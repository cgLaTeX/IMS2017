\documentclass[conference]{IEEEtran}

%\usepackage{array}
%\usepackage{cite}
\usepackage{amsmath}
\usepackage{cleveref}
\usepackage{filecontents}
\usepackage{graphicx}
\usepackage{paralist}
\usepackage{pgfplots}
\usepackage{physics}
\usepackage{siunitx}

\usepackage{microtype}

\pgfplotsset{compat=1.11}

\usetikzlibrary{spy}
%\usetikzlibrary{external}
%\tikzexternalize[prefix=build/tikz/]

\hyphenation{Hamil-tonian band-limitedness op-tical net-works semi-conduc-tor}

\begin{document}
\title{A Self-Consistent Integral Equation Framework for Simulating Optically-Active Media}

%\author{
  %\IEEEauthorblockN{
    %Connor Glosser\IEEEauthorrefmark{1}\IEEEauthorrefmark{2},
    %B.\ Shanker\IEEEauthorrefmark{2}, and
    %Carlo Piermarocchi\IEEEauthorrefmark{1}
  %}
  %\IEEEauthorblockA{\IEEEauthorrefmark{1}Department of Physics \& Astronomy\\
    %Michigan State University, East Lansing, Michigan 48824}
  %\IEEEauthorblockA{\IEEEauthorrefmark{1}Department of Electrical \& Computer Engineering\\
    %Michigan State University, East Lansing, Michigan 48824}
%}

% use for special paper notices
%\IEEEspecialpapernotice{(Invited Paper)}

\maketitle

\begin{abstract}
Here we consider a disordered system of interacting quantum dots---nanoscale semiconductors with wide applicability in systems ranging from lasing to quantum computing to biological contrast imaging and next-generation displays.
Quantum dots facilitate absorptive and emissive processes at specific frequencies over timescales independent of those in the incident radiation; by treating the system \emph{semiclassically} we maintain the discrete dynamics inherent to quantum objects without resorting to second quantization to describe electromagnetic fields (i.e.\ fields behave classically).
Our solution to the coupled system proceeds via
  \begin{inparaenum}[(1)]
  \item determination of source wavefunctions through evolution of the differential Liouville equations, and \label{enum:step 1}
  \item evaluation of radiation patterns through well-known computational electromagnetics techniques. \label{enum:step 2}
  \end{inparaenum}
We employ a highly-tuned predictor-corrector integration scheme to advance the source wavefunctions in time; the subsequent polarizations that arise then serve as point-like sources within time domain integral equations electromagnetic integral equations that is used to propagate the field.
The coupled solution of (1) and (2) produces a complete description of both the quantum and electromagnetic dynamics at each timestep giving rise to lasing effects, non-linear propagation, coupled Rabi oscillations, and other optical phenomena.
This paper presents the theory, computational framework, and selected interesting physical observations for several such systems.
\end{abstract}

\IEEEpeerreviewmaketitle

\section{Introduction}
Conventional models of electromagnetic systems give rise to interesting phenomena through prescription of static boundary conditions.
While these boundary conditions sufficiently describe passive scattering within systems, they fail to account for richer behavior in systems with dynamics.
Such behaviors have proven essential in countless technological applications motivating the need for an efficient fullwave Maxwell solver that couples to an underlying description of the system behavior.

\section{Problem Statement}

\subsection{The quantum Liouville-von Neumann equation}
We wish to model the behavior of $N$ interacting quantum dots immersed in a homogeneous background medium amidst monochromatic radiation of frequency $\omega_L$. Due to the band limit of the incident radiation, we approximate each quantum dot as a two-level system with transition frequency $\omega_0 \approx \omega_L$.
As such, a density matrix formulation fully describes the time-evolution of the population in each level as well as the coherences between levels that give rise to radiation physics.

The density matrix for a quantum dot, $\hat{\rho}$, evolves according to the Liouville-von Neumann equation
\begin{equation}
  i \hbar \pdv{\hat{\rho}}{t} = \commutator{\hat{\mathcal{H}}(t)}{\hat{\rho}} - \hat{\mathcal{D}}\qty[\hat{\rho}]
  \label{eq:liouville}
\end{equation}
where $\hat{\mathcal{H}}$ and $\hat{\mathcal{D}}$ denote the quantum dot's Hamiltonian and dissipator operators, respectively\cite{Breuer2002}.
Here, $\hat{\mathcal{H}}(t)$ describes the  two-level structure of the dot as well as an explicit coupling to an external electric field.
In contrast $\hat{\mathcal{D}}$ models the aggregate effect of processes that destroy coherences and relax populations in the quantum dot system (such as spontaneous emission or phonon losses) through assumed interactions with an external environment.
As matrices,
\begin{subequations}
  \begin{align}
    \hat{\mathcal{H}}(t) &\equiv \mqty(0 & \hbar \chi(t) \\ \hbar \chi^*(t) & \hbar \omega_0) \label{eq:hamiltonian} \\
    \hat{\mathcal{D}}[\hat{\rho}] &\equiv \mqty((\rho_{00} + 1)/T_2 & \rho_{01}/T_1 \\ \rho_{10}/T_1 & \rho_{11}/T_2) \label{eq:dissipator}
  \end{align}
\end{subequations}
where $\chi(t) \equiv \vb{d} \cdot \vb{E}/\hbar$, $\vb{d} \equiv \matrixel{0}{\hat{\vb{d}}}{1}$, $\hat{\vb{d}}$ represents the dipole moment operator $\hat{\vb{d}} \equiv -e \hat{\vb{r}}$, and $T_{1,2}$ denote characteristic relaxation times (determined empirically).
Physically, $\chi(t)$ should contain $\hat{\vb{E}}$ as an \emph{operator} to couple the quantum dot to a quantized electric field of photons; by considering systems with a sufficiently high field intensity so as to ignore single-photon effects, however, we instead treat $\vb{E}$ as a classical object arising from conventional electromagnetic source distributions in a manner consistent with the semiclassical nature of our approach.

\subsection{Determination of electric fields}

The evolution of \cref{eq:liouville} necessarily produces radiation as the quantum dot absorbs and re-radiates light.
Using the density matrix formulation detailed above, we may define a classical polarization as
\begin{equation}
  \begin{aligned}
    \vb{P}(\vb{r}, t) &\equiv \sum_i \Tr[\hat{\vb{d}}_i \hat{\rho}_i(t)] \delta(\vb{r} - \vb{r}_i) \\
                      &\equiv \sum_i 2 \vb{d}_i \Re\!\qty[\rho_{01}^{(i)}(t)] \delta(\vb{r} - \vb{r}_i)
  \end{aligned}
  \label{eq:polarization}
\end{equation}
for a set of dots each having a location $\vb{r}_i$, dipole moment $\vb{d}_i$, and off-diagonal matrix element $\rho_{01}^{(i)}(t)$.
From this we may compute the electric field at any other point in space via the the electric dyadic Green's function\cite{Rothwell2008}
\begin{equation}
  \begin{gathered}
    \vb{E}(\vb{r}, t) = \vb{E}_0(\vb{r}, t) - \frac{\mu_0}{4\pi} \int
      \qty(I - 3\bar{\vb{r}}\bar{\vb{r}}) \frac{c^2 \vb{P}(\vb{r}', t_R)}{\abs{\vb{r} - \vb{r}'}^3} + \\
      \qty(I - 3\bar{\vb{r}}\bar{\vb{r}}) \frac{c \dot{\vb{P}}(\vb{r}', t_R)}{\abs{\vb{r} - \vb{r}'}^2} +
      \qty(I -  \bar{\vb{r}}\bar{\vb{r}}) \frac{\ddot{\vb{P}}(\vb{r}', t_R)}{\abs{\vb{r} - \vb{r}'}} \dd[3]{\vb{r}'}
  \end{gathered}
  \label{eq:efield}
\end{equation}
where $\bar{\vb{r}} \equiv \vb{r} - \vb{r}'/\abs{\vb{r} - \vb{r}'}$ and $t_R \equiv t - \abs{\vb{r} - \vb{r}'}/c$.
These fields serve to couple the dynamics of each quantum dot through the definition of $\chi(t)$ in \cref{eq:hamiltonian}.
Through this coupling \cref{eq:liouville} becomes highly non-linear as $\hat{\rho}_j$ depends on itself arbitrarily far into the past.



\section{Solution Methodology}
\subsection{Discretization and time-evolution}

To solve both \cref{eq:liouville,eq:efield} simultaneously we must first choose an appropriate discretization for $\hat{\rho}$ in both space and time.
As the physical dimensions of these nanostructures remain negligible when compared to the wavelengths present in radiated fields, $\vb{E}(\vb{r}, t)$ varies little throughout the volume occupied by a dot thus we may approximate each as a $\delta$-function in space.
To represent the temporal part of $\hat{\rho}$ we make use of an interpolatory set of Lagrange polynomials.
Such a set has two advantages: it easily accounts for non-integer delay times in $t_R$ by interpolating $\vb{P}(\vb{r}, t)$ between timepoints and it facilitates straightforward evaluation of the polarization derivatives in \cref{eq:efield}.

This discretization, then, suggests a straightforward solution methodology to solve this non-linear equation in two iterated steps; (a) we first use \cref{eq:efield} to evaluate $\vb{E}(\vb{r}, t)$ from a given source distribution at the location of each dot and (b ) then step \cref{eq:rotating liouville} forward in time to update the source distribution.
The first step---evaluating \cref{eq:efield}---proceeds straightforwardly save for pre-calculating the spatial terms and interpolation weights at the start of the simulation.
Numerically integrating \cref{eq:liouville}, however, requires some care.
For this, we make use of a highly-tuned predictor/corrector ($PE(CE)^m$) algorithm detailed in \cite{Glaser2009}.
By approximating the solution to the differential equation as a weighted set of (complex) exponentials we recover the oscillating and decaying modes in $\hat{\rho}(t)$ with a high degree of accuracy without incurring a significant penalty in the size of our timestep $\Delta t$.
Moreover, repeated iterations of the corrector-evaluator step ($CE^m$) facilitate mutual interactions of quantum dots within one timestep; as a result, the timestep becomes unconstrained by delay between quantum dots and we need only consider the frequencies in the system in choosing $\Delta t$.
%The limitation on $\Delta t$ arises due to the non-linear differential equation, wherein one needs to make sure that $\delta t$ is chosen such that $R_{min}> c \Delta t$.
%This ensures a self consistent solution to the non-linear system.
%However, this constraint results excessive number of time steps.

\subsection{The rotating-wave approximation}
\begin{figure}
  \centering
  \begin{filecontents}{frames.dat}
1.666944490748458e-10   1.000000000001851     1.000000000001855
0.0001603274543882296   1.0000000000018556    1.000000000003555
0.0003341406062989621   1.0000000000018603    1.0000000000056366
0.0004964354550654196   1.0000000000018647    1.0000000000077092
0.0006555467231413446   1.0000000000018692    1.0000000000096905
0.0008281438554342214   1.0000000000018736    1.0000000000118514
0.0009892226845828235   1.0000000000018778    1.000000000013864
0.0011637873779483776   1.0000000000018823    1.000000000016046
0.001335168490623399    1.0000000000018865    1.0000000000181875
0.0014950313001541454   1.0000000000018903    1.0000000000201847
0.0016683799739018437   1.0000000000018943    1.0000000000223503
0.0018302103445052673   1.000000000001898     1.000000000024372
0.0019888571344181577   1.0000000000019016    1.0000000000263545
0.002160989788548       1.0000000000019054    1.0000000000285056
0.002321604139533568    1.000000000001909     1.0000000000305131
0.002495704354736088    1.0000000000019125    1.000000000032688
0.0026666209892480747   1.000000000001916     1.0000000000348233
0.002826019320615787    1.0000000000019191    1.000000000036814
0.0029989035162004507   1.0000000000019225    1.0000000000389733
0.0031602694086408398   1.0000000000019256    1.0000000000409883
0.003335121165298181    1.0000000000019287    1.0000000000431728
0.003506789341264989    1.0000000000019318    1.0000000000453173
0.0036669392140875226   1.0000000000019345    1.0000000000473181
0.003840574951127008    1.0000000000019373    1.0000000000494869
0.004002692385022219    1.0000000000019398    1.0000000000515112
0.004161626238226897    1.0000000000019422    1.0000000000534954
0.004334045955648527    1.0000000000019449    1.0000000000556473
0.0044949473699258825   1.000000000001947     1.0000000000576559
0.00466933464842019     1.0000000000019496    1.0000000000598337
0.0048405383462239646   1.0000000000019518    1.0000000000619722
0.005000223740883465    1.0000000000019538    1.000000000063967
0.005173394999759916    1.000000000001956     1.0000000000661302
0.0053350479554920924   1.0000000000019578    1.000000000068148
0.0054935173305337365   1.0000000000019595    1.0000000000701246
0.005665472569792333    1.0000000000019613    1.0000000000722695
0.005825909505906654    1.0000000000019629    1.0000000000742708
0.0059998323062379275   1.0000000000019647    1.0000000000764417
0.006162236803424925    1.000000000001966     1.00000000007847
0.006321457719921391    1.0000000000019673    1.0000000000804588
0.006494164500634808    1.0000000000019686    1.000000000082616
0.006655352978203951    1.0000000000019698    1.0000000000846276
0.0068300273199900455   1.0000000000019709    1.000000000086806
0.007001518081085608    1.000000000001972     1.0000000000889433
0.0071614905390368945   1.0000000000019729    1.0000000000909375
0.007334948861205133    1.0000000000019738    1.0000000000931009
0.007496888880229097    1.0000000000019744    1.0000000000951221
0.007655645318562529    1.000000000001975     1.000000000097105
0.007827887621112911    1.0000000000019755    1.0000000000992562
0.00798861162051902     1.000000000001976     1.0000000001012614
0.00816282148414208     1.0000000000019764    1.0000000001034333
0.008333847767074607    1.0000000000019766    1.0000000001055636
0.008493355746862859    1.0000000000019769    1.0000000001075504
0.008666349590868063    1.0000000000019769    1.0000000001097065
0.008827825131728992    1.0000000000019769    1.000000000111721
0.009002786536806872    1.0000000000019766    1.0000000001139053
0.00917456436119422     1.0000000000019764    1.00000000011605
0.009334823882437294    1.0000000000019762    1.0000000001180498
0.009508569267897319    1.0000000000019758    1.0000000001202147
0.00967079635021307     1.0000000000019753    1.000000000122234
0.009829839851838287    1.0000000000019746    1.0000000001242133
0.010002369217680458    1.000000000001974     1.0000000001263611
0.010163380280378352    1.0000000000019733    1.0000000001283684
0.010337877207293199    1.0000000000019724    1.000000000130546
0.010509190553517513    1.0000000000019713    1.0000000001326848
0.010668985596597551    1.0000000000019704    1.000000000134678
0.010842266503894542    1.000000000001969     1.000000000136837
0.011004029108047259    1.000000000001968     1.0000000001388498
0.011162608131509443    1.0000000000019666    1.0000000001408205
0.01133467301918858     1.000000000001965     1.0000000001429605
0.01149521960372344     1.0000000000019638    1.00000000014496
0.011669252052475253    1.000000000001962     1.0000000001471305
0.01183176619808279     1.0000000000019602    1.000000000149159
0.011991096762999795    1.0000000000019587    1.0000000001511467
0.012163913192133752    1.0000000000019567    1.0000000001532998
0.012325211318123433    1.0000000000019547    1.0000000001553055
0.012499995308330066    1.0000000000019524    1.0000000001574758
0.012671595717846167    1.0000000000019518    1.000000000159607
0.012831677824217994    1.0000000000019513    1.000000000161598
0.013005245794806773    1.0000000000019509    1.0000000001637608
0.013167295462251276    1.0000000000019502    1.0000000001657834
0.013326161549005247    1.0000000000019496    1.0000000001677654
0.01349851349997617     1.0000000000019487    1.0000000001699128
0.013659347147802817    1.0000000000019476    1.0000000001719116
0.013833666659846417    1.0000000000019464    1.000000000174074
0.014004802591199484    1.000000000001945     1.000000000176196
0.014164420219408277    1.0000000000019436    1.0000000001781777
0.01433752371183402     1.0000000000019418    1.0000000001803326
0.01449910890111549     1.00000000000194      1.0000000001823484
0.014657510509706427    1.000000000001938     1.0000000001843248
0.014829397982514316    1.0000000000019358    1.0000000001864664
0.01498976715217793     1.0000000000019333    1.000000000188459
0.015163622186058495    1.0000000000019307    1.0000000001906135
0.015325958916794785    1.0000000000019278    1.0000000001926228
0.015485112066840542    1.000000000001925     1.000000000194595
0.015657751081103254    1.0000000000019214    1.0000000001967402
0.01581887179222169     1.0000000000019178    1.0000000001987475
0.015993478367557077    1.0000000000019138    1.0000000002009262
0.01616490136220193     1.0000000000019096    1.000000000203063
0.01632480605370251     1.0000000000019054    1.0000000002050498
0.01649819660942004     1.0000000000019005    1.0000000002071967
0.016660068861993296    1.0000000000018956    1.0000000002091967
0.01681875753387602     1.0000000000018907    1.0000000002111584
0.016990932069975696    1.000000000001885     1.0000000002132932
0.017151588302931096    1.0000000000018792    1.0000000002152927
0.017325730400103448    1.0000000000018727    1.0000000002174652
0.01749668891658527     1.0000000000018658    1.000000000219596
0.017656129129922815    1.0000000000018594    1.0000000002215772
0.017829055207477313    1.0000000000018519    1.000000000223717
0.017990462981887535    1.0000000000018445    1.000000000225708
0.01816535662051471     1.000000000001836     1.0000000002278653
0.01833706667845135     1.0000000000018276    1.0000000002299896
0.018497258433243718    1.0000000000018192    1.0000000002319802
0.018670936052253038    1.0000000000018097    1.0000000002341451
0.018833095368118082    1.0000000000018001    1.0000000002361675
0.018992071103292592    1.0000000000017895    1.0000000002381433
0.019164532702684055    1.0000000000017775    1.0000000002402765
0.01932547599893124     1.000000000001766     1.0000000002422589
0.01949990515939538     1.000000000001753     1.000000000244404
0.019671150739168988    1.0000000000017397    1.0000000002465166
0.01983087801579832     1.000000000001727     1.0000000002484966
0.020004091156644605    1.0000000000017129    1.0000000002506537
0.020165785994346614    1.000000000001699     1.0000000002526699
0.020324297251358092    1.0000000000016853    1.0000000002546419
0.020496294372586522    1.00000000000167      1.000000000256769
0.020656773190670677    1.0000000000016551    1.0000000002587428
0.020830737872971784    1.0000000000016387    1.0000000002608767
0.020993184252128615    1.000000000001623     1.000000000262873
0.02115244705059491     1.000000000001607     1.0000000002648408
0.02132519571327816     1.0000000000015892    1.0000000002669878
0.021486426072817137    1.0000000000015723    1.000000000268998
0.021661142296573066    1.0000000000015534    1.0000000002711726
0.02183267493963846     1.0000000000015343    1.0000000002732947
0.02199268927955958     1.0000000000015161    1.0000000002752607
0.02216618948369765     1.0000000000014961    1.0000000002773828
0.022328171384691446    1.0000000000014768    1.0000000002793652
0.02248696970499471     1.0000000000014575    1.0000000002813199
0.022659253889514928    1.000000000001436     1.000000000283456
0.02282001977089087     1.0000000000014155    1.0000000002854585
0.022994271516483762    1.0000000000013929    1.0000000002876288
0.02316533968138612     1.0000000000013702    1.0000000002897471
0.023324889543144208    1.0000000000013485    1.0000000002917055
0.023497925269119247    1.0000000000013245    1.0000000002938165
0.02365944269195001     1.0000000000013016    1.0000000002957852
0.023834445978997726    1.0000000000012763    1.0000000002979292
0.024006265685354907    1.000000000001251     1.0000000003000535
0.024166567088567813    1.0000000000012268    1.0000000003020482
0.02434035435599767     1.0000000000012002    1.0000000003042138
0.024502623320283256    1.0000000000011748    1.0000000003062242
0.024661708703878307    1.0000000000011495    1.0000000003081768
0.02483427995169031     1.0000000000011213    1.0000000003102765
0.02499533289635804     1.000000000001095     1.0000000003122302
0.02516987170524272     1.000000000001067     1.000000000314357
0.025341226933436867    1.00000000000104      1.0000000003164666
0.02550106385848674     1.0000000000010145    1.000000000318452
0.025674386647753566    1.0000000000009863    1.000000000320612
0.025836191133876116    1.0000000000009595    1.000000000322619
0.025994812039308132    1.0000000000009326    1.000000000324567
0.0261669188089571      1.0000000000009028    1.000000000326657
0.026327507275461796    1.0000000000008744    1.0000000003285963
0.026501581606183443    1.000000000000843     1.0000000003307041
0.026672472356214557    1.0000000000008116    1.0000000003327953
0.026831844803101395    1.0000000000007816    1.0000000003347687
0.027004703114205185    1.0000000000007485    1.0000000003369218
0.0271660431221647      1.000000000000717     1.000000000338926
0.027340868994341166    1.000000000000682     1.000000000341075
0.0275125112858271      1.000000000000647     1.0000000003431544
0.02767263527416876     1.0000000000006135    1.000000000345078
0.027846245126727374    1.0000000000005766    1.0000000003471667
0.02800833667614171     1.0000000000005416    1.0000000003491372
0.028167244644865516    1.0000000000005063    1.0000000003510954
0.028339638477806274    1.0000000000004674    1.0000000003532403
0.028500514007602756    1.0000000000004303    1.0000000003552416
0.02867487540161619     1.0000000000003892    1.000000000357388
0.02884605321493909     1.0000000000003482    1.0000000003594605
0.029005712725117715    1.0000000000003089    1.0000000003613698
0.029178858099513292    1.0000000000002656    1.000000000363437
0.029340485170764596    1.0000000000002245    1.0000000003653857
0.029498928661325367    1.0000000000001832    1.0000000003673253
0.02967085801610309     1.0000000000001377    1.000000000369459
0.029831269067736536    1.0000000000000941    1.0000000003714562
0.030005165983586934    1.0000000000000462    1.0000000003736018
0.030167544596293058    1.0000000000000007    1.0000000003755702
0.03032673962830865     0.999999999999955     1.0000000003774687
0.030499420524541195    0.9999999999999045    1.0000000003795138
0.030660583117629463    0.9999999999998566    1.0000000003814353
0.030835231574934684    0.9999999999998035    1.0000000003835547
0.03100669645154937     0.9999999999997504    1.0000000003856735
0.031166643025019786    0.9999999999996999    1.000000000387665
0.03134007546270715     0.9999999999996443    1.0000000003898104
0.03150198959725024     0.9999999999995941    1.0000000003917757
0.03166072015110281     0.9999999999995439    1.000000000393664
0.03183293656917232     0.9999999999994883    1.0000000003956877
0.031993634684097556    0.9999999999994355    1.0000000003975824
0.03216781866323975     0.9999999999993769    1.0000000003996736
0.03233881906169141     0.9999999999993183    1.0000000004017737
0.03249830115699879     0.9999999999992625    1.000000000403757
0.03267126911652312     0.9999999999992008    1.000000000405902
0.03283271877290318     0.9999999999991419    1.0000000004078675
0.033007654293500194    0.9999999999990768    1.0000000004099445
0.03317940623340667     0.9999999999990117    1.0000000004119465
0.03333963987016887     0.9999999999989495    1.0000000004138145
0.03351335937114803     0.9999999999988808    1.0000000004158769
0.03367556056898291     0.9999999999988153    1.0000000004178529
0.03383457818612726     0.9999999999987499    1.0000000004198255
0.03400708166748856     0.9999999999986774    1.0000000004219691
0.034168066845705586    0.9999999999986084    1.0000000004239356
0.03434253788813956     0.9999999999985321    1.0000000004260063
0.034513825349883       0.9999999999984556    1.000000000427989
0.034673594508482175    0.9999999999983828    1.0000000004298268
0.034846849531298296    0.9999999999983022    1.0000000004318532
0.03500858625097014     0.9999999999982256    1.0000000004338008
0.03516713938995145     0.999999999998149     1.0000000004357572
0.03533917839314972     0.9999999999980642    1.0000000004378977
0.03549969909320371     0.9999999999979835    1.0000000004398673
0.03567370565747465     0.9999999999978944    1.0000000004419372
0.03583619391860132     0.9999999999978094    1.000000000443809
0.03599549859903746     0.9999999999977245    1.0000000004456175
0.036168289143690545    0.9999999999976307    1.0000000004475993
0.036329561385199355    0.9999999999975414    1.0000000004495073
0.03650431949092512     0.9999999999974427    1.000000000451644
0.03667589401596035     0.9999999999973439    1.000000000453778
0.03683595023785131     0.99999999999725      1.0000000004557499
0.03700949232395922     0.9999999999971462    1.0000000004578207
0.037171516106922854    0.9999999999970475    1.0000000004596796
0.037330356309195956    0.9999999999969489    1.0000000004614598
0.037502682375686006    0.99999999999684      1.0000000004633989
0.03766349013903179     0.9999999999967376    1.0000000004652654
0.03783778376659452     0.9999999999966265    1.00000000046737
0.03800889381346671     0.9999999999965152    1.0000000004694927
0.03816848555719463     0.9999999999964094    1.0000000004714673
0.03834156316513951     0.9999999999962925    1.0000000004735443
0.03850312246994011     0.9999999999961813    1.0000000004753973
0.038661498194050174    0.9999999999960703    1.0000000004771517
0.038833359782377196    0.9999999999959475    1.0000000004790446
0.03899370306755994     0.9999999999958308    1.0000000004808614
0.039167532216959636    0.9999999999957019    1.0000000004829233
0.039329843063215054    0.9999999999955792    1.0000000004849228
0.039488970328779946    0.9999999999954566    1.0000000004869
0.039661583458561786    0.9999999999953213    1.0000000004889922
0.03982267828519935     0.9999999999951925    1.0000000004908505
0.03999725897605387     0.9999999999950504    1.0000000004927692
0.040168656086217856    0.9999999999949082    1.000000000494616
0.040328534893237566    0.999999999994773     1.000000000496379
0.040501899564474225    0.9999999999946236    1.0000000004983909
0.040663745932566615    0.9999999999944817    1.000000000500363
0.04082240871996847     0.9999999999943401    1.0000000005023357
0.040994557371587276    0.9999999999941835    1.0000000005044378
0.041155187720061805    0.9999999999940349    1.0000000005063014
0.04132930393275329     0.9999999999938708    1.0000000005082033
0.04150023656475424     0.9999999999937066    1.0000000005100051
0.041659650893610914    0.9999999999935508    1.0000000005117078
0.041832551086684544    0.9999999999933787    1.0000000005136562
0.0419939329766139      0.9999999999932153    1.0000000005155887
0.0421688007307602      0.9999999999930349    1.0000000005177567
0.04234048490421597     0.9999999999928547    1.000000000519863
0.04250065077452747     0.9999999999926835    1.0000000005217273
0.04267430250905592     0.9999999999924947    1.0000000005236107
0.04283643594044009     0.9999999999923153    1.0000000005252814
0.04299538579113373     0.9999999999921365    1.0000000005269223
0.04316782150604433     0.9999999999919391    1.000000000528799
0.043328738917810646    0.9999999999917518    1.0000000005306826
0.04350314219379391     0.9999999999915453    1.00000000053283
0.04367436188908665     0.999999999991339     1.000000000534944
0.04383406328123512     0.9999999999911434    1.0000000005368224
0.04400725053760053     0.9999999999909337    1.0000000005387022
0.04416891949082167     0.9999999999907359    1.0000000005403367
0.04432740486335228     0.9999999999905387    1.0000000005419116
0.04449937610009984     0.999999999990321     1.0000000005437044
0.04465982903370312     0.9999999999901142    1.0000000005455214
0.04483376783152336     0.9999999999898863    1.0000000005476337
0.044996188326199324    0.9999999999896696    1.0000000005496494
0.04515542524018475     0.9999999999894535    1.000000000551554
0.04532814801838713     0.9999999999892151    1.0000000005534533
0.04548935249344524     0.9999999999889886    1.0000000005550673
0.0456640428327203      0.9999999999887389    1.0000000005567384
0.04583554959130482     0.9999999999884891    1.0000000005584402
0.04599553804674507     0.9999999999882522    1.00000000056018
0.04616901236640227     0.9999999999879907    1.0000000005622436
0.0463309683829152      0.9999999999877422    1.000000000564253
0.04648974081873759     0.9999999999874947    1.000000000566176
0.04666199911877694     0.9999999999872213    1.000000000568091
0.046822739115672016    0.9999999999869619    1.000000000569688
0.04699696497678404     0.9999999999866757    1.0000000005712921
0.047168007257205526    0.9999999999863897    1.000000000572893
0.047327531234482745    0.9999999999861184    1.000000000574536
0.04750054107597691     0.9999999999858191    1.0000000005765286
0.047662032614326805    0.999999999985535     1.0000000005785188
0.04783701001689365     0.9999999999852217    1.0000000005806577
0.048008803838769966    0.9999999999849087    1.0000000005825795
0.048169079357502004    0.9999999999846119    1.0000000005841547
0.04834284074045099     0.9999999999842845    1.000000000585691
0.04850508382025571     0.9999999999839736    1.000000000587116
0.04866414331936989     0.9999999999836638    1.0000000005886518
0.048836688682701024    0.9999999999833222    1.0000000005905565
0.04899771574288788     0.999999999982998     1.0000000005925136
0.04917222866729169     0.999999999982641     1.0000000005946659
0.04934355801100497     0.9999999999822843    1.0000000005966172
0.04950336905157397     0.9999999999819462    1.0000000005981942
0.04967666595635993     0.9999999999815735    1.0000000005996763
0.04983844455800161     0.9999999999812199    1.0000000006009964
0.04999703957895276     0.9999999999808679    1.0000000006024066
0.05016912046412086     0.9999999999804825    1.0000000006041938
0.050329683046144685    0.9999999999801269    1.000000000606094
0.05050373149238546     0.999999999979735     1.0000000006082503
0.050674596357935704    0.9999999999793435    1.0000000006102414
0.05083394292034167     0.9999999999789724    1.000000000611841
0.05100677534696459     0.999999999978563     1.0000000006132872
0.05116808947044324     0.9999999999781745    1.0000000006145044
0.051342889458138835    0.9999999999777461    1.0000000006159167
0.0515145058651439      0.9999999999773183    1.0000000006175855
0.051674603969004695    0.9999999999769122    1.000000000619421
0.051848187937082436    0.9999999999764646    1.000000000621567
0.0520102536020159      0.9999999999760394    1.00000000062349
0.05216913568625884     0.9999999999756158    1.0000000006251175
0.052341503634718727    0.9999999999751484    1.00000000062654
0.05250235328003434     0.9999999999747049    1.0000000006276613
0.052676688789566904    0.999999999974216     1.0000000006289094
0.05284784071840894     0.9999999999737275    1.000000000630411
0.05300747434410669     0.9999999999732642    1.0000000006321397
0.0531805938340214      0.9999999999727532    1.0000000006342538
0.053342195020791836    0.9999999999722683    1.0000000006362115
0.05350061262687174     0.9999999999717851    1.0000000006378869
0.05367251609716859     0.9999999999712521    1.0000000006393124
0.053832901264321165    0.9999999999707464    1.000000000640353
0.054006772295690696    0.9999999999701891    1.000000000641428
0.05416912502391595     0.9999999999696598    1.0000000006426553
0.05432829417145067     0.9999999999691326    1.000000000644215
0.05450094918320235     0.9999999999685513    1.0000000006462493
0.05466208589180975     0.9999999999679997    1.0000000006482366
0.0548367084646341      0.9999999999673921    1.0000000006501588
0.055008147456767915    0.9999999999667853    1.0000000006516006
0.05516806814575746     0.9999999999662101    1.0000000006525729
0.05534147469896396     0.9999999999655762    1.0000000006534768
0.05550336294902618     0.9999999999649746    1.0000000006544973
0.05566206761839786     0.9999999999643757    1.0000000006558765
0.055834258151986504    0.9999999999637155    1.0000000006578027
0.05599493038243087     0.9999999999630895    1.0000000006597887
0.056169088477092184    0.9999999999624001    1.0000000006617775
0.05634006299106297     0.9999999999617121    1.0000000006632679
0.05649951920188948     0.9999999999610729    1.0000000006642007
0.05667246127693294     0.9999999999603768    1.0000000006649419
0.056833885048832126    0.9999999999597161    1.0000000006657297
0.057008794684948266    0.9999999999589885    1.0000000006670275
0.05718052074037387     0.9999999999582616    1.0000000006688354
0.0573407284926552      0.9999999999575724    1.0000000006708005
0.05751442210915349     0.9999999999568127    1.0000000006728293
0.0576765974225075      0.9999999999560913    1.0000000006742882
0.05783558915517097     0.9999999999553729    1.0000000006751986
0.0580080667520514      0.9999999999545807    1.0000000006757876
0.05816902604578755     0.9999999999538288    1.0000000006763379
0.05834347120374066     0.9999999999530005    1.0000000006773566
0.05851473278100323     0.999999999952173     1.0000000006789695
0.05867447605512152     0.9999999999513886    1.0000000006808765
0.05884770519345677     0.9999999999505236    1.0000000006829604
0.059009416028647746    0.9999999999497027    1.0000000006845031
0.059167943283148186    0.9999999999488851    1.000000000685432
0.05933995640186558     0.9999999999479833    1.0000000006858987
0.0595004512174387      0.9999999999471281    1.0000000006862007
0.05967443189722877     0.9999999999461855    1.0000000006868868
0.059836894273874563    0.9999999999452907    1.0000000006881502
0.05999617306982983     0.9999999999443994    1.0000000006899223
0.060168937730002044    0.9999999999434168    1.0000000006920464
0.06033018408702998     0.9999999999424847    1.0000000006937264
0.06050491630827488     0.9999999999414579    1.00000000069483
0.06067646494882924     0.9999999999404328    1.0000000006951948
0.06083649528623932     0.999999999939461     1.0000000006952543
0.061010011487866354    0.9999999999383903    1.0000000006955871
0.06117200938634912     0.9999999999373743    1.0000000006965495
0.06133082370414135     0.9999999999363628    1.000000000698149
0.06150312388615053     0.999999999935248     1.0000000007002603
0.06166390576501543     0.999999999934191     1.0000000007020375
0.06183817350809729     0.999999999933027     1.0000000007032357
0.062009257670488614    0.9999999999318655    1.000000000703549
0.06216882352973566     0.9999999999307652    1.0000000007033807
0.06234187525319967     0.999999999929553     1.0000000007033205
0.06250340867351939     0.9999999999284037    1.000000000703903
0.06267842795805606     0.9999999999271415    1.000000000705425
0.0628502636619022      0.9999999999259157    1.0000000007074918
0.06301058106260407     0.9999999999247536    1.000000000709327
0.0631843843275229      0.9999999999234734    1.0000000007105818
0.06334666928929744     0.9999999999222586    1.0000000007108327
0.06350577067038145     0.999999999921049     1.0000000007104548
0.06367835791568241     0.9999999999197156    1.0000000007099945
0.0638394268578391      0.9999999999184509    1.0000000007101584
0.06401398166421274     0.9999999999170578    1.0000000007113412
0.06418535288989584     0.999999999915667     1.0000000007132936
0.06434520581243466     0.9999999999143486    1.0000000007152041
0.06451854459919044     0.9999999999128958    1.0000000007166079
0.06468036508280195     0.9999999999115171    1.0000000007168983
0.06483900198572293     0.9999999999101444    1.000000000716354
0.06501112475286086     0.999999999908631     1.0000000007154806
0.06517172921685452     0.9999999999071959    1.000000000715151
0.06534581954506512     0.9999999999056148    1.000000000715876
0.06551672629258519     0.9999999999040365    1.0000000007176173
0.06567611473696099     0.9999999999025411    1.00000000071957
0.06584898904555374     0.9999999999008928    1.000000000721163
0.06601034505100221     0.9999999998993294    1.000000000721574
0.06618518692066763     0.9999999998976078    1.000000000720827
0.06635684520964252     0.9999999998958892    1.0000000007195478
0.06651698519547314     0.9999999998942604    1.000000000718737
0.06669061104552071     0.9999999998924659    1.0000000007190006
0.066852718592424       0.9999999998907636    1.000000000720402
0.06701164255863677     0.9999999998890692    1.000000000722335
0.06718405238906648     0.9999999998872018    1.000000000724095
0.06734494391635192     0.9999999998854316    1.0000000007246512
0.0675193213078543      0.9999999998834826    1.0000000007238365
0.06769051511866617     0.9999999998815379    1.0000000007221663
0.06785019062633375     0.9999999998796958    1.0000000007207739
0.0680233519982183      0.999999999877667     1.0000000007203855
0.06818499506695856     0.9999999998757434    1.0000000007213754
0.06834345455500829     0.9999999998738296    1.0000000007232035
0.06851539990727497     0.9999999998717211    1.0000000007251293
0.06867582695639737     0.9999999998697235    1.00000000072591
0.06884973986973673     0.9999999998675249    1.000000000725142
0.06901213447993182     0.9999999998654678    1.0000000007232563
0.06917134550943636     0.9999999998634492    1.0000000007212295
0.06934404240315786     0.9999999998612252    1.0000000007199812
0.06950522099373509     0.9999999998591167    1.000000000720314
0.06967988544852927     0.9999999998567953    1.0000000007220606
0.06985136632263292     0.9999999998544787    1.0000000007240932
0.0700113288935923      0.9999999998522836    1.0000000007250973
0.07018477732876861     0.9999999998498655    1.000000000724426
0.07034670746080066     0.9999999998475718    1.000000000722294
0.07050545401214217     0.9999999998452889    1.0000000007196834
0.07067768642770064     0.9999999998427727    1.0000000007175958
0.07083840054011484     0.9999999998403875    1.0000000007172274
0.07101260051674597     0.9999999998377607    1.0000000007185716
0.07118361691268658     0.9999999998351394    1.000000000720644
0.07134311500548292     0.9999999998326561    1.000000000721922
0.07151609896249621     0.99999999982992      1.0000000007214818
0.07167756461636522     0.9999999998273256    1.000000000719223
0.07185251613445119     0.9999999998244694    1.0000000007157268
0.07202428407184662     0.9999999998216189    1.000000000712823
0.07218453370609777     0.9999999998189177    1.000000000711758
0.07235826920456588     0.999999999815943     1.0000000007126582
0.07252048639988971     0.9999999998131214    1.0000000007145828
0.07267952001452302     0.9999999998103132    1.0000000007160945
0.07285203949337328     0.9999999998072194    1.0000000007159164
0.07301304066907927     0.999999999804287     1.0000000007135987
0.07318752770900219     0.999999999801059     1.0000000007095293
0.07335883116823459     0.9999999997978388    1.0000000007056513
0.07351861632432272     0.9999999997947888    1.0000000007036127
0.0736918873446278      0.9999999997914304    1.0000000007037695
0.0738536400617886      0.9999999997882465    1.0000000007054994
0.07401220919825886     0.9999999997850793    1.0000000007072343
0.07418426419894608     0.9999999997815903    1.0000000007074428
0.07434480089648902     0.9999999997782854    1.0000000007052343
0.07451882345824892     0.999999999774648     1.0000000007007066
0.07468132771686455     0.9999999997711995    1.0000000006960268
0.07484064839478964     0.9999999997677691    1.0000000006927214
0.07501345493693168     0.9999999997639926    1.000000000691715
0.07517474317592944     0.9999999997604146    1.0000000006929544
0.07534951727914416     0.9999999997565677    1.0000000006950178
0.07552110780166835     0.9999999997527447    1.0000000006956025
0.07568118002104826     0.9999999997491233    1.0000000006935659
0.07585473810464512     0.9999999997451362    1.0000000006886687
0.07601677788509771     0.9999999997413556    1.0000000006830603
0.07617563408485976     0.9999999997375938    1.00000000067853
0.07634797614883877     0.9999999997334497    1.0000000006763035
0.0765087999096735      0.9999999997295222    1.000000000676901
0.0766831095347252      0.9999999997251988    1.0000000006789609
0.07685423557908634     0.9999999997208858    1.000000000679995
0.07701384332030321     0.9999999997168008    1.0000000006783214
0.07718693692573704     0.9999999997123017    1.0000000006732592
0.07734851222802659     0.9999999997080361    1.0000000006667802
0.07750690394962562     0.9999999997037922    1.0000000006608938
0.07767878153544158     0.9999999996991157    1.0000000006571348
0.07783914081811329     0.9999999996946849    1.0000000006567638
0.07801298596500193     0.9999999996898067    1.0000000006585943
0.0781753128087463      0.9999999996851803    1.0000000006600664
0.07833445607180015     0.9999999996805767    1.0000000006591188
0.07850708519907094     0.9999999996755061    1.0000000006543543
0.07866819602319747     0.9999999996707006    1.0000000006472394
0.07884279271154093     0.999999999665412     1.0000000006391871
0.07901420581919387     0.9999999996601366    1.0000000006337513
0.07917410062370253     0.9999999996551406    1.000000000632185
0.07934748129242815     0.9999999996496404    1.0000000006335354
0.07950934365800949     0.9999999996444264    1.0000000006352991
0.0796680224429003      0.99999999963924      1.0000000006349241
0.07984018709200806     0.9999999996335279    1.000000000630446
0.08000083343797154     0.9999999996281174    1.00000000062281
0.08017496564815198     0.9999999996221633    1.0000000006132548
0.08034591427764189     0.9999999996162269    1.0000000006058547
0.08050534460398752     0.999999999610608     1.0000000006026875
0.0806782607945501      0.9999999996044228    1.000000000603168
0.08083965868196841     0.999999999598563     1.0000000006050735
0.08101454243360366     0.9999999995921183    1.0000000006052274
0.08118624260454839     0.9999999995856933    1.0000000006008756
0.08134642447234884     0.9999999995796112    1.000000000592669
0.08152009220436625     0.9999999995729729    1.000000000581655
0.08168224163323938     0.9999999995667905    1.000000000572726
0.08184120748142197     0.9999999995606413    1.0000000005677323
0.08201365919382152     0.9999999995538699    1.000000000567022
0.08217459260307679     0.9999999995474544    1.0000000005688214
0.08234901187654901     0.9999999995403949    1.0000000005696386
0.0825202475693307      0.9999999995333544    1.000000000565933
0.08267996495896812     0.9999999995266879    1.0000000005575853
0.08285316821282249     0.999999999519348     1.0000000005452134
0.08301485316353258     0.9999999995123908    1.000000000534131
0.08317335453355214     0.9999999995054704    1.0000000005268748
0.08334534176778866     0.9999999994978469    1.0000000005244234
0.0835058106988809      0.9999999994906252    1.0000000005257665
0.08367976549419008     0.9999999994826763    1.0000000005272203
0.083842201986355       0.999999999475139     1.0000000005247471
0.08400145489782938     0.9999999994676402    1.0000000005169412
0.08417419367352072     0.9999999994593826    1.0000000005036502
0.08433541414606778     0.9999999994515575    1.0000000004902623
0.0845101204828318      0.9999999994429474    1.000000000479289
0.08468164323890527     0.9999999994343604    1.000000000474739
0.08484164769183447     0.9999999994262287    1.000000000475267
0.08501513800898063     0.9999999994172778    1.0000000004771064
0.0851771100229825      0.9999999994087935    1.0000000004755396
0.08533589845629386     0.9999999994003548    1.0000000004682181
0.08550817275382215     0.9999999993910622    1.0000000004541663
0.08566892874820618     0.9999999993822603    1.0000000004386964
0.08584317060680716     0.9999999993725758    1.0000000004246041
0.0860142288847176      0.9999999993629205    1.0000000004172862
0.08617376885948377     0.9999999993537818    1.0000000004164245
0.0863467946984669      0.9999999993437232    1.0000000004183691
0.08650830223430575     0.9999999993341939    1.0000000004178748
0.08668329563436154     0.9999999993237144    1.000000000410418
0.08685510545372681     0.9999999993132673    1.0000000003954825
0.0870153969699478      0.9999999993033775    1.0000000003779754
0.08718917435038574     0.9999999992924977    1.0000000003608847
0.08735143342767941     0.9999999992821891    1.000000000351139
0.08751050892428255     0.9999999992719406    1.0000000003484433
0.08768307028510264     0.9999999992606625    1.000000000350073
0.08784411334277845     0.999999999250159     1.0000000003504779
0.08801864226467121     0.9999999992386894    1.0000000003441094
0.08818998760587345     0.9999999992272549    1.000000000329032
0.0883498146439314      0.9999999992164308    1.0000000003096807
0.08852312754620631     0.9999999992045179    1.0000000002890501
0.08868492214533695     0.9999999991932291    1.0000000002756726
0.08884353316377705     0.9999999991820029    1.0000000002703544
0.0890156300464341      0.9999999991696406    1.0000000002710647
0.08917620862594688     0.9999999991579325    1.000000000272294
0.08935027306967662     0.9999999991450493    1.000000000267464
0.08952115393271581     0.9999999991322048    1.000000000252915
0.08968051649261073     0.9999999991200476    1.0000000002321436
0.0898533649167226      0.9999999991066635    1.0000000002078713
0.0900146950376902      0.999999999093983     1.0000000001901943
0.09018951102287476     0.999999999080034     1.000000000180868
0.09036114342736877     0.9999999990661251    1.0000000001803755
0.09052125752871852     0.9999999990529558    1.000000000181984
0.09069485749428521     0.9999999990384629    1.0000000001781992
0.09085693915670763     0.9999999990247275    1.000000000164841
0.09101583723843952     0.999999999011068     1.0000000001429785
0.09118822118438837     0.9999999989960292    1.0000000001152347
0.09134908682719294     0.9999999989817862    1.0000000000930573
0.09152343833421445     0.999999998966118     1.0000000000792606
0.09169460626054543     0.9999999989504992    1.0000000000764282
0.09185425588373214     0.9999999989357172    1.0000000000781633
0.0920273913711358      0.99999999891945      1.00000000007613
0.09218900855539519     0.9999999989040401    1.0000000000642058
0.09234744215896405     0.9999999988887205    1.00000000004197
0.09251936162674985     0.9999999988718551    1.0000000000110383
0.09267976279139138     0.9999999988558901    0.9999999999839079
0.09285364982024986     0.9999999988383296    0.9999999999645011
0.09301601854596407     0.9999999988216917    0.999999999958145
0.09317520369098775     0.9999999988051517    0.9999999999590897
0.09334787470022837     0.9999999987869522    0.9999999999591476
0.09350902740632473     0.9999999987697215    0.9999999999500144
0.09368366597663803     0.9999999987507786    0.999999999926123
0.09385512096626081     0.9999999987319044    0.999999999892368
0.0940150576527393      0.9999999987140896    0.9999999998602981
0.09418848020343476     0.9999999986948858    0.9999999998347824
0.09435038445098594     0.9999999986766942    0.9999999998241123
0.09450910511784658     0.9999999986586093    0.9999999998234536
0.09468131164892417     0.9999999986387021    0.9999999998246407
0.09484199987685749     0.9999999986198541    0.9999999998177206
0.09501617396900776     0.9999999985991224    0.9999999997949729
0.09518716448046749     0.999999998578459     0.9999999997591353
0.09534663668878296     0.999999998558906     0.9999999997220582
0.09551959476131537     0.9999999985373873    0.9999999996894092
0.09568103453070351     0.9999999985170044    0.9999999996729493
0.0958559601643086      0.9999999984945898    0.999999999669475
0.09602770221722316     0.9999999984722451    0.9999999996710863
0.09618792596699345     0.9999999984510926    0.9999999996656338
0.09636163558098068     0.9999999984278207    0.9999999996434584
0.09652382689182365     0.9999999984057691    0.9999999996075348
0.09668283462197608     0.9999999983838437    0.9999999995657185
0.09685532821634546     0.9999999983597104    0.9999999995256709
0.09701630350757057     0.9999999983368575    0.999999999502622
0.09719076466301263     0.999999998311724     0.9999999994949555
0.09736204223776415     0.999999998286674     0.9999999994964576
0.0975218015093714      0.9999999982629689    0.9999999994933934
0.0976950466451956      0.9999999982368876    0.999999999473707
0.09785677347787552     0.9999999982121835    0.9999999994373362
0.09801531672986492     0.9999999981876273    0.9999999993912456
0.09818734584607126     0.9999999981605984    0.9999999993431392
0.09834785665913334     0.9999999981350147    0.9999999993119413
0.09852185333641236     0.9999999981068792    0.9999999992981038
0.09868433171054711     0.9999999980802238    0.9999999992981077
0.09884362650399132     0.9999999980537279    0.9999999992977825
0.0990164071616525      0.9999999980245781    0.9999999992827371
0.0991776695161694      0.9999999979969816    0.9999999992484153
0.09935241773490323     0.9999999979666472    0.9999999991939648
0.09952398237294655     0.9999999979364252    0.9999999991378646
0.09968402870784558     0.9999999979078348    0.9999999990978944
0.09985756090696157     0.9999999978763967    0.9999999990766013
0.10001957480293329     0.9999999978466291    0.9999999990738747
0.10017840511821448     0.9999999978170522    0.9999999990749106
0.10035072129771261     0.9999999977848658    0.9999999990634457
0.10051151917406646     0.9999999977547355    0.9999999990310035
0.10068580291463727     0.9999999977216074    0.999999998973845
0.10085690307451756     0.9999999976885999    0.9999999989097137
0.10101648493125356     0.9999999976573747    0.9999999988596016
0.10118955265220651     0.9999999976230232    0.9999999988284726
0.1013511020700152      0.9999999975904927    0.9999999988209666
0.10150946790713335     0.9999999975581608    0.9999999988224209
0.10168131960846845     0.9999999975225726    0.9999999988151237
0.10184165300665927     0.9999999974888903    0.9999999987862184
0.10201547226906706     0.9999999974518444    0.9999999987282097
0.10217777322833056     0.9999999974167478    0.9999999986601983
0.10233689060690353     0.9999999973818589    0.9999999985981728
0.10250949384969345     0.999999997343467     0.999999998553218
0.10267057878933909     0.9999999973071176    0.9999999985370629
0.10284514959320169     0.9999999972671507    0.9999999985370341
0.10301653681637375     0.9999999972273234    0.9999999985331974
0.10317640573640155     0.9999999971896391    0.9999999985081397
0.10334976052064629     0.9999999971481861    0.999999998450328
0.10351159700174675     0.9999999971089263    0.9999999983763388
0.10367024990215669     0.9999999970699062    0.9999999983033823
0.10384238866678358     0.9999999970269652    0.999999998244786
0.1040030091282662      0.9999999969863235    0.9999999982189286
0.10417711545396575     0.9999999969416355    0.9999999982152434
0.10434803819897479     0.9999999968971153    0.9999999982145122
0.10450744264083954     0.9999999968550083    0.9999999981947074
0.10468033294692125     0.9999999968086907    0.9999999981395179
0.10484170494985869     0.9999999967648431    0.99999999806123
0.10501656281701308     0.9999999967166519    0.99999999796897
0.10518823710347693     0.9999999966686428    0.99999999789734
0.1053483930867965      0.9999999966232263    0.9999999978617787
0.10552203493433303     0.9999999965732923    0.9999999978534566
0.10568415847872528     0.9999999965260122    0.9999999978543215
0.10584309844242701     0.9999999964790367    0.9999999978394877
0.10601552427034568     0.9999999964273691    0.9999999977877838
0.10617643179512008     0.999999996378483     0.9999999977064586
0.10635082518411143     0.9999999963247613    0.9999999976025846
0.10652203499241225     0.999999996271265     0.9999999975144276
0.1066817264975688      0.999999996221648     0.9999999974644723
0.1068549038669423      0.9999999961671524    0.9999999974473552
0.10701656293317152     0.9999999961155601    0.9999999974483645
0.1071750384187102      0.9999999960642969    0.9999999974391425
0.10734699976846585     0.9999999960078901    0.9999999973934971
0.10750744281507721     0.9999999959545168    0.9999999973118013
0.10768137172590553     0.9999999958958325    0.9999999971975724
0.10784378233358957     0.9999999958402482    0.9999999970959228
0.10800300936058309     0.9999999957850053    0.9999999970260843
0.10817572225179355     0.9999999957242336    0.9999999969937208
0.10833691683985973     0.9999999956667055    0.9999999969914063
0.10851159729214287     0.9999999956034693    0.9999999969869158
0.10868309416373548     0.9999999955404668    0.9999999969476352
0.10884307273218381     0.9999999954808635    0.9999999968669013
0.10901653716484909     0.9999999954153155    0.9999999967437909
0.1091784832943701      0.9999999953532439    0.9999999966253635
0.10933724584320058     0.9999999952915606    0.9999999965361639
0.10950949425624801     0.9999999952236938    0.9999999964871762
0.10967022436615116     0.999999995159468     0.9999999964786805
0.10984444034027127     0.9999999950888624    0.9999999964779444
0.11001547273370084     0.9999999950185312    0.9999999964471226
0.11017498682398613     0.9999999949520179    0.9999999963707045
0.11034798677848838     0.9999999948788666    0.9999999962415606
0.11050946842984635     0.9999999948096215    0.9999999961064262
0.11068443594542128     0.9999999947335296    0.9999999959855932
0.11085621988030567     0.9999999946577327    0.99999999592026
0.11101648551204579     0.999999994586033     0.99999999590448
0.11119023700800286     0.9999999945072129    0.9999999959049856
0.11135247020081565     0.9999999944325852    0.9999999958822127
0.11151151981293791     0.9999999943584432    0.9999999958111166
0.11168405528927713     0.9999999942769064    0.9999999956778539
0.11184507246247208     0.9999999941997612    0.9999999955272106
0.11201957549988396     0.9999999941149951    0.9999999953811068
0.11219089495660532     0.9999999940305893    0.9999999952918107
0.1123506961101824      0.9999999939507872    0.9999999952626818
0.11252398312797644     0.9999999938630697    0.999999995262268
0.1126857518426262      0.9999999937800622    0.9999999952479034
0.11284433697658544     0.9999999936980845    0.999999995185596
0.11301640797476162     0.9999999936091443    0.9999999950525377
0.11317696066979352     0.999999993525011     0.9999999948883587
0.11335099922904238     0.9999999934325401    0.9999999947150741
0.11351351948514696     0.999999993344977     0.9999999946004641
0.11367285616056101     0.9999999932579753    0.999999994548814
0.11384567870019201     0.9999999931622988    0.9999999945414485
0.11400698293667874     0.9999999930717499    0.9999999945363569
0.11418177303738242     0.9999999929722486    0.9999999944806689
0.11435337955739557     0.9999999928731406    0.99999999434974
0.11451346777426444     0.9999999927793985    0.9999999941740364
0.11468704185535027     0.9999999926763369    0.9999999939741697
0.11484909763329182     0.9999999925787597    0.9999999938295066
0.11500796983054283     0.9999999924818124    0.9999999937538382
0.1151803278920108      0.9999999923751751    0.9999999937352222
0.11534116765033449     0.9999999922742753    0.9999999937344793
0.11551549327287514     0.9999999921633805    0.9999999936915815
0.11568663531472524     0.9999999920529374    0.9999999935680985
0.11584625905343109     0.9999999919485042    0.9999999933849467
0.11601936865635387     0.999999991833675     0.999999993158869
0.11618095995613238     0.9999999917249913    0.9999999929798028
0.11633936767522036     0.9999999916170306    0.9999999928729083
0.1165112612585253      0.9999999914982709    0.9999999928354264
0.11667163653868597     0.9999999913859422    0.999999992835462
0.11684549768306357     0.9999999912624826    0.9999999928070907
0.11700784052429691     0.999999991145601     0.9999999927046723
0.11716699978483971     0.9999999910294927    0.9999999925236723
0.11733964490959947     0.9999999909018268    0.9999999922750981
0.11750077173121495     0.9999999907810457    0.9999999920561388
0.11767538441704739     0.9999999906483558    0.9999999918951296
0.11784681352218929     0.9999999905162431    0.9999999918331202
0.1180067243241869      0.9999999903913435    0.9999999918289437
0.11818012099040148     0.9999999902540762    0.9999999918123946
0.11834199935347178     0.9999999901241858    0.9999999917240022
0.11850069413585156     0.9999999899952007    0.9999999915458212
0.11867287478244827     0.9999999898533882    0.9999999912785665
0.11883353712590072     0.9999999897192929    0.9999999910235389
0.11900768533357012     0.9999999895719918    0.9999999908163357
0.11917864996054899     0.9999999894271921    0.999999990719833
0.11933809628438358     0.999999989291187     0.9999999907043524
0.11951102847243512     0.9999999891416776    0.9999999906979887
0.1196724423573424      0.9999999890002187    0.9999999906282018
0.11984734210646661     0.999999988844829     0.9999999904369897
0.1200190582749003      0.9999999886900978    0.9999999901508696
0.1201792561401897      0.9999999885437777    0.9999999898618797
0.12035293986969607     0.9999999883829619    0.9999999896109493
0.12051510529605816     0.999999988230736     0.9999999894837248
0.12067408714172972     0.9999999880795258    0.9999999894504081
0.12084655485161823     0.9999999879132496    0.9999999894489047
0.12100750425836246     0.9999999877559472    0.9999999893965017
0.12118193952932364     0.9999999875831089    0.9999999892185728
0.1213531912195943      0.9999999874110088    0.9999999889224498
0.12151292460672068     0.9999999872482972    0.9999999885988735
0.121686143858064       0.9999999870694309    0.9999999882930138
0.12184784480626305     0.9999999869001593    0.9999999881170185
0.12200636217377157     0.9999999867320383    0.9999999880551962
0.12217836540549705     0.99999998654714      0.9999999880522957
0.12233885033407825     0.9999999863722725    0.9999999880190286
0.1225128211268764      0.9999999861801149    0.9999999878629773
0.12267527361653029     0.9999999859982122    0.999999987583667
0.12283454252549363     0.9999999858175329    0.9999999872316261
0.12300729729867392     0.9999999856189035    0.99999998686311
0.12316853376870994     0.999999985430999     0.9999999866203029
0.12334325610296291     0.9999999852246007    0.9999999865050202
0.12351479485652535     0.9999999850191196    0.9999999864927714
0.12367481530694352     0.999999984824867     0.999999986474488
0.12384832162157863     0.9999999846114096    0.9999999863411326
0.12401030963306947     0.9999999844094319    0.9999999860669372
0.12416911406386978     0.9999999842088741    0.9999999856911992
0.12434140435888705     0.9999999839883993    0.9999999852662882
0.12450217635076004     0.9999999837799266    0.9999999849590621
0.12467643420684997     0.9999999835509502    0.9999999847877652
0.12484750848224938     0.9999999833230718    0.9999999847545223
0.12500706445450452     0.9999999831077524    0.9999999847483435
0.12518010629097662     0.9999999828711651    0.9999999846443276
0.12534162982430444     0.9999999826476615    0.9999999843865897
0.12551663922184922     0.9999999824062392    0.999999983950535
0.12568846503870346     0.9999999821659126    0.9999999834722729
0.12584877255241342     0.9999999819387057    0.9999999831044992
0.12602256593034034     0.9999999816890736    0.999999982878467
0.12618484100512298     0.99999998145283      0.9999999828206059
0.12634393249921508     0.999999981218219     0.9999999828190381
0.12651650985752413     0.9999999809603117    0.9999999827417938
0.12667756891268891     0.9999999807163725    0.9999999825038312
0.12685211383207065     0.9999999804484191    0.999999982056551
0.12702347517076185     0.9999999801816707    0.999999981524805
0.12718331820630877     0.9999999799295178    0.9999999810816986
0.12735664710607264     0.9999999796524042    0.9999999807759671
0.12751845770269224     0.9999999793901978    0.9999999806732682
0.12767708471862133     0.9999999791298197    0.9999999806672663
0.12784919759876737     0.9999999788435276    0.9999999806188864
0.12800979217576913     0.9999999785728044    0.9999999804120492
0.12818387261698783     0.9999999782753821    0.9999999799682794
0.128354769477516       0.9999999779793357    0.9999999793902721
0.1285141480348999      0.9999999776995695    0.9999999788666593
0.12868701245650077     0.9999999773920738    0.9999999784636255
0.12884835857495736     0.9999999771012129    0.9999999782956221
0.1290231905576309      0.999999976781789     0.9999999782708084
0.1291948389596139      0.9999999764638277    0.9999999782396377
0.12935496905845262     0.9999999761632677    0.9999999780538557
0.1295285850215083      0.9999999758330536    0.9999999776083297
0.1296906826814197      0.9999999755206235    0.9999999770184526
0.12984959676064056     0.9999999752104212    0.9999999764169606
0.13002199670407838     0.9999999748694701    0.9999999759108404
0.13018287834437192     0.9999999745471007    0.9999999756649246
0.13035724584888242     0.9999999741930814    0.9999999756055195
0.13052842977270238     0.9999999738407925    0.9999999755932415
0.13068809539337806     0.9999999735079353    0.9999999754476976
0.1308612468782707      0.9999999731422539    0.9999999750280322
0.13102288006001905     0.9999999727964296    0.9999999744138971
0.13118132966107687     0.9999999724531863    0.9999999737364492
0.13135326512635165     0.9999999720759429    0.9999999731131348
0.13151368228848215     0.9999999717194353    0.9999999727658623
0.1316875853148296      0.9999999713312798    0.9999999726471165
0.13184997003803278     0.9999999709669971    0.999999972643384
0.13200917118054542     0.999999970605333     0.9999999725531672
0.13218185818727501     0.999999970207894     0.9999999721944908
0.13234302689086033     0.99999996983207      0.9999999715876046
0.1325176814586626      0.9999999694193847    0.9999999707666897
0.13268915244577437     0.9999999690086637    0.9999999700251341
0.13284910512974185     0.9999999686204968    0.9999999695658177
0.13302254367792626     0.9999999681940333    0.9999999693691767
0.1331844639229664      0.9999999677905901    0.9999999693548519
0.13334320058731602     0.9999999673900433    0.9999999693013468
0.1335154231158826      0.9999999669497531    0.9999999689930067
0.1336761273413049      0.9999999665334768    0.9999999683964561
0.13385031743094417     0.9999999660762657    0.9999999675145314
0.13402132393989288     0.9999999656212585    0.9999999666479983
0.13418081214569733     0.9999999651913346    0.9999999660537845
0.13435378621571872     0.9999999647189104    0.9999999657470634
0.13451524198259585     0.9999999642721027    0.9999999656994296
0.13469018361368992     0.9999999637815278    0.9999999656673841
0.13486194166409346     0.999999963293277     0.9999999653921796
0.13502218141135272     0.9999999628317984    0.9999999647967276
0.13519590702282894     0.9999999623248922    0.9999999638547062
0.13535811433116088     0.9999999618453348    0.9999999629168912
0.13551713805880228     0.9999999613692541    0.9999999621813254
0.13568964765066063     0.9999999608460798    0.9999999617460852
0.1358506389393747      0.9999999603514588    0.9999999616437706
0.13602511609230575     0.9999999598083723    0.9999999616317776
0.13619640966454624     0.9999999592679999    0.9999999614187713
0.13635618493364246     0.9999999587574668    0.999999960862563
0.13652944606695563     0.9999999581966781    0.9999999598889497
0.13669118889712453     0.999999957666372     0.9999999588388977
0.1368497481466029      0.9999999571400655    0.9999999579442329
0.1370217932602982      0.9999999565617107    0.9999999573443862
0.13718232007084924     0.9999999560151696    0.9999999571532284
0.13735633274561723     0.9999999554151041    0.9999999571417937
0.13751882711724095     0.999999954847539     0.9999999570134441
0.13767813790817415     0.9999999542842577    0.9999999565437513
0.13785093456332428     0.9999999536655668    0.999999955589656
0.13801221291533014     0.9999999530891273    0.9999999544464194
0.13818697713155295     0.9999999524572184    0.9999999532717044
0.13835855776708525     0.9999999518284992    0.9999999524921063
0.13851862009947327     0.9999999512344434    0.9999999521881637
0.13869216829607825     0.9999999505819894    0.999999952151471
0.13885419818953895     0.9999999499648901    0.9999999520758335
0.1390130445023091      0.9999999493523615    0.9999999516757004
0.13918537667929623     0.9999999486792567    0.9999999507428438
0.13934619055313907     0.9999999480429852    0.9999999495237634
0.13952049029119887     0.9999999473443433    0.9999999481680361
0.13969160644856812     0.9999999466492199    0.9999999471740318
0.1398512043027931      0.9999999459925251    0.9999999467139621
0.14002428802123504     0.999999945271099     0.9999999466152825
0.1401858534365327      0.999999944588896     0.9999999465841369
0.14034423527113982     0.9999999439117986    0.9999999462728895
0.1405161029699639      0.999999943167596     0.9999999453997387
0.1406764523656437      0.9999999424642909    0.9999999441353875
0.14085028762554044     0.9999999416919253    0.9999999426036733
0.14101260458229292     0.9999999409613176    0.9999999414133947
0.14117173795835486     0.9999999402361057    0.9999999407204697
0.14134435719863375     0.9999999394393183    0.999999940487793
0.14150545813576837     0.9999999386860917    0.9999999404787839
0.14168004493711994     0.9999999378592186    0.9999999402432003
0.14185144815778097     0.9999999370365813    0.999999939441012
0.14201133307529773     0.999999936259432     0.9999999381523961
0.14218470385703144     0.9999999354059308    0.9999999364626032
0.14234655633562088     0.999999934598881     0.9999999350379083
0.1425052252335198      0.9999999337979925    0.9999999341128435
0.14267737999563565     0.9999999329180413    0.9999999337190636
0.14283801645460723     0.9999999320865399    0.9999999336897207
0.14301213877779576     0.9999999311737443    0.9999999335454948
0.14318307752029377     0.9999999302658914    0.9999999328518838
0.1433424979596475      0.9999999294086154    0.9999999315826124
0.1435154042632182      0.9999999284671401    0.999999929761515
0.14367679226364463     0.9999999275773073    0.9999999280898184
0.14385166612828798     0.9999999266009252    0.9999999267939776
0.14402335641224082     0.9999999256298568    0.999999926237035
0.14418352839304938     0.9999999247171321    0.9999999261652134
0.1443571862380749      0.9999999237250533    0.9999999260745803
0.14451932577995613     0.9999999227869604    0.999999925503398
0.14467828174114683     0.9999999218560521    0.9999999242730019
0.14485072356655448     0.9999999208334105    0.9999999223471706
0.14501164708881786     0.999999919866938     0.9999999204403093
0.1451860564752982      0.9999999188060537    0.9999999188229215
0.145357282281088       0.9999999177507686    0.999999918009205
0.1455169897837335      0.9999999167540121    0.9999999178372485
0.14569018315059598     0.9999999156593145    0.9999999178015213
0.14585185821431418     0.999999914624319     0.9999999173610609
0.14601034969734183     0.9999999135972685    0.9999999162201874
0.14618232704458645     0.999999912468725     0.9999999142387691
0.14634278608868678     0.9999999114023669    0.9999999121075953
0.14651673099700407     0.9999999102315905    0.9999999101286522
0.1466791576021771      0.9999999091242742    0.9999999090175192
0.14683840062665957     0.9999999080253148    0.9999999086444609
0.147011129515359       0.9999999068181701    0.999999908621384
0.14717234010091415     0.9999999056771642    0.999999908354959
0.14734703655068626     0.9999999044248713    0.9999999072523857
0.14751854941976783     0.9999999031791853    0.9999999052456308
0.14767854398570512     0.9999999020025055    0.9999999029149559
0.14785202441585937     0.999999900710482     0.9999999005757028
0.14801398654286935     0.999999899488897     0.999999899114503
0.1481727650891888      0.9999998982767818    0.999999898509621
0.14834502949972522     0.9999998969452562    0.9999998984359201
0.14850577560711736     0.9999998956871474    0.9999998982828696
0.14868000757872643     0.99999989430628      0.9999998973461833
0.14885105596964499     0.999999892933043     0.9999998953816124
0.14901058605741926     0.9999998916363962    0.9999998928902998
0.1491836020094105      0.9999998902126288    0.9999998901758796
0.14934509965825746     0.9999998888670403    0.9999998882970575
0.14952008317132137     0.9999998873908021    0.9999998873221625
0.14969188310369475     0.9999998859227299    0.9999998871681444
0.14985216473292384     0.9999998845362164    0.9999998870783472
0.1500259322263699      0.9999998830144644    0.9999998862529155
0.15018818141667167     0.9999998815759645    0.9999998844224973
0.1503472470262829      0.9999998801490151    0.999999881804923
0.1505197985001111      0.9999998785977323    0.9999998787354392
0.15068083167079502     0.999999877138059     0.9999998764234171
0.1508553507056959      0.999999875536329     0.9999998750477619
0.15102668615990622     0.999999873943482     0.9999998747188972
0.15118650331097228     0.9999998724393012    0.9999998746882582
0.1513598063262553      0.9999998707878288    0.9999998740607078
0.15152159103839402     0.9999998692267417    0.9999998723661118
0.15168019216984222     0.9999998676779734    0.9999998696892747
0.15185227916550736     0.9999998659766413    0.9999998662852232
0.15201284785802824     0.9999998643693448    0.9999998634911762
0.15218690241476607     0.9999998626051422    0.9999998616063248
0.15235777339081336     0.9999998608507518    0.9999998609920194
0.15251712606371637     0.9999998591943223    0.999999860964068
0.15268996460083634     0.9999998573752868    0.9999998605530845
0.15285128483481203     0.9999998556561698    0.999999859073713
0.15302609093300468     0.9999998537698338    0.9999998560955027
0.1531977134505068      0.9999998518937807    0.9999998523797612
0.15335781766486467     0.9999998501218798    0.9999998491431588
0.15353140774343946     0.9999998481767208    0.9999998467764345
0.15369347951886997     0.9999998463378279    0.9999998458788603
0.15385236771360997     0.9999998445134423    0.9999998457797145
0.15402474177256692     0.999999842509741     0.999999845539524
0.1541855975283796      0.9999998406167118    0.999999844282955
0.15435993914840923     0.9999998385393793    0.9999998413774627
0.15453109718774832     0.9999998364737969    0.9999998374102056
0.15469073692394314     0.9999998345235857    0.9999998336709176
0.1548638625243549      0.9999998323825677    0.9999998306536584
0.1550254698216224      0.9999998303592726    0.9999998292853376
0.15518389353819936     0.9999998283524276    0.9999998290112032
0.15535580311899327     0.9999998261482667    0.9999998289108528
0.1555161943966429      0.999999824066686     0.9999998279359587
0.1556900715385095      0.9999998217824301    0.9999998252223884
0.15585243037723182     0.9999998196232616    0.9999998213229785
0.1560116056352636      0.9999998174815987    0.9999998170869644
0.15618426675751232     0.9999998151304197    0.9999998132623358
0.15634540957661677     0.9999998129094875    0.9999998111906688
0.15652003825993818     0.999999810473408     0.9999998105258514
0.15669148336256905     0.9999998080569326    0.999999810484448
0.15685141016205564     0.9999998058010433    0.9999998097656573
0.1570248228257592      0.9999998033250701    0.9999998072798487
0.15718671718631846     0.9999998009851344    0.999999803312701
0.1573454279661872      0.9999997986642147    0.9999997986597955
0.15751762461027288     0.9999997961154501    0.9999997941009559
0.1576783029512143      0.9999997937080446    0.9999997913303814
0.15785246715637266     0.9999997910664018    0.9999997901891448
0.15802344778084051     0.9999997884400553    0.9999997901123947
0.1581829101021641      0.999999785960799     0.9999997896650618
0.1583558582877046      0.9999997832389047    0.999999787521853
0.15851728817010083     0.9999997806669562    0.9999997836214988
0.15869220391671401     0.999999777845571     0.9999997780859623
0.1588639360826367      0.9999997750401197    0.9999997728540834
0.15902414994541508     0.999999772390809     0.9999997694307804
0.15919784967241044     0.9999997694831508    0.999999767805895
0.1593600310962615      0.9999997667347241    0.9999997676159917
0.15951902893942205     0.9999997640084085    0.9999997673737863
0.15969151264679954     0.9999997610148019    0.9999997655746582
0.15985247805103275     0.9999997581868859    0.9999997618034677
0.16002692931948292     0.9999997550843152    0.9999997559569412
0.16019819700724255     0.9999997519997579    0.9999997499720262
0.1603579463918579      0.9999997490877882    0.9999997456779737
0.16053118164069022     0.9999997458915525    0.9999997432903422
0.16069289858637825     0.9999997428713537    0.9999997428166739
0.16085143195137575     0.9999997398760628    0.9999997427299286
0.1610234511805902      0.9999997365868815    0.9999997413554307
0.16118395210666037     0.9999997334808769    0.9999997378677882
0.1613579388969475      0.9999997300730578    0.9999997318608748
0.16152040738409035     0.9999997268520987    0.9999997254701732
0.16167969229054266     0.9999997236575666    0.9999997201101581
0.16185246306121193     0.9999997201510968    0.9999997166052053
0.16201371552873692     0.999999716839081     0.9999997155483624
0.16218845386047887     0.9999997132067944    0.99999971548351
0.16236000861153027     0.9999997095964605    0.9999997144859472
0.1625200450594374      0.9999997061886402    0.9999997113187591
0.1626935673715615      0.9999997024497307    0.9999997052480313
0.1628555713805413      0.9999996989173764    0.9999996982630113
0.1630143918088306      0.9999996954380734    0.999999691941949
0.16318669810133682     0.9999996916412254    0.9999996873440821
0.16334748609069877     0.9999996880557559    0.9999996856091811
0.16352175994427767     0.9999996841226669    0.9999996854201252
0.16369285021716606     0.9999996802133381    0.9999996848132288
0.16385242218691018     0.9999996765236887    0.999999682110983
0.16402548002087125     0.9999996724741343    0.999999676181179
0.16418701955168805     0.9999996686483925    0.9999996687204561
0.16436204494672177     0.9999996644528025    0.9999996607157211
0.16453388676106498     0.999999660281822     0.9999996551068638
0.1646942102722639      0.9999996563436465    0.9999996527016533
0.1648680196476798      0.9999996520225765    0.9999996522931915
0.16503031071995142     0.9999996479387955    0.9999996519690814
0.1651894182115325      0.999999643888585     0.9999996497231725
0.16536201156733052     0.9999996394423775    0.9999996440178488
0.16552308661998427     0.9999996352428538    0.9999996361914214
0.16569764753685498     0.9999996306365452    0.9999996271366832
0.16586902487303515     0.999999626057765     0.9999996201892362
0.16602888390607104     0.9999996217357114    0.9999996167404033
0.1662022288033239      0.9999996169927788    0.9999996158391108
0.16636405539743246     0.9999996125115925    0.9999996157223892
0.1665226984108505      0.9999996080679683    0.9999996140303786
0.16669482728848548     0.9999996031893446    0.9999996087793455
0.16685543786297619     0.99999959858288      0.9999996007895928
0.16702953430168385     0.9999995935297897    0.9999995907477309
0.16720044715970098     0.9999995885079156    0.9999995823039552
0.16735984171457383     0.9999995837692839    0.9999995775208159
0.16753272213366363     0.9999995785688945    0.9999995758021797
0.16769408424960916     0.9999995736572992    0.9999995757297683
0.16786893222977164     0.9999995682717044    0.9999995743736355
0.1680405966292436      0.9999995629192604    0.9999995694180313
0.16820074272557126     0.9999995578673966    0.9999995612286395
0.16837437468611588     0.999999552325629     0.9999995502843857
0.16853648834351623     0.9999995470903498    0.9999995408923772
0.16869541842022606     0.9999995419000306    0.9999995346633489
0.16886783436115285     0.9999995362039193    0.9999995318999843
0.16902873199893537     0.9999995308264099    0.9999995316513914
0.1692031155009348      0.9999995249323188    0.9999995308366995
0.16937431542224374     0.9999995191426003    0.9999995265810974
0.1695339970404084      0.9999995136794133    0.9999995185773957
0.16970716452279        0.999999507685193     0.9999995069050668
0.16986881370202733     0.9999995020234443    0.9999994960413318
0.17002727930057412     0.9999994964103323    0.9999994880999581
0.17019923076333787     0.9999994902482345    0.9999994838902762
0.17035966392295734     0.9999994844311711    0.9999994831638538
0.17053358294679377     0.9999994780503366    0.9999994828065831
0.17069598366748592     0.999999472020986     0.9999994797188919
0.17085520080748753     0.9999994660423668    0.9999994724487878
0.1710279038117061      0.9999994594809383    0.9999994604848037
0.1711890885127804      0.9999994532845439    0.9999994481601715
0.17136375907807164     0.9999994464896643    0.999999437256491
0.17153524606267234     0.9999994397366793    0.9999994313937272
0.17169521474412877     0.9999994333632293    0.9999994299220751
0.17186866928980216     0.9999994263707953    0.9999994297833339
0.17203060553233127     0.9999994197651411    0.9999994274802232
0.17218935819416983     0.999999413215861     0.999999420851476
0.17236159672022536     0.9999994060270485    0.9999994087285767
0.1725223169431366      0.9999993992400757    0.9999993951963108
0.1726965230302648      0.9999993917966518    0.9999993821675188
0.17286754553670247     0.9999993844003139    0.9999993742239188
0.17302704973999586     0.9999993774218606    0.9999993715782857
0.1732000398075062      0.9999993697649173    0.9999993713889528
0.17336151157187227     0.9999993625338918    0.9999993699240388
0.1735364692004553      0.9999993546065464    0.9999993633379453
0.1737082432483478      0.999999346729004     0.999999350965837
0.17386849899309603     0.9999993392944643    0.9999993363166964
0.1740422406020612      0.999999331140423     0.9999993213616041
0.17420446390788208     0.9999993234379452    0.9999993118199038
0.17436350363301245     0.9999993158023817    0.9999993077219532
0.17453602922235978     0.9999993074241486    0.9999993071518622
0.17469703650856283     0.999999299515059     0.999999306332206
0.1748715296589828      0.9999992908442288    0.9999993007744534
0.17504283922871228     0.9999992822301884    0.9999992887516743
0.17520263049529747     0.9999992741039186    0.9999992732476535
0.17537590762609961     0.9999992651911883    0.9999992561283428
0.17553766645375748     0.9999992568181962    0.9999992441001437
0.17569624170072481     0.9999992485696125    0.9999992380322604
0.1758683028119091      0.9999992395170954    0.9999992366139697
0.1760288456199491      0.9999992309732174    0.9999992363036598
0.17620287429220607     0.9999992216041087    0.9999992320188823
0.17636538466131876     0.9999992127528455    0.9999992215341553
0.1765247114497409      0.9999992039779263    0.9999992057760703
0.17669752410238002     0.9999991943503244    0.99999918655616
0.17685881845187484     0.9999991852599637    0.9999991714692528
0.17703359866558663     0.9999991752943164    0.9999991619418371
0.17720519529860787     0.9999991653921882    0.9999991592879349
0.17736527362848484     0.9999991560480814    0.9999991591698717
0.17753883782257876     0.9999991457991014    0.9999991560246413
0.1777008837135284      0.9999991361185187    0.999999146476263
0.1778597460237875      0.9999991265221931    0.9999991305983633
0.17803209419826357     0.9999991159913209    0.999999109643606
0.17819292406959536     0.9999991060504622    0.9999990918179892
0.1783672398051441      0.9999990951506248    0.9999990792201758
0.17853837196000233     0.9999990843215564    0.9999990746844767
0.17869798581171628     0.9999990741055523    0.9999990743905098
0.17887108552764716     0.999999062898718     0.9999990724359609
0.17903266694043377     0.9999990523164836    0.9999990642505416
0.17919106477252986     0.9999990418280754    0.9999990487929893
0.1793629484688429      0.9999990303169611    0.999999026476726
0.17952331386201167     0.9999990194542648    0.9999990058269638
0.17969716511939737     0.999999007542618     0.9999989895795532
0.17985949807363882     0.999998996291684     0.9999989825384221
0.18001864744718973     0.999998985139775     0.9999989813147322
0.1801912826849576      0.9999989729054217    0.9999989805573202
0.1803523996195812      0.9999989613570728    0.9999989745022216
0.18052700241842173     0.9999989486987086    0.9999989586088041
0.18069842163657174     0.9999989361246608    0.9999989352824362
0.18085832255157747     0.999998924263434     0.9999989120152368
0.18103170933080015     0.9999989112564301    0.9999988920196307
0.18119357780687856     0.9999988989757006    0.9999988819860469
0.18135226270226643     0.9999988868061802    0.9999988793614558
0.18152443346187125     0.9999988734551895    0.9999988790898827
0.1816850859183318      0.9999988608579634    0.9999988746350058
0.1818592242390093      0.9999988471562913    0.999998860159277
0.18203017897899626     0.9999988335849643    0.9999988364425876
0.18218961541583895     0.9999988207859893    0.999998810756368
0.1823625377168986      0.9999988067475067    0.9999987866250836
0.18252394171481395     0.9999987934950036    0.9999987728020556
0.18269883157694627     0.999998778970663     0.9999987677848695
0.18287053785838805     0.9999987645421813    0.9999987675455307
0.18303072583668556     0.9999987509292121    0.9999987641336541
0.18320439967920002     0.9999987360020477    0.9999987506184349
0.1833665552185702      0.9999987219051341    0.9999987278019213
0.18352552717724988     0.9999987079335353    0.9999987000449785
0.18369798500014647     0.9999986926052372    0.9999986718872326
0.1838589245198988      0.9999986781379623    0.9999986539974481
0.18403334990386808     0.9999986622789914    0.9999986459850712
0.18420459170714684     0.9999986465259055    0.9999986452501162
0.18436431520728133     0.9999986316666879    0.9999986432609422
0.18453752457163275     0.9999986153701127    0.9999986318222828
0.18469921563283992     0.9999985999838356    0.9999986097046096
0.18485772311335655     0.9999985847362826    0.9999985803916084
0.18502971645809013     0.9999985680056551    0.9999985481364627
0.18519019149967944     0.9999985522193098    0.9999985254829337
0.1853641524054857      0.9999985349122497    0.9999985133626973
0.1855265950081477      0.9999985185670067    0.9999985112331005
0.18568585403011914     0.9999985023677714    0.999998510584837
0.18585859891630754     0.9999984845997492    0.999998502300197
0.18601982549935167     0.9999984678296726    0.9999984824179825
0.18619453794661275     0.9999984494512382    0.9999984492517284
0.1863660668131833      0.9999984311976027    0.9999984132260906
0.18652607737660956     0.9999984139802465    0.9999983857377587
0.18669957380425278     0.9999983951031717    0.9999983689633497
0.18686155192875173     0.9999983772816009    0.9999983646885572
0.18702034647256013     0.9999983596232584    0.9999983644752338
0.1871926268805855      0.9999983402539165    0.9999983584931844
0.18735338898546658     0.9999983219794154    0.9999983405673533
0.18752763695456462     0.9999983019517031    0.9999983071920228
0.18769870134297212     0.9999982820650524    0.9999982677346014
0.18785824742823534     0.9999982633148568    0.9999982349708961
0.18803127937771552     0.999998242796879     0.9999982123954053
0.18819279302405142     0.9999982235854036    0.9999982047772745
0.18836779253460428     0.9999982025366404    0.9999982042277997
0.1885396084644666      0.9999981816318934    0.9999981997030414
0.18869990609118464     0.9999981619125115    0.9999981830702012
0.18887368958211964     0.9999981402956564    0.9999981492482195
0.18903595476991036     0.9999981198849452    0.9999981087954012
0.18919503637701057     0.9999980996597784    0.9999980709368796
0.1893676038483277      0.9999980774767141    0.999998042214571
0.1895286530165006      0.9999980565432744    0.9999980304960364
0.18970318804889041     0.9999980336021859    0.9999980288320941
0.18987453950058972     0.9999980108188671    0.9999980262640694
0.19003437264914474     0.9999979893316251    0.9999980123400835
0.19020769166191673     0.9999979657717688    0.9999979797545374
0.19036949237154444     0.9999979435312154    0.9999979371851522
0.1905281095004816      0.9999979214948544    0.9999978941923309
0.19070021249363572     0.9999978973208068    0.9999978583407204
0.19086079718364557     0.9999978745141892    0.9999978411135793
0.19103486773787237     0.9999978495163084    0.9999978370205375
0.19120575471140863     0.9999978246934578    0.9999978360047385
0.19136512338180062     0.9999978012892404    0.9999978253847273
0.19153797791640956     0.9999977756240555    0.9999977953521182
0.19169931414787422     0.9999977514031564    0.9999977518007055
0.19187413624355584     0.9999977248649006    0.9999976992059513
0.19204577475854692     0.9999976985107395    0.9999976569229924
0.19220589497039373     0.9999976736552801    0.9999976344783793
0.19237950104645749     0.9999976464092298    0.9999976275630055
0.19254158881937697     0.9999976206891631    0.9999976272182521
0.1927004930116059      0.999997595207725     0.9999976195407483
0.1928728830680518      0.9999975672625505    0.9999975923982634
0.19303375482135343     0.9999975408992534    0.9999975486399095
0.193208112438872       0.9999975120120104    0.9999974913579224
0.19337928647570005     0.999997483331655     0.9999974411985211
0.1935389422093838      0.9999974562923392    0.9999974112613478
0.19371208380728452     0.9999974266511098    0.9999973992877469
0.19387370710204097     0.9999973986806254    0.9999973986151571
0.1940321468161069      0.9999973709764101    0.9999973939699794
0.19420407239438975     0.999997340592534     0.9999973709153668
0.19436447966952836     0.9999973121059186    0.999997328449702
0.1945383728088839      0.999997280995314     0.9999972675301942
0.19470074764509515     0.9999972516264918    0.9999972117678543
0.1948599389006159      0.9999972225309635    0.9999971718923479
0.1950326160203536      0.9999971906280086    0.9999971516171638
0.19519377483694703     0.9999971605277821    0.9999971485910013
0.19536841951775738     0.9999971275499654    0.9999971462145607
0.19553988061787722     0.9999970948060586    0.9999971271665398
0.19569982341485279     0.999997063930052     0.9999970866391785
0.1958732520760453      0.9999970300847629    0.9999970231010625
0.19603516243409355     0.9999969981399452    0.9999969602787275
0.19619388921145126     0.9999969664941376    0.9999969112845464
0.19636610185302592     0.9999969317871644    0.999996882469898
0.1965267961914563      0.9999968990482748    0.9999968757664291
0.19670097639410364     0.9999968631725686    0.9999968749268022
0.19687197301606044     0.9999968275544969    0.9999968606521626
0.19703145133487296     0.9999967939764968    0.9999968236919086
0.19720441551790244     0.9999967571631888    0.9999967592510975
0.19736586139778764     0.9999967224260045    0.9999966899349073
0.1975407931418898      0.9999966843738338    0.9999966258606843
0.19771254130530141     0.9999966465917325    0.9999965890976941
0.19787277116556876     0.9999966109623424    0.9999965783874081
0.19804648689005305     0.9999965719144208    0.9999965777930996
0.19820868431139307     0.9999965350575953    0.9999965677627443
0.19836769815204255     0.9999964985475664    0.9999965346763552
0.198540197856909       0.9999964585155502    0.9999964703806421
0.19870117925863115     0.9999964207533757    0.9999963955848008
0.19887564652457027     0.9999963793840546    0.9999963207410314
0.19904693020981884     0.9999963383164582    0.9999962726950938
0.19920669559192317     0.9999962996021002    0.9999962550246929
0.19937994683824442     0.9999962571702593    0.9999962533022982
0.1995416797814214      0.9999962171335189    0.9999962473462386
0.19970022914390786     0.9999961774821852    0.9999962195693354
0.19987226437061129     0.9999961340033107    0.9999961576059955
0.20003278129417043     0.9999960930048832    0.9999960787570085
0.2002067840819465      0.9999960480883224    0.9999959929798905
0.20036926856657833     0.9999960056963925    0.999995933884192
0.2005285694705196      0.9999959637108589    0.9999959050864883
0.20070135623867785     0.9999959180844522    0.9999958990604635
0.20086262470369182     0.9999958750484249    0.9999958970264978
0.20103737903292274     0.9999958279119392    0.9999958733910792
0.20120894978146311     0.999995781120615     0.9999958145739768
0.20136900222685922     0.9999957370065584    0.999995732914768
0.20154254053647228     0.9999956886635917    0.9999956371673958
0.20170456054294106     0.9999956430432382    0.9999955652608182
0.2018633969687193      0.9999955978586776    0.9999955252709479
0.2020357192587145      0.9999955483164343    0.9999955131999922
0.20219652324556542     0.9999955015910976    0.9999955125892013
0.2023708130966333      0.9999954504020542    0.999995495386421
0.20254191936701063     0.9999953995905501    0.999995442038565
0.2027015073342437      0.9999953516964221    0.9999953597763966
0.2028745811656937      0.9999952992004626    0.9999952550105266
0.20303613669399945     0.9999952496721772    0.9999951691370906
0.20319450864161465     0.9999952006219671    0.9999951152475056
0.2033663664534468      0.9999951468317483    0.9999950937846949
0.20352670596213468     0.9999950961128988    0.9999950926257408
0.20370053133503951     0.9999950405402771    0.9999950820623441
0.20386283840480007     0.999994988092572     0.9999950397889411
0.20402196189387012     0.9999949361459164    0.9999949615214927
0.2041945712471571      0.9999948792003879    0.9999948503519814
0.2043556622972998      0.9999948254897764    0.9999947491815133
0.20453023921165944     0.9999947666608531    0.9999946714464251
0.20470163254532858     0.9999947082695099    0.9999946385108062
0.20486150757585345     0.999994653229648     0.9999946344937267
0.20503486847059524     0.9999945929166377    0.9999946288779805
0.20519671106219278     0.9999945360136032    0.9999945941053507
0.20535537007309979     0.9999944796650778    0.999994519855259
0.20552751494822374     0.99999441788891      0.9999944042739538
0.20568814152020343     0.9999943596421099    0.9999942903302729
0.20586225395640007     0.9999942958404984    0.9999941940204758
0.20603318281190616     0.9999942325271222    0.9999941457614399
0.20619259336426798     0.9999941728692828    0.9999941356680463
0.20636548978084676     0.9999941074922168    0.9999941336766834
0.20652686789428126     0.9999940458340868    0.9999941077943261
0.20670173187193272     0.9999939783237569    0.9999940310490978
0.20687341226889364     0.9999939115710786    0.9999939111462772
0.20703357436271028     0.9999938489338109    0.9999937859387745
0.20720722232074387     0.9999937803118808    0.9999936730566916
0.2073693519756332      0.9999937155669253    0.9999936122331968
0.20752829804983197     0.9999936514536861    0.9999935935074838
0.2077007299882477      0.9999935811767311    0.9999935925201382
0.20786164362351917     0.9999935149072268    0.9999935744053573
0.20803604312300758     0.9999934423265842    0.9999935059156447
0.20820725904180545     0.9999933702962936    0.9999933860680587
0.20836695665745905     0.9999933024125763    0.9999932504568829
0.2085401401373296      0.9999932280255399    0.9999931175879396
0.20870180531405588     0.9999931578547007    0.9999930370833371
0.20886028691009165     0.9999930883733776    0.9999930055078262
0.20903225437034434     0.9999930121964131    0.9999930024528914
0.20919270352745276     0.9999929403793867    0.9999929920704795
0.20936663854877813     0.9999928617083769    0.9999929348223588
0.20952905526695922     0.9999927874715052    0.9999928260324069
0.2096882884044498      0.9999927139555492    0.9999926837121368
0.20986100740615732     0.9999926333835372    0.9999925294156944
0.21002220810472058     0.9999925573982829    0.9999924231567708
0.21019689466750077     0.9999924741906376    0.9999923683749269
0.21036839764959045     0.9999923916151907    0.9999923599669696
0.21052838232853585     0.9999923137883144    0.9999923550402237
0.2107018528716982      0.9999922285233063    0.9999923086420683
0.2108638051117163      0.9999921480880481    0.9999922058399628
0.21102257377104383     0.9999920684472632    0.9999920590177764
0.21119482829458833     0.9999919811530202    0.9999918869639174
0.21135556451498855     0.9999918988543978    0.9999917573849957
0.21152978659960572     0.9999918087250762    0.9999916801580594
0.21170082510353236     0.9999917192975537    0.9999916616342232
0.21186034530431472     0.9999916350410151    0.9999916602520581
0.21203335136931403     0.9999915427245224    0.9999916264103533
0.21219483913116907     0.9999914556671129    0.9999915337713333
0.21236981275724107     0.999991360364334     0.9999913690200473
0.21254160280262252     0.999991265800365     0.9999911812053922
0.2127018745448597      0.9999911766801849    0.999991030982623
0.21287563215131383     0.9999910790751251    0.9999909330348695
0.2130378714546237      0.9999909870066928    0.9999909038533871
0.213196927177243       0.9999908964406226    0.9999909027146011
0.21336946876407928     0.9999907973174691    0.9999908795597148
0.21353049204777128     0.9999907038640848    0.9999907977498419
0.21370500119568023     0.9999906015395664    0.9999906346946927
0.21387632676289864     0.9999905000133485    0.9999904327418254
0.21403613402697277     0.9999904043481792    0.9999902579127795
0.21420942715526387     0.9999902995464514    0.9999901308874622
0.21437120198041068     0.9999902007014664    0.9999900832638993
0.21452979322486698     0.999990102846052     0.9999900783711937
0.2147018703335402      0.9999899955883704    0.9999900655611922
0.2148624291390692      0.9999898944852582    0.9999899980710181
0.2150364738088151      0.9999897837611175    0.9999898424102294
0.21519900017541674     0.9999896792936305    0.9999896412646619
0.21535834296132786     0.9999895758583459    0.999989440483852
0.21553117161145594     0.9999894625227059    0.9999892758950955
0.21569248195843974     0.9999893556539463    0.9999891991897246
0.21586727816964046     0.999989238654906     0.9999891826620387
0.21603889080015068     0.9999891225650085    0.9999891767966228
0.21619898512751662     0.9999890131649927    0.9999891228250181
0.2163725653190995      0.9999888933363338    0.9999889765968022
0.21653462720753813     0.9999887803094054    0.9999887696992773
0.2166935055152862      0.9999886684151994    0.9999885475383121
0.21686586968725124     0.9999885457944823    0.9999883492878197
0.217026715556072       0.9999884302042595    0.9999882436361835
0.21720104728910972     0.9999883036423425    0.9999882110129085
0.2173721954414569      0.9999881780845528    0.9999882091277019
0.21753182529065979     0.9999880597992155    0.9999881705350171
0.21770494100407964     0.9999879302249157    0.9999880395668467
0.21786653841435522     0.9999878080445979    0.9999878327667562
0.21802495224394025     0.9999876871113483    0.9999875920422748
0.21819685193774224     0.9999875545728897    0.9999873577946137
0.21835723332839996     0.9999874296755531    0.9999872166113724
0.21853110058327463     0.9999872929126988    0.9999871593440773
0.21869344953500502     0.9999871639189909    0.9999871561081516
0.21885261490604488     0.9999870362345268    0.9999871357699909
0.2190252661413017      0.999986896354474     0.9999870309837672
0.21918639907341422     0.9999867645033101    0.9999868361534916
0.2193610178697437      0.999986620367101     0.9999865559425398
0.21953245308538266     0.9999864782239526    0.9999862878051393
0.21969236999787736     0.9999863443111583    0.9999861096333476
0.219865772774589       0.9999861976500527    0.999986022596175
0.22002765724815634     0.9999860593497414    0.999986011500312
0.22018635814103318     0.9999859224610366    0.9999860019016933
0.22035854489812698     0.9999857724599424    0.9999859177714849
0.2205192133520765      0.9999856310896618    0.9999857348305402
0.22069336767024295     0.9999854763072421    0.9999854454641393
0.22086433840771888     0.9999853227737018    0.9999851442124936
0.22102379084205054     0.99998517815812      0.9999849239628631
0.22119672914059915     0.999985019740575     0.9999847976266043
0.2213581491360035      0.9999848703848719    0.9999847703634789
0.22153305499562478     0.9999847069118546    0.9999847658870313
0.22170477727455554     0.9999845447392276    0.9999846956151669
0.221864981250342       0.999984391933408     0.9999845205899752
0.22203867109034545     0.9999842246008717    0.9999842224229203
0.2222008426272046      0.9999840667880643    0.9999839079533414
0.22235983058337322     0.9999839105813265    0.9999836462839142
0.2225323044037588      0.9999837394390417    0.9999834769016616
0.2226932599210001      0.9999835781295585    0.999983427043007
0.22286770130245834     0.9999834015480108    0.9999834246156074
0.22303895910322605     0.9999832263955118    0.9999833754951424
0.22319869860084948     0.9999830614070764    0.999983220480588
0.22337192396268987     0.9999828807109478    0.9999829248956927
0.223533631021386       0.9999827103452333    0.9999825862357802
0.22369215449939156     0.9999825417411052    0.9999822808120339
0.2238641638416141      0.9999823569949579    0.9999820593893162
0.22402465488069234     0.9999821829177896    0.9999819763604693
0.22419863178398755     0.9999819923411022    0.9999819679490372
0.22436109038413848     0.9999818126087399    0.9999819431940222
0.2245203654035989      0.9999816347225163    0.99998182131384
0.22469312628727625     0.9999814398832999    0.9999815456211407
0.22485436886780932     0.9999812562444274    0.99998119266009
0.22502909731255935     0.9999810552793673    0.9999808079840474
0.22520064217661886     0.9999808559691652    0.9999805330607894
0.2253606687375341      0.9999806682345732    0.9999804111139977
0.22553418116266627     0.9999804626924748    0.9999803883602684
0.22569617528465416     0.9999802696997254    0.999980377134371
0.22585498582595154     0.9999800792025205    0.9999802807487571
0.22602728223146587     0.999979870515843     0.9999800236443414
0.22618806033383593     0.9999796738713227    0.9999796620951903
0.22636232430042294     0.9999794586293963    0.9999792352653092
0.2265334046863194      0.999979245169402     0.9999789000925758
0.2266929667690716      0.9999790441412062    0.999978727736033
0.22686601471604076     0.9999788239848464    0.9999786785801263
0.22702754435986564     0.9999786164554793    0.9999786757015077
0.22720255986790747     0.9999783893685845    0.9999785935961849
0.22737439179525876     0.9999781641323962    0.9999783484540092
0.22753470541946577     0.9999779519373188    0.999977977813882
0.22770850490788974     0.9999777196270627    0.9999775142141083
0.22787078609316944     0.9999775005655259    0.9999771419041984
0.2280298836977586      0.9999772837696866    0.9999769152136198
0.2282024671665647      0.9999770463016666    0.9999768310148076
0.22836353233222653     0.9999768225077738    0.9999768279987068
0.22853808336210532     0.9999765775832221    0.9999767728196074
0.22870945081129357     0.9999763346825421    0.9999765568992375
0.22886929995733754     0.9999761059062775    0.9999761917982484
0.22904263496759847     0.9999758554056297    0.9999756962770611
0.22920445167471512     0.9999756192554768    0.9999752647356147
0.22936308480114126     0.9999753855811302    0.9999749730710227
0.22953520379178433     0.9999751295903277    0.9999748388668626
0.22969580447928312     0.9999748884104934    0.999974825432674
0.22986989103099886     0.9999746244268637    0.9999747958172215
0.2300407940020241      0.9999743626667449    0.9999746181337172
0.23020017866990505     0.9999741162037441    0.999974271483739
0.23037304920200294     0.9999738463074609    0.9999737547596416
0.23053440143095655     0.9999735919526874    0.9999732646792798
0.2307092395241271      0.999973313655543     0.999972869777122
0.23088089403660716     0.9999730376883298    0.9999726878109078
0.23104103024594294     0.9999727777736377    0.9999726585358854
0.23121465231949567     0.9999724932598028    0.9999726427323555
0.23137675608990413     0.9999722250554938    0.9999725039434993
0.23153567627962204     0.9999719596949013    0.9999721803845995
0.23170808233355691     0.9999716690815326    0.9999716516003488
0.2318689700843475      0.9999713953587657    0.9999711100562763
0.23204334369935506     0.9999710976105313    0.9999706335486336
0.23221453373367207     0.999970802392219     0.9999703793215796
0.2323742054648448      0.9999705244151744    0.9999703169430062
0.2325473630602345      0.9999702200705032    0.9999703121703041
0.2327090023524799      0.9999699332302568    0.9999702125469719
0.23286745806403478     0.9999696494459426    0.9999699249710245
0.2330393996398066      0.9999693385763122    0.9999693995289207
0.23319982291243416     0.9999690457526189    0.9999688137881065
0.23337373204927866     0.9999687252527538    0.9999682502300634
0.2335361228829789      0.9999684230785715    0.99996791782172
0.2336953301359886      0.9999681240824986    0.9999677946408612
0.23386802325321523     0.9999677966572245    0.9999677856940061
0.2340291980672976      0.999967488131333     0.9999677325181495
0.23420385874559693     0.9999671505564146    0.999967467831338
0.23437533584320572     0.9999668158318594    0.9999669540668945
0.23453529463767023     0.9999665006141453    0.9999663338457557
0.2347087392963517      0.9999661555444658    0.9999656889553793
0.23487066565188888     0.999965830286564     0.9999652677443428
0.23502940842673556     0.9999655084885937    0.9999650792608977
0.23520163706579916     0.9999651560377989    0.9999650497702165
0.2353623474017185      0.9999648240203041    0.9999650243136904
0.23553654360185478     0.999964460690797     0.9999648114811582
0.23570755622130055     0.9999641004791969    0.9999643273677721
0.23586705053760204     0.9999637613572622    0.9999636864164744
0.2360400307181205      0.9999633900716045    0.9999629625064683
0.23620149259549467     0.9999630402055233    0.9999624403540164
0.23637644033708577     0.9999626574869069    0.9999621505739006
0.23654820449798636     0.9999622780287328    0.9999620945157858
0.23670845035574267     0.9999619206803632    0.9999620828611304
0.23688218207771594     0.9999615295905823    0.9999619052494914
0.23704439549654494     0.9999611609576964    0.9999614710023748
0.2372034253346834      0.9999607962791274    0.9999608198744968
0.2373759410370388      0.9999603969743757    0.9999600272523709
0.23753693843624993     0.9999600208287701    0.9999594062292171
0.23771142169967802     0.9999596093288547    0.9999590146526713
0.23788272138241556     0.9999592014144323    0.9999589063582685
0.23804250276200883     0.99995881739824      0.9999589034303401
0.23821577000581906     0.9999583982794382    0.9999587788885821
0.238377518946485       0.9999580044715551    0.9999583942341549
0.23853608430646042     0.9999576149358148    0.9999577518276311
0.2387081355306528      0.9999571883361759    0.9999569016777357
0.23886866845170088     0.9999567865678402    0.9999561765518498
0.23904268723696592     0.9999563469428345    0.9999556617530287
0.2392051877190867      0.9999559325235202    0.9999554780889465
0.23936450462051692     0.9999555225386048    0.9999554635036049
0.2395373073861641      0.9999550736863285    0.9999554001542533
0.23969859184866701     0.9999546508080884    0.999955096924454
0.23987336217538688     0.9999541882305859    0.999954423196857
0.2400449489214162      0.999953729646647     0.9999535288371707
0.24020501736430125     0.9999532978505363    0.9999527074204064
0.24037857167140325     0.9999528252771867    0.9999520662582895
0.24054060767536098     0.9999523799001158    0.9999517922397961
0.2406994600986282      0.999951939333096     0.999951747502276
0.24087179838611236     0.9999514569134711    0.9999517187434647
0.24103261837045226     0.999951002523134     0.9999514786848372
0.24120692421900908     0.9999505053950665    0.9999508504417758
0.24137804648687539     0.9999500126169074    0.9999499316124667
0.24153765045159742     0.9999495487479578    0.9999490178893196
0.2417107402805364      0.9999490409973529    0.9999482346397012
0.24187231180633112     0.9999485625981771    0.99994784303103
0.2420306997514353      0.9999480894329372    0.9999477419964711
0.24220257356075642     0.9999475712465692    0.9999477342223883
0.24236292906693327     0.9999470833087408    0.9999475648935001
0.24253677043732708     0.9999465494118992    0.9999470076195903
0.2426990935045766      0.9999460462285866    0.9999461423317079
0.2428582329911356      0.9999455485097271    0.9999451495997468
0.24303085834191154     0.999945003643928     0.9999442008849495
0.2431919653895432      0.9999444904329488    0.999943644260411
0.24336655830139184     0.99994392909743      0.9999434319168463
0.24353796763254992     0.9999433727302682    0.9999434233701656
0.24369785866056373     0.9999428490087973    0.9999433130452828
0.2438712355527945      0.9999422759036622    0.9999428299039236
0.24403309414188099     0.9999417359429421    0.9999419854439197
0.24419176915027693     0.9999412019476146    0.9999409356857678
0.24436393002288984     0.9999406173166683    0.9999398481760836
0.24452457259235846     0.9999400691493036    0.9999391385046955
0.24469870102604405     0.9999394701317236    0.9999388045714767
0.2448696458790391      0.9999388764826134    0.9999387691092487
0.24502907242888985     0.99993831780016      0.9999387135620655
0.24520198484295758     0.9999377063144528    0.9999383241429286
0.24536337895388102     0.9999371303078494    0.9999375311271623
0.24553825892902142     0.9999365003899046    0.9999363277091627
0.2457099553234713      0.9999358760346164    0.9999351214277914
0.24587013341477693     0.9999352882406223    0.999934277509244
0.24604379737029947     0.9999346450990677    0.9999338285101419
0.24620594302267773     0.9999340390618934    0.9999337522084992
0.24636490509436548     0.9999334396729949    0.9999337315343354
0.2465373530302702      0.9999327835035833    0.9999334289174029
0.24669828266303062     0.9999321655467367    0.9999326976399201
0.246872698160008       0.9999314896299387    0.9999314757900889
0.24704393007629485     0.9999308197472275    0.9999301475725731
0.24720364368943742     0.999930189247922     0.9999291330068941
0.24737684316679695     0.9999294992640376    0.9999285122908375
0.2475385243410122      0.9999288492519154    0.9999283538841257
0.2476970219345369      0.9999282064500413    0.9999283498226621
0.24786900539227857     0.9999275026446939    0.9999281420441267
0.24802947054687596     0.9999268400052277    0.9999275017467462
0.2482034215656903      0.9999261151121202    0.9999262985991308
0.24836585428136038     0.9999254320037452    0.9999249373407072
0.2485251034163399      0.9999247564112035    0.9999237274465028
0.2486978384155364      0.999924016981231     0.9999228703265984
0.2488590551115886      0.9999233205892791    0.999922564522104
0.24903375767185776     0.9999225590565375    0.9999225407111588
0.24920527665143638     0.999921804377261     0.9999224097124207
0.24936527732787073     0.9999210940555217    0.9999218614840077
0.24953876386852203     0.9999203169139937    0.9999206965216086
0.24970073210602906     0.999919584794078     0.9999192648926921
0.24985951676284554     0.9999188608552246    0.9999178930362437
0.250031787283879       0.999918068426956     0.999916820256584
0.2501925395017682      0.9999173223594282    0.9999163583385908
0.2503667775838743      0.9999165064284802    0.9999162816970679
0.2505378320852899      0.9999156979906093    0.9999162193750089
0.2506973682835612      0.9999149381263834    0.9999157846708303
0.25087039034604947     0.999914109856847     0.9999146995948802
0.25103189410539345     0.9999133297763664    0.9999132315261066
0.2512068837289544      0.9999124769108453    0.9999115577739405
0.2513786897718248      0.9999116317508571    0.9999102976994693
0.25153897751155097     0.99991083620936      0.999909691664498
0.25171275111549407     0.9999099659829651    0.9999095488073672
0.2518750064162929      0.9999091460923293    0.9999095269838238
0.25203407813640116     0.9999083353410276    0.9999091951555026
0.2522066357207264      0.9999074480090732    0.9999082012500256
0.2523676750019073      0.9999066124770968    0.9999067220824234
0.2525422001473052      0.9999056988041262    0.9999049002751971
0.2527135417120126      0.9999047934568706    0.9999034041782087
0.2528733649735757      0.9999039414568386    0.9999025855888302
0.2530466740993558      0.9999030092984654    0.9999023157182487
0.25320846492199156     0.9999021312647459    0.9999023102037723
0.25336707216393684     0.9999012631108687    0.999902088755638
0.25353916527009907     0.9999003127886806    0.9999012248532574
0.253699740073117       0.9998994181695692    0.999899779309955
0.2538738007403519      0.9998984397278082    0.99989783834051
0.2540446778268962      0.9998974703181776    0.9998960956604648
0.2542040366102963      0.9998965582756713    0.9998950215615738
0.2543768812579133      0.9998955602766959    0.9998945640016021
0.25453820760238605     0.9998946204819177    0.9998945336497946
0.25471301981107575     0.9998935930050719    0.9998943810910991
0.25488464843907493     0.9998925749377586    0.999893608446579
0.25504475876392985     0.9998916168201638    0.9998921847185105
0.2552183549530017      0.9998905687972912    0.9998901438853915
0.2553804328389293      0.999889581602547     0.9998882776859085
0.25553932714416633     0.9998886055734212    0.9998869438898903
0.2557117073136203      0.9998875374280357    0.9998862662148392
0.25587256917993        0.9998865318825898    0.9998861679202892
0.25604691691045667     0.999885432401401     0.9998860986878431
0.2562180810602928      0.999884343173921     0.9998854829168192
0.2563777269069847      0.9998833184085826    0.9998841573555634
0.2565508586178935      0.9998821973679556    0.9998820687615962
0.25671247202565806     0.9998811417274589    0.9998799963543196
0.25687090185273204     0.9998800982159171    0.9998783732965164
0.257042817544023       0.999878959491247     0.9998774118933669
0.25720321493216963     0.9998778894412389    0.9998771857100895
0.25737709818453325     0.9998767192439861    0.9998771681180842
0.2575394631337526      0.999875616915086     0.9998767679284494
0.2576986445022814      0.9998745270750813    0.999875635709989
0.2578713117350272      0.9998733345985996    0.9998735951976374
0.2580324606646287      0.9998722119111619    0.9998713481932122
0.25820709545844717     0.9998709845384628    0.9998692202528061
0.2583785466715751      0.9998697685852539    0.9998679556621564
0.2585384795815588      0.9998686244509194    0.9998675590571316
0.25871189835575936     0.9998673729821445    0.9998675424174767
0.25887379882681566     0.9998661943520774    0.9998672816277374
0.25903251571718144     0.9998650291802034    0.9998663104388613
0.2592047184717642      0.999863754034184     0.9998643336808686
0.25936540292320265     0.9998625537980215    0.9998619649538415
0.25953957323885807     0.9998612414150939    0.9998595275453283
0.259710559973823       0.9998599413760383    0.9998579040604321
0.2598700284056436      0.9998587184305474    0.9998572653450712
0.2600429827016812      0.9998573805356681    0.9998571897774776
0.2602044186945745      0.9998561208327624    0.9998570582635229
0.26037934055168477     0.9998547439140211    0.999856153793966
0.26055107882810447     0.9998533798328795    0.9998542170902393
0.2607112988013799      0.9998520962346924    0.999851744240689
0.26088500463887226     0.9998506925003462    0.9998490460071681
0.26104719217322037     0.9998493704017902    0.999847180254732
0.26120619612687795     0.9998480634455397    0.9998462655392646
0.2613786859447525      0.9998466334486669    0.9998460738917767
0.26153965745948277     0.9998452874134499    0.9998460253694921
0.26171411483843        0.9998438159474762    0.9998453234167382
0.26188538863668664     0.9998423584249312    0.9998435336498855
0.262045144131799       0.9998409873087398    0.9998410244374072
0.26221838549112836     0.9998394876882049    0.999838059819779
0.2623801085473134      0.999838075705351     0.9998358149042655
0.26253864802280796     0.9998366801276172    0.999834551021545
0.26271067336251946     0.9998351529921187    0.9998341599748779
0.2628711803990867      0.9998337159691421    0.9998341524797377
0.26304517329987087     0.999832144875664     0.9998336707557586
0.2632076478975108      0.9998306656223578    0.9998322033737588
0.26336693891446017     0.9998292090449147    0.9998297760535894
0.26353971579562646     0.999827615709435     0.9998265970377258
0.2637009743736485      0.9998261158651846    0.9998239189395857
0.26387571881588745     0.9998244765916746    0.9998220561780463
0.2640472796774359      0.9998228528931307    0.9998214116142881
0.2642073222358401      0.9998213253310279    0.9998213853350626
0.26438085065846123     0.9998196548854347    0.9998210821289616
0.2645428607779381      0.9998180819052377    0.9998198278308373
0.26470168731672444     0.9998165271566588    0.9998174924778415
0.26487399971972775     0.9998148260825529    0.9998141665575698
0.2650347938195868      0.9998132251731883    0.9998111337044465
0.26520907378366276     0.999811475105363     0.9998087961328893
0.2653801701670482      0.999809741815408     0.9998078009025512
0.2655397482472893      0.999808111536346     0.9998076825586603
0.2657128121917474      0.9998063284442175    0.9998075388055118
0.26587435783306124     0.999804649794545     0.9998065455061401
0.26603271989368454     0.9998029907846145    0.9998043866042284
0.2662045678185248      0.9998011753248149    0.999800996423322
0.2663648974402208      0.9997994672015792    0.9997976322810163
0.26653871292613374     0.9997975996352431    0.9997947663756491
0.2667010101089024      0.9997958409067638    0.9997933517522495
0.2668601237109805      0.9997941025839266    0.999793002332101
0.26703272317727555     0.999792201032856     0.999792975339492
0.26719380434042633     0.9997904113490452    0.9997923265490799
0.26736837136779407     0.9997884553221964    0.9997902343476416
0.2675397548144713      0.9997865181376543    0.9997868408689047
0.26769961995800423     0.9997846960047063    0.9997832035602924
0.26787297096575413     0.9997827035229797    0.9997798329171241
0.26803480367035976     0.9997808276966427    0.9997779434203785
0.26819345279427487     0.999778973918946     0.9997773117648732
0.2683655877824069      0.9997769458128101    0.9997772881702142
0.2685262044673946      0.9997750375860164    0.9997768776351843
0.2687003070165993      0.9997729517558709    0.9997750908891261
0.2688712259851135      0.9997708863529822    0.9997717996098615
0.2690306266504834      0.9997689442127782    0.9997679537079972
0.26920351318007024     0.9997668202650798    0.9997640671111153
0.2693648814065128      0.999764821285473     0.9997616152859169
0.26953973549717236     0.9997626407660485    0.9997605381689637
0.2697114060071414      0.9997604859966115    0.9997604588774607
0.26987155821396613     0.9997584591275304    0.9997601937281557
0.27004519628500784     0.9997562432401063    0.9997586236119839
0.27020731605290527     0.9997541569726821    0.999755597729641
0.2703662522401121      0.9997520952544618    0.9997516122881541
0.27053867429153594     0.9997498400630329    0.9997472653007523
0.2706995780398155      0.9997477179846662    0.9997442502839903
0.270873967652312       0.9997453987666282    0.9997426772044103
0.27104517368411796     0.9997431022159896    0.9997424327781392
0.27120486141277966     0.9997409424614091    0.9997423282374197
0.2713780350056583      0.9997385808372419    0.9997411099842792
0.2715396902953927      0.9997363578615814    0.9997383292583388
0.2716981620044365      0.9997341612565952    0.9997343045417842
0.2718701195776973      0.9997317580750699    0.9997295374860445
0.2720305588478138      0.9997294972871958    0.9997259059149527
0.27220448398214725     0.9997270260488477    0.9997237031986753
0.2723668908133364      0.9997246991468407    0.9997231577773898
0.2725261140638351      0.9997223996061725    0.999723149424248
0.2726988231785507      0.9997198847161138    0.9997223830615286
0.27286001399012205     0.9997175180826342    0.9997200504495194
0.27303469066591035     0.9997149320684944    0.9997157114016433
0.2732061837610081      0.9997123713920566    0.999710603878899
0.27336615855296154     0.9997099630928785    0.999706391543972
0.27353961920913195     0.999707330228677     0.999703525153062
0.2737015615621581      0.9997048518170926    0.9997025897869853
0.2738603203344937      0.9997024028922997    0.9997025502293937
0.2740325649710463      0.9996997242517891    0.9997020850127849
0.2741932913044546      0.9996972042359403    0.9997001223822928
0.27436750350207983     0.9996944502658768    0.9996959643178712
0.2745385321190146      0.9996917236838365    0.9996906092332337
0.27469804243280505     0.9996891601047116    0.9996858122684731
0.27487103861081247     0.99968635713022      0.99968217217199
0.2750325164856756      0.9996837193655168    0.9996806929733311
0.2752074802247557      0.9996808378061349    0.9996805009156096
0.27537926038314525     0.9996779847367849    0.9996802119817345
0.2755395222383905      0.9996753014394741    0.9996785052135507
0.2757132699578527      0.9996723687046649    0.9996744734778277
0.27587549937417066     0.9996696166745629    0.9996692012791097
0.2760345452097981      0.9996668984467797    0.9996638890345011
0.27620707690964247     0.9996639258956493    0.9996594843563178
0.2763680903063426      0.9996611292333099    0.9996573941209539
0.27654258956725963     0.9996580735355838    0.9996569273993733
0.2767139052474861      0.9996550482878641    0.9996568242344592
0.27687370262456834     0.9996522036633176    0.9996555258558234
0.2770469858658675      0.9996490939250054    0.9996518509638563
0.2772087508040224      0.999646167191165     0.9996465491004544
0.2773673321614868      0.9996432756678474    0.9996407758352231
0.27753939938316813     0.9996401129820136    0.999635543620936
0.2776999483017052      0.9996371381145785    0.9996326919800408
0.2778739830844592      0.9996338871064626    0.9996317688760569
0.27803649956406895     0.999630826413819     0.9996317598664756
0.2781958324629881      0.9996278022058465    0.9996309752828444
0.27836865122612425     0.9996244955621628    0.999627927695829
0.2785299516861161      0.9996213842725159    0.9996228571054461
0.2787047380103249      0.999617985367143     0.9996161004942214
0.2788763407538432      0.999614620346804     0.9996100960283759
0.27903642519421723     0.9996114559763426    0.999606454874115
0.2792099954988082      0.999607997328814     0.9996049629767351
0.2793720475002549      0.9996047419984094    0.9996048916710847
0.2795309159210111      0.9996015258854479    0.9996044392661824
0.27970327020598423     0.9995980088774306    0.9996018960834874
0.2798641061878131      0.9995947005480978    0.9995970728093917
0.2800384280338589      0.9995910858778836    0.9995900596556554
0.2802095662992142      0.9995875077291039    0.9995832853772083
0.28036918626142515     0.9995841438852375    0.9995787343448521
0.2805422920878531      0.9995804667097664    0.9995764694649776
0.28070387961113674     0.9995770066751511    0.9995761727597698
0.2808622835537299      0.9995735888317773    0.9995759985206297
0.28103417336054        0.9995698507187173    0.9995740496469988
0.2811945448642058      0.9995663354282315    0.999569651415882
0.2813684022320886      0.9995624941587636    0.999562567061001
0.2815307412968271      0.9995588786654285    0.9995554293198907
0.281689896780875       0.9995553069814089    0.9995497682539668
0.2818625381291399      0.9995514021304167    0.9995463639708708
0.28202366117426053     0.9995477309937252    0.9995455374006254
0.2821982700835981      0.9995437313900284    0.9995455005267155
0.28236969541224516     0.9995397724208848    0.9995440764093619
0.28252960243774794     0.9995360503877038    0.9995401428747954
0.2827029953274677      0.9995319824891536    0.9995331189664057
0.28286486991404314     0.9995281545651005    0.9995254554215028
0.28302356091992803     0.9995243733500261    0.9995188640842113
0.2831957377900299      0.9995202385567151    0.999514391104227
0.28335639635698745     0.999516349879404     0.9995129344640399
0.283530540788162       0.999512101276252     0.9995128753943103
0.283701501638646       0.999507896178083     0.9995119874076354
0.28386094418598573     0.9995039436571616    0.9995086697728767
0.2840338725975424      0.9994996230402402    0.9995019328395913
0.28419528270595484     0.9994955582537566    0.9994938936545752
0.2843701786785842      0.9994911187647529    0.9994856841807296
0.2845418910705231      0.99948672431865      0.9994802864759806
0.28470208515931766     0.999482592460347     0.9994782239594581
0.2848757651123292      0.9994780774046148    0.9994780203264444
0.28503792676219647     0.9994738283401313    0.9994775329051521
0.28519690483137317     0.9994696311227983    0.999474807904048
0.2853693687647668      0.9994650422687603    0.9994684546358406
0.2855303143950162      0.9994607262463396    0.9994601921973814
0.2857047458894825      0.999456011642829     0.9994510673920983
0.28587599380325834     0.9994513454504875    0.9994444424544543
0.2860357234138899      0.9994469592670714    0.9994414299535881
0.2862089388887384      0.9994421655889903    0.999440838073847
0.2863706360604426      0.9994376555377096    0.9994406700191215
0.28652914965145626     0.999433201130707     0.9994386275989051
0.28670114910668687     0.9994283303828994    0.9994328878845508
0.2868616302587732      0.9994237505092096    0.9994246224708547
0.2870355972750765      0.9994187470154388    0.9994146860128963
0.2871980459882355      0.9994140381632273    0.9994070405366211
0.287357311120704       0.9994093870212665    0.9994026571310904
0.28753006211738946     0.999404303095291     0.9994012519974221
0.28769129481093064     0.9993995213483851    0.9994012474867804
0.2878660133686888      0.9993942992777611    0.9993997633302478
0.28803754834575634     0.9993891312499658    0.9993946709257844
0.28819756501967964     0.9993842732927859    0.9993865420968436
0.2883710675578199      0.9993789749391153    0.9993759730099214
0.28853305179281585     0.999373995282435     0.9993671543038136
0.2886918524471213      0.999369077257148     0.9993615116143388
0.2888641389656437      0.9993637006926824    0.9993591916754018
0.28902490718102186     0.9993586449151914    0.9993590838896691
0.28919916126061695     0.9993531225584592    0.9993582205189532
0.28937023175952153     0.9993476577796165    0.9993539651525557
0.28952978395528184     0.999342521979122     0.9993462324045129
0.2897028220152591      0.999336909252689     0.9993352417152951
0.2898643417720921      0.9993316296291668    0.9993252622994663
0.29003934739314197     0.9993258647133184    0.999317642784573
0.29021116943350134     0.9993201593321439    0.9993144701570867
0.29037147317071643     0.9993147956184671    0.9993141399220026
0.2905452627721485      0.999308935856477     0.9993136343131783
0.29070753407043626     0.9993034220759308    0.9993102460494694
0.2908666217880335      0.9992979764546183    0.9993030032402134
0.29103919536984774     0.9992920241002731    0.9992917848051298
0.2912002506485177      0.9992864263952901    0.9992808048140069
0.2913747917914046      0.9992803131653951    0.9992716307808015
0.2915461493536009      0.9992742637489787    0.9992671393800965
0.29170598861265296     0.999268578075374     0.999266286611164
0.29187931373592196     0.9992623655936227    0.9992661354871587
0.2920411205560467      0.9992565214386709    0.9992636084149117
0.2921997437954809      0.9992507502618495    0.9992571125031877
0.2923718528991321      0.9992444410797715    0.9992459620977744
0.292532443699639       0.9992385094049127    0.9992341159165647
0.29270652036436284     0.9992320305111689    0.999223280297281
0.29287741344839613     0.9992256200969228    0.9992171521285429
0.29303678822928514     0.9992195968076358    0.9992154354613609
0.2932096488743911      0.9992130144935942    0.9992154126042372
0.2933709912163528      0.9992068241561849    0.9992137832278569
0.29354581942253144     0.9992000652493415    0.9992074480805218
0.29371746404801957     0.999193377335177     0.9991963384880085
0.29387759037036343     0.9991870913942964    0.9991838004107638
0.29405120255692424     0.9991802246630173    0.9991715908448701
0.2942132964403408      0.9991737649490273    0.9991642744096506
0.2943722067430668      0.9991673863123977    0.9991614873311278
0.2945446029100098      0.9991604164423534    0.9991613007591021
0.2947054807738085      0.9991538789901817    0.9991603728263333
0.2948798445018241      0.9991467400090672    0.9991551070452076
0.2950510246491492      0.9991396768118956    0.9991445324353753
0.29521068649333        0.9991330397916319    0.9991315294112159
0.29538383420172776     0.999125788226819     0.9991177852263502
0.29554546360698125     0.9991189680365834    0.9991086285141436
0.29570390943154423     0.999112234021492     0.9991043974928573
0.29587584112032417     0.9991048725285301    0.9991036873564565
0.29603625450595983     0.9990979528655157    0.99910334231178
0.29621015375581244     0.9990903950811452    0.999099325718567
0.2963725347025208      0.9990832845566037    0.9990902701850453
0.29653173206853856     0.9990762630904679    0.9990772935821569
0.2967044152987733      0.9990685900759788    0.9990621039896175
0.29686558022586373     0.9990613752272375    0.9990507090367151
0.29704023101717114     0.999053497782066     0.9990439895138737
0.29721169822778803     0.9990457039420099    0.9990424600523993
0.29737164713526065     0.9990383797066426    0.9990424160250405
0.2975450819069502      0.9990303786990178    0.9990395002434158
0.2977069983754955      0.9990228530628638    0.9990314277255053
0.29786573126335025     0.999015422561244     0.9990186445673264
0.29803795001542194     0.9990073012211215    0.999002425295972
0.29819865046434935     0.9989996667977817    0.9989891740889689
0.2983728367774937      0.9989913299616987    0.9989803129651432
0.29854383950994756     0.9989830826161367    0.9989775119869571
0.29870332393925714     0.9989753342803945    0.9989774893252934
0.29887629423278367     0.9989668687040569    0.9989756890139091
0.2990377462231659      0.9989589082382833    0.9989689051144371
0.29921268407776513     0.998950218544209     0.9989552012005809
0.29938443835167383     0.9989416215110448    0.9989381516344148
0.29954467432243825     0.9989335421547333    0.9989233796991237
0.2997183961574196      0.9989247182238504    0.998912676770588
0.2998805996892566      0.9989164183107873    0.9989087145531691
0.30003961964040315     0.9989082237364154    0.9989083874128059
0.30021212545576664     0.9988992693782357    0.998907425070491
0.30037311296798586     0.9988908516828777    0.9989018798006343
0.300547586344422       0.9988816616904789    0.9988889907806747
0.3007188761401677      0.9988725710277645    0.9988714473603505
0.30087864763276906     0.9988640407024464    0.9988550083795085
0.3010519049895874      0.9988547333366845    0.9988418659971678
0.30121364404326145     0.9988459809534148    0.9988360215826892
0.30137219951624494     0.9988373406452221    0.9988349708626235
0.3015442408534454      0.9988278975447822    0.9988346631508424
0.30170476388750156     0.9988190224999668    0.9988305314722851
0.3018787727857747      0.9988093313786787    0.9988189215263811
0.30204126338090354     0.998800215103134     0.9988022902294907
0.30220057039534187     0.9987912145240435    0.9987843333075637
0.30237336327399716     0.9987813811148534    0.9987682683231638
0.3025346378495082      0.9987721361821752    0.998759705606943
0.30270939828923615     0.9987620446344608    0.9987571090369878
0.30288097514827356     0.9987520619921493    0.9987570827541941
0.3030410337041667      0.998742682119423     0.9987541678267298
0.30321457812427677     0.9987324379490728    0.998743907108348
0.3033766042412426      0.99872280375472      0.9987275846119834
0.30353544677751787     0.9987132928734728    0.9987085670418734
0.3037077751780101      0.9987029001510953    0.9986901096636058
0.3038685852753581      0.9986931318255544    0.9986790588151774
0.304042881236923       0.9986824672237984    0.9986746821023998
0.30421399361779744     0.9986719189084385    0.9986745535755167
0.3043735876955276      0.9986620100912241    0.9986728442395242
0.3045466676374746      0.9986511865095423    0.9986643080473818
0.3047082292762774      0.9986410100450809    0.9986488171090964
0.30486660733438964     0.998630965100059     0.9986291408804091
0.30503847125671885     0.9986199870642263    0.9986083507284813
0.3051988168759038      0.9986096713495491    0.9985944580919743
0.30537264835930567     0.9985984074645794    0.9985876514595337
0.3055349615395633      0.9985878138117689    0.9985868438920431
0.3056940911391304      0.9985773561442722    0.9985863004560216
0.30586670660291443     0.9985659313602487    0.9985801736773855
0.3060278037635542      0.9985551925787105    0.9985664961564357
0.30620238678841094     0.9985434710932044    0.9985446924615737
0.3063737862325771      0.9985318781183159    0.9985219322142693
0.306533667373599       0.9985209876297771    0.9985052980753044
0.30670703437883784     0.9985090945135734    0.9984958130428336
0.3068688830809324      0.9984979122069559    0.9984938585468626
0.30702754820233646     0.998486875052313     0.9984938118794874
0.3071996991879575      0.9984748358076525    0.9984894523877623
0.3073603318704342      0.9984635272158492    0.998477392061405
0.3075344504171279      0.9984511818163699    0.9984559499328151
0.307705385383131       0.9984389730850454    0.9984315347145067
0.30786480204598987     0.9984275069663805    0.9984120127648473
0.3080377045730657      0.9984149829234633    0.9983992610356489
0.3081990887969972      0.9984032100763109    0.9983954793275369
0.3083739588851457      0.9983903622317329    0.9983954447898348
0.30854564539260365     0.9983776553675943    0.9983922290545041
0.30870581359691734     0.9983657174336081    0.9983813386464053
0.308879467665448       0.9983526826091714    0.99836018635619
0.30904160343083437     0.998340425657563     0.9983357502699786
0.3092005556155302      0.9983283275654791    0.998313622209055
0.309372993664443       0.9983151108658622    0.998297555911584
0.30953391341021147     0.9983026899240331    0.9982915613230863
0.3097083190201969      0.9982891325060046    0.9982910814719337
0.30987954104949184     0.9982757252686283    0.9982893080977611
0.3100392447756425      0.9982631325123663    0.9982804060357995
0.3102124343660101      0.9982493804011049    0.9982605325521947
0.3103741056532334      0.9982364522185602    0.9982354202908148
0.31053259335976624     0.998223693105624     0.9982108001846492
0.310704566930516       0.9982097519627621    0.9981910070228488
0.3108650221981215      0.9981966535917554    0.9981820867583588
0.31103896332994396     0.9981823545336997    0.9981804301523726
0.31120138615862214     0.9981689080530356    0.9981799549433882
0.31136062540660975     0.9981556361981737    0.9981737593673663
0.3115333505188143      0.9981411402213521    0.998156444380411
0.3116945573278746      0.9981275163595139    0.9981316691371931
0.31186925000115184     0.9981126490958485    0.9981021502447834
0.3120407590937386      0.9980979472305007    0.998078811543501
0.31220074988318103     0.9980841378965571    0.9980667469209031
0.31237422653684044     0.9980690605241268    0.9980633430226018
0.3125361848873556      0.9980548859930372    0.9980633567861071
0.31269495965718014     0.9980408974738783    0.9980590907424891
0.31286722029122166     0.9980256165246649    0.9980439750778465
0.3130279626221189      0.9980112589115027    0.9980198392258179
0.3132021908172331      0.9979955888009064    0.9979885772318035
0.3133732354316568      0.9979801027982421    0.9979615511582558
0.3135327617429362      0.9979655810375277    0.9979457180291395
0.3137057739184326      0.9979497233031615    0.9979396854093078
0.31386726779078467     0.9979348189004958    0.9979396013368216
0.3140422475273537      0.9979185575268212    0.9979365372678871
0.3142140436832322      0.997902477575384     0.9979229685232909
0.3143743215359664      0.9978873728153016    0.9978993028497579
0.31454808525291755     0.9978708841706515    0.9978666859020621
0.31471033066672444     0.9978553817420648    0.9978379654049387
0.3148693924998408      0.9978400827775409    0.9978183698211456
0.31504194019717413     0.9978233732369732    0.9978092664032798
0.3152029695913632      0.997807671945545     0.9978084420529996
0.3153774848497692      0.9977905381171844    0.9978069366384802
0.3155488165274847      0.9977735971160324    0.99779596112298
0.3157086299020559      0.9977576874215419    0.9977739077039938
0.31588192914084406     0.9977403170825535    0.9977406609362817
0.31604371007648796     0.9977239896865361    0.9977089351364011
0.3162023074314413      0.9977078783827767    0.9976851586242061
0.31637439065061157     0.9976902785828731    0.9976720843404833
0.3165349555666376      0.9976737449196208    0.9976696839889218
0.31670900634688054     0.997655699844329     0.9976693223368044
0.316879873546433       0.9976378598783712    0.9976612598319918
0.31703922244284116     0.9976211102940409    0.997641617396432
0.3172120572034663      0.9976028200066024    0.9976086771029425
0.31737337366094714     0.9975856323433366    0.997574390683696
0.31754817598264495     0.9975668802984332    0.9975438205724878
0.3177197947236522      0.9975483399941398    0.9975272526373533
0.31787989516151516     0.9975309274323435    0.9975231224859438
0.3180534814635951      0.9975119202418012    0.9975231352685374
0.3182155494625307      0.9974940534796971    0.9975176133708349
0.31837443388077585     0.9974764238225025    0.9975004777943716
0.31854680416323794     0.9974571695955048    0.9974684376144205
0.31870765614255575     0.9974390810009208    0.9974322919149355
0.3188819939860905      0.9974193432041595    0.9973972502330998
0.3190531482489347      0.9973998308621957    0.9973757526784297
0.31921278420863464     0.9973815104571645    0.9973685949667161
0.3193859060325515      0.9973615094207645    0.9973683755401686
0.3195475095533241      0.9973427136128157    0.9973652029397195
0.3197059294934062      0.9973241913299234    0.9973511865668493
0.31987783529770525     0.9973039690789187    0.9973210975433437
0.32003822279886        0.997284975961314     0.9972838985197722
0.32021209616423174     0.9972642477494201    0.9972445480627822
0.3203744512264592      0.9972447621694334    0.9972183864301228
0.3205336227079961      0.9972255355589057    0.9972064133271775
0.32070628005375        0.997204541292304     0.9972045457452343
0.32086741909635963     0.9971848166115829    0.9972036075197649
0.3210420440031862      0.997163297484884     0.9971920086066263
0.3212134853293222      0.9971420243950474    0.9971642362247541
0.3213734083523139      0.9971220490863124    0.9971267058003245
0.3215468172395226      0.9971002450591335    0.9970837862592156
0.321708707823587       0.9970797530446905    0.9970524953107877
0.3218674148269609      0.9970595355213013    0.9970358800890311
0.32203960769455175     0.9970374553270404    0.9970315708100022
0.3222002822589983      0.9970167154906932    0.9970315664352056
0.32237444268766186     0.9969940850488408    0.9970234021548113
0.3225454195356348      0.9969717157506476    0.9969989281032258
0.3227048780804635      0.9969507164289888    0.9969621245904022
0.32287782248950914     0.9969277908107379    0.9969162866064172
0.3230392485954105      0.9969062501273488    0.996879635008166
0.3232141605655288      0.996882754295765     0.9968559711526445
0.3233858889549566      0.9968595277668904    0.9968491632125842
0.32354609904124015     0.9968377168416752    0.9968493079204463
0.32371979499174064     0.9968139139277186    0.9968433742847356
0.32388197263909685     0.9967915421089425    0.9968228387236802
0.3240409667057625      0.9967694705668247    0.9967873508616769
0.3242134466366451      0.9967453705755703    0.9967394854765406
0.3243744082643834      0.9967227324376878    0.996698056383359
0.3245488557563387      0.9966980358557295    0.9966682207601861
0.32472011966760345     0.9966736254022083    0.9966571819620929
0.32487986527572393     0.9966507088927458    0.9966566910627587
0.3250530967480614      0.9966256956817426    0.996653512106565
0.32521480991725454     0.996602192632223     0.9966368408004262
0.3253733395057572      0.9965790081369229    0.996603815660614
0.3255453549584768      0.996553689103299     0.9965549986227525
0.32570585210805214     0.9965299123666039    0.996509064339785
0.32587983512184443     0.9965039739161213    0.9964723273230298
0.32604229983249244     0.9964796258892361    0.9964560568242019
0.3262015809624499      0.9964556055198532    0.9964532555314166
0.3263743479566243      0.996429383112995     0.9964525313689468
0.3265355966476544      0.9964047499889689    0.9964410374328153
0.3267103312029015      0.9963778823797287    0.9964086374469912
0.32688188217745806     0.9963513268267237    0.9963597390815528
0.32704191484887035     0.9963263948043583    0.9963101485336987
0.3272154333844996      0.9962991868088322    0.9962669030730366
0.32737743361698457     0.9962736196342893    0.9962448124638251
0.327536250268779       0.9962483990858984    0.9962389584768487
0.32770855278479033     0.9962208614754328    0.9962390600222706
0.3278693369976574      0.9961949990920484    0.9962312545769179
0.32804360707474145     0.9961667858415191    0.9962031200851578
0.328214693571135       0.9961389031240732    0.996155498861576
0.3283742617643842      0.9961127315693415    0.9961030354704467
0.3285473158218504      0.9960841659774227    0.9960530815021106
0.32870885157617236     0.9960573297002493    0.9960240399156269
0.32888387319471124     0.9960280644893796    0.9960133437065285
0.32905571123255956     0.9959991397947363    0.9960133596556977
0.3292160309672636      0.9959719816005956    0.9960078827240983
0.3293898365661846      0.9959423499346286    0.9959827950681596
0.3295521238619613      0.9959145035559382    0.9959388688383467
0.3297112275770475      0.995887035256234     0.9958844986035491
0.3298838171563507      0.995857049416927     0.9958286398363525
0.33004488843250956     0.9958288861370898    0.9957926997078278
0.3302194455728854      0.9957981690654619    0.9957763867894377
0.33039081913257073     0.9957678130285808    0.9957752145826783
0.3305506743891118      0.9957393184229597    0.9957725558631204
0.3307240155098698      0.9957082238073428    0.9957524248264055
0.33088583832748353     0.9956790102705116    0.9957116591074818
0.3310444775644067      0.9956501970527203    0.9956565973325998
0.3312166026655468      0.9956187381251683    0.9955952552961965
0.33137720946354265     0.995589199181671     0.9955516975607991
0.33155130212575545     0.995556976932662     0.9955281279145286
0.33172221120727774     0.9955251374718544    0.9955241584949404
0.33188160198565575     0.9954952585381256    0.9955236218156819
0.3320544786282507      0.9954626484155321    0.9955089943213357
0.3322158369677014      0.9954320403362334    0.9954728599026597
0.33239068117136905     0.9953986832024297    0.995412357263539
0.33256234179434613     0.9953657195901551    0.9953467865293422
0.33272248411417893     0.9953347756600682    0.995297048879881
0.3328961122982287      0.9953010151027075    0.9952671194501357
0.33305822217913417     0.9952692951016002    0.9952600090513873
0.33321714847934913     0.995238010357516     0.9952603115117417
0.33338956064378106     0.9952038596100528    0.9952503474462774
0.3335504545050687      0.995171790911847     0.9952190124206792
0.3337248342305733      0.9951368155885967    0.9951603557572577
0.33389603037538734     0.9951022562691062    0.9950911655965697
0.3340557082170571      0.9950698223135869    0.9950340519867626
0.3342288719229438      0.9950344298885927    0.9949951786017703
0.33439051732568625     0.9950011847118655    0.994982605501003
0.3345489791477382      0.9949683992948877    0.994982429834078
0.33472092683400706     0.9949326041166733    0.9949769371405601
0.33488135621713166     0.9948989995964916    0.9949514948753316
0.3350552714644732      0.9948623431163558    0.9948966195934829
0.3352176684086705      0.9948278999131057    0.9948289520260417
0.3353768817721773      0.9947939298064351    0.9947641909008144
0.335549580999901       0.9947568548900961    0.9947138530708906
0.33571076192448046     0.9947220380855265    0.9946925804865564
0.3358854287132768      0.9946840729961239    0.9946896330497244
0.33605691192138265     0.9946465606407766    0.994687284067005
0.3362168768263442      0.9946113531353933    0.9946672958605018
0.33639032759552273     0.9945729419464052    0.9946168836659288
0.336552260061557       0.9945368592354026    0.9945487837264302
0.3367110089469007      0.994501276501874     0.9944784648177221
0.3368832436964614      0.9944624353217648    0.9944185323299068
0.3370439601428778      0.9944259693691475    0.9943888900721
0.33721816245351116     0.9943861999196372    0.9943816402733883
0.33738918118345396     0.9943469095146532    0.9943814418964474
0.3375486816102525      0.994310043034754     0.9943673710304989
0.33772166790126795     0.9942698157070182    0.9943231920246608
0.33788313588913915     0.9942320369326089    0.9942566081415062
0.3380580897412273      0.9941908509565811    0.9941742727727096
0.33822986001262495     0.9941501582778902    0.9941066491032216
0.3383901119808783      0.9941119643443735    0.9940698473089887
0.33856384981334864     0.9940703480399505    0.9940581531482398
0.3387260693426747      0.9940312539244144    0.9940585845472801
0.33888510529131016     0.9939927027119212    0.9940494000830834
0.3390576271041626      0.9939506299404354    0.9940114855291174
0.33921863061387075     0.9939111278076194    0.9939475234952316
0.33939311998779587     0.993868055686201     0.9938616927746932
0.33956442578103047     0.9938255036289104    0.9937850000969379
0.3397242132711208      0.9937855740307899    0.9937382383271466
0.33989748662542807     0.9937420126758599    0.9937190083784867
0.3400592416765911      0.9937010999684908    0.993718472369027
0.34021781314706356     0.9936607596874136    0.9937138464149012
0.340389870481753       0.9936167265612479    0.9936834555226187
0.3405504095132982      0.9935753939736551    0.9936241915437809
0.34072443440906025     0.9935303182844089    0.9935368421091475
0.34088694100167805     0.9934879701781865    0.9934553296318265
0.3410462640136053      0.9934462105678431    0.9933959784815072
0.34121907288974956     0.993400644952358     0.9933650971584275
0.3413803634627495      0.99335786047175      0.993360603046248
0.34155513989996644     0.9933112182422735    0.9933591823521105
0.34172673275649285     0.9932651403814555    0.9933356118678002
0.341886807309875       0.993221899452084     0.9932816206450354
0.34206036772747406     0.9931747348776094    0.9931943841773656
0.34222240984192887     0.993130435448189     0.9931064505796011
0.3423812683756931      0.9930867567999999    0.9930367405287868
0.3425536127736743      0.9930390893868212    0.9929949386480268
0.3427144388685112      0.992994342888647     0.9929851003560408
0.3428887508275651      0.9929455540570105    0.9929854114290015
0.34305987920592845     0.9928973612186401    0.9929691276232099
0.34321948928114754     0.9928521473618759    0.992922254303365
0.3433925852205836      0.9928028230138847    0.992837588232082
0.34355416285687534     0.992756507019885     0.9927446770977447
0.34371255691247654     0.9927108457493211    0.9926644065866014
0.3438844368322947      0.9926610066713802    0.9926096703496899
0.34404479844896857     0.9926142339155364    0.9925918474407236
0.3442186459298594      0.9925632279755403    0.9925917210270894
0.34438097510760596     0.9925153185990232    0.9925835844065098
0.344540120704662       0.9924680824241529    0.9925472785478958
0.344712752165935       0.9924165579843582    0.9924697851445302
0.3448738653240637      0.9923682197754455    0.992374657385828
0.3450484643464094      0.9923155257650706    0.9922750975919361
0.34521987978806457     0.9922634775336178    0.9922076479424098
0.34537977692657545     0.9922146438473267    0.9921805883524328
0.34555315992930324     0.9921613813528206    0.9921777206214868
0.34571502462888676     0.9921113645112016    0.9921743032895262
0.34587370574777976     0.9920620556472072    0.992146183557408
0.3460458727308897      0.9920082458007204    0.9920755810930708
0.3462065214108554      0.9919577431702994    0.9919802948879096
0.34638065595503803     0.9919026801829516    0.9918719141472148
0.34655160691853015     0.9918482975557792    0.9917905773468558
0.346711039578878       0.9917972862814838    0.9917517314220816
0.3468839581034428      0.9917416390533335    0.9917431876527414
0.34704535832486333     0.9916893956128755    0.9917430506594832
0.3472202444105008      0.9916324551170279    0.9917197296691322
0.3473919469154477      0.9915762137129857    0.9916542659090484
0.34755213111725036     0.9915234422132698    0.9915591272795679
0.34772580118326996     0.9914658958970615    0.9914441628294703
0.3478879529461453      0.9914118529255552    0.9913553569513365
0.3480469211283301      0.9913585759098286    0.9913042659418793
0.34821937517473184     0.9913004472598977    0.991288035994497
0.3483803109179893      0.9912458879856526    0.9912888293037461
0.34855473252546376     0.991186413888307     0.9912736363286718
0.34872597055224763     0.9911276766467274    0.9912174462315121
0.34888569027588723     0.9910725774944085    0.9911260359772661
0.3490588958637438      0.9910124831708528    0.9910058949559151
0.34922058314845605     0.9909560617025672    0.9909047588594848
0.3493790868524778      0.9909004466053362    0.990839415054174
0.34955107642071653     0.9908397570069063    0.9908122403826122
0.349711547685811       0.9907828088093137    0.9908114838664327
0.3498855048151224      0.9907207208531058    0.990803825183343
0.3500479436412895      0.990662410064922     0.990762121498851
0.3502071988867661      0.9906049276500546    0.9906795067100157
0.3503799399964596      0.9905422240265179    0.9905579707682559
0.35054116280300884     0.9904833680588674    0.9904444366982732
0.35071587147377503     0.9904192238884568    0.9903548466883562
0.3508873965638507      0.9903558775122367    0.9903154213540644
0.3510474033507821      0.990296485715702     0.9903107359207007
0.35122089600193046     0.9902317357026758    0.9903081351515138
0.35138287034993454     0.9901709401430961    0.9902761579275201
0.3515416611172481      0.9901110151588042    0.9902015991857733
0.3517139377487786      0.9900456358825461    0.9900808846853427
0.3518746960771649      0.9899842836069763    0.9899587064108166
0.3520489402697681      0.9899174074080551    0.989852888695747
0.3522200008816807      0.9898513695949286    0.9897981526636276
0.35237954319044906     0.9897894343107088    0.9897863433939835
0.3525525713634344      0.9897218865179294    0.9897868168699904
0.3527140812332754      0.9896584794610016    0.989765130348102
0.3528890769673334      0.9895893882987296    0.9896918509342053
0.3530608891207009      0.9895211578571667    0.9895719547544025
0.3532211829709241      0.9894571459331908    0.9894433768414322
0.35339496268536424     0.9893873588561484    0.9893247690967524
0.3535572240966601      0.9893218296483244    0.9892590957273567
0.35371630192726544     0.9892572398454977    0.98923820294298
0.3538888656220877      0.9891867849831968    0.989238858546151
0.3540499110137657      0.9891206655986831    0.9892254864901333
0.35422444226966066     0.9890486072022426    0.9891635952717666
0.3543957899448651      0.9889774543418677    0.9890491577045738
0.35455561931692525     0.9889107176720566    0.9889159744096272
0.35472893455320237     0.9888379480229891    0.9887825863998557
0.3548907314863352      0.9887696354122633    0.9886998227636842
0.35504934483877754     0.9887023100142367    0.9886664010718369
0.35522144405543676     0.9886288588955173    0.9886640638241333
0.3553820249689517      0.9885599452616046    0.9886581567125133
0.3555560917466836      0.9884848296201885    0.9886095987943715
0.355726974943725       0.9884106664776672    0.988504366662464
0.35588633983762213     0.9883411244377399    0.9883698203481711
0.3560591905957362      0.9882652835296277    0.9882228873423906
0.356220523050706       0.9881941060546215    0.9881212933046233
0.35639534136989276     0.9881165514323114    0.9880691104668732
0.356566976108389       0.988039974799401     0.9880623201807466
0.35672709254374096     0.9879681475033428    0.9880603858029346
0.3569006948433099      0.9878898437037388    0.988021133181344
0.35706277883973453     0.9878163327329712    0.987929309628403
0.3572216792554686      0.987743894320746     0.9877960019121246
0.35739406553541964     0.9876649220227649    0.987638606385773
0.3575549335122264      0.9875908246179496    0.9875196346001076
0.3577292873532501      0.9875100753916324    0.9874488812994363
0.3579004576135833      0.9874303531722554    0.9874329237772673
0.3580601095707722      0.9873555941743585    0.9874340321471041
0.3582332473921781      0.987274080337035     0.9874074571448859
0.3583948669104397      0.9871975744182816    0.9873283255234562
0.35855330284801074     0.9871221848795811    0.9872001378770614
0.3587252246497987      0.9870399388127798    0.9870351461568353
0.35888562814844244     0.9869627887320298    0.9868986703636551
0.3590595175113031      0.986878698473789     0.9868060487843702
0.3592218885710195      0.9867997501631689    0.9867765960319813
0.3593810760500454      0.9867219466347896    0.9867768693473172
0.3595537493932882      0.9866370985631405    0.9867642316411731
0.3597149044333868      0.9865574829596748    0.9867035469510664
0.3598895453377023      0.9864707369706409    0.9865726490212594
0.36006100266132723     0.986385096697127     0.986403413149636
0.3602209416818079      0.9863047829694934    0.9862520596068962
0.36039436656650553     0.9862172298694932    0.9861381161349326
0.3605562731480589      0.9861350509410574    0.9860930543735109
0.3607149961489217      0.9860540729584057    0.9860886884638567
0.3608872050140015      0.9859657480848055    0.9860840921404579
0.36104789557593703     0.9858828910596427    0.9860376819442713
0.3612220720020895      0.9857925987051196    0.9859191233282115
0.36139306484755146     0.9857034672902317    0.9857497597110334
0.36155253938986914     0.9856199010326404    0.9855851653429261
0.3617254997964038      0.985528787241041     0.9854481332903474
0.36188694189979415     0.9854432878815776    0.985383227557367
0.36206186986740146     0.9853501503116119    0.9853698245937382
0.3622336142543182      0.9852582036702237    0.9853691482962407
0.3623938403380907      0.9851719713384907    0.9853324334385559
0.3625675522860801      0.9850779857885404    0.9852232473909374
0.3627297459309253      0.9849897651283471    0.9850636362758523
0.3628887559950799      0.984902836535795     0.9848891439929587
0.3630612519234515      0.9848080413252838    0.9847311602665503
0.36322222954867883     0.9847191103404223    0.9846457529739456
0.3633966930381231      0.9846222194682327    0.9846195336616441
0.3635679729468768      0.9845266072396573    0.9846212120968751
0.36372773455248625     0.9844369763180839    0.9845974863694418
0.36390098202231264     0.9843392705513749    0.9845045427099491
0.36406271118899475     0.9842475808238762    0.9843518854861008
0.36422125677498635     0.9841572442632877    0.9841707435447253
0.3643932882251949      0.9840587156539224    0.9839919993897948
0.3645538013722592      0.9839663051544387    0.9838828858006639
0.3647278003835404      0.9838656061989353    0.9838384168959161
0.36489028109167737     0.983771078642269     0.9838379446960897
0.36504957821912376     0.9836779375412302    0.983828068437187
0.3652223612107871      0.9835763878133962    0.9837581544171169
0.3653836258993062      0.9834811142752902    0.9836209286385724
0.3655583764520422      0.9833773332728402    0.9834189841103409
0.3657299434240877      0.9832748940683826    0.9832228804189668
0.36588999209298895     0.9831788398803493    0.9830910621461344
0.36606352662610714     0.9830741528571654    0.9830262208145333
0.36622554285608105     0.9829759059609308    0.983019382601298
0.36638437550536446     0.9828791110605575    0.9830170487804142
0.3665566940188648      0.9827735597038394    0.982964496569311
0.3667174942292209      0.9826745567196521    0.9828414585886163
0.3668917803037939      0.9825666956002508    0.9826416210753466
0.36706288279767635     0.9824602407911639    0.9824304161287354
0.36722246698841454     0.9823604467182163    0.9822743419040552
0.3673955370433697      0.9822516656173783    0.9821841581943548
0.36755708879518056     0.9821496021683426    0.9821656149215673
0.3677154569663009      0.9820490588378309    0.9821675918047597
0.36788731100163824     0.9819394014396929    0.9821330407911472
0.3680476467338313      0.9818365733620573    0.9820284428719895
0.3682214683302413      0.981724526722639     0.9818368379791879
0.368383771623507       0.9816193677420636    0.981625255150917
0.36854289133608215     0.9815157659730063    0.9814421332489462
0.36871549691287425     0.981402815594925     0.981317647414341
0.3688765841865221      0.9812968673042314    0.9812778452606647
0.36905115732438687     0.9811814634472453    0.9812786910130075
0.36922254688156114     0.9810675690646389    0.9812584265415277
0.36938241813559114     0.9809607954656798    0.9811717956706325
0.3695557752538381      0.9808444308336106    0.9809915664285329
0.36971761406894077     0.9807352471267194    0.9807747325686333
0.3698762693033529      0.9806277284989566    0.980571576095267
0.37004841040198194     0.9805104908194624    0.9804177686311027
0.37020903319746673     0.980400548875754     0.9803561186473257
0.3703831418571685      0.9802807753232003    0.9803503530972936
0.3705540669361797      0.9801625813105533    0.9803425065652575
0.37071347371204666     0.9800518048551763    0.9802763649021204
0.37088636635213057     0.9799310571571563    0.9801135212941005
0.3710477406890702      0.9798177887254175    0.9798971026485973
0.3712226008902268      0.9796944362124864    0.9796552216972201
0.37139427751069287     0.9795727013089907    0.9794781164870109
0.3715544358280147      0.9794585705644933    0.97939737232066
0.3717280800095534      0.9793342127553761    0.9793829429828195
0.37189020588794786     0.9792175223751542    0.9793823587162862
0.37204914818565177     0.9791025761191192    0.9793341970138705
0.37222157634757264     0.978977261888337     0.9791899121988699
0.37238248620634923     0.9788597391625914    0.978978135786171
0.3725568819293428      0.9787317325686399    0.9787215491964548
0.3727280940716458      0.9786054184245174    0.9785154476033658
0.37288778791080457     0.9784870244817021    0.9784068935006632
0.3730609676141803      0.9783579999005102    0.9783756890984486
0.3732226290144117      0.9782369607316475    0.9783786629453325
0.3733811068339526      0.9781177444096852    0.9783486838848645
0.3735530705177104      0.977987752864097     0.9782277376578612
0.37371351589832397     0.9778658744876058    0.978027024886533
0.37388744714315447     0.9777331019672435    0.9777609635316045
0.3740498600848407      0.9776085093813436    0.9775359480355963
0.3742090894458364      0.9774857825080645    0.9773904855872888
0.3743818046710491      0.9773520136218333    0.9773306100936702
0.3745430015931175      0.9772265554420514    0.9773300887367087
0.37471768437940284     0.9770899336398775    0.9773145419387848
0.3748891835849976      0.9769551236543651    0.9772157469898967
0.37504916448744813     0.9768287599291203    0.9770288515582934
0.3752226312541156      0.9766910786923788    0.9767589171642819
0.3753845797176388      0.9765619124202911    0.9765116514852185
0.37554334460047145     0.9764346953486222    0.9763353568692796
0.3757155953475211      0.9762960092259984    0.9762472226071928
0.37587632779142643     0.9761659725164039    0.9762373313414116
0.37605054609954874     0.9760243529602866    0.9762336298906131
0.37622158082698054     0.9758846496439579    0.9761596030683997
0.37638109725126806     0.9757537326730112    0.9759926006082685
0.3765540995397725      0.9756110666277338    0.9757258046034408
0.3767155835251327      0.9754772573197329    0.9754596013795388
0.3768905533747098      0.9753315712022389    0.9752338085179106
0.37706233964359637     0.975187822760913     0.9751216602827915
0.3772226076093387      0.9750530732622477    0.9751010837365391
0.37739636143929794     0.9749062852804574    0.975102655521747
0.37755859696611294     0.9747685683113424    0.9750505785341709
0.3777176489122374      0.9746329331318729    0.9749040927420304
0.37789018672257885     0.9744851002389247    0.9746450960715314
0.378051206229776       0.9743464794455875    0.9743656767040545
0.3782257116011901      0.9741955299844134    0.9741076580624782
0.37839703339191366     0.9740466044288465    0.9739608959616414
0.37855683687949293     0.9739070371874097    0.97392070463387
0.37873012623128915     0.9737549756139411    0.9739241958220134
0.3788918972799411      0.9736123464705208    0.9738926125122621
0.37905048474790254     0.9734718894380667    0.9737713049519426
0.37922255808008093     0.9733187749710158    0.973527785008037
0.37938311310911504     0.9731752378997452    0.973240920582806
0.3795571540023661      0.9730189092501423    0.9729517814804024
0.3797196765924729      0.9728722337834212    0.9727716953278206
0.37987901560188914     0.972727779610467     0.9726992370363041
0.3800518404755223      0.9725703672286079    0.9726954297464415
0.3802131470460112      0.9724227562371723    0.9726847011013026
0.3803879394807171      0.9722620500042336    0.9725821081881479
0.38055954833473243     0.9721035048187161    0.972356801158652
0.3807196388856035      0.9719549145301989    0.9720681067856394
0.38089321530069153     0.9717930557661384    0.9717538150004654
0.3810552734126353      0.9716412297298526    0.9715381835999707
0.3812141479438885      0.9714917203760132    0.9714349389775082
0.38138650833935867     0.9713287720488778    0.9714175557403218
0.3815473504316846      0.9711760085313612    0.9714174505070925
0.3817216783882274      0.9710096658225743    0.9713428211125084
0.3818928227640797      0.9708455783988478    0.9711428215107403
0.38205244883678774     0.970691833259696     0.9708596843322573
0.3822255607737127      0.9705243317463004    0.9705246257161182
0.38238715440749343     0.970367268523572     0.9702717201088733
0.38254556446058363     0.9702126243096798    0.9701310806802714
0.3827174603778908      0.9700440509514138    0.9700910542786846
0.38287783799205366     0.9698860539603961    0.9700958019941675
0.3830517014704335      0.9697139842266187    0.9700493657256819
0.38321404664566905     0.9695525722667108    0.9698928189912179
0.38337320824021404     0.9693936279421237    0.9696288488617004
0.383545855698976       0.9692204325973313    0.9692809242606245
0.38370698485459365     0.9690580540860254    0.9689875010486705
0.38388159987442827     0.968881277943592     0.9687844008515649
0.3840530313135724      0.9687069052622186    0.9687168186122314
0.3842129444495722      0.9685435140992921    0.9687191733118062
0.384386343449789       0.9683655400579013    0.9686944847437904
0.3845482241468615      0.9681986310486751    0.96856746954765
0.38470692126324346     0.9680342925004592    0.9683232472748154
0.38487910424384236     0.9678551888100342    0.9679713928537864
0.385039768921297       0.9676873132055156    0.9676489972618705
0.38521391946296857     0.9675045225641515    0.9674004266429979
0.38538488642394964     0.9673242370232439    0.9672962760922599
0.3855443350817864      0.9671553483448151    0.9672880549211638
0.3857172696038402      0.9669713553111877    0.967281332880921
0.38587868582274965     0.9667988447413715    0.967187149176463
0.38605358790587607     0.9666110768503476    0.9669413454657123
0.38622530640831193     0.9664258694077906    0.9665882976782898
0.38638550660760357     0.9662523167383464    0.9662451859339914
0.3865591926711121      0.9660633135860908    0.9659611047454152
0.38672136043147637     0.965886052604988     0.9658286886177637
0.3868803446111501      0.9657115268408993    0.9658034759695172
0.3870528146550408      0.9655213605980079    0.9658065675233316
0.38721376639578725     0.9653431070209694    0.965740662622793
0.38738820400075064     0.9651490567196234    0.9655260436978856
0.3875594580250235      0.9649576738069561    0.96518283794557
0.3877191937461521      0.9647783799564822    0.9648212582578665
0.38789241533149765     0.9645830921542362    0.9644937674021957
0.38805411861369893     0.9643999828974484    0.9643176137220452
0.38821263831520963     0.9642197199684128    0.9642660797444977
0.3883846438809373      0.9640232691484778    0.9642716178222628
0.3885451311435207      0.9638391720828063    0.9642334687849049
0.388719104270321       0.9636387343339392    0.9640569305733137
0.3888815590939771      0.9634507403220314    0.9637525239396368
0.38904083033694264     0.9632656529523336    0.9633816614345203
0.38921358744412515     0.9630640209609104    0.9630080524664767
0.3893748262481634      0.962875010550479     0.9627748257513544
0.3895495509164186      0.9626692928528604    0.9626766315782431
0.3897210920039832      0.9624664113757213    0.9626762432496392
0.38988111478840354     0.9622763354229545    0.9626587427623257
0.39005462343704084     0.962069346926026     0.9625177749751549
0.39021661378253386     0.9618752572766817    0.9622375505834414
0.39037542054733637     0.9616841904664047    0.9618657896099189
0.39054771317635584     0.9614760093085492    0.9614600397399452
0.390708487502231       0.9612809088435535    0.9611800252271608
0.3908827476923232      0.9610685280766053    0.9610371731140842
0.39105382430172475     0.960859098539628     0.9610209481051849
0.3912133826079821      0.9606629377388664    0.9610196297970313
0.39138642677845636     0.9604492873013777    0.9609174311805971
0.39154795264578635     0.9602490010149314    0.9606704972960461
0.3917229643773333      0.9600310559618384    0.960266022527059
0.3918947925281897      0.959816124810033     0.9598375864960067
0.39205510237590185     0.9596147495080756    0.9595217289957239
0.39222889808783096     0.9593955021324445    0.9593414035172063
0.3923911754966158      0.9591899078812095    0.9593070293087936
0.3925502693247101      0.9589875220524745    0.959313460120264
0.39272284901702137     0.9587670544809408    0.959243921797834
0.39288391040618836     0.9585604294487804    0.9590307055134225
0.3930584576595723      0.9583355502592616    0.9586417200322616
0.3932298213322657      0.9581138053730461    0.9581940557621831
0.3933896667018148      0.9579060988203523    0.9578347702291123
0.39356299793558086     0.9576799207306954    0.9576011856141428
0.39372481086620265     0.957467880338303     0.9575352478091944
0.3938834402161339      0.9572591729488925    0.9575415569423971
0.39405555543028215     0.9570317804051071    0.9575034461216683
0.3942161523412861      0.9568187189320265    0.9573304666211336
0.394390235116507       0.9565867966366866    0.9569687249755016
0.39456113431103734     0.9563581320512128    0.9565115418007737
0.3947205152024234      0.9561439983876828    0.9561111049281867
0.3948933819580264      0.9559107762813766    0.955817602601667
0.39505473041048517     0.9556921826888801    0.9557083823885354
0.39522956472716086     0.9554543272120696    0.9557049578085388
0.39540121546314605     0.9552197970427021    0.9556854470759545
0.39556134789598696     0.9550001036399501    0.9555411915300626
0.3957349661930448      0.9547609230857552    0.9552018607806751
0.3958970661869584      0.9545366826809919    0.9547645801821973
0.3960559826001815      0.954315974321933     0.9543318742242205
0.39622838487762146     0.9540755575146671    0.9539827435646757
0.39638926885191716     0.9538502818678127    0.9538266853913014
0.3965636386904298      0.953605115799411     0.9538027874958906
0.39673482494825196     0.9533634041758275    0.9538020075914553
0.39689449290292983     0.9531370413524399    0.9536978783234057
0.39706764672182465     0.9528905588919232    0.9533990840573141
0.3972292822375752      0.9526595306403831    0.9529722175885417
0.39738773417263523     0.9524321676239287    0.9525140198631901
0.3975596719719122      0.9521844598471292    0.9521074831102698
0.39772009146804493     0.9519524111279363    0.9518949213871787
0.3978939968283946      0.9516998325215839    0.9518367807611857
0.3980563838856         0.9514630202344886    0.951845869321114
0.3982155873621148      0.9512299457229109    0.9517884613489287
0.3983882767028466      0.9509761123052536    0.9515500908749034
0.3985494477404341      0.9507382536492559    0.9511543187525244
0.39872410464223856     0.950479447771971     0.9506326390800969
0.3988955779633525      0.9502242996050945    0.9501769517910471
0.3990555329813222      0.9499853414478233    0.9499088661923876
0.3992289738635088      0.9497251989566703    0.9498091390067311
0.39939089644255116     0.9494813557604381    0.9498150882641839
0.39954963544090294     0.9492413888179044    0.9497874624495053
0.3997218603034717      0.948980004701303     0.9495973210515416
0.39988256686289614     0.9487351321550431    0.9492335196716227
0.40005675928653756     0.9484686509701582    0.9487079303700064
0.40022776812948846     0.9482059648385526    0.9482066824990734
0.4003872586692951      0.9479600094552295    0.9478772267835046
0.40056023507331867     0.9476922044151473    0.9477225765508692
0.400721693174198       0.947441241412948     0.947712929892758
0.40089663713929424     0.9471682340425116    0.9477038458697736
0.40106839752369994     0.9468990951754496    0.9475472355536615
0.40122863960496136     0.9466469928017729    0.9472086234838116
0.40140236755043973     0.9463725986063154    0.9466836321102204
0.40156457719277383     0.9461153849710606    0.946174552238472
0.4017236032544174      0.9458622686284233    0.9457891908572155
0.40189611518027796     0.9455866203068652    0.945576820282476
0.4020571088029942      0.9453283726051407    0.9455418444591684
0.40223158828992744     0.9450473954555273    0.9455483866291057
0.40240288419617015     0.9447704324923208    0.9454381416905507
0.4025626617992686      0.9445110972018658    0.9451433978674016
0.40273592526658397     0.9442287841188837    0.9446356092584086
0.4028976704307551      0.9439642140327941    0.9441005504328818
0.4030562320142356      0.9437038887555277    0.9436583621135503
0.4032282794619331      0.9434203419849366    0.9433776628276586
0.40338880860648635     0.9431547618274703    0.9433034638002222
0.4035628236152565      0.9428657597863529    0.9433144085387594
0.40372532032088243     0.9425948415086163    0.9432570041914948
0.4038846334458178      0.9423282483870307    0.9430259129771066
0.40405743243497017     0.9420379858492491    0.9425602474087782
0.40421871312097823     0.9417660342327513    0.9420131493543927
0.40439347967120326     0.9414702096744818    0.9414661925036247
0.4045650626407377      0.9411786248130539    0.9411195817738732
0.4047251273071279      0.9409055851915896    0.9409998626587743
0.40489867783773503     0.940608416742604     0.9410024779103798
0.4050607100651979      0.9403299125601794    0.9409770284391481
0.40521955871197024     0.9400558861018806    0.9407954823207376
0.40539189322295954     0.939757479744772     0.9403707285964469
0.40555270943080457     0.9394779683643395    0.9398245437443992
0.40572701150286655     0.9391738706420725    0.9392315126184169
0.40589812999423797     0.9388741641343924    0.9388129306426085
0.4060577301824651      0.9385935905661823    0.9386346388453806
0.4062308162349092      0.9382881711517813    0.9386141455917884
0.406392383984209       0.9380020055614231    0.9386137843290402
0.4065507681528183      0.9377204729289593    0.938485374057951
0.4067226381856446      0.9374138398759029    0.938115227620582
0.40688298991532657     0.9371266948739274    0.9375852880038602
0.4070568275092255      0.9368142401199837    0.9369563591559612
0.40721914679998017     0.9365213961507004    0.9364823648762185
0.4073782825100443      0.9362332529234184    0.9362214594980495
0.4075509040843254      0.9359195017977632    0.9361516738318009
0.40771200735546226     0.935625603333438     0.9361651019621101
0.407886596490816       0.935305920347344     0.9360801225842379
0.4080580020454792      0.9349908678630351    0.9357645884625295
0.40821788929699815     0.9346959138264291    0.935260209876535
0.40839126241273405     0.9343749091323373    0.9346101963852287
0.40855311722532567     0.9340741277959071    0.9340762757393372
0.4087117884572268      0.9337782267578326    0.9337444248066837
0.40888394555334484     0.9334560141516333    0.9336209302898713
0.40904458434631863     0.933154266518528     0.9336298474864845
0.4092187090035094      0.9328259927911493    0.9335889612420025
0.40938965008000955     0.9325025105372619    0.9333372341582813
0.40954907285336545     0.9321997422263707    0.9328738960341854
0.4097219814909383      0.9318701785276361    0.9322180996991197
0.4098833718253669      0.931561455304333     0.9316292368742581
0.4100582480240124      0.931225719164524     0.931189390830531
0.41022994064196744     0.9308948624284376    0.9310185933694031
0.4103901149567782      0.9305850986095441    0.9310153030397929
0.4105637751358059      0.9302480488317514    0.9309997144386329
0.4107259170116893      0.9299322202730367    0.9308097166646424
0.41088487530688217     0.9296215276984277    0.9303917763142193
0.411057319466292       0.9292832815022161    0.929743640839807
0.4112182453225575      0.9289665046401342    0.9291137514465662
0.41139265704304        0.9286219540860784    0.9285955772184158
0.411563885182832       0.9282824485379185    0.9283528954874566
0.4117235950194797      0.9279646679452941    0.9283203498657331
0.41189679072034435     0.9276188374648286    0.9283282035329181
0.41205846811806474     0.9272948618622305    0.9281966384114849
0.4122169619350946      0.9269761910868703    0.9278370619876698
0.41238894161634143     0.9266291997117366    0.9272144522631977
0.412549402994444       0.9263043143943633    0.9265548319370714
0.41272335023676343     0.925950885891401     0.9259575875903839
0.41288577917593866     0.9256196951318709    0.9256387277786902
0.4130450245344233      0.9252939011706078    0.9255496760739058
0.41321775575712494     0.9249392896244494    0.9255657586298045
0.4133789686766823      0.9246071704211634    0.9254998752977359
0.4135536674604566      0.924246008031307     0.9251792557154205
0.41372518266354036     0.9238900862712124    0.9245915575462248
0.41388517956347987     0.9235569056998879    0.9239168443588994
0.41405866232763633     0.9231943930434307    0.9232538316662867
0.4142206267886485      0.922854774670938     0.9228556754557093
0.41437940766897013     0.922520729214383     0.9227084031281515
0.4145516744135087      0.9221570754766032    0.9227130844654307
0.414712422854903       0.9218165743761042    0.9226877317062927
0.41488665716051426     0.9214462379177959    0.9224383578579454
0.415057707885435       0.9210813818397486    0.9219030474850994
0.41521724030721147     0.9207399442684293    0.9212290336277784
0.4153902585932049      0.9203683868154273    0.9205072958994038
0.41555175857605403     0.9200203830700496    0.9200230118278878
0.41572674442312013     0.919642029628995     0.9197910764614443
0.41589854668949566     0.9192692523874348    0.9197775389005809
0.4160588306527269      0.9189202978966865    0.9197742719351445
0.41623260048017513     0.9185407057723954    0.9195742638776828
0.41639485200447907     0.9181850732190269    0.919111188616516
0.4165539199480925      0.917835290485693     0.9184504173269846
0.41672647375592287     0.9174545881385561    0.9176863312511502
0.41688750926060897     0.9170981092845599    0.9171252133780292
0.417062030629512       0.9167104789915593    0.9168101076335253
0.41723336841772457     0.9163286041390936    0.9167574596028409
0.41739318790279284     0.9159712243459356    0.9167730583315118
0.417566493252078       0.9155824026793038    0.9166389089612268
0.41772828029821896     0.9152182141768442    0.9162423064667259
0.41788688376366934     0.9148600573050225    0.9156125439356615
0.41805897309333667     0.9144701741705893    0.9148198842508669
0.41821954411985973     0.9141051905489144    0.9141824706059721
0.41839360101059975     0.9137082468503224    0.9137704517954086
0.41856447432064925     0.9133172396916085    0.9136580704084024
0.4187238293275545      0.9129514061078       0.9136772390678884
0.41889667019867666     0.9125533195432342    0.913605971458948
0.41905799276665456     0.9121805459292806    0.9132877805316966
0.4192328011988494      0.9117752827549436    0.9126360293367562
0.4194044260503537      0.9113760551356068    0.9118284307360744
0.41956453259871374     0.9110024177364192    0.9111369874405724
0.4197381250112907      0.910595994424739     0.9106486871111513
0.4199001991207234      0.9102152769317851    0.9104835224560854
0.4200590896494656      0.9098408014896538    0.9104897013312466
0.42023146604242473     0.9094332419565225    0.910464066412759
0.4203923241322396      0.9090516938120847    0.9102179191892976
0.4205666680862714      0.9086368240946464    0.9096276158845827
0.42073782845961266     0.9082281789931835    0.908822904838799
0.42089747052980964     0.9078458260221499    0.9080745638118207
0.42107059846422357     0.9074298545195396    0.907486923489542
0.4212322080954932      0.9070403180023172    0.9072395073689307
0.42139063414607236     0.9066572917962981    0.9072134359319507
0.42156254606086846     0.9062403566020798    0.9072218461616828
0.4217229396725203      0.9058501311823768    0.9070518139527449
0.42189681914838906     0.9054257579846408    0.9065417696347896
0.42205918032111356     0.905028238802525     0.9058044148039505
0.42221835791314755     0.9046373337838505    0.9050105432716373
0.42239102136939843     0.9042119876596215    0.9043082407305127
0.42255216652250505     0.9038137730274102    0.9039455857485539
0.4227267975398286      0.9033808761545274    0.9038539476218185
0.42289824497646167     0.9029544969627329    0.9038769514476607
0.42305817410995045     0.9025555343865318    0.9037696992236887
0.4232315891076562      0.9021215880657272    0.9033392816904263
0.42339348580221764     0.9017152034501102    0.9026413866184944
0.4235521989160886      0.9013156266800552    0.9018269864832797
0.4237243978941765      0.9008807712603167    0.9010417546035699
0.4238850785691201      0.9004737566191204    0.9005811917087486
0.4240592451082807      0.9000312208875101    0.9004149120191658
0.4242302280667507      0.8995953960568037    0.900434205516933
0.42438969272207644     0.8991876987773655    0.9003851569093951
0.42456264324161913     0.8987441772806667    0.9000463229773539
0.42472407545801755     0.8983289292628892    0.8994093475487484
0.4248989935386329      0.8978776122103379    0.8985058561316256
0.4250707280385578      0.8974331117834007    0.8976632180043417
0.42523094423533836     0.8970171743759714    0.8971276104313534
0.4254046462963359      0.8965648618481918    0.8968959501610685
0.42556683005418916     0.8961412596821563    0.8968975267050391
0.42572583023135185     0.8957247688114675    0.8968888905622668
0.4258983162727315      0.895271603431154     0.8966322020284093
0.4260592840109669      0.8948474318047708    0.8960619007847755
0.4262337376134192      0.8943862552655891    0.8951728884373577
0.426405007635181       0.8939320413047438    0.8942713684387247
0.42656475935379856     0.8935071222129308    0.8936388554776886
0.42673799693663306     0.8930449657954216    0.8933083499424859
0.4268997162163233      0.8926122533684578    0.8932699509212941
0.427058251915323       0.8921868549854921    0.8932892315629104
0.4272302734785396      0.8917239204870533    0.8931177158482525
0.4273907767386119      0.8912907159717337    0.8926314775546063
0.4275647658629012      0.8908197297224638    0.8917817684916801
0.4277272366840462      0.8903786236850371    0.8908821071282038
0.4278865239245007      0.8899449417687546    0.8901395224654207
0.42805929702917217     0.8894731770592843    0.8896732578249432
0.42822055183069935     0.8890315807000579    0.8895575090449945
0.4283952924964435      0.8885516540776386    0.8895858290892275
0.4285668495814971      0.8880790544851002    0.8894829664317166
0.42872688836340644     0.8876369188836502    0.8890779174281634
0.42890041300953274     0.8871561442715598    0.8882804798552071
0.42906241935251477     0.886705984194056     0.8873633065187682
0.4292212421148062      0.8862634522248142    0.886543240171173
0.42939355074131463     0.8857819796244076    0.8859645124549126
0.42955434106467877     0.8853314102795576    0.8857693202827062
0.42972861725225986     0.8848416523636329    0.885783177572841
0.42989970985915044     0.8843594253065429    0.885742568546883
0.43005928416289674     0.8839083977589828    0.8854291342043921
0.43023234433086        0.8834178723050572    0.8847076386851076
0.430393886195679       0.8829586972128269    0.8837949446494908
0.4305689139247149      0.8824597744043493    0.8828243935255932
0.4307407580730603      0.8819684940770082    0.8821613623047463
0.4309010839182614      0.8815088625583505    0.8818991415659333
0.4310748956276794      0.8810091716672671    0.8818887652463892
0.4312371890339532      0.8805412815608937    0.8818876458424776
0.43139629885953645     0.8800813378790794    0.8816548230869884
0.43156889454933667     0.8795810303526427    0.881013543460351
0.4317299719359926      0.8791128150805441    0.8801228467492461
0.4319045351868655      0.8786039856853587    0.8790976409635702
0.4320759148570478      0.8781030045588922    0.8783260574917662
0.4322357762240859      0.8776344147454469    0.8779634798628541
0.4324091234553409      0.8771248839675171    0.8779006616646817
0.4325709523834516      0.8766477614697126    0.877927031367407
0.43272959773087183     0.8761787953848011    0.8777766607242188
0.432901728942509       0.8756685868494908    0.8772342728602407
0.4330623418510019      0.87519122639222      0.8763897523204197
0.43323644062371175     0.8746723744209012    0.875329107102495
0.4334073558157311      0.874161583307048     0.8744497290723209
0.43356675270460615     0.8736839412292595    0.8739701759292656
0.43373963545769817     0.8731644976452755    0.8738272259327975
0.4339009999076459      0.8726783564529568    0.8738600695569658
0.4340758502218106      0.8721501634411415    0.8737639580763029
0.43424751695528474     0.8716301473601459    0.8732944224819309
0.4344076653856146      0.8711437361168969    0.8724913260860298
0.4345812996801614      0.8706149615211686    0.8714146946357975
0.43474341567156394     0.870119945885834     0.8705001114351609
0.43490234808227596     0.8696334154251804    0.8699101206052082
0.43507476635720493     0.8691042176872885    0.8696755356460167
0.43523566632898963     0.868609073556786     0.8696920859885917
0.4354100521649913      0.8680710123387008    0.8696601424339963
0.4355812544203024      0.8675413404524736    0.8692955882236157
0.4357409383724692      0.8670460240262665    0.8685696025966296
0.43591410818885296     0.8665074795966813    0.8675009915747379
0.43607575970209245     0.8660034444419566    0.8665124437600091
0.4362342276346414      0.8655081062473416    0.8658043222333931
0.43640618143140736     0.8649692367481636    0.8654537258672387
0.436566616925029       0.8644651705703578    0.8654280771134369
0.43674053828286763     0.8639173237822098    0.8654443825589816
0.43690294133756197     0.8634044348969464    0.8652128826606181
0.43706216081156574     0.8629003583649765    0.8646068360576685
0.43723486614978646     0.8623521963844093    0.8635894301845346
0.4373960531848629      0.8618392878380894    0.8625423485333993
0.4375707260841563      0.8612820432869237    0.861629603892255
0.4377422154027592      0.8607335149591275    0.8611621024863418
0.4379021864182178      0.8602205428118286    0.8610769387536134
0.4380756432978934      0.8596629228212386    0.8611148421926722
0.4382375818744247      0.859141013284731     0.8609702182712153
0.43839633687026547     0.8586281285173324    0.8604643940029142
0.4385685777303232      0.8580702916922331    0.8595084879269356
0.43872930028723667     0.857548388004161     0.8584374139974433
0.4389035087083671      0.8569811733232227    0.8574166943387636
0.43907453354880693     0.8564228919275456    0.8568152414138606
0.4392340400861025      0.8559009316901106    0.8566442199717427
0.43940703248761503     0.8553334469872208    0.8566791549141148
0.4395685065859833      0.8548024385777296    0.8566149910126433
0.4397434665485685      0.8542256564999733    0.8561579124428034
0.4399152429304632      0.8536579279891061    0.8552540167996505
0.4400755010092136      0.8531269806666171    0.8541752040580998
0.440249244952181       0.8525499503712022    0.8530816888756272
0.4404114705920041      0.8520098567600677    0.852401692185609
0.4405705126511366      0.8514791277142769    0.8521362207381951
0.4407430405744861      0.850902014409528     0.8521440001834667
0.4409040501946913      0.8503621340308564    0.8521360485648927
0.4410785456791135      0.8497756217775461    0.8517909971656933
0.44124985758284513     0.8491983814779691    0.8509801529888393
0.4414096511834325      0.8486586770608655    0.8499188932240686
0.44158293064823684     0.8480720334246394    0.8487507301215612
0.4417446918098969      0.8475230801717828    0.8479450923162415
0.4419032693908664      0.8469837078810827    0.8475626757438439
0.4420753328360528      0.8463970970157654    0.8475142933686153
0.442235877978095       0.8458484708696723    0.8475457771081879
0.4424099089843541      0.8452523601129431    0.8473164750755809
0.44257242168746896     0.8446943889388174    0.8466678312794854
0.4427317508098933      0.8441461166419402    0.8456674716624849
0.4429045657965346      0.8435500597077075    0.8444446584375838
0.4430658624800316      0.8429924373923627    0.843496210125844
0.4432406450277456      0.8423867838634198    0.842921393201253
0.44341224399476903     0.8417907370073188    0.8427996389423486
0.4435723246586482      0.8412334264780519    0.8428450693633498
0.44374589118674435     0.8406277786924963    0.8427092476366465
0.4439079394116962      0.8400610211027062    0.8421699353742975
0.4440668040559575      0.8395041779302636    0.8412365772583561
0.44423915456443575     0.8388986994040905    0.839995878049025
0.4443999867697697      0.8383324042051508    0.8389484903925526
0.44457430483932064     0.8377172286548977    0.8382296523943871
0.44474543932818106     0.8371118759041359    0.8380057689881341
0.4449050555138972      0.836546006260177     0.8380409455505773
0.4450781575638303      0.8359307821006303    0.8379899342053864
0.4452397413106191      0.8353551834447026    0.8375734204610367
0.44539814147671736     0.8347897182670481    0.8367346577110882
0.44557002750703256     0.8341747598315272    0.8355066192713843
0.4457303952342035      0.8335997441563343    0.8343740251793423
0.4459042488255914      0.8329749939339843    0.833501067025502
0.446066584113835       0.8323903408657453    0.8331519302612808
0.4462257358213881      0.8318159413128668    0.8331354266971448
0.44639837339315813     0.8311915138653029    0.8331580860906014
0.4465594926617839      0.830607476595728     0.8328920278186241
0.44673409779462664     0.8299731703678896    0.8321001557629687
0.4469055193467788      0.8293490301337668    0.8309087747691767
0.4470654225957867      0.8287655791671119    0.8297194950808034
0.44723881170901153     0.8281315609571336    0.8287124470487958
0.4474006825190921      0.8275383846784934    0.8282334103476495
0.44755936974848215     0.8269556797366318    0.8281481824800582
0.44773154284208916     0.8263221175315398    0.828198562719094
0.4478921976325519      0.8257296869798716    0.8280378737046684
0.4480663382872316      0.8250861608304114    0.8273791827246576
0.44823729536122076     0.8244530195005821    0.8262571057041742
0.44839673413206566     0.8238613055575131    0.8250348911536611
0.4485696587671275      0.8232182011064441    0.8238977567679499
0.4487310650990451      0.822616674986954     0.8232695246155322
0.44890595729517957     0.821963520283417     0.8230827069145916
0.44907766591062354     0.8213208705669391    0.8231358676733709
0.4492378562229232      0.8207200946856423    0.823041535544732
0.4494115323994399      0.8200673956640819    0.8224808165257906
0.44957369027281224     0.8194567213193367    0.821481959952873
0.4497326645654941      0.8188568536602684    0.8202540369473035
0.4499051247223929      0.8182047764683946    0.8190153858739753
0.45006606657614745     0.8175950101917827    0.8182487082511744
0.45024049429411894     0.8169327990329827    0.8179427867082666
0.45041173843139987     0.8162813092063437    0.8179739450313317
0.4505714642655365      0.8156724224449761    0.8179541265606287
0.4507446759638901      0.8150107996171813    0.8175301877263352
0.45090636935909945     0.8143919289785729    0.8166376788701681
0.45106487917361826     0.8137840779352623    0.8154336606771034
0.45123687485235403     0.8131231569967804    0.8141105802660573
0.4513973522279455      0.8125051362184192    0.8131982914155963
0.451571315467754       0.8118338491120746    0.8127437592962587
0.45173376040441815     0.8112057612507564    0.8127146820556
0.45189302176039176     0.8105888150862648    0.8127556325153928
0.4520657689805823      0.8099183227610584    0.8124963397267838
0.4522269978976286      0.8092913142323501    0.8117631124873183
0.45240171267889184     0.8086105296150364    0.8105008462199339
0.45257324387946457     0.8079408061084468    0.8091248991814204
0.452733256776893       0.8073148562199399    0.8080886784252277
0.45290675553853843     0.8066348467515638    0.8074866323305038
0.45306873599703956     0.8059987589806027    0.8073795474855637
0.4532275328748502      0.8053740285033415    0.8074387152933483
0.45339981561687775     0.8046949633727247    0.8072915901512201
0.45356058005576105     0.8040600997584479    0.8066901742128272
0.45373483035886125     0.8033706754698428    0.8055151919231066
0.45390589708127094     0.8026925293209851    0.8041134270869132
0.45406544550053635     0.8020588696912762    0.8029590544658806
0.4542384797840187      0.8013703707301534    0.8021903495048265
0.4543999957643568      0.8007265039014174    0.8019726845177069
0.45457499760891185     0.8000275728194396    0.802027074991286
0.4547468158727764      0.7993400391816514    0.8019486468276559
0.45490711583349663     0.7986974214901401    0.8014417407303853
0.45508090165843385     0.797999462297632     0.800340682984825
0.4552431691802268      0.797346564075171     0.7990052848976217
0.45540225312132915     0.7967053533687937    0.7977611166287191
0.45557482292664847     0.7960085321704413    0.7968406509230318
0.4557358744288235      0.7953570459155719    0.7965031430825076
0.4559104117952155      0.7946497281500859    0.796520897646314
0.456081765580917       0.7939540206945929    0.7965186415270264
0.4562416010634742      0.7933039271817425    0.7961427415112644
0.4564149224102484      0.7925977297401384    0.7951664298272082
0.4565767254538783      0.7919372888009669    0.7938623768997727
0.4567353449168176      0.7912887433156541    0.7925457875431782
0.4569074502439739      0.7905838296316757    0.791467506421116
0.4570680372679859      0.7899249417283817    0.7909835726919315
0.45724211015621485     0.7892094684836173    0.7909284198383522
0.4574129994637533      0.7885058149070266    0.79097965862785
0.45757237046814747     0.7878483323664768    0.7907375489903953
0.4577452273367586      0.7871339286145742    0.7899145955736048
0.45790656590222545     0.7864659794435246    0.7886788784487723
0.45808139033190926     0.7857409483444684    0.7871802168719092
0.45825303118090255     0.785027858802147     0.7859923166155901
0.45841315372675157     0.7843614987028737    0.7853935925977471
0.4585867621368175      0.7836377937801153    0.7852653116708752
0.4587488522437391      0.7829609588935516    0.785333978760138
0.45890775876997025     0.7822963467832148    0.7851991053342944
0.45908015116041834     0.7815741353764466    0.7845231978779694
0.45924102524772215     0.7808990583878428    0.7833751747959554
0.4594153851992429      0.7801661730418272    0.7818576726551645
0.45958656157007316     0.7794454404336787    0.7805409900279973
0.45974621963775913     0.7787721108545845    0.7797834314317369
0.45991936356966207     0.7780407162855906    0.7795314513000132
0.4600809891984207      0.7773568621411117    0.779591384111888
0.4602394312464888      0.7766854355585286    0.7795526071786303
0.46041135915877385     0.7759556956281052    0.7790393738169291
0.4605717687679146      0.7752737544158352    0.7780127011299216
0.46074566424127233     0.7745332958175719    0.776514177790983
0.4609080414114858      0.773840772283495     0.7751477650881642
0.4610672350010087      0.7731607879283398    0.7741938551705548
0.4612399144547486      0.7724220401696406    0.7737535680015303
0.4614010756053442      0.7717314837294568    0.7737544294075155
0.4615757226201568      0.770981961776509     0.7737891851538381
0.4617471860542788      0.7702449116702134    0.7734179557273106
0.4619071311852565      0.7695563131513012    0.7725177693599128
0.4620805621804512      0.768808501154631     0.7710684100487321
0.4622424748725016      0.768109273896481     0.7696345797997993
0.4624012039838615      0.7674227848204216    0.7685358600188835
0.4625734189594383      0.7666768423577686    0.7679302852786067
0.4627341156318709      0.7659797348052517    0.767850182287578
0.4629082981685204      0.7652229767747454    0.7679243775317095
0.4630792971244794      0.7644788914286632    0.7676958632933932
0.46323877777729416     0.7637838950612402    0.7669470838712418
0.4634117442943258      0.7630290064617716    0.7655866134561676
0.46357319250821316     0.7623233368207112    0.7641145594334405
0.4637481265863175      0.7615575563292322    0.7627718670702807
0.4639198770837313      0.7608043982408415    0.762040564421335
0.4640801092780008      0.7601007241509211    0.7618849199366204
0.4642538273364873      0.7593367108225725    0.7619663315361591
0.4644160270918295      0.7586223116999682    0.7618471602289447
0.4645750432664812      0.7579209597862885    0.7612402494671185
0.46474754530534984     0.7571590401263466    0.7599860899181357
0.4649085290410742      0.7564469781421709    0.7585099467415138
0.46508299864101554     0.7556741607177824    0.7570457594120917
0.4652542846602663      0.7549143311124508    0.7561409706094836
0.4654140523763728      0.754204605829028     0.7558607374931806
0.46558730595669623     0.7534338959261879    0.755920370603396
0.4657490412338754      0.7527134166752588    0.7558975372772885
0.46590759293036405     0.7520061779448317    0.755445197939452
0.46607963049106965     0.7512377334917757    0.7543324961101674
0.466240149748631       0.7505197557824757    0.7528881199658972
0.46641415487040927     0.7497403907408634    0.7513231701493535
0.4665766416890433      0.7490116172514761    0.7502732559611781
0.4667359449269867      0.7482961896586372    0.7498076190268407
0.4669087340291471      0.7475191569739789    0.7497885807500274
0.46707000482816324     0.7467929489780681    0.7498465787262831
0.4672447614913963      0.7460049570317758    0.7495195689915743
0.4674163345739389      0.7452302550305537    0.7485502794561699
0.4675763893533372      0.744506613762376     0.7471635469077704
0.4677499299969524      0.7437209703784284    0.7455370181412752
0.4679119523374234      0.742986508662214     0.7443397177946826
0.46807079109720384     0.7422655788019603    0.7437160132270161
0.4682431157212012      0.7414824364631782    0.7436000109063974
0.4684039220420543      0.7407507015575079    0.7436857044044762
0.4685782142271243      0.7399565813927517    0.7435000059197846
0.46874932283150383     0.7391759393513169    0.7426995489746725
0.4689089131327391      0.7384469332689046    0.7414070018790478
0.4690819892981913      0.7376553312867206    0.7397512992060851
0.4692435471604992      0.736915482353893     0.7384134297490442
0.4694019214421166      0.7361893457664522    0.7376119215370925
0.469573781587951       0.7354004102561442    0.7373587244405021
0.4697341234306411      0.7346634462459659    0.7374368757281006
0.46990795113754813     0.7338635165637472    0.7373784428483122
0.4700702605413109      0.7331156005332925    0.736811962849994
0.4702293863643831      0.7323814181520548    0.735679500036707
0.47040199805167227     0.7315840587391779    0.7340470456039099
0.47056309143581715     0.730839014184963     0.7325730782619134
0.470737670684179       0.7300306303088043    0.7314787237302682
0.4709090663518503      0.7292360129815415    0.7310775978608006
0.47106894371637736     0.728493929862324     0.7311152465681889
0.47124230694512137     0.7276883106167551    0.7311431801107083
0.4714041518707211      0.7269353381119704    0.7307348390170983
0.47156281321563026     0.726196357270363     0.7297477132281106
0.4717349604247564      0.7253936511987923    0.7281731764921678
0.4718955893307382      0.7246438006598456    0.7266258419449878
0.472069704100937       0.7238300696060638    0.7253514909755254
0.4722406352904453      0.7230302852885249    0.7247705629551231
0.4724000481768093      0.722283567181976     0.7247288036150812
0.47257294692739027     0.7214727804301606    0.7248140356475993
0.47273432737482696     0.7207151679892226    0.724564467455059
0.4729091936864806      0.7198933351674672    0.7236306186545869
0.4730808764174437      0.7190855485187787    0.7221161449207304
0.4732410408452625      0.7183311428653594    0.720534263746482
0.4734146911372982      0.7175123331819597    0.719138500566946
0.4735768231261897      0.7167470105803058    0.7184374495797828
0.47373577153439067     0.7159959463735581    0.7182950750116283
0.4739082058068086      0.7151803018652669    0.7183969949994223
0.47406912177608224     0.7144183408911845    0.7182727902635032
0.47424352360957284     0.7135916548779503    0.717517320137405
0.4744147418623729      0.7127791864358823    0.7161169627774692
0.47457444181202874     0.7120205995495191    0.7145230257490077
0.4747476276259015      0.7111971126396625    0.7129862920817778
0.47490929513663        0.7104276089433783    0.7121020657118271
0.4750677790666679      0.7096725261827475    0.711823018432009
0.4752397488609228      0.7088523755739723    0.7119032494559243
0.4754002003520334      0.7080863962083165    0.7118889210071785
0.47557413770736096     0.7072552111999483    0.7113276527595506
0.47573655675954424     0.7064782970919278    0.7101564196053141
0.475895792231037       0.7057158925366421    0.7086018052761116
0.47606851356674673     0.7048881188668095    0.7069327840228158
0.4762297165993122      0.7041148002422493    0.7058244233547492
0.4764044054960946      0.7032758672673216    0.7053232872127219
0.4765759108121864      0.7024513873389874    0.7053490882179387
0.476735897825134       0.7016815576086394    0.705406029421451
0.4769093707022985      0.7008460540992791    0.7050124506727311
0.47707132527631874     0.7000652969261569    0.7039955077602621
0.47723009626964846     0.6992992078437407    0.7025073533304003
0.47740235312719514     0.6984672920765099    0.7007723868976414
0.47756309168159755     0.6976902993505825    0.6995024392413309
0.4777373161002169      0.6968473544254985    0.6988112028133934
0.4779083569381457      0.6960190458150196    0.6987415448096806
0.4780678794729302      0.6952458376700044    0.6988417194034147
0.4782408878719317      0.6944065268546897    0.6986140554081453
0.4784023779677889      0.6936224077496456    0.6977785692088887
0.47857735392786305     0.6927720646667285    0.6962311702771591
0.4787491463072467      0.6919364466555702    0.6944690179313212
0.47890942038348605     0.6911561930293175    0.693093548079434
0.47908318032394237     0.6903095702582033    0.6922599796957083
0.4792454219612544      0.6895184007192311    0.6920982810973236
0.47940448001787594     0.6887421340013524    0.6922052269979883
0.47957702393871443     0.6878993597465629    0.6921059892447586
0.47973804955640864     0.6871122014713108    0.6914389249355528
0.4799125610383198      0.6862584220732943    0.6900222952692293
0.4800838889395404      0.6854195183602054    0.6882588198038432
0.4802436985376167      0.6846363937368493    0.6867611174535289
0.48041699399991        0.6837865123857594    0.6857320826269604
0.480578771159059       0.6829924940829555    0.6854275744037899
0.4807373647375175      0.6822135194368943    0.6855059067971241
0.48090944418019294     0.6813676589885289    0.6855172590770645
0.4810700053197241      0.6805778150407135    0.6850310955102132
0.4812440523234722      0.6797209793648484    0.6837843774852952
0.4814149157465298      0.6788791630002442    0.6820626298276017
0.48157426086644306     0.6780935167572071    0.6804651403189773
0.4817470918505733      0.6772407526672576    0.6792316520449573
0.48190840453155925     0.6764442378075243    0.6787484874312224
0.48208320307676217     0.675580503459703     0.6787682676036186
0.48225481804127457     0.6747318678952867    0.6788380075703863
0.4824149147026427      0.67393963033878      0.6784757489378546
0.4825884972282278      0.6730800190443252    0.6773638323858518
0.4827505614506686      0.6722768292221       0.6757851972520392
0.4829094420924188      0.6714888937337322    0.6741297386135633
0.483081808598386       0.6706334972169329    0.6727253610113136
0.4832426568012089      0.6698347221366384    0.6720659874100176
0.4834169908682488      0.6689683930296917    0.6719827786249006
0.48358814135459816     0.6681173001605735    0.6720992027986255
0.48374777353780324     0.6673229687567489    0.6718916466136767
0.4839208915852253      0.666460973746809     0.670976103123816
0.48408249132950304     0.6656558124548135    0.6695057817470556
0.4842409074930903      0.6648660319647425    0.6678262213298695
0.4844128095208945      0.6640084844119524    0.66626056223905
0.4845731932455544      0.6632079011566893    0.6654030193983116
0.4847470628344313      0.662339465983491     0.6651715439142994
0.4849094141201639      0.6615280647216828    0.6652881649636065
0.48506858182520596     0.6607321132337717    0.66524228645202
0.48524123539446495     0.6598682120705505    0.6645813777213084
0.4854023706605797      0.6590614702758283    0.6632928982532131
0.48557699179091135     0.6581866986302564    0.6614634921499083
0.4857484293405525      0.6573273636504998    0.6597792843636242
0.4859083485870494      0.6565253126825147    0.658743874153897
0.48608175369776324     0.6556551382637404    0.658351447481951
0.4862436405053328      0.6548423122770447    0.6584312718110263
0.4864023437322118      0.6540450518556745    0.6584779773051581
0.48657453282330776     0.6531795801186072    0.6580080503133792
0.48673520361125944     0.6523715723983727    0.6568862162095201
0.48690936026342807     0.6514952813210214    0.6551211142425085
0.4870803333349062      0.650634545662211     0.6533436833390718
0.48723978810324003     0.6498313885822539    0.6521248708113179
0.48741272873579083     0.6489598646192402    0.6515359846248014
0.48757415106519736     0.6481459785035775    0.6515357004246584
0.48774905925882084     0.6472636582691893    0.6516434802041214
0.48792078387175375     0.6463969593486026    0.6513012234604996
0.4880809901815424      0.6455880073923557    0.65030752087385
0.488254682355548       0.644710543842209     0.6486119473799032
0.4884168562264093      0.6438908837722962    0.6468748314445937
0.4885758465165801      0.6430869581069697    0.6455060238108498
0.48874832267096785     0.6422144487175658    0.6447279745219011
0.48890928052221133     0.6413998119428244    0.6446227643510842
0.48908372423767177     0.6405165170057218    0.6447605594625376
0.4892549843724417      0.6396489525588398    0.644577448488656
0.48941472620406734     0.6388393946424005    0.6437684607041744
0.48958795389990994     0.6379611236836634    0.6422006667163097
0.4897496632926083      0.6371409109212143    0.640447269881863
0.48990818910461603     0.6363365320415844    0.6389389220029228
0.49008020078084075     0.6354633785695308    0.6379508315227918
0.4902406941539212      0.6346483742665298    0.637700803465638
0.4904146733912186      0.633764544966911     0.6378267064418395
0.4905771343253717      0.6329389136616169    0.6378009847364846
0.4907364116788343      0.6321291702669041    0.6372286231387319
0.49090917489651387     0.6312505464427591    0.6358666687042878
0.49107041981104915     0.6304302060719988    0.6341513646015351
0.4912451505898014      0.6295409398887855    0.6323599570620924
0.49141669778786307     0.6286675624511608    0.6311865869813106
0.49157672668278046     0.6278525520060572    0.6307842782168791
0.4917502414419148      0.6269685656574152    0.6308605522766807
0.4919122378979049      0.6261429898077104    0.6309250505250449
0.49207105077320445     0.6253333897774397    0.6305266420592308
0.492243349512721       0.6244547685364559    0.629347499329098
0.4924041299490932      0.623634633145138     0.6277038346690067
0.4925783962496824      0.6227454394871059    0.6258354014505818
0.49274947896958105     0.6218722255131577    0.6244724404055623
0.4929090433863354      0.6210575706032728    0.6238872500222171
0.4930820936673067      0.6201738176492876    0.6238691848791387
0.49324362564513374     0.6193486619901735    0.6239941332358011
0.49341864348717773     0.6184543763827376    0.6237093077579187
0.4935904777485312      0.6175761211381539    0.6226671587394004
0.4937507937067404      0.6167565305549796    0.6210946794477704
0.49392459552916657     0.6158677767275839    0.6191958127005743
0.49408687904844845     0.6150377228339243    0.6177667051855382
0.4942459789870398      0.6142237699088887    0.6170087230636946
0.49441856478984814     0.6133406248235863    0.6168737214130773
0.4945796322895122      0.6125162392572311    0.6170177567123217
0.49475418565339313     0.6116226379440435    0.6168850304130483
0.49492555543658356     0.6107451467022174    0.6160401103199779
0.4950854069166297      0.6099264726597362    0.6145929177724407
0.49525874426089284     0.6090385633820822    0.6126882526865168
0.4954205633020117      0.6082095003832915    0.6111220297270189
0.49557919876244        0.6073966074118966    0.6101736146507383
0.4957513200870853      0.6065144570764825    0.6098811338892115
0.4959119231085863      0.6056912025503045    0.6100064491520097
0.49608601199430424     0.6047986752760174    0.6100080632126212
0.49625691729933163     0.6039223300598002    0.6093773069451439
0.49641630430121475     0.6031049274447527    0.6080947912820311
0.4965891771673148      0.6022182392286206    0.6062303299471251
0.4967505317302706      0.6013905183708885    0.6045497318393004
0.49692537215744337     0.6004935017501332    0.6033198966813759
0.4970970290039256      0.5996127074920523    0.6029007486202549
0.49725716754726357     0.5987909214505001    0.6029888365920159
0.4974307919548185      0.5978998333603873    0.6030644202185622
0.4975928980592291      0.5970677752408792    0.6026244173375244
0.4977518205829492      0.5962519830530654    0.6015096941382817
0.49792422897088623     0.5953668862265936    0.5997214261987006
0.498085119055679       0.5945408531726747    0.5979727699659162
0.4982594950046887      0.5936455135331007    0.5965565039709818
0.4984306873730079      0.5927664570811487    0.5959469536212852
0.4985903614381828      0.5919464950699554    0.5959513144799841
0.4987635213675747      0.5910572303631964    0.5960923382030088
0.4989251629938223      0.5902270766451397    0.5958300732593206
0.49908362103937937     0.5894132392712911    0.5949110977108664
0.4992555649491534      0.5885301065160915    0.5932451732641353
0.4994159905557832      0.5877061086604135    0.5914636383898301
0.49958990202662984     0.5868128216229718    0.5898689822985742
0.49975229519433223     0.5859786830457423    0.5890632313049949
0.4999115047813441      0.5851608883475786    0.5889130514214409
0.5000842002325729      0.5842738181893717    0.5890751738001228
0.5002453773806574      0.5834459146045428    0.5890027490341657
0.500420040392959       0.5825487470161425    0.5882264114887876
0.5005915198245698      0.5816679457754672    0.5867014880474484
0.5007514809530365      0.580846325694549     0.5849284114289958
0.5009249279457201      0.5799554622008255    0.5832013953312413
0.5010868566352594      0.579123788226532     0.5822070456906461
0.5012456017441083      0.5783084973801123    0.5819085228244585
0.501417832717174       0.5774240166044382    0.5820401221685091
0.5015785453870955      0.5765987556731369    0.5820810241728839
0.5017527439212339      0.5757042985258657    0.581517046312409
0.5019237588746818      0.5748262482876554    0.5801724958166407
0.5020832555249853      0.574007395857865     0.5784489133548912
0.5022562380395059      0.5731193770261374    0.5766144598993035
0.5024177022508822      0.5722905588693934    0.5754232291985454
0.5025926523264754      0.5713925985908201    0.5749268992221859
0.5027644188213781      0.5705110680372731    0.575013792702736
0.5029246670131365      0.5696887361855338    0.5751147504088867
0.503098401069112       0.5687972985187602    0.5746948343251762
0.5032606168219431      0.5679650594428534    0.5735684563898938
0.5034196489940836      0.5671492499284858    0.5719238666065372
0.5035921670304412      0.5662643734086018    0.570029640736004
0.5037531667636544      0.5654386878141036    0.5686761640616864
0.5039276523610846      0.5645439672858186    0.5679868563112026
0.5040989543778243      0.5636657058599871    0.5679794824547645
0.5042587380914196      0.5628466237174713    0.5681317322983004
0.504432007669232       0.5619585525999654    0.5678930203169525
0.5045937589439001      0.5611296557154781    0.5669638898958503
0.5047523266378777      0.5603172078910816    0.5654386094104917
0.5049243801960722      0.5594358192929307    0.563521028850657
0.5050849154511224      0.5586135878800703    0.562013914616625
0.5052589365703897      0.5577224554912269    0.5611071847059247
0.5054214393865126      0.5568904722751278    0.560955627547479
0.505580758621945       0.5560749488095362    0.5611182545877239
0.5057535637215943      0.5551905789085284    0.5610719922383621
0.5059148505180994      0.554365335383799     0.5603980418779078
0.5060896231788213      0.5534712904208905    0.5588945122457473
0.5062612122588528      0.5525937397758452    0.556994564600693
0.50642128303574        0.5517752883569575    0.5553760470055701
0.5065948396768442      0.5508880975627876    0.5542757386335057
0.5067568780148041      0.5500599930477793    0.5539769256138484
0.5069157327720734      0.5492483570149748    0.5541067169786067
0.5070880733935597      0.5483680455281077    0.5541729527098498
0.5072488957119017      0.5475467881515714    0.5536901477521693
0.5074232038944606      0.5466569080354532    0.5523741160561719
0.507594328496329       0.5457835350635958    0.5505333500035533
0.5077539347950532      0.5449692761700412    0.5488259278498051
0.5079270269579943      0.5440864739905535    0.5475246634099937
0.5080886008177912      0.5432626634703339    0.5470449032548063
0.5082469910968974      0.542455320073136     0.5470999509491056
0.5084188672402207      0.5415795050990868    0.5472476409818982
0.5085792250803997      0.5407626393683763    0.5469568464162144
0.5087530687847955      0.5398773611690012    0.5458626785728866
0.5089153941860471      0.5390510107801579    0.5442232364177456
0.5090745360066082      0.5382411267380934    0.5424561079314754
0.5092471636913863      0.5373629078720852    0.5409270329069172
0.5094082730730202      0.5365435690097117    0.5401992367706125
0.5095828683188709      0.5356559592990842    0.5401211314616737
0.5097542799840311      0.534784856282442     0.5403036486735915
0.509914173346047       0.5339725808232234    0.5401656096415846
0.51008755257228        0.5330921196769416    0.5392804034253813
0.5102494134953687      0.5322704601723856    0.5377716753287697
0.5104080908377667      0.5314652555839333    0.5360041379937038
0.5105802540443818      0.5305919514717147    0.5343313810286683
0.5107408989478526      0.5297773926812265    0.5334091132901759
0.5109150297155403      0.5288948053515372    0.533174949017762
0.5110859769025374      0.5280287158299812    0.5333523366749925
0.5112454057863903      0.5272213104810388    0.533351480633615
0.5114183205344602      0.5263459704376051    0.5326924603773668
0.5115797169793859      0.5255292842061281    0.5313548443436245
0.5117545992885284      0.5246447392467817    0.5294427252007503
0.5119262980169804      0.5237766879459439    0.5276909647805004
0.5120864784422882      0.5229672236238955    0.5266361728460517
0.5122601447318128      0.5220900006737408    0.5262763442907515
0.5124222927181932      0.5212713314602415    0.5264134848530088
0.512581257123883       0.5204690894745895    0.5265061042132043
0.5127537073937899      0.5195991896083062    0.5260424934722927
0.5129146393605525      0.5187877731287722    0.5248796902164239
0.513089057191532       0.5179087817358916    0.5230435483936237
0.513260291441821       0.5170462651215761    0.5212100998892676
0.5134200073889656      0.5162421563668422    0.5199805198513038
0.5135932092003272      0.5153705814720051    0.5194304527863405
0.5137548927085446      0.5145573767543091    0.5194911165633566
0.5139133926360714      0.5137606193263152    0.5196493889743948
0.5140853784278152      0.5128966075232022    0.5193806254830485
0.5142458459164148      0.5120908762482215    0.5184211907497485
0.5144197992692312      0.5112178881254372    0.5167077026802697
0.5145822343189034      0.5104031388967251    0.514915164114297
0.514741485787885       0.5096047688917874    0.5134879182599797
0.5149142231210837      0.5087392549577181    0.5126791226342953
0.5150754421511381      0.5079318943718596    0.5125925241162236
0.5152501470454094      0.5070574828032808    0.5127912521123877
0.5154216683589902      0.5061994998826261    0.5126756701595075
0.5155816713694267      0.5053995796544065    0.5119045917347819
0.5157551602440801      0.5045327289445837    0.5103340129917937
0.5159171308155892      0.5037238954367415    0.5085453444118799
0.5160759178064078      0.5029313996426771    0.5069969437787083
0.5162481906614435      0.5020720936080802    0.5059897916368138
0.5164089452133348      0.5012707119085307    0.505759369175646
0.516583185629443       0.5004026189407751    0.5059399386800183
0.5167542424648608      0.49955091529692985   0.5059613759689047
0.5169137809971343      0.49875703784542025   0.5053925985660337
0.5170868053936246      0.4978965770933571    0.5040046996006602
0.5172483114869707      0.49709389321992753   0.5022605931763453
0.5174233034445338      0.4962247295410511    0.5004634126101
0.5175951118214064      0.4953719342614224    0.4993247830991932
0.5177554018951347      0.49457681254841485   0.49898258806887474
0.5179291778330799      0.49371534467179673   0.49912651538696734
0.5180914354678808      0.4929114984256642    0.4992293581547434
0.5182505095219913      0.49212391675335154   0.4988334907333236
0.5184230694403186      0.49127012198953246   0.49762864244057137
0.5185841110555016      0.4904738436450325    0.4959604564945136
0.5187586385349017      0.48961146172287084   0.4940962852826702
0.5189299824336112      0.48876540008260133   0.49278176360121895
0.5190898080291765      0.48797674407048475   0.49226972932114854
0.5192631194889586      0.4871221255803056    0.4923301158104041
0.5194249126455965      0.4863248569629619    0.49249822812955973
0.5195835222215439      0.48554379716358864   0.4922744704125275
0.5197556176617083      0.4846969152072187    0.49128074243496866
0.5199161947987284      0.4839072706633126    0.48972854436289787
0.5200902577999654      0.4830519194423863    0.4878322769996716
0.5202611372205119      0.4822129830126501    0.48634758988444443
0.5204204983379142      0.48143119096082043   0.48564153476368105
0.5205933453195333      0.48058385395327174   0.4855733736886843
0.5207546739980081      0.47979356209627694   0.48576977907827396
0.5209294885407         0.47893784339573486   0.4856675282495408
0.5211011195027013      0.4780983553152014    0.4848278199788416
0.5212612321615584      0.47731578733262664   0.48337966729212184
0.5214348306846324      0.47646794343720983   0.481491436799956
0.5215969109045621      0.4756769564912649    0.47996982702122554
0.5217558075438012      0.474902073906689     0.4790894727346216
0.5219281900472573      0.47406206484206037   0.4788779483193842
0.5220890542475691      0.4732787871901801    0.4790618829886743
0.522263404312098       0.4724305059971942    0.47909278504765446
0.5224345707959362      0.4715983874565581    0.4784616835771672
0.5225942189766303      0.47082286927374634   0.47717447472041913
0.5227673530215412      0.4699825044622282    0.4753291715536683
0.5229289687633079      0.469198673795696     0.4737039111574674
0.523087400924384       0.46843087339406403   0.4726402116375549
0.523259318949677       0.46759838160733813   0.47224910651171914
0.5234197186718258      0.4668222926147232    0.47237970405306146
0.5235936042581916      0.4659816399596858    0.47251642476280664
0.5237559715414131      0.4651973216587478    0.4721386025132123
0.523915155243944       0.46442899329176407   0.4710806081390705
0.5240878248106919      0.46359626122975334   0.4693443933494732
0.5242489760742955      0.4628197278500227    0.4676364174931346
0.5244236132021162      0.4619789223681987    0.4662619097066815
0.5245950667492463      0.46115416847543844   0.46569645050559244
0.5247550019932321      0.46038547182740375   0.4657467613306689
0.5249284231014348      0.45955267076917955   0.46594135728005065
0.5250903259064933      0.4587758554545388    0.46573017340391193
0.5252490451308612      0.45801494876497095   0.46486128659289244
0.5254212502194461      0.45719010343616845   0.4632514397130008
0.5255819370048866      0.45642110237146577   0.4615256598087269
0.5257561096545442      0.45558829903840975   0.4599959141237796
0.5259270987235113      0.45477146771583365   0.45923277709284666
0.5260865694893341      0.45401033327631557   0.4591645515565963
0.5262595261193738      0.4531855701819388    0.45937996216864974
0.5264209644462692      0.45241646556970716   0.4593242056094444
0.5265958886373816      0.4515840539907111    0.4585525530925313
0.5267676292478034      0.4507675662069374    0.45705192366331965
0.5269278515550809      0.4500065327336852    0.4553406093622587
0.5271015597265755      0.449182203032927     0.4537236129606756
0.5272637495949257      0.44841324936344157   0.45284905823028465
0.5274227558825855      0.44766006295938354   0.4526523384948736
0.5275952480344621      0.4468437531759337    0.4528493659743953
0.5277562218831945      0.44608266620097936   0.45290592636016475
0.5279306815961439      0.4452585980008938    0.452335182100556
0.5281019577284026      0.4444503598607718    0.4510021580924149
0.5282617155575171      0.4436971857458171    0.4493360526367484
0.5284349592508486      0.44288121022532134   0.4476210098039624
0.5285966846410358      0.4421202182408286    0.4465700248157858
0.5287552264505325      0.4413748953665933    0.4462141725306238
0.5289272541242461      0.4405669481174317    0.4463525851874056
0.5290877634948155      0.43981382699181826   0.4464976391069656
0.5292617587296018      0.4389982270905233    0.44613209912525176
0.5294242356612439      0.4382373710156615    0.44507154672185884
0.5295835290121954      0.43749213060188086   0.44350847872564614
0.5297563082273639      0.43668459225953643   0.44171626680696335
0.529917569139388       0.43593163682406866   0.4404580591752888
0.5300923159156291      0.43511653218436425   0.43985688056327016
0.5302638791111797      0.43431710855372463   0.43991182258333306
0.530423924003586       0.4335721011645103    0.440104168044997
0.5305974547602093      0.432765132573023     0.4399050204203306
0.5307594672136883      0.4320124957183422    0.4390277205218303
0.5309182960864768      0.43127537089216117   0.43757931864121935
0.5310906108234822      0.43047646988788707   0.4357741336007936
0.5312514072573433      0.4297317355160427    0.43438403044119944
0.5314256895554215      0.4289253771310909    0.4335925137353596
0.531596788272809       0.42813459736587184   0.433526447367707
0.5317563686870522      0.42739781328111226   0.43373464910486564
0.5319294349655125      0.4265995974506483    0.4336908525981635
0.5320909829408285      0.4258552906088451    0.43301452927484424
0.5322493473354539      0.42512638903176897   0.43171667588170093
0.5324211975942963      0.4243362448708021    0.42993786344334706
0.5325815295499944      0.4235998404483499    0.4284330687941331
0.5327553473699095      0.4228024772896474    0.42743736702410373
0.5329176468866803      0.42205885155113215   0.42720808145104183
0.5330767628227605      0.4213305673928702    0.42738287811337133
0.5332493646230577      0.4205414045109964    0.42748930890499487
0.5334104481202107      0.4198057018811922    0.42705771773326184
0.5335850174815805      0.41900927677172545   0.425830785115197
0.5337564032622598      0.41822825838745825   0.42411325120303944
0.5339162707397949      0.41750051964330365   0.42253737508053246
0.5340896240815469      0.416712254701896     0.42136981339631563
0.5342514591201547      0.4159771776313613    0.4209838963177013
0.534410110578072       0.4152573237659335    0.42109512255106285
0.5345822479002061      0.41447713619916077   0.4212768924690669
0.534742866919196       0.4137499588362004    0.4210193474280234
0.5349169718024028      0.41296260604985885   0.41999407950422607
0.5350878931049191      0.41219054234503555   0.41837766735886356
0.5352472961042911      0.4114713056896193    0.41675856146491236
0.5354201849678801      0.4106920924164578    0.4154220702683018
0.5355815555283249      0.4099656131274736    0.41485440397747214
0.5357564119529865      0.40917931781505057   0.4148834978246071
0.5359280847969576      0.40840824733936576   0.4150957699465196
0.5360882393377845      0.4076897250169632    0.41494891534417794
0.5362618797428284      0.4069115883697334    0.4140726516158589
0.5364240018447279      0.4061859054468612    0.4126362577583647
0.5365829403659369      0.4054752585212968    0.41101280841767945
0.5367553647513629      0.4047051950612484    0.4095491395407292
0.5369162708336446      0.4039874025622367    0.4088162742935661
0.5370906627801432      0.4032103561542404    0.4087215680453051
0.5372618711459513      0.40244841330863934   0.408945232683345
0.5374215612086151      0.40173855305379136   0.4089304659935365
0.5375947371354959      0.4009696431083475    0.4082569234808724
0.5377563947592324      0.40025272003034057   0.4069661989172886
0.5379148688022783      0.3995507090864644    0.405371548694943
0.5380868287095413      0.398789848680014     0.40379767102929365
0.5382472703136599      0.3980807899287051    0.40288847462480964
0.5384211977819955      0.39731304635082954   0.40263416546257963
0.5385836069471868      0.3965970073458962    0.4028180285081605
0.5387428325316875      0.3958958130065158    0.402931323445868
0.5389155439804053      0.39513613691317034   0.4025047412864031
0.5390767371259788      0.3944281782689226    0.4014248212038972
0.5392514161357692      0.3936619343849714    0.3997430731784276
0.5394229115648691      0.39291059826604613   0.39810222338856643
0.5395828886908247      0.3922105675957165    0.3970463862046056
0.5397563516809972      0.3914524452780973    0.3966339974189041
0.5399182963680255      0.39074552858539124   0.39675864110525316
0.5400770574743632      0.390053320018488     0.3969351546953745
0.5402493044449179      0.38930322140176904   0.3966819942864568
0.5404100331123284      0.3886041366545717    0.3957797354695467
0.5405842476439557      0.38784732758389784   0.39420353051643503
0.5407552785948926      0.3871052923438718    0.3925224237119694
0.5409147912426852      0.38641407405643996   0.39132232591823357
0.5410877897546947      0.38566533845663575   0.3907274632632025
0.54124926996356        0.38496731957623204   0.3907565514538624
0.5414242360366422      0.3842119505674731    0.3909796802005956
0.5415960185290339      0.38347128207652037   0.39083498227383257
0.5417562827182814      0.3827811316859414    0.39006131064758115
0.5419300327717457      0.3820338399906288    0.38858083910931884
0.5420922645220657      0.38133696529168276   0.38697809614565726
0.5422513126916952      0.3806545889106843    0.38567026966563794
0.5424238467255418      0.37991527537109926   0.38491124407701266
0.542584862456244       0.3792261846171636    0.38483171014683026
0.5427593640511632      0.37848032452211927   0.38505915543072017
0.5429306820653919      0.3777490292610476    0.38504640730325096
0.5430904817764762      0.3770677573604043    0.38444914030279537
0.5432637673517775      0.3763299257308222    0.3831164740643664
0.5434255346239345      0.375642015948858     0.3815383867465085
0.543584118315401       0.3749684679833787    0.38013641407307175
0.5437561878710845      0.37423856518730136   0.3792018156611304
0.5439167391236237      0.3735583892619669    0.37898349459870495
0.5440907762403798      0.37282202694781397   0.37917882045060125
0.5442532950539917      0.3721352891229105    0.37928448398575454
0.544412630286913       0.371462838748625     0.3789034352261639
0.5445854513840512      0.3707344086749496    0.37778399515029015
0.5447467541780452      0.37005540642979257   0.37627965353686776
0.5449215428362562      0.36932059438935616   0.3746563918250347
0.5450931479137766      0.36860013712506473   0.3735764727167356
0.5452532346881528      0.3679289821981571    0.37322230329764544
0.5454268073267459      0.3672023682985936    0.37335844893811715
0.5455888616621947      0.3665248620461068    0.37352906810347974
0.545747732416953       0.3658614991624216    0.37329875650728106
0.5459200890359283      0.3651427595430639    0.3723568176246251
0.5460809273517593      0.36447292808247367   0.3709450528988649
0.5462552515318072      0.3637478879500037    0.3692886184717455
0.5464263921311646      0.3630370549456125    0.3680644257656667
0.5465860144273778      0.3623749258898811    0.3675533631329383
0.5467591225878078      0.3616577970838753    0.3675951914811537
0.5469207124450935      0.360989268131008     0.36780408858830843
0.5470791187216888      0.3603347381137081    0.3677181258932149
0.547251010862501       0.3596254120082967    0.36697330468468314
0.547411384700169       0.35896448706762385   0.3656881100101192
0.5475852444020539      0.35824893343708497   0.3640323907647968
0.5477475858007945      0.3575816764796072    0.3627278847302939
0.5479067436188446      0.35692833975935734   0.36201062373085796
0.5480793873011116      0.3562205787888904    0.36189292899141834
0.5482405126802343      0.3555609150363135    0.36210095941902665
0.548415123923574       0.35484699506251344   0.36214338903273424
0.5485865515862233      0.3541470590696751    0.3615735988664146
0.5487464609457282      0.3534950161272703    0.3604228779454637
0.54891985616945        0.3527889258373521    0.35880346913599087
0.5490817330900276      0.35213062452154126   0.35741448743052945
0.5492404264299147      0.35148609903256794   0.3565484165156026
0.5494126056340186      0.35078772982003953   0.35628772201763903
0.5495732665349783      0.35013695092610736   0.35645835867475983
0.549747413300155       0.34943249567228857   0.35659517845000077
0.5499183764846411      0.34874187973854887   0.3562070238263079
0.550077821365983       0.3480986507608286    0.35521728699066246
0.5502507521115418      0.34740195347670483   0.3536700079494319
0.5504121645539564      0.3467525394865823    0.35221836443233107
0.5505870628605879      0.34604982518132393   0.35112910027459804
0.5507587775865289      0.3453608708729631    0.3507644500248815
0.5509189740093257      0.34471899573731557   0.3508920816445918
0.5510926562963393      0.3440240291087168    0.35107462674056306
0.5512548202802087      0.34337603759750096   0.3508342228588207
0.5514138006833875      0.34274159797104736   0.3499979590662703
0.5515862669507833      0.34205440862350367   0.3485446258344546
0.5517472149150348      0.3414140307498754    0.3470678833076182
0.5519216487435032      0.3407209453850364    0.3458464219245371
0.5520928989912812      0.34004146972461574   0.3453263656240503
0.5522526309359149      0.3394085523476193    0.3453720866463111
0.5524258487447655      0.3387231325779744    0.34559099243376584
0.5525875482504718      0.33808416634744914   0.3454899769415173
0.5527460641754877      0.3374586003229131    0.34482394192566235
0.5529180659647204      0.3367807313582649    0.3434972677882114
0.5530785494508088      0.33614911706380396   0.3420233368304399
0.5532525188011143      0.3354653637404291    0.34067990274135224
0.5534149698482754      0.3348277603656269    0.34000534180986747
0.5535742373147461      0.33420347593289423   0.33991319121256613
0.5537469906454336      0.33352725180404486   0.3401310080434567
0.5539082256729769      0.33289697880092955   0.3401753336625265
0.5540829465647372      0.33221492990765994   0.33963895415028045
0.5542544838758069      0.33154625959564615   0.33844527877726394
0.5544145028837323      0.3309233374381913    0.3370034748840872
0.5545880077558747      0.33024884213667827   0.3355775285701415
0.5547499943248728      0.3296199913946116    0.33476198557507947
0.5549087973131804      0.32900431203373853   0.3345481333462915
0.555081086165705       0.32833725669365516   0.3347298713340708
0.5552418567150852      0.32771564924071167   0.3348554707246727
0.5554161131286824      0.327042827836793     0.33448972046252423
0.5555871859615892      0.32638323679856557   0.33345505290253363
0.5557467404913516      0.32576889286172567   0.3320748516627678
0.555919780885331       0.3251035355217376    0.3305864455611637
0.5560813029761661      0.32448332280755404   0.3296267224971971
0.5562563109312181      0.3238122585921648    0.32926124860677736
0.5564281353055796      0.32315434335748816   0.3294024348603926
0.5565884413767967      0.3225413720628326    0.32956723504538404
0.556762233312231       0.3218777496363427    0.3293112776896688
0.5569245069445209      0.32125896871599074   0.3284549504242112
0.5570835969961203      0.3206531318987314    0.3271540009600756
0.5572561729119366      0.3199968388029064    0.3256394921757161
0.5574172305246087      0.31938519276067195   0.3245666990296192
0.5575917740014977      0.31872325056559614   0.324054978932161
0.5577631338976962      0.31807434324428774   0.3241180698673091
0.5579229754907503      0.3174699873889839    0.32431639854935435
0.5580963029480215      0.31681554422249947   0.32419949697676426
0.5582581121021484      0.316205437290489     0.323503251018168
0.5584167376755848      0.3156081256523785    0.32230878818004927
0.558588849113238       0.31496091727720155   0.320794012918108
0.558749442247747       0.31435785152398743   0.31961545417072146
0.558923521246473       0.3137050456262255    0.3189418179023172
0.5590944166645084      0.3130650900052031    0.31889579158617437
0.5592537937793995      0.31246907977300553   0.31910008425324965
0.5594266567585077      0.3118235225077944    0.31911044099759933
0.5595880014344715      0.31122180987014203   0.3185867674007893
0.5597628319746523      0.3105707060719954    0.31739221257353195
0.5599344789341425      0.30993237015774766   0.3159013903153004
0.5600946075904885      0.3093376824144108    0.3146686127284972
0.5602682221110514      0.30869379569784083   0.3138884472403969
0.5604303183284701      0.3080934567753912    0.31375024357661446
0.5605892309651983      0.30750568496683894   0.3139332813076968
0.5607616294661434      0.30686890067177197   0.3140326981449298
0.5609225096639442      0.3062754744863212    0.31365777927208577
0.5610968757259619      0.3056331890452124    0.3126100082702662
0.5612680582072891      0.3050035243917821    0.3111690942692445
0.561427722385472       0.30441702464947834   0.3098741749914138
0.5616008724278719      0.30378185457017287   0.3089482137651885
0.5617625041671276      0.3031897506926436    0.3086797034760774
0.5619209523256927      0.3026100701677174    0.3088129936689933
0.5620928863484748      0.30198190259709534   0.3089807273002926
0.5622533020681125      0.3013966147212979    0.30875551101159815
0.5624272036519673      0.30076299040743976   0.30787790785773234
0.5625895869326778      0.30017214745924153   0.30659153715421483
0.5627487866326977      0.2995936495150129    0.3052499442273181
0.5629214721969346      0.29896699754290146   0.3041529944149422
0.5630826394580272      0.2983829411692589    0.30370344538006505
0.5632572925833367      0.297750880645999     0.3037564392826792
0.5634287621279557      0.29713121749480975   0.30395805426953887
0.5635887133694304      0.29655396076694435   0.3038517005312856
0.5637621504751221      0.2959288846266753    0.3031317249767442
0.5639240692776696      0.29534611824662604   0.3019421084234006
0.5640828044995264      0.2947756071777344    0.3005995660814022
0.5642550255856003      0.2941575009860971    0.299397907896959
0.5644157283685298      0.29358151492236645   0.2988101987684149
0.5645899170156763      0.2929580423446066    0.2987584294049258
0.5647609220821322      0.2923468219428151    0.2989682092311845
0.5649204088454439      0.2917775348082868    0.29897184346603584
0.5650933814729726      0.2911609410771951    0.2984222890356475
0.565254835797357       0.29058618507137157   0.29735639762066113
0.5654297759859583      0.28996426767335776   0.2959012362447822
0.5656015325938691      0.28935452247776244   0.29464656670934875
0.5657617708986357      0.2887864294116766    0.2939698059071561
0.5659354950676191      0.2881713534109159    0.2938367138688035
0.5660977009334582      0.28759783462997096   0.29402380055274363
0.5662567232186069      0.2870362982112746    0.294105117280979
0.5664292313679725      0.2864279517022499    0.2937021897436761
0.5665902212141939      0.28586098373125346   0.29276116518513196
0.5667646969246322      0.28524734767405113   0.29135480986878637
0.5669359890543799      0.28464574206677307   0.29003629457620267
0.5670957628809834      0.2840853333541755    0.28923474784489744
0.5672690225718039      0.2834784310521276    0.2889744789514926
0.5674307639594801      0.2829126327570127    0.2891213268037144
0.5675893217664657      0.2823586791156028    0.289264271883419
0.5677613654376683      0.2817584008609664    0.28901031722993
0.5679218908057266      0.28119905178186744   0.2882141977382985
0.5680959020380019      0.28059351691675355   0.28688629955358197
0.5682583949671328      0.28002881896630155   0.2855847898578934
0.5684177043155733      0.27947589061318523   0.2846340413512875
0.5685904995282307      0.27887694404493346   0.28419515628723385
0.5687517764377439      0.27831866077834577   0.2842559726908355
0.568926539211474       0.2777144969648864    0.28445191180089474
0.5690981184045135      0.2771221491484945    0.28431684326024753
0.5692581792944088      0.27657028819340795   0.28365456112536563
0.569431726048521       0.2759727157819681    0.28241913120247436
0.5695937544994889      0.27541553999115803   0.2811094716863969
0.5697525993697663      0.2748699996953038    0.2800670429281894
0.5699249301042607      0.27427891164979024   0.2794925575835548
0.5700857425356108      0.2737280504587837    0.2794665389354195
0.5702600408311779      0.27313177650651965   0.27967060454715253
0.5704311555460544      0.2725472228685072    0.2796477528726146
0.5705907519577866      0.2720027207075411    0.2791311928677781
0.5707638342337358      0.27141297325078506   0.27801441914535524
0.5709253982065406      0.270863188744622     0.2767203550460008
0.571083778598655       0.2703249071501288    0.2755964904710947
0.5712556448549864      0.26974153951567664   0.2748763434722397
0.5714159928081735      0.26919796855754424   0.2747386241438955
0.5715898266255774      0.26860944239215123   0.27492138521812926
0.5717521421398372      0.26806062511184414   0.2750013488309589
0.5719112740734064      0.2675232385374096    0.2746641909454149
0.5720838918711926      0.2669410543771296    0.27371913021646754
0.5722449913658345      0.26639841425622585   0.2724789577414649
0.5724195767246933      0.26581110603558555   0.27117030755739807
0.5725909785028617      0.26523527127555024   0.27032895227270615
0.5727508619778857      0.26469881320405025   0.2700811039662859
0.5729242313171267      0.26411784576341457   0.27021936381332795
0.5730860823532233      0.2635761693783742    0.2703566067427261
0.5732447498086295      0.2630457951655061    0.27014489513014056
0.5734169031282527      0.26247106507929113   0.2693436571690222
0.5735775381447316      0.2619354653116453    0.2681748781013694
0.5737516590254274      0.2613556357676446    0.2668335098458152
0.5739225963254326      0.2607871499117863    0.2658708329720753
0.5740820153222936      0.2602576312123318    0.26549521045107677
0.5742549201833715      0.2596840372719835    0.26556047107593744
0.5744163067413052      0.2591493269974069    0.2657339008957704
0.5745911791634558      0.2585706661840612    0.2656075310448204
0.5747628680049159      0.25800327761332303   0.26490879134880985
0.5749230385432318      0.25747461109827074   0.26380460433629593
0.5750966949457645      0.2569021468108822    0.2624621130842495
0.575258833045153       0.2563683218673502    0.2614685800587481
0.575417787563851       0.25584560372017534   0.2609779725202345
0.5755902279467658      0.25527923548163983   0.2609595597785768
0.5757511500265364      0.2547513514402173    0.2611412539757538
0.575925557970524       0.2541799386239033    0.261118748902721
0.5760967823338211      0.2536196722594916    0.26056313115885166
0.5762564883939739      0.25309773266702973   0.25955855009774803
0.5764296803183436      0.25253241287520006   0.25822897911835074
0.5765913539395691      0.2520053318384503    0.25715387608856694
0.576749843980104       0.2514892259112816    0.2565387233801349
0.5769218198848559      0.2509298819871921    0.2564123127764129
0.5770822774864636      0.2504086346427052    0.2565783413370316
0.5772562209522881      0.24984426744836308   0.2566468682939131
0.5774186461149684      0.2493179171635883    0.25627589156559694
0.5775778876969582      0.24880248665940025   0.25541659148149076
0.5777506151431648      0.2482440785069616    0.25413590907209804
0.5779118242862272      0.24772353782785778   0.2529816342742431
0.5780865192935066      0.24716013631488473   0.25216037090274307
0.5782580307200955      0.24660769274068597   0.25192765999974487
0.5784180238435401      0.2460929650180513    0.2520587755071887
0.5785915028312016      0.2455355195136867    0.2521862508664171
0.5787534635157189      0.2450157122556608    0.25193559121416836
0.5789122406195456      0.24450670770425362   0.25119741378743987
0.5790845035875892      0.24395512362684463   0.24998295287018701
0.5792452482524886      0.2434410322898498    0.24879769613384176
0.5794194787816049      0.24288447499955956   0.24786239375864658
0.5795905257300308      0.24233875738956967   0.2475052673278752
0.5797500543753124      0.2418303849055158    0.247577904171044
0.5799230688848108      0.24127968550752935   0.2477453213939587
0.580084565091165       0.2407662556796784    0.2476132813694622
0.5802595471617362      0.24021061113145598   0.24692023157291157
0.5804313456516168      0.2396657411405693    0.24576544285741603
0.5805916258383531      0.23915799464291637   0.2445769167340446
0.5807653918893064      0.2386081707165473    0.24357722432807719
0.5809276396371155      0.23809539549977182   0.24314405457620214
0.581086703804234       0.23759324525130934   0.2431491679493349
0.5812592538355694      0.23704915021311415   0.24332947485028836
0.5814202855637606      0.2365419637786427    0.243288593094027
0.5815948031561687      0.23599294150965586   0.24272927463065036
0.5817661371678864      0.23545457938578435   0.2416679227185686
0.5819259528764598      0.23495298379241544   0.24048549994550594
0.58209925444925        0.23440968564748202   0.23940172200580465
0.582261037718896       0.23390308129694887   0.2388509529350955
0.5824196374078515      0.23340699159764486   0.23876781301328961
0.5825917229610239      0.23286932807396807   0.2389389145398445
0.582752290211052       0.23236822213577313   0.23898011340249944
0.5829263433252971      0.2318255866669833    0.23856323825060857
0.5830972128588517      0.23129348394715865   0.237618089590775
0.5832565640892621      0.2307978064800695    0.236464424761823
0.5834294011838893      0.23026078511278877   0.23530898741408365
0.5835907199753723      0.22976011882714978   0.23463396774224393
0.5837655246310722      0.2292182138485596    0.23444383177708955
0.5839371457060816      0.22868679948146964   0.23459425768958111
0.5840972484779466      0.2281916042623956    0.23467883110319365
0.5842708371140287      0.22765529932647274   0.2343559209183164
0.5844329074469665      0.22715514398103048   0.2335533264289745
0.5845917941992138      0.22666534011382108   0.23244623140455575
0.584764166815678       0.2261345512625444    0.23124935202964825
0.5849250211289979      0.2256397814260344    0.23047261838675015
0.5850993613065348      0.2251041290733553    0.23016852987923736
0.5852705179033811      0.22457886195953616   0.23027272664427964
0.5854301561970832      0.22408948131237186   0.23040239794191808
0.5856032803550022      0.2235593437343559    0.23020221214325662
0.585764886209777       0.22306502479702628   0.22952385996766111
0.5859233084838612      0.22258095542847356   0.22848556699849887
0.5860952166221624      0.2220562505521743    0.22726549848710853
0.5862556064573192      0.221567237037038     0.22638902900069818
0.5864294821566931      0.22103768776314647   0.22595475877640134
0.5865918395529227      0.22054376263325406   0.22597899083073925
0.5867510133684618      0.2200600315253995    0.22613611803473255
0.5869236730482178      0.21953588473223504   0.22607369332868613
0.5870848144248295      0.21904723598637624   0.22555835366751284
0.5872594416656581      0.218518270192581     0.22451639506210513
0.5874308853257962      0.21799953069050557   0.22329842775998776
0.5875908106827901      0.21751616128414325   0.22234829749130047
0.5877642219040009      0.21699259563957765   0.22179826777711273
0.5879261148220675      0.21650433460359916   0.22174404511154552
0.5880848241594435      0.21602616886691156   0.22189722740528525
0.5882570193610365      0.2155079236592372    0.22191873167175524
0.5884176962594851      0.21502486025656772   0.22152567043512897
0.5885918590221508      0.21450181330411988   0.2205924623065647
0.5887628382041259      0.21398889292811937   0.2193977202918432
0.5889222990829567      0.2135110297631143    0.21838295276234762
0.5890952458260045      0.21299327148142747   0.21770929718832274
0.589256674265908       0.2125104112376867    0.21755614721599412
0.5894315885700284      0.2119877699742562    0.2177000771782905
0.5896033192934583      0.21147520412727985   0.21776632349807537
0.589763531713744       0.21099751885166404   0.21745310981429172
0.5899372299982466      0.2104801705689317    0.2166027519371315
0.590099409979605       0.20999764056185158   0.215499868112009
0.5902584063802727      0.20952506136758645   0.2144459478652831
0.5904308886451575      0.20901293355497788   0.21367045309818644
0.590591852606898       0.20853550655584827   0.21342142946573847
0.5907663024328553      0.20801862492165452   0.21352328838647539
0.5909375686781222      0.20751172638095303   0.21363914429570569
0.5910973166202448      0.2070394091506785    0.21343291551125326
0.5912705504265844      0.20652775293283415   0.2127034638466315
0.5914322659297797      0.20605061693677282   0.21165795067588736
0.5915907978522845      0.2055833418447268    0.2105788722717493
0.5917628156390062      0.20507683984328853   0.20970161091942988
0.5919233151225837      0.2046047429006465    0.20934203656696954
0.592097300470378       0.20409351113448979   0.20937806198688869
0.5922597675150282      0.2036166236778881    0.20952184411241878
0.5924190509789877      0.20314954812325364   0.20943882628426305
0.5925918203071643      0.2026434490550577    0.20887066822017625
0.5927530713321966      0.2021715797933103    0.2079213281803198
0.5929278082214458      0.20166077846965297   0.20672608438425355
0.5930993615300044      0.20115982034294835   0.2057722674104442
0.5932593965354188      0.20069297555763468   0.20531406046152553
0.5934329174050502      0.20018731120099822   0.20527652375169456
0.5935949199715373      0.19971570064254912   0.20542519067594533
0.5937537389573339      0.19925381418243576   0.2054185114817302
0.5939260438073474      0.19875321715267322   0.20497160101255407
0.5940868303542166      0.19828656171510922   0.20411229015640622
0.5942611027653028      0.1977812852639881    0.202935795895564
0.5944321915956984      0.19728576351236773   0.20191221148297697
0.5945917621229497      0.19682406922831425   0.20134716970716457
0.5947648185144181      0.1963238640753761    0.20121554790585647
0.5949263566027422      0.1958574280441837    0.20134918811312877
0.5951013805552832      0.19535257003983517   0.20139891808680857
0.5952732209271336      0.19485741805529655   0.20103067798941773
0.5954335429958398      0.19439580821037783   0.20023953249834014
0.5956073509287629      0.19389581921470278   0.1990913106412177
0.5957696405585418      0.19342944110255714   0.19807949414476844
0.5959287466076301      0.19297265811087275   0.19742566215223709
0.5961013385209354      0.19247765739526485   0.1972022825250797
0.5962624121310964      0.19201616003181615   0.1973052396303163
0.5964369716054744      0.1915165340558549    0.19740506753489012
0.5966083474991618      0.19102653600347658   0.1971454009099055
0.596768205089705       0.190569931449433     0.19645567654772206
0.596941548544465       0.19007530870468203   0.19535791674550707
0.5971033736960809      0.1896140234084015    0.1943127988218986
0.5972620152670062      0.18916225393182268   0.19356829318919488
0.5974341427021485      0.18867257371422108   0.19323750838610262
0.5975947518341465      0.1882161243877743    0.19328991277821733
0.5977688468303614      0.18772185262591223   0.19342439750446475
0.5979397582458857      0.18723712918935245   0.19327323600130838
0.5980991513582659      0.18678552777890967   0.1926990114569787
0.598272030334863       0.186296213358773     0.19167452550333444
0.5984333910083158      0.18583996542169007   0.19061292719021042
0.5986082375459856      0.18534609281769554   0.1897149870537297
0.5987799005029648      0.18486172484709987   0.18931183060978804
0.5989400451567998      0.1844103144284534    0.1893209353121921
0.5991136756748516      0.18392138880461462   0.18946554349188383
0.5992757878897592      0.18346536515135273   0.18939433500766098
0.5994347165239763      0.18301873496458432   0.1889244050127102
0.5996071310224104      0.18253469608972925   0.18798076208690137
0.5997680272177002      0.1820834535231423    0.18692463366047518
0.5999424092772069      0.18159488980608493   0.18595394163650927
0.6001136077560231      0.1811157517614357    0.1854440780591566
0.600273287931695       0.18066930208204557   0.18538021274735378
0.6004464539715838      0.18018563971860002   0.18552284245924833
0.6006081017083283      0.17973461064548082   0.1855273248723047
0.6007665658643824      0.17929289709075522   0.18516949756163217
0.6009385158846534      0.17881407639232172   0.1843264261425239
0.6010989476017802      0.17836778428957673   0.18329502828448224
0.6012728651831238      0.17788447178024677   0.18226165846233852
0.6014352644613232      0.1774336327522766    0.18165704643708505
0.6015944801588321      0.17699206220922664   0.1814818679870851
0.6017671817205579      0.1765133684540496    0.18159234142806943
0.6019283649791395      0.1760670584883831    0.18167041095391115
0.602103034101938       0.17558390581159297   0.18140161795707666
0.602274519644046       0.17511006481886268   0.18065701627130817
0.6024344868830097      0.1746685027353802    0.17966539027756778
0.6026079399861903      0.17419020884100847   0.1785951539652083
0.6027698747862267      0.17374414048565284   0.1779018246964437
0.6029286260055725      0.17330727707929475   0.17763623395903044
0.6031008630891354      0.17283379033509702   0.17770145643820995
0.6032615818695539      0.17239242647557618   0.17781395431972274
0.6034357865141894      0.17191452845333005   0.17764928372269093
0.6036068075781343      0.17144587103585815   0.1770181353977568
0.603766310338935       0.1710092310178382    0.1760855247530542
0.6039392989639526      0.17053616833264373   0.174993669036868
0.6041007692858259      0.1700950693692448    0.1742125387778778
0.6042757254719162      0.1696176377870402    0.17382830213581865
0.604447498077316       0.16914940791688765   0.17385464982288423
0.6046077523795715      0.1687130352663447    0.17397934211384986
0.604781492546044       0.1682404425796044    0.17387780532822977
0.6049437144093721      0.16779965291506962   0.1733644485831511
0.6051027526920097      0.16736795657318898   0.1724978293471933
0.6052752768388643      0.16690015031476582   0.1714061607940936
0.6054362826825745      0.16646404283832245   0.1705596245480858
0.6056107743905018      0.16599191597179827   0.17007488621284989
0.6057820825177386      0.16552891860812324   0.17003359296123838
0.6059418723418311      0.16509751292017902   0.17016246804130597
0.6061151480301404      0.16463020101906548   0.1701407107712325
0.6062769054153055      0.16419442628932346   0.16973480384265335
0.6064354792197801      0.16376767205910747   0.16895140674010198
0.6066075388884716      0.16330512230556363   0.16787799618001892
0.606768080254019       0.16287400490566137   0.16697312206193973
0.6069421074837832      0.16240718298091      0.16638011466952018
0.6071046164104031      0.16197173838710702   0.16624988030019983
0.6072639417563326      0.1615452746200542    0.16635908852859443
0.6074367529664789      0.161083218090835     0.16641863516087282
0.607598045873481       0.16065243306997568   0.16614680347159896
0.6077728246447001      0.16018614716177093   0.16539376986345697
0.6079444198352286      0.15972876255806812   0.1643540540707842
0.6081044967226129      0.15930244371598212   0.16341296804283362
0.6082780594742141      0.1588407223620023    0.16273025628039187
0.608440103922671       0.1584101260217585    0.1625180743512355
0.6085989647904374      0.1579884437519653    0.16259461763945468
0.6087713115224207      0.15753147472479068   0.16269525317829792
0.6089321399512597      0.15710552630017666   0.1625164721303618
0.6091064542443158      0.1566443863084876    0.16187026027476945
0.6092775849566813      0.1561922014015191    0.16088291083089257
0.6094371973659025      0.15577092914258753   0.15991724250424544
0.6096102956393407      0.15531458536776052   0.15914387279308911
0.6097718756096345      0.15488909946053114   0.15883620258911801
0.6099302719992379      0.1544724590928524    0.158861331524298
0.6101021542530581      0.1540208651977696    0.1589873954142213
0.6102625182037341      0.15360002251957589   0.15889957418051573
0.6104363680186271      0.15314432334500075   0.15837343983564228
0.6105986995303758      0.15271931956725704   0.15751040596263796
0.610757847461434       0.152303124005859     0.15653599762146442
0.6109304812567091      0.15185219212534146   0.15566016607176247
0.61109159674884        0.15143184697445233   0.15522445621096997
0.6112661981051878      0.1509768643271434    0.15517040617097286
0.611437615880845       0.15053073232197622   0.15530130762738814
0.611597515353358       0.15011507473416766   0.15528257807498572
0.611770900690088       0.14966490422450326   0.15486379328396277
0.6119327677236737      0.1492451511315297    0.1540786700253425
0.6120914511765688      0.1488341346950971    0.1531204495147124
0.6122636204936809      0.14838872774506467   0.15218699434558816
0.6124242715076487      0.1479736273076526    0.15165842982591313
0.6125984083858335      0.1475242370176828    0.15152075137735654
0.6127693616833277      0.1470836264733213    0.15163840766783035
0.6129287966776777      0.14667320783663004   0.1516807646542798
0.6131017175362447      0.1462286262014413    0.1513755853980848
0.6132631200916673      0.14581417830379817   0.15068551273485345
0.6134380085113069      0.14536566989406122   0.14966125117458925
0.6136097133502559      0.14492590257656368   0.14870227585945284
0.6137698998860607      0.14451615228898615   0.14811684629826022
0.6139435722860824      0.1440724706260685    0.1479186870680207
0.6141057263829599      0.1436587467529185    0.14800950574658214
0.6142646968991469      0.14325345883975127   0.14809172619609773
0.6144371532795507      0.1428143252259992    0.14788149852008442
0.6145980913568103      0.1424050511087401    0.14728326372474693
0.6147725152982869      0.14196206196525607   0.14630798285263177
0.6149437556590729      0.14152774796703688   0.14532038342221024
0.6151034777167146      0.14112317658788948   0.14465545617138695
0.6152766856385733      0.14068502540393027   0.1443640793165672
0.6154383752572877      0.14027655766674985   0.14441305587712328
0.6155968812953116      0.139876646756227     0.1445229122409052
0.6157688731975525      0.13944328942251305   0.14440714399588478
0.615929346796649       0.13903949918687933   0.14391297015354318
0.6161033062599626      0.13860237224211025   0.1430069316634297
0.6162657474201318      0.13819475164507297   0.1420526111402408
0.6164250049996106      0.13779564980595924   0.14129530175799956
0.6165977484433063      0.13736334797804858   0.14087659470306535
0.6167589735838577      0.13696043308781206   0.14085007947427083
0.616933684588626       0.1365244306574494    0.1409800630479179
0.6171052120127037      0.1360969976846912    0.14093737796209338
0.6172652211336372      0.13569882766595018   0.14053673293557348
0.6174387161187878      0.13526771270588905   0.13970588971614792
0.617600692800794       0.13486579790769768   0.13876043938099955
0.6177594859021097      0.13447232700883952   0.13795035860844826
0.6179317648676423      0.13404605175630757   0.13743926783047242
0.6180925255300307      0.1336488533833424    0.13734399652183651
0.618266772056636       0.1332189662900551    0.13746492472002034
0.6184378350025508      0.13279757494782546   0.13748906702237118
0.6185973796453212      0.1324051325369563    0.1371885887038746
0.6187704101523087      0.13198014795209187   0.13645038959265082
0.6189319223561519      0.13158404748308775   0.1355311210912551
0.619106920424212       0.13115552326376295   0.13460066066879461
0.6192787349115816      0.13073545474175488   0.1340337263494664
0.6194390310958069      0.13034413918351498   0.13388940903525348
0.6196128131442492      0.12992054991533009   0.13399465743404412
0.6197750768895472      0.1295256471670665    0.1340578247094292
0.6199341570541547      0.1291390672544948    0.13384190846377395
0.6201067230829791      0.12872036148885205   0.13319385390446117
0.6202677708086592      0.12833021196602398   0.13231371254569863
0.6204423043985563      0.1279079960203572    0.1313544347024184
0.6206136544077628      0.12749401393253393   0.13070675783492094
0.6207734861138251      0.12710846784174318   0.13048472922968632
0.6209468036841044      0.12669105746676024   0.13055429418979161
0.6211086029512394      0.12630201609303135   0.1306527310196061
0.6212672186376839      0.12592122186660573   0.13052251541763857
0.6214393201883452      0.1255087182593617    0.1299777466055644
0.6215999034358624      0.1251244511701934    0.12915393192760907
0.6217739725475964      0.124708602131095     0.1281796954724735
0.6219448580786399      0.12430105930926387   0.1274521067912537
0.6221042253065391      0.12392161503265227   0.12714081924486242
0.6222770783986553      0.12351075098421996   0.1271552241627451
0.6224384131876273      0.1231279156947489    0.12727410733948827
0.6226132338408161      0.12271379171247704   0.12720317472750572
0.6227848709133145      0.12230793336946891   0.12673134510002299
0.6229449896826685      0.1219299616300343    0.12595671925969273
0.6231185943162396      0.1215208679470387    0.12498759173929697
0.6232806806466664      0.12113958895692613   0.12424587909942447
0.6234395833964026      0.12076643182004139   0.12385566985694979
0.6236119720103558      0.12036231742667511   0.12380930690592595
0.6237728423211647      0.11998587582930564   0.12393033002739777
0.6239471984961905      0.11957861244339764   0.12392976001306254
0.6241183710905258      0.1191795331044095    0.12355991641665987
0.6242780253817168      0.11880797894530887   0.12286048258915977
0.6244511655371249      0.11840577518653413   0.12190819564256361
0.6246127873893886      0.11803102190954429   0.12111532602836937
0.6247712256609618      0.11766430331851588   0.12064028007985374
0.6249431497967519      0.11726710509697089   0.12051664132341267
0.6251035556293978      0.11689721017310646   0.12062309629952016
0.6252774473262606      0.11649697543589758   0.12068358237837006
0.6254398207199791      0.1161239671830682    0.12044356496873086
0.625599010533007       0.11575894616023802   0.11985215808599162
0.625771686210252       0.11536375977148143   0.11894370592134204
0.6259328435843527      0.11499564857952874   0.11810311866742326
0.6261074868226704      0.11459751550712258   0.1174829462454506
0.6262789464802975      0.1142074349748642    0.11728483000258322
0.6264388878347803      0.11384427240437885   0.11736455430743639
0.62661231505348        0.11345127012844558   0.11746419509772113
0.6267742239690355      0.11308499401013138   0.11730924533777096
0.6269329493039004      0.11272657187625726   0.11680699142410512
0.6271051605029824      0.11233847804864722   0.11595240027075214
0.62726585339892        0.11197707867887756   0.11509717513702393
0.6274400321590746      0.11158615713097968   0.11440194980461885
0.6276110273385387      0.11120320215975621   0.11411729521130372
0.6277705042148585      0.11084678040511746   0.11415393569873875
0.6279434669553953      0.11046102689733      0.11427916170187367
0.6281049113927878      0.11010172507399903   0.1142077277639819
0.6282798416943972      0.10971324541106708   0.11374256217315354
0.6284515884153161      0.10933268546772891   0.11293542233905643
0.6286118168330908      0.10897841092659104   0.11208388538135544
0.6287855311150823      0.10859515448194172   0.11134876987482381
0.6289477270939295      0.10823809934710775   0.111014628094277
0.6291067394920863      0.10788879578989755   0.11100317256149553
0.62927923775446        0.10751070392072186   0.11113585131377736
0.6294402177136895      0.10715864723270828   0.11112860900056983
0.6296146835371359      0.10677796147837827   0.11076116468197958
0.6297859657798918      0.10640510079834162   0.11002674122675653
0.6299457297195034      0.10605810238369433   0.10918713898594887
0.6301189795233318      0.1056826774472486    0.10839954026047578
0.6302807110240161      0.10533302729760938   0.107985593351521
0.6304392589440098      0.1049910273077786    0.10791183844610011
0.6306112927282205      0.10462080080534675   0.10803633652360609
0.6307718082092869      0.10427617653765127   0.10808652732018215
0.6309458095545702      0.10390349028703867   0.10782260866789428
0.6311082925967093      0.10355631624350424   0.10721582906188765
0.6312675920581579      0.10321673714885865   0.10640969473346705
0.6314403773838233      0.10284930147914512   0.10557030938497763
0.6316016444063445      0.10250720051917518   0.10505545835078609
0.6317763972930827      0.10213741193618454   0.10488824329157581
0.6319479665991303      0.10177529557173874   0.10499226680008408
0.6321080176020337      0.10143832935620986   0.10508114801902817
0.632281554469154       0.10107389015191137   0.10490437764740469
0.63244357303313        0.10073450768626999   0.10438084470759347
0.6326024080164155      0.10040261066140836   0.10361816577583652
0.6327747288639181      0.10004345246516817   0.1027610271662654
0.6329355314082763      0.09970915977583471   0.10218123216617706
0.6331098198168514      0.0993476802986646    0.10193476867438285
0.6332809246447361      0.09899376791381882   0.10200266565050697
0.6334405111694764      0.09866454515006982   0.10211984314124124
0.6336135835584337      0.09830844985408616   0.10203011559870441
0.6337751376442466      0.09797694812632973   0.10160071983145277
0.6339335081493691      0.09765282143254955   0.10089750037915242
0.6341053645187086      0.09730204209802115   0.10003687862836326
0.6342657025849038      0.09697566681303407   0.09939531517959035
0.6344395265153159      0.0966228197778209    0.09905973632164912
0.6346018321425837      0.09629427791586513   0.09906787707869613
0.634760954189161       0.09597305056811684   0.09919872556110523
0.6349335620999552      0.09562557734175085   0.099204275779449
0.6350946517076052      0.09530221429249629   0.09889547909899145
0.6352692271794721      0.09495279115394516   0.09820454214487054
0.6354406190706485      0.09461076760607073   0.0973578955593311
0.6356004926586807      0.09429265144467203   0.09667395970499616
0.6357738521109297      0.09394871126078663   0.09626163629238092
0.6359356932600345      0.09362857583275036   0.09621373979000583
0.6360943508284488      0.09331563478161042   0.09633876298139528
0.63626649426108        0.09297710269365013   0.09640204197291016
0.6364271193905668      0.09266217324412224   0.09618316050934954
0.6366012303842707      0.09232184440478836   0.09557912022403822
0.6367721577972841      0.09198879713888253   0.09476225393035138
0.6369315669071532      0.09167914243112121   0.0940439471009922
0.6371044618812393      0.09134433166074604   0.09355121536105991
0.637265838552181       0.09103280709719602   0.09343390489899298
0.6374407010873397      0.09069632311846795   0.0935524207073866
0.637612380041808       0.09036705665562467   0.09364737856451942
0.6377725406931319      0.09006086058186061   0.09348791628234045
0.6379461872086728      0.08972995468358853   0.09295022817957352
0.6381083154210694      0.08942200990792051   0.09220811482642739
0.6382672600527755      0.08912106416274156   0.09147378492121112
0.6384396905486984      0.08879565478819348   0.09091777546303999
0.6386006027414771      0.08849299170846298   0.0907354119435197
0.6387750007984728      0.08816606749720637   0.09082264340409285
0.638946215274778       0.08784623373008422   0.09095071018397399
0.6391059114479389      0.08754892315828151   0.09086861027071184
0.6392790934853168      0.08722759049765058   0.09042440515018414
0.6394407572195503      0.08692865354432625   0.0897349871647445
0.6395992373730933      0.08663658194632462   0.08899576930862924
0.6397712033908532      0.08632075697482913   0.08837847526608383
0.6399316511054689      0.08602712307401882   0.0881222467202585
0.6401055846843016      0.08570994446662335   0.08816204337478271
0.64026799995999        0.08541484125729608   0.08830442134320078
0.6404272316549878      0.0851265307717187    0.08830894403516636
0.6405999492142026      0.08481493556427076   0.08798556972853179
0.6407611484702731      0.08452518845600096   0.08737702557081513
0.6409358335905606      0.08421237045544294   0.08657350636976432
0.6411073351301575      0.08390644145509586   0.08591384995721452
0.6412673183666101      0.08362212458013253   0.08559420152958316
0.6414407874672797      0.08331500792956412   0.08558334284222358
0.6416027382648051      0.08302938388683967   0.08572777566910864
0.6417615054816399      0.08275040985830186   0.08578616380336342
0.6419337585626916      0.08244890332108479   0.08555337633232833
0.6420944933405991      0.0821686548011262    0.08501853861969094
0.6422687139827236      0.08186609359896352   0.08424283539081587
0.6424397510441575      0.0815702808566875    0.08354730219260349
0.6425992698024472      0.08129548269218517   0.08316024415708756
0.6427722744249538      0.08099865039557583   0.08308554289600381
0.6429337607443161      0.08072270934904933   0.08321803755963864
0.6431087329278953      0.08042495926457464   0.08332301116814321
0.643280521530784       0.08013388125907282   0.083150196601031
0.6434407918305284      0.079863445015311     0.0826716721587834
0.6436145479944898      0.07957148487972789   0.08192825560129988
0.643776785855307       0.07930004010383346   0.08125219078778423
0.6439358401354336      0.07903501465194923   0.08081205481273739
0.6441083802797771      0.07874874627974543   0.0806774964601625
0.6442694021209764      0.07848274548939572   0.08078872087561216
0.6444439098263927      0.07819573286224904   0.08092862087709468
0.6446152339511183      0.07791524237955334   0.08083545643038138
0.6447750397726997      0.07765476253284191   0.08043659746422607
0.6449483314584981      0.0773735634196719    0.07974253403719694
0.6451101048411522      0.07711224473454878   0.07905712925605798
0.6452686946431158      0.07685718782700493   0.07856433365043466
0.6454407703092963      0.07658169982967845   0.07836065340594432
0.6456013276723325      0.07632587320664472   0.07843736777964781
0.6457753708995857      0.07604985968318165   0.07860083769941328
0.6459378958236947      0.07579333514376965   0.07859370144209289
0.646097237167113       0.07554298534611291   0.07829891797668645
0.6462700643747483      0.07527273869011127   0.07768089971216637
0.6464313732792394      0.07502171968704295   0.07699783935779811
0.6466061680479473      0.07475104562531171   0.07640102107301025
0.6467777792359648      0.0744866527195273    0.07613794666690517
0.646937872120838       0.0742412167664058    0.07617711243376778
0.6471114508699282      0.07397643147064303   0.07634941529773533
0.6472735113158741      0.07373046588324132   0.07639931040542355
0.6474323881811295      0.07349050701034225   0.07618324530004165
0.6476047509106018      0.07323149970255624   0.07563551830448337
0.6477655953369298      0.07299104387711047   0.07497317043336346
0.6479399256274747      0.07273178708539618   0.07434242245573891
0.6481110723373291      0.07247864547096136   0.074014857307076
0.6482707007440393      0.0722437775764229    0.0740052758370846
0.6484438150149664      0.0719904214594089    0.0741738254781467
0.6486054109827493      0.07175519821446398   0.07427516147851483
0.6487638233698416      0.07152580849817221   0.07414139453901547
0.6489357216211509      0.07127823828472109   0.07367634588170918
0.6490961015693159      0.07104852588664501   0.07304830104953207
0.6492699673816978      0.07080088607600202   0.07239249787900505
0.6494323148909356      0.07057096022494154   0.07200962819674839
0.6495914788194826      0.07034677359044372   0.07192592594602137
0.6497641286122467      0.0701049736135544    0.07206863523985155
0.6499252601018666      0.0698806067048072    0.07221717883380414
0.6500998774557033      0.0696388847332176    0.07215972588667889
0.6502713112288495      0.06940301464245782   0.07177342704977495
0.6504312266988515      0.06918428688745383   0.07118830074471473
0.6506046280330704      0.06894853027710211   0.0705252189469067
0.6507665110641451      0.06872976854715084   0.07009441263742108
0.6509252105145291      0.06851656383442828   0.06995507271348231
0.6510973958291302      0.06828665114515999   0.07006750219724657
0.651258062840587       0.06807344562215911   0.07023848767525895
0.6514322157162606      0.06784379601102916   0.07025475698545854
0.6516031850112438      0.0676198181072839    0.06995578525583576
0.6517626360030827      0.06741227666040889   0.0694261601278179
0.6519355728591386      0.0671887160301506    0.06876717917198732
0.6520969914120502      0.06698140532199015   0.06829109332191993
0.6522718958291788      0.0667582609141276    0.06808475636553579
0.6524436166656168      0.06654068621279394   0.06817521623305152
0.6526038191989105      0.06633905644208307   0.06835624026431374
0.6527775075964212      0.06612193083844842   0.06841872314284551
0.6529396776907875      0.06592059536362961   0.06820116494780276
0.6530986642044634      0.06572452003358309   0.06772963879630368
0.6532711365823562      0.06551328083380172   0.0670885650153739
0.6534320906571048      0.06531753065928833   0.06658337373597856
0.6536065305960703      0.06510688959033849   0.0663198353542658
0.6537777869543453      0.0649016253207796    0.06636855460138123
0.653937525009476       0.06471153898964735   0.06655277970965838
0.6541107489288237      0.06450690569070913   0.06667084865441114
0.6542724545450271      0.06431729241924408   0.06653249721309146
0.65443097658054        0.06413274016700407   0.06612956156820539
0.6546029844802698      0.06393397878342442   0.06551859168297526
0.6547634740768553      0.06374993043387583   0.06499033574375711
0.6549374495376578      0.06355195076836213   0.06466650098388274
0.6550999066953159      0.06336852340661585   0.06465641603023667
0.6552591802722836      0.06319004861405522   0.0648270068528441
0.6554319397134682      0.06299798627141089   0.06499956338195062
0.6555931808515086      0.06282016377077483   0.06495623403709869
0.6557679078537659      0.06262903638843967   0.0645988102945102
0.6559394512753327      0.06244298199635843   0.0640265481399163
0.6560994763937552      0.06227084473094769   0.06348999073380758
0.6562729873763946      0.0620857585805682    0.06312013247106678
0.6564349800558897      0.06191442572096618   0.06306295188431749
0.6565937891546944      0.06174783822164478   0.06321485627055173
0.656766084117716       0.061568651468582605  0.06341677405039835
0.6569268607775933      0.06140289966172784   0.06343985303827669
0.6571011233016876      0.061224836103285406  0.063164016862569
0.6572722022450913      0.06105164006905047   0.06264214781345584
0.6574317628853508      0.06089155023456836   0.062105428427563406
0.6576048093898271      0.06071951063945404   0.06169041827107225
0.6577663375911592      0.06056041033220808   0.06157886659407741
0.6579413516567083      0.06038965270899881   0.06172173185327646
0.6581131821415669      0.060223782812692625  0.06193943069259202
0.6582734943232812      0.06007055701792873   0.062004788274591265
0.6584472923692124      0.05990605513485968   0.061787176653290324
0.6586095721119993      0.05975397402190061   0.06133642280503539
0.6587686682740957      0.059606302845361295  0.060811216464377316
0.6589412503004092      0.05944771430804059   0.06036624604051997
0.6591023140235783      0.059301215176772136  0.060208903535000705
0.6592768636109644      0.05914409363076432   0.06031881507934049
0.6594482296176599      0.05899150462340547   0.06054842998218271
0.6596080773212112      0.05885066348747162   0.06066669913098172
0.6597814108889793      0.058699570334283656  0.06052810560777801
0.6599432261536032      0.058560051498356865  0.06013747260333129
0.6601018578375366      0.05842471812621138   0.059632863227707694
0.660273975385687       0.058279495915778606  0.05916186643900832
0.6604345746306931      0.05814551218667651   0.0589547150084859
0.6606086597399161      0.05800193823659949   0.05902111520185062
0.6607795612684486      0.05786267367211615   0.059251928428056704
0.6609389444938368      0.05773430146378432   0.059418410224122684
0.661111813583442       0.057596713986538636  0.05936267958132281
0.6612731643699028      0.057469842971531744  0.05904249218936322
0.6614480010205807      0.05733405956779116   0.05851781825202028
0.661619654090568       0.05720246366715056   0.05804210991709349
0.6617797888574111      0.05708123317591506   0.05780965194803362
0.661953409488471       0.05695147057118971   0.05784897843016328
0.6621155118163867      0.056831894970876234  0.0580650921868461
0.6622744305636119      0.056716150097193836  0.05826647147640311
0.662446835175054       0.05659224594515927   0.05828123082092754
0.6626077214833518      0.056478183666645726  0.05802862442571649
0.6627820936558666      0.05635626865609519   0.05754502676821444
0.6629532822476909      0.05623831137366548   0.05706114391974528
0.6631129525363709      0.0561298403269393    0.05678799497484075
0.6632861086892679      0.05601390137171895   0.05677942514875801
0.6634477465390205      0.0559072679103691    0.056979865627998186
0.6636062008080826      0.0558042307268983    0.05720933975367046
0.6637781409413618      0.05569410286896831   0.05729482393047549
0.6639385627714967      0.05559293095356774   0.05711731975185976
0.6641124704658484      0.055484978538516506  0.056686715289425545
0.6642748598570559      0.05538582402034667   0.056225031819541624
0.6644340656675729      0.055290331115905984  0.05590147911069038
0.6646067573423068      0.055188457573726306  0.05582285276261333
0.6647679307138964      0.055094984655563374  0.055987159923605055
0.664942589949703       0.054995443229343285  0.05625916795146106
0.6651140656048191      0.05489949193952581   0.05640195682341582
0.665274022956791       0.054811573991273135  0.056292505964479586
0.6654474661729797      0.05471797820401644   0.05591798297374962
0.6656093910860241      0.054632228920056314  0.05546975907462724
0.6657681324183781      0.05454969611881887   0.05512155006645476
0.6659403596149489      0.05446186781925506   0.05499590978095464
0.6661010685083755      0.05438152614911593   0.05512895760119348
0.666275263266019       0.05429620331269262   0.05540904259575481
0.6664462744429721      0.05421422376750064   0.05560645686991129
0.6666057673167809      0.05413936079899347   0.055569725805240215
0.6667787460548066      0.05405991006863395   0.05526172637204514
0.666940206489688       0.05398738766717103   0.054836556350870244
0.6671151527887864      0.05391059524673227   0.05443873646018974
0.6672869155071942      0.05383701110559805   0.05429016923175591
0.6674471599224577      0.053769981277631974  0.05440502424419503
0.6676208902019383      0.05369907908517046   0.054689030997253786
0.6677831021782745      0.05363454093451196   0.0549140321324003
0.6679421305739203      0.0535728305433236    0.054940360838816114
0.6681146448337829      0.05350763696237819   0.05469705820450974
0.6682756407905013      0.053448440968586856  0.05430255606054042
0.6684501226114367      0.0533860817267674    0.05389402552624624
0.6686214208516815      0.053326679230543746  0.053704817522874246
0.6687812007887821      0.05327289753926143   0.05378330080219
0.6689544665900996      0.053216352985522374  0.05406050775829493
0.6691162140882728      0.05316523755782673   0.054321775641120604
0.6692747780057555      0.05311669554076327   0.05441318209311333
0.6694468277874551      0.053065782465015955  0.05424307817796307
0.6696073592660104      0.053019929406571874  0.053888969852379294
0.6697813766087828      0.052972027404312116  0.0534763883910121
0.6699438756484108      0.05292899172286227   0.05325031242186756
0.6701031911073484      0.05288839089170178   0.05327180458768182
0.6702759924305028      0.052846137110387534  0.05352305571397056
0.670437275450513       0.05280837662936855   0.05381545162318479
0.6706120443347402      0.05276942776563585   0.053986989659177274
0.6707836296382768      0.052733173358585086  0.05388438965486241
0.670943696638669       0.0527010091734717    0.053574146687119156
0.6711172495032783      0.052667942874348415  0.0531689357510082
0.6712792840647434      0.05263876980179488   0.0529152099273137
0.6714381350455179      0.05261176316048761   0.05289807441113225
0.6716104718905093      0.0525842487587146    0.05312584581003587
0.6717712904323564      0.0525602498386519    0.053434001782955674
0.6719455948384205      0.052536067324783935  0.05366354628627962
0.672116715663794       0.05251417824224729   0.053634818085439476
0.6722763181860233      0.05249541707546695   0.05337768268729324
0.6724494065724695      0.052476876699662875  0.05298805208094952
0.6726109766557715      0.052461266883475584  0.05270831702131746
0.672769363158383       0.05244755579707152   0.05264735371670835
0.6729412355252113      0.05243446049215288   0.05284115434587442
0.6731015895888954      0.05242391733695037   0.0531559048139001
0.6732754295167965      0.05241431466466203   0.05343983564966352
0.6734377511415534      0.05240706535320744   0.053494117027703396
0.6735968891856197      0.05240156892170276   0.05331107119222819
0.6737695130939029      0.05239741089721562   0.05295150511157857
0.6739306186990418      0.052395225047035844  0.052643979056655306
0.6741052101683976      0.05239470469835387   0.052524777215944105
0.674276618057063       0.05239606536447978   0.052685833981625074
0.674436507642584       0.052399007127805826  0.05299978707495776
0.6746098830923221      0.05240402241671683   0.05332531718725312
0.6747717402389158      0.0524104196540779    0.05344514858402982
0.6749304138048191      0.05241829961949619   0.05332503445165987
0.6751025732349393      0.05242865166952892   0.05300262481809899
0.6752632143619152      0.05244000381826514   0.052688374636943154
0.6754373413531081      0.05245415567924274   0.0525284771659683
0.6756082847636103      0.052469918580198664  0.052647744285511736
0.6757677098709683      0.052486289801403345  0.05295108774372634
0.6759406208425434      0.05250586942699551   0.05331191543554725
0.6761020135109741      0.05252585787183817   0.05349885278807295
0.6762768920436217      0.05254938476475744   0.05343135709601902
0.6764485869955789      0.05257437484188978   0.053142801556333996
0.6766087636443917      0.05259937912431969   0.052834184106968295
0.6767824261574216      0.0526283335447281    0.05265616015520364
0.6769445703673072      0.05265735547698691   0.05274099203254829
0.6771035309965021      0.05268744432332373   0.05302992002917332
0.6772759774899141      0.05272190522709062   0.05341505857740098
0.6774369056801818      0.05275577214034203   0.0536581924838266
0.6776113197346664      0.05279433873610196   0.053662695155983285
0.6777825502084605      0.052834084968028446  0.05342305385378196
0.6779422623791103      0.052872839543209774  0.05312167304804035
0.6781154604139771      0.05291670116593991   0.05291256065324181
0.6782771401456996      0.052959368434806114  0.052955307295483194
0.6784356362967316      0.05300280994364129   0.053219993965927415
0.6786076183119806      0.053051755667722675  0.05362105555976963
0.6787680820240852      0.053099120030595706  0.05391899013708856
0.6789420316004068      0.05315231505791029   0.054002116240223254
0.6791044628735841      0.05320372521062225   0.05383578667417871
0.6792637105660708      0.05325575658294201   0.05355228712564107
0.6794364441227746      0.05331401739748831   0.05330652564610505
0.6795976593763341      0.053370104749932055  0.053288450432719146
0.6797723604941105      0.053432749300883896  0.05354110719864529
0.6799438780311964      0.05349613967870259   0.053949719043658244
0.680103877265138       0.05355695877251682   0.05429142538679527
0.6802773623632966      0.053624742584520414  0.0544447423797011
0.680439329158311       0.053689752331227365  0.05433660998442443
0.6805981123726347      0.053755102340253345  0.0540794772158294
0.6807703814511754      0.05382781440273621   0.053819639573225504
0.6809311322265719      0.053897365223881326  0.05376251082270685
0.6811053688661852      0.05397460467917791   0.053976934466568205
0.6812764219251081      0.054052308559876604  0.05438321722384273
0.6814359566808866      0.054126454795758526  0.054764550069427584
0.6816089773008822      0.05420869573690397   0.054991780995174204
0.6817704796177335      0.05428717698265084   0.05495132173496812
0.6819454677988017      0.054374080817239094  0.05470283134631134
0.6821172723991794      0.05446129478625581   0.05444343012228808
0.6822775586964128      0.0545443510869926    0.054367746186479646
0.6824513308578631      0.05463623764605105   0.05456087547058306
0.6826135847161692      0.054723763673766565  0.05494403722450895
0.6827726549937847      0.054811194107817755  0.05535485664702761
0.6829452111356172      0.054907852298259854  0.0556456627124692
0.6831062489743055      0.05499984376689139   0.05566944061480256
0.6832807726772107      0.055101591113767195  0.05546445608201418
0.6834521127994253      0.055203360280501206  0.05520419528388435
0.6836119346184957      0.055299963911737235  0.055095423066330716
0.683785242301783       0.05540654571013626   0.055245404903355805
0.6839470316819259      0.05550775799201484   0.05561185378609638
0.6841056374813784      0.05560858411719669   0.05604555892273733
0.6842777291450479      0.055719779850349836  0.05640051396977018
0.6844383025055731      0.0558252181987481    0.05649620318788531
0.6846123617303153      0.05594134785632551   0.05634641634860552
0.6847832373743669      0.05605720996157027   0.05609352790758224
0.6849425947152742      0.05616691837203709   0.05595192996277547
0.6851154379203985      0.05628771680011971   0.05605034978825233
0.6852767628223786      0.05640215934551066   0.05638799685176257
0.6854515735885756      0.05652801379986976   0.056881516132815266
0.685623200774082       0.056653441872054566  0.05728057053027471
0.6857833096564442      0.05677211744861206   0.05742880013487116
0.6859569044030233      0.05690260384249206   0.05732471805561526
0.6861189808464581      0.057026135381760726  0.05709858034147072
0.6862778737092023      0.05714883580635202   0.056935179536454175
0.6864502524361636      0.05728373526025825   0.05698922575192231
0.6866111128599807      0.057411295089683685  0.05729620501214268
0.6867854591480146      0.0575513729903468    0.057796298962465915
0.686956621855358       0.05769073734279783   0.05824752996452015
0.6871162662595571      0.05782236895350699   0.05846690519258383
0.6872893965279732      0.05796691402170401   0.05842966102170939
0.6874510084932449      0.05810352578102231   0.05822847950158999
0.6876094368778262      0.05823902284336027   0.05804693689031722
0.6877813511266244      0.05838781811786295   0.05805141748318788
0.6879417470722784      0.05852829873794141   0.05831649112522854
0.6881156288821493      0.058682393779001245  0.05881043233466449
0.6882779923888759      0.05882797342363248   0.059286689338893736
0.688437172314912       0.05897228384110475   0.05959253059027671
0.688609838105165       0.05913059367234416   0.0596478543268818
0.6887709855922737      0.05928000642618455   0.05949107463412827
0.6889456189435994      0.05944373498445908   0.059280384595554805
0.6891170687142346      0.05960631084713161   0.059245152989976554
0.6892770001817256      0.059759599697900005  0.05947033265521904
0.6894504175134334      0.05992779094224152   0.05995077216619848
0.689612316541997       0.06008654767107042   0.06045772443008617
0.68977103198987        0.06024374724070794   0.06082689939543704
0.6899432333019601      0.06041605356529366   0.06095888506874333
0.6901039163109058      0.06057847465219189   0.06084880331670679
0.6902780851840685      0.060756313164486     0.06064002535617506
0.6904490704765407      0.06093270541794137   0.060563663795267314
0.6906085374658686      0.06109882399508057   0.06073965808555218
0.6907814903194134      0.061280743756152375  0.06119290552647286
0.6909429248698139      0.06145219148482654   0.061721165886599344
0.6911178452844314      0.061639750177083466  0.06218661316139325
0.6912895821183584      0.061825702506833044  0.062374975506158135
0.6914498006491412      0.0620007952947557    0.062304158899441624
0.6916235050441408      0.06219238164708953   0.06210651370889261
0.6917856911359962      0.06237291055482948   0.062009494991328955
0.691944693647161       0.06255143726815848   0.06214187553274868
0.6921171820225428      0.0627468291699811    0.06256448877594219
0.6922781520947803      0.06293078888010013   0.06310338812605777
0.6924526080312348      0.06313191914936897   0.06362603956707945
0.6926238803869987      0.06333115606004189   0.06389255563696154
0.6927836344396184      0.06351857870146974   0.06388092697883277
0.692956874356455       0.06372354904867315   0.06370353446295315
0.6931185959701474      0.0639165100046985    0.06357972516353251
0.6932771340031492      0.0641071873319656    0.06366211853950463
0.6934491579003679      0.06431577865242866   0.06404081797965358
0.6936096634944423      0.06451199113704624   0.0645783727609531
0.6937836549527338      0.06472641860613637   0.06515284644906971
0.6939461281078809      0.06492827256387354   0.06549283699502703
0.6941054176823376      0.06512768951892998   0.06556562546464975
0.6942781931210111      0.0653456866574581    0.06542881039251146
0.6944394502565404      0.0655507417101809    0.06527993962925673
0.6946141932562867      0.06577467703180649   0.06531241395893912
0.6947857526753425      0.06599628212745413   0.06564799168929233
0.694945793791254       0.06620456940268948   0.06617619088673415
0.6951193207713824      0.0664321074999061    0.06678891353393795
0.6952813294483665      0.06664613587982111   0.06719820832172643
0.6954401545446601      0.06685745021363791   0.06734188179095917
0.6956124655051706      0.06708840710093993   0.06725032775452923
0.6957732581625369      0.06730568053952422   0.06709663759717474
0.6959475366841201      0.06754287778879975   0.06708185873968565
0.6961186316250129      0.06777745957791224   0.06736284579978043
0.6962782082627613      0.06799777947615675   0.06786832554201024
0.6964512707647267      0.0682383847693657    0.06850964243272341
0.6966128149635478      0.06846453756076382   0.06898872828189068
0.6967711755816783      0.06868769363319507   0.06921275439687637
0.6969430220640258      0.06893148558731442   0.06918146758865895
0.697103350243229       0.06916046555962986   0.06903378366002108
0.6972771642866492      0.06941036938682688   0.06897312941530455
0.6974394600269251      0.06964527162123908   0.06917118554734063
0.6975985721865106      0.06987702453786229   0.06962698903212552
0.6977711702103129      0.07013004952540527   0.07028012960231393
0.6979322499309709      0.0703677153845138    0.07083354550723814
0.698106815515846       0.07062693941237697   0.07117807432591654
0.6982781975200304      0.07088311241055564   0.07121078038397484
0.6984380612210705      0.07112356253247601   0.07108075913885224
0.6986114107863277      0.07138592287597971   0.07098857374507775
0.6987732420484406      0.07163237439800654   0.07113162733753065
0.698931889729863       0.07187540296293003   0.07154386738327156
0.6991040232755023      0.07214068294824136   0.07219466181331327
0.6992646385179973      0.07238970355897349   0.07279676659752751
0.6994387396247093      0.07266125592560661   0.07322841351877021
0.6996096571507308      0.07292948151366288   0.073339097614027
0.699769056373608       0.07318109109303458   0.07324029057932303
0.6999419414607021      0.07345557734991372   0.07312353371358972
0.700103308244652       0.07371326529243838   0.07320680630405359
0.7002781608928188      0.07399410829346026   0.07361436595982508
0.700449829960295       0.07427147285153977   0.07426132936302177
0.700609980724627       0.07453168460804448   0.07489511224235083
0.700783617353176       0.0748153938532298    0.07538888591313779
0.7009457356785807      0.07508176906260002   0.07555947184607814
0.7011046704232948      0.07534430332350187   0.07550034913026189
0.7012770910322259      0.07563066689415539   0.07537386045802744
0.7014379933380127      0.07589935488975026   0.07540858770465353
0.7016123815080165      0.07619214481944075   0.0757566253927638
0.7017835860973297      0.07648118697733616   0.07637955460807513
0.7019432723834986      0.07675231176610192   0.07704201007908358
0.7021164445338846      0.07704797704024251   0.07761368127274244
0.7022780983811262      0.07732542658851828   0.07786757659081266
0.7024365686476773      0.07759876734533697   0.0778626997820879
0.7026085247784454      0.07789688312132655   0.07773815916189881
0.7027689626060691      0.07817644659373378   0.07772512857390079
0.7029428862979099      0.07848104958645828   0.07800233356366844
0.7031052916866064      0.07876692255440701   0.07855006141866785
0.7032645134946123      0.07904854034058789   0.07922581317870434
0.7034372211668352      0.07935551622538818   0.07988157104476243
0.7035984105359138      0.0796434284724431    0.08024298966659792
0.7037730857692093      0.07995696052744705   0.0803168712258753
0.7039445774218143      0.0802663235650039    0.08020755888553024
0.704104550771275       0.08055628433538362   0.080160572088861
0.7042780099849527      0.0808721856424293    0.08037450587612283
0.7044399508954862      0.08116851139279167   0.08087689882829176
0.7045987082253291      0.08146032150032514   0.08155130759209323
0.704770951419389       0.08177838225215747   0.08226211833594425
0.7049316763103045      0.08207654227060707   0.08270666005038263
0.7051058870654371      0.0824012076840944    0.08286152979491346
0.705276914239879       0.08272144107584571   0.08278233296628063
0.7054364231111767      0.08302144370552073   0.08270763491306667
0.7056094178466914      0.0833482638453576    0.08285243141339715
0.7057708942790618      0.08365468433243141   0.08329454305080299
0.7059458565756491      0.08398817418932897   0.08402642609913119
0.7061176352915459      0.08431708721847041   0.08477255570933043
0.7062778957042984      0.08462527378485853   0.08527575048563295
0.706451641981268       0.08496083817047116   0.0854937271025677
0.7066138699550932      0.08527550881811526   0.08545149697844966
0.7067729143482279      0.08558526804112317   0.08536351266358529
0.7069454446055795      0.08592270321875291   0.08545147961098956
0.7071064565597869      0.08623893099177764   0.08583477583745515
0.7072809543782111      0.08658307962761157   0.08653948116694547
0.7074522686159449      0.08692239662955857   0.08731861329860431
0.7076120645505344      0.0872401880099321    0.08789637687031468
0.7077853463493409      0.08758620009762155   0.08820495962142427
0.7079471098450031      0.0879105236833102    0.0882149673310727
0.7081056897599747      0.08822968982793693   0.08812453409844882
0.7082777555391633      0.08857752704926213   0.0881559831366824
0.7084383030152076      0.08890337530054575   0.08846927737461273
0.7086123363554688      0.08925798442956338   0.08912821703045239
0.7087748513925858      0.08959042460218496   0.08988724785799262
0.7089341828490122      0.08991756738031796   0.09054655045883409
0.7091070001696557      0.09027375448770088   0.09097326746323489
0.7092682991871548      0.09060746896129998   0.09106604063424804
0.7094430840688709      0.09097046057815664   0.09098284059776784
0.7096146853698965      0.09132822750988256   0.09097306404740328
0.7097747683677778      0.09166321436747109   0.09122350031384162
0.709948337229876       0.09202776230434048   0.09183154718400567
0.7101103877888301      0.0923693726471366    0.09259161555465173
0.7102692547670935      0.09270544321050717   0.09330395344472378
0.7104416076095739      0.0930713486119462    0.09382171945377732
0.71060244214891        0.09341402240124479   0.09398980403792893
0.710776762552463       0.09378675587793664   0.09393676009903969
0.7109478993753255      0.09415401937705305   0.09389210148898988
0.7111075178950438      0.09449775367602847   0.09407366887929293
0.711280622278979       0.09487182173794623   0.09461413567638394
0.71144220835977        0.0952222081175444    0.09535813055911471
0.7116006108598704      0.09556681950093841   0.09611381227702338
0.7117724992241877      0.09594202878817282   0.09672574868673686
0.7119328692853608      0.09629327196021892   0.09698348667982606
0.7121067252107508      0.09667532998186674   0.09698018882004333
0.7122690628329965      0.09703327145414162   0.09691333629917434
0.7124282168745517      0.09738530970995563   0.09701089722060063
0.7126008567803239      0.09776842246750178   0.0974501190111657
0.7127619783829519      0.09812713822030661   0.09814694697299786
0.7129365858497967      0.09851714172842406   0.0990110239810819
0.713108009735951       0.09890130730957103   0.09970172351455264
0.713267915318961       0.09926079221209864   0.10004532457965819
0.713441306766188       0.09965182449114837   0.10010190789782134
0.7136031799102707      0.10001803069057849   0.10003258319914456
0.7137618694736628      0.10037810810667266   0.10007666008122262
0.713934044901272       0.10076998294384132   0.10043649342913329
0.7140947020257369      0.10113676082201069   0.10108413775322064
0.7142688450144187      0.10153554147947272   0.10195851367566434
0.71443980442241        0.10192829970735469   0.10272390073900523
0.714599245527257       0.10229578101028286   0.10316252368503774
0.714772172496321       0.10269552578477144   0.10329791103172284
0.7149335811622407      0.10306974802801966   0.10324257319318007
0.7151084756923773      0.10347643379554204   0.10325069544500745
0.7152801866418235      0.10387692171762955   0.10355754814318775
0.7154403792881253      0.10425161613865283   0.10416685000470043
0.7156140577986441      0.10465901655293981   0.10504106124098413
0.7157762180060185      0.10504048442713194   0.10581961622544987
0.7159351946327025      0.10541547631665285   0.10634305195404754
0.7161076571236035      0.10582340712430406   0.10656017457901505
0.7162686013113602      0.1062051475494484    0.10653339203191385
0.7164430313633338      0.106620018158918     0.10650641127879606
0.7166142778346168      0.10702846773874702   0.10673599421533018
0.7167740060027556      0.10741046684208039   0.10727851092638521
0.7169472200351114      0.10782582771953705   0.1081318864220124
0.7171089157643228      0.10821460464155379   0.10895550706929581
0.7172674279128437      0.10859669351503648   0.1095664982333399
0.7174394259255816      0.10901236648100028   0.10988077959007227
0.7175999056351753      0.10940120822464625   0.1099009735738191
0.7177738712089858      0.10982381660915198   0.10985200401150276
0.7179363184796521      0.11021946276823678   0.10998730797396697
0.7180955821696279      0.11060830526398155   0.11043118161300168
0.7182683317238205      0.11103113132701561   0.11122966216568143
0.7184295629748689      0.11142675270522671   0.11208399334096358
0.7186042800901343      0.111856535767438     0.11285196340812312
0.7187758136247092      0.11227956500512283   0.1132586669596208
0.7189358288561398      0.11267514524203225   0.1133349053591225
0.7191093299517873      0.11310510236348256   0.1132831115138806
0.7192713127442906      0.11350748503600037   0.11336010148920948
0.7194301119561033      0.1139028629265675    0.11372690727337387
0.719602397032133       0.11433282403318733   0.11446909593401924
0.7197631638050185      0.11473497879665438   0.11532973949917538
0.7199374164421208      0.11517188618428692   0.11617093192110643
0.7201084854985327      0.11560183545361803   0.11667983290304769
0.7202680362518002      0.11600374484577723   0.11682967134583286
0.7204410728692847      0.11644061127936671   0.11679288671148784
0.7206025911836249      0.11684931789925684   0.11681766034991045
0.7207775953621821      0.11729323183668636   0.11714831280804568
0.7209494159600488      0.11773010938145106   0.1178467994368789
0.7211097182547712      0.118138589851428     0.11870426882354311
0.7212835064137105      0.11858239879101276   0.11959076498894533
0.7214457762695056      0.11899769247174957   0.12015296149611662
0.7216048625446101      0.11940567616244988   0.12037854144663837
0.7217774346839315      0.11984917657719311   0.12037223097471325
0.7219384885201087      0.12026394444197526   0.12036352290564589
0.7221130282205028      0.12071438356513067   0.12061340177625657
0.7222843843402065      0.12115755197606998   0.12123724398036023
0.7224442221567658      0.12157176963487154   0.12207214964482052
0.7226175458375421      0.12202184387066271   0.1230045562521871
0.7227793512151741      0.12244285513905569   0.12365648973458374
0.7229379730121156      0.12285637119824389   0.12397198574644896
0.723110080673274       0.12330592071032329   0.12401584324545649
0.7232706700312881      0.12372620131372927   0.12398600013810548
0.7234447452535192      0.12418266055035054   0.12415389442248606
0.7236156368950598      0.12463166044698215   0.12468715374578915
0.7237750102334561      0.12505118481835625   0.12547880046748866
0.7239478694360694      0.1255070615581565    0.12644021979199985
0.7241092103355383      0.12593335647912454   0.12717990050816066
0.7242840370992243      0.12639614381679745   0.12762262223606125
0.7244556802822197      0.1268513675135419    0.12771316750615258
0.7246158051620709      0.12727680857124546   0.12768031570376978
0.724789415906139       0.12773890931091367   0.1277978597611923
0.7249515083470628      0.12817112404439598   0.12823328423444638
0.7251104172072961      0.12859556833607758   0.12897229593167467
0.7252828119317463      0.12905683157245013   0.12994150570850393
0.7254436883530522      0.12948801979985403   0.1307494160873795
0.7256180506385751      0.12995615767642266   0.13129556031648423
0.7257892293434075      0.13041655534968347   0.13146168714624953
0.7259488897450956      0.13084668876829858   0.1314384960514291
0.7261220360110007      0.13131392753040153   0.1314959222014992
0.7262836639737614      0.13175080458662058   0.1318414998218529
0.7264421083558317      0.13217974647065853   0.13250918298968561
0.7266140386021188      0.13264594168675853   0.13346514196746698
0.7267744505452617      0.13308159744187953   0.13433131560903278
0.7269483483526216      0.1335546418489043    0.1349868235212152
0.7271107278568372      0.13399712450365622   0.13524310862360492
0.7272699237803623      0.13443158007941047   0.13525726385727044
0.7274426055681043      0.13490355841808024   0.13526047221912393
0.727603769052702       0.13534472349127447   0.13549547468535553
0.7277784184015167      0.13582352662105943   0.13613376429873017
0.7279498841696408      0.1362943256987592    0.1370601584895457
0.7281098316346206      0.13673413884272806   0.1379633972904278
0.7282832649638175      0.1372117261056191    0.13871165366176624
0.72844517998987        0.1376582383592629    0.1390589501352668
0.7286039114352321      0.1380965683391657    0.13912133600822488
0.728776128744811       0.13857280103413627   0.13910429452166057
0.7289368277512457      0.1390177975243869    0.13926351327952183
0.7291110126218974      0.1395008022146526    0.13980557028821508
0.7292820139118585      0.13997564408618454   0.14068059706771072
0.7294414968986754      0.14041908923917512   0.14160491002721873
0.7296144657497092      0.14090066650698516   0.14244283703534275
0.7297759162975987      0.14135076419236192   0.14289341833682975
0.7299508527097052      0.1418390937999099    0.14302581445004978
0.7301226055411211      0.14231917259870014   0.14300698004645518
0.7302828400693927      0.14276761772858626   0.14312024466963622
0.7304565604618813      0.14325441154451712   0.1435933550972345
0.7306187625512257      0.14370949207572398   0.1443760100176683
0.7307777810598796      0.14415616249578841   0.1453021528401277
0.7309502854327503      0.1446412913672873    0.14620827225888103
0.7311112715024768      0.14509456386917144   0.14675263040044564
0.7312857434364203      0.14558638483195321   0.14696478940550464
0.7314570317896731      0.14606980911963785   0.14695813596085344
0.7316168018397817      0.14652123509391823   0.14701702130765867
0.7317900577541073      0.1470113141586912    0.1473930891348593
0.7319517953652886      0.14746932149972436   0.14810075801458003
0.7321103493957795      0.14791878527444108   0.14900946095481168
0.7322823892904872      0.1484070000861022    0.14997266026765982
0.7324429108820506      0.14886301188453485   0.15061528901645016
0.732616918337831       0.14935785501093643   0.15092593170384377
0.7327794074904672      0.14982042494336403   0.15095457102031581
0.7329387130624128      0.1502743785141719    0.15096524426215716
0.7331115044985753      0.15076725439435446   0.15122311863965793
0.7332727776315936      0.1512277611086288    0.15181996180757432
0.7334475366288288      0.15172729595029688   0.15277737110557407
0.7336191120453734      0.15221822035902347   0.15377510462604999
0.7337791691587738      0.15267661858603823   0.154499655694794
0.7339527121363912      0.15317410274971843   0.15490642833963614
0.7341147368108644      0.15363899636015865   0.15498308093309876
0.734273577904647       0.1540951507212511    0.1549764705193383
0.7344459048626465      0.15459046919501968   0.1551542452824904
0.7346067135175017      0.15505308347429658   0.15565994523314586
0.7347810080365739      0.15555492590164882   0.15655808697121504
0.7349521189749555      0.1560480318694529    0.15757337468118046
0.7351117116101928      0.156508322014154     0.15837637258627693
0.7352847901096472      0.15700791236711292   0.15889134883158804
0.7354463503059573      0.15747462890242012   0.15903619392440674
0.7356047269215769      0.15793249291593597   0.1590302541787299
0.7357765894014133      0.15842972317264428   0.1591345629970078
0.7359369335781055      0.15889397808985734   0.1595392776544813
0.7361107636190147      0.15939765330129108   0.1603550227527623
0.7362730753567797      0.1598682986047222    0.16131522217959404
0.736432203513854       0.16033002958970033   0.16219649483844667
0.7366048175351453      0.16083123952865255   0.16284656465395456
0.7367659132532923      0.16129932468774877   0.1630966852720704
0.7369404948356563      0.16180693679839234   0.16311880336519435
0.7371118928373297      0.16230563114238858   0.1631724651995856
0.7372717725358588      0.16277110832095718   0.16348866387932898
0.737445138098605       0.1632761645263998    0.1642174581659572
0.7376069853582069      0.16374795536872352   0.16515093477818513
0.7377656490371182      0.16421073197574523   0.166073686712513
0.7379377985802464      0.1647131341422159    0.16682204355394173
0.7380984298202304      0.1651821881807097    0.16716505747604327
0.7382725469244313      0.16569090580290782   0.16723311771868138
0.7384434804479417      0.16619060211331377   0.16725035130500027
0.7386028956683078      0.16665687027705514   0.1674769250632826
0.738775796752891       0.16716284174680457   0.16810167596699835
0.7389371795343298      0.16763534319107734   0.1689855971643035
0.7391120481799855      0.1681475797553106    0.17002648344440519
0.7392837332449508      0.16865073725647653   0.17083842141268346
0.7394439000067717      0.1691203517647624    0.17125112476594662
0.7396175526328096      0.16962975534421046   0.17136433856786112
0.7397796869557032      0.17010557752068406   0.17136681058536987
0.7399386376979062      0.17057224333610888   0.17152230982729805
0.7401110743043263      0.17107870455456642   0.1720489288284517
0.7402719926076021      0.17155151874308827   0.17287172890471222
0.7404463967750948      0.17206414992414806   0.17391856435104552
0.740617617361897       0.17256761096641277   0.17480700935839033
0.7407773196455549      0.17303736419320973   0.17531651725582936
0.7409505077934299      0.17354695354728128   0.17550466527610492
0.7411121776381605      0.17402280298323117   0.17550963599034988
0.7412706639022005      0.17448941800083237   0.1756010444855162
0.7414426360304576      0.17499588349989784   0.1760206682841069
0.7416030898555703      0.17546855717447793   0.17676241066358342
0.7417770295449         0.17598109290007377   0.17779220572380888
0.7419394509310854      0.17645980829957306   0.17870505129413133
0.7420986887365802      0.17692924666750204   0.1793343664918203
0.7422714124062921      0.1774385537083144    0.17963576512359633
0.7424326177728597      0.17791399550581188   0.17967217203087887
0.7426073090036442      0.17842931120114852   0.179719763454019
0.7427788166537382      0.17893532972261725   0.18004698347795625
0.742938806000688       0.17940744184990723   0.1807055989897953
0.7431122812118546      0.1799194266191223    0.18169943935381497
0.7432742381198769      0.18039748288848423   0.18264883043153846
0.7434330114472087      0.180866197698256     0.18336280244845302
0.7436052706387576      0.18137477960324214   0.18376126071610668
0.7437660115271622      0.1818494002433182    0.18384307920365242
0.7439402382797836      0.18236388319483024   0.18386154861235574
0.7441112814517146      0.1828690009054751    0.18409629637696212
0.7442708063205012      0.1833401286163027    0.18465760925063057
0.7444438170535048      0.18385110498483448   0.18559237740621948
0.7446053094833641      0.18432807559109357   0.1865608379120158
0.7447802877774404      0.184844883567477     0.18742206132775405
0.7449520824908262      0.18535228798266964   0.18789115831942965
0.7451123589010678      0.18582566529715422   0.18801513368029246
0.7452861211755262      0.18633885815178802   0.18802627153021398
0.7454483651468403      0.18681801188802813   0.18818650363010664
0.745607425537464       0.18728773795320464   0.188657581475461
0.7457799717923046      0.18779724801442976   0.18952310980623915
0.745940999744001       0.18827269504180438   0.19048906676404403
0.7461155135599142      0.1887879094261037    0.19141777481104255
0.746286843795137       0.18929366644468512   0.19198528514265747
0.7464466557272155      0.1897653633452595    0.19217835561459745
0.7466199535235108      0.19027679365634906   0.19219641177983562
0.746781733016662       0.19075415827625716   0.19229360197310782
0.7469403289291225      0.19122205431059267   0.19266759089169921
0.7471124107058001      0.1917296463820094    0.19344423761675883
0.7472729741793335      0.19220317145691862   0.19438726174088938
0.7474470235170837      0.19271636162476852   0.1953710769928578
0.7476178892741434      0.19322004973567053   0.1960417220144391
0.7477772367280588      0.19368967427373837   0.19632139543696514
0.7479500700461912      0.19419891780815812   0.19636944339183302
0.7481113850611792      0.19467409841383565   0.1964170394583538
0.7482861859403843      0.19518886053206871   0.19673987912528748
0.7484578032388989      0.1956940951048493    0.19744524284498907
0.7486179022342692      0.19616527761697503   0.19835758289754069
0.7487914870938563      0.19667598732679772   0.19936604847009817
0.7489535536502993      0.1971526493905053    0.20007739900097218
0.7491124366260516      0.19761979266420748   0.20044198529341156
0.7492848054660209      0.19812640588051453   0.20053601287778006
0.749445656002846       0.1985989892699269    0.20055747266903629
0.749619992403888       0.19911099532120505   0.20078995087674972
0.7497911452242395      0.19961344275891385   0.20139331456223689
0.7499507797414467      0.20008188339614957   0.20225017282933622
0.7501239001228709      0.2005896802440024    0.20327494342842295
0.7502855022011508      0.2010634809167975    0.20406439368094018
0.7504439206987401      0.20152774415342647   0.2045244995965258
0.7506158250605465      0.20203129453740062   0.20468601125782976
0.7507762111192086      0.2025008783624195    0.2046999644697316
0.7509500830420875      0.20300969233979377   0.20484909326059325
0.7511124366618223      0.20348455413483957   0.20530648523347014
0.7512716067008665      0.20394986825429404   0.2060727144926541
0.7514442626041276      0.20445433543803343   0.2070879596581339
0.7516054002042444      0.20492488733525804   0.20795587309845423
0.7517800236685782      0.2054345287201039    0.2085768225397958
0.7519514635522215      0.20593457626875522   0.20881199040437262
0.7521113851327206      0.20640070654992268   0.20883794951567378
0.7522847925774365      0.20690582145144126   0.20892787778221653
0.7524466817190082      0.2073770888290159    0.20928979326509892
0.7526053872798893      0.2078388052458917    0.20997158222313264
0.7527775787049874      0.20833943171190822   0.21095360361707077
0.7529382518269412      0.20880625991431487   0.2118610938420576
0.753112410813112       0.20931192562139858   0.21257718538026427
0.7532833862185923      0.209807991791989     0.2129025103083454
0.7534428433209283      0.21027031473488844   0.21296051570974256
0.7536157862874812      0.21077137761236256   0.21300597086454778
0.7537772109508898      0.21123872424820173   0.21327162310630246
0.7539521214785154      0.21174473178300798   0.2139332172569052
0.7541238484254504      0.21224113526173766   0.21487705059788992
0.7542840570692412      0.21270388493513157   0.21579835689184465
0.7544577515772489      0.213205189785248     0.2165742196591352
0.7546199277821124      0.2136728715665494    0.2169567259074486
0.7547789204062854      0.21413101069394838   0.21705903938774415
0.7549513988946752      0.2146275977615085    0.2170834580258643
0.7551123590799208      0.21509062952748786   0.21727193017383556
0.7552868051293834      0.21559202088194182   0.21783102898004128
0.7554580675981555      0.21608381486085435   0.21871095279075298
0.7556178117637833      0.21654212754125127   0.21963754770138422
0.755791041793628       0.21703868254624112   0.22048572009455383
0.7559527535203284      0.2175017928971783    0.22095918593609565
0.7561112816663383      0.21795537909082924   0.221123774350961
0.7562832956765652      0.218447089146762     0.22114618626778515
0.7564437913836477      0.21890543353476138   0.2212649507520028
0.7566177729549473      0.21940180419686506   0.22171324232059583
0.7567802362231025      0.219864849567449     0.22246349943207896
0.7569395159105673      0.22031838077945837   0.2233718012362361
0.757112281462249       0.22080981183732523   0.22429075883493724
0.7572735287107863      0.22126800383165313   0.22487720161475866
0.7574482618235407      0.22176399155207657   0.22514906142875665
0.7576198113556045      0.22225040426102208   0.2251879404042279
0.757779842584524       0.22270367082785517   0.22525859851036414
0.7579533596776605      0.22319459621247473   0.22561060506748892
0.7581153584676528      0.22365242173338284   0.22627590816995136
0.7582741736769545      0.22410075262684123   0.227147499478955
0.7584464747504731      0.22458650784955847   0.2280982172547227
0.7586072575208475      0.22503926841249758   0.22876325675695192
0.7587815261554388      0.2255294293275243    0.22912343309922356
0.7589526112093395      0.22601004700162264   0.22919846160011245
0.759112177960096       0.22645777547704282   0.2292344334907061
0.7592852305750695      0.2269427577026759    0.22949083065346187
0.7594467648868987      0.2273949034849064    0.23005784544988048
0.7596217850629448      0.22788418440846536   0.23096666479097228
0.7597936216583003      0.22836393885049472   0.23192433174318422
0.7599539399505116      0.22881096944832913   0.23263613171956019
0.76012774410694        0.2292949807401359    0.2330601600121597
0.760290029960224       0.22974632456291738   0.23317004775267908
0.7604491322328175      0.2301882545885067    0.2331912827093223
0.7606217203696279      0.2306670108270771    0.23337231022610438
0.7607827902032941      0.23111321573802238   0.23384772400088152
0.7609573459011771      0.23159611969562516   0.2346867022059843
0.7611287180183697      0.23206954172302058   0.2356420808595333
0.7612885718324179      0.23251053565236085   0.23641069102809584
0.7614619115106832      0.23298806357823842   0.23692387944442506
0.7616237328858042      0.23343322521009563   0.2370935777594845
0.7617823706802347      0.2338690284539968    0.23711647867887406
0.7619544943388821      0.23434120090487706   0.23723045843712165
0.7621150996943852      0.2347811335095049    0.23760847893564482
0.7622891909141053      0.23525729967147233   0.23835929949475917
0.7624600985531348      0.2357240362348469    0.2392921010419507
0.76261948788902        0.2361586667213576    0.24010672387777024
0.7627923630891222      0.23662935520365055   0.24071283108338629
0.7629537199860802      0.23706800479532592   0.24095868057157518
0.763128562747255       0.23754257038841511   0.24100452452991122
0.7633002219277394      0.2380077341847978    0.2410810073302846
0.7634603628050795      0.23844100011090438   0.2413909455847627
0.7636339895466365      0.23890999851643818   0.24206880312214157
0.7637960979850493      0.23934716991622335   0.24292230895517047
0.7639550228427715      0.23977508118242494   0.2437607621879826
0.7641274335647107      0.2402385421025093    0.2444421572828905
0.7642883259835056      0.24067031981624254   0.2447629148569252
0.7644627042665174      0.2411374967162755    0.24484506303741221
0.7646338989688387      0.24159523626214488   0.24488640593609307
0.7647935753680157      0.24202140220456292   0.24511292209565386
0.7649667376314098      0.24248276238988256   0.2456906972307026
0.7651283815916595      0.24291267897363625   0.2464892740128418
0.7652868419712188      0.24333341574840822   0.24733602162959284
0.7654587882149949      0.2437891560310186    0.24808778244614377
0.7656192161556268      0.24421360801457115   0.24849270058724718
0.7657931299604755      0.2446729065475104    0.24863101870698126
0.76595552546218        0.24510099747837047   0.24865285795745445
0.766114737383194       0.2455199522077899    0.24879106005991555
0.766287435168425       0.24597355570000073   0.24924256550416266
0.7664486146505117      0.24639611333002048   0.24995493834446791
0.7666232799968153      0.2468531569855309    0.2508758139003194
0.7667947617624284      0.24730098473952455   0.2516760749485391
0.7669547252248973      0.24771793611109622   0.25215482553386964
0.7671281745515831      0.24816916491931681   0.25235565589520226
0.7672901055751247      0.2485896028234142    0.2523823986768009
0.7674488530179756      0.24900099572364784   0.2524672125614583
0.7676210863250436      0.24944645920256217   0.2528221941428121
0.7677818013289672      0.24986130259120554   0.2534527407859711
0.7679560021971078      0.2503100463544052    0.25433971224933616
0.7681270194845579      0.25074966270741117   0.2551768872266947
0.7682865184688636      0.2511588377329489    0.2557310092395662
0.7684595033173864      0.2516016952911461    0.25600888489261475
0.768620969862765       0.2520142017397128    0.25605892269091507
0.7687959222723604      0.2524602146117878    0.25611620235500093
0.7689676911012653      0.2528971477880812    0.2564050680952274
0.7691279416270259      0.2533039153777508    0.2569697631521731
0.7693016780170034      0.2537439641456724    0.2578163569996466
0.7694638961038367      0.2541539409765598    0.2586256306784396
0.7696229306099794      0.2545550253517992    0.2592385976068153
0.7697954509803392      0.25498916778126424   0.2595898015844424
0.7699564530475547      0.2553934247375077    0.259675064605759
0.770130940978987       0.25583055618104855   0.25970710473856473
0.7703022453297289      0.2562587075684491    0.25991586482301055
0.7704620313773265      0.2566571682788593    0.2603921938694447
0.770635303289141       0.25708826898931314   0.2611766219529039
0.7707970568978112      0.25748976746868824   0.26198717960597473
0.7709556269257909      0.2578823148300584    0.2626536775259054
0.7711276828179876      0.25830725581302705   0.26308568433168456
0.77128822040704        0.25870281433814285   0.2632220092955215
0.7714622438603094      0.25913057739990875   0.2632462041461117
0.7716247490104344      0.25952906113223645   0.26336977118594335
0.771784070579869       0.25991882798491084   0.26373698952574187
0.7719568780135205      0.26034056319211685   0.26442914467224404
0.7721181671440278      0.2607332225670394    0.2652206637649922
0.772292942138752       0.261157656029236     0.2660033779721339
0.7724645335527857      0.26157328342351965   0.26650422578353994
0.772624606663675       0.26196004666370104   0.26669486512938334
0.7727981656387813      0.26237833707818403   0.26672866001146345
0.7729602063107434      0.26276787045417294   0.26680421523916886
0.7731190634020149      0.26314881194317385   0.2670897252064676
0.7732914063575034      0.26356103686844157   0.26769726643256486
0.7734522310098476      0.2639447158156675    0.2684518459733944
0.7736265415264088      0.26435947771057117   0.2692584026102895
0.7737976684622794      0.26476555627838494   0.26982833483396074
0.7739572770950057      0.26514330828801425   0.2700842597550954
0.774130371591949       0.26555188862142054   0.2701449659486987
0.774291947785748       0.2659322534637963    0.27018490150723035
0.7744503403988564      0.2663041570632386    0.2703906532242761
0.7746222188761819      0.2667066377616643    0.2709026716045422
0.7747825790503631      0.26708112152431374   0.2716029628377398
0.7749564250887612      0.2674859757325056    0.2724168424081753
0.775118752824015       0.26786294710277475   0.2730265649872511
0.7752778969785783      0.2682315283595662    0.2733702708415224
0.7754505269973585      0.26863022242334134   0.27348617982572654
0.7756116387129944      0.2690012580838939    0.27350633131419916
0.7757862362928474      0.26940219464450715   0.2736547548227438
0.7759576502920098      0.2697946452132648    0.2740784607699859
0.776117545988028       0.27015967049765094   0.27471633735663475
0.776290927548263       0.270554327836385     0.27551699384629524
0.7764527908053538      0.2709216779359211    0.27616746771854395
0.776611470481754       0.27128077808371065   0.2765761550520865
0.7767836360223713      0.2716692451700593    0.2767470240119591
0.7769442832598443      0.2720306369382018    0.2767702920721765
0.7771184163615342      0.2724210949518166    0.27686112202099505
0.7772893658825336      0.2728030660090334    0.2771930696731623
0.7774487971003887      0.2731582186631039    0.2777553766857979
0.7776217141824607      0.27354222861144284   0.2785250766181073
0.7777831129613885      0.2738995436831753    0.2792053743282534
0.7779579976045332      0.27428549505606475   0.27971482688610455
0.7781296986669874      0.27466318255666106   0.2799280077249896
0.7782898814262974      0.27501442389878855   0.2799626798146945
0.7784635500498243      0.27539402242781924   0.28002159819042627
0.7786257003702068      0.27574730094439703   0.28026929559160835
0.7787846671098989      0.2760925678404573    0.28075662223942743
0.7789571197138079      0.27646591735689074   0.28148149417092855
0.7791180540145726      0.2768131930092496    0.2821718999469334
0.7792924741795543      0.2771883254870564    0.2827375476533623
0.7794637107638455      0.27755534818762106   0.28301343600661055
0.7796234290449924      0.2778965517261084    0.28307510017930754
0.7797966331903562      0.2782653270226486    0.2831060913222686
0.7799583190325757      0.27860841235616746   0.28328207030001934
0.7801168212941048      0.27894364627041185   0.2836870392373168
0.7802888094198507      0.27930617160247706   0.2843514984955442
0.7804492792424523      0.2796432589061454    0.28503795469315907
0.780623234929271       0.2800074070761054    0.28565353634081286
0.7807856723129454      0.2803462490256837    0.28598819880650717
0.7809449261159292      0.28067732679951357   0.2861000056721106
0.78111766578313        0.2810351796009006    0.2861195101906176
0.7812788871471865      0.28136798317718026   0.2862226128895933
0.78145359437546        0.281727326728004     0.28657462925229515
0.781625118023043       0.2820788043485155    0.2871731476953904
0.7817851233674817      0.28240549855276076   0.28783969597743947
0.7819586145761374      0.28275843613477086   0.28848564604117266
0.7821205874816487      0.28308672515818734   0.2888762323849349
0.7822793768064695      0.28340741792642327   0.289034781937395
0.7824516519955073      0.28375406246761486   0.28906225122554924
0.7826124088814007      0.28407632122492116   0.28912227068773066
0.7827866516315112      0.2844242919516325    0.2893917663582595
0.7829577108009311      0.2847645629535628    0.2899132458316234
0.7831172516672068      0.2850807199025163    0.2905457442848713
0.7832902783976994      0.2854222863340716    0.29121084934950187
0.7834517868250477      0.2857396928155923    0.2916565409506097
0.783626781116613       0.2860822024708247    0.29188196556482016
0.7837985918274877      0.28641710939356185   0.29192286938696255
0.7839588842352181      0.28672833590829955   0.29196009527816297
0.7841326625071655      0.28706440474969386   0.2921729525143685
0.7842949224759687      0.28737693531470726   0.2926039948673476
0.7844539988640813      0.2876821473811166    0.2931926610993433
0.7846265611164108      0.28801190296003637   0.293858574657798
0.7847876050655961      0.28831839556758143   0.2943443457290871
0.7849621348789984      0.2886491859978437    0.2946254432547486
0.78513348111171        0.2889725553902434    0.2946947599153152
0.7852933090412774      0.28927294566223694   0.2947134266942618
0.7854666228350617      0.28959732507127844   0.2948610836059237
0.7856284183257018      0.28989886957942484   0.29521594403990375
0.7857870302356514      0.29019328145946655   0.2957490327910001
0.7859591280098179      0.29051137972774604   0.29640247335841796
0.7861197074808401      0.290806922351523     0.2969220810541468
0.7862937728160794      0.29112590242917646   0.2972627539365286
0.7864646545706281      0.29143764617248924   0.2973730665959402
0.7866240180220325      0.2917271221774528    0.2973865238268373
0.7867968673376539      0.29203972292575814   0.2974771940518492
0.7869581983501309      0.2923302022223744    0.2977532574978178
0.7871330152268249      0.2926435536550726    0.2982770888235228
0.7873046485228283      0.29294976961458397   0.2989056011490398
0.7874647635156875      0.29323415627200133   0.2994363115678505
0.7876383643727637      0.29354109776747955   0.29981365449594094
0.7878004469266956      0.29382635840948446   0.2999543272509684
0.787959345899937       0.29410478044622884   0.2999751978751432
0.7881317307373953      0.29440544615367537   0.30002690578056335
0.7882925972717093      0.2946847183674138    0.300235670130918
0.7884669496702403      0.29498597840451785   0.30068681631160615
0.7886381184880807      0.29528029344494466   0.30127852834290636
0.7887977690027769      0.2955535111915733    0.30181938686100035
0.78897090538169        0.29584839544417085   0.302243984440111
0.7891325234574589      0.2961223329161956    0.3024323424955146
0.7892909579525372      0.29638962638876076   0.3024727798736669
0.7894628783118325      0.2966782713112292    0.3024971984674203
0.7896232803679835      0.296946209489111     0.30264216838999336
0.7897971682883514      0.2972350397779394    0.3030139173109616
0.7899595379055752      0.29750338512114105   0.3035265689721128
0.7901187239421082      0.29776519914630595   0.3040671126991536
0.7902913958428583      0.2980477701690619    0.3045422332534822
0.7904525494404642      0.2983101541475987    0.30479355647098855
0.7906271889022869      0.2985930352386274    0.3048747717787925
0.7907986447834191      0.2988692803693364    0.30488733115151867
0.790958582361407       0.29912564471497827   0.3049832205798881
0.791132005803612       0.29940218099484966   0.30528266591779735
0.7912939109426727      0.29965899230004306   0.30574052063242063
0.7914526325010428      0.29990947924762645   0.30626200555526106
0.7916248399236299      0.30017981994002846   0.306759923659159
0.7917855290430726      0.30043073564084943   0.3070555945347715
0.7919597040267323      0.30070124349729993   0.30717623070268407
0.7921306954297015      0.3009653217476449    0.30718916803539226
0.7922901685295264      0.3012102834113381    0.307243735718501
0.7924631274935683      0.30147450993544916   0.3074704706271298
0.7926245681544659      0.30171977677472867   0.3078648485018951
0.7927994946795804      0.301984044129781     0.308407975328533
0.7929712376240045      0.3022419947583711    0.3089083152868623
0.7931314622652842      0.30248129731064816   0.309228647207436
0.7933051727707809      0.30273926965099845   0.3093786814123523
0.7934673649731334      0.30297875245308203   0.3093990249272095
0.7936263735947953      0.3032122355866961    0.30942677233779964
0.7937988680806741      0.3034640648891948    0.30959305696902634
0.7939598442634087      0.30369770986533545   0.30992567219496153
0.7941343063103602      0.3039494349643987    0.3104266915626086
0.7943055847766212      0.30419505341656944   0.3109274307299809
0.7944653449397379      0.304422801293671     0.31127951037970203
0.7946385909670716      0.30466829645243587   0.3114723694902814
0.794800318691261       0.304896080664038     0.31151195960345596
0.7949588628347598      0.305118077466511     0.31152256722713595
0.7951308928424756      0.30535749589857725   0.3116331338984044
0.7952914045470472      0.3055795106493101    0.3119004560781601
0.7954654021158356      0.30581867930268414   0.3123487130810197
0.7956278813814798      0.30604060553412876   0.3128155311331042
0.7957871770664334      0.3062568599891394    0.3132000934712535
0.7959599586156041      0.3064897899321624    0.3134478429389014
0.7961212218616305      0.3067057094612721    0.31352264521226103
0.7962959709718738      0.306938166036855     0.31352955473087546
0.7964675365014265      0.30716484942897093   0.3135985564836203
0.796627583727835       0.30737493937923477   0.3138080516835023
0.7968011168184604      0.3076012317994116    0.314200580909438
0.7969631316059417      0.30781109459787626   0.31464399643266017
0.7971219628127323      0.3080155118460489    0.31503904767262414
0.7972942798837399      0.3082358048400469    0.3153226415471512
0.7974550786516033      0.3084399821069603    0.3154291792528053
0.7976293632836835      0.3086597665508827    0.3154420298927828
0.7978004643350732      0.30887400076179666   0.3154773877273164
0.7979600470833187      0.3090724411829479    0.3156303361136591
0.7981331156957812      0.30928615393976394   0.3159608652440204
0.7982946660050994      0.30948423687099547   0.3163708764608001
0.7984530327337269      0.30967709749049044   0.31676769967100993
0.7986248853265715      0.3098849034884451    0.31708403047169065
0.7987852196162718      0.31007739381221094   0.31722710307460955
0.798959039770189       0.31028456074488925   0.3172568626560714
0.7991213416209619      0.3104765770656154    0.31726698017068006
0.7992804598910443      0.31066349270205845   0.3173612041100452
0.7994530640253438      0.3108647557505477    0.31761764372786533
0.7996141498564989      0.3110511844662443    0.31797983104041566
0.799788721551871       0.3112516899983505    0.3184074205786841
0.7999601096665525      0.3114469893784248    0.3187436859081568
0.8001199794780898      0.31162777879343456   0.31891829949361816
0.8002933351538439      0.3118223076653449    0.3189703306916875
0.8004551725264538      0.31200249177786904   0.31897182958699927
0.8006138263183732      0.3121778001426209    0.3190276613434748
0.8007859659745096      0.31236651853563585   0.319224525314455
0.8009465873275017      0.3125412087039898    0.31953783486171045
0.8011206945447107      0.31272903810530994   0.3199423439703061
0.8012916181812292      0.3129118862323919    0.3202917524305188
0.8014510235146034      0.3130810307832784    0.32049783712974456
0.8016239147121945      0.3132629769539766    0.32057919142509284
0.8017852876066415      0.31343138489256894   0.3205819464850343
0.8019601463653053      0.313612321814395     0.32061467086962553
0.8021318215432787      0.31378838640407997   0.320767823288893
0.8022919784181077      0.31395104226020276   0.3210382627204548
0.8024656211571537      0.31412587041727685   0.3214128443047509
0.8026277455930554      0.3142876706332729    0.321744280050945
0.8027866864482666      0.3144449524180849    0.3219727755014136
0.8029591131676947      0.3146140778832439    0.3220822618534616
0.8031200215839787      0.3147704964074902    0.3220939946616686
0.8032944158644795      0.31493848851719225   0.3221047747963073
0.8034656265642898      0.3151018599577168    0.322209506977781
0.8036253189609558      0.31525285300205363   0.3224287830383049
0.8037984972218387      0.31541508395640344   0.3227649866854629
0.8039601571795774      0.31556510400192284   0.3230896802438959
0.8041186335566255      0.3157108383561118    0.32333667404185346
0.8042905957978906      0.31586748343662036   0.32347669687080244
0.8044510397360115      0.3160122371975183    0.3235048843363592
0.8046249695383493      0.3161676327627307    0.3235032227491622
0.8047873810375428      0.3163113050939505    0.32355938211581264
0.8049466089560458      0.3164508185494424    0.3237203921705801
0.8051193227387656      0.3166006454583704    0.32400630600946523
0.8052805182183412      0.3167390698802733    0.324315149595762
0.8054551995621339      0.31688753788745366   0.32460103080409775
0.805626697325236       0.3170317448723197    0.32476531061624175
0.8057866767851938      0.3171648776414047    0.32481218117861904
0.8059601421093685      0.3173077186101175    0.3248080967122817
0.806122089130399       0.31743965271520846   0.3248352429229917
0.8062808525707389      0.31756766050544155   0.32495104275606007
0.8064531018752957      0.3177050495807399    0.3251887539307945
0.8066138328767084      0.3178318511289701    0.3254712692611103
0.806788049742338       0.3179677652665196    0.3257578705788321
0.806959083027277       0.31809965138268353   0.32594459359345135
0.8071185980090718      0.31822127682148005   0.3260140585731816
0.8072915988550835      0.31835168090993343   0.32601581552599035
0.8074530813979509      0.3184719909419441    0.3260213835526408
0.8076280498050353      0.31860080998842893   0.32610847697498113
0.8077998346314292      0.31872572878722766   0.3263052327929927
0.8079601011546789      0.31884088151984735   0.32655830109868367
0.8081338535421454      0.3189642081788225    0.32683335190586627
0.8082960876264677      0.31907793598474027   0.3270214730759872
0.8084551381300994      0.3191880185196819    0.32711085047874766
0.808627674497948       0.31930582514369515   0.3271236208424722
0.8087886925626524      0.31941436666059303   0.3271174762149418
0.8089631964915738      0.31953047339348817   0.32716778688952614
0.8091345168398046      0.31964291943323137   0.3273177226895469
0.8092943188848912      0.3197464289012233    0.32753508079767857
0.8094676067941947      0.3198571732404363    0.3277945677633175
0.8096293764003539      0.31995914863022024   0.32799134348347975
0.8097879624258226      0.32005779858373695   0.3281008224161483
0.8099600343155082      0.32016336194693507   0.32813000772846496
0.8101205879020495      0.32026047505273336   0.3281195449881419
0.8102946273528079      0.32036423737936714   0.3281393271305284
0.8104571485004219      0.3204597172093876    0.3282360263792037
0.8106164860673455      0.3205520007887503    0.3284098458978467
0.8107893094984859      0.3206506119447587    0.3286451141987137
0.8109506146264821      0.3207412595616007    0.32884670220449014
0.8111254056186953      0.32083797042764556   0.3289877661550645
0.8112970130302178      0.3209313877329764    0.3290335366090146
0.8114571021385962      0.3210171673373677    0.3290252782357513
0.8116306771111915      0.3211086826528166    0.32902468760043485
0.8117927337806425      0.321192726520339     0.3290849369438342
0.811951606869403       0.3212738093454347    0.32921913346929876
0.8121239658223804      0.32136030914593877   0.3294234667244537
0.8122848064722136      0.3214396535747442    0.32961676007668583
0.8124591329862637      0.32152415303504406   0.3297693748256271
0.8126302759196233      0.32160559502293296   0.3298335829338595
0.8127899005498387      0.3216802045526414    0.3298331884295617
0.8129630110442709      0.32175964452203215   0.3298194816178697
0.8131246032355589      0.3218324167871731    0.3298475281735382
0.8132830118461564      0.3219024610999913    0.32994169574083965
0.8134549063209707      0.32197702000852235   0.3301097453610825
0.8136152824926408      0.3220452244245118    0.3302878250084017
0.8137891445285279      0.32211768386205913   0.33044659494943385
0.8139514882612707      0.32218395368862707   0.3305270421753498
0.814110648413323       0.3222476225789746    0.33054217359372395
0.8142832944295922      0.32231523048561384   0.3305238764660823
0.8144444221427172      0.3223769622374408    0.3305238136909368
0.8146190357200591      0.32244237331015985   0.33058552235112715
0.8147904657167104      0.3225049522787422    0.3307160221379626
0.8149503774102175      0.3225619523038526    0.33087218481952846
0.8151237749679415      0.3226222978117303    0.33102786695734526
0.8152856542225213      0.3226772640226014    0.3311201625687794
0.8154443498964106      0.3227298660592279    0.33114907637290725
0.8156165314345167      0.32278550305463705   0.3311337831779004
0.8157771946694786      0.32283607277851356   0.3311177671547566
0.8159513437686575      0.32288942228753015   0.33114454826697975
0.8161223092871458      0.3229403160400579    0.3312356995635857
0.8162817565024899      0.3229864607962881    0.3313645194783443
0.8164546895820509      0.3230350699589006    0.33151035686841296
0.8166161043584677      0.3230790926277744    0.3316108955006207
0.8167910049991013      0.3231253251156183    0.33165607972869177
0.8169627220590444      0.32316923244025314   0.3316457657080206
0.8171229208158433      0.3232088709485508    0.3316213976018679
0.8172966054368591      0.32325040466511634   0.3316231734395851
0.8174587717547307      0.3232878317792931    0.331676670717903
0.8176177544919118      0.3233232581781484    0.33177528670809925
0.8177902230933097      0.3233602741029578    0.33190370999910923
0.8179511733915634      0.32339349079957813   0.3320051282046423
0.8181256095540341      0.3234280458318193    0.3320628283981837
0.8182968621358142      0.3234605101184536    0.33206275942698427
0.81845659641445        0.32348948867103006   0.3320348817689176
0.8186298165573028      0.3235194951505582    0.33201420096806966
0.8187915183970114      0.32354617596807506   0.3320345458203111
0.8189500366560294      0.32357108620401354   0.3321002485772985
0.8191220407792643      0.3235967227975155    0.33220514385432887
0.819282526599355       0.3236193370180684    0.33230187033759284
0.8194564982836626      0.32364242978047947   0.33236967212286067
0.819618951664826       0.3236626599586001    0.3323832076868037
0.8197782214652988      0.32368124496971756   0.33235925811108236
0.8199509771299885      0.32370000766883383   0.3323227340968359
0.820112214491534       0.3237162105196045    0.3323115764773191
0.8202869377172964      0.32373234384030697   0.33234640097936224
0.8204584773623683      0.32374674351156674   0.3324234361395371
0.8206184987042959      0.3237588920612698    0.3325085901829343
0.8207920059104405      0.32377066562840284   0.3325801291100703
0.8209539948134409      0.32378031938159807   0.33260472288289517
0.8211128001357507      0.3237884554126179    0.33258774598941704
0.8212850913222774      0.3237959097180936    0.33254442353477526
0.8214458642056599      0.32380157944745985   0.33251108796061063
0.8216201229532593      0.32380632494935807   0.33251138032620137
0.8217911981201681      0.323809569823293     0.33255663196449564
0.8219507549839327      0.3238113359190331    0.33262440949807565
0.8221237977119142      0.3238118788142597    0.3326945626690475
0.8222853221367515      0.32381109916969447   0.33272928735547197
0.8224603324258057      0.323808855004451     0.33271984939992066
0.8226321591341694      0.3238052382308853    0.33267420730784214
0.8227924675393888      0.32380060343702405   0.3326255504398528
0.8229662618088253      0.3237942066646324    0.3325975759259542
0.8231285377751174      0.3237869474764755    0.3326108608786795
0.8232876301607189      0.3237786269152124    0.3326565352375951
0.8234602084105375      0.3237682557775052    0.3327184375478361
0.8236212683572117      0.32375731618292364   0.3327588095773499
0.8237958141681029      0.32374408888322287   0.3327605586259523
0.8239671763983035      0.32372971738558476   0.33271855715943877
0.8241270203253599      0.32371507678018      0.33265913683410975
0.8243003501166333      0.32369785619815766   0.3326041926443014
0.8244621616047624      0.32368051946691      0.3325857686218248
0.8246207895122009      0.32366234448089026   0.3326048452424943
0.8247929032838563      0.32364130599010826   0.33265227718950807
0.8249534987523676      0.32362044026699616   0.3326944547218401
0.8251275800850957      0.32359647807256275   0.33270788331602186
0.8252984778371333      0.32357159632383514   0.3326746770403627
0.8254578572860266      0.32354718155737716   0.33261113716454743
0.825630722599137       0.3235193831982622    0.332534668754362
0.825792069609103       0.3234922021524571    0.33248529996466075
0.825966902483286       0.3234614057662868    0.3324750588362106
0.8261385517767784      0.32342981332717835   0.33250392580461324
0.8262986827671266      0.32339913109284285   0.33254200751801
0.8264722996216917      0.3233645479575514    0.3325620308608463
0.8266343981731125      0.32333102470106273   0.33253922794482954
0.8267933131438427      0.32329700476582307   0.3324768829565806
0.82696571397879        0.3232588069396592    0.33238582351755885
0.8271265965105931      0.32322195162557255   0.33230969573287517
0.827300964906613       0.32318062481938914   0.33226590291257735
0.8274721497219424      0.3231386889829464    0.332269802527382
0.8276318162341275      0.3230983907506757    0.33229920531561535
0.8278049686105297      0.32305340001529836   0.3323253016231636
0.8279666026837875      0.3230101946245193    0.33231453784781084
0.8281250531763548      0.32296671100566504   0.3322590965066657
0.828296989533139       0.3229182645116034    0.33216056968409646
0.828457407586779       0.32287188174470655   0.3320617913011202
0.8286313115046359      0.322820313961813     0.33198370634076013
0.8287936971193485      0.3227709567728359    0.3319572527893827
0.8289528991533706      0.3227214401224251    0.3319698775740014
0.8291255870516097      0.32266647005501387   0.3319975126998332
0.8292867566467045      0.32261398726565393   0.3320011043653132
0.8294614121060162      0.32255583054521486   0.3319530527857082
0.8296328839846374      0.32249743923895124   0.3318531069284634
0.8297928375601143      0.32244181656851467   0.3317376015701031
0.8299662769998082      0.32238024883956845   0.33162849732792765
0.8301281981363577      0.32232159360672996   0.33157123443776193
0.8302869356922168      0.32226299153922816   0.3315625997634021
0.8304591591122928      0.32219818176526144   0.33158407870331325
0.8306198642292246      0.3221365554456614    0.331596078151764
0.8307940552103733      0.3220685056301832    0.33156221179699075
0.8309650626108315      0.32200043558461755   0.3314678545820336
0.8311245517081454      0.32193582451678376   0.33134196766748425
0.8312975266696762      0.32186452467300697   0.3312048672950172
0.8314589833280627      0.32179682468246174   0.33111450233393486
0.8316339258506662      0.32172222181877386   0.33107774081159863
0.8318056847925792      0.32164771664845243   0.33108888321522123
0.831965925431348       0.32157708498870724   0.33110417450534635
0.8321396519343336      0.3214992873294967    0.3310807761565892
0.832301860134175       0.32142550312355306   0.3309982279142007
0.8324608847533258      0.32135209652701      0.33086780867523236
0.8326333952366937      0.32127126908150727   0.3307084360372087
0.8327943874169172      0.3211947185512625    0.3305869188183519
0.8329688654613577      0.3211105377948278    0.33051799995234865
0.8331401599251077      0.32102666374833766   0.3305116537099069
0.8332999360857134      0.32094733444513185   0.3305276456235433
0.833473198110536       0.3208601104432958    0.3305182166217277
0.8336349418322143      0.32077751513726493   0.33044940263842565
0.8337935019732021      0.3206955019378366    0.3303215303832329
0.8339655479784069      0.3206053476431202    0.3301459359205341
0.8341260756804675      0.32052013783746486   0.3299944411087675
0.8343000892467449      0.3204265829173519    0.3298890321313119
0.8344625845098781      0.3203381089317913    0.3298578888239593
0.8346218961923207      0.32025032885698596   0.32986762753326765
0.8347946937389803      0.3201539577103519    0.3298722138377564
0.8349559729824956      0.32006292389628554   0.329824631961285
0.8351307380902279      0.3199630969523628    0.3296946443201067
0.8353023196172696      0.31986389672233356   0.329509256482258
0.8354623828411671      0.3197702942126455    0.3293324421602922
0.8356359319292815      0.3196676507130255    0.3291913555646865
0.8357979627142517      0.31957073815327125   0.32913240071762534
0.8359568099185313      0.31947471877996664   0.3291299676387101
0.8361291429870278      0.3193694186272068    0.32914061886406115
0.83628995775238        0.319270099670537     0.3291095249957332
0.8364642583819493      0.3191613029428228    0.32899438176616463
0.836635375430828       0.3190533340569639    0.32880694973118807
0.8367949741765625      0.3189516004943323    0.3286092178260328
0.8369680587865138      0.31884014757170553   0.3284315515259801
0.8371296250933209      0.31873505998961443   0.3283390664069522
0.8372880078194375      0.3186310603174088    0.32831712449963973
0.837459876409771       0.31851710761745655   0.32832828666532443
0.8376202266969603      0.3184097644181079    0.32831375892236914
0.8377940628483665      0.3182922762282327    0.3282195848404952
0.8379563806966285      0.31818152645889997   0.3280494928342853
0.8381155149641999      0.3180719707334672    0.3278384116727185
0.8382881350959882      0.31795203878780426   0.32762331503976094
0.8384492369246322      0.31783908729055643   0.32748840599124035
0.8386238246174933      0.31771556961885966   0.3274332863817047
0.8387952287296638      0.3175931833857101    0.3274383365681562
0.8389551145386901      0.3174780232833992    0.3274359215547964
0.8391284862119333      0.31735206419156664   0.3273626918525912
0.8392903395820321      0.31723345695111077   0.32720452274514716
0.8394490093714405      0.317116232623749     0.326987368258729
0.8396211650250658      0.3169879841925854    0.32674427712621973
0.8397818023755468      0.31686731273982743   0.32657247347743246
0.8399559255902448      0.3167354124253416    0.3264826870117518
0.8401268652242523      0.3166048350010967    0.3264738198020192
0.8402862865551155      0.31648208713850867   0.32647990368239554
0.8404591937501956      0.3163479017524509    0.3264301479411972
0.8406205826421315      0.3162216686713131    0.3262913595159372
0.8407954573982843      0.3160838155705419    0.326051332114116
0.8409671485737465      0.3159473909897838    0.3257871264630057
0.8411273214460645      0.31581915608437494   0.32558517650569757
0.8413009801825995      0.3156790777839295    0.3254642837355117
0.8414631206159902      0.3155473106709673    0.3254389882589116
0.8416220774686903      0.3154172151371819    0.3254473772668603
0.8417945201856074      0.31527506037721326   0.32541739601632413
0.8419554445993802      0.31514144451345205   0.3252995074651027
0.84212985487737        0.3149955921288765    0.3250674632081178
0.8423010815746693      0.3148513539263894    0.32478681215335564
0.8424607899688243      0.3147158855599505    0.3245512739633723
0.8426339842271963      0.31456796373618423   0.3243890565685751
0.8427956601824239      0.31442892997189176   0.32433747562097626
0.842954152556961       0.31429174687133743   0.32434168992305973
0.8431261307957151      0.3141419007656547    0.32432992226204527
0.8432865907313248      0.3140011641688798    0.32423769856001944
0.8434605365311516      0.31384759243468835   0.32402310686124475
0.8436229640278341      0.3137032472108177    0.32374930009314723
0.8437822079438261      0.3135608502665753    0.3234812050115124
0.843954937724035       0.31340541133399835   0.32326840906731386
0.8441161492010996      0.3132594179873016    0.3231768134720854
0.8442908465423812      0.31310021274002725   0.3231665917021579
0.8444623603029722      0.31294290124847723   0.3231667292698695
0.8446223557604189      0.3127952574241514    0.3230981017922817
0.8447958370820826      0.31263419396589615   0.3229044856930544
0.8449578001006021      0.31248291189050176   0.322631087388974
0.845116579538431       0.31233375103696337   0.32234069792128084
0.8452888448404768      0.3121709699753117    0.3220866487916433
0.8454495918393784      0.3120181831186953    0.32195723561113915
0.845623824702497       0.31185161129880706   0.3219253950695859
0.8457948739849249      0.3116871080604506    0.32193216424152654
0.8459544049642086      0.3115328147314485    0.32188829222099824
0.8461274218077093      0.31136453508422013   0.32172324034046396
0.8462889203480657      0.31120657580006805   0.3214597363479219
0.846463904752639       0.31103446774247967   0.3211222577756154
0.8466357055765218      0.31086452457781555   0.32083629887696674
0.8467959880972603      0.31070511491717745   0.3206752326129439
0.8469697564822158      0.3105313580389887    0.32062182609470746
0.8471320065640271      0.31036824383006006   0.3206283508438632
0.8472910730651478      0.3102075128794431    0.320604153547444
0.8474636254304855      0.31003224319549494   0.32046738365293936
0.8476246594926788      0.30986782029378673   0.320218884897896
0.8477991794190891      0.3096887013500942    0.31987152698628063
0.8479705157648089      0.30951191592204513   0.31955044321078424
0.8481303338073843      0.3093461840445638    0.3193475379919554
0.8483036377141768      0.3091655640586755    0.31925974845064486
0.848465423317825       0.30899610353052503   0.3192587849104993
0.8486240253407827      0.30882918836718004   0.31925167559386974
0.8487961132279573      0.30864719980169114   0.3191474998274581
0.8489566828119877      0.30847656860705525   0.3189228844538769
0.849130738260235       0.308290711814184     0.31857553682345113
0.8493016101277917      0.3081073525102062    0.31822389343568647
0.8494609636922041      0.3079355510220354    0.3179763904027995
0.8496338031208336      0.30774833819858044   0.3178450807811691
0.8497951242463188      0.307572785864721     0.31782695916860015
0.8499699312360208      0.30738167234369856   0.31782949581713316
0.8501415546450324      0.30719314667059977   0.31774789446744073
0.8503016597508997      0.30701647930752907   0.3175423009561818
0.850475250720984       0.30682406798604356   0.31719785660195554
0.850637323387924       0.30664361632809123   0.31684257942693905
0.8507962124741734      0.30646595525995446   0.3165554781193415
0.8509685874246398      0.3062723740042298    0.3163795988955538
0.8511294440719619      0.30609094160959105   0.316337700008611
0.8513037865835009      0.305893444282703     0.3163451060228282
0.8514749455143494      0.305698692569048     0.316293276474069
0.8516345861420537      0.3055162812193323    0.31611980474294515
0.8518077126339749      0.305317628488957     0.3157919750136899
0.8519693208227519      0.3051314142456719    0.3154228635950607
0.8521277454308384      0.30494814189670677   0.3150975810656253
0.8522996559031417      0.3047484650201669    0.31487100897370685
0.8524600480723008      0.3045614268243292    0.3147954376879436
0.8526339261056768      0.3043578393914815    0.3147983881048358
0.8527962858359085      0.3041669683547691    0.31477758417341967
0.8529554619854497      0.30397912090255397   0.3146488443780785
0.8531281239992079      0.30377455664451397   0.3143562611004868
0.8532892677098218      0.30358288905647596   0.31398618311231585
0.8534638972846527      0.30337436703342235   0.31359040491214357
0.853635343278793       0.30316882604355677   0.3133164207717449
0.853795270969789       0.30297636414784407   0.31320377246431613
0.8539686845250021      0.3027668801178829    0.3131934463519099
0.8541305797770709      0.30257056866708215   0.3131898497632845
0.854289291448449       0.30237742525316635   0.3130949271357385
0.8544614889840442      0.3021670981080938    0.3128351689412105
0.8546221682164951      0.3019701178642264    0.31247228668339894
0.8547963333131628      0.3017558211674598    0.3120501568969798
0.8549673148291401      0.3015446517474415    0.31172676873723315
0.8551267780419731      0.30134700558149136   0.3115688202412747
0.8552997271190231      0.3011318813391321    0.3115341575711194
0.8554611578929289      0.3009303707315905    0.3115411610135045
0.8556360745310515      0.30071125166114904   0.3114671813217402
0.8558078075884836      0.30049534049305765   0.3112322711127483
0.8559680223427715      0.300293216693076     0.31087745642643544
0.8561417229612762      0.3000733257587986    0.31043871701577497
0.8563039052766367      0.2998673112545159    0.310093244488744
0.8564629040113066      0.2996646828100928    0.3098906067506409
0.8566353886101936      0.2994441344392458    0.3098253251453359
0.8567963549059363      0.2992376283543055    0.3098337156720729
0.8569708070658959      0.29901307685872924   0.30979063021865333
0.857142075645165       0.2987918734787978    0.3095954260873068
0.8573018259212898      0.2985848800523824    0.3092627400151074
0.8574750620616315      0.29835968861333967   0.3088146799398553
0.8576367798988289      0.29814879308836817   0.3084305139179413
0.8577953141553358      0.29794141763379783   0.308178069598012
0.8579673342760598      0.29771569729348446   0.3080714279096512
0.8581278360936394      0.29750443282309585   0.3080717220697912
0.858301823775436       0.29727470284713986   0.3080557465353338
0.8584642931540882      0.2970595133916137    0.3079170423438921
0.85862357895205        0.29684794495052486   0.3076248716565813
0.8587963506142288      0.2966177855831362    0.3071830131832512
0.8589576039732633      0.2964023207470704    0.3067627072536752
0.8591323431965147      0.2961681311982311    0.30642587494833445
0.8593038988390755      0.2959374980513856    0.3062754965723661
0.8594639361784921      0.29572171799784397   0.3062597443163905
0.8596374593821257      0.29548706885909626   0.3062608381226407
0.8597994642826149      0.2952673541076997    0.30615969803901016
0.8599582856024136      0.295051358378005     0.3059033245382579
0.8601305927864293      0.2948163545950815    0.3054751698297181
0.8602913816673008      0.29459643649033385   0.3050339906102607
0.8604656564123891      0.2943573962041475    0.3046466365264108
0.860636747576787       0.2941220406838258    0.30444391851097435
0.8607963204380406      0.2939019237368834    0.30440150026932333
0.860969379163511       0.2936625459060234    0.30441170802395506
0.8611309195858372      0.2934384850453143    0.3043483552042971
0.8613059458723804      0.2931950514136226    0.3041035306300109
0.8614777885782331      0.29295537334511007   0.3036882103251206
0.8616381129809415      0.29273116262438803   0.30323501000610836
0.8618119232478668      0.29248744326170023   0.3028119698429244
0.8619742152116479      0.2922592683753152    0.30257570030840286
0.8621333235947384      0.2920350033408014    0.30250156916832044
0.8623059178420458      0.29179109898838923   0.3025107911361716
0.862466993786209       0.29156288242968925   0.30247673722636326
0.8626415555945891      0.29131491924071756   0.3022766655055745
0.8628129338222787      0.2910708342995026    0.3018903325623951
0.8629727937468241      0.29084258186812284   0.3014326885476668
0.8631461395355864      0.29059445259951133   0.3009695300668241
0.8633079670212044      0.29036223007472      0.30068033937584904
0.8634666109261319      0.2901340342972352    0.30056450884901875
0.8636387406952764      0.28988583652945205   0.3005621038264917
0.8637993521612766      0.28965368331683583   0.300553114600816
0.8639734494914937      0.28940142534067526   0.30040235290745465
0.8641443632410203      0.28915316416342357   0.3000565293950726
0.8643037586874026      0.2889210866393884    0.2996059594921807
0.8644766399980018      0.2886687797650528    0.29910927073844834
0.8646380030054568      0.288432727905776     0.2987643485511216
0.8648128518771288      0.28817635548296405   0.2985885351606193
0.8649845171681102      0.28792409693875826   0.2985711630162669
0.8651446641559474      0.28768822556738066   0.29857498990807113
0.8653182970080014      0.28743190685186554   0.2984579241863038
0.8654804115569112      0.28719204493265765   0.2981639223243825
0.8656393425251305      0.2869563833322984    0.29772866063706704
0.8658117593575667      0.286700156582719     0.2972104306853459
0.8659726578868586      0.2864605159684194    0.29681773761890207
0.8661470422803675      0.2862002134631437    0.2965860146786937
0.866318243093186       0.28594408286545836   0.29653908424519626
0.8664779256028601      0.2857046689901791    0.2965521395477888
0.8666510939767511      0.28544447587100236   0.2964782469023673
0.8668127440474979      0.2852010664624857    0.29623198600983325
0.8669712105375541      0.2849619629668659    0.29582393792175116
0.8671431628918272      0.284701967450176     0.29529466264111487
0.8673035969429561      0.2844588799704376    0.2948564581480197
0.867477516858302       0.2841948078602734    0.2945614975075311
0.8676399184705036      0.28394770962498284   0.29447279952091604
0.8677991365020147      0.28370497486900276   0.29448259761626516
0.8679718403977427      0.2834411452140177    0.2944540833053245
0.8681330259903264      0.28319441152644437   0.2942716744122018
0.8683076974471271      0.28292649239268236   0.2938654867595353
0.8684791853232373      0.28266291073724387   0.29333622148442917
0.8686391548962032      0.2824165482034294    0.2928632121573763
0.8688126103333861      0.282148890591502     0.29251043528620485
0.8689745474674246      0.2818985153063224    0.2923765779340652
0.8691333010207727      0.28165260393856206   0.29237238200526067
0.8693055404383376      0.28138529224654774   0.2923693523887897
0.8694662615527583      0.28113537996602517   0.29223485418819345
0.869640468531396       0.2808639809421133    0.2918759513498407
0.8698114919293432      0.2805970213400394    0.29135927509248755
0.869970997024146       0.28034757919355346   0.29085821466415535
0.8701439879831658      0.2800765457831106    0.29044536663890164
0.8703054606390414      0.2798230904210293    0.29025623720135196
0.8704804191591339      0.27954795949978145   0.29022725317072706
0.8706521940985358      0.2792773242758401    0.29023656035963136
0.8708124507347934      0.2790243827526081    0.290133943847256
0.870986193235268       0.2787496638304244    0.2898109455669849
0.8711484174325984      0.2784927364748139    0.2893355441877482
0.8713074580492383      0.27824044935128117   0.2888177906393924
0.871479984530095       0.27796629217112545   0.28835452920723553
0.8716409927078075      0.2777099925317354    0.28811155573317015
0.871815486749737       0.27743174195509757   0.2880477076752867
0.8719867972109758      0.2771580811752721    0.28806496350815153
0.8721465893690704      0.2769023878792326    0.2880029797001019
0.872319867391382       0.2766246457040822    0.287734523826057
0.8724816271105493      0.27636492743424235   0.28729019544249546
0.8726402032490261      0.2761099118299501    0.286765748397176
0.8728122652517198      0.2758327533187514    0.28625501373971385
0.8729728089512693      0.27557372324381574   0.2859519002203609
0.8731468385150357      0.2752924730541955    0.2858400594108248
0.8733093497756579      0.27502940668030607   0.2858523720035857
0.8734686774555895      0.27477109182292736   0.28583259029441366
0.8736414909997381      0.27449046509263664   0.28563599491821823
0.8738027862407424      0.27422812468559765   0.285245152616694
0.8739775673459635      0.2739433970632456    0.2846739525935146
0.8741491648704942      0.2736633998463279    0.28412774470634433
0.8743092440918806      0.27340179225041067   0.28377067069091655
0.874482809177484       0.27311770640002564   0.2836080312945971
0.8746448559599431      0.2728520632098084    0.2836052203248902
0.8748037191617117      0.2725912565290907    0.28360849896513174
0.8749760682276972      0.27230788429938313   0.2834647490812012
0.8751368989905385      0.27204305278620894   0.28312137609233357
0.8753112156175967      0.27175558405060896   0.2825685545191034
0.8754823486639643      0.2714729320511549    0.281994301762186
0.8756419634071877      0.27120891952356435   0.28158154476163694
0.875815064014628       0.270922183277407     0.28135757091865465
0.8759766463189241      0.2706541372202185    0.2813269089419221
0.8761350450425297      0.27039100935904525   0.28134455851812834
0.8763069296303522      0.2701050749131059    0.2812537707649134
0.8764672959150304      0.26983792429638953   0.28096724959588204
0.8766411480639256      0.2695478990615398    0.2804478887271914
0.8768034819096765      0.269276707308229     0.2798850372433137
0.8769626321747368      0.2690104782868255    0.27941074320686365
0.8771352683040141      0.26872129409546774   0.2791044902253989
0.8772963861301472      0.268451034958489     0.2790231971740014
0.8774709898204971      0.26815781746605016   0.2790432231522325
0.8776424099301565      0.26786955327912043   0.2789938009542928
0.8778023117366717      0.26760030093465753   0.27876046592072445
0.8779756994074038      0.2673079523546595    0.2782811062974534
0.8781375687749917      0.26703466245874086   0.277716621570974
0.8782962545618891      0.26676640977340765   0.2772023185275318
0.8784684262130034      0.2664749835649389    0.27683109189959537
0.8786290795609734      0.26620270267445156   0.2767005496536051
0.8788032187731603      0.26590718480280806   0.2767085864028017
0.8789741744046566      0.2656166865305662    0.27669527562270485
0.8791336117330087      0.2653454208482273    0.27651981820736476
0.8793065349255779      0.26505084156682884   0.2760936980521182
0.8794679398150027      0.2647755396890545    0.27554113700641925
0.8796428305686445      0.26447686242554286   0.2749426191322879
0.8798145377415957      0.26418324677973826   0.27452557080769563
0.8799747266114026      0.263908993926152     0.27435561361516786
0.8801484013454266      0.2636112911779316    0.27434735685293404
0.8803105577763063      0.2633329950857026    0.27435557042515485
0.8804695306264954      0.26305984836124174   0.274229661836464
0.8806419893409014      0.2627631803822999    0.27385742128734486
0.8808029297521632      0.26248599999289657   0.27332650134953806
0.8809773560276419      0.2621852397631254    0.272704636885351
0.8811485987224301      0.2618896131852424    0.27222877803369194
0.881308323114074       0.261613555622245     0.27200038707072355
0.881481533369935       0.26131384767335986   0.2719589369306079
0.8816432253226516      0.26103375057719597   0.2719812075134724
0.8818017336946776      0.2607588708310438    0.2719044760169514
0.8819737279309207      0.2604602732021774    0.2715956194415293
0.8821342038640194      0.26018136350685356   0.27110007105176925
0.8823081656613351      0.2598786805184342    0.27046700039784843
0.8824706091555065      0.25959572635732703   0.2699560463823925
0.8826298690689873      0.259318026612586     0.2696497497651101
0.8828026148466852      0.2590164881277714    0.26955021037573507
0.8829638423212388      0.25873475371812416   0.2695723895639621
0.8831385556600093      0.25842912684221275   0.26953578028747494
0.8833100854180893      0.25812874055745966   0.26928523661109366
0.883470096873025       0.2578482339421453    0.26882999339266916
0.8836435941921776      0.2575437927553557    0.2681989878570952
0.883805573208186       0.2572593110944905    0.2676480242402523
0.8839643686435038      0.2569801462087352    0.267281595218283
0.8841366499430386      0.25667696737451595   0.26712748042816326
0.8842974129394292      0.2563937731642287    0.2671360337734752
0.8844716618000367      0.25608651372951874   0.26713350109560785
0.8846427270799536      0.25578455722863347   0.26694594234595953
0.8848022740567263      0.25550265648175885   0.2665428789903766
0.884975306897716       0.2551966287372308    0.2659282855910634
0.8851368214355614      0.2549106932897596    0.26534484988908136
0.8853118218376237      0.25460058103313654   0.2648832357756214
0.8854836386589955      0.2542958063811269    0.26468679993364774
0.885643937177223       0.2540111935048925    0.2646790495715685
0.8858177215596674      0.253702343893673     0.26469387984145515
0.8859799876389676      0.25341369176808304   0.26456143615064615
0.8861390701375772      0.2531304486863037    0.26421011131371264
0.8863116385004038      0.2528229115565298    0.2636214533115166
0.8864726885600862      0.25253563765110104   0.2630191482058833
0.8866472244839855      0.2522240226481979    0.2624994669622999
0.8868185768271942      0.25191780449191326   0.2622398655135888
0.8869784108672587      0.2516319156183283    0.2621998225148657
0.8871517307715401      0.25132162914156614   0.2622280487190213
0.8873135323726772      0.25103170590898144   0.26214678033555405
0.8874721503931238      0.25074724746540544   0.2618557036853892
0.8876442542777874      0.25043833743890465   0.2613073368208528
0.8878048398593067      0.2501498530705999    0.2606973149385697
0.8879789113050429      0.2498368728100224    0.26012241212443465
0.8881497991700886      0.24952934613985855   0.25979131396129074
0.8883091687319901      0.24924230779261664   0.2597051764760071
0.8884820241581084      0.24893072044056305   0.25973335867063807
0.8886433612810825      0.24863965363830834   0.2596991624051033
0.8888181842682735      0.2483239950139912    0.25943772542778565
0.8889898236747741      0.24801382134875483   0.25892328211816557
0.8891499447781304      0.24772422921336273   0.2583148132231812
0.8893235517457035      0.24740999402470787   0.25770577051840976
0.8894856404101324      0.2471163717423423    0.2573353909551183
0.8896445454938708      0.24682829719141533   0.25719961627468474
0.8898169364418262      0.24651553075327687   0.257214735700462
0.8899778090866374      0.24622347600582414   0.25721352016435256
0.8901521675956654      0.24590670552632857   0.2570170590977478
0.8903233425240029      0.2455954717277026    0.2565583939566885
0.8904829991491962      0.2453049614947507    0.2559646212295545
0.8906561416386063      0.24498967489129864   0.2553200030526021
0.8908177658248722      0.2446951411326744    0.2548849698437533
0.8909762064304475      0.24440620263007987   0.25468828998939874
0.8911481329002399      0.24409244126133395   0.25467530143601497
0.891308541066888       0.243799486536811     0.25469799002024673
0.8914824350977529      0.24348167078066077   0.2545678749259556
0.8916448108254736      0.24318469023007294   0.254202352835761
0.8918040029725038      0.2428933311550674    0.2536459079507586
0.8919766809837509      0.24257706610965335   0.2529759004668755
0.8921378406918538      0.24228168867556885   0.25246695131476293
0.8923124862641736      0.24196136838459298   0.25217087034542757
0.8924839482558029      0.24164666050210065   0.2521212933474083
0.892643891944288       0.2413528927994922    0.2521539720118809
0.8928173214969899      0.24103413811325977   0.25207717255999107
0.8929792327465476      0.240736350655462     0.25177390843059577
0.8931379604154147      0.24044422972013102   0.25125748815468685
0.8933101739484988      0.24012707974595107   0.2505826229937471
0.8934708691784385      0.23983094652948145   0.25002498254181277
0.8936450502725953      0.23950974975824213   0.2496560571150345
0.8938160477860616      0.23919421130608726   0.24955555046605812
0.8939755269963836      0.23889973924382782   0.2495859096437366
0.8941484920709225      0.23858016243565638   0.24955775199449895
0.8943099388423171      0.23828167756510868   0.24932324703343364
0.8944848714779288      0.23795805474785933   0.24880096792914141
0.8946566205328499      0.2376401156459469    0.24812994339779954
0.8948168512846267      0.23734331662775796   0.2475431578622472
0.8949905679006205      0.23702134012935852   0.24712296882002954
0.89515276621347        0.23672052851197256   0.24698237895523403
0.8953117809456289      0.23642545022141437   0.2469990842293931
0.8954842815420048      0.23610515687684588   0.24700451352726271
0.8956452638352363      0.23580607363274733   0.24683007550289274
0.8958197319926849      0.23548174450182      0.2463672482894844
0.895991016569443       0.23516314192693816   0.24571420219915516
0.8961507828430568      0.23486580208790556   0.24509691000210232
0.8963240349808875      0.2345432135417681    0.2446088024694051
0.896485768815574       0.2342419006695456    0.2444063931618815
0.8966443190695699      0.23394635987559292   0.2443958968962554
0.8968163551877827      0.23362550449156483   0.24442536171449936
0.8969768730028512      0.23332596675043601   0.24431196496710073
0.8971508766821368      0.23300108547095746   0.24392032073666087
0.897313362058278       0.23269754414595717   0.24333461488033895
0.8974726638537288      0.23239979591892812   0.2426961905182955
0.8976454515133965      0.23207667029574952   0.24213023897182706
0.8978067208699199      0.23177492532530392   0.24184268760413258
0.8979814760906603      0.2314477751743527    0.24178243505612954
0.89815304773071        0.23112641199122005   0.24182152768667287
0.8983131010676155      0.2308264701579404    0.24175678976527987
0.898486640268738       0.23050109015096887   0.2414320048346938
0.8986486611667163      0.23019715246493702   0.2408871350880033
0.898807498484004       0.22989904370394476   0.2402459514254988
0.8989798216655086      0.22957546546587448   0.2396295045166462
0.899140626543869       0.22927336769840406   0.23927508763452804
0.8993149172864463      0.2289457747696256    0.2391610641211024
0.8994860244483331      0.22862400540066494   0.23919706598922535
0.8996456133070757      0.2283237545974569    0.23917677501570409
0.8998186880300352      0.22799797824054951   0.23892555254693443
0.8999802444498504      0.22769374009065546   0.238435309475222
0.900138617288975       0.22739536479810638   0.23780525531805116
0.9003104759923166      0.22707143518219267   0.23714574449125034
0.9004708163925139      0.22676907939485236   0.23672029220871735
0.9006446426569282      0.2264411456831442    0.23653799020406854
0.9008069506181983      0.22613480477342895   0.2365520771104289
0.9009660749987778      0.22583434522566959   0.236572195740054
0.9011386852435742      0.22550828042719495   0.23641057561633513
0.9012997771852264      0.2252038427558218    0.2360013408045274
0.9014743549910955      0.22487377741794418   0.23533738688018224
0.901645749216274       0.22454958934822872   0.23465095782179182
0.9018056251383083      0.22424706257904892   0.2341655256218129
0.9019789869245596      0.22391888182414427   0.2339154747810898
0.9021408304076666      0.22361238001463313   0.23389885284561598
0.902299490310083       0.22331179077803143   0.233936369284733
0.9024716360767164      0.22298553275744193   0.23383706757937966
0.9026322635402055      0.2226809942368934    0.23349550878162775
0.9028063768679117      0.2223507580441612    0.23287355675594276
0.9029773066149271      0.2220264304979907    0.23217209438729933
0.9031367180587984      0.22172384369371798   0.23162833570073899
0.9033096153668866      0.22139553516708566   0.23130069395983852
0.9034709943718305      0.22108898358812273   0.23123801367006164
0.9036458592409914      0.22075669091731437   0.23128294162403754
0.9038175405294617      0.22043032420848327   0.23122254400025802
0.9039777035147878      0.22012574454261163   0.2309297121664283
0.9041513523643309      0.21979540116761964   0.23034433340274182
0.9043134829107297      0.21948686039036022   0.2296750764451485
0.9044724298764379      0.21918427669877466   0.22908705564510717
0.9046448627063631      0.21885590806229308   0.2286889893287725
0.904805777233144       0.21854936993760976   0.22857537423181962
0.904980177624142       0.21821702945585236   0.2286128354301496
0.9051513944344493      0.21789064359459231   0.22859825825110983
0.9053110929416123      0.21758611591461807   0.22837323087838157
0.9054842773129923      0.2172557656843622    0.22784782141477808
0.9056459433812281      0.2169472879504835    0.22719046478422225
0.9058044258687733      0.21664479348139556   0.2265649385798917
0.9059763942205354      0.2163164575872439    0.2260922959817292
0.9061368442691533      0.21601001974719453   0.22591491137666042
0.9063107801819882      0.21567772501132002   0.22592869737355853
0.9064731977916788      0.21536734199733742   0.22595302825474303
0.9066324318206788      0.215062956527914     0.22581055768876718
0.9068051517138958      0.21473269668360406   0.22537314660848792
0.9069663533039685      0.21442437286241695   0.2247511817636793
0.9071410407582581      0.21409016034138378   0.22403257523652562
0.9073125446318572      0.2137619441005828    0.2234975162116936
0.907472530202312       0.2134556878481847    0.22325785388788946
0.9076460016369838      0.21312352654984706   0.2232375492065628
0.9078079547685114      0.21281333766920915   0.2232799458568789
0.9079667243193483      0.21250916994153257   0.2231942376917973
0.9081389797344023      0.21217908204716765   0.2228290856477181
0.9082997168463119      0.21187098851778732   0.2222477727816498
0.9084739398224385      0.21153696243604345   0.22151815816504145
0.9086449792178746      0.21120895694870165   0.22092229828331392
0.9088045003101664      0.2109029635177525    0.2206116094089501
0.9089775072666751      0.210571023065524     0.2205414072470978
0.9091389959200397      0.21026111006666495   0.2205891608737657
0.9093139704376211      0.2099252390782039    0.2205417526030252
0.909485761374512       0.20959540236018775   0.2202276491309398
0.9096460340082586      0.20928761336380106   0.21968060761094788
0.9098197925062222      0.20895385398785796   0.21895187152244802
0.9099820327010416      0.20864215288366675   0.21834494072865904
0.9101410893151703      0.20833650716452312   0.2179703976935468
0.9103136317935161      0.20800487981729768   0.21784606084005384
0.9104746559687176      0.20769532906584723   0.217885900000876
0.910649166008136       0.20735978748332776   0.21788239786819646
0.9108204924668638      0.20703030202320605   0.21764010020544328
0.9109803006224474      0.2067229109924722    0.21714896134040199
0.911153594642248       0.206389519172808     0.21643431413882372
0.9113153703589043      0.20607823108859322   0.21578889711819904
0.91147396249487        0.2057730177532172    0.2153466280906402
0.9116460404950527      0.2054417947484808    0.2151548557435086
0.9118066001920911      0.20513269143904933   0.21517235459327544
0.9119806457533465      0.20479757120479092   0.21520367005391747
0.9121515077339113      0.20446852716522146   0.21503626760079678
0.9123108514113318      0.20416161823961904   0.21461360662368623
0.9124836809529694      0.20382868484355454   0.21393017078168855
0.9126449921914627      0.2035178946336065    0.2132568855813439
0.9128197892941728      0.20318107391149176   0.21270666039476657
0.9129914028161925      0.20285034058030169   0.21246414864161955
0.9131514980350679      0.20254176447570812   0.21245821896165934
0.9133250791181602      0.20220715184020813   0.21250555391681497
0.9134871418981082      0.2018947038061477    0.21239981249783646
0.9136460210973657      0.20158835792704036   0.21204298763186138
0.9138183861608402      0.20125597048089947   0.2114005253407283
0.9139792329211704      0.2009457598984488    0.21071556731493152
0.9141535655457176      0.20060950365258307   0.21010427360066863
0.9143247145895742      0.2002793522360308    0.209787594423767
0.9144843453302866      0.19997138931108063   0.2097385330236492
0.9146574619352159      0.19963737711351592   0.20979458285843894
0.914819060237001       0.19932555954972897   0.2097458695032594
0.9149774749580954      0.19901984859510732   0.2094628734289194
0.9151493755434069      0.19868806515508972   0.20887745051367967
0.9153097578255741      0.19837848851546297   0.20819486809944066
0.9154836259719582      0.19804285564150953   0.2075281742678464
0.915645975815198       0.19772943523976416   0.20714208624126518
0.9158051420777473      0.197422141210406     0.20702282522416024
0.9159777942045136      0.19708879023630255   0.20706725197711612
0.9161389280281357      0.19677766055533794   0.20707551146904077
0.9163135477159746      0.19644047339200077   0.20684928982349796
0.916484983823123       0.19610941707319868   0.2063240252179582
0.9166449016271272      0.19580059005619813   0.2056568787215522
0.9168183052953482      0.1954657065270765    0.20495220918946297
0.916980190660425       0.19515305662058646   0.20449849692833139
0.9171388924448113      0.19484654621601907   0.20431736726131772
0.9173110800934146      0.19451398115912943   0.20433666955176763
0.9174717494388736      0.19420365597900346   0.2043749927577813
0.9176459046485494      0.1938672777229758    0.20422277389055454
0.9178168762775348      0.19353704420961595   0.20377059593433455
0.9179763296033759      0.1932290559590994    0.20313410184603653
0.918149268793434       0.19289501822162697   0.20240244039120006
0.9183106896803477      0.19258322872798878   0.20188042128044864
0.9184855964314784      0.19224539270113455   0.20161452586478018
0.9186573196019187      0.191913709248172     0.20160893767092622
0.9188175244692146      0.19160427785859943   0.20166075111839488
0.9189912152007275      0.19126880525619458   0.2015565398910056
0.919153387629096       0.19095558690502504   0.20118538470511782
0.9193123764767741      0.19064852551919265   0.2005875870155016
0.9194848511886692      0.19031542902630294   0.19984546206877318
0.91964580759742        0.1900045889546091    0.19926938590804888
0.9198202498703877      0.1896677188412322    0.19892921508648984
0.9199915085626649      0.18933701229061348   0.19887790527244972
0.9201512489517978      0.18902856336078053   0.19893676519251932
0.9203244752051476      0.18869409232539203   0.19889174861295697
0.9204861831553531      0.18838187976016302   0.19859495800713745
0.9206447075248682      0.18807583247215048   0.19805012664140476
0.9208167177586002      0.18774377172987602   0.19731186786735969
0.920977209689188       0.18743396908509374   0.19668676795928683
0.9211511874839926      0.18709816014536454   0.196265318922233
0.921313646975653       0.18678456613150854   0.19615247911236391
0.9214729228866229      0.1864771314380298    0.1962001452207269
0.9216456846618096      0.18614369667367597   0.1962147047209464
0.9218069281338521      0.18583252265540817   0.19601169086277506
0.9219816574701116      0.18549535771597675   0.1954851692663834
0.9221532032256806      0.18516437280826545   0.1947638514891705
0.9223132306781053      0.1848556461677204    0.19410546522154004
0.922486743994747       0.18452094190342733   0.19361387687369846
0.9226487390082443      0.1842084949228541    0.19344020490391522
0.9228075504410511      0.18390222602546893   0.19346519358819045
0.9229798477380748      0.18356999354040457   0.19351117513083874
0.9231406267319543      0.18326001412156306   0.19337557976344272
0.9233148915900508      0.18292408270717547   0.19292351049230133
0.9234859728674567      0.18259433922612914   0.19223465800483155
0.9236455358417184      0.18228684330126796   0.19155253806477585
0.923818584680197       0.18195341170394042   0.1909898883275865
0.9239801152155313      0.18164222513096367   0.19074412108732283
0.9241551316150826      0.18130511613061526   0.19073726670774327
0.9243269644339434      0.18097419964392086   0.1907975120015703
0.9244872789496599      0.1806655207889157    0.1907051298759081
0.9246610793295933      0.18033093790636917   0.19030751302467333
0.9248233614063824      0.18001858916994903   0.18968537139680874
0.9249824599024811      0.17971242599444953   0.18899417948778505
0.9251550442627966      0.1793803778082983    0.18837546050536896
0.9253161103199679      0.17907055475953856   0.18806280093629096
0.9254906622413561      0.17873486238482553   0.1880092452852999
0.9256620305820539      0.17840536701246035   0.18807779762301813
0.9258218806196074      0.17809808635676272   0.1880396441285404
0.9259952165213777      0.17776495779716178   0.18772020339182663
0.9261570341200038      0.17745403892187392   0.18714966469716598
0.9263156681379394      0.17714930700256615   0.1864618458202931
0.9264877880200919      0.17681874913648166   0.18579176214686915
0.9266483895991001      0.1765103890081992    0.18540574593165535
0.9268224770423253      0.17617622103678668   0.1852901737098361
0.92699338090486        0.17584825234139465   0.1853521348152711
0.9271527664642505      0.17554246796024972   0.18536221027040337
0.9273256378878578      0.17521090015479465   0.18512706908733406
0.9274869910083209      0.17490149331637656   0.18462299374524171
0.927661829993001       0.17456625475329934   0.18388105271839064
0.9278334853969904      0.17423721896494648   0.1831829845199252
0.9279936224978356      0.1739303518911139    0.18274771928993672
0.9281672454628977      0.1735977424302881    0.18258375782799835
0.9283293501248157      0.17328729470894394   0.18262801334941844
0.928488271206043       0.17298303583438746   0.18267032428850558
0.9286606781514873      0.17265306289797333   0.18250740350982803
0.9288215667937874      0.17234523555160539   0.18207007014999144
0.9289959413003044      0.17201171749095798   0.1813592152358223
0.9291671322261309      0.17168440374237684   0.1806325273952914
0.9293268048488131      0.17137921760781652   0.18013175768239642
0.9294999633357123      0.17104837209972215   0.17989419360528283
0.9296616035194671      0.17073964537831907   0.17990649563584624
0.9298200601225315      0.17043710533640238   0.17997067433722747
0.9299920025898127      0.17010893776628588   0.17987979777972532
0.9301524267539497      0.16980286931576868   0.17951991343222692
0.9303263367823037      0.16947119958089893   0.17885762942023262
0.9304887285075134      0.16916161898463875   0.17815034660058712
0.9306479366520326      0.1688582239112548    0.1775753975348385
0.9308206306607687      0.16852926186305375   0.1772377757227718
0.9309818063663605      0.16822236704764829   0.17719064917511734
0.9311564679361694      0.16788993352194917   0.17726740303195454
0.9313279459252876      0.16756370289636194   0.177232956130883
0.9314879056112615      0.16725951562819588   0.176944286290171
0.9316613511614524      0.1669298271163295    0.1763372504447139
0.9318232784084991      0.1666221701304641    0.17563275839232434
0.9319820220748553      0.1663206924039873    0.17501249321585471
0.9321542516054283      0.16599375125934962   0.17459892722862339
0.9323149628328571      0.16568881622098916   0.17449545629789395
0.9324891599245029      0.16535844892192372   0.17456308915749894
0.932660173435458       0.16503427930977072   0.1745792636699335
0.9328196686432689      0.16473208833616254   0.17436735741863632
0.9329926497152968      0.16440450615261856   0.17383076370629563
0.9331541124841805      0.1640988889414461    0.1731443953262694
0.933329061117281       0.16376791374581565   0.17242666934602788
0.933500826169691       0.1634431336780899    0.17196321484581267
0.9336610729189567      0.16314028875514935   0.1718169959963323
0.9338348055324395      0.16281205893620962   0.17187003461848696
0.9339970198427779      0.1625057213700254    0.1719159917522006
0.9341560505724259      0.16220555182254087   0.17176964060474226
0.9343285671662908      0.1618801048328851    0.17130309041536476
0.9344895654570113      0.16157655528999396   0.17064610795815383
0.9346640496119488      0.16124776557558917   0.16990292550692845
0.9348353501861958      0.16092516607666896   0.16937175713905026
0.9349951324572985      0.16062443141154234   0.1691596633793157
0.9351684005926182      0.16029850573048277   0.16917963061114277
0.9353301504247936      0.15999442866872451   0.1692498913275021
0.9354887166762785      0.1596965094202765    0.16916932746654462
0.9356607687919803      0.15937344865120698   0.16878399251201387
0.9358213026045379      0.15907220211666898   0.1681717954108835
0.9359953222813124      0.15874585482661355   0.16741619487058118
0.9361578236549427      0.15844130413007124   0.16684182355717542
0.9363171414478824      0.1581429057167687    0.16653948010078073
0.936489945105039       0.1578194590867222    0.16649788613380773
0.9366512304590514      0.15751777189727906   0.1665783949103672
0.9368260016772807      0.1571910797514947    0.16655603463975235
0.9369975893148195      0.15687056441474498   0.16624510905103343
0.937157658649214       0.1565717687192378    0.1656831398296697
0.9373312138478255      0.15624802468508736   0.16493096561395618
0.9374932507432927      0.15594598041804264   0.16431052451003073
0.9376521040580693      0.15565007323861552   0.1639404839781355
0.9378244432370629      0.15532927450112252   0.16384139974758913
0.9379852641129123      0.15503013413773103   0.16391453865939506
0.9381595708529785      0.1547061489670716    0.16394173425266856
0.9383306940123542      0.15438832667655578   0.16371142504034675
0.9384902988685857      0.15409211862542865   0.1632132050745531
0.9386633895890342      0.15377112670931475   0.16247983956068884
0.9388249620063384      0.15347172694437083   0.16182055532237766
0.938983350842952       0.15317844590991483   0.16137922636822263
0.9391552255437826      0.15286044199592236   0.16120860484008714
0.9393155819414689      0.1525639846365789    0.161258643763519
0.9394894242033721      0.15224285460873072   0.16132503135807744
0.939651748162131       0.15194324759705927   0.16119242853066043
0.9398108885401995      0.15164974934251949   0.16078540573217612
0.9399835147824849      0.15133164249515893   0.16009750500018752
0.940144622721626       0.1510348997460566    0.1594078859741927
0.9403192165249841      0.1507135908324657    0.15884275821980795
0.9404906267476516      0.1503984220497833    0.15860331286418608
0.9406505186671749      0.1501046858579323    0.1586206377290359
0.9408238964509151      0.14978645515017505   0.15870827635608747
0.9409857559315111      0.14948963203982088   0.15864032893097968
0.9411444318314165      0.14919889910572984   0.15830773912951515
0.9413165935955389      0.14888374198603674   0.15767047780488921
0.941477237056517       0.148589940615519     0.15697492611955094
0.941651366381712       0.1482717729652721    0.15634929985144633
0.9418223121262165      0.1479597271065482    0.1560316578165204
0.9419817395675767      0.14766898180809107   0.15600128745443878
0.9421546528731539      0.1473539455998847    0.156095961652502
0.9423160478755869      0.14706018230920173   0.15608875464202546
0.9424909287422367      0.14674218908198844   0.1557952992028452
0.9426626260281961      0.14643030802695098   0.15520064373314632
0.9428228050110111      0.1461396411971524    0.15451074683457292
0.9429964698580431      0.14582482365018004   0.15385101124706768
0.9431586164019308      0.14553119090892008   0.15349039091465036
0.943317579365128       0.1452436115330496    0.153409006982029
0.9434900281925421      0.14493196058100677   0.15349608821416696
0.943650958716812       0.144641434110255     0.15353319065003998
0.9438253751052988      0.14432690120800518   0.15331945663263558
0.9439966079130951      0.14401845610055566   0.15279159820194432
0.9441563224177472      0.14373107159974147   0.1521199818964721
0.9443295227866161      0.14341976505538162   0.15142152096685824
0.9444912048523407      0.14312948705161607   0.1509907919832965
0.9446497033373749      0.14284523274274571   0.15084603853039172
0.944821687686626       0.14253714055610164   0.15090978432577673
0.9449821537327329      0.1422500115616519    0.15098232457285413
0.9451561056430566      0.14193911396932102   0.15085079506435436
0.9453185392502361      0.14164914573479      0.15043213908713343
0.9454777892767251      0.14136518506454696   0.14980238889170325
0.9456505251674311      0.14105754372678952   0.14907328845860057
0.9458117427549928      0.14077076296723093   0.14855934797618717
0.9459864462067714      0.14046037390580943   0.14831619404948546
0.9461579660778595      0.14015602964559903   0.14834732142961662
0.9463179676458033      0.13987242540274392   0.14843957371650426
0.9464914550779641      0.1395652278688697    0.14837532963826605
0.9466534242069805      0.13927879100192578   0.14803032252359324
0.9468122097553064      0.13899832945407647   0.1474463850063313
0.9469844811678494      0.1386944380630644    0.14671169714909987
0.9471452342772481      0.13841123474906483   0.14614460297960033
0.9473194732508636      0.13810467953702604   0.14582528280050178
0.9474905286437887      0.13780414060187426   0.14580869656969186
0.9476500657335695      0.13752421294345213   0.145908558419197
0.9478230886875672      0.1372210342768944    0.14590811262001518
0.9479845933384206      0.13693842831671657   0.14564472168751533
0.948159583853491       0.13663265294771518   0.14505496366630555
0.9483313907878709      0.1363328779539224    0.1443264793593035
0.9484916794191065      0.13605359469022443   0.14373033737932447
0.948665453914559       0.13575124749231865   0.14336039681475995
0.9488277101068673      0.13546935133574445   0.14330309303194846
0.948986782718485       0.13519337421480482   0.1433979229223452
0.9491593411943197      0.13489443824174133   0.14344452335843585
0.94932038136701        0.13461587072357314   0.14325320501887587
0.9494949074039174      0.13431443083514777   0.1427305499601374
0.9496662498601343      0.13401895425877328   0.1420204470320786
0.9498260740132068      0.13374375909401393   0.14139001088773326
0.9499993840304963      0.13344580308348744   0.1409487194354504
0.9501611757446415      0.13316808469769797   0.140831663836758
0.9503197838780962      0.1328962420880499    0.14090800797869882
0.9504918778757679      0.13260174959542353   0.1409933521665593
0.9506524535702954      0.13232740626040618   0.1408767699646431
0.9508265151290397      0.13203050436421      0.14043440645903893
0.9509973931070935      0.13173952427571214   0.1397589447559908
0.9511567527820031      0.1314686002866125    0.1391041669512508
0.9513295983211295      0.13117523525818325   0.13859053082332523
0.9514909255571117      0.13090187947098586   0.13840199701196787
0.9516657386573109      0.13060617767426846   0.13845540380695542
0.9518373681768195      0.13031637544168462   0.1385602308747944
0.9519974793931839      0.13004648515123143   0.13849276472252656
0.9521710764737652      0.12975437102391682   0.13811070359111155
0.9523331552512022      0.12948211982857938   0.13750469889015154
0.9524920504479487      0.1292156702149804    0.1368416063844832
0.9526644315089121      0.12892701579548652   0.13627159622147672
0.9528252942667312      0.12865811557613233   0.13601678101833511
0.9529996428887674      0.12836720428114218   0.13602676979063208
0.953170807930113       0.12808214765396864   0.13614588081474188
0.9533304546683143      0.12781676191347052   0.13613936583525202
0.9535035872707326      0.127529495159795     0.1358419046531655
0.9536652015700066      0.12726184792542194   0.13529116531276705
0.95382363228859        0.12699995266735836   0.13463265692229642
0.9539955488713904      0.1267163057452256    0.13401120073706044
0.9541559471510465      0.12645217471805942   0.13368353074935377
0.9543298312949196      0.1261663984807446    0.13363443928571386
0.9544921971356484      0.1259000843581565    0.13374657387694663
0.9546513793956867      0.1256394942969559    0.13380306595807182
0.9548240475199419      0.1253573935061905    0.13361215915145805
0.9549851973410528      0.12509464652420701   0.13314491680282767
0.9551598330263807      0.12481049947225019   0.13244220521405714
0.9553312851310182      0.12453213163404592   0.13178671681110118
0.9554912189325113      0.12427300386537166   0.1313970182805594
0.9556646385982214      0.12399261820091824   0.13128776503153464
0.9558265399607871      0.12373141529769012   0.13138480774184946
0.9559852577426624      0.12347587703491857   0.1314769535054325
0.9561574613887546      0.12319922205199252   0.13136368700330153
0.9563181467317025      0.12294163482823417   0.13096898860570927
0.9564923179388674      0.12266304705442749   0.13030314317302719
0.9566633055653417      0.12239018107113991   0.12962314522547408
0.9568227748886718      0.12213626225802682   0.1291702383114312
0.9569957300762189      0.12186149189760452   0.12898896688908493
0.9571571669606216      0.12160560790168877   0.12905581846703615
0.9573320897092413      0.12132899282695923   0.1291798399770639
0.9575038288771704      0.12105806864807603   0.1291172813720367
0.9576640497419553      0.12080590532587925   0.12877634633997528
0.9578377564709571      0.12053316531761267   0.12814591221766083
0.9579999448968147      0.12027912286379429   0.12749135126503172
0.9581589497419818      0.12003064474473364   0.12698950154922786
0.9583314404513658      0.11976174304693613   0.12674202447709315
0.9584924128576054      0.11951141242194961   0.12677185158347518
0.9586668711280621      0.11924078419058867   0.12690961974677345
0.9588381458178282      0.11897574918419034   0.12691122380563058
0.95899790220445        0.11872908812894577   0.12664624596930085
0.9591711444552888      0.11846228308641797   0.12607286915089844
0.9593328684029834      0.11821385787380363   0.1254208209780007
0.9594914087699874      0.1179709275950558    0.12487371437193648
0.9596634350012083      0.1177080145913671    0.12455301006224867
0.959823942929285       0.11746334922357796   0.12453227522851397
0.9599979367215786      0.11719883382871712   0.12466957976986813
0.960160412210728       0.11695249734754157   0.1247322768355921
0.9603197041191869      0.11671161806257807   0.12456370195527965
0.9604924818918626      0.11645105578923833   0.12407632244920597
0.9606537413613941      0.11620853496076725   0.12344567091332666
0.9608284866951425      0.1159464685754024    0.12280148272928974
0.9610000484482004      0.11568992619103377   0.12241872883643791
0.961160091898114       0.11545128133824731   0.12234692354561494
0.9613336212122446      0.11519326690362972   0.12247139191845975
0.961495632223231       0.11495307738722552   0.12257297559479735
0.9616544596535268      0.11471826669312586   0.12247465721856357
0.9618267729480395      0.11446426083769741   0.12206159990830513
0.961987567939408       0.11422793504730516   0.1214635603314838
0.9621618487949934      0.11397255760331083   0.12079717144816779
0.9623329460698883      0.11372262798840908   0.12035039075659196
0.9624925250416388      0.11349022697048235   0.12021699472040195
0.9626655898776064      0.11323895780391394   0.12031384983080462
0.9628271364104297      0.11300514076385343   0.12044498762535724
0.9629854993625625      0.11277661865986317   0.12041710140041752
0.9631573481789122      0.11252941014347116   0.1200893618617663
0.9633176786921176      0.1122995015814861    0.11953918678415243
0.96349149506954        0.11205105613711268   0.11886364154491384
0.9636537931438182      0.11181983141556853   0.11837318715599562
0.9638129076374057      0.11159385597448739   0.11815433228802093
0.9639855079952102      0.1113495312472846    0.11819601558319746
0.9641465900498705      0.11112226971387965   0.11834343835904529
0.9643211579687476      0.11087681322597541   0.1183845602819337
0.9644925423069343      0.11063667923309749   0.11813534269836673
0.9646524083419766      0.11041344397985624   0.11763874895545011
0.964825760241236       0.11017221075978287   0.11697036560741217
0.9649875938373511      0.10994779319470721   0.11643795164787883
0.9651462438527757      0.10972849032666579   0.11615623841360641
0.9653183797324172      0.10949135022617815   0.1161463086332895
0.9654789973089144      0.1092708664897553    0.11629176693488107
0.9656531007496285      0.10903273498271389   0.11638751757940952
0.9658240206096521      0.10879983781516613   0.11622280236570168
0.9659834221665314      0.10858342579741023   0.11579293555727649
0.9661563095876278      0.1083495716927725    0.11514678013107126
0.9663176787055798      0.10813211625291294   0.11457962946565789
0.9664925336877488      0.10789738499182684   0.11421096835115076
0.9666642050892272      0.10766784050719132   0.11416400624785703
0.9668243581875614      0.1074545175107776    0.11430241918540514
0.9669979971501126      0.10722413090177      0.11443017548632553
0.9671601178095195      0.1070098762686809    0.11433625309817567
0.9673190548882358      0.10680062943893871   0.11397250803341313
0.9674914778311691      0.1065745289618791    0.11336003581925684
0.9676523824709581      0.10636438287205183   0.11277455797466787
0.967826772974964       0.1061375558661758    0.11234625168494433
0.9679979798982794      0.10591581787396519   0.11223901518633532
0.9681576685184505      0.10570984905117316   0.11235552023210972
0.9683308430028387      0.10548741953379469   0.1125155625481652
0.9684924991840825      0.10528066556838338   0.11249267013851093
0.9686509717846358      0.10507881343309422   0.11220483121266374
0.968822930249406       0.10486071835175531   0.11164101946210873
0.968983370411032       0.10465811344526108   0.11104792213166718
0.9691572964368749      0.10443944473175899   0.11056149520495565
0.9693197041595735      0.1042361695618155    0.11038573448239804
0.9694789283015816      0.1040377385375071    0.11045652927881909
0.9696516383078067      0.10382346784429813   0.11063677646322073
0.9698128300108875      0.10362439947044468   0.1106929012149081
0.9699875075781852      0.10340967580151131   0.11046762135895986
0.9701590015647924      0.10319988201434034   0.10995791565643132
0.9703189772482553      0.1030050913606249    0.10937116036510793
0.9704924387959352      0.10279488013515119   0.10884304507549299
0.9706543820404707      0.10259957131102639   0.1086082075068586
0.9708131417043158      0.10240899095589888   0.10863675266217218
0.9709853872323778      0.10220322194062237   0.10881873976189049
0.9711461144572956      0.10201215632277096   0.1089254767177951
0.9713203275464303      0.10180609242213406   0.10878311280112571
0.9714913570548744      0.10160480919641707   0.10834072244357028
0.9716508682601743      0.1014180248053579    0.10777275458174261
0.9718238653296912      0.10121648265170129   0.10720923540242132
0.9719853440960639      0.10102933479939781   0.10691109493260045
0.9721603087266534      0.1008276256906985    0.10689588357824913
0.9723320897765524      0.10063067311341524   0.10707439791415091
0.9724923525233071      0.10044790186436499   0.1072113945179697
0.9726661011342788      0.10025081936495585   0.10712587808333213
0.9728283314421061      0.10006781046413382   0.10676159905553788
0.972987378169243       0.09988934268861821   0.10622220509015749
0.9731599107605968      0.09969681059599295   0.10563956776299291
0.9733209250488064      0.09951813948423271   0.10529025382042827
0.9734954252012329      0.09932560717407579   0.10521936419495194
0.9736667417729689      0.09913770932386087   0.105380529502398
0.9738265400415606      0.09896345125429071   0.10554982291356876
0.9739998241743691      0.09877559018780001   0.10553870430982129
0.9741615900040335      0.09860125697272586   0.10524747406054855
0.9743201722530073      0.09843133441217147   0.1047492374025929
0.9744922403661981      0.09824806374633027   0.10415713226957084
0.9746527901762446      0.09807810034814789   0.10375698906435557
0.9748268258505081      0.09789499851617696   0.10362091072922139
0.9749893432216272      0.09772508884018914   0.10374075866829915
0.9751486770120559      0.09755951879011807   0.10393374297131251
0.9753214966667014      0.09738107193613663   0.10400740423374825
0.9754827980182027      0.09721559048040333   0.10381245548570842
0.975657585233921       0.09703744741176627   0.10332437691431749
0.9758291888689488      0.09686374199625807   0.10273668394601067
0.9759892742008323      0.09670276629535118   0.10229984614103182
0.9761628453969327      0.09652940221940275   0.10210640928451618
0.9763248982898888      0.09636864855126637   0.10219116607313113
0.9764837676021544      0.09621209397580588   0.1023910524560079
0.976656122778637       0.09604342060349004   0.10252035043383818
0.9768169596519753      0.09588712292421533   0.10239965660389366
0.9769912823895306      0.09571892818386314   0.101976407443808
0.9771624215463953      0.09555503293175842   0.10140464893314781
0.9773220424001158      0.09540326954358072   0.10093530558696287
0.9774951491180531      0.09523989044424354   0.10067922669437258
0.9776567375328462      0.09508852479322008   0.10071728726395891
0.9778151423669488      0.09494121783901814   0.10091223717889544
0.9779870330652684      0.0947825735032349    0.10109088138665256
0.9781474054604437      0.09463569404884521   0.10104824027128277
0.9783212637198359      0.09447770531299947   0.10070273251740046
0.9784836036760838      0.09433135468395887   0.10018948084086914
0.9786427600516412      0.09418897645329388   0.09969006313840918
0.9788154022914155      0.09403577302272159   0.09935489213149541
0.9789765262280455      0.09389395874989546   0.09932119914045136
0.9791511360288926      0.09374155292456882   0.09951351625748915
0.9793225622490491      0.09359322436211467   0.09972640832140255
0.9794824701660614      0.09345602658443493   0.09975000137987997
0.9796558639472905      0.09330853338357295   0.09947968088795946
0.9798177394253754      0.09317204008640712   0.09900595218311617
0.9799764313227698      0.09303936234248122   0.09849973773392112
0.9801486090843811      0.09289668103366705   0.09811519107280195
0.9803092685428481      0.0927647428921412    0.09802627526122862
0.9804834138655321      0.092623041237485     0.09818883151765952
0.9806543756075256      0.09248526231691698   0.09842663947010925
0.9808138190463748      0.09235796021701922   0.09851591116329492
0.9809867483494409      0.09222119867194802   0.09832990503488534
0.9811481593493627      0.09209477899720929   0.09790906150142799
0.9813230562135016      0.09195914558877141   0.09735680851457701
0.9814947694969499      0.09182735091008912   0.09694814053258093
0.981654964477254       0.09170562525498697   0.09682544911245634
0.9818286453217749      0.09157499702062119   0.09696595790854073
0.9819908078631516      0.09145429955556796   0.09720504751285733
0.9821497868238378      0.09133716436434956   0.09734662834316618
0.9823222516487409      0.09121143312944004   0.09723705930512988
0.9824831981704997      0.09109536180409654   0.0968724735306744
0.9826576305564755      0.09097094654282611   0.0963356330448135
0.9828288793617608      0.09085020542141634   0.09589301852613018
0.9829886098639018      0.09073884339791455   0.09571706663122945
0.9831618262302597      0.09061945652170683   0.09581474769755711
0.9833235242934734      0.09050930642444659   0.0960564791186558
0.9834820387759965      0.09040254588159398   0.0962462153068616
0.9836540391227365      0.09028807471586923   0.09621809201274369
0.9838145211663323      0.09018256176271745   0.09592121838551947
0.983988489074145       0.09006964978944956   0.09541304951542576
0.9841509386788135      0.08996556013537364   0.0949617796118697
0.9843102047027915      0.08986476429963229   0.09471601195970242
0.9844829565909864      0.08975684185677188   0.09474485436177678
0.984644190176037       0.08965744169297407   0.0949705834269289
0.9848189096253046      0.08955117831105125   0.0952230616583999
0.9849904454938816      0.08944832410679492   0.09526498384247602
0.9851504630593143      0.08935369643438454   0.09503384476132312
0.985323966488964       0.08925253836409515   0.09456337870117919
0.9854859516154695      0.08915945685209295   0.09410152112745465
0.9856447531612844      0.08906948462018832   0.09381217567147887
0.9858170405713162      0.08897330946691204   0.09378850137872766
0.9859778096782038      0.08888491802761105   0.0939935601527198
0.9861520646493084      0.08879059298275427   0.09427494288850002
0.9863231360397224      0.0886994947185557    0.09438798708561186
0.9864826891269922      0.08861587703633239   0.09423134536261062
0.9866557280784789      0.08852666591655435   0.09381174261330194
0.9868172487268213      0.08844478163727836   0.09334844626578369
0.9869922552393806      0.08835757877095791   0.0929948969897102
0.9871640781712494      0.08827350336599872   0.09294053438737065
0.9873243827999739      0.0881964453037785    0.09313121257028548
0.9874981732929154      0.08811441574821936   0.09342918749831486
0.9876604454827127      0.08803924655640705   0.0935874483515627
0.9878195340918194      0.08796689091760565   0.09349896678558146
0.9879921085651431      0.08788990520729788   0.09313401245130974
0.9881531647353224      0.08781947359646872   0.0926811672700615
0.9883277067697188      0.08774469268386063   0.09229414553365706
0.9884990652234246      0.0876728472755292    0.09218767358278299
0.988658905373986       0.0876072390923754    0.09234609972366271
0.9888322313887645      0.0875376360396083    0.09265432615003708
0.9889940391003987      0.08747410946942012   0.09286665306025683
0.9891526632313424      0.08741319696780708   0.09285207187369161
0.989324773226503       0.08734863813563382   0.09255370267699245
0.9894853649185194      0.08728984238888077   0.09212199578173805
0.9896594424747527      0.08722768694477084   0.09170617062018466
0.9898303364502954      0.08716826963779517   0.09154242851409776
0.9899897121226938      0.08711429123349074   0.09165743758817364
0.9901625736593093      0.08705731483377596   0.09196345180950044
0.9903239168927804      0.08700561286561113   0.0922243913995752
0.9904987459904685      0.08695120514032167   0.09227905233922504
0.990670391507466       0.08689942802132138   0.09203270943044599
0.9908305187213193      0.08685259470049822   0.0916255513850698
0.9910041317993896      0.08680342433664355   0.09120130268758113
0.9911662265743156      0.08675903001841803   0.09100804589860763
0.9913251377685511      0.08671693085096704   0.09108156847169763
0.9914975348270035      0.08667285713367888   0.09137679738097891
0.9916584135823117      0.08663323263602873   0.09167281548219451
0.9918327782018367      0.08659193168270639   0.0918016158301311
0.9920039592406712      0.08655305440195452   0.09162971127314355
0.9921636219763614      0.08651828840725653   0.09126088055135477
0.9923367705762687      0.08648222193147535   0.09082719720092562
0.9924984008730316      0.08645009526498088   0.09058813121436685
0.9926568475891041      0.08642004936696576   0.09061134925678875
0.9928287801693935      0.08638907252055356   0.09088306276078477
0.9929891944465385      0.08636170160061377   0.09120578531974344
0.9931630945879006      0.08633370362663911   0.09140885675480133
0.9933254764261185      0.08630913683389597   0.09133232462344418
0.9934846746836457      0.08628653405415822   0.09102399671661122
0.9936573588053899      0.08626368063494211   0.0905910895990174
0.9938185246239898      0.086243918174424     0.09029742894205628
0.9939931763068066      0.08622421460216541   0.09025887727045802
0.9941646444089329      0.086206608007744     0.09050339733677398
0.994324594207915       0.0861917407217332    0.09084152536758448
0.994498029871114       0.08617732253668509   0.0911051759615885
0.9946599472311688      0.08616546519475357   0.09110254289448344
0.994818681010533       0.08615534794959982   0.0908511313150699
0.9949909006541141      0.0861460632500629    0.09043558098211307
0.995151601994551       0.08613899212267635   0.0901135937914754
0.9953257891992048      0.08613306886137016   0.0900200409889697
0.995496792823168       0.08612902075178178   0.09022545472016205
0.995656278143987       0.08612682733995058   0.09056875994445604
0.9958292493290231      0.08612617918966133   0.09088917627577998
0.9959907022109148      0.08612720358688924   0.09096599122416936
0.9961656409570234      0.08613009430552092   0.09075146313128239
0.9963373961224415      0.08613473923962524   0.09035789935860435
0.9964976329847154      0.08614069123587736   0.09002660819208473
0.9966713557112061      0.0861489141264919    0.0899030058828821
0.9968335601345527      0.08615825856482631   0.09006904041051055
0.9969925809772087      0.08616898623939137   0.09040956974659699
0.9971650876840816      0.08618238228522695   0.09077326014782641
0.9973260760878103      0.08619653910319043   0.09091962368785579
0.9975005503557559      0.08621369113741041   0.09077778734522943
0.997671841043011       0.08623236594041551   0.09041877316917928
0.9978316134271218      0.08625142881431556   0.09007453349003831
0.9980048716754496      0.08627389865089793   0.08990225898426604
0.9981666116206331      0.08629656736330743   0.09002262348295967
0.998325167985126       0.08632038050384223   0.09034927308018777
0.998497210213836       0.08634800503628516   0.09074918436355954
0.9986577341394016      0.08637546069664795   0.09096701837663426
0.9988317439291842      0.08640706032331029   0.0909085380079038
0.9989942354158224      0.08643829853159825   0.09061594651048577
0.9991535433217702      0.08647055110548317   0.09026501755889106
0.999326337091935       0.0865073591220286    0.09003130922993414
0.9994876125589555      0.0865434314722938    0.0900820031849989
0.9996623738901929      0.08658439760732119   0.09041155286180085
0.9998339516407397      0.08662652210790671   0.09083593746344165
0.9999999998333056      0.08666909059126325   0.09111962451147294
0.00008016381054133934  1.0000000000018532    1.0000000000026863
0.05075426963913869     0.9999999999791588    1.0000000006110765
0.05092035913365313     0.9999999999787686    1.000000000612595
0.05125548946429104     0.9999999999779612    1.0000000006151841
0.05142869766164137     0.9999999999775331    1.0000000006167098
0.052089694644137374    0.9999999999758284    1.0000000006243421
0.05225531966048878     0.9999999999753831    1.0000000006258696
0.05258952103480062     0.9999999999744615    1.0000000006282634
0.05276226475398792     0.9999999999739727    1.0000000006296155
0.05292765753125782     0.9999999999734968    1.0000000006312382
0.05342140382383179     0.9999999999720276    1.0000000006370893
0.05358656436202017     0.9999999999715198    1.0000000006386505
0.05375270868074487     0.9999999999710003    1.000000000639856
0.053919836780005934    0.999999999970469     1.0000000006408774
0.05408794865980332     0.9999999999699255    1.0000000006420005
0.05424870959768331     0.9999999999693973    1.0000000006433891
0.05441462167732651     0.9999999999688433    1.0000000006452001
0.05474939717822193     0.9999999999676973    1.000000000649246
0.05492242796070101     0.99999999996709      1.0000000006509395
0.05508810780126269     0.9999999999664988    1.000000000652121
0.05525477142236071     0.9999999999658945    1.0000000006530219
0.05542241882399507     0.9999999999652766    1.0000000006539478
0.05558271528371202     0.9999999999646764    1.0000000006551344
0.05574816288519219     0.999999999964047     1.0000000006567935
0.05608200942976153     0.9999999999627462    1.0000000006608274
0.05625457573407758     0.9999999999620575    1.00000000066259
0.05641979109647623     0.99999999996139      1.000000000663782
0.056585990239411216    0.9999999999607263    1.0000000006645837
0.05675317316288253     0.9999999999600477    1.000000000665302
0.0569213398668902      0.9999999999593538    1.000000000666308
0.05709465771266107     0.9999999999586265    1.0000000006678724
0.057427575300904345    0.9999999999571941    1.0000000006718552
0.05759550976583049     0.9999999999564535    1.0000000006736252
0.05775609328883924     0.9999999999557336    1.0000000006748038
0.057921827953611185    0.9999999999549785    1.0000000006755219
0.05808854639891947     0.9999999999542063    1.0000000006760368
0.058256248624764105    0.9999999999534165    1.0000000006767724
0.058429101992371946    0.9999999999525885    1.0000000006780891
0.05859460441806237     0.9999999999517823    1.0000000006798944
0.05876109062428915     0.9999999999509579    1.0000000006819476
0.058928560611052255    0.9999999999501148    1.000000000683801
0.059088679655897966    0.9999999999492954    1.0000000006850411
0.059253949842506884    0.9999999999484361    1.0000000006857148
0.05942020380965214     0.9999999999475574    1.0000000006860368
0.05958744155733374     0.9999999999466588    1.0000000006864695
0.059755663085551664    0.9999999999457398    1.000000000687438
0.05991653367185219     0.9999999999448468    1.0000000006889864
0.06024956090851601     0.9999999999429525    1.0000000006929506
0.06041755019765243     0.9999999999419735    1.0000000006943817
0.06059069062855206     0.9999999999409475    1.000000000695083
0.060756480117534276    0.9999999999399488    1.000000000695227
0.06092325338705284     0.9999999999389279    1.0000000006953516
0.06109101043710774     0.9999999999378842    1.0000000006959775
0.061251416545245234    0.9999999999368705    1.0000000006972802
0.061416973795145935    0.9999999999358077    1.0000000006991914
0.06158351482558298     0.9999999999347216    1.0000000007012062
0.06175103963655636     0.9999999999336114    1.0000000007027512
0.06192371558929295     0.9999999999324486    1.0000000007034873
0.06208904060011214     0.9999999999313174    1.0000000007034886
0.062255349391467665    0.9999999999301615    1.0000000007032928
0.06242264196335953     0.9999999999289805    1.0000000007035135
0.06259091831578772     0.9999999999277739    1.0000000007045566
0.06276434580997914     0.9999999999265311    1.0000000007064267
0.06293042236225313     0.999999999925337     1.0000000007084602
0.06309748269506349     0.9999999999241163    1.0000000007100784
0.06326552680841016     0.9999999999228685    1.0000000007108139
0.06342621997983944     0.9999999999216561    1.0000000007106906
0.06359206429303194     0.9999999999203851    1.0000000007101828
0.06375889238676075     0.9999999999190857    1.000000000709974
0.06392670426102592     0.9999999999177573    1.0000000007106191
0.0640996672770543      0.9999999999163653    1.000000000712256
0.06426527935116524     0.9999999999150104    1.0000000007142835
0.06443187520581256     0.9999999999136252    1.0000000007160323
0.06459945484099619     0.9999999999122091    1.0000000007168806
0.06475968353426244     0.9999999999108334    1.0000000007167014
0.0649250633692919      0.9999999999093909    1.0000000007159007
0.06509142698485769     0.9999999999079162    1.000000000715216
0.06525877438095981     0.9999999999064086    1.0000000007153604
0.06543127291882515     0.999999999904829     1.0000000007166503
0.06559642051477309     0.9999999999032917    1.0000000007186027
0.06576255189125736     0.9999999999017204    1.0000000007204861
0.06592966704827798     0.9999999999001141    1.0000000007215144
0.06609776598583492     0.9999999998984722    1.0000000007213277
0.06627101606515508     0.999999999896752     1.0000000007201943
0.06643691520255783     0.9999999998950779    1.0000000007190462
0.06660379812049692     0.9999999998933669    1.0000000007186964
0.06677166481897236     0.999999999891618     1.0000000007195835
0.06693218057553038     0.9999999998899196    1.0000000007213485
0.06709784747385163     0.9999999998881393    1.0000000007233216
0.0672644981527092      0.99999999988632      1.0000000007245338
0.06743213261210311     0.9999999998844611    1.0000000007244105
0.06760491821326023     0.9999999998825142    1.0000000007230474
0.06777035287249997     0.9999999998806203    1.00000000072139
0.06793677131227602     0.9999999998786855    1.000000000720394
0.06810417353258844     0.9999999998767087    1.0000000007207281
0.06826422481098343     0.9999999998747899    1.0000000007222318
0.06842942723114162     0.9999999998727795    1.0000000007242476
0.06859561343183618     0.999999999870726     1.0000000007256875
0.06876278341306705     0.9999999998686285    1.0000000007257328
0.06893093717483428     0.9999999998664862    1.0000000007242889
0.0690917399946841      0.9999999998644623    1.0000000007221979
0.0692576939562971      0.9999999998623417    1.000000000720421
0.06942463169844648     0.999999999860175     1.0000000007199594
0.06959255322113217     0.9999999998579608    1.000000000721061
0.06976562588558109     0.9999999998556417    1.0000000007231251
0.0699313476081126      0.9999999998533853    1.0000000007247631
0.07009805311118045     0.9999999998510796    1.000000000725005
0.07026574239478464     0.9999999998487231    1.0000000007234973
0.07042608073647141     0.9999999998464346    1.0000000007209806
0.0705915702199214      0.9999999998440359    1.0000000007184646
0.07075804348390774     0.9999999998415846    1.0000000007171963
0.07092550052843041     0.9999999998390795    1.0000000007177157
0.07109810871471628     0.9999999998364554    1.0000000007196073
0.07126336595908475     0.9999999998339024    1.0000000007214387
0.07142960698398956     0.9999999998312937    1.0000000007219756
0.07159683178943071     0.9999999998286278    1.0000000007205425
0.0717650403754082      0.9999999998259034    1.0000000007175156
0.0719384001031489      0.9999999998230499    1.0000000007141086
0.0721044088889722      0.9999999998202734    1.000000000712052
0.07227140145533183     0.9999999998174365    1.0000000007119738
0.0724393778022278      0.9999999998145376    1.0000000007135759
0.07260000320720636     0.9999999998117225    1.0000000007154737
0.07276577975394816     0.9999999998087725    1.0000000007163017
0.07293254008122627     0.9999999998057587    1.000000000715
0.07310028418904073     0.9999999998026795    1.0000000007116718
0.07327317943861839     0.9999999997994553    1.000000000707457
0.07343872374627866     0.9999999997963195    1.0000000007043788
0.07360525183447525     0.9999999997931163    1.0000000007033962
0.0737727637032082      0.9999999997898443    1.0000000007045262
0.07393292463002374     0.9999999997866685    1.0000000007064636
0.07409823669860247     0.9999999997833416    1.0000000007076435
0.07426453254771756     0.9999999997799438    1.0000000007066328
0.07443181217736897     0.9999999997764739    1.0000000007031609
0.07460007558755674     0.9999999997729301    1.0000000006983023
0.0747609880558271      0.9999999997694904    1.0000000006941283
0.07492705166586067     0.9999999997658882    1.0000000006918623
0.07509409905643055     0.99999999976221      1.0000000006921401
0.0752621302275368      0.9999999997584917    1.000000000694032
0.07543531254040625     0.9999999997546637    1.0000000006956111
0.0756011439113583      0.9999999997509407    1.0000000006949243
0.0757679590628467      0.9999999997471378    1.0000000006913927
0.07593575799487141     0.999999999743253     1.0000000006858616
0.07609620598497874     0.9999999997394816    1.0000000006805652
0.07626180511684927     0.99999999973553      1.0000000006770149
0.07642838802925614     0.9999999997314932    1.0000000006763317
0.07659595472219935     0.9999999997273693    1.0000000006778933
0.07676867255690577     0.9999999997230509    1.0000000006797523
0.07693403944969478     0.9999999997188509    1.0000000006795358
0.07710039012302013     0.9999999997145602    1.0000000006761611
0.07726772457688182     0.9999999997101768    1.000000000670101
0.0774277080888261      0.9999999997059219    1.000000000663646
0.0775928427425336      0.9999999997014632    1.000000000658582
0.07775896117677744     0.9999999996969086    1.0000000006565972
0.07792606339155761     0.9999999996922556    1.000000000657532
0.07809414938687412     0.9999999996875021    1.000000000659519
0.07825488444027323     0.999999999682887     1.0000000006599845
0.07842077063543554     0.9999999996780515    1.0000000006572127
0.07858764061113421     0.9999999996731123    1.0000000006509997
0.0787554943673692      0.9999999996680669    1.000000000643065
0.07892849926536741     0.9999999996627846    1.0000000006360232
0.0790941532214482      0.9999999996576477    1.0000000006325411
0.07926079095806535     0.9999999996524014    1.0000000006325973
0.07942841247521881     0.999999999647043     1.0000000006345362
0.0795886830504549      0.9999999996418426    1.0000000006355025
0.07975410476745418     0.9999999996363952    1.0000000006332455
0.07992051026498981     0.9999999996308324    1.000000000626942
0.08008789954306175     0.999999999625152     1.0000000006179885
0.08026043996289693     0.9999999996192065    1.0000000006091245
0.0804256294408147      0.9999999996134274    1.0000000006037755
0.0805918026992688      0.9999999996075273    1.0000000006025278
0.08075895973825925     0.9999999996015034    1.0000000006041365
0.08092710055778604     0.9999999995953531    1.0000000006056071
0.08110039251907603     0.9999999995889179    1.0000000006036844
0.08126633353844862     0.999999999582663     1.000000000597184
0.08143325833835754     0.999999999576278     1.0000000005872132
0.08160116691880281     0.999999999569893     1.0000000005768261
0.08176172455733068     0.999999999563727     1.0000000005696803
0.08192743333762174     0.9999999995572688    1.0000000005668432
0.08209412589844915     0.9999999995506739    1.0000000005678207
0.0822618022398129      0.9999999995439386    1.0000000005696281
0.08243462972293986     0.9999999995368883    1.000000000568477
0.0826001062641494      0.9999999995300333    1.0000000005622944
0.0827665665858953      0.9999999995230325    1.0000000005516063
0.08293401068817753     0.9999999995158823    1.0000000005393774
0.08309410384854235     0.9999999995089431    1.0000000005299223
0.0832593481506704      0.9999999995016735    1.000000000524974
0.08342557623333477     0.9999999994942492    1.0000000005248453
0.08359278809653549     0.9999999994866665    1.0000000005267835
0.08376098374027255     0.9999999994789216    1.0000000005266279
0.0839218284420922      0.9999999994714032    1.0000000005215126
0.08408782428567504     0.9999999994635276    1.0000000005107228
0.08425480390979426     0.9999999994554843    1.0000000004968062
0.08442276731444978     0.9999999994472695    1.0000000004840741
0.08459588186086853     0.9999999994386706    1.000000000476211
0.08476164546536988     0.9999999994303093    1.0000000004745973
0.08492839285040754     0.9999999994217708    1.0000000004763416
0.08509612401598157     0.9999999994130511    1.00000000047696
0.08525650423963818     0.9999999994045893    1.0000000004726517
0.08542203560505801     0.9999999993957265    1.0000000004618177
0.08575604967750666     0.9999999993774369    1.0000000004309921
0.08592869974576238     0.9999999993677665    1.0000000004200287
0.08609399887210069     0.9999999993583674    1.0000000004162701
0.08626028177897534     0.9999999993487718    1.0000000004173595
0.08642754846638633     0.9999999993389755    1.000000000418702
0.08659579893433364     0.9999999993289744    1.000000000415195
0.08676920054404418     0.9999999993185106    1.00000000040375
0.0869352512118373      0.9999999993083397    1.0000000003868563
0.08710228566016677     0.9999999992979582    1.0000000003688156
0.08727030388903258     0.9999999992873616    1.000000000355099
0.08743097117598098     0.9999999992770826    1.000000000349031
0.0875967896046926      0.9999999992663227    1.0000000003490064
0.08776359181394054     0.9999999992553912    1.0000000003507659
0.08793137780372483     0.9999999992444464    1.000000000348418
0.08810431493527232     0.9999999992329939    1.000000000337587
0.08826990112490243     0.9999999992218621    1.0000000003196945
0.08843647109506886     0.9999999992104974    1.0000000002988896
0.08860402484577162     0.9999999991988939    1.0000000002814027
0.08876422765455699     0.9999999991876359    1.0000000002720721
0.08892958160510558     0.9999999991758455    1.0000000002701888
0.08909591933619049     0.9999999991638076    1.0000000002720149
0.08926324084781176     0.999999999151516     1.0000000002710228
0.08943571350119622     0.9999999991386517    1.000000000261423
0.08960083521266327     0.9999999991261479    1.0000000002431255
0.08976694070466668     0.9999999991133814    1.0000000002197542
0.0899340299772064      0.9999999991003462    1.0000000001980875
0.09010210303028249     0.9999999990870359    1.0000000001841858
0.09027532722512177     0.9999999990731063    1.0000000001798637
0.09044120047804365     0.999999999059564     1.000000000181373
0.09060805751150186     0.9999999990457374    1.0000000001812424
0.09077589832549643     0.99999999903162      1.000000000172806
0.09093638819757358     0.9999999990179218    1.0000000001547686
0.09126865400579065     0.999999998988933     1.0000000001032627
0.0914362625807037      0.9999999989739823    1.0000000000846405
0.09160902229737994     0.9999999989583381    1.0000000000767655
0.09177443107213878     0.9999999989431342    1.0000000000772324
0.09194082362743397     0.9999999989276146    1.000000000078193
0.0921081999632655      0.9999999989117724    1.0000000000715954
0.09226822535717963     0.9999999988964068    1.0000000000542382
0.09243340189285695     0.9999999988803194    1.0000000000268685
0.09259956220907062     0.9999999988639005    0.9999999999967404
0.09276670630582062     0.999999998847143     0.9999999999725698
0.09293483418310697     0.9999999988300399    0.9999999999600181
0.09309561111847592     0.99999999881345      0.9999999999581637
0.09326153919560806     0.9999999987960857    0.9999999999598845
0.09342845105327655     0.9999999987783666    0.9999999999560636
0.09359634669148138     0.9999999987602853    0.9999999999398325
0.09376939347144941     0.9999999987413759    0.9999999999099909
0.09393508930950006     0.9999999987230066    0.9999999998758043
0.09410176892808703     0.9999999987045238    0.9999999998458554
0.09426943232721036     0.999999998685822     0.9999999998278601
0.09442974478441626     0.9999999986676831    0.9999999998229719
0.09459520838338537     0.9999999986486932    0.9999999998245136
0.09476165576289083     0.9999999986293113    0.9999999998226585
0.09492908692293262     0.9999999986095278    0.9999999998084275
0.09510166922473762     0.9999999985888295    0.9999999997782593
0.09526690058462522     0.9999999985687167    0.9999999997403813
0.09543311572504917     0.9999999985481876    0.9999999997041077
0.09560031464600943     0.999999998527232     0.9999999996793343
0.09576849734750606     0.9999999985058402    0.9999999996697808
0.09594183119076588     0.9999999984834596    0.9999999996704801
0.0961078140921083      0.9999999984617061    0.9999999996698249
0.09627478077398707     0.999999998439501     0.9999999996568946
0.09644273123640217     0.9999999984168342    0.9999999996269799
0.09676908141916077     0.9999999983718226    0.9999999995442105
0.09693581586195801     0.9999999983483242    0.9999999995121014
0.0971035340852916      0.9999999983243387    0.9999999994968558
0.09727640345038839     0.9999999982992458    0.9999999994954364
0.09744192187356777     0.9999999982748627    0.9999999994962246
0.0976084240772835      0.9999999982499774    0.9999999994861156
0.09777591006153556     0.999999998224579     0.9999999994574662
0.09793604510387022     0.9999999981999476    0.9999999994149112
0.0981013312879681      0.9999999981741631    0.9999999993660066
0.09826760125260231     0.9999999981478508    0.9999999993253768
0.09843485499777285     0.9999999981209996    0.9999999993025641
0.09860309252347974     0.9999999980935979    0.999999999297269
0.09876397910726922     0.999999998067021     0.9999999992988425
0.0989300168328219      0.9999999980392067    0.9999999992928748
0.09909703833891095     0.9999999980108272    0.9999999992680173
0.09926504362553631     0.9999999979818706    0.9999999992226605
0.0994382000539249      0.999999997951591     0.9999999991651152
0.09960400554039606     0.9999999979221783    0.9999999991156823
0.09977079480740358     0.9999999978921731    0.9999999990843166
0.09993856785494742     0.9999999978615635    0.9999999990738279
0.10009898996057388     0.9999999978318896    0.9999999990748771
0.10026456320796354     0.9999999978008433    0.9999999990717487
0.10043112023588954     0.9999999977698523    0.9999999990500529
0.10059866104435186     0.9999999977382332    0.9999999990046234
0.10077135299457742     0.9999999977051642    0.9999999989415556
0.10093669400288556     0.9999999976730408    0.9999999988825722
0.10110301879173003     0.9999999976402628    0.9999999988406857
0.10127032736111086     0.9999999976068145    0.9999999988226473
0.10143028498857427     0.999999997574382     0.9999999988215941
0.1015953937578009      0.9999999975404327    0.9999999988210958
0.10176148630756386     0.9999999975057897    0.9999999988038063
0.10192856263786317     0.9999999974704369    0.9999999987602202
0.10209662274869881     0.9999999974343576    0.9999999986947747
0.10225733191761704     0.9999999973993632    0.9999999986275142
0.10242319222829849     0.9999999973627344    0.9999999985720872
0.10259003631951627     0.9999999973253555    0.9999999985422953
0.10275786419127039     0.9999999972872093    0.9999999985359981
0.10293084320478772     0.9999999972473105    0.9999999985370592
0.10309647127638766     0.9999999972085459    0.9999999985240005
0.10326308312852392     0.9999999971689898    0.999999998482999
0.10343067876119652     0.9999999971286244    0.9999999984146805
0.10359092345195173     0.9999999970894826    0.999999998338655
0.10375631928447013     0.9999999970485148    0.9999999982703766
0.1039226988975249      0.999999997006714     0.9999999982283874
0.10409006229111598     0.9999999969640625    0.9999999982150535
0.10426257682647028     0.9999999969194564    0.9999999982161779
0.10442774041990716     0.9999999968761331    0.9999999982079643
0.1045938877938804      0.9999999968319343    0.9999999981716147
0.10476101894838996     0.9999999967868416    0.9999999981026564
0.10492913388343589     0.9999999967408363    0.9999999980143195
0.10510239996024501     0.999999996692734     0.9999999979293875
0.1052683150951367      0.9999999966460108    0.9999999978755647
0.10543521401056477     0.9999999965983499    0.9999999978547297
0.10560309670652915     0.9999999965497323    0.9999999978544518
0.10576362846057615     0.9999999965026021    0.9999999978501358
0.10592931135638634     0.9999999964532953    0.9999999978187536
0.10609597803273288     0.9999999964030074    0.9999999977503401
0.10643643008826184     0.9999999962981072    0.9999999975550099
0.10660188074499052     0.9999999962465084    0.9999999974849725
0.10676831518225555     0.9999999961944993    0.9999999974518894
0.10693573340005691     0.999999996141444     0.9999999974474714
0.10709580067594086     0.9999999960900141    0.9999999974465957
0.10726101909358803     0.9999999960361959    0.999999997421895
0.10742722129177153     0.9999999959812941    0.9999999973568845
0.10759440727049137     0.9999999959252828    0.9999999972561787
0.10776257702974755     0.9999999958681361    0.9999999971442323
0.10792339584708632     0.99999999581272      0.9999999970562851
0.10808936580618832     0.9999999957547308    0.9999999970045848
0.10825631954582664     0.999999995695568     0.9999999969907618
0.1084242570660013      0.9999999956352046    0.9999999969917363
0.10859734572793917     0.9999999955720826    0.9999999969731173
0.10876308344795965     0.9999999955107662    0.9999999969124446
0.10892980494851645     0.99999999544821      0.9999999968082105
0.10909751022960959     0.9999999953843861    0.999999996682837
0.10925786456878533     0.9999999953225058    0.9999999965759917
0.1094233700497243      0.9999999952577507    0.9999999965053217
0.10958985931119958     0.9999999951916899    0.9999999964798487
0.10975733235321122     0.9999999951242949    0.999999996479769
0.10992995653698606     0.9999999950538234    0.999999996468328
0.11009522977884348     0.999999994985386     0.9999999964149818
0.11026148680123726     0.999999994915575     0.9999999963106236
0.11042872760416736     0.9999999948443612    0.9999999961734132
0.11059695218763382     0.9999999947717148    0.9999999960403465
0.11077032791286348     0.999999994695767     0.9999999959457103
0.11093635269617573     0.9999999946220024    0.999999995908189
0.11110336126002432     0.999999994546765     0.999999995905194
0.11127135360440926     0.9999999944700244    0.999999995898677
0.11143199500687678     0.9999999943956361    0.9999999958534628
0.11159778755110752     0.9999999943178198    0.9999999957505862
0.1117645638758746      0.9999999942384614    0.9999999956032678
0.11193232398117803     0.9999999941575297    0.9999999954490691
0.11210523522824464     0.9999999940729399    0.999999995328479
0.11227079553339386     0.9999999939908182    0.9999999952716245
0.11243733961907942     0.9999999939070828    0.9999999952612336
0.11260486748530132     0.9999999938217019    0.9999999952595077
0.11276504440960583     0.9999999937389767    0.9999999952240596
0.11293037247567353     0.9999999936537721    0.9999999951268922
0.11309668432227757     0.9999999935672171    0.999999994972886
0.11326397994941795     0.9999999934789421    0.9999999947977577
0.11343225935709467     0.999999993388906     0.9999999946502026
0.11359318782285399     0.99999999330162      0.9999999945674098
0.11375926743037651     0.9999999932103089    0.9999999945414072
0.11392633081843537     0.9999999931171764    0.9999999945418347
0.11409437798703058     0.9999999930221803    0.9999999945175844
0.11426757629738898     0.9999999929228717    0.9999999944246117
0.11443342366583001     0.9999999928264258    0.9999999942661062
0.11460025481480735     0.999999992728054     0.9999999940715762
0.11476806974432105     0.999999992627713     0.9999999938942268
0.11492853373191733     0.9999999925304464    0.9999999937831763
0.11509414886127681     0.999999992428685     0.9999999937385899
0.11526074777117265     0.9999999923248939    0.9999999937362672
0.1154283304616048      0.9999999922190287    0.9999999937218482
0.11560106429380018     0.9999999921083551    0.9999999936407253
0.11576644718407816     0.9999999920008936    0.999999993482825
0.11610016430624312     0.9999999917795148    0.9999999930622044
0.11626016381567637     0.9999999916711875    0.9999999929167898
0.11642531446687283     0.9999999915578611    0.9999999928457282
0.11659144889860563     0.9999999914422919    0.9999999928347635
0.11675856711087476     0.9999999913244328    0.9999999928291217
0.11692666910368024     0.9999999912042363    0.9999999927667573
0.11708742015456831     0.9999999910877359    0.9999999926229844
0.1172533223472196      0.9999999909658847    0.9999999924021999
0.11742020832040721     0.9999999908416344    0.9999999921600332
0.11758807807413117     0.9999999907149362    0.9999999919631763
0.11776109896961834     0.9999999905825289    0.9999999918533119
0.1179267689231881      0.9999999904539951    0.9999999918280729
0.1180934226572942      0.9999999903229496    0.9999999918270791
0.11826106017193663     0.9999999901893423    0.999999991779665
0.11842134674466168     0.9999999900598985    0.9999999916457863
0.11858678445914991     0.9999999899245385    0.9999999914180161
0.11875320595417449     0.9999999897865552    0.9999999911471803
0.11892061122973542     0.9999999896458972    0.9999999909072516
0.11909316764705954     0.9999999894993877    0.999999990754892
0.11925837312246629     0.999999989359409     0.9999999907063392
0.11942456237840934     0.9999999892166946    0.9999999907052365
0.11959173541488877     0.9999999890711803    0.9999999906745247
0.1197598922319045      0.9999999889228005    0.999999990548103
0.11993320019068346     0.9999999887677342    0.9999999903023559
0.120099157207545       0.999999988617177     0.9999999900040444
0.1202660980049429      0.9999999884636551    0.9999999897236476
0.12043402258287711     0.9999999883071012    0.9999999895335672
0.12059459621889394     0.9999999881553767    0.9999999894584961
0.12076032099667397     0.9999999879966811    0.999999989450966
0.12092702955499035     0.9999999878348574    0.9999999894335792
0.12109472189384304     0.9999999876698364    0.9999999893250605
0.12126756537445897     0.9999999874973602    0.9999999890824958
0.12159953423239234     0.9999999871591805    0.9999999884342649
0.12176699433216354     0.9999999869850744    0.9999999881896839
0.12192710349001731     0.9999999868163706    0.9999999880744102
0.12209236378963431     0.9999999866399131    0.9999999880513327
0.12225860786978765     0.9999999864599917    0.999999988045057
0.12242583573047733     0.9999999862765333    0.999999987959955
0.12259404737170335     0.9999999860894633    0.9999999877378286
0.12275490807101196     0.999999985908164     0.9999999874123549
0.12292091991208377     0.9999999857185651    0.9999999870386538
0.12308791553369193     0.9999999855252569    0.9999999867256881
0.12325589493583643     0.999999985328163     0.9999999865444927
0.12342902547974413     0.9999999851222143    0.9999999864925057
0.12359480508173443     0.999999984922305     0.9999999864909977
0.12376156846426108     0.9999999847185088    0.9999999864275034
0.12392931562732405     0.9999999845107473    0.9999999862216866
0.12408971184846962     0.9999999843094702    0.9999999858876338
0.12425525921137842     0.999999984099014     0.9999999854731741
0.12442179035482354     0.9999999838844951    0.9999999850966307
0.124589305278805       0.9999999836658326    0.9999999848518354
0.12476197134454967     0.9999999834373952    0.9999999847599402
0.12492728646837695     0.9999999832157497    0.9999999847556365
0.12509358537274057     0.9999999829898598    0.999999984715661
0.12526086805764053     0.9999999827596433    0.9999999845360366
0.12542913452307683     0.9999999825273704    0.999999984183758
0.12560255213027632     0.9999999822864873    0.9999999837085969
0.12576861879555845     0.9999999820526727    0.9999999832724232
0.1259356692413769      0.9999999818143235    0.9999999829670859
0.12610370346773164     0.9999999815713356    0.9999999828353424
0.12626438675216903     0.9999999813358986    0.9999999828203966
0.1264302211783696      0.9999999810897119    0.9999999827985314
0.12659703938510652     0.9999999808387365    0.9999999826457221
0.12676484137237978     0.9999999805828657    0.999999982300786
0.12693779450141623     0.9999999803155042    0.9999999817927758
0.12710339668853532     0.9999999800559993    0.9999999812889173
0.12726998265619072     0.9999999797914436    0.9999999809020406
0.12743755240438243     0.9999999795217273    0.9999999807055996
0.1275977712106568      0.9999999792604236    0.9999999806660267
0.12776314115869436     0.9999999789871682    0.9999999806583949
0.12792949488726824     0.999999978708602     0.9999999805399133
0.12809683239637848     0.9999999784246121    0.999999980216478
0.12826932104725192     0.9999999781278653    0.999999979687707
0.12843445875620796     0.9999999778398985    0.999999979117301
0.12860058024570034     0.9999999775463524    0.9999999786371425
0.12876768551572906     0.9999999772471113    0.9999999783556727
0.12893577456629413     0.999999976942057     0.999999978271864
0.1291090147586224      0.9999999766233508    0.9999999782679786
0.12927490400903324     0.9999999763140253    0.999999978172324
0.12944177703998044     0.9999999759987285    0.9999999778621942
0.129609633851464       0.9999999756773391    0.9999999773265229
0.12977013972103013     0.9999999753660086    0.9999999767105356
0.12993579673235947     0.9999999750405243    0.999999976135874
0.13010243752422515     0.9999999747087948    0.9999999757596457
0.13027006209662717     0.9999999743706959    0.9999999756164407
0.1304428378107924      0.9999999740175264    0.9999999756067403
0.1306082625830402      0.9999999736748821    0.9999999755455574
0.13077467113582436     0.9999999733257104    0.9999999752739399
0.1309420634691449      0.9999999729698839    0.9999999747410869
0.13126729739371426     0.9999999722651903    0.9999999733987063
0.1314334737074169      0.9999999718982393    0.9999999729076465
0.13160063380165588     0.9999999715244489    0.999999972679329
0.1317687776764312      0.9999999711497167    0.9999999726444111
0.13192957060928912     0.9999999707867289    0.9999999726199545
0.13209551468391023     0.9999999704072867    0.9999999724135954
0.13226244253906766     0.9999999700205771    0.9999999719196156
0.13243035417476146     0.9999999696264363    0.999999971184765
0.1326034169522185      0.9999999692147177    0.9999999703731144
0.1327691287877581      0.999999968815192     0.9999999697611038
0.13293582440383406     0.9999999684079943    0.9999999694324067
0.13310350380044633     0.999999967992956     0.9999999693536362
0.1332638322551412      0.999999967590944     0.9999999693457915
0.1334293118515993      0.9999999671706463    0.9999999691889416
0.13359577522859375     0.9999999667422749    0.9999999687301905
0.13376322238612454     0.9999999663056567    0.999999967973045
0.13393582068541854     0.9999999658495291    0.9999999670643634
0.1341010680427951      0.999999965406972     0.9999999663155866
0.134267299180708       0.9999999649559269    0.9999999658573923
0.1344345140991573      0.9999999644962161    0.9999999657056101
0.1346027127981429      0.9999999640276585    0.9999999656989191
0.1347760626388917      0.9999999635382254    0.9999999655730327
0.1349420615377231      0.9999999630632624    0.9999999651356564
0.13510904421709083     0.9999999625792071    0.9999999643518197
0.1352770106769949      0.9999999620858736    0.9999999633761616
0.13543762619498156     0.9999999616080331    0.9999999625143096
0.13560339285473144     0.9999999611085459    0.999999961913891
0.1357701432950177      0.9999999605995433    0.9999999616679767
0.13593787751584024     0.9999999600808349    0.9999999616443606
0.136110762878426       0.9999999595390823    0.9999999615670818
0.13627629729909435     0.9999999590135215    0.9999999611876337
0.13644281550029905     0.9999999584780092    0.9999999604135819
0.1367704685218637      0.9999999574040196    0.9999999583597727
0.13693577070345053     0.9999999568518406    0.9999999575889846
0.13710205666557373     0.9999999562892778    0.9999999572115195
0.13726932640823325     0.9999999557161313    0.9999999571419926
0.13743757993142908     0.9999999551321975    0.999999957110382
0.13759848251270757     0.9999999545667484    0.999999956829475
0.13776453623574922     0.9999999539759222    0.9999999561184367
0.1379315737393272      0.9999999533778526    0.9999999550313161
0.13809959502344155     0.9999999527742339    0.9999999538279656
0.1382727674493191      0.9999999521438973    0.9999999528229029
0.13843858893327926     0.999999951532388     0.9999999522931909
0.13860539419777576     0.9999999509093095    0.9999999521513583
0.1387731832428086      0.999999950274406     0.9999999521400457
0.13893362134592402     0.9999999496595668    0.9999999519285666
0.13909921059080266     0.9999999490169318    0.9999999512722324
0.13926578361621766     0.9999999483621118    0.999999950159153
0.13943334042216898     0.9999999476948439    0.9999999488247482
0.1396060483698835      0.999999946997934     0.9999999476111089
0.1397714053756806      0.9999999463218878    0.9999999468875392
0.13993774616201407     0.9999999456330215    0.9999999466304909
0.14010507072888387     0.9999999449310647    0.9999999466160262
0.14026504435383624     0.9999999442513852    0.9999999464802507
0.14043016912055184     0.9999999435409344    0.9999999459096182
0.1405962776678038      0.9999999428170334    0.9999999448080422
0.14076336999559208     0.9999999420794046    0.9999999433629201
0.14093144610391667     0.9999999413277654    0.9999999419579764
0.1410921712703239      0.9999999405998236    0.999999941002192
0.1412580475784943      0.9999999398390356    0.9999999405500982
0.14142490766720106     0.9999999390638711    0.9999999404827312
0.14159275153644416     0.9999999382740402    0.9999999404176962
0.14176574654745044     0.9999999374492505    0.9999999399234977
0.14193139061653937     0.9999999366491948    0.9999999388513633
0.1420980184661646      0.9999999358340939    0.9999999373171559
0.14226563009632615     0.9999999350036504    0.9999999357056434
0.14242589078457035     0.999999934199645     0.9999999345054081
0.14259130261457773     0.9999999333594543    0.9999999338453796
0.14275769822512144     0.9999999325035553    0.9999999336879913
0.1429250776162015      0.9999999316316436    0.99999993366447
0.14309760814904476     0.9999999307212804    0.99999993328521
0.14326278773997064     0.9999999298385382    0.9999999322867295
0.14342895111143286     0.9999999289394045    0.9999999307023411
0.14359609826343142     0.9999999280235672    0.9999999288919864
0.14376422919596632     0.9999999270907093    0.9999999273530366
0.14393751127026438     0.9999999261169422    0.9999999264311974
0.1441034424026451      0.9999999251726379    0.9999999261711321
0.14427035731556215     0.9999999242227172    0.9999999261577288
0.1444382560090155      0.9999999232574437    0.9999999258698715
0.14459880376055148     0.9999999223229062    0.9999999249708548
0.14476450265385066     0.9999999213464019    0.9999999233615625
0.14493118532768617     0.9999999203516493    0.9999999213735632
0.14509885178205803     0.9999999193382524    0.9999999195436201
0.14527166937819308     0.9999999182801279    0.9999999183167922
0.14543713603241076     0.9999999172539037    0.9999999178737428
0.14560358646716476     0.9999999162084663    0.9999999178391451
0.14577102068245507     0.9999999151434084    0.9999999176596848
0.145931103955828       0.9999999141123422    0.999999916885048
0.14609633837096414     0.9999999130348428    0.9999999153052288
0.14642975854284543     0.9999999108189145    0.999999911037428
0.14659794429959058     0.999999909679641     0.9999999094719364
0.14675877911441831     0.9999999085764559    0.9999999087572582
0.14692476507100927     0.9999999074237204    0.9999999086233889
0.1470917348081366      0.9999999062494104    0.9999999085546021
0.14725968832580022     0.9999999050530888    0.9999999079290297
0.14743279298522705     0.9999999038040484    0.9999999063479988
0.14759854670273648     0.9999999025926232    0.999999904099954
0.14776528420078225     0.9999999013586073    0.9999999016762396
0.14793300547936436     0.9999999001015523    0.999999899736811
0.14809337581602908     0.9999998988846486    0.9999998987184711
0.14825889729445701     0.9999998976131715    0.9999998984362769
0.1484254025534213      0.9999998963180963    0.9999998984121234
0.14859289159292188     0.9999998949989631    0.999999897945028
0.14876553177418572     0.9999998936218532    0.9999998964868358
0.1489308210135321      0.9999998922866458    0.9999998941815023
0.14909709403341487     0.9999998909268014    0.9999998914838631
0.149264350833834       0.9999998895418492    0.9999998891263284
0.14943259141478943     0.9999998881313106    0.9999998876737057
0.14960598313750806     0.9999998866590931    0.999999887185818
0.1497720239183093      0.9999998852315186    0.9999998871638479
0.14993904847964687     0.9999998837777682    0.9999998867995549
0.15010705682152078     0.9999998822973518    0.9999998854675334
0.15026771422147728     0.999999880864563     0.9999998831854836
0.150433522763197       0.9999998793726379    0.9999998802446761
0.15060031508545307     0.9999998778700737    0.9999998774721307
0.15076809118824547     0.9999998763397884    0.9999998755792567
0.15094101843280106     0.9999998747424421    0.9999998747915881
0.15110659473543925     0.9999998731936287    0.9999998747207671
0.15127315481861378     0.9999998716162312    0.9999998745022589
0.15144069868232465     0.9999998700096399    0.9999998733604079
0.1516008916041181      0.999999868454649     0.999999871129136
0.15193256351176782     0.9999998651754025    0.9999998647917804
0.15209987513639717     0.9999998634901107    0.9999998623764343
0.1522723379027897      0.9999998617307461    0.999999861171663
0.15243744972726486     0.9999998600250013    0.999999860963799
0.15260354533227635     0.9999998582877389    0.999999860869629
0.15277062471782418     0.9999998565183154    0.9999998599724217
0.15293868788390835     0.999999854716076     0.9999998577425504
0.15311190219175574     0.9999998528348067    0.9999998542731753
0.15327776555768574     0.9999998510104707    0.9999998506756014
0.15344461270415205     0.9999998491524401    0.9999998477756314
0.15361244363115473     0.9999998472600427    0.9999998461837185
0.15377292361623995     0.9999998454283244    0.9999998457817035
0.15393855474308843     0.999999843514792     0.999999845747611
0.15410516965047327     0.9999998415660439    0.9999998450760781
0.15427276833839443     0.9999998395813915    0.9999998430232441
0.15444551816807878     0.999999837509849     0.9999998394725248
0.15461091705584573     0.9999998355015579    0.9999998354794989
0.15477729972414903     0.9999998334564836    0.9999998319754684
0.15494466617298866     0.9999998313739199    0.9999998297957765
0.15510468167991087     0.9999998293587613    0.9999998290607249
0.1552698483285963      0.9999998272538098    0.9999998290133116
0.1554359987578181      0.9999998251105209    0.9999998285846601
0.15560313296757622     0.9999998229281719    0.999999826804441
0.15577125095787064     0.9999998207060283    0.9999998233927161
0.1559320180062477      0.9999998185555179    0.9999998191877182
0.15609793619638795     0.9999998163096775    0.9999998150037045
0.15626483816706455     0.9999998140231796    0.9999998120267594
0.15643272391827748     0.9999998116952727    0.9999998106968596
0.1566057608112536      0.9999998092663401    0.9999998105166734
0.15677144676231236     0.9999998069322673    0.999999810275435
0.15693811649390743     0.999999804566966     0.9999998087729384
0.15710577000603881     0.9999998021585565    0.9999998054603102
0.15726607257625283     0.9999997998280374    0.999999801010596
0.15743152628823004     0.9999997973938435    0.9999997962311313
0.1575979637807436      0.9999997949152857    0.9999997924998134
0.15776538505379348     0.9999997923914347    0.9999997905470971
0.1579379574686066      0.9999997897573414    0.9999997901088157
0.1581031789415023      0.9999997872040495    0.9999997900152113
0.15826938419493436     0.9999997846041658    0.999999788859266
0.1584365732289027      0.9999997819567356    0.9999997857816174
0.1586047460434074      0.999999779260786     0.999999780936877
0.15877806999967536     0.9999997764472593    0.9999997753406678
0.15894404301402587     0.9999997737193516    0.9999997709147361
0.15911099980891275     0.9999997709416029    0.99999976836358
0.159278940384336       0.9999997681130149    0.9999997676268992
0.15943953001784178     0.9999997653755286    0.9999997675904126
0.1596052707931108      0.9999997625163208    0.9999997667456195
0.15977199534891615     0.999999759604997     0.9999997639395201
0.15993970368525784     0.9999997566405332    0.9999997590282768
0.16011256316336275     0.999999753546845     0.9999997528790132
0.16027807169955022     0.9999997505480016    0.9999997475978842
0.16044456401627405     0.9999997474946964    0.9999997441834858
0.16061204011353425     0.9999997443858798    0.9999997429096312
0.160772165268877       0.9999997413780054    0.999999742822731
0.16093744156598297     0.9999997382365838    0.9999997423037873
0.16110370164362528     0.999999735038375     0.9999997398981114
0.16127094550180393     0.9999997317823045    0.9999997350866726
0.16160004983731652     0.9999997252593952    0.9999997225872508
0.1617660776758773      0.9999997219097525    0.9999997180191817
0.1619330892949744      0.9999997184998574    0.9999997158584023
0.16210108469460788     0.999999715028592     0.9999997154999593
0.16227423123600457     0.999999711407132     0.9999997152242178
0.16244002683548384     0.9999997078973859    0.9999997132158103
0.16260680621549944     0.999999704324924     0.9999997085765224
0.1627745693760514      0.9999997006886026    0.999999701803522
0.16293498159468595     0.9999996971722525    0.9999996949294121
0.1631005449550837      0.999999693545491     0.9999996892822056
0.1632670920960178      0.9999996898536461    0.9999996861976307
0.16343462301748823     0.9999996860953491    0.99999968541708
0.16360730508072185     0.9999996821739986    0.9999996853131797
0.16377263620203814     0.9999996783737964    0.9999996837905409
0.16393895110389073     0.9999996745052048    0.9999996795119518
0.16410624978627963     0.9999996705668164    0.9999996725771967
0.1642745322492049      0.9999996665571986    0.9999996645536539
0.16444796585389337     0.999999662373757     0.9999996575338875
0.16461404851666445     0.9999996583184121    0.9999996535757776
0.16478111495997186     0.9999996541898653    0.9999996523309888
0.1649491651838156      0.9999996499866449    0.9999996522673846
0.16510986446574194     0.9999996459194825    0.9999996511776901
0.1652757148894315      0.9999996416723759    0.999999647299834
0.1654425490936574      0.9999996373486898    0.9999996403102444
0.16561036707841964     0.999999632946915     0.9999996315699516
0.16578333620494506     0.9999996283541908    0.9999996232905766
0.1659489543895531      0.9999996239029274    0.9999996180835443
0.16611555635469746     0.9999996193716024    0.9999996160255996
0.16628314210037817     0.9999996147586677    0.9999996158472689
0.16644337690414146     0.9999996102960743    0.9999996151915822
0.16660876284966797     0.999999605636144     0.9999996118894419
0.16677513257573084     0.9999996008926997    0.9999996050795585
0.16711499073069241     0.9999995910264706    0.9999995861847675
0.1672801444371374      0.9999995861452925    0.9999995794887714
0.16744628192411873     0.9999995811770416    0.9999995762896922
0.1676134031916364      0.9999995761200943    0.9999995757386357
0.1677815082396904      0.9999995709727997    0.9999995754002753
0.16795476442950763     0.9999995656035623    0.9999995724234532
0.1681206696774074      0.9999995604004283    0.9999995656924944
0.16828755870584355     0.999999555104939     0.9999995558440166
0.16845543151481607     0.9999995497154051    0.999999545315669
0.16861595338187113     0.9999995445023822    0.9999995373272609
0.16878162639068944     0.9999995390605179    0.9999995328106852
0.16894828318004412     0.999999533522673     0.9999995316488053
0.1691159237499351      0.999999527887121     0.9999995315258335
0.16928871546158927     0.99999952204614      0.9999995292616428
0.16945415623132606     0.999999516418658     0.9999995230366676
0.1696205807815992      0.9999995106914191    0.9999995129570948
0.16978798911240867     0.9999995048623656    0.9999995012726475
0.16994804650130074     0.9999994992247163    0.9999994916147134
0.170113255031956       0.9999994933386157    0.9999994854276762
0.1702794473431476      0.9999994873479279    0.9999994832809588
0.17044662343487554     0.9999994812505392    0.999999483162608
0.17061478330713986     0.9999994750442981    0.9999994817534095
0.17077559223748673     0.9999994690400741    0.9999994766273987
0.17094155230959682     0.99999946277165      0.9999994668527117
0.17110849616224325     0.9999994563915519    0.9999994542504168
0.171276423795426       0.999999449897572     0.9999994421844903
0.171449502570372       0.9999994431233817    0.9999994336784077
0.17161523040340054     0.9999994365589389    0.9999994302947587
0.17178194201696545     0.999999429877694     0.9999994299104272
0.17194963741106672     0.9999994230773822    0.9999994290931135
0.17210998186325055     0.999999416499644     0.9999994247690133
0.1722754774571976      0.9999994096323329    0.9999994153189796
0.17244195683168098     0.9999994026431355    0.9999994020127587
0.1726094199867007      0.999999395529731     0.9999993882186551
0.17278203428348365     0.9999993881095577    0.9999993774885184
0.17294729763834915     0.9999993809208199    0.9999993724112051
0.17311354477375102     0.9999993736049544    0.9999993713841607
0.17328077568968925     0.999999366159584     0.9999993710501267
0.1734489903861638      0.9999993585822915    0.9999993674173152
0.17362235622440153     0.9999993506795332    0.9999993577984824
0.17378837112072193     0.9999993430220687    0.999999343795622
0.1739553697975786      0.9999993352297099    0.9999993284325712
0.17412335225497164     0.9999993272999822    0.9999993159138971
0.17428398377044726     0.9999993196306377    0.9999993091659202
0.17444976642768611     0.9999993116257075    0.9999993071736464
0.1746165328654613      0.9999993034805423    0.9999993070465847
0.1747842830837728      0.9999992951926123    0.9999993043607952
0.17495718444384756     0.999999286549828     0.9999992955475304
0.17512273486200486     0.9999992781781327    0.9999992813099937
0.17528926906069853     0.9999992696606997    0.9999992644076063
0.17545678703992856     0.9999992609949427    0.9999992494345551
0.17561695407724115     0.9999992527051514    0.9999992403513419
0.17578227225631696     0.999999244056765     0.999999236866376
0.1759485742159291      0.9999992352569806    0.999999236629929
0.1761158599560776      0.9999992263027327    0.9999992349438921
0.17628412947676242     0.9999992171909       0.999999227603479
0.17644504805552985     0.9999992083774685    0.9999992141688674
0.17677817127712742     0.9999991898178272    0.9999991783977404
0.17694620855873072     0.9999991802922107    0.9999991657243651
0.17711939698209725     0.9999991703579557    0.9999991599703265
0.17728523446354635     0.9999991607330779    0.9999991592509355
0.1774520557255318      0.9999991509389813    0.9999991583209654
0.17761986076805358     0.9999991409723777    0.9999991521610221
0.17778031486865797     0.9999991313335361    0.9999991392198702
0.17794592011102556     0.9999991212724404    0.9999991202605472
0.17811250913392945     0.999999111034698     0.999999100199099
0.17828008193736972     0.999999100616939     0.999999084466327
0.17845280588257323     0.999999089752068     0.9999990761063543
0.1786181788858593      0.9999990792275995    0.9999990743671882
0.17878453566968172     0.9999990685188266    0.9999990740161881
0.17895187623404046     0.9999990576222961    0.999999069303432
0.1791118658564818      0.9999990470865406    0.9999990573769436
0.17927700662068638     0.9999990360894803    0.9999990380255425
0.1794431311654273      0.9999990249005258    0.9999990157626578
0.17961023949070454     0.999999013516143     0.9999989966309518
0.1797783315965181      0.9999990019327382    0.9999989851057282
0.17993907276041426     0.9999989907308525    0.9999989814891435
0.18010496506607365     0.9999989790405643    0.9999989812999196
0.1802718411522694      0.9999989671470458    0.9999989784658722
0.18043970101900147     0.9999989550466221    0.9999989677917616
0.18061271202749674     0.9999989424299155    0.9999989475888079
0.1807783720940746      0.9999989302100574    0.9999989234298743
0.1809450159411888      0.99999891777893      0.9999989009733135
0.18111264356883935     0.9999989051327752    0.9999988859050178
0.18127292025457248     0.9999988929071393    0.9999988800080933
0.18143834808206882     0.999998880149924     0.9999988793551132
0.18160475969010154     0.9999988671734764    0.9999988777385171
0.18177215507867056     0.9999988540083746    0.9999988687809988
0.18194470160900278     0.999998840390208     0.999998849227397
0.18227607656636877     0.9999988137872815    0.9999987977520625
0.18244323971585627     0.999998800139364     0.9999987785048643
0.1826113866458801      0.9999987862543542    0.9999987692179576
0.18278468471766718     0.9999987717774246    0.9999987675949101
0.1829506318475368      0.9999987577541907    0.9999987666559074
0.18311756275794278     0.999998743487623     0.9999987588710535
0.18328547744888513     0.9999987289729863    0.9999987402745786
0.18344604119791003     0.9999987149381807    0.9999987141852248
0.18361175608869817     0.999998700291814     0.9999986851813665
0.18377845476002264     0.9999986853913496    0.9999986616691844
0.18394613721188344     0.9999986702319327    0.9999986486390713
0.18411897080550746     0.9999986544253117    0.9999986452320286
0.1842844534572141      0.9999986391164006    0.9999986449096221
0.18445091988945705     0.9999986235422944    0.9999986390913511
0.18461837010223633     0.999998607698017     0.9999986220724895
0.18477846937309822     0.9999985923804844    0.9999985956117852
0.18494371978572333     0.9999985763952642    0.9999985637272004
0.1851099539788848      0.9999985601338487    0.9999985355371531
0.1852771719525826      0.9999985435911434    0.999998517806923
0.18544537370681669     0.9999985267619682    0.9999985115834075
0.18560622451913342     0.9999985104890678    0.9999985112592722
0.18577222647321334     0.999998493509516     0.9999985079210815
0.1859392122078296      0.9999984762373655    0.9999984939003477
0.1861071817229822      0.9999984586673171    0.9999984669426848
0.186280302379898       0.9999984403505691    0.9999984309889729
0.18644607209489644     0.9999984226118935    0.9999983982643378
0.18661282559043119     0.999998404568966     0.9999983755240167
0.18678056286650224     0.9999983862163664    0.9999983657815783
0.18694094920065593     0.9999983684756806    0.9999983646426456
0.18710648667657281     0.9999983499662032    0.9999983628427537
0.18727300793302604     0.9999983311409296    0.9999983512257081
0.1874405129700156      0.999998311994321     0.9999983253864874
0.18761316914876836     0.999998292036353     0.9999982875971337
0.18777847438560374     0.9999982727145017    0.9999982502824935
0.18794476340297545     0.9999982530649606    0.9999982217047975
0.18811203620088346     0.9999982332168038    0.9999982071895755
0.18828029277932784     0.9999982130915244    0.9999982042027434
0.18845370049953544     0.9999981921140425    0.9999982032135021
0.18861975727782562     0.9999981717984305    0.9999981931979034
0.18878679783665214     0.9999981511352783    0.9999981679901557
0.188954822176015       0.9999981301178196    0.9999981294730887
0.18911549557346047     0.9999981097991075    0.9999980889959937
0.18928132011266913     0.9999980886000787    0.9999980545183368
0.18944812843241415     0.9999980670380343    0.9999980346436077
0.1896159205326955      0.9999980451060352    0.999998028901481
0.18978886377474008     0.9999980222429975    0.9999980285179574
0.18995445607486722     0.9999980001038067    0.9999980211550971
0.19012103215553072     0.999997977585645     0.9999979982514141
0.1902885920167306      0.9999979546813975    0.9999979593600402
0.19044880093601302     0.9999979325420687    0.9999979151282236
0.19061416099705866     0.9999979094423681    0.9999978742363043
0.19078050483864065     0.9999978859478797    0.9999978477217115
0.19094783246075897     0.9999978620513172    0.9999978377641207
0.19112031122464052     0.9999978371400112    0.9999978370675596
0.1912854390466046      0.9999978130222069    0.9999978324820972
0.19145155064910507     0.9999977884933108    0.9999978129035054
0.1916186460321419      0.9999977635458623    0.9999977749659079
0.19178672519571505     0.999997738172277     0.9999977252463675
0.19195995550105138     0.99999771172506      0.9999976760803942
0.19212583486447032     0.9999976861157291    0.9999976434699912
0.1922926980084256      0.9999976600710844    0.9999976292734073
0.19246054493291723     0.9999976335833657    0.9999976275192478
0.19262104091549143     0.9999976079815794    0.9999976250216631
0.19278668803982885     0.9999975812744947    0.999997608746891
0.19295331894470263     0.9999975541154895    0.9999975723789741
0.19312093363011273     0.9999975264966326    0.9999975203039533
0.19329369945728603     0.9999974977117188    0.9999974645299423
0.19345911434254193     0.999997469847003     0.9999974238234857
0.19362551300833417     0.9999974415132574    0.9999974029677032
0.19379289545466274     0.999997412702377     0.9999973985455566
0.19395292695907393     0.9999973848638957    0.99999739764835
0.19411810960524833     0.9999973558264845    0.9999973853526384
0.19428427603195905     0.9999973263370148    0.9999973520210871
0.19445142623920614     0.999997296594509     0.9999972989904529
0.1946195602269895      0.9999972663496358    0.9999972384903193
0.19478034327285554     0.9999972371163817    0.9999971894024794
0.19494627746048476     0.9999972066243062    0.9999971588555158
0.1951131954286503      0.9999971756173879    0.9999971489568465
0.1952810971773522      0.9999971440857867    0.999997148501967
0.1954541500678173      0.9999971112237609    0.9999971396207454
0.195619852016365       0.9999970794083063    0.9999971096503852
0.19578653774544905     0.9999970470552554    0.9999970565676908
0.19595420725506943     0.9999970141545159    0.9999969909971914
0.1961145258227724      0.999996982357983     0.9999969334295058
0.19627999553223857     0.9999969491893571    0.9999968935523125
0.19644644902224112     0.9999969154605767    0.9999968773400392
0.19661388629277998     0.9999968811613038    0.9999968758036895
0.19678647470508204     0.9999968454130984    0.9999968705659031
0.1969517121754667      0.9999968108090483    0.9999968452685299
0.1971179334263877      0.9999967756215901    0.9999967939483569
0.19745332726983872     0.9999967034539212    0.9999966552330695
0.19762666722359562     0.999996665535371     0.9999966038285079
0.19779265623543507     0.9999966288232528    0.9999965814520468
0.1979596290278109      0.9999965914932295    0.9999965779997017
0.19812758560072308     0.9999965535342846    0.9999965750958006
0.19828819123171781     0.9999965168494046    0.9999965545600169
0.19845394800447577     0.9999964785871757    0.9999965057290334
0.19862068855777007     0.9999964396833516    0.9999964336070478
0.1987884128916007      0.9999964001266706    0.9999963559086459
0.19896128836719457     0.9999963589066452    0.9999962928364686
0.199126812900871       0.9999963190087798    0.9999962609353615
0.1992933212150838      0.9999962784449111    0.9999962532112063
0.1994608133098329      0.9999962372035279    0.9999962522064808
0.19962095446266465     0.9999961973578979    0.9999962369071436
0.1997862467572596      0.9999961558021798    0.999996192512183
0.19995252283239084     0.9999961135562838    0.9999961196308268
0.20011978268805847     0.9999960706084554    0.9999960342805875
0.2002880263242624      0.9999960269467557    0.9999959599105999
0.20044891901854897     0.9999959847563391    0.9999959159074597
0.20061496285459873     0.9999959409583578    0.9999958999471001
0.20078199047118483     0.9999958966216528    0.9999958991062909
0.20095000186830728     0.9999958515457769    0.9999958892995401
0.2011231644071929      0.9999958045802589    0.9999958485167976
0.20128897600416118     0.9999957591198955    0.999995776007636
0.20145577138166576     0.9999957129020167    0.9999956842274308
0.20162355053970665     0.9999956659124177    0.9999955978267845
0.20178397875583018     0.9999956205082644    0.9999955412282644
0.2019495581137169      0.9999955731557351    0.9999955161245582
0.20211612125213996     0.9999955250137733    0.9999955131734359
0.20228366817109936     0.9999954760678244    0.9999955077701029
0.20245636623182195     0.9999954250657206    0.9999954737526835
0.20262171335062718     0.9999953757044969    0.9999954040684693
0.20295535892984656     0.9999952745001345    0.9999952090702515
0.20311532266780705     0.9999952252089909    0.9999951378311414
0.20328043754753072     0.9999951738004931    0.9999951003633265
0.20344653620779074     0.9999951215370518    0.9999950924793863
0.2036136186485871      0.9999950684034103    0.9999950904911095
0.2037816848699198      0.9999950143840591    0.9999950657466856
0.2039424001493351      0.9999949621848544    0.9999950048252978
0.20410826657051362     0.9999949077510868    0.999994907653891
0.20427511677222843     0.999994852413604     0.9999947976251986
0.2044429507544796      0.9999947961565473    0.9999947048797673
0.20461593587849403     0.9999947375442312    0.9999946499061401
0.204781570060591       0.9999946808189879    0.9999946347490762
0.20494818802322434     0.9999946231554973    0.9999946340220386
0.205115789766394       0.9999945645375453    0.9999946163630665
0.20527604056764628     0.9999945079095417    0.9999945619066675
0.20544144251066176     0.9999944488603625    0.999994465081778
0.20560782823421359     0.9999943888387314    0.9999943460430236
0.20577519773830175     0.9999943278280883    0.9999942368907103
0.2059477183841531      0.9999942642681966    0.9999941639747331
0.2061128880880871      0.9999942027722208    0.9999941377781951
0.2062790415725574      0.9999941402685574    0.9999941358393466
0.206446178837564       0.9999940767402944    0.9999941253585365
0.20661429988310698     0.999994012170258     0.999994076173842
0.20678757207041318     0.999993944914987     0.9999939751771307
0.20695349331580196     0.9999938803306002    0.9999938480444373
0.20712039834172707     0.9999938147157669    0.9999937244101552
0.20728828714818853     0.9999937480213265    0.9999936367079143
0.2074488250127326      0.9999936835898895    0.9999935988132371
0.20761451401903985     0.9999936164098977    0.9999935929002708
0.20778118680588342     0.9999935481253406    0.9999935876238979
0.20794884337326336     0.9999934787158893    0.9999935474846554
0.20812165108240652     0.9999934064078987    0.9999934514434619
0.20828710784963225     0.9999933364392333    0.9999933190095557
0.20845354839739433     0.9999932653198244    0.9999931796508422
0.20862097272569274     0.9999931930288414    0.9999930710003567
0.20878104611207376     0.9999931232001392    0.9999930160747184
0.20894627064021798     0.9999930503872915    0.9999930022612165
0.20911247894889856     0.999992976377931     0.999993000583272
0.20927967103811546     0.9999929011507386    0.9999929710291291
0.20944784690786866     0.9999928246840395    0.9999928866745702
0.20960867183570453     0.9999927508048236    0.9999927573778206
0.20977464790530356     0.9999926737779917    0.9999926036671378
0.20994160775543896     0.9999925954862815    0.9999924699899508
0.21010955138611068     0.9999925159075257    0.9999923880623528
0.2102826461585456      0.9999924330129583    0.9999923608373281
0.21044838998906315     0.9999923527983967    0.9999923598062631
0.21061511760011703     0.9999922712704811    0.9999923393751288
0.21078282899170725     0.999992188406543     0.9999922645149673
0.21094318944138007     0.9999921083654383    0.9999921364343902
0.2111087010328161      0.9999920249162657    0.9999919715499572
0.21127519640478842     0.9999919401057198    0.9999918161707598
0.21144267555729712     0.9999918539106439    0.9999917100225951
0.21161530585156904     0.9999917641288971    0.999991665728503
0.21178058520392354     0.9999916772724404    0.9999916618251748
0.21194684833681438     0.999991589005142     0.9999916503429284
0.21211409525024155     0.9999914993033485    0.9999915882065595
0.21228232594420507     0.9999914081430387    0.9999914580170138
0.21262173867374112     0.9999912213489129    0.9999911003987714
0.21278875334808678     0.9999911280063966    0.9999909724844326
0.21295675180296875     0.9999910331540703    0.999990912315831
0.21311739931593335     0.9999909417780997    0.9999909026452249
0.21328319797066114     0.9999908470096038    0.9999908973178552
0.21344998040592528     0.9999907507057454    0.9999908473191248
0.21361774662172575     0.9999906528383463    0.9999907246957934
0.21379066397928942     0.9999905509095212    0.9999905354835161
0.21395623039493572     0.999990452297769     0.9999903405617514
0.21412278059111833     0.9999903520863584    0.999990184388753
0.21429031456783726     0.9999902502464101    0.999990099332442
0.21445049760263885     0.999990151892612     0.9999900783028901
0.2146158317792036      0.9999900493585632    0.9999900767338114
0.21478214973630472     0.9999899451611254    0.9999900406506276
0.21494945147394215     0.9999898392707348    0.9999899305297888
0.21511773699211592     0.9999897316573328    0.9999897457668331
0.2152786715683723      0.9999896277020961    0.9999895379216694
0.2154447572863919      0.9999895193403284    0.999989348412786
0.21561182678494784     0.999989409220093     0.9999892281803975
0.21577988006404009     0.9999892973106419    0.9999891852089644
0.21595308448489559     0.9999891807620206    0.9999891828369479
0.21611893796383363     0.9999890679985749    0.9999891584753056
0.21628577522330805     0.9999889534091536    0.9999890614891577
0.21645359626331884     0.999988836962308     0.9999888791857621
0.21661406636141217     0.9999887244974964    0.9999886574052445
0.21677968760126873     0.9999886072653933    0.9999884392152807
0.21694629262166162     0.9999884881404382    0.9999882860023467
0.21711388142259086     0.9999883670904989    0.9999882190807048
0.21728662136528332     0.9999882410260686    0.999988210909023
0.21745201036605832     0.9999881190845806    0.999988197777793
0.2176183831473697      0.9999879951813437    0.9999881180073191
0.21778573970921744     0.9999878692835316    0.9999879444961207
0.21811090209084125     0.9999876210132673    0.9999874670290183
0.2182770426330711      0.9999874922744472    0.9999872760347791
0.2184441669558373      0.9999873614721496    0.9999871771039978
0.21861227505913983     0.9999872285723533    0.9999871556475941
0.21877303222052497     0.9999871002283759    0.9999871520969209
0.2189389405236733      0.9999869664743393    0.9999870968117027
0.21910583260735794     0.999986830586798     0.9999869444207171
0.21927370847157895     0.9999866925310489    0.9999867006924554
0.2194467354775632      0.9999865494775737    0.9999864156946613
0.21961241154163003     0.9999864114276709    0.9999861872941423
0.21977907138623318     0.9999862711708863    0.999986052964604
0.21994671501137267     0.999986128667511     0.9999860123104388
0.22010700769459476     0.9999859910681039    0.9999860110099694
0.22027245151958008     0.999985847654012     0.9999859731705237
0.22043887912510174     0.9999857019450327    0.9999858389969464
0.2206062905111597      0.9999855539005071    0.9999855981215592
0.22077885303898093     0.9999853997372424    0.9999852911894604
0.2209440646248847      0.9999852506387561    0.9999850230473735
0.22111025999132483     0.9999850991546714    0.9999848456178851
0.22127743913830134     0.9999849452433649    0.9999847762356519
0.22144560206581415     0.9999847888625243    0.9999847708530074
0.22161891613509016     0.9999846260340387    0.9999847438623877
0.22178487926244878     0.9999844685195018    0.9999846221916417
0.22195182617034373     0.9999843084844944    0.9999843821767
0.2222803366052889      0.9999839888701613    0.9999837669399222
0.222446067493566       0.9999838252304369    0.999983544939993
0.22261278216237945     0.9999836589778593    0.9999834413900986
0.22278048061172923     0.9999834900681965    0.9999834252303849
0.2229533302028422      0.9999833141949211    0.9999834118358335
0.22311882885203776     0.9999831440971488    0.9999833132125415
0.22328531128176968     0.9999829712913478    0.9999830867823309
0.22345277749203793     0.9999827957323226    0.9999827577369481
0.22361289276038876     0.9999826262410622    0.9999824250379978
0.2237781591705028      0.9999824496029994    0.9999821526111974
0.22394440936115323     0.9999822701626543    0.99998200441376
0.22411164333233996     0.999982087873892     0.9999819674397915
0.224279861084063       0.999981902689862     0.9999819637956849
0.2244407278938687      0.9999817238738739    0.9999818977373099
0.22460674584543758     0.9999815375499103    0.9999817012367016
0.2247737475775428      0.9999813482807791    0.999981375787162
0.22494173309018434     0.9999811560186834    0.9999809932442261
0.2251148697445891      0.9999809558738356    0.9999806528696698
0.22528065545707648     0.999980762320762     0.9999804561820177
0.22544742495010017     0.9999805657228902    0.9999803909271445
0.22561517822366023     0.9999803661783057    0.9999803882980305
0.22577558055530283     0.999980174672417     0.9999803440891352
0.2259411340287087      0.9999799751223776    0.9999801725992629
0.2261076712826509      0.9999797724251493    0.9999798532933235
0.22627519231712945     0.9999795665252202    0.9999794453729468
0.22644786449337118     0.9999793521671083    0.9999790508640187
0.2266131857276955      0.9999791448905103    0.9999787958986891
0.22677949074255618     0.9999789343424296    0.999978689762264
0.2269467795379532      0.9999787204660269    0.999978679022858
0.22711505211388655     0.9999785032035184    0.999978652063271
0.22728847583158313     0.9999782770343053    0.9999784934144321
0.22745454860736225     0.9999780582842623    0.9999781765762108
0.2277896455005296      0.9999776103567471    0.9999773139280276
0.227950334895464       0.9999773924201935    0.9999770090613886
0.22811617543216164     0.9999771653356357    0.9999768554877788
0.22828299974939562     0.9999769346683738    0.9999768274985441
0.22845080784716593     0.999976700357983     0.9999768155961714
0.22862376708669946     0.9999764564368862    0.9999766887653719
0.22878937538431554     0.9999762205612646    0.9999763915785601
0.228955967462468       0.9999759809725074    0.9999759489241975
0.2291235433211568      0.9999757376088647    0.9999754688320796
0.22928376823792818     0.9999755026880052    0.9999750986773611
0.2294491442964628      0.9999752579059499    0.9999748840832339
0.22961550413553372     0.9999750092815836    0.9999748254075682
0.229782847755141       0.9999747567518644    0.999974822131114
0.22995534251651148     0.9999744938707062    0.9999747313366226
0.23012048633596457     0.9999742397192817    0.9999744657127425
0.230286613935954       0.9999739815924892    0.9999740239755618
0.23045372531647973     0.9999737194259731    0.9999735018559139
0.23062182047754182     0.9999734531543902    0.999973042582004
0.23079506678036715     0.9999731760123991    0.9999727536436331
0.23096096214127504     0.9999729080296447    0.9999726625784133
0.2311278412827193      0.9999726358705893    0.9999726588983576
0.23129570420469991     0.9999723594685659    0.9999725951807612
0.23145621618476309     0.9999720926762683    0.9999723660294159
0.23162187930658948     0.9999718147452807    0.9999719326204981
0.2317885262089522      0.9999715325027972    0.9999713773000144
0.23195615689185128     0.9999712468561053    0.9999708485352907
0.23212893871651358     0.9999709503632109    0.9999704780558135
0.23229436959925842     0.9999706637216601    0.9999703323353486
0.23246078426253963     0.9999703726205438    0.9999703171776609
0.2326281827063572      0.9999700769828054    0.9999702824620377
0.23278823020825734     0.9999697916605901    0.9999700949083984
0.2329534288519207      0.99996949439453      0.9999696850869879
0.23311961127612038     0.9999691925014461    0.9999691087583856
0.2332867774808564      0.9999688859025173    0.999968511928582
0.23345492746612878     0.9999685745176287    0.9999680563845578
0.23361572650948376     0.9999682739219099    0.9999678343945054
0.23378167669460193     0.9999679607752515    0.9999677844246941
0.2339486106602564      0.9999676427506651    0.9999677748387896
0.23411652840644726     0.9999673197662577    0.9999676333967115
0.23428959729440132     0.9999669836051015    0.99996723932182
0.23445531524043797     0.999966658583809     0.9999666518359449
0.23462201696701096     0.9999663285072968    0.9999659955215595
0.2347897024741203      0.9999659932918645    0.9999654499900059
0.23495003703931222     0.9999656697523225    0.9999651471279735
0.23511552274626735     0.9999653326962473    0.9999650515171219
0.23528199223375884     0.9999649904093588    0.9999650482181598
0.23544944550178665     0.9999646428061942    0.9999649509629294
0.23562204991157765     0.9999642810231263    0.9999646030544868
0.2357873033794513      0.9999639313024976    0.9999640215164779
0.23595354062786128     0.9999635761701827    0.9999633147937653
0.23612076165680757     0.9999632155389274    0.9999626739317456
0.23628896646629022     0.9999628493201338    0.9999622587681263
0.23646232241753606     0.9999624682185045    0.9999621037222729
0.23662832742686452     0.9999620997587332    0.9999620959776747
0.2367953162167293      0.9999617256143519    0.9999620265812308
0.23696328878713044     0.9999613456949581    0.9999617222510506
0.23712391041561415     0.9999609790259384    0.9999611664839686
0.23745643973664438     0.9999602093257461    0.9999596913091099
0.23762418006796399     0.9999598155808745    0.9999591700057542
0.2377970715410468      0.9999594058593778    0.999958933909238
0.2379626120722122      0.9999590098338716    0.9999589055878585
0.23812913638391395     0.9999586077507145    0.9999588704763751
0.23829664447615204     0.9999582018213701    0.9999586235768915
0.2384568016264727      0.999957810136279     0.9999581008807671
0.2386221099185566      0.999957402150284     0.9999573326891827
0.23878840199117685     0.9999569879041045    0.9999565177808422
0.23895567784433341     0.9999565672917398    0.9999558765312048
0.2391239374780263      0.9999561402054443    0.9999555381727713
0.2392848461698018      0.9999557279891641    0.9999554619102003
0.2394509060033405      0.9999552986564538    0.9999554538770683
0.23961794961741556     0.9999548627255166    0.999955286359214
0.23978597701202695     0.9999544200862023    0.999954802098107
0.23995915554840153     0.9999539594902087    0.9999539907852866
0.24012498314285874     0.9999535142329403    0.9999531019099327
0.24029179451785226     0.9999530621383769    0.9999523437242747
0.2404595896733821      0.9999526030939244    0.9999518914358704
0.2406200338869946      0.9999521601063414    0.9999517528265751
0.2407856292423703      0.9999516987047085    0.9999517478833612
0.2409522083782823      0.9999512302289912    0.9999516357951924
0.24111977129473067     0.9999507545642126    0.9999512131355625
0.24129248535294223     0.9999502595943461    0.9999504159858951
0.2414578484692364      0.9999497811985406    0.9999494658469029
0.2416241953660669      0.9999492954847399    0.9999485850800497
0.24179152604343376     0.9999488023355445    0.9999479957537751
0.2419515057788832      0.9999483265364387    0.9999477663997691
0.24211663665609584     0.9999478309580776    0.9999477429629487
0.24228275131384486     0.9999473278202298    0.9999476834602097
0.2424498497521302      0.9999468170031341    0.9999473398976748
0.24261793197095183     0.9999462983852052    0.9999466082159653
0.24294454566652357     0.9999452767256882    0.9999446407048681
0.24311141186572738     0.9999447476080078    0.9999438760482222
0.24327926184546753     0.999944210439313     0.9999434948175792
0.24345226296697087     0.9999436515687874    0.9999434220524211
0.24361791314655684     0.9999431114437192    0.9999433973806288
0.24378454710667913     0.9999425631364588    0.9999431280448258
0.24395216484733773     0.999942006520622     0.9999424497434096
0.24411243164607896     0.9999414695234401    0.9999414730731028
0.24427784958658338     0.9999409103177475    0.9999403648143946
0.24444425130762415     0.9999403435431353    0.9999394457441079
0.24461163680920125     0.9999397703544337    0.9999389184865192
0.24478417345254155     0.9999391740020923    0.9999387686574901
0.24494935915396449     0.9999385977515381    0.9999387629361168
0.24511552863592373     0.9999380127818187    0.999938575640558
0.24528268189841929     0.9999374189482866    0.9999379780534515
0.2454508189414512      0.9999368161039439    0.9999369567444127
0.24562410712624638     0.9999361889470475    0.9999357039202038
0.2457900443691241      0.9999355827829277    0.9999346516427723
0.2459569653925382      0.999934967435248     0.9999339926191693
0.2461248701964886      0.9999343427537347    0.999933763722643
0.2462854240585216      0.9999337400200485    0.9999337547881406
0.24645112906231784     0.9999331123630351    0.9999336348814181
0.2466178178466504      0.9999324752057496    0.999933120197179
0.24678549041151931     0.9999318283947235    0.9999321276647722
0.24695831411815145     0.9999311554727045    0.999930802613452
0.24712378688286613     0.9999305051855043    0.9999295955652646
0.24729024342811717     0.9999298450717926    0.9999287553237949
0.2474576837539046      0.9999291749748533    0.9999283942287703
0.24761777313777456     0.9999285285453818    0.999928353298325
0.24778301366340774     0.9999278553716255    0.9999282947644482
0.24794923796957727     0.9999271720483415    0.9999278834849198
0.24811644605628314     0.9999264784156473    0.9999269555162403
0.24828463792352534     0.9999257743112199    0.999925624450463
0.24844547884885015     0.999925094937081     0.9999242961780002
0.24861147091593816     0.999924387561377     0.9999232277778893
0.24877844676356248     0.9999236695447942    0.9999226646877974
0.24894640639172316     0.9999229407218191    0.9999225371258131
0.24911951716164707     0.9999221825902865    0.9999225161083956
0.24928527698965355     0.9999214499821955    0.9999221995059907
0.24945202059819638     0.9999207063919342    0.9999213469289383
0.24961974798727554     0.9999199516507464    0.9999200009736953
0.2497801244344373      0.9999192235958597    0.9999185517204641
0.24994565202336227     0.9999184655555556    0.9999172850262844
0.2501121633928236      0.9999176961953232    0.9999165263843159
0.25027965854282125     0.9999169153432447    0.9999162873869475
0.25045230483458214     0.9999161031313456    0.9999162793105875
0.2506176001844256      0.9999153184632735    0.9999160651400635
0.25078387931480534     0.999914524949884     0.9999153216673797
0.25095114222572146     0.9999137206594241    0.9999140016744648
0.2511193889171739      0.9999129043423685    0.9999123738078604
0.2512927867503896      0.9999120553028575    0.9999108565022402
0.25145883364168786     0.9999112348338658    0.9999099242146281
0.2516258643135225      0.9999104021087453    0.9999095739601339
0.2517938787658935      0.9999095569284413    0.9999095531357272
0.25195454227634706     0.9999087415802098    0.999909420933782
0.25212035692856377     0.999907892700133     0.9999087865392728
0.25228715536131685     0.9999070311437066    0.9999075126572816
0.25245493757460624     0.9999061567076301    0.9999058068311176
0.2526278709296589      0.9999052471681236    0.9999040866059705
0.2527934533427941      0.9999043683673091    0.9999029176209405
0.25296001953646574     0.9999034764572827    0.9999023865130225
0.25312756951067367     0.9999025712304284    0.9999023124126124
0.2532877685429642      0.9999016981069436    0.9999022520099254
0.2534531187170179      0.9999007890407124    0.9999017510480745
0.253619452671608       0.9998998664366698    0.9999005685435135
0.2537867704067345      0.9998989300829918    0.9998988251455981
0.25395923928362407     0.9998979561252878    0.9998969122530562
0.2541243572185963      0.9998970152632046    0.9998954776812067
0.2542904589341048      0.9998960604219005    0.999894710633533
0.2544575444301497      0.9998950913852739    0.9998945289589994
0.25462561370673087     0.9998941079339628    0.9998945099068303
0.2547988341250753      0.9998930851281433    0.9998940925926367
0.25496470360150236     0.9998920968933271    0.999892974859148
0.2551315568584658      0.9998910940102584    0.9998911975535648
0.2552993938959655      0.9998900762552655    0.9998891718429515
0.25545987999154784     0.9998890946096048    0.9998875295317415
0.2556255172288933      0.9998880727120238    0.9998865079243034
0.25579213824677516     0.9998870357179364    0.9998861781952264
0.2559597430451933      0.9998859833994692    0.9998861688310583
0.2561324989853747      0.9998848890087824    0.9998858872868496
0.2562979039836387      0.9998838318611771    0.9998849110521946
0.2564642927624391      0.9998827591556076    0.9998831699099926
0.2566316653217758      0.9998816706599294    0.9998810106642051
0.2567916869391951      0.9998806210477851    0.9998791079543959
0.2569568596983775      0.9998795292565228    0.9998777828428448
0.2571230162380963      0.9998784255874881    0.9998772389581417
0.2572901565583514      0.999877305672347     0.9998771894287102
0.2574582806591429      0.9998761692494796    0.999877047294489
0.257619053818017       0.9998750731292868    0.9998763026927141
0.2577849781186543      0.9998739321831298    0.9998747008472211
0.25811977806153796     0.999871599626244     0.9998702032465279
0.25829282106501117     0.9998703779247581    0.9998684698021306
0.258458513126567       0.9998691977140953    0.9998676785251891
0.25862518896865905     0.9998680001346664    0.9998675390695211
0.2587928485912875      0.9998667849135717    0.9998674803404674
0.2589531572719985      0.9998656129736678    0.9998669017400854
0.2591186170944728      0.9998643930403113    0.9998654293517016
0.2592850606974834      0.9998631551741367    0.9998631769508192
0.2594524880810304      0.9998618990967006    0.9998606823735804
0.25962506660634055     0.9998605928437323    0.9998585941455158
0.2597902941897333      0.9998593311729166    0.9998574871625403
0.2599565055536624      0.999858050988375     0.9998571869238684
0.26012370069812785     0.9998567520060287    0.9998571748627414
0.2602918796231296      0.9998554339375091    0.9998567363587247
0.2604652096898946      0.9998540633933108    0.9998553092948707
0.2606311888147422      0.9998527393667463    0.999853026977037
0.2607981517201261      0.9998513959462988    0.9998503462643897
0.26096609840604634     0.9998500328379216    0.999848003045282
0.2611266941500492      0.9998487182662646    0.9998466101647006
0.2612924410358152      0.9998473500388344    0.9998461010766738
0.26145917170211763     0.9998459618275817    0.9998460804442649
0.26162688614895635     0.9998445533329247    0.9998457998933351
0.2617997517375583      0.9998430887909667    0.9998445695775664
0.26196526638424283     0.9998416742730603    0.9998423510139806
0.2621317648114637      0.9998402391640179    0.9998395198341049
0.2622992470192209      0.9998387831586303    0.9998368343155969
0.2624593782850607      0.9998373793309215    0.999835057953931
0.2626246606926637      0.9998359182368062    0.9998342564300801
0.2627909268808031      0.9998344359507091    0.9998341607325962
0.26295817684947875     0.9998329321619355    0.9998340238626536
0.2631264105986908      0.9998314065553419    0.9998330762929372
0.26328729340598545     0.9998299388131555    0.9998310921426897
0.26345332735504334     0.9998284141331888    0.9998282042043803
0.2636203450846375      0.9998268673304664    0.999825174208274
0.26378834659476796     0.9998252980563943    0.9998228266482617
0.2639614992466617      0.9998236665201753    0.9998216076581585
0.264127300956638       0.9998220906723235    0.999821375939169
0.2642940864471507      0.9998204919580762    0.9998213278333096
0.2644618557181997      0.9998188700214145    0.9998205996157867
0.2646222740473313      0.9998173061062988    0.9998187855185737
0.2647878435182261      0.9998156784890853    0.9998158817916618
0.26495439676965726     0.9998140272691267    0.999812586811874
0.26512193380162474     0.999812352083165     0.9998098026655265
0.26529462197535547     0.9998106103495843    0.9998081459311604
0.26545995920716875     0.9998089283323324    0.9998076873258653
0.2656262802195184      0.9998072219539899    0.9998076772159139
0.26579358501240435     0.9998054908439395    0.9998071851254989
0.2659535388633729      0.9998038219592775    0.9998056122134279
0.2661186438561047      0.9998020850360213    0.9998027831348771
0.2662847326293728      0.9998003230010477    0.9997992799532296
0.26645180518317724     0.9997985354765652    0.9997960455190779
0.26661986151751804     0.9997967220791479    0.9997938998577338
0.26678056690994145     0.999794973495868     0.9997930824992876
0.266946423444128       0.999793153883549     0.9997930104997845
0.2671132637588509      0.9997913080118082    0.9997927786272345
0.2672810878541102      0.9997894354899907    0.9997914778281812
0.2674540630911327      0.9997874888221909    0.9997886657559792
0.26778629546187915     0.9997837019353897    0.9997813799889818
0.2679538873180569      0.9997817675159816    0.9997787159347797
0.2681141282323173      0.9997799026522148    0.9997775016077749
0.2682795202883409      0.9997779620523928    0.9997772810517671
0.2684458961249008      0.9997759936164118    0.9997771915207744
0.26861325574199696     0.9997739969391044    0.9997761918769559
0.2687857665008564      0.9997719212560265    0.9997736108179027
0.26895092631779843     0.9997699172107606    0.9997699118635762
0.26911706991527684     0.9997678845222219    0.999765899526378
0.26928419729329156     0.9997658227779599    0.9997626635626219
0.2694523084518426      0.9997637315596826    0.9997608899188849
0.2696255707521569      0.9997615656834546    0.9997604468256384
0.26979148211055376     0.9997594745826947    0.9997604177370832
0.269958377249487       0.9997573535792414    0.9997596214547884
0.2701262561689566      0.9997552022122451    0.9997572877143391
0.2702867841465087      0.9997531281537849    0.9997536780684582
0.27045246326582406     0.9997509700796224    0.9997493606944611
0.27061912616567574     0.9997487811493827    0.9997455823946128
0.27078677284606373     0.999746560892846     0.9997432458547253
0.27095957066821497     0.9997442529377455    0.9997424565836724
0.2711250175484488      0.9997420244839422    0.9997424395303431
0.271291448209219       0.9997397641922149    0.9997419251838738
0.27145886265052555     0.9997374715828187    0.9997399224881541
0.2716189261499146      0.9997352617215082    0.9997364322006603
0.27178414079106694     0.9997329622315385    0.9997318876199537
0.2719503392127556      0.9997306299317814    0.9997275593908671
0.2721175214149805      0.9997282643332064    0.9997245618133069
0.27228568739774184     0.9997258649394921    0.9997232921052301
0.27244650243858576     0.999723551643594     0.9997231616898279
0.2726124686211929      0.9997211448483748    0.9997229448228201
0.2727794185843364      0.9997187037575141    0.9997214381896896
0.2729473523280162      0.9997162278653348    0.999718075166515
0.27312043721345924     0.999713654439689     0.9997131712872784
0.27328617115698484     0.9997111696171523    0.9997083555469662
0.27345288888104674     0.9997086494728193    0.999704700591189
0.273620590385645       0.9997060934915009    0.9997028784083017
0.2737809409483259      0.9997036297434085    0.9997025332041659
0.27394644265277        0.9997010664034985    0.9997024648682943
0.27411292813775046     0.9996984667265333    0.9997013323429237
0.2742803974032672      0.9996958301880516    0.9996982836983
0.2744530178105472      0.9996930898258748    0.9996933573226578
0.2746182872759098      0.9996904443910646    0.9996881013346427
0.2747845405218088      0.9996877615742339    0.9996837311001278
0.27495177754824407     0.9996850408414948    0.9996812144951642
0.27511999835521567     0.999682281651401     0.9996804984471703
0.2752933703039505      0.9996794142469988    0.9996804766145855
0.2754593913107679      0.9996766456951055    0.9996795900440495
0.2756263960981216      0.999673838156465     0.9996767638323957
0.2757943846660117      0.9996709949309661    0.9996719469019847
0.2759550222919844      0.9996682601849349    0.9996664734404885
0.2761208110597203      0.9996654152847256    0.9996614328207012
0.27628758360799255     0.9996625302983533    0.9996581905480342
0.2764553399368011      0.9996596046216906    0.9996570013933497
0.2766282474073729      0.999656564057986     0.9996569463891013
0.27679380393602726     0.9996536287347128    0.9996563955112856
0.2769603442452179      0.9996506520643209    0.9996539967952255
0.27712786833494496     0.9996476334304657    0.9996493649824724
0.2772880414827546      0.9996447242106274    0.9996436408404261
0.2774533657723275      0.9996416976243195    0.9996379289798357
0.2776196738424367      0.9996386284427584    0.9996338459489343
0.2777869656930822      0.9996355160376887    0.9996320061638785
0.2779552413242641      0.9996323597714994    0.9996317645691248
0.2781161660135285      0.9996293172253475    0.9996315539797782
0.2782822418445562      0.999626152339415     0.9996297816949018
0.2784493014561202      0.9996229429498181    0.9996256215968016
0.27861734484822054     0.9996196884069447    0.9996195282430534
0.2787905393820841      0.9996163063407992    0.9996129000045968
0.27895638297403025     0.99961304121524      0.9996079909321428
0.2791232103465127      0.999609730268223     0.9996054278724051
0.27929102149953156     0.9996063728379355    0.9996048679109055
0.279451481710633       0.9996031370135546    0.9996048153029471
0.2796170930634977      0.9995997710219469    0.9996035042934142
0.2797836881968987      0.9995963579050787    0.9995997608535906
0.279951267110836       0.9995928969892439    0.999593691030215
0.2801239971665366      0.9995893004690589    0.9995865247853964
0.2802893762803197      0.9995858290175652    0.9995807255569811
0.2804557391746391      0.9995823090988902    0.9995772673764645
0.2806230858494949      0.9995787400272361    0.9995761926275574
0.2807830815824333      0.9995753009787509    0.9995761825962791
0.28094822845713496     0.9995717235981461    0.9995753491089127
0.2811143591123729      0.9995680964231392    0.9995721681133627
0.2812814735481472      0.9995644187561468    0.9995663182036667
0.28144957176445784     0.9995606898898272    0.999558934444259
0.28161031903885103     0.9995570961869975    0.9995523397288697
0.28177621745500747     0.9995533585390538    0.999547686176327
0.2819430996517002      0.9995495690395717    0.9995457370735397
0.28211096562892934     0.9995457353236059    0.9995455519906948
0.28228398274792166     0.9995417559210458    0.9995450884126272
0.2824496489249966      0.9995379149260071    0.9995424557669036
0.2826162988826078      0.9995340206118699    0.9995369167040984
0.28294421541698556     0.9995262675071238    0.9995219336502312
0.28310964935497895     0.9995223101638393    0.9995162240400846
0.28327606707350866     0.999518297912034     0.9995133837516721
0.28344346857257474     0.9995142299510843    0.999512849494951
0.283616021213404       0.9995100029748712    0.9995126841647558
0.2837812229123159      0.9995059236396215    0.9995106898086744
0.2839474083917641      0.9995017877593537    0.9995056639754023
0.28411457765174863     0.9994975945180529    0.9994980014933911
0.2842827306922695      0.9994933430877243    0.9994895638385813
0.2844560348745536      0.9994889259883243    0.9994825685176197
0.28462198811492034     0.9994846622873466    0.9994789275034747
0.28478892513582343     0.9994803395470633    0.999477995702858
0.28495684593726284     0.9994759569239725    0.9994779601312209
0.28511741579678485     0.999471733652003     0.999476532446237
0.28528313679807        0.9994673413418148    0.9994720571226927
0.2854498415798915      0.9994628883318775    0.9994644925671597
0.28561753014224933     0.9994583737635722    0.999455477957959
0.2857903698463704      0.9994536832243955    0.9994473453154016
0.2859558586085741      0.9994491564559501    0.9994425561812702
0.28612233115131414     0.999444567278817     0.9994409085550884
0.2862897874745905      0.9994399148190078    0.9994408655795768
0.28644989285594946     0.9994354324501704    0.9994399920761985
0.28661514937907157     0.9994307706347813    0.9994362366484532
0.28678138968273004     0.9994260447204958    0.9994290129762461
0.2869486137669248      0.999421253818345     0.9994196002077704
0.28711682163165597     0.9994163970269454    0.9994105274236753
0.28727767855446973     0.9994117168842355    0.999404427518257
0.28744368661904673     0.9994068501401958    0.9994016094352624
0.28761067846416005     0.9994019166769711    0.9994012509587369
0.28777865408980974     0.9993969155780826    0.999400870536271
0.28795178085722256     0.9993917203719661    0.9993977309404823
0.288117556682718       0.999386706744185     0.9993909437280291
0.28828431628874973     0.9993816265595433    0.999381305103373
0.28845205967531784     0.9993764897598618    0.9993712744499964
0.28861245211996855     0.9993715407711078    0.9993638925154094
0.2887779957063825      0.9993663943135328    0.9993599163123736
0.2889445230733328      0.9993611774878891    0.999359042801951
0.28911203422081944     0.9993558892812736    0.9993589465399858
0.2892846965100693      0.9993503955541926    0.9993566214096697
0.2894500078574017      0.9993450945982831    0.999350513958794
0.28961630298527047     0.9993397212070899    0.999340905370488
0.2897835818936756      0.9993342743483268    0.9993300358915062
0.289951844582617       0.9993287529745952    0.9993209108782954
0.29012525841332165     0.9993230176588573    0.9993155547024989
0.2902913213021089      0.9993174824161134    0.9993141343235352
0.29045836797143243     0.9993118715859419    0.9993141202547148
0.29062639842129234     0.9993061841014155    0.9993124185780091
0.29078707792923486     0.999300704234325     0.9993071025679073
0.2909529085789406      0.9992950061651662    0.999297677878873
0.2911197230091827      0.9992892304112792    0.9992861615113379
0.29128752121996115     0.9992833758866989    0.9992756967026637
0.29146047057250274     0.9992772943844739    0.999268813555603
0.29162606898312693     0.9992714261041074    0.9992664377709923
0.29179265117428743     0.9992654779805382    0.9992663362182389
0.2919602171459843      0.9992594489084458    0.9992653222892045
0.2921204321757638      0.9992536410655237    0.9992608893068136
0.2922857983473065      0.9992476018507344    0.9992519462105448
0.29261948203200094     0.9992352763630108    0.9992282313026281
0.2927919669063795      0.9992288315182039    0.999219593022709
0.29295710083884063     0.9992226138913959    0.9992159084835324
0.2931232185518381      0.999216312089878     0.9992154118551427
0.2932903200453719      0.9992099249702823    0.9992149880126991
0.2934584053194421      0.9992034513734829    0.9992112991781523
0.2936316417352755      0.9991967277641111    0.9992023969766523
0.2937975272091915      0.9991902400320265    0.9991901319345083
0.2939643964636438      0.9991836647320007    0.9991772814171035
0.2941322494986325      0.9991770006855802    0.9991673422705509
0.29429275159170376     0.9991705813148307    0.9991624006496728
0.2944584048265383      0.9991639072726689    0.999161243740649
0.29462504184190913     0.9991571536175905    0.9991611492401888
0.2947926626378163      0.9991503164838929    0.9991584361971281
0.29496543457548663     0.999143215193795     0.9991504345142468
0.2951308555712396      0.9991363642460352    0.9991382220539029
0.2952972603475289      0.9991294210510766    0.9991243496324752
0.29546464890435453     0.9991223843125308    0.9991126175020403
0.29562468651926277     0.9991156070098557    0.9991059451397708
0.2957898752759342      0.9991085603671734    0.9991037341672993
0.295956047813142       0.9991014189144745    0.9991037160562468
0.2961232041308861      0.9990941813317997    0.9991020053423894
0.2962913442291666      0.999086846280168     0.999095446614756
0.29645213338552967     0.9990797800756184    0.9990841371712981
0.2966180736836559      0.9990724339900077    0.9990695667356441
0.2967849977623185      0.9990649891475789    0.9990558699016722
0.29695290562151744     0.9990574441856002    0.9990465727659102
0.2971259646224796      0.9990496083175361    0.9990427701111105
0.29729167268152434     0.9990420483551632    0.9990425190085762
0.2974583645211054      0.9990343869326308    0.9990415760152954
0.29762604014122285     0.9990266226630603    0.9990361715720988
0.2977863648194229      0.9990191443713954    0.999025519707899
0.2981183002398856      0.9990034908200496    0.9989953325384627
0.29828574362092153     0.9989955064333215    0.9989839184895992
0.29845833814372064     0.9989872141023987    0.9989783061745157
0.29862358172460235     0.9989792152875742    0.9989774324566987
0.2987898090860204      0.9989711095879336    0.9989771081056507
0.29895702022797477     0.9989628955691321    0.9989730354054884
0.29912521515046553     0.9989545717771204    0.9989627784384344
0.2992985612147195      0.9989459281633395    0.9989468300943093
0.29946455633705604     0.9989375889575585    0.9989303615722175
0.2996315352399289      0.9989291386152074    0.9989171776020646
0.2997994979233381      0.9989205756581249    0.9989100374110585
0.2999601096648299      0.9989123281686568    0.9989083358669053
0.30012587254808487     0.9989037550160883    0.9989083047817049
0.3002926192118762      0.9988950679416208    0.998905391306357
0.30046034965620394     0.9988862654435925    0.998896290293638
0.30063323124229485     0.9988771248512376    0.9988805561796025
0.30079876188646837     0.998868308492356     0.9988629449396643
0.3009652763111782      0.9988593958220294    0.9988476027014441
0.3011327745164244      0.9988503648713334    0.9988381780550442
0.3012929217797532      0.9988416682752542    0.9988351157669522
0.30145822018484514     0.9988326279585007    0.9988350489883503
0.30162450237047345     0.9988234677928491    0.9988332962821178
0.3017917683366381      0.9988141861341956    0.9988256899271603
0.3019600180833391      0.9988047813148536    0.9988111072448901
0.3021209168881227      0.9987957226258012    0.9987932222385358
0.3022869668346695      0.9987863070728826    0.9987755589853752
0.30245400056175265     0.9987767667642429    0.9987631362305336
0.30262201806937217     0.9987671000028486    0.9987577309551352
0.3027951867187548      0.9987570626258023    0.998757152097817
0.3029610044262201      0.9987473802135384    0.9987562567363085
0.3031278059142217      0.9987375696879988    0.9987500710461198
0.30329559118275967     0.9987276293223146    0.9987364093638171
0.30362161097776397     0.9987081062163221    0.9986987068301817
0.3037881802266841      0.9986980244928642    0.9986836953631424
0.30395573325614056     0.9986878095796917    0.9986760103729427
0.30412843742736023     0.9986772028209981    0.9986744594181295
0.3042937906566625      0.9986669730387874    0.9986742156322654
0.3044601276665011      0.9986566084063078    0.9986696383594255
0.304627448456876       0.9986461071380409    0.9986573847453835
0.3047874183053335      0.998635996138186     0.9986392606634045
0.3049525392955542      0.9986254862301467    0.9986182832429497
0.3051186440663113      0.9986148380941424    0.9986005216881636
0.3052857326176047      0.9986040499158385    0.998590025655991
0.30545380494943447     0.9985931198563284    0.99858686315487
0.3056145263393468      0.9985825938894851    0.9985868864515014
0.3057803988710224      0.9985716543005644    0.9985842485695563
0.3059472551832343      0.998560571211673     0.9985743059110895
0.3061150952759826      0.9985493427548656    0.9985562081700775
0.306288086510494       0.9985376851942268    0.9985330392855782
0.306453726803088       0.998526442140653     0.9985127766625497
0.3066203508762184      0.9985150520306889    0.9984994016734446
0.3067879587298851      0.9985035129668623    0.9984942439749228
0.30694821564163444     0.998492402912947     0.9984939611556047
0.30711362369514694     0.9984808638915662    0.9984925560658793
0.3072800155291958      0.9984691911506127    0.9984844798714686
0.30744739114378106     0.9984573659196814    0.9984675291247207
0.3077850937145604      0.9984332497172969    0.9984210356672504
0.3079512533095278      0.9984212564222479    0.9984043997090455
0.30811839668503144     0.998409106566157     0.9983965649318842
0.30828652384107147     0.998396798052282     0.9983953817401031
0.3084598021388747      0.9983840203474711    0.9983946764059155
0.3086257294947605      0.998371696516616     0.9983878934663927
0.30879264063118267     0.9983592119903024    0.9983718027103489
0.3089605355481412      0.9983465646352838    0.998348141797794
0.3091210795231823      0.9983343867676738    0.9983240792596253
0.3092867746399866      0.9983217312443126    0.9983043204517436
0.3094534535373272      0.9983089109369252    0.9982935740204715
0.30962111621520416     0.9982959236766841    0.9982909634363254
0.30979393003484434     0.9982824409763793    0.9982908554283785
0.30995939291256713     0.9982694394729108    0.9982859756065147
0.3101258395708263      0.9982562689788831    0.998271716159639
0.31029327000962176     0.9982429272892906    0.9982484067946816
0.31045334950649983     0.9982300832754575    0.9982226937685336
0.3106185801451411      0.9982167351060258    0.9981996677484102
0.31078479456431873     0.9982032137874826    0.9981854030952741
0.31095199276403274     0.9981895170795652    0.9981805891072311
0.31112017474428305     0.9981756427118232    0.9981805401824617
0.31128100578261597     0.9981622831637725    0.998177896866735
0.31144698796271203     0.998148401273156     0.9981665325736108
0.31161395392334446     0.9981343397366023    0.9981448072527226
0.3117819036645132      0.9981200962481321    0.9981167102261909
0.3119550045474452      0.9981053112739112    0.9980893276508604
0.3121207544884598      0.998091054037883     0.9980715233868853
0.31228748821001073     0.9980766127780452    0.9980640921383088
0.312455205712098       0.9980619851522913    0.9980634517798145
0.3126155722722679      0.998047903226573     0.9980621539157964
0.3127810899742009      0.9980332706034881    0.9980530633185167
0.3129475914566703      0.9980184496302149    0.9980329028211623
0.313115076719676       0.9980034379294428    0.9980043287260416
0.3132877131244449      0.997987855636756     0.9979740720122029
0.31345299858729647     0.9979728538592718    0.9979523128867458
0.3136192678306844      0.9979576663090405    0.9979414624540827
0.3137865208546086      0.9979422835021237    0.9979394463881415
0.3139547576590692      0.9979267028676947    0.9979390320110171
0.31412814560529295     0.9979105317723754    0.9979313404046152
0.3142941826095993      0.9978949376532714    0.9979123110461814
0.31446120339444195     0.9978791432297637    0.9978833687216464
0.31462920795982097     0.9978631458868121    0.9978515832768052
0.3147898615832826      0.9978477447637395    0.9978268323129961
0.31495566634850747     0.9978317428135746    0.9978123428947993
0.31512245489426866     0.9978155355675945    0.997808367290369
0.3152902272205662      0.9977991203676643    0.9978084009164733
0.31546315068862696     0.997782082493315     0.9978030274337376
0.3156287232147703      0.9977656552914792    0.9977863093050768
0.31579527952145        0.9977490176587777    0.9977580170123267
0.315962819608666       0.9977321668932575    0.9977243057879657
0.3161230087539646      0.9977159470922279    0.9976957654675337
0.31628834904102643     0.9976990939469176    0.9976769426531522
0.3164546731086246      0.9976820252945805    0.9976700773750606
0.3166219809567591      0.9976647383905612    0.9976698761888998
0.3167944399466568      0.9976467953837418    0.9976667717416731
0.3169595479946371      0.9976294986647145    0.9976529743444832
0.3171256398231537      0.9976119812171159    0.9976262657714575
0.3172927154322067      0.9975942402528671    0.9975913687266789
0.317460774821796       0.9975762729469108    0.9975576675852346
0.3176339853531486      0.9975576262675472    0.997533724507722
0.3177998449425837      0.9975396478225784    0.9975241392887179
0.31796668831255515     0.9975214405174201    0.9975232222191571
0.31813451546306293     0.9975030014832595    0.9975215948587324
0.3182949916716533      0.9974852527804724    0.9975106781127058
0.3184606190220069      0.9974668134306766    0.9974859148323342
0.31862723015289685     0.9974481399403141    0.9974505345917495
0.3187948250643231      0.9974292293979344    0.9974135540409329
0.3189675711175126      0.9974096037972298    0.9973846028854647
0.3191329662287847      0.9973906853220432    0.9973708325827156
0.3192993451205931      0.9973715272764607    0.9973681594119769
0.31946670779293784     0.9973521267056714    0.9973677806786749
0.3196267195233652      0.9973334625287739    0.997359854254701
0.3197918823955557      0.9973140975927227    0.9973379227450474
0.31995802904828263     0.9972944877520211    0.9973030577661733
0.32012515948154585     0.9972746298695802    0.9972633400509623
0.32029327369534544     0.9972545207651973    0.9972297675097415
0.32045403696722763     0.9972351641473504    0.997210765645703
0.32061995138087307     0.9972150565189829    0.9972046043082652
0.3207868495750548      0.997194694808789     0.9972046708070433
0.32095473154977294     0.9971740757849126    0.9971997461785104
0.3211277646662542      0.9971526791135692    0.9971801533373129
0.32129344684081806     0.9971320526492012    0.9971463966211809
0.32146011279591824     0.9971111658896281    0.9971047348493467
0.3216277625315548      0.9970900155504737    0.9970665324789382
0.32178806132527393     0.9970696602294514    0.9970423715163048
0.32195351126075633     0.997048514305653     0.9970324155737322
0.32211994497677504     0.9970271019450034    0.9970317944578212
0.3222873624733301      0.9970054198119435    0.9970292461694815
0.32245993111164833     0.9969829193469569    0.9970133862777829
0.32262514880804916     0.9969612326642249    0.9969818379068082
0.3229585355424598      0.9969170376495127    0.9968965524945274
0.32312670458046966     0.9968945224977134    0.996865449613175
0.3233000247602427      0.9968711607001897    0.9968509344669328
0.3234659939980984      0.996848639518521     0.9968491652492606
0.32363294701649037     0.9968258357317443    0.9968479313392153
0.3238008838154187      0.9968027458552017    0.9968352154785616
0.32396146967242967     0.9967805235722376    0.9968067379012646
0.32412720667120376     0.9967574409635722    0.9967638517356009
0.3242939274505142      0.9967340693630335    0.99671762161038
0.32446163201036105     0.996710405234858     0.9966807311498391
0.32463448771197106     0.9966858510679653    0.996660683507334
0.3247999924716637      0.99666218502425      0.9966564297421635
0.3249664810118926      0.996638223421634     0.9966563408253608
0.32513395333265793     0.9966139626723826    0.9966472962413578
0.32529407471150584     0.9965906182682391    0.9966222805435531
0.325459347232117       0.9965663697853607    0.9965803811468472
0.3256256035332645      0.9965418192565038    0.996531253400244
0.3257928436149483      0.9965169630432982    0.9964883585720298
0.32596106747716846     0.9964918190902922    0.9964620395579231
0.32612194039747117     0.9964676342558946    0.996453587980126
0.3262879644595371      0.9964425162736931    0.9964535493931688
0.3264549723021394      0.9964170857933627    0.9964487470055043
0.32662296392527795     0.9963913389161211    0.9964275219753972
0.3267961066901798      0.9963646266494935    0.9963856610597046
0.3269618985131642      0.9963388801134468    0.9963345661158733
0.327128674116685       0.9963128136272946    0.9962863553240436
0.3272964335007421      0.996286423229984     0.9962534959877153
0.32745684194288177     0.996261028697254     0.996240382432734
0.3276224015267847      0.9962346531717584    0.9962391399732088
0.3277889448912239      0.9962079503309199    0.9962369026850177
0.32795647203619943     0.9961809161530016    0.9962200549398106
0.3281291503229382      0.996152867445696     0.9961813012618308
0.32846078879311735     0.9960984725311474    0.9960762034811822
0.3286280836990114      0.9960707686551538    0.9960360796378459
0.3287963623854418      0.9960427216687785    0.9960164017980077
0.3289697922136354      0.9960136259661211    0.9960130691961895
0.3291358710999116      0.9959855815482384    0.9960121687731767
0.3293029337667241      0.9959571904075383    0.9959983045314938
0.32947098021407295     0.995928448345086     0.9959630065289677
0.3296316757195044      0.9959007902747473    0.9959122904363528
0.3297975223666991      0.9958720670240362    0.9958551006230887
0.32996435279443015     0.9958429893941365    0.9958081711461119
0.3301321670026975      0.995813553125521     0.9957818111701792
0.33030513235272807     0.9957830157835875    0.9957749139875023
0.33047074676084126     0.99575358736126      0.9957750475157657
0.3306373449494908      0.9957237966879825    0.9957654283155024
0.3308049269186767      0.9956936394426085    0.9957345893243498
0.3309651579459451      0.9956646252995126    0.9956852851010928
0.3311305401149768      0.9956344931921464    0.9956249943106439
0.33129690606454476     0.9956039910585571    0.995571088877953
0.33146425579464905     0.9955731145175637    0.9955368027953982
0.3316367566665166      0.9955410828375224    0.9955245892132828
0.33180190659646674     0.9955102204127315    0.9955245333959677
0.33196804030695326     0.9954789799685756    0.9955190499456008
0.3321351577979761      0.9954473587212596    0.9954937713833077
0.33230325906953523     0.9954153891641049    0.9954446289364708
0.3324765114828576      0.9953822279614656    0.9953790822569594
0.33264241295426256     0.9953502708773314    0.9953196634520663
0.3328092982062038      0.9953179228713432    0.9952787335978863
0.3329771672386814      0.9952851792031154    0.9952616585726088
0.33313768532924165     0.9952536760194182    0.9952602639872045
0.3333033545615651      0.9952209625472257    0.9952577623715909
0.3334700075744249      0.9951878493988622    0.9952376972222319
0.333637644367821       0.9951543317628073    0.9951923345709093
0.33381043230298035     0.9951195635675029    0.9951259573098273
0.33397586929622225     0.9950860634672956    0.9950606864860824
0.33414229007000046     0.9950521546884756    0.9950111137423344
0.33430969462431503     0.9950178323471844    0.9949864413712713
0.3344697482367122      0.9949848161985687    0.9949819780338829
0.33463495299087265     0.9949505303465249    0.9949816201255507
0.3348011415255694      0.9949158269218716    0.9949672732782359
0.33496831384080245     0.9948807009702688    0.9949273057703296
0.33513646993657187     0.9948451474739138    0.9948637481470565
0.3352972750904239      0.9948109399380335    0.9947952597850852
0.3354632313860392      0.9947754220089543    0.9947355995413811
0.33563017146219076     0.9947394724589701    0.9947002241523818
0.3357980953188786      0.9947030861992484    0.9946895682940143
0.3359711703173297      0.9946653465279015    0.9946898276008316
0.3361368943738634      0.9946289828724992    0.9946802617009112
0.33630360221093347     0.9945921782473672    0.9946457949387869
0.33647129382853985     0.9945549274914377    0.9945844742865863
0.33663163450422884     0.9945190938476364    0.9945128888101361
0.336797126321681       0.994481886645008     0.9944452922976388
0.33696360191966956     0.9944442292393557    0.994400388323967
0.3371310612981945      0.9944061163992359    0.9943829239054018
0.33730367181848253     0.994366585479103     0.9943821676267754
0.3374689313968532      0.9943285031633798    0.9943771206718482
0.3376351747557602      0.9942899611535602    0.9943493279294573
0.33780240189520355     0.9942509541466231    0.9942922526144243
0.33797061281518326     0.9942114767740086    0.9942152650844913
0.33814397487692616     0.9941705364219814    0.994137477283361
0.33830998599675166     0.9941310891261823    0.9940847234775738
0.3384769808971135      0.9940911884445804    0.994061067317296
0.33864495957801166     0.9940508297914721    0.9940583226623072
0.3388055873169924      0.9940120061543811    0.9940563467218504
0.3389713661977364      0.9939716992614888    0.994034664034379
0.3391381288590167      0.9939309077143273    0.9939824641163227
0.33930587530083334     0.9938896258052506    0.9939053000325736
0.3394787728844132      0.9938468126675232    0.9938208665355157
0.33964431952607566     0.993805567702511     0.9937579993842968
0.33981084994827443     0.9937638274857017    0.9937249885269074
0.33997836415100957     0.993721586225655     0.9937178926203233
0.3401385274118273      0.9936809587112492    0.9937179689756372
0.3403038418144083      0.9936387773059475    0.9937028603257476
0.3404701399975256      0.9935960901803672    0.9936573445577892
0.3406374219611792      0.9935528914614281    0.9935821969254185
0.3408056877053691      0.9935091752016874    0.9934945720201599
0.34096660250764166     0.9934671202892348    0.9934221861912199
0.34113266845167745     0.9934234631341196    0.9933762285827328
0.34129971817624954     0.9933792836844447    0.9933609757286039
0.341467751681358       0.993334575910491     0.9933611923864137
0.34164093632822967     0.9932882147298232    0.9933514337002627
0.34180677003318394     0.9932435508927365    0.9933125519548471
0.3419735875186745      0.9931983537612874    0.9932405904392437
0.34214138878470146     0.993152617222082     0.9931496321716423
0.34230183910881096     0.9931086270820544    0.9930683970031987
0.3424674405746837      0.9930629597081528    0.9930111376179134
0.34263402582109276     0.9930167481773513    0.99298735394743
0.3428015948480382      0.9929699862947389    0.9929856018833408
0.3429743150167468      0.9929214942733361    0.9929809001207929
0.343139684243538       0.9928747863120372    0.9929499167293291
0.34330603725086556     0.9928275230293246    0.9928834484517927
0.3436333598846759      0.9927337083672475    0.9927018336320232
0.3437984968723856      0.9926859640469645    0.9926321329911607
0.34396461764063163     0.9926376533830017    0.9925973128770187
0.344131722189414       0.9925887700167764    0.9925909915952771
0.3442998105187327      0.9925393075125148    0.9925902517529868
0.344460547906134       0.9924917335544455    0.9925697158839764
0.3446264364352985      0.9924423522247832    0.9925130522209874
0.34479330874499936     0.9923924230744567    0.9924235538002251
0.34496116483523653     0.992341913140289     0.9923224780736736
0.34513417206723696     0.9922895407706553    0.9922364715583893
0.34529982835732        0.9922390949058922    0.9921900503704794
0.3454664684279394      0.9921880530373848    0.9921772737657096
0.34563409227909503     0.992136408356528     0.9921778705612575
0.3457943651883333      0.992086744290703     0.992164392752291
0.34595978923933474     0.9920351912002495    0.9921160500025318
0.34612619707087255     0.99198302990453      0.992030249341564
0.3462935886829467      0.9919302535015619    0.9919247158940961
0.34646613143678406     0.9918755293873176    0.9918266101831827
0.34663132324870405     0.9918228273340011    0.9917665752578908
0.3467974988411604      0.9917695045328072    0.9917444042572823
0.34696465821415307     0.9917155539867807    0.9917440075507873
0.3471328013676821      0.9916609686122437    0.9917361065557374
0.34730609566297427     0.991604376313205     0.9916925822213862
0.34747203901634904     0.9915498646045009    0.9916097358078
0.34763896615026013     0.9914947123339347    0.9915010840153379
0.3478068770647076      0.9914389123221218    0.991395901787446
0.34796743703723765     0.9913852510221388    0.9913249171418507
0.34813314815153096     0.9913295548669606    0.9912920683655695
0.3482998430463606      0.9912732054946748    0.9912883301145482
0.34846752172172657     0.9912161956326615    0.9912853258162261
0.3486403515388557      0.9911570885558951    0.9912514257706788
0.34880583041406743     0.9911001649056586    0.9911756775151512
0.3489722930698155      0.9910425750329769    0.9910666509544434
0.3491397395060999      0.9909843115703978    0.9909521236836057
0.3492998350004669      0.9909282919279382    0.9908670768430446
0.3494650816365972      0.9908701464833214    0.9908207542437666
0.34963131205326375     0.9908113219784145    0.9908106036455795
0.3497985262504667      0.9907518109533973    0.990810794959976
0.34996672422820596     0.9906916058589719    0.9907882996613244
0.3501275712640278      0.990633707857954     0.99072567042969
0.35029356944161283     0.9905736219282665    0.990621070257763
0.3504605513997342      0.9905128363704145    0.9904991470558092
0.35062851713839194     0.9904513435420046    0.9903937314216982
0.35080163401881287     0.9903875967613305    0.9903292918578512
0.3509673999573164      0.9903262161626027    0.9903107390294379
0.35113414967635626     0.9902641582799097    0.9903115114256732
0.3513018831759325      0.9902013795922875    0.9902972492419999
0.3514622657335913      0.990141017887294     0.990244310168202
0.35162779943301337     0.9900783731111004    0.9901448694684849
0.35179431691297175     0.9900150013851639    0.990018803090803
0.3519618181734665      0.9899508946648025    0.9899003550629919
0.3521344705757244      0.9898844361148396    0.989819073479103
0.3522997720360649      0.9898204435664839    0.9897887616094654
0.3524660572769417      0.9897557095928872    0.9897872992004065
0.35263332629835487     0.9896902260422024    0.9897805437785205
0.3528015791003044      0.9896239846630877    0.9897355664892273
0.35297498304401714     0.9895553222692977    0.9896364778586214
0.35330807282814414     0.9894223031888983    0.9893790974653994
0.35347609339101216     0.9893546387428163    0.989285776586849
0.3536367630119628      0.9892895776986794    0.9892441788396906
0.35380258377467655     0.989222063187899     0.9892378944078221
0.3539693883179267      0.9891537697146605    0.9892360370823396
0.3541371766417132      0.9890846888144271    0.9892018384211161
0.3543101161072629      0.9890130815284421    0.9891121069408829
0.35447570463089517     0.9889441303663774    0.9889836809344787
0.3546422769350638      0.9888743852422156    0.98884527227541
0.35480983301976876     0.988803837584413     0.9887349609609072
0.35497003816255635     0.9887360169810344    0.9886777100603157
0.3551353944471072      0.9886656367989969    0.9886630388587617
0.35530173451219427     0.9885944478508604    0.9886639505453824
0.35546905835781767     0.988522441460862     0.9886410546102605
0.3556415333452043      0.9884478003461813    0.9885637628500392
0.3558066573906736      0.9883759411382663    0.9884396228290382
0.3559727652166792      0.988303257954345     0.9882937002175236
0.35613985682322113     0.9882297420126341    0.9881660961453579
0.3563079322102994      0.9881553844281816    0.9880877575028264
0.3564811587391409      0.9880783170292257    0.9880623436666124
0.356647034326065       0.9880041082699618    0.9880633306955465
0.35681389369352545     0.987929051228393     0.9880476909118951
0.35698173684152223     0.9878531369127173    0.9879820143985244
0.35714222904760157     0.9877801567143232    0.9878664662122085
0.35730787239544415     0.9877044637236384    0.9877160940880935
0.35747449952382304     0.9876279219210933    0.9875736339029527
0.35764211043273825     0.9875505073608719    0.9874761582902412
0.3578148724834167      0.9874702698260494    0.9874360002219659
0.35798028359217776     0.9873930222130813    0.9874340902351688
0.3581466784814752      0.9873148946017865    0.9874269237052812
0.35831405715130893     0.9872358775750155    0.9873751605840715
0.3584740848792252      0.9871599280972763    0.9872693430736555
0.3588054263991206      0.9870014138752836    0.9869623130976602
0.35897257282987277     0.9869208027477697    0.9868439439364963
0.3591407030411613      0.986839276118141     0.9867854294759224
0.35930148231053244     0.9867608983993658    0.986775452719011
0.3594674127216668      0.9866795816874657    0.9867752287082295
0.3596343269133375      0.9865973424553801    0.9867412238369083
0.35980222488554453     0.9865141709394353    0.9866458030535439
0.3599752739995148      0.9864279758701777    0.986490271150859
0.3601409721715676      0.9863449914246769    0.9863242062335075
0.3603076541241567      0.9862610673388349    0.9861867241908453
0.3604753198572822      0.9861761937299907    0.9861086339658954
0.3606356346484903      0.9860946134082289    0.9860880554282172
0.36080110058146164     0.9860099713513134    0.9860895959002627
0.3609675502949693      0.9859243727605048    0.9860679462343113
0.36113498378901326     0.9858378076372641    0.9859871970565555
0.3613075684248205      0.9857480937421188    0.985838551722638
0.3614728021187103      0.9856617372165856    0.985665362268028
0.3616390195931365      0.9855744068033611    0.9855087937232881
0.361806220848099       0.9854860923860185    0.985407873451221
0.3619744058835978      0.9853967837311634    0.985371403627029
0.36214774206085987     0.9853042395603947    0.9853714916582785
0.3623137272962045      0.9852151421844266    0.9853575024068968
0.3624806963120854      0.9851250430977938    0.9852873149201087
0.3626486491085027      0.9850339319482886    0.9851485464029918
0.3628092509630026      0.9849463553374428    0.9849757687142938
0.36297500395926574     0.9848555033300275    0.9848031156832262
0.3631417407360652      0.9847636321439156    0.9846801426672672
0.36330946129340097     0.9846707313112012    0.9846257715444853
0.3634823329925         0.9845744690492335    0.9846205096935604
0.36364785374968156     0.9844818479390296    0.9846151882728799
0.36381435828739944     0.984388189761834     0.9845609567665193
0.3639818466056537      0.9842934837387488    0.9844348423321939
0.36414198398199055     0.9842024685617645    0.9842624536588397
0.36430727250009065     0.9841080461913837    0.9840756552206414
0.36447354479872707     0.9840125683094643    0.9839289732292377
0.3646408008778998      0.983916024010128     0.9838521083855732
0.3648090407376089      0.9838184022602564    0.9838361640623798
0.36496992965540054     0.9837245658427011    0.983837227157722
0.36513596971495543     0.9836272309036208    0.9838029303638768
0.36530299355504664     0.9835288107256387    0.9836977166905099
0.3654710011756742      0.9834292941476485    0.9835237563845264
0.365644159938065       0.9833261817946664    0.9833167258673845
0.36580996775853836     0.9832269265010095    0.9831486914507379
0.36597675935954804     0.9831265666365763    0.9830488218805665
0.3661445347410941      0.9830250909100595    0.9830189458660749
0.36630495918072276     0.9829275678497851    0.983020909920144
0.36647053476211466     0.9828264055064315    0.9830000733063952
0.3666370941240429      0.9827241195207241    0.9829120982369421
0.36680463726650736     0.9826206984745697    0.9827475310502914
0.3669773315507351      0.982513538181473     0.9825337865859254
0.3671426748930454      0.9824104048755833    0.9823447385514908
0.3673090020158921      0.982306128340031     0.9822184478518183
0.3674763129192751      0.9822006970277297    0.9821690936472461
0.36763627288074074     0.9820993914033972    0.9821673838873742
0.36780138398396955     0.9819943021312563    0.982158517308514
0.36796747886773473     0.9818880503103284    0.9820904643243656
0.3681345575320363      0.9817806242679394    0.9819408447085188
0.3684633314797946      0.9815676295337028    0.9815275125664309
0.3686291941244782      0.9814593648166637    0.981368629617744
0.36879604054969817     0.9813499061786426    0.9812898296040463
0.36896387075545445     0.9812392416886538    0.9812764736884652
0.369136852102974       0.9811245901012227    0.981275348780219
0.3693024825085761      0.9810142467602447    0.9812250065522428
0.3694690966947146      0.9809026892706607    0.9810916890339716
0.36963669466138943     0.9807899055726804    0.9808856434112699
0.36979694168614685     0.9806815508528617    0.9806684795539213
0.3699623398526674      0.9805691855827932    0.980483545382298
0.37012872179972434     0.980455586215501     0.9803776090198484
0.3702960875273176      0.9803407403926359    0.9803488419015602
0.37046860439667406     0.9802217540839173    0.9803512303136005
0.37063377032411315     0.9801072592412713    0.9803190132757078
0.3707999200320886      0.9799915091384755    0.9802065106305609
0.3709670535206004      0.9798744912783207    0.9800101470538874
0.3711351707896485      0.9797561930219975    0.9797724486637401
0.37130843920045986     0.9796336467223135    0.9795558663720877
0.3714743566693538      0.9795157040504184    0.9794274989392151
0.371641257918784       0.9793964720377633    0.979383825219848
0.3718091429487506      0.9792759379055824    0.9793852839058049
0.37196967703679984     0.9791601171022597    0.979367157121444
0.3721353622666122      0.9790399991658816    0.9792746572417997
0.37230203127696093     0.9789185706017528    0.9790909317841495
0.372469684067846       0.978795818494842     0.9788486202919308
0.37264248800049427     0.9786686554490136    0.9786087512524674
0.37280794099122516     0.9785462912700617    0.9784501646708952
0.3729743777624924      0.9784225946056934    0.9783825126617091
0.373141798314296       0.9782975524022259    0.9783774700584089
0.3733018679241822      0.9781774220935749    0.9783712937818922
0.3734670886758315      0.9780528307968854    0.9783013657591572
0.3736332932080172      0.9779268854620449    0.9781363308380067
0.3738004815207392      0.9777995729003989    0.9778957284884052
0.37396865361399756     0.9776708797775802    0.9776413980563283
0.37412947476533853     0.9775472174228608    0.9774520238588721
0.37429544705844275     0.9774189824688493    0.977349162246268
0.3744624031320833      0.9772893583211055    0.9773274934373051
0.3746303429862602      0.9771583315092034    0.9773293022614247
0.37480343398220023     0.9770226127882903    0.977278310342575
0.3749691740362229      0.976892015271858     0.9771328635899307
0.37513589787078183     0.976760006010726     0.9768984202994095
0.37530360548587716     0.9766265713960208    0.9766301190769069
0.3754639621590551      0.9764983770210596    0.9764126412015183
0.37562946997399627     0.9763654386793034    0.9762782278593621
0.37579596156947376     0.976231066360852     0.976236912884367
0.3759634369454876      0.9760952463220705    0.9762399032514747
0.37613606346326467     0.9759545874079824    0.976209141227752
0.3763013390391243      0.975819266332163     0.9760880051098655
0.37646759839552024     0.9756824883972696    0.9758667551441875
0.3766348415324526      0.9755442395817242    0.9755900022894871
0.37680306844992123     0.9754045057092551    0.975334316536828
0.3769764465091531      0.9752597854667198    0.9751636265779872
0.3771424736264675      0.9751205253073622    0.9751041796594644
0.37730948452431834     0.974979770452194     0.9751042192488671
0.37747747920270547     0.9748375065775046    0.975087123056224
0.3776381229391752      0.9747008276684035    0.9749900678440349
0.37780391781740813     0.9745591075564562    0.9747847663659185
0.37813845891548303     0.9742710983362409    0.9742258222138773
0.3783113724965519      0.9741211577747901    0.974019375949043
0.3784769351357033      0.9739768998838525    0.9739313292427142
0.37864348155539107     0.9738310997128794    0.9739216749259415
0.37881101175561516     0.9736837426485786    0.9739173035687335
0.37897119101392185     0.973542196643593     0.9738450329195069
0.37913652141399173     0.9733954251658875    0.9736622505811594
0.379302835594598       0.9732470876481146    0.9733870578889677
0.37947013355574055     0.9730971693348281    0.9730878556695204
0.3796384152974195      0.9729456553115269    0.9728481828679847
0.379799346097181       0.9728000875080551    0.9727236239033225
0.3799654280387057      0.9726491688128494    0.972692420286484
0.38013249376076674     0.972496645118637     0.972696149526941
0.38030054326336415     0.9723425013676104    0.9726489344651483
0.38047374390772476     0.9721828724455103    0.9724841190140975
0.38063959361016797     0.9720292926571587    0.9722178590158896
0.3808064270931475      0.9718740830290131    0.9719050817473379
0.3809742443566634      0.9717172283574266    0.9716329505321756
0.3811347106782619      0.971566557596587     0.971473249014502
0.3813003281416236      0.9714103436856525    0.9714181080988853
0.3814669293855216      0.9712524754526661    0.9714209836396159
0.381634514409956       0.9710929375520498    0.97139458551362
0.38180725057615356     0.9709277191838485    0.9712590570311685
0.3819726358004337      0.9707687905511624    0.9710095231247745
0.3821390048052502      0.9706081824707605    0.9706896006519483
0.38230635759060305     0.97044588527675      0.9703863729266298
0.3824663594340385      0.9702900306749754    0.97018707954442
0.3826315124192372      0.9701284371767018    0.9700996360883205
0.3827976491849722      0.9699651393980864    0.9700937283767099
0.3829647697312436      0.9698001216145825    0.9700850114300635
0.3831328740580513      0.9696333679342332    0.9699866234576975
0.38329362744294154     0.9694731865850426    0.9697723091640581
0.38345953196959504     0.9693071323616259    0.9694569679438133
0.38362642027678484     0.9691393325594864    0.969124994872158
0.38379429236451096     0.9689697711389744    0.9688681127348364
0.3839673155940003      0.9687941932735324    0.968736474777432
0.3841329878815723      0.9686252984353161    0.9687152161277005
0.38429964394968064     0.9684546317738968    0.9687167698100522
0.3844672837983253      0.9682821770993081    0.9686465682399804
0.3846275727050525      0.9681165500289686    0.9684589942303592
0.38479301275354294     0.9679448448789938    0.9681531837479614
0.3849594365825697      0.9677713420449521    0.967803572395006
0.3851268441921328      0.9675960251906098    0.9675073557035505
0.3852994029434591      0.9674144835540788    0.9673318140647167
0.38546461075286803     0.9672398832191628    0.9672859837147776
0.38563080234281333     0.9670634586623369    0.9672912551163902
0.38579797771329494     0.9668851933984829    0.9672490781098189
0.38596613686431286     0.9667050707705516    0.9670827967596969
0.38613944715709403     0.9665185794831566    0.9667734773504202
0.38630540650795775     0.9663391859266981    0.9664122595783605
0.38647234963935784     0.966157924648394     0.9660865394720761
0.38664027655129424     0.9659747788411054    0.9658784899781575
0.38680085252131324     0.9657988820213553    0.9658068795903971
0.38696657963309544     0.9656165526746829    0.9658080586888259
0.387133290525414       0.9654323289904768    0.9657871035798722
0.38730098519826894     0.9652461940155649    0.9656533147684263
0.38747383101288707     0.9650534736961869    0.9653666962380077
0.3876393258855878      0.9648681215053528    0.9650008835462279
0.38780580453882485     0.9646808476695453    0.9646428624339232
0.38797326697259826     0.9644916350869369    0.9643882957203498
0.3881333784644543      0.9643099454631043    0.9642796217716442
0.38829864109807344     0.9641216055996742    0.9642676767327799
0.38846488751222896     0.9639313171923276    0.9642637948217747
0.38863211770692085     0.9637390674914513    0.9641657500948807
0.38880033168214906     0.9635448372838902    0.9639192002661345
0.38896119471545987     0.9633582929938802    0.9635704528750866
0.38912720889053387     0.9631649506616263    0.9631839118043464
0.38929420684614424     0.9629696151018862    0.9628740424401608
0.389462188582291       0.9627722687131927    0.962707460711367
0.38963532146020086     0.9625679652502337    0.9626712723773194
0.38980110339619334     0.9623714721434189    0.9626761447682082
0.3899678691127222      0.9621729576076343    0.9626085325349583
0.39013561860978735     0.9619724038909366    0.9623945258185068
0.3902960171649351      0.9617798219771055    0.9620586304958177
0.3904615668618461      0.9615802156999852    0.9616555277810763
0.3906281003392934      0.9613785602166951    0.9613034215279276
0.3907956175972771      0.9611748376289161    0.9610879149472858
0.39096828599702393     0.9609639285092871    0.9610195490238486
0.39113360345485343     0.9607611186388443    0.9610255277235529
0.39129990469321924     0.9605562310678168    0.960987472386997
0.39146718971212136     0.9603492477488591    0.9608127159696807
0.3916354585115598      0.9601401504525189    0.9604806347580983
0.39180887845276147     0.9599237083045651    0.9600472634329726
0.39197494745204575     0.9597155400913411    0.9596639744168252
0.3921420002318664      0.959505247145328     0.9594094542252112
0.39231003679222337     0.9592928110868271    0.9593124575053522
0.39247072241066294     0.9590888172045744    0.9593120172330213
0.3926365591708657      0.9588774089107142    0.9592956115831851
0.39280337971160484     0.9586638473392006    0.9591571237005007
0.39297118403288034     0.9584481139641963    0.9588525685378263
0.3933097440170402      0.9580100567737015    0.9580008542919868
0.39347633231869783     0.9577931332038797    0.957694930473793
0.39364390440089175     0.9575740084047328    0.9575530172892972
0.3938041255411683      0.957363630587671     0.9575361064584432
0.393969497823208       0.9571455993821024    0.9575362829415054
0.3941358538857841      0.9569253567942811    0.9574370126561395
0.39430319372889655     0.9567028840048789    0.9571697097260017
0.39447568471377215     0.9564725863358146    0.9567449751684409
0.39464082475673035     0.9562511716092542    0.9563009054412362
0.3948069485802249      0.9560275151228913    0.955941457827896
0.3949740561842558      0.9558015891076259    0.9557446797898025
0.39514214756882304     0.9555733840102341    0.9556992495368918
0.3953153900951535      0.9553371868739908    0.9557062811396487
0.39548128167956653     0.9551100591946995    0.9556331203213695
0.3956481570445159      0.9548806416620442    0.9553940872496263
0.3958160161900016      0.9546489150212414    0.9549914133656695
0.39597652439356995     0.9544264365512213    0.9545412557105292
0.3961421837389015      0.9541958934166823    0.9541355324346718
0.3963088268647693      0.9539630310130826    0.953884208041896
0.39647645377117346     0.9537278299440892    0.953802331576174
0.39664923181934086     0.9534843866956739    0.9538090457428424
0.39681465892559087     0.9532503332790507    0.9537683695420095
0.39698106981237724     0.9530139304305268    0.9535733528715795
0.3971484644796999      0.9527751586096136    0.9531982591392468
0.3973085082051052      0.952545958806192     0.9527404477194951
0.3974737030722737      0.9523084432020518    0.9522913712466652
0.39763988171997855     0.952068548477226     0.9519792528933922
0.39780704414821977     0.9518262549510914    0.9518484069329907
0.3979751903569973      0.9515815427525885    0.9518420126768452
0.39813598562385744     0.9513465951097646    0.951832427215962
0.3983019320324807      0.951103161280311     0.9516953671857231
0.39846886222164035     0.9508572984800998    0.9513697038977913
0.3986367761913363      0.9506089866967492    0.9508963785307489
0.3988098413027955      0.9503520051055626    0.9503884987546785
0.3989755554723373      0.9501049351671157    0.9500204924277657
0.3991422534224155      0.9498554053267116    0.9498374707886453
0.39930993515303        0.9496033954280352    0.9498080512254384
0.3994702659417271      0.9493614860339491    0.9498132544451371
0.3996357478721873      0.949110830982962     0.949718313106943
0.3998022135831839      0.9488576855882195    0.9494361627140238
0.3999696630747168      0.9486020295535015    0.9489794210445673
0.400142263708013       0.9483374412377911    0.9484453466706343
0.40030751339939175     0.9480831034073063    0.9480201254844428
0.4004737468713069      0.9478262440232166    0.9477747950286977
0.4006409641237583      0.9475668426467931    0.9477084836358937
0.40080916515674614     0.947304878644374     0.9477186820096417
0.4009825173314971      0.9470338015244089    0.9476508122444492
0.40114851856433065     0.946773162795332     0.9474007085982526
0.4013155035777005      0.9465099356714674    0.946959075865377
0.40148347237160675     0.9462441140936994    0.9464219774228707
0.4016440902235956      0.9459889446104518    0.9459614728108885
0.4018098592173477      0.9457245834219956    0.9456555068302352
0.4019766119916361      0.9454576177503573    0.9455454758731682
0.40214434854646086     0.9451880268226694    0.9455500167932405
0.4023172362430488      0.9449090519269212    0.9455163752504
0.4024827729977194      0.9446408851411983    0.9453153506981761
0.4026492935329263      0.9443700824372842    0.9449078816797721
0.4029769512224953      0.9438341706575937    0.9438616765236748
0.4031422557380844      0.9435622560958578    0.9434891283176893
0.40330854403420974     0.9432876746935033    0.9433221640699387
0.40347581611087147     0.943010405414667     0.9433074393086072
0.4036440719680695      0.9427304270289312    0.9433029622660918
0.40380497688335015     0.9424616666865221    0.9431667020497158
0.403971032940394       0.9421832606561777    0.9428173471149052
0.4041380727779742      0.9419021353585664    0.9422915086084888
0.4043060963960907      0.941618269432552     0.9417228225327403
0.4044792711559705      0.9413245597242055    0.9412638102736501
0.4046450949739328      0.9410422292668125    0.9410377325090572
0.4048119025724315      0.9407571473737277    0.940993636463684
0.4049796939514665      0.9404692925506228    0.9410030556210625
0.4051401343885841      0.940193022512273     0.9409109671021637
0.4053057259674649      0.9399068282220363    0.9406111528750647
0.40547230132688206     0.9396178508548083    0.9401078738596845
0.40563986046683553     0.9393260687868857    0.9395165609126183
0.40581257074855226     0.9390241615914044    0.938994297819865
0.40597793008835154     0.9387340030626358    0.9386987520566823
0.4061442732086872      0.9384410290319661    0.9386103395125489
0.40631160010955913     0.9381452177438497    0.9386221081477566
0.4064715760685137      0.9378613638399699    0.9385724457558747
0.40663670316923145     0.9375673034296255    0.9383312180305274
0.4068028140504856      0.9372703956284117    0.9378661357573351
0.406969908712276       0.9369706185527473    0.9372660899156647
0.4071379871546028      0.9366679501194914    0.9366969559376399
0.4072987146550122      0.9363774600395212    0.9363247171208324
0.4074645932971849      0.9360765272261337    0.9361649999320725
0.40763145571989384     0.9357726833480698    0.9361591630138429
0.4077993019231391      0.9354659157294374    0.9361463483781142
0.4079722992681476      0.9351485427259262    0.9359547296445088
0.4081379456712387      0.9348435204061459    0.9355329847478289
0.4084721898190299      0.934224651738265     0.9343243869830603
0.40863245284127625     0.9339263055921508    0.9338826317432599
0.4087978670052858      0.9336172717909906    0.9336559302534878
0.40896426494983174     0.9333052723455689    0.9336200467846331
0.409131646674914       0.9329902850529865    0.933628044692714
0.40930417954175946     0.9326644017223017    0.9334955360265228
0.4094693614666875      0.9323512571406953    0.9331304724545075
0.4096355271721519      0.932035114479985     0.9325553650198312
0.4098026766581526      0.9317159514224531    0.9319098437033669
0.40997080992468965     0.9313937454551084    0.931376157053127
0.41014409433298993     0.9310604436030875    0.9310738956323383
0.4103100277993728      0.9307401137559357    0.9310070721769652
0.41047694504629206     0.9304167306277832    0.9310218208756976
0.4106448460737476      0.9300902715897967    0.9309332149973898
0.41080539615928574     0.9297770059283036    0.9306287507197499
0.4109710973865871      0.929452560164149     0.9300837902311937
0.4111377823944248      0.9291250287961081    0.9294202464955216
0.41130545118279876     0.9287943890829513    0.9288236784895783
0.411478271112936       0.9284523555461043    0.9284408224512528
0.41164374010115584     0.9281236926664754    0.9283207999038997
0.41181019286991205     0.9277919110776064    0.9283317622236658
0.41197762941920457     0.9274569879236438    0.9282884278776573
0.4121377150265797      0.9271356595749631    0.9280470259544352
0.412302951775718       0.9268028523952557    0.9275486847676149
0.4124691723053927      0.9264668939702051    0.9268827215793177
0.4126363766156037      0.9261277613327704    0.926229279170553
0.4128045647063511      0.9257854313167966    0.9257660478289944
0.412965401855181       0.9254569337259091    0.9255716713414336
0.41313139014577416     0.9251167551923399    0.9255556920377933
0.41329836221690364     0.9247733694668143    0.9255536716605063
0.4134663180685694      0.924426753270975     0.9253773042101822
0.41363942506199847     0.9240682151429928    0.9249139184558616
0.4138051811135101      0.9237236340315784    0.9242585558330615
0.4139719209455581      0.9233758119128561    0.9235633874706339
0.41413964455814245     0.9230247257443484    0.9230224145025179
0.4143000172288093      0.9226878885026237    0.9227550362990393
0.41446554104123945     0.922339063325629     0.9227010306835955
0.4146320486342059      0.9219869653073158    0.9227160816539279
0.41479954000770863     0.9216315713165568    0.9225999714655649
0.41497218252297463     0.9212639693123094    0.9222042167787519
0.41513747409632323     0.9209108019423192    0.9215773128931639
0.4153037494502082      0.9205543291452345    0.9208529529752408
0.4154710085846295      0.920194527699406     0.9202341794228143
0.4156392514995871      0.9198313741926206    0.9198701590488355
0.4158126455563079      0.919455803047529     0.9197690260657444
0.4159786886711113      0.9190949163888545    0.9197873445486129
0.416145715566451       0.9187306680938068    0.9197097680075969
0.4163137262423271      0.9183630346587897    0.9193760265747332
0.4164743859762858      0.9180103215716792    0.9187981770318094
0.4166401968520077      0.9176451039658212    0.91806015402116
0.41680699150826594     0.9172764923265413    0.9173773579922828
0.4169747699450605      0.9169044630617607    0.9169277811480925
0.4171476995236183      0.9165197046282214    0.9167612410660178
0.4173132781602587      0.9161500563231761    0.9167701388841686
0.4174798405774354      0.9157769808253342    0.9167377809849586
0.41764738677514845     0.9154004544522598    0.9164760687715403
0.41780758203094415     0.9150392762700568    0.9159509248436091
0.4181392586065982      0.9142878268114369    0.914477036932899
0.41830657256522974     0.9139068886798848    0.9139351543363525
0.4184790376656245      0.913512907333995     0.9136846360754013
0.41864415182410186     0.9131344657928661    0.9136648845909033
0.41881024976311554     0.9127525311678976    0.9136675830736313
0.4189773314826656      0.9123670795998535    0.9134829163490542
0.419145396982752       0.9119780870351237    0.9129965175708962
0.41931861362460154     0.9115758356526094    0.9122386585794381
0.4194844793245337      0.9111893817236492    0.9114624020702197
0.4196513288050022      0.9107993771178782    0.9108514664436237
0.41981916206600706     0.9104057976914912    0.9105346725287057
0.4199796443850945      0.9100281828687552    0.9104767274797303
0.42014527784594513     0.9096371909806461    0.9104962484680055
0.42031189508733213     0.909242615434276     0.9103760208217442
0.4204794961092555      0.9088444324656205    0.9099626421194583
0.420652248272942       0.9084326689609286    0.9092403126125482
0.4208176494947111      0.9080371483030708    0.908434802229008
0.4209840344970166      0.907638011919991     0.9077413225529884
0.4211514032798584      0.9072352359886703    0.9073275749027994
0.4213114211207828      0.9068489489318915    0.9072090400560051
0.4214765901034704      0.9064489939738881    0.907228579133924
0.42164274286669434     0.9060453918340035    0.9071689507313757
0.42180987941045467     0.905638118632388     0.9068405513935978
0.4219779997347513      0.9052271503080197    0.9061954283838483
0.42213876911713055     0.9048329324564002    0.9054028795512872
0.422304689641273       0.9044248328801388    0.9046253646601286
0.42247159394595174     0.9040130304561397    0.9040883970180702
0.42263948203116686     0.9035975010659111    0.9038693962360413
0.42281252125814517     0.9031678568424436    0.9038679924831984
0.4229782095432061      0.9027551640005851    0.9038513671377915
0.4231448816088033      0.9023387358035591    0.9036000680625638
0.4233125374549369      0.9019185480739986    0.9030192952873285
0.42363829840513256     0.9010983716389012    0.9014063385849121
0.4238047382316483      0.9006774144375113    0.900772447935443
0.4239721618387004      0.9002526657580112    0.9004604513281966
0.4241447365875157      0.899813479248196     0.9004175038713107
0.4243099603944136      0.899391696111885     0.9004317651421597
0.4244761679818478      0.898966113117855     0.9002611132615886
0.42464335934981834     0.898536705973924     0.8997628003104414
0.4248115344983252      0.8981034502043175    0.8989702939061283
0.4249848607885953      0.8976555351767901    0.8980616198738718
0.42515083613694804     0.8972252939627864    0.8973569007623984
0.42531779526583713     0.8967911956475827    0.8969696049327066
0.42548573817526253     0.8963532156966515    0.8968831686192218
0.42564633014277053     0.8959331633029004    0.896908810994133
0.4258120732520417      0.8954983617135459    0.8968038819815077
0.4259788001418492      0.8950596706985593    0.8963862234010034
0.426146510812193       0.8946170501322676    0.8956395164502146
0.4263193726243001      0.8941593218653463    0.8947075148860268
0.42648488349448976     0.893719732866472     0.8939190990754612
0.4266513781452158      0.8932762218070762    0.8934269743175104
0.42681885657647817     0.8928287646201912    0.8932672424767689
0.42697898406582313     0.892399703256664     0.8932871605082773
0.4271442626969313      0.8919555633615592    0.8932424508500875
0.42731052510857576     0.8915074712138435    0.8929166104305736
0.4274777713007566      0.8910554027273391    0.8922382741762461
0.4278068803042735      0.8901619336694804    0.8904805388968366
0.42797291047683644     0.8897092371025588    0.8898571825953729
0.42813992442993576     0.8892525337581806    0.8895841546592024
0.4283079221635714      0.8887917993647579    0.8895694455762432
0.42848107103897026     0.888315529795114     0.8895677659679617
0.42864686897245174     0.8878581395115729    0.8893234518457306
0.4288136506864696      0.8873967113529947    0.8887185306043118
0.42898141618102376     0.8869312210273844    0.8878273998867653
0.4291418307336605      0.8864848689881671    0.8869289656149272
0.42930739642806043     0.8860228935005166    0.8862047946592045
0.4294739459029967      0.8855568496720269    0.8858295682162463
0.4296414791584693      0.8850867131915655    0.8857633159368385
0.4298141635557051      0.8846007142285043    0.8857883928109292
0.42997949701102356     0.8841340641948345    0.8856279835074479
0.43014581424687837     0.883663314693625     0.8851145022480623
0.4303131152632695      0.8831884413937854    0.8842664170313231
0.430481400060197       0.8827094197969814    0.8832878287912054
0.4306548359988876      0.8822143117055824    0.882444916155445
0.4308209209956608      0.8817388328804483    0.8819891671597582
0.43098798977297037     0.8812591988785642    0.8818733113406653
0.4311560423308163      0.8807753851812601    0.8819046373160114
0.4313167439467448      0.8803114622136116    0.8818109398626205
0.43148259670443656     0.8798313636600951    0.8813848623494152
0.43164943324266464     0.8793470791883502    0.8805917400732796
0.4318172535614291      0.878858584260643     0.8795981031976852
0.43199022502195666     0.8783536724619668    0.8786675677314969
0.43215584554056685     0.8778688640451963    0.8781001929063149
0.43232244983971335     0.8773798383031897    0.8779014636792978
0.4324900379193962      0.8768864810765916    0.8779211857929399
0.4326502750571617      0.876413430630157     0.8778870479283665
0.4328156633366904      0.875923870313603     0.877558976026551
0.43298203539675545     0.875430062652694     0.8768439333404447
0.43332189821972145     0.8744171555897062    0.8748514145745471
0.43348705426016865     0.873922915991553     0.8741636607336115
0.43365319408115216     0.8734244002902721    0.8738584679202307
0.43382031768267204     0.8729215846216241    0.8738404487349272
0.43398842506472823     0.8724144449761921    0.8738482453580811
0.4341616835885477      0.8718903337713246    0.8735835717495498
0.43432759117044967     0.87138709699101      0.8729303136780918
0.434494482532888       0.8708795313373562    0.8719610396238976
0.4346623576758627      0.8703676128252982    0.8709289290863833
0.43482288187692        0.8698768336034157    0.8701591153845255
0.43498855721974045     0.8693689965788767    0.8697452500426125
0.4351552163430973      0.8688568024107017    0.8696708654701106
0.43532285924699043     0.8683402271383889    0.8697032918947595
0.43549565329264683     0.8678063539403417    0.8695310377071811
0.4356610963963858      0.8672938367126232    0.8689760591603003
0.4358275232806611      0.8667769334957107    0.8680550750950238
0.43599493394547273     0.8662556203535288    0.8669869615602136
0.43615499366836696     0.865755927517958     0.8661148857536178
0.4363202045330244      0.8652388506907479    0.8655755126573718
0.4364863991782182      0.8647173596649489    0.8654176934880347
0.4366535776039483      0.8641914305291223    0.8654519313452892
0.43682173981021477     0.8636610392264162    0.8653726469944452
0.43698255107456385     0.8631525503251183    0.8649582018350512
0.43714851348067607     0.8626264582246749    0.8641317984808101
0.43731545966732466     0.8620958996579687    0.8630588070151253
0.4374833896345096      0.8615608505927229    0.8620392364554975
0.43765647074345776     0.8610079574879667    0.8613389424173711
0.4378222009104885      0.8604771841098219    0.8610876725033768
0.4379889148580556      0.8599419153331535    0.8611001943362468
0.438156612586159       0.8594021271501224    0.8610810404740892
0.43831695937234505     0.858884723818481     0.8607673912008805
0.43848245730029434     0.8583493901229863    0.8600297687381666
0.43864893900877994     0.8578095327336734    0.8589764551259694
0.4388164044978019      0.857264964759483     0.8578883787399428
0.438989021128587       0.8567022099714734    0.8570578852080563
0.4391542868174547      0.8561620660024425    0.8566897203682122
0.43932053628685874     0.8556173706263975    0.8566530423834204
0.43948776953679913     0.8550681006612688    0.8566775653186375
0.4396559865672759      0.8545142328057687    0.8564473365587584
0.43982935473951584     0.8539419707468074    0.8557554823857554
0.4399953719698384      0.8533926096243236    0.8547258953679805
0.44016237298069727     0.852838647969243     0.8535967077568571
0.4403303577720925      0.8522800625544883    0.852689838618121
0.44049099162157035     0.8517446449806237    0.852222916071095
0.44065677661281133     0.8511907507263703    0.8521197348970813
0.4408235453845887      0.8506322306251787    0.8521617926731342
0.4409912979369024      0.8500690615232017    0.8520217759309799
0.4411642016309793      0.8494871785247465    0.8514417234867555
0.4413297543831388      0.8489286831092595    0.8504712565294089
0.44149629091583464     0.8483655360632708    0.8493141958766479
0.44166381122906684     0.8477977143067963    0.8482993172924862
0.44182398060038164     0.8472535453303217    0.8477031802087334
0.4419893011134596      0.8466905805010443    0.8475059307265391
0.4421556054070739      0.8461229388846101    0.8475410477621684
0.44232289348122455     0.845550597473922     0.8474836650690151
0.44249116533591154     0.8449735331437571    0.8470474261694926
0.44265208624868113     0.844420405187693     0.8462015635218177
0.4428181583032139      0.8438482673810459    0.8450512803139834
0.44298521413828307     0.8432714045926414    0.843929344513021
0.4431532537538886      0.8426897937704314    0.8431439162867174
0.4433264445112573      0.8420889368834833    0.8428178945236853
0.4434922843267086      0.8415122352286414    0.8428229408518184
0.44365910792269625     0.8409307829388648    0.8428222799213804
0.44382691529922025     0.8403445570359152    0.8424963388882228
0.44398737173382685     0.8397827504749175    0.841745414421011
0.4441529793101966      0.8392016162623409    0.8406243374869353
0.4443195706671027      0.8386157063848331    0.839438988209516
0.4444871458045452      0.8380249979369864    0.838524166941704
0.44465987208375085     0.8374147271344698    0.8380657132737931
0.4448252474210391      0.836829093122819     0.8380122792871448
0.4449916065388637      0.8362385849181393    0.8380502155149641
0.44515894943722467     0.8356431383121036    0.8378374274689943
0.44531894139366823     0.8350726001749552    0.8372033831858108
0.44548408449187493     0.834482414691701     0.8361427231330202
0.445650211370618       0.8338874047235167    0.8349175467489635
0.44581732202989743     0.8332875483509621    0.8338758934024141
0.4459854164697132      0.832682823566212     0.8332719765405658
0.44614615996761153     0.8321032910366022    0.8331183218421615
0.4463120546072731      0.831503903837966     0.8331658149705974
0.446478933027471       0.8308996485817121    0.8330756324739833
0.4466467952282053      0.8302905033852459    0.8325626562521251
0.4468198085707027      0.8296612734615166    0.831538657607056
0.4469854709712827      0.8290574552059903    0.8303020142037912
0.4471521171523991      0.8284487468941658    0.8291599861987767
0.4473197471140518      0.827835126770143     0.8284144593528943
0.4474800261337871      0.8272471800111811    0.8281551775850868
0.4476454562952856      0.8266390724416681    0.8281788906096584
0.4478118702373205      0.8260260534235091    0.8281621772208193
0.44797926795989174     0.8254081013252398    0.8277779815298437
0.44815181682422617     0.82476976096719      0.8268642368173039
0.44848319644959656     0.8235399277132549    0.8244193531034412
0.4486503619330863      0.822917589806375     0.8235236713695977
0.44881851119711236     0.8222902756195976    0.8231227243082873
0.4489918116029016      0.8216423667956776    0.8231046791491311
0.4491577610667734      0.8210206316157022    0.8231266890152852
0.44932469431118155     0.8203939201156727    0.8228310412756036
0.44949261133612606     0.8197622108307233    0.8220308113284772
0.44965317741915317     0.8191569337525136    0.8208798175024205
0.44981889464394353     0.8185309870083296    0.8195980390555548
0.4499855956492702      0.8179000429101854    0.8185731943127185
0.4501532804351332      0.8172640801175212    0.818033380114296
0.45032611636275943     0.8166072230851827    0.8179397967581913
0.4504916013484682      0.8159770126659485    0.8179920986444219
0.4506580701147133      0.815341783525182     0.8178095618105368
0.4508255226614948      0.8147015144480909    0.8171406306361161
0.45098562426635885     0.814088147600662     0.8160597591340557
0.4511508770129862      0.8134538123654496    0.8147486055162687
0.4513171135401498      0.8128142939586256    0.8135996058071805
0.45148433384784975     0.8121696655191197    0.8129017065785467
0.45165253793608606     0.8115199556373328    0.8126981578491598
0.4518133910824049      0.8108974325404422    0.8127485865261234
0.45197939537048704     0.8102537384840002    0.8126857550403792
0.45214638343910546     0.8096049659390931    0.8121919702982564
0.45231435528826025     0.8089510947557318    0.811176322001846
0.45248747827917823     0.8082758341473187    0.8098024577180296
0.4526532503281788      0.8076279756130579    0.8085578799441094
0.4528200061577157      0.8069750210080554    0.8077147298375345
0.452987745767789       0.8063169503631363    0.8073911879398129
0.45314813443594487     0.8056865352557518    0.8074103264445616
0.453313674245864       0.8050346622141492    0.8074161013116796
0.45348019783631943     0.8043776760935442    0.8070546513863713
0.4536477052073111      0.803715557101995     0.8061589846012813
0.45382036372006607     0.8030317654438666    0.8048189300685846
0.4539856712909036      0.802375841084745     0.8034961232333995
0.45415196264227753     0.801714786436352     0.8025011487827292
0.45431923777418776     0.8010485818862934    0.8020295820055876
0.45448749668663435     0.8003772077719471    0.8019876229265946
0.45466090674084414     0.799683968997534     0.8020312329559725
0.45482696585313653     0.7990188719557635    0.8017585955625404
0.45499400874596524     0.7983486080423791    0.8009549646028717
0.4551620354193303      0.7976731577753722    0.7996883480326971
0.45532271115077794     0.7970260974494126    0.7983527274325497
0.4554885380239888      0.7963571057184041    0.7972296333720462
0.455655348677736       0.7956829307061277    0.7966127760754531
0.45582314311201955     0.7950035531076004    0.7964849318623174
0.4559960886880663      0.7943020341895657    0.7965507774881976
0.45616168332219564     0.793629112694815     0.7963910330362894
0.4563282617368613      0.7929509913202006    0.7957261100772666
0.45649582393206334     0.7922676509400846    0.7945429236758804
0.4566560351853479      0.7916131519662104    0.7931856413679967
0.45682139758039575     0.7909364461802428    0.7919411132868491
0.4569877437559799      0.7902545244678244    0.7911611267089873
0.4571550737121004      0.7895673678790834    0.7909137289703781
0.4573275548099841      0.7888577982671143    0.7909699200148073
0.4574926849659504      0.7881772362458183    0.7909126955253716
0.45765879890245303     0.7874912899652294    0.7904027130034595
0.457825896619492       0.7868000926029137    0.7893383634638965
0.4579939781170673      0.7861036261819061    0.7879213618334472
0.4581672107564059      0.7853845595870125    0.7865265763448397
0.45833309245382703     0.7846948141816942    0.7856262895003564
0.45849995793178455     0.7839998050322279    0.7852770339544142
0.45866780719027833     0.7832995143954236    0.7853021275747174
0.4588283055068547      0.7826287851955684    0.785312664405631
0.4589939549651943      0.7819353964530463    0.7849393337356351
0.45916058820407024     0.7812367318338754    0.7840011623164057
0.4593282052234825      0.7805327738237491    0.782624458584606
0.459500973384658       0.7798059587115862    0.7811502234663346
0.4596663906039161      0.7791089075088149    0.7800952445448518
0.4598327916037106      0.7784065682440225    0.7795933123793088
0.4600001763840414      0.7776989236349477    0.7795492290758442
0.4601602102224548      0.7770212776994886    0.7796072121243504
0.4603253952026313      0.7763207169112628    0.7793722366723896
0.46049156396334423     0.7756148563942681    0.7785869670155343
0.46065871650459345     0.7749036790934101    0.7772889814833753
0.46082685282637903     0.7741871679422878    0.7758004574267797
0.4609876382062472      0.7735009084656613    0.7746078254505887
0.46115357472787866     0.772791564672814     0.773898852890943
0.4613204950300464      0.7720768927946983    0.7737239167266767
0.4614883991127505      0.7713568759935215    0.773797898522507
0.4616614543372178      0.770613584011926     0.7736749961233654
0.46182715861976764     0.7699007402286056    0.7730345571431397
0.4619938466828538      0.7691825570234592    0.7718337268798906
0.46216151852647636     0.7684590177904695    0.7703342266428297
0.4623218394281815      0.7677661542207105    0.7690282982230615
0.4624873114716499      0.7670499601682231    0.7681532088517183
0.4626537672956546      0.7663284158573619    0.7678473599357407
0.4628212069001957      0.765601504908998     0.7678965846770593
0.46299379764649995     0.7648510774067506    0.7678730348617697
0.4631590374508868      0.7641315175507       0.7673914978615187
0.46332526103580995     0.7634065965722173    0.7663222534491045
0.4636606595472653      0.7619406066457775    0.7633830086280399
0.4638340018350244      0.7611811197600112    0.7623246078452542
0.46399999318086604     0.7604526847656738    0.7619115850447549
0.464166968307244       0.7597188622166745    0.7619223844325795
0.46433492721415837     0.7589796369828403    0.7619555406865985
0.4644955351791553      0.7582717561623965    0.7616140531852656
0.4646612942859155      0.7575401411620779    0.7606799426813283
0.464828037173212       0.7568031316702666    0.7592603974288848
0.4649957638410449      0.7560607128335909    0.7577301715774114
0.4651686416506409      0.7552943836140931    0.7565125214168442
0.46533416851831955     0.7545595878406675    0.755940244012409
0.4655006791665345      0.753819390738481     0.7558701084399638
0.4656681735952858      0.7530737777349217    0.755945663818284
0.4658283170821197      0.7523599135987891    0.7557390137787512
0.46599361171071685     0.7516220921300384    0.7549651943526114
0.4661598901198503      0.7508788629558243    0.7536378127725072
0.46632715230952015     0.7501302117781875    0.7520739619506421
0.4664953982797263      0.7493761243268876    0.7507297738135102
0.466656293308015       0.7486540186989487    0.7499714802181227
0.46682233947806695     0.7479078083908743    0.7497569145874324
0.4669893694286552      0.7471561701979119    0.7498364995594805
0.4671573831597798      0.7463990901262775    0.749757767402786
0.4673305480326676      0.7456177376805225    0.7491171274465745
0.46749636196363803     0.7448685485283694    0.7478971980764881
0.4676631596751448      0.7441139257316045    0.7463345430343933
0.46783094116718793     0.7433538555768654    0.7448768807928567
0.46799137171731364     0.7426261548585014    0.7439554174949436
0.4681569534092025      0.7418741379268352    0.7436023526950127
0.4683235188816277      0.7411166820280981    0.7436468685473591
0.4684910681345893      0.7403537737232321    0.7436571320428973
0.4686637685293141      0.7395663873156808    0.7431849466486969
0.46882911798212146     0.73881154630749      0.7421058917269873
0.4691627682293452      0.7372855186556433    0.7390312961608239
0.4693227343013079      0.7365525211208584    0.7379397282739298
0.4694878515150338      0.7357950035821057    0.737416388605758
0.46965395250929604     0.73503203711712      0.7373852854090642
0.46982103728409463     0.73426360883726      0.7374573983273104
0.46998910583942954     0.7334897058896218    0.7371711748576643
0.470149823452847       0.7327486152070959    0.7363106338552227
0.4703156922080277      0.731982862419279     0.7348888367750855
0.47048254474374473     0.7312116439209597    0.7332756203414383
0.47065038105999807     0.7304349478346592    0.7319410280708821
0.47082336851801465     0.7296334420824295    0.7311998690503472
0.47098900503411384     0.7288650757058178    0.7310692864457566
0.4711556253307494      0.7280912422553839    0.7311635023862009
0.47132322940792126     0.7273119301766504    0.7310105735697102
0.4714834825431757      0.7265659488958606    0.7303128116739497
0.47164888682019335     0.7257951227845587    0.7290030260954389
0.4718152748777473      0.7250188286288246    0.7273799441044302
0.4719826467158376      0.724237055189991     0.725911485240458
0.47215516969569116     0.7234302925471654    0.7249759823616546
0.4723203417336273      0.7226570258401921    0.7247077676610059
0.4724864975520998      0.721878290378323     0.7247871820533183
0.47265363715110864     0.7210940752433492    0.7247526001047506
0.4728217605306538      0.7203043695843978    0.7241889756549991
0.47299503505196216     0.7194895549765182    0.7229272508279774
0.4731609586313531      0.7187084436417882    0.7213169732776167
0.47332786599128035     0.717921852556526     0.7197668635783687
0.47349575713174397     0.7171297711924559    0.7187084468807177
0.4736562973302902      0.7163715734380499    0.7183125253407571
0.47382198867059966     0.7155882352847018    0.7183442175074642
0.47398866379144544     0.714799417668558     0.718388102742259
0.47415632269282754     0.714005110376738     0.7179875931394744
0.4743291327359729      0.7131855284696239    0.7168845450828875
0.47449459183720083     0.7123999862795065    0.7153276748503238
0.47466103471896515     0.7116089652007075    0.7136984260282727
0.4748284613812658      0.7108124553403968    0.7124649460372023
0.47498853710164896     0.7100501579407232    0.7118982026704699
0.4751537639637954      0.7092625567030478    0.7118421336340401
0.4753199746064781      0.7084694774993355    0.7119357862982438
0.4754871690296972      0.7076709107496235    0.7116977444327036
0.4756553472334526      0.7068668469493137    0.7108143371816537
0.4758161744952906      0.7060971835241958    0.7094065979268397
0.4759821528988919      0.7053021094776044    0.7077315266629922
0.4761491150830295      0.7045015494284286    0.7063048271754676
0.4763170610477034      0.7036954500084955    0.7054853437213512
0.47649015815414053     0.7028637277243676    0.7052985321012795
0.47665590431866023     0.7020665592864909    0.7054033778941535
0.47682263426371624     0.7012639072414132    0.7052922630109022
0.4769903479893086      0.7004557632892092    0.7045826245133088
0.4771507107729836      0.6996823361985731    0.7032931609212365
0.47731622469842183     0.6988833479533305    0.7016218225149164
0.4774827224043964      0.6980788804589557    0.7000714186526129
0.47765020389090723     0.697268925763642     0.6990630111782571
0.47782283651918134     0.6964332947333882    0.6987220728608043
0.477988118205538       0.6956325234737784    0.6988011398088521
0.47815438367243096     0.6948262777246318    0.6988001847655522
0.4783216329198603      0.6940145498874761    0.6982783335714623
0.47848986594782594     0.6931973324670883    0.6970696701828679
0.47866325011755484     0.6923543477581663    0.6953451704855498
0.47882928334536634     0.6915463994277162    0.6937225077415065
0.4789963003537142      0.6907329745045393    0.6925814934052317
0.4791643011425984      0.6899140658491477    0.6921191849387538
0.4793249509895652      0.6891303440490721    0.692146101371848
0.47949075197829516     0.6883208364516867    0.6922153519560849
0.4796575367475615      0.6875058580421956    0.6918542988196319
0.4798253052973642      0.6866854020287627    0.6908083677524961
0.4799982249889301      0.6858390565247507    0.6891540479944868
0.48016379373857854     0.6850280305707067    0.6874633233015781
0.48033034626876336     0.6842115400161415    0.6861528880086815
0.4804978825794845      0.6833895784220959    0.6855091526589059
0.48065806794828825     0.6826030784734965    0.6854447322413778
0.4808234044588552      0.6817906729798933    0.6855550528577715
0.4809897247499585      0.6809728093640255    0.6853522839614833
0.48115702882159817     0.6801494815317498    0.6844962169366984
0.481329484035001       0.679300151774833     0.6829564843126158
0.4814945883064864      0.6784864094053195    0.6812325969867812
0.4816606763585082      0.6776672158353684    0.6797607732116919
0.4818277481910663      0.6768425653208628    0.6789110960944625
0.48199580380416074     0.6760124522315547    0.6787145089713386
0.4821690105590184      0.675156263641366     0.6788337035086393
0.48233486637195866     0.6743358163926242    0.6787301754438341
0.48250170596543523     0.6735099198650261    0.6780136551425275
0.4826695293394482      0.6726784918469765    0.6766178515573381
0.48283000177154367     0.6718829259689802    0.6749413962623415
0.4829956253454024      0.6710612706642227    0.6733490852889582
0.48316223269979747     0.6702341745262909    0.6723124775225734
0.4833298238347289      0.6694016330228008    0.6719635670716441
0.4835025661114235      0.6685429185695424    0.6720527234546397
0.4836679574462007      0.6677201964591167    0.6720593606562016
0.48383433256151426     0.6668920434460405    0.671531389058319
0.48400169145736416     0.6660584553729215    0.670298749747455
0.4843268585069924      0.6644373272575196    0.6669784243032905
0.4844930013832245      0.6636082522867485    0.6657478004241941
0.48466012803999287     0.6627737527657395    0.6652108700193445
0.4848282384772976      0.6619338250401926    0.6652198150807925
0.4849889979726849      0.6611301457557468    0.6653133759335287
0.48515490860983546     0.6603002287405787    0.6650078378957421
0.4853218030275223      0.6594648981082376    0.664009500366302
0.48548968122574554     0.658624150573312     0.6624024018488647
0.48566271056573196     0.6577570941328869    0.6605712227024346
0.485828388963801       0.6569263923633036    0.6591802601461443
0.4859950511424063      0.6560902884680319    0.658460231527545
0.486162697101548       0.6552487795361726    0.6583644478154215
0.4863229921187723      0.6544437336322656    0.658488241731209
0.4864884382777598      0.6536123759764162    0.6583332992867582
0.4866548682172836      0.6527756278662568    0.6575273913019418
0.4868222819373438      0.6519334867567232    0.6560475081213143
0.48699484679916716     0.6510649704859813    0.6542005140077259
0.48716006071907314     0.6502330160856792    0.6526594148473568
0.48732625841951543     0.6493956834744365    0.6517349130888954
0.4874934399004941      0.6485529704786333    0.6514918338207654
0.48766160516200907     0.6477048750710986    0.6516132402478549
0.4878349215652873      0.6468303627063843    0.6515563011720681
0.48800088702664807     0.6459925296613381    0.6508882200715115
0.4881678362685452      0.6451493292977849    0.6495165580492275
0.48833576929097866     0.6443007599625795    0.6477288995144864
0.4884963513714947      0.6434889647109131    0.6461242873720576
0.48866208459377397     0.6426507542077413    0.6450179351359842
0.4888288015965896      0.6418071798633893    0.6446176635980628
0.4889965023799415      0.6409582149250335    0.6446954802294514
0.4891693543050567      0.6400827826458947    0.6447412721489921
0.4893348552882545      0.6392442146073148    0.6442587335577354
0.48950134005198864     0.6384003066344051    0.6430570589830825
0.4896688085962591      0.6375510580120755    0.6413274668999976
0.4898289261986122      0.6367387599987129    0.6396392041800698
0.4899941949427284      0.6358999999195244    0.6383461719341041
0.49016044746738097     0.635055914613054     0.6377557417885907
0.49032768377256986     0.6342065037389468    0.6377458709332882
0.4904959038582951      0.6333517671198762    0.6378634816944032
0.490656773002103       0.632534077688899     0.6375979937926946
0.4908227932876741      0.6316898996398358    0.6366356425916349
0.4909897973537815      0.6308404115678035    0.6350350100151098
0.4911577852004253      0.6299856136728609    0.6332092838785514
0.49133092418883223     0.6291042896267878    0.6316786113615682
0.49149671223532176     0.6282600900559687    0.6309069092269732
0.4916634840623476      0.627410596669859     0.6307856391041159
0.4918312396699098      0.6265558100517299    0.6309271808870481
0.49199164433555465     0.6257382202454623    0.6308033176015831
0.4921572001429627      0.6248941142621368    0.6300330665804473
0.4923237397309071      0.6240447307617526    0.6285679001680898
0.49249126309938784     0.6231900707018034    0.6267427596087701
0.49266393760963173     0.6223088648807941    0.6250675103270578
0.4928292611779582      0.621464925592448     0.624095399398584
0.492995568526821       0.6206157257623457    0.623822478995754
0.4931628596562202      0.6197612667280137    0.6239480152889904
0.49333113456615574     0.6189015500000424    0.6239368046975868
0.49350456061785447     0.6180152776930503    0.623287571545247
0.4936706357276358      0.6171663503884114    0.6219335882606206
0.49383769461795346     0.6163121817267488    0.6201329281134125
0.4940057372888075      0.6154527736013605    0.6184114157119182
0.4941664290177441      0.6146307686548103    0.6173010628492666
0.494332271888444       0.6137822228444242    0.6168705272415622
0.49449909853968016     0.612928453580337     0.6169452557044254
0.49466690897145266     0.6120694631295496    0.6170233066314875
0.49483987054498835     0.6111839152014961    0.6165633037123927
0.49500548117660664     0.6103358286365506    0.6153818392007688
0.4951720755887613      0.6094825399279583    0.613648693675418
0.4953396537814523      0.6086240503170298    0.6118481067511292
0.49549988103222586     0.6078030710091186    0.610562198083642
0.49566525942476264     0.6069555516540772    0.6099432837924217
0.4958316215978358      0.6061028460689682    0.6099250647984978
0.49599896755144524     0.6052449572528271    0.6100618632713556
0.49617146464681794     0.6043605195874553    0.6097906119476267
0.4963366108002732      0.6035136428389719    0.6088122073062922
0.4965027407342648      0.6026615991635447    0.6071923962884248
0.49666985444879275     0.6018043919354282    0.6053494160224092
0.496837951943857       0.6009420247131596    0.6038352320944976
0.4970112005806845      0.6000531179795204    0.6030188849477612
0.4971770982755946      0.5992018254592033    0.6029135456669049
0.49734397975104105     0.5983453895676238    0.6030673315207853
0.49751184500702383     0.597483814241202     0.6029281255572672
0.49767235932108916     0.5966598880930449    0.6021499658684359
0.49783802477691774     0.5958094444302233    0.6006637629614764
0.49800467401328263     0.5949538775787229    0.5988229236271951
0.49817230703018384     0.5940931918438774    0.5971739558915424
0.4983450911888483      0.593205992731609     0.5961533837602018
0.49851052440559535     0.5923564819021114    0.5959019315265175
0.49867694140287877     0.5915018688206823    0.5960421899973368
0.4988443421806985      0.5906421581673673    0.5960356493136031
0.49900439201660085     0.5898201618369471    0.595457342101716
0.4991695929942664      0.588971676728371     0.5941441512350172
0.4993357777524683      0.5881181102850568    0.5923490327306117
0.4995029462912065      0.5872594675528175    0.5905892599428169
0.49967109861048103     0.5863957537717457    0.589374252468627
0.49983189998783817     0.5855697864670678    0.5889235935368242
0.4999978525069585      0.5847173534327885    0.5889876802105758
0.5001647888066152      0.5838598658898096    0.5890972956188183
0.5003327088868081      0.5829973294466334    0.5887202408129819
0.5005057801087645      0.5821083443182447    0.5875434138258089
0.5006715003888031      0.5812571332913421    0.5858266460007449
0.5008382044493782      0.5804008903226155    0.5840031677251689
0.5010058922904898      0.5795396213947824    0.5826140154542192
0.5011662291896839      0.5787161385272512    0.5819828472799466
0.5013317172306411      0.5778662382527784    0.5819475254832239
0.5014981890521347      0.5770113804979916    0.5821037037069023
0.5016656446541647      0.576151519741819     0.5818999932005454
0.5018382513979578      0.5752652655898564    0.5809355570945495
0.5020035071998336      0.5744168146690223    0.57934035249614
0.5021697467822457      0.573563377018219     0.5774891036454408
0.5023369701451941      0.5727049591104668    0.5759343082466275
0.5025051772886788      0.5718415676148164    0.5750755432976259
0.5026785355739267      0.5709518218643309    0.5749299248211279
0.5028445429172573      0.5700998915329549    0.575095592162072
0.5030115340411243      0.5692430041937063    0.5749997816314009
0.5031795089455275      0.5683811668738793    0.5742195056506633
0.5033401329080134      0.5675571424652947    0.5727907341145653
0.5035059080122624      0.5667067965784471    0.5709529297121065
0.5036726668970478      0.5658515168475974    0.5692767287168429
0.5038404095623695      0.5649913106496789    0.5682265646315591
0.5040133033694545      0.5641048195543571    0.567925269561249
0.504178846234622       0.563256149374303     0.5680697268707597
0.5043453728803258      0.5624025693137934    0.5680960438162024
0.5045128833065661      0.5615440871049595    0.5675198764305887
0.504673042790889       0.5607234148362643    0.5662601888947468
0.5050046478235973      0.5590246849672624    0.5627034507580121
0.5051719260107561      0.5581679990785372    0.5614541650018738
0.5053401879784511      0.5573064435309547    0.5609629401916513
0.5055010990042288      0.5564826903925826    0.5610285581705541
0.5056671611717696      0.5556327394408102    0.561159789160774
0.5058342071198468      0.554777935249954     0.5608256698860586
0.5060022368484604      0.5539182864535435    0.559733578821726
0.5061754177188371      0.5530324888713419    0.5579609523895297
0.5063412476472964      0.5521844906314669    0.5561344971684264
0.506508061356292       0.5513316646874975    0.5547222592005984
0.5066758588458241      0.5504740200276864    0.5540475283403571
0.5068363053934387      0.549654150155235     0.5540166723209802
0.5070019030828166      0.5488081712853656    0.5541871007321165
0.5071684845527307      0.5479573901085759    0.554017487204917
0.5073360498031811      0.5471018159592491    0.5531296670755921
0.5075087661953948      0.5462201892055727    0.5514900639015222
0.5076741316456911      0.5453763774610253    0.5496450767073696
0.5078404808765238      0.5445278413074647    0.5480787810574318
0.5080078138878927      0.543674538726107     0.5471980644162869
0.5081677959573443      0.5428589624082507    0.5470302910649735
0.508332929168559       0.5420173773655077    0.5472003336041116
0.5084990461603102      0.5411710410116202    0.5471793883456894
0.5086661469325976      0.5403199629075596    0.5465139642968762
0.5088342314854213      0.5394641528155578    0.5450952011672344
0.5089949650963277      0.5386460363505642    0.5433274548857584
0.5091608498489972      0.5378019785184723    0.5416096523384931
0.5093277183822033      0.5369532040857197    0.5404717395071961
0.5094955706959455      0.5360997231314846    0.5400891893086318
0.5096685741514511      0.5352203675791382    0.5402194082340132
0.5098342266650391      0.5343786830046879    0.5403008823736689
0.5100008629591635      0.5335323077912673    0.5398289733197875
0.5101684830338243      0.5326812523401089    0.5385919345782395
0.5103287521665677      0.5318678212247772    0.5368933503564847
0.5104941724410743      0.531028559732079     0.535100906186599
0.5106605764961172      0.5301846333779046    0.533779651636449
0.5108279643316964      0.5293360528776576    0.5332065405279867
0.5110005033090388      0.5284617154580173    0.5332493708466315
0.5111656913444639      0.527624973345567     0.533403720636267
0.5113318631604253      0.5267835929736412    0.5331253528834411
0.511499018756923       0.5259375853769127    0.5321013455562834
0.5116671581339571      0.5250869618001153    0.5304242164210017
0.5118404486527544      0.524210664796742     0.5285122366939232
0.5120063882296343      0.5233719127509603    0.527075931096843
0.5121733115870505      0.5225285608989568    0.5263628573619898
0.5123412187250029      0.5216806208070798    0.5263170110573469
0.5125017749210381      0.5208701664316572    0.5264968162388024
0.5126674822588364      0.5200340870665354    0.5263713997051045
0.5128341733771712      0.519193435099372     0.5255462373174578
0.5130018482760422      0.5183482224102313    0.5240064816651352
0.5131746743166765      0.5174774697278703    0.5220906709168236
0.5133401494153933      0.5166441634675943    0.5205144721790066
0.5135066082946464      0.5158063126627287    0.5196043872743832
0.513674050954436       0.5149639295121571    0.519415949167696
0.513834142672308       0.5141589259311276    0.5195899197204998
0.5139993855319434      0.5133285561004337    0.5196014461564166
0.514165612172115       0.5124936914890763    0.5189905499168588
0.514332822592823       0.5116543223801151    0.5176282670707608
0.5145010167940673      0.5108104608506885    0.5157997501986771
0.5146618600533942      0.5100039028075298    0.5141336755419432
0.5148278544544844      0.5091719512482933    0.512978428811975
0.5149948326361109      0.5083355213101023    0.5125723730886469
0.5151627945982737      0.507494625342331     0.5126921867722367
0.5153359077021997      0.5066284297945227    0.5128072651039222
0.5155016698642084      0.5057994857196768    0.5123803909636129
0.5156684158067534      0.5049660901770423    0.5111975242587986
0.5158361455298346      0.5041282557926965    0.5094456984902324
0.5159965243109985      0.503327592870784     0.5077161279260287
0.5161620542339256      0.5025016817096059    0.506389993240049
0.5163285679373891      0.5016713457368651    0.5057998784506608
0.5164960654213889      0.5008365978484518    0.5058287979344704
0.516668714047152       0.4999767014070523    0.5060080463456655
0.5168340117309975      0.49915391892591576   0.5057643811111034
0.5170002931953794      0.49832673909273706   0.5047891985245229
0.5171675584402977      0.4974951750776709    0.5031573844406407
0.5173358074657523      0.49665924025535113   0.5013108575644382
0.5175092076329701      0.49579826275035566   0.49979444640572157
0.5176752568582705      0.4949743127233273    0.4990730954442825
0.5178422898641073      0.4941460067066446    0.49901952957985374
0.5180103066504804      0.4933133583512565    0.499216569792196
0.518170972494936       0.49251764637890766   0.4991131489317752
0.5183367894811549      0.491696946773112     0.4983292102627682
0.51850359024791        0.49087191908856287   0.49683493304771403
0.5186713747952016      0.49004257724562206   0.4949960809226959
0.5188443104842564      0.4891883576061046    0.493347342396077
0.5190098952313938      0.48837100781550985   0.49243941488817816
0.5191764637590676      0.4875493586854894    0.4922465754585867
0.5193440160672775      0.4867234244102806    0.4924354555139836
0.5195042174335702      0.485934262342624     0.49245894705700827
0.519669569941626       0.48512027942643265   0.49188017325020933
0.5198359062302184      0.48430202561287095   0.49056023007998634
0.5200032262993469      0.4834795153631694    0.48876960797829383
0.5201756975102387      0.48263235471817395   0.4870104902267127
0.520340817779213       0.48182201931579116   0.48590565511685047
0.5205069218287237      0.4810074423795295    0.4855367704500793
0.5206740096587708      0.4801886378534405    0.4856734344362803
0.520842081269354       0.4793656198676217    0.48579433223254925
0.5210153040217007      0.4785180189862791    0.4853505958332858
0.5211811758321299      0.47770700097841534   0.4841684287220073
0.5213480314230954      0.47689178221540623   0.48244012744649256
0.5215158707945973      0.47607237704877364   0.48066983912959366
0.5216763592241817      0.47528944473707924   0.47944184400240825
0.5218419987955293      0.4744819859795558    0.4789004002848736
0.5220086221474132      0.4736703529863216    0.4789544754080365
0.5221762292798335      0.4728545603246744    0.4791366004338453
0.522348987554017       0.4720143631015378    0.47887753386507614
0.5225143948862833      0.4712105552177411    0.47789309412228315
0.5226807859990857      0.47040260036966963   0.47627701456682137
0.5228481608924245      0.46959051334404994   0.47447122519977053
0.5230081848438459      0.468814700386873     0.4730888794100319
0.5231733599370305      0.46801454083812755   0.4723511892163877
0.5233395188107515      0.467210261268252     0.4722811305698983
0.5235066614650087      0.4664018766777685    0.47248733190593695
0.5236747878998023      0.46558940225953904   0.47241126125492355
0.5238355633926786      0.46481308159088486   0.471693448814254
0.524001490027318       0.4640125375068159    0.4702620077827357
0.5241684004424938      0.4632079159516167    0.46846688164641337
0.5243362946382059      0.4623992323336093    0.46685840075386986
0.5245093399756813      0.4615664555175653    0.4658802029564203
0.5246750343712392      0.46076974152069283   0.4656738409166462
0.5248417125473335      0.45996897837462936   0.4658641132479319
0.5250093745039641      0.45916418170681234   0.4659101996251404
0.5251696855186773      0.4583953234932153    0.4653824505325582
0.5253351476751537      0.4576024330934388    0.4641220069524126
0.5255015936121663      0.4568055215166265    0.4623826512500548
0.5256690233297154      0.4560046046036885    0.46068331896889825
0.5258416041890277      0.4551797902803262    0.45951324992948417
0.5260068341064227      0.45439081912247714   0.4591372795728228
0.5261730478043539      0.45359785554103227   0.45927098942260414
0.5263402452828214      0.4528009155940232    0.45941356405814
0.5265084265418254      0.4520001605066809    0.4590437062352676
0.5266817589425925      0.45117571404461176   0.45787728132802474
0.5268477404014422      0.45038696556723073   0.45620265119140213
0.5270147056408282      0.44959426890555504   0.4544658972613493
0.5271826546607505      0.448797639599116     0.45319610580101577
0.5273432527387556      0.4480365726416738    0.4526795537011549
0.5275090019585238      0.44725180943141335   0.4527308404899231
0.5276757349588284      0.4464631234800768    0.4529253879666114
0.5278434517396692      0.44567053048222055   0.45272205637300955
0.5280163196622732      0.4448543806286025    0.4517545432032575
0.5281818366429598      0.44407368697956306   0.4501920792076342
0.5283483374041829      0.4432890967075802    0.4484295572726275
0.5285158219459423      0.44250062566496784   0.447009902002883
0.5286759555457842      0.44174747140376336   0.4463131494912483
0.5288412402873893      0.4409708208499011    0.44624408683028904
0.5290075088095307      0.4401902994181488    0.4464563504996643
0.5291747611122086      0.4394059231143589    0.4464082812773235
0.5293429971954229      0.43861770811665696   0.4456874937471886
0.5295038823367196      0.4378646631125331    0.44433259809939357
0.5296699186197796      0.4370882579067873    0.4425875351718638
0.5298369386833759      0.4363080240536543    0.4410118601211219
0.5300049425275086      0.435523977886044     0.44005532989957796
0.5301780975134044      0.4347167172400805    0.43982952021276644
0.5303439015573828      0.43394451481277496   0.44002362001623996
0.5305106893818976      0.43316851065296363   0.44008745890451423
0.5306784609869488      0.432388721251082     0.4395549524682858
0.5308388816500825      0.4316438437395735    0.4383589176931862
0.5310044534549795      0.43087581462147384   0.43667107845934405
0.5311710090404127      0.4301040102965774    0.4350149014167135
0.5313385484063824      0.42932844740861753   0.4338850174615864
0.5315112389141152      0.42852988191895663   0.43348999021288037
0.5316765784799307      0.42776611340053755   0.4336289247551458
0.5318429018262824      0.42699859689786984   0.433781243787993
0.5320102089531704      0.4262273492107175    0.4334409827063439
0.5321701651381412      0.4254907484180844    0.4324324721179742
0.5323352724648751      0.42473120897451994   0.4308422704662813
0.5325013635721454      0.4239679483703243    0.4291357153946722
0.5326684384599519      0.42320099671262423   0.4278350733293406
0.5328364971282948      0.42243056729748624   0.4272474333329387
0.5329972048547205      0.42169461593735225   0.4272724908571711
0.5331630637229091      0.42093587566616064   0.4274829815544377
0.5333299063716341      0.420173456942255     0.42735652392897433
0.5334977328008956      0.4194073760492518    0.42653681891293116
0.5336707103719202      0.41861865816581556   0.42500460359547654
0.5338363370010273      0.4178642936220763    0.42329014089551
0.5340029474106709      0.41710627477922146   0.42186007484251875
0.5341705416008509      0.41634461801365985   0.4210947533709751
0.5343307848491133      0.4156171561863891    0.4210015082053693
0.5344961792391391      0.4148671184973726    0.42121442737160286
0.534662557409701       0.4141134502583197    0.42122321427837284
0.5348299193607995      0.41335616793571983   0.4206051157522544
0.535002432453661       0.4125764636108272    0.4192359522383429
0.5351675946046051      0.41183082764285317   0.4175498046641261
0.5353337405360856      0.4110815854564372    0.4160075421554153
0.5355008702481026      0.41032875360990345   0.4150522196289502
0.5356689837406556      0.4095723488093433    0.41480821378131755
0.5358422483749721      0.40879366989303134   0.4150047010111083
0.5360081620673711      0.40804888791462346   0.4150900041586154
0.5361750595403064      0.4073005409561372    0.4146105312147809
0.5363429407937781      0.4065486458163018    0.41341079630539257
0.5365034711053325      0.4058304846393917    0.41182165123943315
0.5366691525586499      0.4050901120029722    0.41021069286892486
0.5368358177925037      0.40434619865161414   0.4090965060900649
0.5370034668068939      0.40359876147397233   0.4086947060823214
0.5371762669630472      0.40282927088211523   0.40882981569295446
0.5373417161772831      0.4020933839439494    0.4089935546374336
0.5375081491720555      0.4013539811483848    0.4086924399933024
0.5376755659473642      0.40061107947575664   0.4076791205041389
0.5378356317807553      0.39990161626397136   0.40618251694666097
0.5380008487559098      0.3991701629259677    0.40453026542784976
0.5381670495116007      0.39843521816694044   0.403259983346252
0.5383342340478277      0.3976967990561987    0.40267569524379204
0.5385024023645911      0.3969549228138407    0.40270344269843156
0.5386632197394372      0.39624630999534644   0.4029126301425083
0.5388291882560464      0.39551585686577045   0.4028104112862744
0.538996140553192       0.39478204825893914   0.4020427166936787
0.5391640766308741      0.3940449352476931    0.40062113628371565
0.5393371638503192      0.3932861495172504    0.39888414836490155
0.5395029001278469      0.39256048120114223   0.3974969346771708
0.539669620185911       0.391831386732163     0.3967477228067986
0.5398373240245113      0.3910988825081801    0.39665897494330243
0.5399976769211943      0.3903993238624862    0.3968699777366279
0.5401631809596406      0.38967815238383074   0.3968920953173603
0.5403296687786232      0.38895357591156404   0.3963130335603881
0.540497140378142       0.3882256108707415    0.3950454144982236
0.5406697631194242      0.387476193004893     0.39334272914393886
0.5408350349187889      0.38675958137977895   0.3918539022382008
0.54100129049869        0.38603958639184244   0.3909285887278733
0.5411685298591273      0.38531622449410147   0.3906904848322445
0.5413367530001012      0.38458951225896293   0.3908796901103911
0.5415101272828381      0.38384149783499544   0.3909828238039364
0.5416761506236576      0.3831261036654118    0.39053107898033246
0.5418431577450136      0.3824073644196392    0.3893847270630208
0.5420111486469057      0.38168529669840745   0.38777450724494317
0.5421717886068804      0.38099567519402333   0.3862658125087709
0.5423375797086185      0.38028481212034926   0.385194483405411
0.542504354590893       0.3795706253769146    0.38480942591158934
0.5426721132537036      0.3788531315915679    0.3849389182047692
0.5428450230582775      0.3781145582578777    0.3851153696238827
0.5430105819209341      0.3774082900081865    0.38482970090318636
0.5431771245641268      0.37669871997106874   0.38385832044683543
0.543344650987856       0.3759858648022664    0.3823386157800302
0.5435048264696678      0.3753051399802423    0.3807917976992069
0.5436701530932427      0.3746033964180194    0.3795762719897497
0.543836463497354       0.37389837252073876   0.3790216487245284
0.5440037576820018      0.373190084970841     0.37905565668953234
0.5441720356471857      0.37247855057149687   0.3792724001766996
0.5443329626704523      0.37179896055885964   0.37917071566697846
0.5444990408354822      0.3710985019954446    0.3784303883454461
0.5446661027810482      0.370394801432565     0.37706259510880996
0.5448341485071506      0.369687875700441     0.375431148900129
0.5450073453750164      0.3689602454471826    0.37402967394528375
0.5451731913009648      0.36826441665650167   0.37332302377358806
0.5453400210074493      0.3675655520684188    0.3732492207501458
0.5455078344944703      0.3668635078144253    0.37347020425981253
0.5456682970395739      0.366193077437733     0.3734836501443875
0.5458339107264406      0.3655020079459586    0.3729190376804572
0.5460005081938437      0.36480773810624756   0.37169488952042107
0.5461680894417833      0.364110283858628     0.37009775684589175
0.5463408218314859      0.3633923518019915    0.36859914569631075
0.5465062032792711      0.3627058863503495    0.3677294325014216
0.5466725685075928      0.36201623911026065   0.36751790030889875
0.5468399175164507      0.3613234259903226    0.36770994719187516
0.5469999155833911      0.36066190067948856   0.3678208509971715
0.5471650647920949      0.35997995445047337   0.36743840818926365
0.5473311977813351      0.35929484456533617   0.36638694062650146
0.5476664151014242      0.35791519742626193   0.3633229514971396
0.5478271647098196      0.3572549047783228    0.3622899096323985
0.5479930654599781      0.3565743376962244    0.3618793669604322
0.548159949990673       0.35589064103193935   0.36198672352371786
0.5483278183019042      0.35520383072161127   0.36217903826562303
0.5485008377548987      0.35449690724900856   0.36194915471841593
0.5486665062659757      0.353820933356975     0.3610631284974629
0.5488331585575892      0.3531418484353127    0.3596307445656649
0.5490007946297388      0.35245966838874593   0.3580644532115812
0.5491610797599711      0.3518082591601397    0.3569056757008511
0.5493265160319667      0.3511367936452178    0.3563369526564413
0.5494929360844985      0.35046223522688685   0.3563479729324826
0.5496603399175666      0.3497845997786653    0.35656731650886236
0.5498328948923981      0.3490870686780783    0.35648639072302263
0.5499980989253122      0.34842016173577445   0.35578387394328465
0.5501642867387624      0.3477501803719433    0.35447865964465786
0.5503314583327491      0.3470771404287594    0.3529147373468344
0.5504996137072722      0.34640105783819936   0.35158798046780576
0.5506729202235584      0.345705228040952     0.35086186704041067
0.5508388757979272      0.3450398288914336    0.3507938580950617
0.5510058151528325      0.34437138970784814   0.3510117959986063
0.551173738288274       0.3436999263907649    0.3510255782738066
0.551334310481798       0.3430587149951791    0.3504912564008812
0.5515000338170855      0.3423978583947433    0.3493202354404931
0.551666740932909       0.34173411448023244   0.34779097816896487
0.551834431829269       0.3410673646155578    0.3463810166696092
0.5520072738673922      0.34038108869095496   0.3454982467718338
0.5521727649635981      0.3397249077264952    0.34530285425592994
0.5523392398403402      0.33906572109343475   0.3454927560499185
0.5525066984976187      0.33840354380403903   0.3456022253773147
0.5526668062129798      0.33777128189358313   0.3452332097956869
0.552832065070104       0.3371195465228888    0.34422261760465067
0.5531655341259616      0.3358071185837572    0.3412884580181134
0.5533337443246948      0.335146455939213     0.34026270112895757
0.5534946035815107      0.3345155162527052    0.3399001115327335
0.5536606139800899      0.3338652441054454    0.3400112639404102
0.5538276081592053      0.3332120110826958    0.34020016540701886
0.553995586118857       0.33255583209565376   0.33999770310357674
0.5541687152202721      0.33188047703129914   0.3391133688641287
0.5543344933797696      0.33123469617798795   0.33773813485026905
0.5545012553198034      0.3305859695938169    0.33624081177596843
0.5546690010403738      0.32993431210742086   0.3350928720967508
0.5548293958190267      0.3293120512273891    0.3345889877843621
0.5549949417394426      0.32867066620431956   0.3346123589077607
0.5551614714403951      0.3280263501830673    0.33482672052853407
0.5553289849218839      0.32737911791115487   0.3347575973878131
0.5555016495451358      0.32671291617997844   0.33405109950675677
0.5556669632264704      0.326075963907077     0.3327925403207796
0.5558332606883414      0.32543609560799164   0.33129697501419886
0.5560005419307485      0.3247933259481955    0.3300356370870275
0.5561688069536921      0.3241476696532921    0.32935858827124687
0.5563422231183988      0.3234831844162946    0.32929541393180783
0.5565082883411882      0.3228477563584144    0.32950947253208257
0.5566753373445139      0.3222094418503393    0.3295184287343385
0.5568433701283759      0.3215682555351928    0.3289575504912254
0.5570040519703205      0.3209559507917       0.32784310010796797
0.5571698849540285      0.32032486836819946   0.32637804602869736
0.5573367017182727      0.3196909139986208    0.325039699053449
0.5575045022630531      0.3190541022450567    0.3242225901738231
0.557677453949597       0.31839866566259023   0.3240372209728773
0.5578430546942232      0.31777206542732267   0.3242283038051104
0.5580096392193858      0.31714264855248      0.3243272994089796
0.5581772075250849      0.31651038874680626   0.32392750455817
0.5583374248888666      0.3159066835994059    0.32295513948503857
0.5585027933944113      0.31528440644195727   0.32154922666132824
0.5586691456804925      0.31465928443162466   0.3201514708487114
0.5588364817471101      0.3140313313164473    0.3191903889344621
0.5590089689554907      0.31338495501050984   0.3188575863301472
0.559174105221954       0.3127669869472241    0.3189942557403133
0.5593402252689537      0.31214618612316464   0.319161676176913
0.5595073290964896      0.31152256616483087   0.31892366414985296
0.5596754167045619      0.310896140735098     0.31805850455759366
0.5598486554543973      0.3102514253161286    0.3166557165510485
0.5600145432623155      0.3096349282901051    0.3152399238189857
0.56018141485077        0.3090156240623268    0.314192624723254
0.5603492702197608      0.3083935261734112    0.3137572515555862
0.5605097746468342      0.3077994748637786    0.3138254354152698
0.5606754302156709      0.30718718002470224   0.3140261925843245
0.5608420695650438      0.30657208952750836   0.31391666889721825
0.561009692694953       0.3059542167929744    0.3132106057526802
0.5611824669666254      0.3053182461105442    0.311913501224761
0.5613478902963805      0.30471017846738324   0.31048879488070485
0.561514297406672       0.30409932684707547   0.3093305648425122
0.5616816882974998      0.3034857045489967    0.3087457926348325
0.5618417282464101      0.3028998163360203    0.30871717456529413
0.5620069193370838      0.3022958757870533    0.30892456066202884
0.5621730942082936      0.30168916255861544   0.30893316889611494
0.5623402528600399      0.3010796898310913    0.308398766454318
0.5625083952923225      0.3004674708185455    0.3072712425788791
0.5626691867826877      0.29988280434662584   0.30590470445448664
0.5628351294148162      0.29928021296374996   0.3046312556582063
0.5630020558274809      0.298674873223993     0.3038558155310382
0.5631699660206819      0.298066798221717     0.3036793383551761
0.5633430273556461      0.29744094066097526   0.3038706173900414
0.563508737748693       0.29684249496837417   0.3039621801780833
0.5636754319222763      0.29624131218966776   0.30357553469985804
0.5638431098763959      0.2956374052979495    0.30258378092322397
0.564003436888598       0.2950607516121024    0.30126779079038235
0.5641689150425633      0.29446644576746867   0.29993918148518195
0.564335376977065       0.2938694138635889    0.29903167432936606
0.5645028226921031      0.2932696683483855    0.2987225364743426
0.5646754195489043      0.2926523261112988    0.29886103563097
0.5648406654637881      0.29206208636108993   0.29901719145662925
0.5650068951592082      0.29146912996375485   0.2987798671867398
0.5651741086351648      0.29087346922022783   0.2979454939271497
0.5653423058916577      0.29027511644811205   0.29663875234495424
0.5655156542899138      0.2896592893773893    0.2952227890325868
0.5656816517462524      0.2890703841658918    0.2942376153319965
0.5658486329831274      0.2884787837917921    0.2938356084238803
0.5660165980005387      0.287884500424796     0.29391740481108997
0.5661772120760326      0.2873169766707062    0.2941013507830233
0.5663429772932898      0.2867320195913229    0.2939828204034051
0.5665097262910832      0.28614437617312477   0.2932939425314006
0.566677459069413       0.28555405844098325   0.29208222966902947
0.566850342989506       0.2849464417468919    0.2906572294381204
0.5670158759676817      0.2843654482056719    0.2895689285458554
0.5671823927263937      0.2837817772069703    0.28903034500039543
0.567349893265642       0.2831954406288458    0.2890224841268462
0.5675100428629729      0.28263556842441956   0.28921760911738464
0.567675343602067       0.28205843720732676   0.2892100222214576
0.5678416281216975      0.28147863706135473   0.28867892534600814
0.5680088964218643      0.28089617972080244   0.2875888186672293
0.5681771485025673      0.2803110769332226    0.28621560772875587
0.568338049641353       0.27975226752153964   0.2850507344230893
0.5685041019219019      0.27917631491431427   0.2843353591446548
0.5686711379829873      0.2785977134177946    0.2841855945820152
0.5688391578246089      0.27801647463487206   0.2843675153865704
0.5690123288079938      0.27741822284388173   0.28444926672591664
0.5691781488494612      0.2768461316786889    0.2840540060440963
0.5693449526714649      0.2762713999759433    0.28308650704522004
0.569512740274005       0.2756940391918913    0.28175674340831447
0.5696731769346276      0.2751426848068419    0.280537670792383
0.5698387647370136      0.2745743558357299    0.2797001506822885
0.5700053363199358      0.27400339433772863   0.27943059394855846
0.5701728916833944      0.27342981162549834   0.2795671817627296
0.5703455981886161      0.2728394020255013    0.27971373021071766
0.5705109537519205      0.2722748870657656    0.27945736335417926
0.5706772930957612      0.27170774761121763   0.27863263909425573
0.5708446162201382      0.27113799465822336   0.2773731960895166
0.5710045884025978      0.2705939652002547    0.27611811033229217
0.5711697117268206      0.27003312616563846   0.27515908508081366
0.5713358188315799      0.2694696696946626    0.27475076370182605
0.5715029097168755      0.2689036066303161    0.274812852033665
0.5716709843827072      0.2683349478188527    0.27499794441777115
0.5718317081066218      0.26779184945579665   0.27489699782178195
0.5719975829722994      0.26723204981095955   0.2742613234191456
0.5721644416185135      0.26666965037888596   0.27312116410870485
0.5723322840452639      0.26610466185147286   0.27179061680285704
0.5725052776137776      0.2655230941960619    0.27067687884229197
0.5726709202403737      0.26496696028480277   0.2701435333590076
0.5728375466475062      0.2644082333988406    0.2701200274143629
0.5730051568351751      0.2638469240735604    0.27031296974542984
0.5731654160809264      0.2633109022515528    0.27030959636160856
0.5733308264684411      0.2627583361941246    0.26981818921930206
0.5734972206364921      0.2622031836545873    0.26879243619492194
0.5736645985850795      0.26164545501513725   0.2674846493225803
0.5738371276754299      0.2610713010568741    0.26628665759910874
0.5740023058238631      0.26052231096530637   0.26561852569480476
0.5741684677528326      0.25997074088542443   0.265484844341754
0.5743356134623383      0.2594166010440889    0.2656590379640869
0.5745037429523805      0.25885990166771605   0.26573474217939214
0.5746770235841858      0.25828688068460587   0.2653330798161875
0.5748429532740739      0.25773886520682193   0.26439633367202964
0.5750098667444982      0.2571882861963882    0.2631239135050599
0.5751777639954587      0.2566351537233969    0.26191271174440006
0.575338310304502       0.25610688553845506   0.2611582885818738
0.5755040077553084      0.2555623289548378    0.26091615230704746
0.5756706889866512      0.2550152147601638    0.2610505136813066
0.5758383539985302      0.25446555287152944   0.2611840497100471
0.5760111701521726      0.2538997168969106    0.26091575192914634
0.5761766353638975      0.2533586256654747    0.2601086044163999
0.5763430843561588      0.252814982733655     0.2588984426420463
0.5765105171289564      0.2522687978633999    0.25764825097716754
0.5766705989598365      0.2517472039247292    0.2567828434010297
0.57683583193248        0.2512094659630034    0.25641439995231746
0.5770020486856597      0.25066918194902477   0.2564829835346395
0.5771692492193758      0.2501263615743644    0.25665424161373723
0.5773374335336283      0.2495810145292553    0.2565264163514203
0.5774982669059633      0.24906012737204564   0.25590212991693906
0.5776642514200615      0.2485231951489399    0.25479842921440365
0.577831219714696       0.2479837322376555    0.2535293828258507
0.5779991717898669      0.2474417481808309    0.25249854262580057
0.578172275006801       0.24688382912004864   0.25197750309558314
0.5783380272818178      0.2463502547850563    0.25197055133557583
0.5785047633373709      0.24581415542280524   0.2521520685019454
0.5786724831734602      0.24527554042626826   0.25212141218598066
0.5788328520676322      0.244761137675253     0.25162605873330257
0.5789983721035674      0.24423083081956745   0.2506242269344657
0.579164875920039       0.24369800430966182   0.2493722329069235
0.5793323635170468      0.24316266739204057   0.24826444868229802
0.5795050022558179      0.2426115333354629    0.24761361379896127
0.5796702900526716      0.24208449929727985   0.24750866276733421
0.5798365616300616      0.24155495096133142   0.24767738271688075
0.580003816987988       0.2410228974255862    0.2477329796061406
0.5801720561264506      0.2404883477831049    0.24734115475240265
0.5803454464066765      0.2399380938746665    0.2463837838780956
0.5805114857449849      0.2394117965193907    0.24516086013368596
0.5806785088638298      0.23888299906155427   0.24401752577208177
0.580846515763211       0.23835171044473083   0.24329620703747468
0.5810071717206747      0.2378442507532313    0.24310589286379936
0.5811729788199017      0.23732111606812822   0.24324273713010328
0.5813397696996649      0.2367954861020711    0.24335512824683386
0.5815075443599647      0.23626736965234885   0.24308338464291684
0.5816804701620275      0.23572368072786593   0.24224855631642914
0.5818460450221731      0.23520371245395053   0.2410774913043417
0.5820126036628549      0.2346812536924369    0.23989368809819095
0.582180146084073       0.23415631309226043   0.23906252562217836
0.5823403375633738      0.23365496902560864   0.23876156531145581
0.5825056801844377      0.23313808072283043   0.2388435159990024
0.582672006586038       0.23261870645757654   0.2389961086331455
0.5828393167681746      0.2320968328101754    0.2388438006734089
0.5830117780920745      0.23155945804305747   0.23814854895290558
0.5831768884740569      0.23104557820320853   0.23705376923903734
0.5833429826365757      0.23052921718638683   0.23584868654589108
0.5835100605796308      0.23001038368329385   0.23491055710407052
0.5836781223032222      0.22948908638699983   0.2344749015060667
0.5838513351685769      0.22895242982854894   0.23450071057483357
0.584017197092014       0.22843913519370487   0.2346659572540701
0.5841840427959877      0.22792337363557755   0.23458614238957048
0.5843518722804977      0.2274051537207327    0.23401104241305223
0.5845123508230902      0.22691017693868187   0.23302199438441107
0.5846779805074459      0.22639986927853753   0.23182131285206076
0.5848445939723379      0.22588709999845533   0.23080462662216272
0.5850121912177664      0.22537187754117652   0.23025222377444243
0.585184939604958       0.22484142084694142   0.23019084263236053
0.5853503370502322      0.22433410686731453   0.23035641607611965
0.5855167182760427      0.22382433657308973   0.23036493753465268
0.5856840832823895      0.22331211828145225   0.22992269496744733
0.5858440973468191      0.22282292688269806   0.22903653163989554
0.5860092625530118      0.22231852875587368   0.22786217181445526
0.5861754115397408      0.22181167936633864   0.2267776151353665
0.5863425443070062      0.22130238690839604   0.22610125704411357
0.5865106608548079      0.2207906595757283    0.22592889923952755
0.5866714264606923      0.22030183418776217   0.2260623032762558
0.5868373432083398      0.21979788435070613   0.22615702142749922
0.5870042437365237      0.21929149629079223   0.22587699947851952
0.5871721280452438      0.21878267807803137   0.22508702850209775
0.5873451634957272      0.21825882836822702   0.22390577100873726
0.5875108480042932      0.21775778346305918   0.22278109897118828
0.5876775162933955      0.21725430517286767   0.22200299392539685
0.5878451683630341      0.21674840144225396   0.22172589753139624
0.5880054694907555      0.21626519071680683   0.22181492604334585
0.58817092176024        0.2157669746542852    0.22195028923622492
0.5883373578102609      0.2152663298013236    0.22178160921070808
0.588504777640818       0.21476326397979387   0.2211163659591337
0.5886773486131384      0.2142452831789303    0.22000598509293026
0.5888425686435412      0.21374990070227243   0.21885725606115045
0.5890087724544806      0.21325209401785347   0.2179784776129067
0.5891759600459563      0.21275177955453306   0.21758122945919128
0.5893441314179682      0.21224901819798123   0.21761046911227538
0.5895174539317434      0.21173141740840432   0.21776818370726386
0.5896834255036012      0.21123630100412316   0.21766671506817525
0.5898503808559953      0.21073877376529912   0.21708889235211928
0.5900183199889257      0.2102388438504027    0.2160695348224152
0.5901789081799389      0.20976129177105662   0.21494873720620167
0.5903446475127151      0.20926892794871785   0.21399494874616742
0.5905113706260277      0.2087741596469176    0.21349072372242883
0.5906790775198767      0.2082769949375615    0.21344369868273808
0.5908519355554888      0.20776510756327876   0.21360520439174005
0.5910174426491834      0.20727550865530125   0.21358856015361205
0.5911839335234146      0.20678351168018358   0.21313304930812774
0.5913514081781821      0.20628912462210314   0.21220855936145705
0.5915115318910321      0.2058169215504663    0.2111044622670735
0.5916768067456454      0.20533002289174085   0.2100836139958841
0.5918430653807949      0.20484073234409328   0.2094643734121153
0.5920103077964809      0.20434905780576712   0.20932058628261155
0.592178533992703       0.20385500718871283   0.20945925515068978
0.592339409247008       0.2033830281427718    0.2095250145126294
0.592505435643076       0.20289643078626718   0.20922176192616113
0.5926724458196804      0.20240745549129194   0.20843476576669598
0.5928404397768212      0.20191611008333177   0.20732049340207995
0.5930135848757251      0.2014102330046177    0.20619992943282667
0.5931793790327116      0.2009263402936583    0.20548554800257163
0.5933461569702345      0.20044007574715197   0.20524774537873813
0.5935139186882938      0.19995144710295565   0.20535006748071144
0.5936743294644355      0.1994847010044864    0.20545914652751726
0.5938398913823406      0.19900344942117798   0.20526103690430467
0.594006437080782       0.19851983187799618   0.20458712270229645
0.5941739665597596      0.19803385602696774   0.20353290524241588
0.5943466471805006      0.19753345952000775   0.20238407988728785
0.5945119768593241      0.19705486006917067   0.20157361069046778
0.5946782903186839      0.19657390058203658   0.20122663594444687
0.5948455875585801      0.19609058862363796   0.20127085119083413
0.5950138685790127      0.19560493177070448   0.20141067152272377
0.5951873007412084      0.19510492935539356   0.20127870129685482
0.5953533819614867      0.19462658343306055   0.20068341457580652
0.5955204469623014      0.19414574771569093   0.19968191704289584
0.5956884957436523      0.19366257266456516   0.19855653353613437
0.5958491935830859      0.1932009943879041    0.19769944649527993
0.5960150425642827      0.1927250928274676    0.1972547038034978
0.596181875326016       0.19224685220904125   0.19723336253223148
0.5963496918682853      0.1917662807339979    0.19738136792035277
0.5965226595523181      0.1912714711685306    0.19733508662168778
0.5966882762944334      0.190798178204529     0.1968524679879157
0.5968548768170849      0.19032255484344449   0.1959338063721233
0.597022461120273       0.18984460924915111   0.19481605940300126
0.5971826944815435      0.18938808411720776   0.1938920671808
0.5973480789845773      0.1889173496512082    0.1933404819108639
0.5975144472681475      0.18844429322422207   0.19323442728140836
0.5976817993322538      0.1879689229632403    0.193371455208903
0.5978543025381235      0.1874794277908789    0.19340247488549706
0.5980194548020759      0.1870112736310214    0.1930402434002895
0.5981855908465644      0.1865408060919843    0.19222408207113148
0.5983527106715893      0.18606803326347715   0.1911352440608779
0.5985208142771508      0.18559296327188157   0.19011135124637643
0.5986940690244752      0.18510384541298008   0.18945035441152772
0.5988599728298822      0.18463596448712058   0.1892819581525041
0.5990268604158258      0.18416578683781873   0.18939933449073648
0.5991947317823054      0.18369332055399482   0.1894729385100896
0.5993552522068677      0.18324199586989903   0.18921369612539
0.5995209237731933      0.18277665180315156   0.18849761624206715
0.5996875791200553      0.18230901935652497   0.1874538412757099
0.5998552182474535      0.18183910658208968   0.18639502493402368
0.600028008516615       0.1813552581077902    0.18563626363542818
0.6001934478438591      0.18089247243902312   0.18537118942227868
0.6003598709516393      0.18042740687741773   0.1854460389109337
0.6005272778399561      0.17996006943779716   0.18556019143259644
0.6006873337863554      0.17951370033916336   0.18540130344106576
0.6008525408745179      0.17905342376394986   0.18479982617249285
0.6010187317432167      0.1785908755618883    0.18382208857172677
0.6011859063924521      0.17812606371116788   0.18274441856418452
0.6013540648222235      0.1776589962259679    0.1819049659353912
0.6015148723100776      0.17721279593666692   0.18152196392816888
0.601680830939695       0.176752651992245     0.18151720362984888
0.6018477733498486      0.17629015824113092   0.18165497649700973
0.6020156995405388      0.1758254172230656    0.18159627246045226
0.602188776872992       0.17534692266570734   0.18108549891556852
0.6023545032635278      0.17488922920453068   0.1801810240242922
0.6025212134346001      0.17442929155957745   0.17910635654339238
0.6026889073862085      0.17396711862322595   0.17819795873172162
0.6028492503958995      0.17352565487034446   0.1777186163794444
0.6030147445473539      0.17307047018098637   0.1776392120233437
0.6031812224793447      0.1726130530343403    0.17777194870794227
0.6033486841918716      0.17215341234358364   0.17778610040297943
0.6035212970461619      0.1716801369148932    0.17739284306393607
0.6036865589585347      0.1712274963193346    0.17658008465458747
0.6038528046514438      0.17077263525940764   0.1755271182055513
0.6040200341248892      0.17031556266960202   0.174558298450894
0.6041882473788711      0.16985628755189408   0.17395755920545863
0.6043616117746161      0.16938345913860853   0.1738062053667702
0.6045276252284437      0.1689311660796571    0.173924854128985
0.6046946224628078      0.16847667360776986   0.17397805159646515
0.604862603477708       0.16801999074647436   0.17367532973082334
0.6050232335506909      0.16758374990446734   0.1729658107818186
0.6051890147654371      0.16713398884386738   0.17194975419590056
0.6053557797607194      0.16668204025675867   0.17094441341770622
0.6055235285365381      0.16622791318742064   0.17025441309829847
0.6056964284541202      0.16576035339896306   0.1700115331373461
0.6058619774297849      0.16531316011973346   0.17009667637469292
0.6060285101859857      0.164863791469755     0.17019303980787098
0.606196026722723       0.16441225651457547   0.16999139995138804
0.6063561923175429      0.16398099420797405   0.16938364831845976
0.6065215090541258      0.16353633240418275   0.16842349109323407
0.6066878095712454      0.1630895071530437    0.16739501238589144
0.606855093868901       0.16264052754058053   0.16661575850728058
0.6070233619470932      0.16218940272118879   0.16626978180302077
0.6071842790833679      0.16175845073511486   0.16629143153950718
0.6073503473614057      0.16131418068048597   0.16641835878938271
0.60751739941998        0.16086776831060082   0.16633336363023268
0.6076854352590906      0.16041922280056464   0.16582556775749105
0.6078586222399643      0.15995744633100403   0.16489193178041464
0.6080244582789207      0.15951554644525115   0.16386069392462163
0.6081912780984136      0.15907151619764062   0.16301430057383817
0.6083590816984425      0.15862536575680355   0.16257535639085458
0.6085195343565541      0.1581992285681064    0.1625353270141015
0.6086851381564291      0.15775989276088967   0.16266465284247544
0.6088517257368402      0.15731844245646537   0.1626523771367361
0.6090192970977877      0.15687488790616236   0.16225189108792332
0.6091920196004985      0.15641822773602773   0.16140400752855574
0.609357391161292       0.15598150759053828   0.16038620345023102
0.6095237465026215      0.15554268922120812   0.15947875185373095
0.6096910856244876      0.1551017829635652    0.15893887268249487
0.6098510738044363      0.15468072198989524   0.15881989619145298
0.610016213126148       0.15424659449906483   0.15893300416121092
0.6101823362283961      0.1538103848091245    0.15898406860951964
0.6103494431111807      0.1533721033374882    0.15869651489533126
0.6105175337745015      0.15293176060579472   0.15797574405478518
0.6106782734959049      0.15251116314151889   0.15702124468966686
0.6108441643590716      0.15207758886861686   0.15605554263451626
0.6110110390027745      0.15164195911049638   0.15539051266541243
0.6111788974270138      0.15120428447136391   0.15515341580657996
0.6113519069930164      0.15075372951979976   0.15523465581905646
0.6115175656171015      0.15032284349480873   0.1553261091027771
0.6116842080217231      0.14988991869819984   0.15513253306489574
0.6118518342068808      0.14945496581825257   0.1545109334974782
0.6120121094501212      0.14903958330625022   0.15360636609538447
0.6121775358351249      0.14861136086061322   0.15261966286741338
0.6123439460006648      0.14818111609949372   0.1518725599286089
0.612511339946741       0.1477488597929223    0.15153945510129557
0.6126838850345806      0.14730386180194766   0.1515678801903088
0.6128490791805028      0.14687835615604952   0.15168582541857387
0.6130152571069611      0.14645084507155395   0.151584906853484
0.613182418813956       0.14602133940174267   0.15107521069681695
0.6133505643014872      0.14558985010671116   0.15019058811797095
0.6135238609307814      0.14514571471406826   0.14915396518653276
0.6136898066181583      0.1447209650199596    0.1483616236913326
0.6138567360860716      0.1442942378980079    0.1479647651837328
0.6140246493345212      0.14386554439225002   0.1479465389477268
0.6141852116410533      0.14345605287670368   0.14806922119428842
0.6143509250893489      0.14303381875821103   0.14803932065246078
0.6145176223181805      0.14260962407680264   0.1476299104294369
0.6146853033275486      0.14218348092811905   0.14682199218463318
0.61485813547868        0.1417448317669437    0.14579580324893102
0.6150236166878937      0.14132539832355603   0.1449440359104193
0.6151900816776439      0.14090402546597738   0.14445381248596625
0.6153575304479305      0.14048072543598822   0.14436301284754716
0.6155176282762997      0.14007653843136395   0.14447785990716872
0.615682877246432       0.13965989268096585   0.14451181234860838
0.6158491099971007      0.1392413283836419    0.14420915510367752
0.6160163265283058      0.1388208579234849    0.14349525186191212
0.6161845268400472      0.13839849382826516   0.1425237294775405
0.6163453762098712      0.1379951349921806    0.14163706169922666
0.6165113767214585      0.13757942123696518   0.14102864340184496
0.616678361013582       0.13716182261002058   0.14082878432737758
0.6168463290862418      0.1367423517825588    0.14091538102811277
0.6170194483006648      0.13631063664641435   0.1409990072393032
0.6171852165731705      0.13589784494055698   0.14078600824322648
0.6173519686262126      0.13548319023035033   0.1401632309570709
0.6175197044597909      0.1350666853329349    0.1392356824865533
0.6176800893514518      0.13466899494852222   0.13832496598301194
0.6178456253848761      0.13425910960384635   0.13763873378033972
0.6180121451988365      0.1338473828261353    0.13735187594521261
0.6181796487933333      0.1334338275746222    0.13739455583495908
0.6183523035295935      0.13300819101119965   0.1375091361867714
0.618517607323936       0.13260128422432874   0.13738615864446663
0.6186838948988149      0.13219255815800948   0.13686719953514634
0.6188511662542303      0.13178202591604704   0.13600233226791766
0.619019421390182       0.13136970075020485   0.13503647313287348
0.6191928276678969      0.13094540710204958   0.1342630940860158
0.6193588830036942      0.13053972540120537   0.13391932398032952
0.619525922120028       0.1301322601167667    0.13392570809779666
0.6196939450168982      0.12972302464620883   0.13404902864090326
0.619854616971851       0.12933228592608165   0.13399441995079545
0.6200204400685669      0.12892963017167686   0.13356915478401105
0.6201872469458192      0.12852521311497128   0.13277281303226807
0.6203550376036078      0.12811904829531182   0.13181365437595471
0.6205279794031595      0.12770092091032065   0.13098027001440218
0.620693570260794       0.12730116737477654   0.13055070063542581
0.6208601448989648      0.12689967577753644   0.13049437614514242
0.6210277033176719      0.12649646068476228   0.13061803235201155
0.6211879107944616      0.12611154549874223   0.13062792816529256
0.6213532694130146      0.12571488312916979   0.13030372143908772
0.6215196118121038      0.12531650865453156   0.12959226908759383
0.6216869379917294      0.12491643683730504   0.12865664339163163
0.6218594153131181      0.1245047437234217    0.12777118986923178
0.6220245416925895      0.12411126114153616   0.12724988947048246
0.6221906518525973      0.12371609313497346   0.12711432806760625
0.6223577457931413      0.12331925466588839   0.1272198975494981
0.6225258235142217      0.122920760880443     0.1272819603677085
0.6226990523770652      0.12251077262334815   0.12702113013891334
0.6228649302979915      0.12211886887530939   0.12637469461352752
0.6230317919994541      0.12172532192427364   0.1254690002304285
0.623199637481453       0.12133014711762687   0.12458123100180835
0.6233601320215345      0.120952931864872     0.12400433775954728
0.6235257777033791      0.12056428177710628   0.12379246995508726
0.6236924071657602      0.12017401539893571   0.12386684021684413
0.6238600204086776      0.11978214827376221   0.123965969912321
0.6240327847933582      0.11937897993926858   0.12379805352903551
0.6241981982361213      0.11899367489649704   0.12324617710396002
0.6243645954594208      0.11860678121863728   0.12239103791913428
0.6245319764632568      0.11821831464847214   0.12148322173246512
0.6246920065251752      0.1178475816373459    0.12083293926768034
0.6248571877288569      0.11746560843415557   0.12053295416890014
0.6250233527130749      0.11708207389223432   0.12055823996372324
0.6251905014778292      0.11669699394855353   0.12068019608540081
0.6253586340231199      0.11631038472989745   0.12060926098354892
0.625519415626493       0.11594137309902948   0.12018902226534896
0.6256853483716296      0.11556125420279836   0.11941677990602219
0.6258522648973024      0.11517961777143396   0.11850491520617619
0.6260201652035116      0.11479648012744281   0.1177423932243687
0.6261932166514839      0.1144023765638815    0.11733504667094327
0.6263589171575389      0.11402576745591465   0.11730596849592989
0.6265256014441302      0.11364766944102814   0.1174325305124104
0.6266932695112577      0.11326806435893183   0.11742878043060255
0.626853586636468       0.11290569683990032   0.11710142921783015
0.6270190549034413      0.11253242308919133   0.11640664442039936
0.6271855069509512      0.11215768920935931   0.11551437179077943
0.6273529427789973      0.11178151262305369   0.11470423384654145
0.6275255297488067      0.11139457764515169   0.11420868239295066
0.6276907657766986      0.11102490211810065   0.1141098987089339
0.6278569855851268      0.11065379824523106   0.11422485066824299
0.6280241891740915      0.11028128369180912   0.11428028210476339
0.6281923765435925      0.10990737634324668   0.11402874439126849
0.6283657150548567      0.1095228599289859    0.11337055935732061
0.6285317026242034      0.10915545591536018   0.11250471069011418
0.6286986739740865      0.10878667370369949   0.11167551551228501
0.628866629104506       0.10841653142390587   0.11113533781962243
0.6290272332930079      0.10806335536173832   0.11097795048027115
0.6291929886232732      0.10769964082543532   0.11106907341711052
0.6293597277340748      0.10733458016596668   0.11116340555152836
0.6295274506254127      0.1069681917514187    0.11099801657041192
0.6297003246585138      0.10659142207179169   0.11043153808390302
0.6298658477496976      0.10623150625491476   0.10960982716915768
0.6300323546214176      0.10587027727854002   0.10875966917052142
0.6301998452736739      0.1055077537530145    0.10814708730084302
0.6303599849840129      0.10516193210337883   0.10791307948300882
0.6305252758361151      0.10480580145793562   0.10796432764060813
0.6306915504687537      0.10444839019445122   0.1080856130616539
0.6308588088819286      0.10408971715924505   0.10800551695731934
0.6310270510756397      0.10372980142581024   0.10755841788872604
0.6311879423274336      0.10338642837811994   0.10682547759484484
0.6313539847209906      0.10303290312223741   0.10596678953221139
0.631521010895084       0.10267814932456494   0.10527045245210666
0.6316890208497136      0.10232218629642541   0.10492390113873437
0.6318621819461065      0.10195623761548624   0.10492278958106058
0.632027992100582       0.10160671094836446   0.10505415682961158
0.6321947860355939      0.10125598988102605   0.10504026330345531
0.6323625637511421      0.10090409396670617   0.10468447865157436
0.6325229905247728      0.1005684578789438    0.10401880787872683
0.6326885684401669      0.1002229118191864    0.10317482716998799
0.6328551301360972      0.09987620505457753   0.10243256768461795
0.6330226756125639      0.09952829643819652   0.10200777273234803
0.6331953722307937      0.09917060438907822   0.10194350828968239
0.6333607179071062      0.09882905189707272   0.10207101071027386
0.633527047363955       0.0984863738520611    0.10211737599010046
0.6336943606013401      0.09814259073931701   0.10185883750019681
0.6338543228968079      0.09781478028294228   0.10127480291734928
0.6340194363340388      0.09747730814328637   0.10046186332198613
0.6341855335518062      0.09713874634903531   0.09968290738810129
0.6343526145501098      0.0967991156473816    0.09917657857177956
0.6345206793289497      0.09645843703669801   0.09903370264486855
0.6346813931658724      0.09613355629843133   0.0991329524643176
0.6348472581445581      0.09579918636038905   0.09923560044100528
0.6350141069037802      0.09546378417626994   0.09909326983195152
0.6351819394435386      0.09512737100842794   0.09858835447249818
0.6353549231250604      0.09478165183013906   0.09778417172141843
0.6355205558646646      0.09445159804738862   0.09698842456498868
0.6356871723848052      0.0941205496879175    0.09641790276871277
0.6358547726854822      0.09378852828259393   0.09620296874164617
0.6360150220442417      0.09347199406063267   0.09626858383211007
0.6361804225447644      0.0931462372013388    0.09639706649013925
0.6363468068258233      0.09281952294333322   0.096334162100207
0.6365141748874188      0.09249187307900367   0.09592417617687117
0.6366866940907774      0.0921551893547525    0.09518235150318884
0.6368518623522187      0.0918338549911558    0.09438249115014229
0.6370180143941963      0.09151160142582569   0.09375040068172888
0.6371851502167101      0.09118845071719123   0.09345409003438446
0.6373532698197604      0.09086442518266939   0.09347648978063139
0.6375265405645738      0.09053155441231626   0.0936211447319931
0.63769246036747        0.0902138402142311    0.09360702619235746
0.6378593639509024      0.08989526785940995   0.09326412286222606
0.638027251314871       0.08957585993339234   0.09259540747518351
0.6381877877369224      0.08927141895618913   0.09182687340314447
0.6383534753007369      0.08895821993209449   0.09115240869957633
0.6385201466450878      0.08864420121969034   0.09078613786821996
0.638687801769975       0.08832938566563185   0.0907549332803217
0.6388606080366255      0.08800601129016451   0.09089989726612398
0.6390260633613585      0.08769745674162202   0.090945262586581
0.6391925024666278      0.0873881215587328    0.09069351015797127
0.6393599253524336      0.08707799626201629   0.09010235719560394
0.6395199972963218      0.08678249640432725   0.08935865743891198
0.6396852203819733      0.08647852599889973   0.08864911097740567
0.639851427248161       0.08617381461973302   0.08820905460037576
0.6400186178948852      0.085868385786982     0.0881109491410684
0.6401867923221458      0.08556226329267926   0.08823631472480704
0.6403476158074889      0.08527056098171534   0.08833605928884035
0.6405135904345952      0.08497058546055614   0.08819474851993765
0.6406805488422378      0.08466993282183265   0.08771105786938257
0.6408484910304169      0.08436862712767063   0.08697680251493407
0.641021584360359       0.08405925851305993   0.08621138068303909
0.6411873267483839      0.08376415420210083   0.08571325755817105
0.6413540529169449      0.08346841420001498   0.08555246628514106
0.6415217628660423      0.08317206284370683   0.08565159018471347
0.6416821218732225      0.08288976849920072   0.08578073936757667
0.6418476320221658      0.08259950488773882   0.08571574604304105
0.6420141259516454      0.08230864644868648   0.08531964740563944
0.6421816036616613      0.08201721778649927   0.08464054907508366
0.6423542325134406      0.08171803586757526   0.08386976999011
0.6425195104233024      0.08143274960539662   0.08331476867685234
0.6426857721137005      0.08114691048088614   0.08308225407679166
0.642853017584635       0.0808605433707216    0.08314047155999017
0.6430212468361056      0.08057367343152612   0.08329296603733007
0.6431946272293396      0.08027926456854889   0.08328058650609953
0.6433606566806562      0.07999852710636705   0.08294634711788527
0.6435276699125091      0.07971730445114596   0.08231492746136564
0.6436956669248983      0.07943562203388115   0.08157263727149143
0.6438563129953703      0.07916739188670745   0.08099575472671594
0.6440221102076054      0.07889172042758652   0.08070164971049727
0.6441888912003768      0.07861560597745222   0.08071586181953075
0.6443566559736845      0.07833907423443322   0.08087359152905699
0.6445295718887555      0.07805532802881668   0.08092251056807169
0.6446951368619089      0.07778486309040616   0.08067333117541474
0.6448616856155989      0.07751399849109651   0.0801114860977728
0.6450292181498252      0.07724276020082893   0.07938909792960176
0.645189399742134       0.0769845775242258    0.07877816465340404
0.645354732476206       0.07671927987762056   0.07841794160155673
0.6455210489908143      0.07645364127935833   0.07837597264650861
0.6456883492859591      0.07618769753439744   0.07852560043946816
0.6458566333616402      0.07592144960382481   0.07862898440834946
0.6460175664954038      0.07566801769441224   0.07848459632190151
0.6461836507709307      0.07540769374044692   0.07801978517043717
0.6463507188269939      0.07514708208847055   0.07733770951093956
0.6465187706635933      0.07488620933531245   0.07666593665366493
0.6466919736419561      0.07461868151476526   0.07622497912012881
0.6468578256784014      0.074363788341953     0.07613023960579292
0.6470246614953831      0.074108651429522     0.07626239659242802
0.6471924810929011      0.07385329763385781   0.07640118312348262
0.6473529497485018      0.07361034080901654   0.07632884551608736
0.6475185695458656      0.07336083137336392   0.07594380202854259
0.6476851731237658      0.07311112152370769   0.07530922169726632
0.6478527604822022      0.07286123836904211   0.0746306977731708
0.6480254989824019      0.07260504498768722   0.07413535979408815
0.6481908865406842      0.07236106202358372   0.07397899600888126
0.6483572578795028      0.07211692311097484   0.07408088046927971
0.6485246129988578      0.07187265561595374   0.07424535451583397
0.6486846171762954      0.07164035468545003   0.07424397966255104
0.6488497724954962      0.07140184776288057   0.07394705849740948
0.6490159115952334      0.07116322870848074   0.0733739507218098
0.6491830344755068      0.07092452513851438   0.072700903943002
0.6493511411363166      0.07068576496011925   0.07216453762349441
0.6495118968552092      0.07045871438737715   0.07193318735397464
0.6496778037158647      0.07022569356612926   0.07197915517773354
0.6498446943570566      0.06999263284058722   0.07215542417894708
0.650012568778785       0.06975956037010594   0.0722280456885
0.6501855943422764      0.0695207704448572    0.07200682167733041
0.6503512689638504      0.06929349430645704   0.07149787541585573
0.6505179273659609      0.0690662240400871    0.07084435332928951
0.6506855695486078      0.06883898806181853   0.0702766511687621
0.650845860789337       0.06862301065515274   0.06998896478242289
0.6510113031718296      0.06840142382698473   0.0699870788708491
0.6511777293348586      0.0681798879719068    0.07015900399950134
0.6513451392784237      0.06795843175576513   0.0702814791558196
0.6515177003637522      0.06773162426119543   0.07014656084968386
0.6516829105071633      0.06751587439210771   0.06971318827759641
0.6518491044311107      0.06730030821334239   0.06909212647503757
0.6520162821355944      0.06708489632128364   0.06850052124867174
0.6521844436206146      0.06686963962348806   0.06814422277531282
0.6523577562473978      0.06664928652191482   0.06810248633837415
0.6525237179322636      0.0664397080972302    0.06826747692295972
0.6526906633976659      0.06623030125277053   0.06841850528138545
0.6528585926436044      0.06602109492894394   0.06834707563115305
0.6530191709476254      0.06582239564614424   0.0679913805752793
0.6531849003934098      0.06561870921523584   0.06741136577387502
0.6533516136197306      0.06541523877053916   0.06681207896759363
0.6535193106265875      0.06521201346976224   0.06640879744846526
0.6536921587752078      0.06500406739050321   0.06631237631725127
0.6538576559819107      0.06480641636177302   0.06645634023144707
0.6540241369691498      0.06460902685594405   0.06663693452571452
0.6541916017369254      0.06441192825201021   0.06663766248589612
0.6543517155627836      0.06422485173459826   0.06636030023283718
0.6545169805304049      0.06403316522291706   0.06583349593322926
0.6546832292785625      0.06384178505966893   0.06523624818982682
0.6548504618072566      0.06365074083898395   0.06478779685124986
0.6550186781164868      0.06346006244321947   0.06462901186074987
0.6551795434837997      0.06327911772317542   0.06472982248784827
0.6553455599928759      0.06309381894025613   0.06492996119063081
0.6555125602824885      0.0629089016580095    0.06501104304270283
0.6556805443526372      0.06272439597574561   0.06481625384913951
0.6558536795645493      0.06253581193329737   0.06432788443661182
0.656019463834544       0.06235674126759066   0.0637453690551854
0.6561862318850749      0.06217809881370831   0.06326746236821766
0.6563539837161422      0.06199991489190579   0.06305736388603358
0.6565143846052921      0.06183096119420872   0.06312209171956507
0.6566799366362052      0.06165804332560652   0.0633258767968111
0.6568464724476546      0.061485599645353405  0.06345818558284749
0.6570139920396405      0.06131366068849039   0.06334204031388499
0.6571866627733894      0.06113803787249162   0.06292405445631738
0.657351982565221       0.060971420556467155  0.062366871425839264
0.657518286137589       0.06080532456799394   0.06186454657776856
0.6576855734904932      0.060639780660880646  0.061599570166148684
0.6578538446239337      0.0604748198840389    0.06162527771109594
0.6580272668991376      0.0603064871782242    0.061836359374398055
0.6581933382324241      0.06014699154979567   0.06199931891005579
0.6583603933462467      0.05998809602288694   0.06193608293002235
0.6585284322406059      0.05982983107906374   0.06158418349985664
0.6586891201930476      0.059679961727578     0.0610721279013366
0.6588549592872525      0.05952680023597277   0.06055982161929039
0.6590217821619937      0.059374282926063546  0.0602527115873909
0.6591895888172714      0.059222440443640396  0.06023447358182475
0.6593625466143122      0.05906759248115528   0.06043307174034047
0.6595281534694355      0.05892090390578324   0.06063057916710308
0.6596947441050953      0.058774904668316946  0.06063696861312113
0.6598623185212913      0.05862962558196638   0.060358882154371386
0.6600225419955699      0.058492206365872636  0.059889091210028694
0.6601879166116118      0.058351896539588506  0.059373942332507104
0.6603542750081901      0.05821232044421211   0.05902454219558303
0.6605216171853046      0.058073509055442854  0.058954653293550555
0.6606941105041824      0.05793209721578175   0.0591292494136819
0.6608592528811428      0.05779830567163936   0.05935308617799678
0.6610253790386393      0.05766529333515056   0.05942840852897057
0.6611924889766724      0.05753309134769034   0.05923184091843589
0.6613605826952418      0.05740173112552177   0.05879056626253593
0.6615338275555743      0.057268048996001186  0.05826046655498467
0.6616997214739895      0.05714166302617491   0.057893558703379824
0.6618665991729411      0.057016133521543874  0.057794313868116835
0.6620344606524289      0.056891492065514214  0.057946080115091766
0.6621949711899993      0.056773838910894034  0.058178873393368906
0.662360632869333       0.05665398150457202   0.05830919370661318
0.6625272783292029      0.056535025897518874  0.058186026260234495
0.6626949075696091      0.05641700383534481   0.05780305066667714
0.6628676879517788      0.0562970753148364    0.057289518438396365
0.663033117392031       0.05618388872190321   0.05689475000171596
0.6631995306128193      0.05607165036404017   0.05674675506332365
0.6633669276141442      0.05596039215225387   0.05686359340986505
0.6635269736735516      0.05585556400579354   0.05710220146034992
0.6636921708747222      0.05574894819597876   0.05728378959776484
0.6638583518564292      0.055643326263537705  0.05723828244875311
0.6640255166186726      0.05553873027994889   0.05692397560036551
0.6641936651614522      0.055435192597083564  0.056450283043979976
0.6643544627623144      0.05533788882647024   0.05603771765852741
0.6645204115049399      0.05523917199439863   0.05582408641247522
0.6646873440281016      0.05514152719157179   0.0558830679664234
0.6648552603317996      0.05504498592068898   0.05612503358942156
0.6650283277772611      0.05494724751254504   0.056358346152147215
0.6651940442808051      0.05485534122671988   0.056379517392735085
0.6653607445648853      0.05476455038276969   0.05613148253506586
0.6655284286295019      0.05467490658316771   0.05569362750021787
0.6656887617522012      0.05459077298223184   0.05527401836265546
0.6658542460166634      0.054505558582934155  0.0550211625936493
0.6660207140616622      0.05442150223605779   0.05503706116206391
0.6661881658871973      0.05433863564048258   0.05526445049962505
0.6663607688544956      0.05425499246233206   0.05553062690599602
0.6665260208798764      0.05417659974658066   0.05561963448697154
0.6666922566857938      0.05409940868049136   0.05544586189115034
0.6668594762722473      0.054023451061527165  0.05505462152384011
0.6670276796392371      0.0539487589371697    0.05461585219731967
0.6672010341479904      0.05387357875364379   0.05432788678882004
0.667367037714826       0.053803300618870946  0.05432056785395579
0.6675340250621979      0.05373430002134256   0.05453854318543293
0.6677019961901064      0.053666609108125415  0.0548185113558923
0.6678626163760975      0.0536034924166624    0.05495731737504016
0.6680283877038515      0.053540005974183943  0.054851394180747054
0.6681951428121421      0.053477840344085155  0.054510022971610936
0.668362881700969       0.05341702776964453   0.05408189788791855
0.6685357717315591      0.05335615506059129   0.053764791241016435
0.6687013108202318      0.05329959202419098   0.05371493400949787
0.6688678336894409      0.05324439407428479   0.05390776291587667
0.6690353403391862      0.05319059355254334   0.054203140975081485
0.6691954960470141      0.05314077246567885   0.054395346118185264
0.6693608028966054      0.05309101022591526   0.05436276502987542
0.6695270935267328      0.05304265652398581   0.05408108494332112
0.6696943679373966      0.05299574379658895   0.053672526881565506
0.6698626261285968      0.05295030473420238   0.05333500351565013
0.6700235333778797      0.0529084941948514    0.05323019570149623
0.6701895917689256      0.052867031826052     0.0533764207599849
0.6703566339405079      0.052827054370259495  0.05367435267199663
0.6705246598926267      0.052788597107033076  0.053931238172237765
0.6706978369865084      0.0527510709534166    0.05397116426657023
0.6708636631384729      0.05271689133665723   0.053748096094120956
0.6710304730709737      0.05268424085348356   0.0533669793828395
0.6711982667840108      0.05265315123039836   0.05301686107059979
0.6713587095551306      0.05262506924664363   0.05287549814802137
0.6715243034680136      0.05259777368415881   0.05298671645837072
0.6716908811614328      0.052572046920117556  0.05327992494383542
0.6718584426353884      0.05254792070301222   0.05357454427222704
0.6720311552511073      0.05252489338079467   0.053684753737812625
0.6721965169249087      0.052504597987679626  0.05352872575043421
0.6723628623792464      0.052485911915576354  0.05318448853604517
0.6725301916141205      0.05246886693599164   0.05282699988343052
0.6726901699070773      0.05245421437168493   0.05264718177989536
0.6728552993417971      0.052440776076796286  0.052715083455985016
0.6730214125570534      0.052428986795984     0.052993009612965006
0.673188509552846       0.052418878322188237  0.05331831807276348
0.6733565903291749      0.05241048266569162   0.05349824424246185
0.6735173201635865      0.052404117739664344  0.05342896070550152
0.6736832011397613      0.05239925515341129   0.0531408680834838
0.6738500658964723      0.05239611338415629   0.05278260215440488
0.6740179144337197      0.05239472446408234   0.05254882606112784
0.6741909141127302      0.0523951531755103    0.05257346609059526
0.6743565628498235      0.05239733439156276   0.05283265386467537
0.6745231953674531      0.052401277298048325  0.05317784312144855
0.674690811665619       0.05240701395007692   0.05341482459305863
0.6748510770218674      0.05241416051069307   0.053413411127145816
0.6750164935198792      0.052423241102054464  0.053178855665380526
0.6751828937984272      0.05243412342174682   0.052835253051215725
0.6753502778575116      0.05244683954622318   0.052575497151262814
0.6755228130583593      0.05246180549647859   0.05255415046184213
0.6756879973172893      0.052477902609824975  0.052784457084395324
0.6758541653567558      0.05249584235533079   0.05314075050938622
0.6760213171767587      0.05251565683181591   0.05343233291119879
0.6761894527772979      0.052537378357210514  0.05350093167079985
0.6763627395196004      0.052561645472223736  0.05330530323072575
0.6765286753199853      0.05258667292509308   0.05298151582107163
0.6766955949009066      0.05261361633572733   0.05271445924001998
0.6768634982623644      0.052642629317299204  0.05266697598974516
0.6770240506819046      0.05267219878248913   0.05286668852120326
0.6771897542432082      0.05270443810911571   0.05322624929057259
0.6773564415850479      0.05273863259817553   0.05356058401140628
0.6775241127074241      0.05277481338998529   0.05369697236864678
0.6776969349715634      0.05281397858392962   0.053565623144085944
0.6778624062937854      0.05285325933527132   0.053270401234524416
0.6780288613965437      0.052894531743003105  0.052990255300597325
0.6781963002798383      0.0529378268917586    0.05290195910942248
0.6783563882212156      0.052980889407182256  0.05306515273022474
0.6785216273043562      0.053027047603476714  0.053418139514354326
0.6786878501680329      0.053075233117482516  0.05379005673389533
0.678855056812246       0.05312547697843915   0.05399685092472547
0.6790232472369955      0.05317781039392063   0.05394386037314158
0.6791840867198274      0.053229539317246444  0.053698787726570545
0.6793500773444228      0.05328464984756417   0.053408681402082735
0.6795170517495543      0.053341854524123065  0.053266290681691
0.6796850099352223      0.053401184499017665  0.05338322877608798
0.6798581192626534      0.05346421074664067   0.053737714594792735
0.6800238776481672      0.0535263458430173    0.05413668923027442
0.6801906198142174      0.053590611590287926  0.05440327772878982
0.6803583457608038      0.05365703908487998   0.05441830838392394
0.6805187207654728      0.053722227093275024  0.054217526542635945
0.680684246911905       0.053791222694143444  0.05393393957542332
0.6808507568388736      0.053862384619025254  0.053761416015033016
0.6810182505463785      0.05393574390897008   0.05383539282583327
0.6811908953956467      0.05401322432941113   0.05416666506655739
0.6813561893029974      0.05408917963850105   0.054585240743409175
0.6815224669908844      0.05416733764963364   0.05491119608282299
0.6816897284593079      0.05424772934829392   0.055001435918173365
0.6818579737082676      0.054330385897978464  0.05484340021265444
0.6820313700989906      0.054417453586053044  0.05456095485674189
0.6821974155477961      0.054502619093900855  0.05437731872908732
0.682364444777138       0.054590054804171416  0.05442866997251906
0.6825324577870162      0.054679791824710604  0.05473658017570637
0.6826931198549769      0.05476727818998046   0.05515636317817646
0.682858933064701       0.054859287053042764  0.05553060193953734
0.6830257300549614      0.054953601802997604  0.05568868898632657
0.6831935108257581      0.05505047605311883   0.05558828650066278
0.6833664427383179      0.05515224321796641   0.05532767986904693
0.6835320237089605      0.05525145997453236   0.05512420717174518
0.6836985884601393      0.05535301731910468   0.05513350098609248
0.6838661369918544      0.05545694503544119   0.05540832467483924
0.6840263345816522      0.055557972442369374  0.05583048376228515
0.6841916833132131      0.05566394833597782   0.05624958074335738
0.6843580158253105      0.05577229576031142   0.05648005254294917
0.6845253321179442      0.05588304436852258   0.056447637402195924
0.6846977995523411      0.055999049080691435  0.0562196596985285
0.6848629160448205      0.05611186442873236   0.05600082849573435
0.6850290163178363      0.05622708278897062   0.05596391511049039
0.6851961003713886      0.05634473368257174   0.05619457985083574
0.6853641682054771      0.05646484676632186   0.056631356207404844
0.6855373871813288      0.05659049686729618   0.0571047046439157
0.6857032552152631      0.05671257880534474   0.05738626872866791
0.6858701070297337      0.05683712470989063   0.057406007498378604
0.6860379426247407      0.056964164105096894  0.0572158440899544
0.6861984272778302      0.05708728822617133   0.05699882724608099
0.686364063072683       0.05721605341714518   0.05692567295377416
0.6865306826480722      0.05734731319673411   0.05711495364298547
0.6866982860039976      0.05748109695861962   0.05753666564642321
0.6868710405016862      0.05762082684564133   0.05804067214132714
0.6870364440574576      0.057756354673701785  0.05838788047377252
0.6872028313937651      0.057894408243449894  0.058481152243826884
0.687370202510609       0.0580350168167012    0.0583388514438948
0.6875302226855355      0.058171079293659805  0.058124468152706416
0.6876953940022252      0.05831319102278325   0.05801450024744456
0.6878615490994514      0.058457858840595935  0.058153498844314096
0.6880286879772138      0.05860511187918297   0.05854727341469962
0.6881968106355125      0.05875497940386333   0.0590594562365667
0.6883575823518939      0.058899932509999     0.059467671389704164
0.6885235052100385      0.05905120817249801   0.0596563675179506
0.6886904118487194      0.059205099358081804  0.059586185973893285
0.6888583022679365      0.05936163520170816   0.05937664870559789
0.6890313438289171      0.059524796102582454  0.05923063164099344
0.6891970344479801      0.059682758062820335  0.059325674937506095
0.6893637088475795      0.05984342877235233   0.05968887055314695
0.6895313670277152      0.06000696768779272   0.060209843526086375
0.6896916742659335      0.060164953950530224  0.06066753438448294
0.6898571326459151      0.06032967293531918   0.06093038486174287
0.690023574806433       0.06049706633236581   0.060925246691188616
0.6901910007474872      0.06066716187049847   0.06074199387102774
0.6903635778303046      0.06084428598614896   0.060573496522704925
0.6905288039712046      0.0610155707464882    0.06061874707680042
0.690695013892641       0.061189556045478585  0.06093911268063336
0.6908622075946136      0.06136626940881058   0.061456522184466435
0.6910303850771227      0.06154573845388303   0.06198086912357088
0.6912037137013949      0.06173250267844222   0.062318651440863605
0.6913696913837498      0.0619130545098118    0.06236379420568449
0.691536652846641       0.06209636030537241   0.06220744116527071
0.6917045980900685      0.06228244747823086   0.06203554343595664
0.6918651923915786      0.062461983273462605  0.06204285866776253
0.6920309378348519      0.06264890919175545   0.06232179380829852
0.6921976670586616      0.06283861419911808   0.06282793405638362
0.6923653800630075      0.0630311255082507    0.06338673660179586
0.6925382442091168      0.06323131769123316   0.06379673076014906
0.6927037574133086      0.06342467632591361   0.06391458318326737
0.6928702543980367      0.06362083952686826   0.0638010490161322
0.6930377351633012      0.06381983430387955   0.06362392623665557
0.6931978649866483      0.0640116613173409    0.06358916876253452
0.6933631459517585      0.06421126272951978   0.06381629520061699
0.693529410697405       0.06441369341917853   0.06429761383041124
0.693696659223588       0.06461898019705661   0.06488135150164802
0.6938648915303074      0.06482714996127034   0.06535494017137565
0.6940257728951092      0.06502779327390695   0.06556061751001378
0.6941918054016744      0.06523646750288498   0.06551458255401844
0.6943588216887757      0.06544802231135967   0.06534385320744233
0.6945268217564136      0.06566248439725997   0.06526033376340215
0.6946999729658145      0.0658852630595593    0.06544253414173154
0.6948657732332982      0.06610023761856543   0.06589504021627479
0.6950325572813182      0.06631811758659795   0.06649217843364544
0.6952003251098744      0.06653892945888215   0.06702320935296467
0.6953607419965133      0.066751608559202     0.06730310898940849
0.6955263100249154      0.06697269589702819   0.067319380928835
0.6956928618338538      0.0671968553614331    0.0671687663627454
0.6958603974233285      0.06742405823151726   0.06705712374237892
0.6960330841545666      0.0676599562099339    0.0671830117088649
0.6961984199438871      0.06788743508275392   0.06759335810859182
0.696364739513744       0.0681178656408841    0.0681914166106181
0.6965320428641373      0.06835127293888889   0.06877510291237739
0.696691995272613       0.06857593508256479   0.06913456364908929
0.6968570988228521      0.06880937748673534   0.0692261941169128
0.6970231861536273      0.0690457913224433    0.06910955808253713
0.6971902572649391      0.06928520138296292   0.06897664593929219
0.6973583121567872      0.06952763250848643   0.06903621295876174
0.6975190161067178      0.06976096775834614   0.06937112782755307
0.6976848711984117      0.07000332529896384   0.06994634470858552
0.6978517100706418      0.07024869843425777   0.07057645472618268
0.6980195327234084      0.07049711174152332   0.07104638748494188
0.6981925065179382      0.07075481851238065   0.0712279163788972
0.6983581293705505      0.07100315739977502   0.07115367718489322
0.6985247360036991      0.07125453142673191   0.07101352326126069
0.6986923264173841      0.07150896490411753   0.0710250943568989
0.6988525658891518      0.07175371247339345   0.07130637644664305
0.6990179565026826      0.07200783596545036   0.0718548769523841
0.6991843308967498      0.07226501342977423   0.07250973432755475
0.6993516890713533      0.07252526891625072   0.07305113577866716
0.6995241983877201      0.07279516599322819   0.0733210337093742
0.6996893567621694      0.073055110363058     0.07330400663506235
0.699855498917155       0.07331812770435847   0.0731678618247596
0.7000226248526771      0.07358424180251338   0.07313248935887463
0.7001907345687354      0.07385347648374266   0.07336959004670765
0.7003639954265569      0.07413258831781637   0.07391874546185988
0.700529905342461       0.07440140309384885   0.07458811987554566
0.7006967990389015      0.07467333322879659   0.07517844576418393
0.7008646765158784      0.07494840228275618   0.07550990531053843
0.7010252030509377      0.07521286435589326   0.07554960335776967
0.7011908807277604      0.07548728332263599   0.07542993162356325
0.7013575421851193      0.07576483555904531   0.07536168495051189
0.7015251874230146      0.07604554436455327   0.07553971564251699
0.7016979838026731      0.07633646830478615   0.0760425981190915
0.7018634292404142      0.07661653414382749   0.07671432159332343
0.7020298584586916      0.07689994327044418   0.07736005675945448
0.7021972714575053      0.07718652704818617   0.07777764527221574
0.7023573335144018      0.07746192947619285   0.07789017381891011
0.7025225467130614      0.07774762858118994   0.07780120444220687
0.7026887436922573      0.0780364941572759    0.07770667726013031
0.7028559244519895      0.07832854806502663   0.07782030566389904
0.7030240889922581      0.07862381216902929   0.07824704719656596
0.7031849025906094      0.07890756476330132   0.07888286918604398
0.7033508673307237      0.07920183273323211   0.07957859004682417
0.7035178158513744      0.07949930250875606   0.0800989859860944
0.7036857481525616      0.07979999563755988   0.08031640529620013
0.7038588315955118      0.0801114509486403    0.08027018035042482
0.7040245640965446      0.08041113814578686   0.08016402275317797
0.7041912803781138      0.08071404062490706   0.08022500050794786
0.7043589804402195      0.0810201796123586    0.08059249722411915
0.7045193295604076      0.08131425457912334   0.08120260494584444
0.704684829822359       0.08161916190210357   0.08192505330827315
0.7048513138648467      0.08192729733747939   0.0825194635094196
0.7050187816878708      0.0822386817971125    0.08282471934512496
0.7051914006526581      0.0825611387730268    0.08283766789629707
0.7053566686755279      0.08287128142312075   0.08273115153671243
0.7055229204789341      0.08318466500779953   0.08274002829360905
0.7056901560628766      0.08350131012001213   0.0830369896859027
0.7058583754273555      0.08382123734909229   0.08363978165588373
0.7060317459335975      0.08415244629751169   0.08441299363654528
0.7061977654979221      0.08447102048377331   0.08505747738864226
0.7063647688427832      0.08479286847220545   0.08542752353100462
0.7065327559681807      0.08511801053119605   0.0854921822854658
0.7066933921516605      0.08543023226439053   0.08539972008548492
0.7068591794769037      0.08575380242271845   0.08537103559729421
0.7070259505826832      0.08608065803806672   0.08560480005789708
0.707193705468999       0.08641081906472155   0.08615950972650649
0.707366611497078       0.08675255919765829   0.08693501067533824
0.7075321665832397      0.08708113712415252   0.08763721583670846
0.7076987054499376      0.08741301212936124   0.0880952140102573
0.7078662280971719      0.0877482038490462    0.08823557996403425
0.7080263998024889      0.08806995533421613   0.0881689911788439
0.708191722649569       0.088403422073071     0.08810905311817298
0.7083580292771854      0.08874029707012954   0.08827365397924992
0.7085253196853383      0.08908049946310549   0.08876444531861959
0.7086935938740273      0.08942404779935266   0.08950456156251675
0.7088545171207989      0.08975384590246605   0.09024030963146744
0.7090205915093339      0.09009548502806661   0.09080439048937614
0.7091876496784053      0.09044045906837678   0.09105180347106667
0.7093556916280128      0.09078878620848929   0.09103302036416053
0.7095288847193837      0.09114917261225433   0.09095245787546818
0.7096947268688372      0.09149557231836584   0.09105993584956146
0.7098615527988269      0.09184531429446151   0.09148824550442447
0.710029362509353       0.09219841635817619   0.09220139520566396
0.7101898212779618      0.09253726323365861   0.09296550223040605
0.7103554311883337      0.09288822626747656   0.09360560674589079
0.7105220248792419      0.09324253835764106   0.09394171369573198
0.7106896023506866      0.09360021696234701   0.09398023372969028
0.7108623309638943      0.09397022236016818   0.0938961980273256
0.7110277086351846      0.094325743322175     0.09394660214430278
0.7111940700870114      0.09468461995083217   0.09430050857838286
0.7113614153193746      0.0950468693397113    0.0949685217669477
0.7115214096098201      0.09539437443986361   0.09574668728390374
0.711686555042029       0.09575426068848471   0.09645897243833658
0.7118526842547743      0.09611750866366416   0.0968933396682986
0.7120197972480558      0.09648413510234678   0.09700735784180019
0.7121878940218737      0.09685415669363998   0.09693896691905718
0.7123486398537742      0.09720915269725024   0.09693053422138044
0.7125145368274378      0.09757670450980324   0.09718423116925219
0.7126814175816378      0.09794764018740452   0.09777232118870806
0.7128492821163743      0.09832197606017243   0.0985820625537914
0.7130222977928738      0.0987090672094105    0.09939097841992653
0.713187962527456       0.09908091344901448   0.09991341511853256
0.7133546110425746      0.09945614875010741   0.1001060323618828
0.7135222433382293      0.09983478907862633   0.1000670314728656
0.7136825246919667      0.10019793682240982   0.10002794244651761
0.7138479571874674      0.10057389011591952   0.10020995735978276
0.7140143734635045      0.10095323714747745   0.10072851218149957
0.7141817735200777      0.10133599352685062   0.10151512936411267
0.7143543247184143      0.10173174738358969   0.10236939909041891
0.7145195249748335      0.10211190943239763   0.10298283852412798
0.714685709011789       0.10249550013505888   0.10326896347158901
0.7148528768292808      0.10288250403707384   0.10327840079857294
0.7150210284273091      0.10327293568706766   0.10322046879415828
0.7151943311671004      0.1036765289013186    0.10335820610076547
0.7153602829649743      0.10406414001495844   0.10382712123860094
0.7155272185433847      0.10445516557241753   0.10459131335058079
0.7156951379023313      0.10484961971722541   0.10545046374133407
0.7158557063193605      0.10522785529480269   0.10611937119761923
0.716021425878153       0.1056192952655337    0.10649475824507834
0.7161881292174819      0.10601415044186778   0.10656225516281953
0.7163558163373469      0.10641243457125363   0.10650167486541724
0.7165286545989753      0.10682410079283042   0.10657778033875188
0.7166941419186863      0.10721934425062607   0.10696867212039947
0.7168606130189334      0.10761800333988511   0.10768384924540393
0.7170280678997171      0.10802009140615058   0.1085563822176634
0.7171881718385833      0.10840552977487873   0.10929583675435964
0.7173534269192127      0.10880439026743276   0.10976996620972732
0.7175196657803784      0.10920666635544511   0.1099137969830039
0.7176868884220806      0.1096123709915397    0.10986774704358122
0.717855094844319       0.11002151703969931   0.10988490000259249
0.7180159503246399      0.11041376674195115   0.11016823918382611
0.7181819569467243      0.11081958106613535   0.11079899571864878
0.7183489473493447      0.11122882313056133   0.11165918743094395
0.7185169215325016      0.11164150540363954   0.11250425500665462
0.7186900468574218      0.1120679173343644    0.11310294882900546
0.7188558212404246      0.11247723997980935   0.11332565052863942
0.7190225794039635      0.11288998918908383   0.11330917794874337
0.719190321348039       0.11330617702918898   0.11329198252389007
0.719350712350197       0.11370506246104894   0.1135022858030634
0.7195162544941182      0.11411771295412426   0.1140600819576639
0.7196827804185757      0.1145337884070514    0.11489368126959722
0.7198502901235697      0.1149533004944253    0.11577953854254835
0.7200229509703268      0.11538673436319752   0.11647265888037124
0.7201882608751664      0.11580268078350302   0.11678905163698874
0.7203545545605425      0.11622205017702783   0.11682125196279562
0.7205218320264548      0.11664485382014037   0.11678247566977515
0.7206900932729035      0.11707112703858522   0.11693401799037989
0.7208635056611155      0.11751154686042684   0.11745588812789551
0.7210295671074101      0.11793424260309496   0.11826424611462488
0.7211966123342408      0.11836036937251168   0.11917108898927975
0.721364641341608       0.1187899374222407    0.11991323664904772
0.7215253194070579      0.1192015809792906    0.12030378954561348
0.7216911486142707      0.11962730557948187   0.12039387012378838
0.7218579616020201      0.12005645601517284   0.12035205530784941
0.7220257583703058      0.12048904211528817   0.12044249116486304
0.7221987062803547      0.12093585112125935   0.12087960909747411
0.7223643032484861      0.12136456000505158   0.12163589221891156
0.722530883997154       0.12179668904617026   0.12255319388689587
0.7226984485263581      0.12223224764255398   0.12336864362223171
0.7228586621136448      0.12264951593652226   0.12385496936669901
0.7230240268426948      0.12308103228968756   0.12402105342796313
0.723190375352281       0.12351596275604788   0.12399337279659416
0.7233577076424036      0.12395431631235107   0.12402909321695567
0.7235301910742895      0.12440705085938214   0.12437223004135102
0.723695323564258       0.12484132796297205   0.1250567239020573
0.7238614398347627      0.12527901263338634   0.1259641636036374
0.7240285398858038      0.12572011341985104   0.1268429732847257
0.7241966237173814      0.12616463874251083   0.12745222384517027
0.724369858690722       0.1266236491173498    0.12770052704186016
0.7245357427221453      0.1270639960124745    0.1276952608266776
0.724702610534105       0.12750775158171732   0.12770291267309272
0.7248704621266009      0.12795492381478632   0.1279712824888129
0.7250309627771794      0.1283832576041356    0.12857077713211618
0.7251966145695212      0.12882609650194665   0.12945215502644866
0.7253632501423992      0.1292723363035007    0.13037238353549363
0.7255308694958136      0.12972198457788148   0.13107308104910925
0.7257036399909913      0.13018625695212416   0.13141782432863286
0.7258690595442516      0.13063153613753775   0.13145708326326327
0.7260354628780481      0.13108020791835498   0.13143877318836383
0.726202849992381       0.131532279436693     0.13162542276474581
0.7263628861647966      0.13196519294463388   0.13213831682793534
0.7265280734789752      0.13241274763927333   0.13297233645543832
0.7266942445736903      0.13286368631519044   0.13391748366721043
0.7268613994489417      0.13331801569731735   0.1347070828887135
0.7270295381047294      0.13377579935132483   0.13515574713170933
0.7271903258185997      0.13421427243102968   0.13526757750417845
0.7273562646742333      0.13466747614666366   0.13524193578111174
0.7275231873104031      0.13512406061381252   0.1353385117705514
0.7276910937271093      0.13558403159926696   0.13576499218577454
0.7278641512855788      0.1360588369653181    0.13657359175389133
0.7280298579021307      0.13651415539544226   0.13752339281697704
0.728196548299219       0.13697284296230924   0.13838135960150902
0.7283642224768437      0.1374349049810045    0.13892877052831862
0.7285245457125511      0.13787732980934334   0.13911477854735868
0.7286900200900215      0.13833459897484943   0.1391055307455357
0.7288564782480283      0.13879522539800157   0.13914921021418275
0.7290239201865716      0.13925921395258517   0.13948294200596184
0.729196513266878       0.13973814122445388   0.14021131524738203
0.7293617554052669      0.14019729613380336   0.14114557050609502
0.7295279813241924      0.14065979577829218   0.14206133067588406
0.729695191023654       0.14112564458368013   0.14271267167989982
0.729863384503652       0.14159484681084933   0.14299654467583753
0.7300367291254131      0.14207905486876243   0.14301606483477927
0.7302027228052569      0.14254332772934694   0.14303313383426478
0.7303697002656371      0.1430109362513226    0.14330509898934288
0.7305376615065535      0.1434818842444494    0.1439506658104164
0.7306982718055526      0.14393276305371247   0.14483389492841126
0.730864033246315       0.14439865220205644   0.14578553180010237
0.7310307784676136      0.14486786328781884   0.1465242581990369
0.7311985074694485      0.14534039968006293   0.1469019196909246
0.7313713876130468      0.1458280258889326    0.14697082342213827
0.7315369168147274      0.14629546099492602   0.1469639295359666
0.7317034297969445      0.14676620363299805   0.1471549414260868
0.731870926559698       0.1472402567247767    0.14770777704131258
0.7320310723805341      0.1476939953657284    0.1485414150307096
0.7321963693431333      0.1481628252212729    0.14951239414725756
0.7323626500862689      0.1486349480031633    0.15033518316431713
0.7325299146099409      0.14911036619598653   0.1508188239160221
0.7326981629141491      0.1495890821078315    0.15095867728348014
0.73285906027644        0.15004734681607745   0.1509462477132778
0.733025108780494       0.15052075271441204   0.15104891351837474
0.7331921410650845      0.1509974396240282    0.15147815148730034
0.7333601571302112      0.1514774651466276    0.15227155608154158
0.7335333243371012      0.15197269798166593   0.15328874803365555
0.7336991406020736      0.15244736787544716   0.15417473022694916
0.7338659406475825      0.15292530091788908   0.15475391747993028
0.7340337244736278      0.15340649826942374   0.15497062765198355
0.7341941573577557      0.1538670249825608    0.1549743486543899
0.7343597413836467      0.1543427536936083    0.15502582104026497
0.7345263091900741      0.15482172811888886   0.15536241949970556
0.7346938607770378      0.1553039489662731    0.15607364542463936
0.7350319152925742      0.15627813178056815   0.15800655967062752
0.73519825085992        0.1567580649753188    0.15868556048013174
0.7353655702078022      0.1572412259275568    0.15899663172796408
0.7355255386137671      0.15770351867193985   0.15903691890601387
0.735690658161495       0.15818105918945397   0.1590507627874447
0.7358567614897594      0.15866180887756923   0.15929281551835073
0.7360238485985602      0.15914576753584223   0.15990445663097894
0.7361919194878972      0.15963293476461526   0.16082654379487962
0.7363526394353168      0.16009912524445225   0.1617785660563183
0.7365185105244996      0.16058058973542075   0.16257116555209733
0.7366853653942188      0.1610652438497469    0.16301122124942652
0.7368532040444743      0.16155308673582067   0.16312394233173863
0.737026193836493       0.16205624247223482   0.16312230208872294
0.7371918326865943      0.16253833438906906   0.1632887407768225
0.7373584553172319      0.1630235957726701    0.16380553913581236
0.737526061728406       0.16351202531247938   0.1646670707697653
0.7376863171976625      0.1639793111156524    0.16562704315015156
0.7378517238086824      0.16446189561898988   0.1664935258374172
0.7380181142002384      0.16494762934014037   0.16703640991894625
0.7381854883723309      0.1654365105203918    0.16722509758576556
0.7383580136861865      0.1659407197195118    0.16722834064518644
0.7385231880581248      0.16642370717429597   0.1673260425125515
0.7386893462105995      0.16690982277126534   0.16773863651788085
0.7388564881436104      0.16739906429657564   0.16851807891145681
0.7390246138571577      0.1678914293251185    0.16951473971918532
0.7391978907124681      0.16839912844105998   0.1704740964713973
0.7393638166258613      0.16888551911406574   0.17108876442691906
0.7395307263197907      0.1693750238390704    0.17133998625339014
0.7396986197942563      0.16986764201581328   0.17136107459760616
0.7398591623268047      0.170338887705036     0.17141220146481415
0.7400248560011162      0.1708254481060511    0.17173415931034017
0.7401915334559642      0.17131508993772424   0.17242826852045745
0.740532007068496       0.17231585769563745   0.17439759837365373
0.740697468503726       0.17280246856191075   0.1751058420489794
0.7408639137194923      0.17329213742035507   0.17545020535483158
0.7410313427157952      0.17378486037614055   0.17551195446374226
0.7411914207701805      0.17425609403786302   0.17552993625496718
0.7413566499663291      0.17474263227293527   0.17576068987481408
0.741522862943014       0.17523220503675005   0.17635386448898868
0.7416900597002352      0.17572480798444526   0.17726464217623214
0.7418582402379927      0.1762204365443063    0.17827137778542002
0.7420190698338328      0.17669451472146544   0.17906139600348478
0.7421850505714362      0.17718388608072308   0.17953158864079105
0.7423520150895759      0.17767626311433377   0.17967008434960108
0.742519963388252       0.1781716407973196    0.17967552675245413
0.7426930628286912      0.178682309292922     0.17983629253218056
0.7428588113272131      0.17917137684513343   0.18033510263024352
0.7430255436062713      0.17966342464265217   0.1811805274784999
0.7431932596658657      0.18015844720013374   0.18218867516198978
0.7433536247835428      0.18063183377509065   0.18304373525981868
0.7435191410429831      0.1811204818788055    0.1836116314457192
0.7436856410829599      0.18161208481558605   0.1838261724682778
0.7438531249034729      0.1821066366506791    0.18384220251614947
0.7440257598657491      0.1826164380914707    0.1839370509325345
0.7441910438861079      0.18310456208875597   0.18433368921468532
0.744357311687003       0.18359561457020734   0.18509392902229532
0.7445245632684345      0.18408958914458134   0.1860817414028947
0.7446927986304023      0.1845864791815504    0.18703203230489612
0.7448661851341334      0.18509858633008372   0.18770703319330412
0.745032220695947       0.185588977905153     0.18798200778640187
0.745199240038297       0.18608226413635556   0.18801813859141728
0.7453672431611833      0.1865784379336127    0.18807346599178032
0.7455278953421521      0.18705287846645544   0.18837859398859846
0.7456936986648843      0.187542501019573     0.18905243805156519
0.7458604857681528      0.18803497673672168   0.19000266452250328
0.7460282566519576      0.18853030927855516   0.19098651948201534
0.7462011786775256      0.18904079564627377   0.19175155118797618
0.7463667497611762      0.18952952237114568   0.19211638535285833
0.7465333046253632      0.19002108819951904   0.19219510777056065
0.7467008432700863      0.19051548521418438   0.19221924447620992
0.7468610309728922      0.19098811591194684   0.19243866809330798
0.7470263698174613      0.19147586255846383   0.1930122146130553
0.7471926924425668      0.19196642032947317   0.19390332378377212
0.7473599988482086      0.19245978086757384   0.19490287624892985
0.7475324563956136      0.1929682204011458    0.19575383666191615
0.7476975630011011      0.19345487556844404   0.19622067790874398
0.747863653387125       0.19394431289784833   0.19636375034776585
0.7480307275536853      0.19443652358476723   0.19637626577746134
0.7481987855007818      0.19493149857374897   0.19653087710526
0.7483719945896417      0.19544149715542086   0.19704606615406234
0.748537852736584       0.19592970396065262   0.19788299986239008
0.7487046946640628      0.19642065407363282   0.19887812432280713
0.7488725203720779      0.19691433798849647   0.19976128111248587
0.7490329951381755      0.1973862406285638    0.20030123964929825
0.7491986210460363      0.1978731232350819    0.20051598024588072
0.7493652307344334      0.19836271922450222   0.20053829831842904
0.749532824203367       0.1988550186513958    0.2006316884051263
0.7497055688140637      0.19936224536021122   0.20104289236829842
0.7498709624828431      0.1998476867389881    0.20179595058614994
0.7500373399321588      0.20033581055293706   0.20276847140832793
0.7502047011620108      0.20082660641005173   0.2037036812912642
0.7503647114499454      0.20129563808567974   0.20433681802005435
0.7505298728796432      0.2017795503168135    0.20464026688178766
0.7506960180898775      0.20226611418536347   0.20469396966808637
0.750863147080648       0.20275531886356457   0.20474029705387908
0.7510312598519548      0.2032471532606993    0.20503355798881093
0.7511920216813444      0.203717240791011     0.2056561791932697
0.751357934652497       0.20420213756760716   0.20657308740743463
0.751524831404186       0.20468964328719147   0.20754708527233962
0.7516927119364113      0.2051797464200923    0.20831650726182468
0.7518657436103999      0.20568459042244983   0.2087350920408007
0.752031424342471       0.20616768188671242   0.20883424422144178
0.7521980888550786      0.20665330457381456   0.20885703354669533
0.7523657371482224      0.20714149126910875   0.2090665835799075
0.7525260344994488      0.2076079824639157    0.20959313215779635
0.7526914829924384      0.20808916103788025   0.2104456850369809
0.7528579152659642      0.20857288361856785   0.21142448298360103
0.7530253313200266      0.20905913807208182   0.21226564621172928
0.7531978985158521      0.20956000325007862   0.2127848645516902
0.7533631147697603      0.21003919274045574   0.21294918150597833
0.7535293148042047      0.21052089347790334   0.21296733575176272
0.7536964986191854      0.21100509290353434   0.21310050889326076
0.7538646662147026      0.2114917781927493    0.2135538988063439
0.7540379849519829      0.21199298278151615   0.21438199862569546
0.7542039527473459      0.21247255371461454   0.21534870946420065
0.754370904323245       0.21295458950734864   0.2162288048695563
0.7545388396796806      0.21343907690702738   0.2168079178602547
0.7546994240941989      0.21390198627251383   0.2170322247195032
0.7548651596504803      0.21437935821316853   0.21706478597723428
0.7550318789872981      0.21485916141405267   0.21714450898150436
0.7551995821046521      0.21534138220482485   0.21750178472906084
0.7553724363637695      0.2158379736947753    0.2182402100313646
0.7557044267787056      0.21679046389570258   0.2200974976834684
0.7558718976569783      0.21727028977181528   0.22076530487135823
0.7560320175933334      0.2177286367171061    0.22107189035430286
0.7561972886714518      0.2182012946922925    0.22113878051092145
0.7563635435301064      0.21867631482102137   0.22117918692624555
0.7565307821692975      0.2191536825945674    0.22144032791107365
0.756699004589025       0.21963338322685247   0.22205393154105899
0.7568598760668349      0.2200916700437793    0.22290954115657283
0.7570258986864081      0.22056416172832724   0.22385694695006012
0.7571929050865176      0.22103896557985997   0.22462487437879697
0.7573608952671635      0.22151606639523833   0.22505618839228544
0.7575340365895726      0.2220072650184138    0.22518115233008837
0.7576998269700643      0.22247709668764795   0.22520393175019993
0.7578666011310923      0.22294920392764914   0.22538867476952895
0.7580343590726566      0.2234235711114554    0.22590508056582098
0.7581947660723036      0.22387665365737885   0.2266960186552808
0.7583603242137138      0.22434370218236752   0.22763956057671242
0.7585268661356603      0.22481295147705285   0.22846808217600043
0.7586943918381431      0.22528442411448613   0.22898973171218842
0.7588670686823892      0.22576981150590092   0.22918244199422624
0.7590323945847177      0.2262339757213747    0.22920532999607454
0.7591987042675827      0.2267003432408821    0.2293216261646305
0.7593659977309841      0.22716889808967541   0.22973386407100677
0.7595342749749217      0.2276396240201601    0.23048580850535666
0.7597077033606225      0.22812413964814957   0.23145554600636886
0.759873780804406       0.2285875227526445    0.2323140025257232
0.7600408420287258      0.22905305654427657   0.23289542074758537
0.7602088870335819      0.2295207243824486    0.23314018136883957
0.7603695810965208      0.2299673591806077    0.2331771626336301
0.7605354263012227      0.23042771541485207   0.23324674796678907
0.7607022552864611      0.23089018602144237   0.23356920212335092
0.7608700680522356      0.23135475397650146   0.23423391806733201
0.7612086449253939      0.23229011243410835   0.2360544373096337
0.7613752416715506      0.23274938714265003   0.23671434713476724
0.7615428221982437      0.23321072138847151   0.23703950651107966
0.7617030517830194      0.23365120148159435   0.23710968073420557
0.7618684325095584      0.23410520335623558   0.2371461027720656
0.7620347970166337      0.2345612451097864    0.23737997516910525
0.7622021453042452      0.23501930894893858   0.2379448043949189
0.76237464473362        0.23549075778296957   0.23881689634236863
0.7625397932210773      0.23594143028922696   0.23972025246348602
0.7627059254890711      0.23639410448058257   0.24045451706650658
0.7628730415376013      0.23684876217759318   0.24087109023028197
0.7630411413666676      0.237305384915191     0.24099568953306205
0.7632143923374972      0.2377752469271944    0.24102162115285627
0.7633802923664095      0.23822445020482677   0.24119870775586294
0.763547176175858       0.23867559776565422   0.24168804938980415
0.7637150437658429      0.23912867075558933   0.24248099373821602
0.7638755604139105      0.23956120938141812   0.24335520719347448
0.7640412282037411      0.2400069111055062    0.24414224289728853
0.7642078797741081      0.24045451827729267   0.24464029841437884
0.7643755151250116      0.24090401166338402   0.2448264018298166
0.7645483016176781      0.24136649326592813   0.24485340232573516
0.7647137371684272      0.24180840724762775   0.24496644826815175
0.7648801564997127      0.24225218653159758   0.2453577341716183
0.7650475596115347      0.24269781214633873   0.24606833407984047
0.7652076117814391      0.2431231358424697    0.2469182573396265
0.7653728150931068      0.24356139078254657   0.24774680202858518
0.7655390021853108      0.24400147395496902   0.24832888260709216
0.7657061730580512      0.2444433660537933    0.24859195297275447
0.7658743277113278      0.24488704749940413   0.2486403424838768
0.766035131422687       0.2453105672149378    0.24869551323305947
0.7662010862758095      0.24574686335646764   0.24897252842417958
0.7663680249094684      0.24618493044116435   0.24956980434660453
0.7665359473236635      0.24662474855240624   0.25041169629756105
0.7667090208796219      0.24707718090099648   0.25130427218124074
0.7668747434936629      0.24750955679424533   0.25195320467312315
0.7670414498882402      0.2479436645470488    0.2522907136807042
0.7672091400633538      0.24837948390015027   0.25237520623105475
0.7673694792965502      0.2487953959951795    0.2524047929075639
0.7675349696715096      0.24922384202970566   0.2526025981251843
0.7677014438270053      0.24965398127736124   0.2531046014548579
0.7678689017630376      0.25008579314493856   0.2538835378949345
0.7680415108408328      0.25052996963894825   0.2547786077486649
0.7682067689767107      0.2509543509560542    0.25548919025322475
0.768373010893125       0.25138038572590415   0.2559094919748137
0.7685402365900758      0.2518080530157254    0.2560478894552404
0.7687084460675626      0.25223733160938083   0.25607084085252985
0.7688818066868128      0.25267880092574035   0.2562213906280082
0.7690478163641457      0.25310063640987346   0.25665294704964353
0.7692148098220146      0.25352406370715963   0.2573745779750209
0.7693827870604201      0.25394906125931455   0.25823304604378117
0.7695434133569081      0.2543545882283056    0.2589637218253858
0.7697091907951593      0.2547722209218977    0.25945559916516575
0.7698759520139469      0.2551914051922812    0.2596525792353171
0.7700436970132709      0.2556121191477222    0.25968298041152427
0.770216593154358       0.25604475664762916   0.25977707182355164
0.7703821383535276      0.25645804709586073   0.26011843861195166
0.7705486673332338      0.25687284773888014   0.2607589238284612
0.7707161800934761      0.25728913634588046   0.2615862108255409
0.770876341911801       0.25768615042507825   0.2623469835208278
0.7710416548718892      0.25809491458125794   0.2629112933501543
0.7712079516125139      0.25850514813917186   0.26317952822706353
0.7713752321336746      0.2589168293540267    0.2632352863482184
0.7715434964353719      0.25932993621631606   0.26328323203687737
0.7717044097951518      0.2597240574776779    0.26351833673444996
0.7718704742966948      0.2601298290475026    0.264050538116197
0.7720375225787741      0.2605370096789451    0.2648196796618802
0.77220555464139        0.260945577082106     0.26563650946832607
0.7723787378457688      0.2613656031684964    0.26629388640580764
0.7725445701082303      0.26176678169869616   0.26662891733642397
0.7727113861512283      0.26216932963276357   0.26672090713496677
0.7728791859747624      0.2625732243956028    0.2667479687549178
0.773039634856379       0.2629584576365329    0.26691414076554815
0.7732052348797591      0.2633550620619689    0.267357452806373
0.7733718186836755      0.26375299674698255   0.26806254024532955
0.7735393862681281      0.2641522388386931    0.26887165336128677
0.7737121049943441      0.2645626545546374    0.2695802769189818
0.7738774727786426      0.26495455247074917   0.2699882930013367
0.7740438243434773      0.2653477404281327    0.27013161479257997
0.7742111596888485      0.2657421952913338    0.27015374024694466
0.7743711440923022      0.2661183251617269    0.2702586575909061
0.7745362796375191      0.2665055392012088    0.27060860190026453
0.7747023989632724      0.26689400359423      0.2712343660987941
0.7748695020695622      0.26728369493015036   0.27201749206331144
0.7750375889563881      0.2676745895239004    0.27275023685811844
0.7751983249012966      0.2680473613664036    0.27323174665197614
0.7753642119879685      0.2684310214808899    0.2734536117833986
0.7755310828551765      0.2688158680353719    0.27349435795540294
0.7756989375029208      0.26920187706741316   0.2735514718251185
0.7758719432924286      0.2695985658164502    0.2738285037608368
0.7760375981400189      0.26997728532059484   0.27437425350917843
0.7763718591768084      0.2707381345929921    0.27586542397503255
0.776532130643554       0.27110135508685984   0.2764043046257
0.7766975532520627      0.2714751618295571    0.27669180099353613
0.7768639596411078      0.27185007236350284   0.2767637931846728
0.7770313498106893      0.27222606216933076   0.2767936505883956
0.7772038911220339      0.2726122302946988    0.27699057909793434
0.7773690814914611      0.2729807730582815    0.27744727714413436
0.7775352556414248      0.27335037790231276   0.2781315572404938
0.7777024135719246      0.2737210209768375    0.2788819070039051
0.7778705552829608      0.2740926781820131    0.27949726188732477
0.7780438481357603      0.2744744923958954    0.2798538495234997
0.7782097900466425      0.27483893732394166   0.27995510809552865
0.7783767157380608      0.2752043812884044    0.2799757235656101
0.7785446252100156      0.2755707999696663    0.2801148201074114
0.7787051837400529      0.2759200677015728    0.2804839570675463
0.7788708934118533      0.2762793999742637    0.28110374888518214
0.7790375868641902      0.276639692666176     0.2818370475157774
0.7792052640970635      0.27700092124384784   0.28248799688410575
0.7793780924716999      0.277371993475824     0.28290990336997585
0.7795435699044189      0.2777260866498913    0.2830596830025971
0.7797100311176743      0.2780811006204789    0.28308184936013264
0.779877476111466       0.2784370106364471    0.2831673374107528
0.7800375701633402      0.2787761651632647    0.2834546739078702
0.7802028153569778      0.27912506937154286   0.2839978351654009
0.7803690443311515      0.2794748553442179    0.2846981700815696
0.7805362570858616      0.2798254981183888    0.2853735711008657
0.7807044536211082      0.28017697247465384   0.2858524961450034
0.7808652992144374      0.28051192713520623   0.28606539062777553
0.7810312959495296      0.2808564174914373    0.28611057499529474
0.7811982764651583      0.2812017249311364    0.28615040305789124
0.7813662407613233      0.2815478240213916    0.28636291250643137
0.7815393561992515      0.2819032290064835    0.28684819020466834
0.7817051206952623      0.28224229412206064   0.28750384123656264
0.7818718689718096      0.28258213557608747   0.2881845966845729
0.782039601028893       0.28292272771830157   0.2887113348128457
0.7821999821440591      0.28324721329047126   0.2889799320683134
0.7823655144009884      0.2835809075318298    0.2890563184057665
0.782532030438454       0.2839153379797689    0.2890772563740723
0.7826995302564559      0.28425047877260234   0.2892236949824578
0.7828721812162212      0.28459459390333647   0.2896235321388366
0.783037481234069       0.2849227866346271    0.29022090308868675
0.783203765032453       0.28525167440437293   0.29089328444741763
0.7833710326113735      0.28558116109936943   0.2914615065532221
0.7835392839708304      0.28591112368598826   0.29180106096299996
0.7837126864720503      0.2862498259976312    0.2919146351394914
0.7838787380313529      0.28657287098627565   0.29193085391070805
0.7840457733711919      0.2868965450384071    0.2920360812346135
0.7842137924915671      0.2872208226686195    0.2923610372490396
0.7843744606700249      0.2875296883472173    0.29288503714571185
0.7845402799902461      0.2878471985070332    0.2935340711673581
0.7847070830910035      0.2881653005450443    0.2941258274502997
0.7848748699722972      0.28848396883060573   0.29451764751116133
0.7850478079953542      0.28881104267239516   0.29467766932349343
0.7852133950764937      0.28912290046565847   0.2946993029616588
0.7853799659381695      0.2894353120380514    0.29476127011062275
0.7855475205803818      0.28974825161089496   0.2950106819864987
0.7857077242806767      0.29004622408420283   0.2954649842360689
0.786039417745329       0.2906593039308624    0.2966821310641052
0.7862067401484598      0.2909665924064672    0.2971247080500986
0.7863792136933537      0.29128194815608965   0.2973401048684313
0.7865443362963302      0.29158253568555415   0.29738110961503267
0.7867104426798432      0.2918836011482726    0.2974115740506641
0.7868775328438924      0.2921851184753343    0.2975883174653267
0.7870456067884779      0.2924870613643151    0.2979904493089622
0.7872188318748266      0.29279683880910784   0.2985880667867337
0.7873847060192579      0.2930921174423476    0.29918694857033723
0.7875515639442257      0.2933878090037322    0.2996556403602549
0.7877194056497296      0.293683887043045     0.29990641654326927
0.7878798964133162      0.2939657224921396    0.2999711667850483
0.7880455383186662      0.29425529380362525   0.2999865594573645
0.7882121640045523      0.2945452397599612    0.3001064815469219
0.7883797734709748      0.2948355337674587    0.30043406804476014
0.7885525340791606      0.295133314963689     0.3009735152132934
0.7887179437454288      0.295417058376652     0.30155969497352614
0.7888843371922334      0.2957011372149242    0.3020594325546096
0.7890517144195744      0.29598552473855205   0.30236233920557215
0.7892117407049981      0.29625613423786207   0.30246379581393035
0.7893769181321848      0.2965341312244782    0.30247696387651907
0.789543079339908       0.2968124250797411    0.3025481241089409
0.7897102243281675      0.29709081183191527   0.3027996946828098
0.7898783530969633      0.29736937586686824   0.30325662004691767
0.7900391309238417      0.2976344493871943    0.30380091215162336
0.7902050598924832      0.29790666990453585   0.30432751123512564
0.7903719726416613      0.29817912370840016   0.3046927184618606
0.7905398691713755      0.29845178464079025   0.30485301876659526
0.790712916842853       0.29873134111620825   0.3048788333077218
0.7908786135724131      0.29899762224096604   0.30491748311617195
0.7910452940825095      0.2992641008440175    0.305105008335682
0.7912129583731423      0.29953075069025503   0.30549458963511406
0.7915387362123363      0.3000448356080058    0.3065288738829705
0.7917051844833513      0.3003054399447677    0.30693153631048453
0.7918726165349025      0.30056618030955606   0.3071381152446466
0.792045199728217       0.3008334666788233    0.30718638081212446
0.7922104319796139      0.3010879628615253    0.30720320970032167
0.7923766480115473      0.3013425854317834    0.307330859171451
0.7925438478240171      0.30159730800387136   0.3076480942846662
0.7927120314170232      0.3018521039870116    0.30813017335610693
0.7928853661517925      0.30211320622629856   0.30867203178018293
0.7930513499446443      0.30236180879288876   0.30909055948318725
0.7932183175180325      0.3026104750150212    0.3093271721898669
0.7933862688719572      0.30285917822495695   0.30939606712809614
0.7935468692839643      0.30309565487548623   0.30940407931123
0.7937126208377348      0.3033383397554029    0.30948636050855644
0.7938793561720414      0.3035810526222912    0.3097386411337175
0.7940470752868845      0.30382376673411693   0.3101651599742125
0.7942199455434906      0.30407243170857884   0.3106856225958061
0.7943854648581795      0.3043090906887196    0.31112289329187354
0.7945519679534048      0.30454574116823296   0.31140043125789135
0.7947194548291663      0.30478235632906575   0.31150337674328776
0.7948795907630104      0.30500724047542244   0.3115132143188918
0.7950448778386177      0.30523797693783566   0.3115580134326319
0.7952111486947614      0.3054686690958978    0.3117458744384918
0.7953784033314414      0.3056992900581963    0.3121093583640035
0.7957075292239566      0.3061488966419399    0.3130231779620765
0.7958735678410187      0.306373565632186     0.3133482650283916
0.7960405902386173      0.30659791790546365   0.31350039471192975
0.7962085964167521      0.3068221353573783    0.31352712946741473
0.7963817537366502      0.30705169829600376   0.31354825571556616
0.7965475601146308      0.30727006031302656   0.3136832578338384
0.7967143502731477      0.30748828072372236   0.3139863834852509
0.7968821242122011      0.30770633336566644   0.3144201835131184
0.797042547209337       0.3079134668380143    0.3148530616625909
0.797208121348236       0.30812585100898937   0.315203497525769
0.7973746792676716      0.30833806131871067   0.31539292609535047
0.7975422209676434      0.3085500715975081    0.3154415327833258
0.7977149138093784      0.3087670738724411    0.315448576294233
0.797880255709196       0.3089733865140226    0.3155355730094531
0.7980465813895499      0.3091794923513162    0.31577583664604153
0.7982138908504403      0.3093853652067306    0.3161596234150716
0.7983738493694131      0.30958083042592616   0.316576547940612
0.7985389590301493      0.3097811928064028    0.3169458235670352
0.7987050524714217      0.3099813161261448    0.31717356804791885
0.7988721296932304      0.3101811742006831    0.3172523966410547
0.7990401906955755      0.31038074067127763   0.3172568250334813
0.7992009007560031      0.31057020004875685   0.3172988295716096
0.7993667619581941      0.3107643186608428    0.3174687741692892
0.7995336069409213      0.31095813953440743   0.3177882468600611
0.799701435704185       0.31115163630352477   0.3181969466370254
0.7998744156092117      0.3113495316540609    0.3185922534943781
0.8000400445723211      0.31153755119029486   0.318848842679828
0.8002066573159669      0.3117252398027913    0.3189578576371036
0.8003742538401488      0.31191257111776044   0.3189701096709822
0.8005344994224135      0.31209031075012217   0.31898767233520575
0.8006998961464413      0.3122723534203713    0.31910622851704046
0.8008662766510056      0.3124540326722378    0.31936839476186185
0.80120615636297        0.31282065377673507   0.32012979712256967
0.8013713208479163      0.3129966252349729    0.3204116320443683
0.801537469113399       0.31317220008664853   0.3205544147095015
0.801704601159418       0.3133473519436597    0.3205836382299941
0.8018727169859734      0.3135220542424662    0.32058749283081583
0.8020459839542919      0.31370055495960725   0.32067289427932694
0.8022118999806932      0.31386988295382473   0.3208893873326809
0.8023787997876307      0.3140386545143418    0.32122106109279114
0.8025466833751045      0.31420694324743      0.321586859420491
0.802707216020661       0.3143664774996892    0.3218736321693051
0.8028728998079806      0.3145297104466792    0.32204447025379407
0.8030395673758367      0.3146924571874848    0.32209432446563413
0.8032072187242291      0.31485469216216105   0.32209281586197613
0.8033800212143847      0.3150203666735618    0.32214107935024755
0.8035454727626228      0.3151775238618037    0.3223048491613952
0.8037119080913973      0.3153341652867069    0.3225889147928566
0.8038793272007081      0.3154902654435913    0.3229318764632947
0.8040393953681014      0.31563813592551665   0.3232258870365744
0.8042046146772581      0.3157893548369412    0.32342380444258584
0.8043708177669511      0.3159400291137685    0.32349966836164434
0.8045380046371804      0.3160901333061219    0.32350189794077744
0.8047061752879461      0.31623964182196823   0.3235196924738206
0.8048669949967943      0.3163812279721073    0.3236257226451251
0.8050329658474057      0.31652592745219255   0.32385205206532586
0.8053678588902375      0.31681350373155664   0.32446993108754474
0.8055409484436848      0.31695983397009264   0.3246994650208653
0.8057066870552149      0.31709847881312797   0.3247992848674725
0.8058734094472811      0.31723649508301677   0.3248116881749241
0.8060411156198837      0.31737385729790457   0.32481304466285277
0.8062014708505689      0.3175038215325546    0.32488010999998784
0.8063669772230173      0.31763654913531797   0.3250571068793875
0.8065334673760021      0.3177686193238275    0.32532712347259474
0.8067009413095232      0.3179000066710054    0.32562271302927887
0.8068735663848075      0.3180338995713993    0.32586592360874744
0.8070388405181744      0.31816063042454257   0.32599089744208937
0.8072050984320777      0.3182866744612353    0.3260199194655032
0.8073723401265172      0.3184120063094755    0.3260132705337155
0.8075405656014931      0.3185366004556211    0.3260510287891192
0.8077139422182322      0.31866346212716473   0.32619380789760744
0.807879967893054       0.3187834728797597    0.3264270309481195
0.8080469773484121      0.31890274195129703   0.3267010668370087
0.8082149705843065      0.31902124388334746   0.3269388872117422
0.8083756128782835      0.31913318207217617   0.3270778747833703
0.8085414063140237      0.319247115967657     0.327124676667703
0.8087081835303003      0.31936026485214775   0.3271182015096427
0.8088759445271131      0.31947261830821083   0.3271312713613575
0.8090488566656893      0.31958688737245733   0.32722979546741066
0.8092144178623479      0.3196948401844262    0.3274197026344338
0.809380962839543       0.31980199614000837   0.3276666023833711
0.8095484915972744      0.31990833080171666   0.32790205549020635
0.8097086694130883      0.3200086367315936    0.32805730496154123
0.8098739983706654      0.3201107721653157    0.32812475670874813
0.8100403111087788      0.320212085439003     0.32812535576121205
0.8102076076274287      0.32031255222437943   0.3281211152798953
0.810375887926615       0.3204121480824403    0.32817672883942195
0.8105368172838837      0.3205060230381805    0.3283145778815782
0.8107028977829157      0.32060149919865033   0.32852548353467437
0.810869962062484       0.320696103606636     0.3287519132524581
0.8110380101225887      0.32078981193287665   0.3289294906951952
0.8112112093244566      0.32088486875586053   0.3290209151439325
0.811377057584407       0.32097444247472984   0.3290322881077911
0.8115438896248939      0.32106311874217786   0.3290196542225868
0.8117117054459171      0.3211508733398583    0.32904529122630893
0.8118721703250228      0.32123342999864835   0.3291432267166479
0.8120377863458917      0.3213172498442832    0.3293168501983888
0.812204386147297       0.32140014720549004   0.3295234545296234
0.8123719697292386      0.32148209797303084   0.3297033060466663
0.8125447044529435      0.3215650614990571    0.3298120046958546
0.812710088234731       0.3216430627453844    0.3298382293897845
0.8128764557970548      0.3217201160423385    0.32982410995232964
0.8130438071399149      0.321796197390957     0.32982591010712964
0.8132038075408576      0.32186759905349066   0.3298859685592655
0.8133689590835635      0.3219399289385553    0.3300193214472398
0.8135350944068058      0.32201128607106005   0.33019946743462086
0.8137022135105844      0.322081646560517     0.3303749030447784
0.8138703163948993      0.3221509864080777    0.33049591077172835
0.8140310683372969      0.3222159491333652    0.33054112557044424
0.8141969714214576      0.3222816158217713    0.3305341831169017
0.8143638582861548      0.32234626110524706   0.3305188404557077
0.8145317289313881      0.32240986109395053   0.3305451434047204
0.8147047507183848      0.3224738664478267    0.33064332422940235
0.8148704215634639      0.32253361406125247   0.3307926258185778
0.8150370761890795      0.32259231501674684   0.33095508876702934
0.8152047145952315      0.3226499462612065    0.33108198234729036
0.8153650020594659      0.322703723746738     0.3311417047983585
0.8155304406654637      0.32275787114084914   0.3311448938910398
0.8156968630519976      0.3228109501604585    0.3311231301215361
0.8158642692190681      0.32286293790716986   0.33112329384044714
0.8160368265279017      0.3229150523928892    0.3311822719074969
0.8162020328948179      0.32296354758312396   0.3312967822353331
0.8163682230422704      0.3230109523604534    0.3314397280178319
0.8165353969702593      0.3230572439830885    0.3315669588083503
0.8167035546787844      0.32310239962859855   0.3316420217042819
0.8168768635290728      0.3231474624020621    0.3316557298313866
0.8170428214374439      0.3231892113010539    0.3316328003681957
0.8172097631263512      0.3232298251686764    0.33161623127131273
0.8173776885957948      0.3232692813399812    0.3316431384578466
0.8175382631233212      0.32330570155504135   0.3317214600020302
0.8177039887926107      0.3233419501643049    0.3318393623725126
0.8178706982424366      0.32337704250317884   0.33195887662970336
0.8180383914727988      0.3234109560612516    0.33204180351796087
0.8182112358449242      0.3234444586807045    0.33206877609965346
0.818376729275132       0.3234751564017039    0.3320500700600553
0.8185432064858764      0.32350467630218627   0.33202077436947763
0.8187106674771572      0.32353299602818064   0.33201841668525217
0.8188707775265204      0.3235587851004345    0.3320621761418807
0.8190360387176469      0.3235840855945617    0.33215035521033054
0.8192022836893096      0.3236081873488971    0.3322558605496409
0.8193695124415088      0.32363106816297316   0.3323422531815977
0.8195377249742444      0.3236527057589638    0.33238238805326403
0.8196985865650623      0.3236721069086232    0.3323744349203402
0.8198645992976437      0.3236908077817       0.332339966448974
0.8200315958107613      0.32370826695195604   0.33231269305017647
0.8201995761044152      0.32372446229593616   0.3323224982365054
0.8203727075398324      0.32373972185668554   0.3323809800441835
0.8205384880333322      0.3237529726330786    0.33246638995315897
0.8207052523073682      0.32376496064365146   0.3325490810904151
0.8208730003619407      0.32377566392122714   0.3325981618848323
0.8210333974745958      0.32378453924187484   0.3326005710462599
0.8211989457290141      0.3237923609783807    0.33256730888670905
0.8213654777639687      0.32379889966099273   0.3325250134281344
0.8215329935794595      0.32380413405672087   0.3325053706008039
0.8217056605367137      0.3238081223352235    0.3325289520575909
0.8218709765520504      0.323810604784        0.3325889973105492
0.8220372763479235      0.3238117857237556    0.3326620675445161
0.8222045599243328      0.3238116441143088    0.33271687812789885
0.8223728272812787      0.32381015886168324   0.3327305667455745
0.8225462457799876      0.3238072215096044    0.3326999249874021
0.8227123133367791      0.3238030727895269    0.3326487457715606
0.822879364674107       0.3237975833233742    0.33260673849168626
0.8230473997919714      0.3237907322111231    0.3325992505493244
0.8232080839679181      0.3237829360471501    0.33263066627908966
0.8233739192856282      0.32377361618482947   0.33268790673716886
0.8235407383838746      0.3237629379806624    0.3327423586961823
0.8237085412626572      0.3237508807247251    0.33276582471973654
0.8238814952832032      0.32373707455826967   0.33274407168754155
0.8240470983618318      0.32372254596400074   0.3326896109064701
0.8242136852209966      0.3237066412297818    0.33262845035960004
0.8243812558606978      0.3236893398382793    0.33259011291804275
0.8245414755584817      0.32367157779767075   0.33259109873292525
0.8247068463980286      0.3236519965925358    0.3326267688107328
0.8248732010181119      0.32363102204154953   0.3326754659316982
0.8250405394187317      0.3236086338161257    0.33270677844848223
0.8252130289611145      0.32358420519086206   0.33269694917873044
0.8253781675615799      0.32355953478137567   0.3326455484427971
0.8255442899425818      0.32353345362381464   0.33257175640944886
0.82571139610412        0.3235059415808241    0.3325058287931454
0.8258794860461944      0.32347697846631773   0.3324742360410372
0.8260527271300322      0.3234457774277       0.3324859678625301
0.8262186172719526      0.323414618042259     0.3325234931646955
0.8263854911944091      0.3233820106313215    0.33255670275907556
0.8265533488974021      0.3233479352011757    0.33255624017146684
0.8267138556584777      0.32331415755420284   0.3325122435088058
0.8268795135613164      0.32327807362526156   0.33243236420308103
0.8270461552446915      0.32324052510823764   0.33234477738889034
0.8272137807086031      0.32320147682862344   0.3322816340446412
0.8273865573142777      0.32315982129408816   0.33226263614348395
0.8275519829780349      0.32311868256313053   0.33228309350533575
0.8277183924223286      0.3230760628258092    0.33231532537894587
0.8278857856471586      0.3230319428966533    0.3323254701190587
0.8280458279300711      0.32298859241045813   0.33229224285836995
0.8282110213547469      0.3229426517588504    0.33221312707528006
0.8283771985599591      0.3228952155683677    0.33210986070554593
0.8285443595457074      0.32284626486766876   0.33201708304856925
0.8287125043119922      0.32279578065583714   0.33196460515429177
0.8288732981363596      0.3227463377876537    0.3319600031070474
0.8290392431024901      0.3226941186618827    0.33198434684581135
0.829206171849157       0.3226403708132945    0.33200397636696805
0.8293740843763604      0.32258507545643184   0.3319846075441651
0.8295471480453268      0.3225267950426993    0.3319083927518443
0.8297128607723758      0.32246976694139      0.33179586865204935
0.8298795572799613      0.32241119579645006   0.3316784631491872
0.830047237568083       0.32235106304080346   0.3315934629988426
0.8302075669142872      0.32229242856393986   0.33156171584663013
0.8303730474022548      0.3222307463611015    0.33157168246495794
0.8305395116707588      0.32216750734159016   0.3315934715834554
0.8307069597197989      0.32210269315220247   0.3315868809000147
0.8308795589106024      0.32203462695109025   0.33152214168653293
0.8310448071594885      0.3219682657252344    0.3314075059357344
0.8312110391889107      0.321900333808502     0.3312705819477691
0.8313782549988694      0.3218308130644762    0.33115336165961184
0.8315464545893645      0.32175968533219784   0.3310882436276246
0.8317198053216227      0.3216851250838554    0.33107969766277945
0.8318858051119635      0.3216125361451372    0.33109861946042096
0.8320527886828408      0.32153834484202676   0.33109999848815413
0.8322207560342543      0.32146253323169327   0.3310470867577114
0.8323813724437503      0.3213889321599632    0.3309375620243052
0.8325471399950097      0.3213118381632779    0.33078734134379334
0.8327138913268055      0.3212331287956016    0.33064196555404163
0.8328816264391374      0.3211527863277155    0.3305434761126713
0.8330545126932327      0.32106875282460623   0.33050891745366745
0.8332200480054106      0.32098713106945764   0.3305197762706132
0.8333865670981246      0.32090388085200494   0.3305294383149493
0.8335540699713752      0.3208189473552781    0.3304921436837838
0.8337142219027083      0.3207366375249264    0.3303919112062859
0.8338795249758045      0.3206505762481489    0.33023546592325875
0.8340458118294372      0.3205628742430884    0.3300657347517035
0.8342130824636063      0.3204735144850123    0.3299322894050348
0.8343813368783115      0.3203824799391231    0.3298660749222761
0.8345422403510994      0.32029434737195944   0.3298602607108433
0.8347082949656506      0.32020229402389055   0.3298742898396608
0.8348753333607379      0.3201085717584998    0.3298567512804958
0.8350433555363617      0.32001316377179784   0.32976950988820647
0.8352165288537488      0.31991364422783247   0.32960616400738824
0.8353823512292184      0.319817223382642     0.32941800224162604
0.8355491573852243      0.31971912242568246   0.32925263328749615
0.8357169473217666      0.31961932478697275   0.32915303445526184
0.8358773863163915      0.319522853414075     0.32912647147680535
0.8360429764527795      0.31942221536117965   0.3291374325683815
0.8362095503697039      0.3193198865014385    0.3291328432073702
0.8363771080671647      0.3192158504939426    0.32906321216845086
0.8365498169063886      0.3191074618695335    0.3289075083658171
0.8367151748036952      0.3190025916459504    0.3287073802130074
0.8368815164815382      0.3188960198980256    0.32851208685865224
0.8370488419399174      0.31878773051711273   0.32837545588258793
0.8372088164563791      0.3186831816132265    0.3283211477645359
0.8373739421146043      0.31857422658888124   0.3283220389418591
0.8375400515533657      0.31846355981066854   0.3283275097589373
0.8377071447726634      0.31835116539728153   0.32827865940687645
0.8378752217724975      0.3182370274638195    0.32814317100990215
0.8380359478304142      0.3181268694815795    0.327946231011828
0.838201825030094       0.31801214659493793   0.32772481199534254
0.8383686860103102      0.31789568621919984   0.32754576693197274
0.8385365307710628      0.31777747269825835   0.3274498945171907
0.8387095266735786      0.3176545151135629    0.327432339163355
0.8388751716341769      0.31753572359282056   0.3274419224674825
0.8390418003753117      0.31741518472452274   0.3274113054165619
0.8392094128969827      0.31729288308455283   0.32729403617013497
0.8393696744767363      0.31717496219529534   0.3271007545506045
0.8395350871982532      0.3170522462170826    0.3268621540449155
0.8397014837003063      0.31692777350203905   0.32664869646637745
0.8398688639828957      0.31680150270868884   0.3265148238308713
0.8400413954072485      0.3166702583288799    0.32647172348000425
0.840206575889684       0.3165435777944335    0.32647932050758177
0.8403727401526555      0.3164151313264927    0.3264661761919148
0.8405398881961635      0.3162849040893422    0.32637254579097
0.840708020020208       0.31615288125364477   0.32618000920387097
0.8408813029860154      0.3160157369541731    0.3259176302792739
0.8410472350099055      0.31588338951117123   0.3256770940006671
0.841214150814332       0.31574925282677446   0.325510856608877
0.8413820503992948      0.3156133123117702    0.3254434338922688
0.8415425990423402      0.31548237604816187   0.3254431780274995
0.8417082988271489      0.31534627049042085   0.3254420895153375
0.8418749823924938      0.31520836766794036   0.3253709686741469
0.8420426497383751      0.3150686532264903    0.32519494025439033
0.8422154682260197      0.3149236026174695    0.3249286325664174
0.8423809357717469      0.3147837321218381    0.3246615333468948
0.8425473870980102      0.3146420563775245    0.3244557129041758
0.8427148222048101      0.3144985612686047    0.3243526850202045
0.8428749063696925      0.3143604480290244    0.32433669462557413
0.843040141676338       0.31421695245207754   0.3243431884377524
0.84320636076352        0.3140716440794727    0.3242963637212156
0.8433735636312383      0.313924509028465     0.32414437731670737
0.8435417502794929      0.31377553342841874   0.32389099559253653
0.84370258598583        0.31363215758411905   0.3236105262671781
0.8438685728339306      0.3134832584362466    0.32336092123297333
0.8440355434625673      0.31333252546867535   0.3232098292674473
0.8442034978717403      0.3131799450454628    0.3231643459272779
0.8443766034226767      0.3130216815549653    0.3231713976728844
0.8445423580316955      0.3128691873639392    0.32314439235269765
0.8447090964212507      0.3127148522699441    0.323017067992807
0.8448768185913423      0.3125586628750599    0.3227754993446141
0.8450371898195166      0.3124084367833009    0.3224837897514946
0.8452027121894539      0.31225248405384837   0.32220098208996806
0.8453692183399276      0.3120946837555581    0.32200793504665487
0.8455367082709377      0.31193502272339646   0.32193056317838536
0.8457093493437109      0.3117694802152471    0.32193015966742633
0.8458746394745668      0.3116100658830752    0.321920828744142
0.846040913385959       0.3114487973800365    0.32182269423927035
0.8462081710778875      0.31128566177486217   0.3216021338140238
0.8465498051645803      0.3109496156252315    0.3209678077929417
0.846715846836891       0.31078492336398733   0.32074111604619915
0.8468828722897381      0.31061835783081315   0.32063534865656407
0.8470508815231215      0.31044990635248965   0.32062361017609414
0.8472115398145874      0.3102879793270196    0.32062504163156696
0.8473773492478167      0.310119996432319     0.32055305802474404
0.8475441424615822      0.3099501344889915    0.320356153673043
0.847711919455884       0.3097783810564086    0.320049591566447
0.8478848475919489      0.30960042408761523   0.31970196605570766
0.8480504247860966      0.3094291500703796    0.31943429617775304
0.8482169857607806      0.3092559913126182    0.3192877815211783
0.8483845305160009      0.30908093560961986   0.31925438728911976
0.8485447243293038      0.3089127434499738    0.3192614501834407
0.8487100692843701      0.3087383084506031    0.3192163490167528
0.8488763980199725      0.3085619834046539    0.31905023502830193
0.8490437105361113      0.3083837563382005    0.31875766409551376
0.8492161741940134      0.30819914364775014   0.3183940501105753
0.8493812869099979      0.30802154836763973   0.31808625573744664
0.8495473834065188      0.3078420578277203    0.3178927289933887
0.8497144636835762      0.3076606602881685    0.31782752925322877
0.8498825277411698      0.3074773440337649    0.3178324871309943
0.8500557429405267      0.30728751985013036   0.31780464872609276
0.850221607197966       0.30710490865255596   0.31766135445704763
0.8503884552359418      0.3069203856726002    0.31738151302492373
0.850556287054454       0.3067339394306722    0.31701793393867295
0.8507167679310487      0.3065548789354702    0.3166865542866427
0.8508823999494066      0.30636927383247614   0.3164482906802435
0.8510490157483008      0.30618175253054314   0.31634706947394536
0.8512166153277314      0.3059923037805696    0.3163416250921755
0.8513893660489251      0.3057961748145075    0.3163331025293466
0.8515547658282016      0.30560757908329006   0.3162236058401517
0.8517211493880144      0.30541706284811476   0.31597089583770976
0.8518885167283634      0.3052246150936916    0.3156088697964182
0.8520485331267951      0.30503986779605646   0.31525004260793593
0.8522137006669901      0.3048484052321143    0.31496468379043846
0.8523798519877213      0.3046550371663865    0.3148187573787769
0.8525469870889888      0.30445973992100805   0.31479251309494544
0.8527151059707927      0.30426249663539723   0.3147974692272017
0.8528758739106791      0.30407313345038406   0.31473003819937556
0.8530417929923289      0.3038769428723388    0.3145211142743248
0.8532086958545149      0.3036788131636105    0.31417748837529397
0.8533765824972372      0.303478733681573     0.313779815541235
0.8535496202817228      0.30327169793604936   0.3134342975391053
0.853715307124291       0.303072682974418     0.3132435039508394
0.8538819777473956      0.3028717250407537    0.31318998022013866
0.8540496321510365      0.30266881371881516   0.31319794236824094
0.8542099356127599      0.30247408247242263   0.31315823894601846
0.8543753902162465      0.30227236193484297   0.3129856798889909
0.8545418286002696      0.30206869491992      0.3126636835067452
0.854709250764829       0.30186307123283457   0.31225693639041696
0.8548818240711515      0.30165033407069325   0.31187089313277716
0.8550470464355566      0.301445913220062     0.31162964206807703
0.8552132525804981      0.3012395425094338    0.31153850369956404
0.855380442505976       0.3010312119686187    0.31154000013172894
0.8555486162119902      0.3008209116595284    0.31152166378223944
0.8557219410597676      0.3006033924918902    0.3113714813776658
0.8558879149656275      0.30039436215436355   0.3110673961045664
0.8562228141189565      0.2999704034011148    0.31025177496042233
0.8563834046439717      0.2997660782892908    0.30997303909018764
0.8565491463107502      0.29955450384057425   0.3098413891364482
0.856715871758065       0.2993409639615985    0.30982801130794035
0.8568835809859161      0.2991254491615159    0.30982681380338634
0.8570564413555304      0.29890256781059593   0.30971532459947715
0.8572219507832274      0.29868845701466623   0.3094446631395785
0.8573884439914606      0.29847237828771533   0.30904244582389206
0.8575559209802301      0.29825432236324384   0.30861124335051904
0.8577160470270824      0.29804518336140307   0.3082854898159798
0.8578813242156977      0.29782864889241156   0.3081049182339467
0.8580475851848496      0.2976101442971858    0.30806578633923876
0.8582148299345377      0.2973896605286175    0.30807440839163686
0.858383058464762       0.2971671885772409    0.30800598052293815
0.8585439360530691      0.29695379863132265   0.30778946703037346
0.8587099647831393      0.2967329554704623    0.307413638711206
0.858876977293746       0.2965101314038852    0.30696607091214034
0.859044973584889       0.2962853174453812    0.30657274899464004
0.8592181210177952      0.29605290239770093   0.30632857122177554
0.8593839175087838      0.29582968407606486   0.30625803676597174
0.8595506977803089      0.29560448244587734   0.3062667920782051
0.8597184618323703      0.29537728873213664   0.3062285006745063
0.8598788749425142      0.29515943017005736   0.30605159003899673
0.8600444391944215      0.2949339431183566    0.3057033485391222
0.8602109872268651      0.29470647064387      0.3052518037741522
0.860378519039845       0.2944770041781466    0.3048210412454905
0.860551201994588       0.2942398027013799    0.30452160589901733
0.8607165340074138      0.29401205517671064   0.3044092946249964
0.8608828498007758      0.2937823202539078    0.3044080883162882
0.8610501493746741      0.2935505895740821    0.30439585877712355
0.8612184327291088      0.2933168548144763    0.3042513694259151
0.8613918672253067      0.293075295444103     0.3039129372349035
0.8617250181144042      0.29260938713652584   0.303006320621345
0.8618930692297573      0.29237342887791085   0.302671841555123
0.8620537694031931      0.29214720576223235   0.3025221488093258
0.8622196207183921      0.29191313303598354   0.302502915349743
0.8623864558141274      0.29167706168015867   0.30250672066001194
0.862554274690399       0.29143898378923      0.3024022037425231
0.862727244708434       0.29119295633008724   0.3021040743961718
0.8628928637845514      0.2909567769789306    0.30166606764867737
0.8630594666412053      0.2907185978531784    0.30118770249768506
0.8632270532783954      0.2904784112566001    0.3008032279337008
0.8633872889736681      0.290248199067253     0.3006032105030924
0.8635526758107042      0.2900100137668344    0.3005548445368892
0.8637190464282765      0.2897698278067832    0.3005667498439023
0.8638864008263851      0.28952763369551304   0.3005021369660931
0.8640589063662569      0.2892773708500285    0.30025298346200907
0.8642240609642114      0.28903719126115357   0.2998404114338218
0.8643901993427022      0.2887950102907643    0.2993490853271981
0.8645573215017293      0.2885508206549266    0.29891650946677845
0.8647254274412928      0.288304615111969     0.2986507735558242
0.8648986845226194      0.28805030107345775   0.2985677308340684
0.8650645906620288      0.2878062261038271    0.29857917235508386
0.8652314805819744      0.2875601420843903    0.2985394577525299
0.8653993542824563      0.2873120417204385    0.29833383110752715
0.8655598770410209      0.28707427711214983   0.2979593919675171
0.8657255509413486      0.28682834372838034   0.2974659149076568
0.8658922086222127      0.2865804002122213    0.2969959379660988
0.8660598500836131      0.2863304394567554    0.2966747363834545
0.8662326426867768      0.2860722198409422    0.29654580457450075
0.866398084348023       0.28582443798406987   0.29654716637783585
0.8665645097898056      0.2855746450562559    0.29653520112820525
0.8667319190121245      0.2853228341411242    0.29637896295943583
0.866891977292526       0.2850815749470815    0.29604486750477416
0.8670571867146907      0.2848320357814843    0.295561319330814
0.8672233799173916      0.2845804848409047    0.2950606780524704
0.867390556900629       0.2843269153952646    0.2946815059758501
0.8675587176644028      0.28407132075265557   0.2944977761499093
0.8677195274862591      0.28382640155966327   0.29447325139639663
0.8678854884498787      0.28357312947519764   0.29448313346707855
0.8680524331940345      0.2833178385425976    0.2943863349357059
0.8682203617187267      0.28306052225750605   0.2940932631408417
0.8683934413851822      0.28279476896164907   0.2936080365074249
0.8685591701097203      0.2825397878157232    0.29308844615163465
0.8687258826147947      0.28228278763803266   0.29266014527744416
0.8688935789004053      0.28202376211452723   0.2924213499792847
0.8690539242440987      0.2817756161971253    0.2923654576524936
0.8692194207295552      0.28151901433808424   0.2923808474129573
0.869385900995548       0.2812603934819669    0.2923241848933821
0.8695533650420771      0.2809997475007795    0.2920828023159314
0.8697259802303696      0.2807305654004922    0.2916302475701393
0.8698912444767446      0.28047235586292374   0.2911020058298504
0.8700574925036559      0.2802121275290593    0.2906272165426792
0.8702247243111036      0.27994987445970826   0.2903265796613508
0.8703929398990876      0.27968559075859417   0.2902260040736145
0.8705663066288348      0.27941270494932174   0.29023818436767884
0.8707323224166645      0.279150908098837     0.29020584784523634
0.8708993219850307      0.2788870870908226    0.29000121275216184
0.8710673053339333      0.2786212418768161    0.289588438820772
0.8712279377409183      0.2783666458147709    0.2890743569974721
0.8713937212896666      0.2781034327020165    0.28856446411696546
0.8715604886189512      0.2778381960235485    0.288207855105
0.8717282397287722      0.2775709299592323    0.28805917563814937
0.8719011419803564      0.2772949716916722    0.28805712026446684
0.8720666932900232      0.2770302865721394    0.28805165010145434
0.8722332283802262      0.27676357767497567   0.2878982708473931
0.8724007472509656      0.2764948393493721    0.287531577755016
0.8725609151797877      0.2762374700998349    0.2870306916848759
0.872726234250373       0.27597139168065105   0.28649282999058756
0.8728925371014946      0.2757032894671909    0.28607830852613
0.8730598237331525      0.2754331579741867    0.28587120855106285
0.8732280941453467      0.2751609917535445    0.28584088732733515
0.8733890136156237      0.27490029886590256   0.28585528709555647
0.8735550842276638      0.27463083651197323   0.2857629468296352
0.8737221386202403      0.27435934518681643   0.2854637853271785
0.8738901767933529      0.27408581960899747   0.2849694781575282
0.8740633661082289      0.2738034547458221    0.2843877367041621
0.8742292044811875      0.2735326447666114    0.2839250236573755
0.8743960266346823      0.27325980627909413   0.28366154517115705
0.8745638325687135      0.2729849341709511    0.2835964029010292
0.8747242875608274      0.27272170705806803   0.2836153144656629
0.8748898936947045      0.27244962564265307   0.2835633870555219
0.8750564836091179      0.272175516362682     0.2833184179676196
0.8752240573040676      0.2718993742711609    0.2828605878786667
0.8753967821407805      0.27161431156140836   0.2822737934745996
0.875562156035576       0.27134097206996743   0.2817658216087354
0.8757285137109079      0.27106560551825404   0.28143957187574
0.8758958551667761      0.2707882071270496    0.28132704079178145
0.8760558456807269      0.2705226180842915    0.2813390038363741
0.8762209873364409      0.2702480945735822    0.2813226326014185
0.8763871127726912      0.2699715449789597    0.2811370732621733
0.876554221989478       0.26969296468519977   0.28072875522152946
0.8768830570422066      0.26914363668970315   0.27962986973373927
0.8770489502393755      0.2688659375235326    0.2792267613191106
0.8772158272170807      0.26858620896338437   0.27904326190148726
0.8773836879753221      0.2683044731894952    0.2790300574814822
0.8775566998753268      0.2680137350936932    0.2790380624355311
0.8777223608334142      0.2677349701357627    0.27890376963820207
0.8778890055720378      0.2674541769597827    0.2785462447017363
0.8780566340911977      0.2671713510153739    0.2780036546946598
0.8782169116684404      0.2669005778024759    0.27744545772229573
0.8783823403874462      0.26662074547015874   0.27698655335764605
0.8785487528869884      0.26633888537136385   0.27674193208809406
0.8787161491670668      0.2660549931001023    0.2766949657532417
0.8788886965889084      0.26576198296020936   0.276716314610913
0.8790538930688327      0.26548109459187097   0.2766331044960966
0.8792200733292933      0.2651981790465068    0.2763358016690484
0.8793872373702902      0.2649132320646576    0.2758279241547876
0.8795553851918236      0.2646262494224407    0.2752298044672142
0.8797286841551201      0.26433010093969717   0.2747051196735419
0.8798946321764991      0.26404616044572      0.27441487012522864
0.8800615639784146      0.26376018940038987   0.2743374370506769
0.8802294795608665      0.2634721837273399    0.2743602601914845
0.8803900442014008      0.26319646051563905   0.27431624099059343
0.8805557599836984      0.26291155975138886   0.27407476466192593
0.8807224595465324      0.26262462946853343   0.27360729226292224
0.8808901428899025      0.2623356657349044    0.2730095014429347
0.881062977375036       0.2620374703913984    0.27244046572928693
0.8812284609182521      0.26175162237755484   0.27208726902815544
0.8813949282420045      0.2614637460278141    0.27195996246808224
0.8815623793462932      0.26117383755634976   0.2719733143311469
0.8817224795086647      0.2608963474136999    0.27196335634743
0.8818877308127991      0.2606096149673512    0.271782356158517
0.88205396589747        0.26032085550776785   0.2713678307750821
0.8823893874084208      0.25973724101700396   0.2701919457433197
0.8825502391122468      0.25945691237934115   0.269775165767868
0.8827162419578363      0.2591672993262233    0.2695742092448637
0.8828832285839621      0.25887565722977335   0.26955722163508794
0.883051198990624       0.2585819826304855    0.2695737408203994
0.8832243205390493      0.2582789742383908    0.26944264015977193
0.8833900911455572      0.257988522291449     0.2690811027872415
0.8835568455326013      0.25769604307584637   0.2685215582774662
0.8837245837001818      0.2574015873741426    0.26790823395763913
0.883884970925845       0.25711976252708807   0.2674379220711777
0.8840505092932712      0.25682859642958356   0.26717513600747295
0.8842170314412339      0.25653540457789503   0.2671221349877494
0.884384537369733       0.25624018350893685   0.26714870519469164
0.8845571944399951      0.25593557384828264   0.26707036706575155
0.8847225005683399      0.2556436400295195    0.2667707944042429
0.8848887904772211      0.2553496813902208    0.26624924077905404
0.8850560641666387      0.2550536945937241    0.2656266819837563
0.8852243216365925      0.2547556763362831    0.2650832293046297
0.8853977302483096      0.25444823122407095   0.2647536679574441
0.8855637879181093      0.25415353239375477   0.26466957831087723
0.8857308293684452      0.25385680659788246   0.2646960107817738
0.8858988545993175      0.2535580506599824    0.2646535315231954
0.8860595288882724      0.253272101585722     0.26441385674820034
0.8862253543189905      0.2529767167851435    0.26393493838908677
0.886392163530245       0.2526793063290273    0.2633156731484518
0.8865599565220359      0.2523798671653063    0.26273172811381135
0.8867329006555899      0.2520709490041222    0.26233663716841366
0.8868984938472264      0.2517748906826371    0.26220172214309867
0.8870650708193994      0.25147680815383633   0.2622169841365009
0.8872326315721086      0.2511766984919138    0.26220999953206925
0.8873928413829004      0.25088950625390044   0.2620301037929699
0.8875582023354556      0.2505928270252464    0.26160597056797186
0.8877245470685471      0.2502941251492455    0.2610036691809868
0.8878918755821747      0.24999339782348834   0.26038698535428423
0.8880643552375658      0.24968314285224738   0.2599233885544105
0.8882294839510394      0.24938585579457903   0.25972567132506724
0.8883955964450492      0.2490865477922149    0.2597150627669102
0.8885626927195954      0.24878521616799198   0.2597342384549065
0.888730772774678       0.24848185828143693   0.2596033083316675
0.8889040039715238      0.24816894066647022   0.2592081698586424
0.8890698842264522      0.24786905334847115   0.2586246388603675
0.889236748261917       0.2475671443721641    0.257991361732138
0.889404596077918       0.2472632112231757    0.25749086713683667
0.8895650929520016      0.24697236150984397   0.2572417342410226
0.8897307409678485      0.24667194556410407   0.2571965386010251
0.8898973727642319      0.24636952471108822   0.2572272165264106
0.8900649883411513      0.2460651226364776    0.25714853355166095
0.8902377550598342      0.24575111913343792   0.2568188951939401
0.8904031708365996      0.245450242974686     0.25627279578286744
0.8905695703939012      0.24514734899061869   0.25562955085351996
0.8907369537317392      0.2448424346662051    0.2550746770599746
0.8908969861276599      0.24455069732862977   0.2547584158070533
0.8910621696653437      0.2442493517059725    0.2546645534399943
0.8912283369835639      0.24394598962728425   0.2546938260198303
0.8913954880823205      0.2436406086844004    0.25466295545235296
0.8915636229616133      0.2433332065005423    0.2544156836043956
0.8917244068989887      0.24303903550875025   0.2539420530164778
0.8918903419781274      0.24273522762520047   0.2533067941301057
0.8920572608378023      0.24242940246607358   0.25269772997920403
0.8922251634780136      0.2421215577621893    0.2522829161403333
0.8923982172599882      0.24180404240760464   0.2521234872834773
0.8925639201000455      0.24149980080764288   0.25213919616904795
0.8927306067206389      0.24119354364984597   0.25214153416112484
0.8928982771217687      0.24088526877424496   0.25195682062454233
0.8930585965809812      0.2405903134598015    0.2515379900585596
0.8932240671819567      0.24028568192319427   0.2509228456880229
0.8933905215634687      0.23997903663337183   0.25028436511949775
0.893557959725517       0.23967037553709028   0.24980457085724672
0.8937305490293285      0.23935200672307      0.2495782413197212
0.8938957873912226      0.2390469978813062    0.24956608073750033
0.8940620095336531      0.23873997722396198   0.2495922685436161
0.8942292154566198      0.23843094280612573   0.2494712327766617
0.8943974051601229      0.23811989271704648   0.24909319392608353
0.8945707460053893      0.23779911058662312   0.24847304333037348
0.8947367359087384      0.23749173805782015   0.24782056349306014
0.8949037095926236      0.23718235393497902   0.24729798185124743
0.8950716670570452      0.23687095641642059   0.24702480198468732
0.8952322735795495      0.23657301043449566   0.24698064388217245
0.8953980312438168      0.23626532813657244   0.2470165051776132
0.8955647726886206      0.2359556364884146    0.24694639682869035
0.8957324979139607      0.23564393379492424   0.24663300242373354
0.895905374281064       0.2353224668366495    0.2460548085325934
0.8960708997062499      0.23501448867214403   0.24539493452240138
0.8962374089119722      0.2347045315840403    0.24482019802067354
0.8964049018982307      0.2343925776569346    0.24447719085621344
0.8965650439425719      0.23409414987343766   0.24438548967568735
0.8967303371286763      0.2337859550792099    0.24441847457510868
0.896896614095317       0.23347575539451326   0.24439480810427716
0.897063874842494       0.2331635491584903    0.2441526268889432
0.8972321193702074      0.23284933474116778   0.24364661709948968
0.8973930129560035      0.2325486890402878    0.24301226436130868
0.8975590576835626      0.2322382552858045    0.242385519657231
0.8977260861916582      0.23192581696952969   0.24195456980771154
0.8978940984802901      0.23161137255678144   0.2417869652792712
0.8980672619106851      0.23128711489534023   0.24180338046584723
0.8982330743991628      0.2309764594640538    0.24181171870380128
0.8983998706681768      0.23066380158712188   0.24163150211012233
0.8985676507177272      0.23034913982659055   0.2411832738326282
0.8987280798253601      0.23004811573235937   0.24056927436127595
0.8988936600747564      0.2297372751744271    0.23991500962450474
0.8990602241046888      0.22942443435144402   0.23942061772256581
0.8992277719151576      0.2291095919199664    0.2391875903637433
0.8994004708673897      0.22878490983477476   0.23917334295128745
0.8995658188777045      0.22847389702305437   0.23920468974183776
0.8997321506685554      0.22816088625695585   0.23908749044973357
0.8998994662399428      0.22784587628897907   0.23870816222172334
0.9000594308694128      0.22754456875074314   0.238129308021619
0.9002245466406458      0.22723341901093627   0.2374591626127804
0.9003906461924153      0.22692027368675174   0.23690292384299633
0.9005577295247211      0.22660513162442314   0.23659465626105525
0.9007257966375632      0.22628799170397462   0.2365324823357432
0.900886512808488       0.22598459068432403   0.23657277287647802
0.901052380121176       0.22567133109916077   0.2365245735442992
0.9012192312144003      0.22535607734633165   0.23623713201443386
0.9013870660881609      0.2250388284002462    0.23568815846148106
0.9015600521036848      0.22471170084614836   0.23498419042431895
0.9017256871772912      0.22439834100138836   0.23438070602618968
0.9018923060314339      0.22408298969754925   0.23400372153512003
0.9020599086661131      0.22376564600486226   0.23388900044353236
0.9022201603588749      0.22346209974454254   0.2339222816711684
0.9023855631933997      0.22314867347836462   0.23391600699039913
0.902551949808461       0.22283327789551172   0.23369872155991794
0.9027193202040587      0.22251589288121285   0.2332096123741624
0.9028918417414193      0.22218861022948455   0.23252018386309548
0.9030570123368628      0.2218751508296833    0.23187658844031914
0.9032231667128425      0.2215597054131088    0.23142676331272163
0.9033903048693586      0.22124227315001807   0.23124585202267617
0.903558426806411       0.22092285324395622   0.23125949223541944
0.9037316998852265      0.2205935227991626    0.23127862839508606
0.9038976220221248      0.22027804748760757   0.2311086900332467
0.9040645279395594      0.21996058809281657   0.23066595697363226
0.9042324176375303      0.21964114390845274   0.23001273759806715
0.9043929563935837      0.21933558102211206   0.22936179938141796
0.9045586462914006      0.21902010689350968   0.22885077472800164
0.9047253199697536      0.21870265148836626   0.22860474853470095
0.9048929774286429      0.21838321418837675   0.2285858217504802
0.9050657860292957      0.21805385031605648   0.22862583548533846
0.9052312436880308      0.21773839160527012   0.22851741239595547
0.9053976851273022      0.21742095456214844   0.2281438250211639
0.9055651103471102      0.21710153865776674   0.22752868300843385
0.9057251846250007      0.21679605195441592   0.2268639403741266
0.9058904100446543      0.21648063859651717   0.2262933954446934
0.9060566192448444      0.21616324988908822   0.22597299623297365
0.9062238122255708      0.21584388539050597   0.22590591467407026
0.9063919889868335      0.21552254469475982   0.22595234725342384
0.9065528148061788      0.21521515987653692   0.22591056664911108
0.9067187917672872      0.21489783892105307   0.22562884381756354
0.9068857525089321      0.2145785453492969    0.22507943817597373
0.9070536970311133      0.21425727884317328   0.22438313505522486
0.9072267926950577      0.21392606384268764   0.2237328693600454
0.9073925374170846      0.21360882593960673   0.22334531885991493
0.9075592659196479      0.2132896187401987    0.22322554726289365
0.9077269782027476      0.2129684420156329    0.22326408620778737
0.9078873395439299      0.21266126318495512   0.22326245382255636
0.9080528520268754      0.21234413686383738   0.22304992206045632
0.9082193482903571      0.21202504459700303   0.22256090518866792
0.9083868283343752      0.2117039862428038    0.22188183469920844
0.9085594595201565      0.21137296989238508   0.22119259464556695
0.9087247397640205      0.21105596905736967   0.22073391645653156
0.9088910037884208      0.2107370033783229    0.22054839554456016
0.9090582515933574      0.21041607520026046   0.2205637190494568
0.9092264831788304      0.21009318452685874   0.22059173628264342
0.9093998659060665      0.20976033013384632   0.22042302100353628
0.9095658976913853      0.2094415159057673    0.2199797268222064
0.9097329132572405      0.20912074295200264   0.21932046259166205
0.9099009126036319      0.20879801136659748   0.21862736654194004
0.910061561008106       0.20848933750310086   0.21812515380500946
0.9102273605543432      0.20817070211747563   0.21787575083259525
0.9103941438811168      0.20785011180199292   0.21785827968126734
0.9105619109884268      0.20752756673895084   0.21790449365659792
0.9107348292374999      0.2071950527322384    0.21779849552095598
0.9109003965446556      0.20687661330058466   0.2174237236618863
0.9110669476323476      0.20655622289344447   0.21680321363843738
0.9112344825005761      0.20623388178273552   0.21609617626230385
0.9113946664268872      0.20592563067132114   0.2155370103670237
0.9115600014949614      0.2056074134364479    0.2152147162461342
0.9117263203435719      0.20528724919807223   0.21514951264728982
0.9118936229727188      0.20496513831562235   0.21520109363467999
0.9120660767436288      0.20463305574726778   0.21515455916756177
0.9122311795726216      0.2043150782615905    0.21485684699108118
0.9123972661821507      0.20399515790623626   0.2142909721604566
0.912564336572216       0.2036732951295077    0.21358483887621638
0.9127323907428178      0.20334949042106665   0.21294741459085778
0.9129055960551826      0.20301571299220897   0.21254773015795617
0.9130714504256302      0.20269605737861185   0.21244292820958913
0.913238288576614       0.20237446368252407   0.2124895018232908
0.9134061105081341      0.20205093248310937   0.21248113067765256
0.913566581497737       0.2017415352020865    0.21225459096446037
0.9137322036291029      0.20142216913280375   0.21174688064162298
0.9138988095410053      0.20110086932944526   0.21105574421496523
0.914066399233444       0.20077763645856392   0.21038016583277738
0.914239140067646       0.20044443227905373   0.20990717371973144
0.9144045299599304      0.20012537439553638   0.20973964066251305
0.9145709036327512      0.19980438729589628   0.2097663756327608
0.9147382610861085      0.19948147173544004   0.2097941383652406
0.9148982675975482      0.19917271246093465   0.2096376661717147
0.9150634252507512      0.19885396037246506   0.20920085847780287
0.9152295666844905      0.19853327971386142   0.2085408862214289
0.9153966918987662      0.1982106752655141    0.20783799848185727
0.9155648008935782      0.19788614804748486   0.20730055040349038
0.9157255589464727      0.19757579057269642   0.2070537477674907
0.9158914681411305      0.19725546829376864   0.20703495274634096
0.9160583611163247      0.19693322746659803   0.2070881773694484
0.9162262378720551      0.19660906920698054   0.2070012781513282
0.9163992657695488      0.19627494718792873   0.20662126157380234
0.9165649427251251      0.1959550051017302    0.20600130544471246
0.9167316034612377      0.19563314990441716   0.20528748234646663
0.9168992479778866      0.19530938280902232   0.20469258550366526
0.9170595415526182      0.1949998024487624    0.20437639912190675
0.9172249862691129      0.19468026471030148   0.2043097724862793
0.917391414766144       0.19435881929286258   0.20436642127763638
0.9175588270437115      0.19403546750439793   0.20433463955830097
0.9177313904630422      0.19370216140033666   0.20403423300242188
0.9178966029404554      0.19338305029892003   0.203469443381711
0.9180627991984049      0.19306203714991457   0.2027589084626669
0.9182299792368909      0.19273912335768897   0.2021119520628543
0.918398143055913       0.19241431037746873   0.2017071650205282
0.9185714580166986      0.1920795504514388    0.2015899048655157
0.9187374220355666      0.19175899293275217   0.20164087486743396
0.918904369834971       0.1914365406324786    0.20164138296295864
0.9190723014149118      0.19111219510294988   0.20140612276415684
0.9192328820529351      0.19080205511498102   0.2009085636370813
0.9193986138327217      0.19048197579033474   0.200214379116027
0.9195653293930446      0.1901600075318666    0.1995319493861291
0.9197330287339038      0.18983615198669326   0.19905824594593324
0.9199058792165262      0.18950236352713767   0.19887595556136925
0.9200713787572313      0.1891827858880959    0.19890681414510436
0.9202378620784727      0.1888613253707024    0.1989417642762673
0.9204053291802503      0.18853798371835961   0.19877870952644452
0.9205654453401106      0.18822885372625633   0.19834923106699867
0.9207307126417341      0.18790979909748662   0.19768678890193017
0.9208969637238941      0.18758886762660607   0.19697917484422972
0.9210641985865903      0.18726606115098815   0.1964357948723
0.9212324172298227      0.18694136581186546   0.19617884145357148
0.9213932849311379      0.1866308455233696    0.19616744127867528
0.9215593037742162      0.18631040999451193   0.19622677452634588
0.9217263063978309      0.18598810593017498   0.19614686997014535
0.9218942928019819      0.18566393556841065   0.19578562204150846
0.9220674303478962      0.18532986058094889   0.195137183942397
0.922233216951893       0.185010005222181     0.19441996205420142
0.9223999873364261      0.1846882887932325    0.19382155423337807
0.9225677415014957      0.18436471364444673   0.1934939852090919
0.9227281447246477      0.18405535570794995   0.1934378565616164
0.922893699089563       0.18373610395043918   0.19350052882396762
0.9230602372350145      0.18341499855738969   0.1934743464981238
0.9232277591610025      0.18309204198873064   0.19318948518996523
0.9234004322287537      0.18275920454489508   0.19259880379945205
0.9235657543545875      0.18244058548702768   0.19188536730663222
0.9237320602609577      0.18212012048279855   0.19123684681581626
0.923899349947864       0.1817978121034722    0.19083195982637166
0.9240676234153069      0.18147366298522213   0.19071670704047508
0.9242410480245129      0.18113965028555165   0.19077468463345884
0.9244071216918016      0.18081985340647547   0.1907797629677339
0.9245741791396266      0.18049822111538655   0.1905472759545441
0.9247422203679878      0.18017475616059864   0.19001886808819202
0.9249029106544318      0.17986550030715978   0.18933756011799907
0.9250687520826388      0.1795463931171603    0.1886549432094097
0.9252355772913823      0.17922545843721227   0.1881835128702252
0.925403386280662       0.17890269912505954   0.1880063019773578
0.925576346411705       0.17857010536308285   0.18804345527919597
0.9257419556008306      0.1782517183703002    0.188082790903931
0.9259085485704925      0.17793151207403735   0.187921099222674
0.9260761253206908      0.17760948944321933   0.18746222455741965
0.9262363511289716      0.17730166421080995   0.1868101717162384
0.9264017280790157      0.17698401754515702   0.18610288700779898
0.926568088809596       0.17666455971521425   0.1855638138908044
0.9267354333207127      0.17634329379838548   0.18531311495176836
0.9269079289735926      0.1760122256437051    0.1853130965670985
0.9270730736845553      0.17569535035421752   0.185375566194974
0.9272392021760542      0.17537667230826073   0.1852844147091567
0.9274063144480893      0.1750561946934648    0.18490647691900725
0.9275744105006609      0.17473386151820916   0.18426439626595434
0.9277476576949957      0.1744017245157272    0.1835127754419093
0.927913553947413       0.17408377445353007   0.1829317044591701
0.9280804339803667      0.1737640339855559    0.18262814625610185
0.9282482977938566      0.17344250684398715   0.18259303394140425
0.9284088106654294      0.173135153780958     0.1826618612156667
0.9285744746787652      0.17281803557396305   0.18262596916406096
0.9287411224726374      0.1724991369786074    0.18232274626538145
0.9289087540470459      0.17217846185806512   0.18173424336067504
0.9290815367632177      0.1718480462077887    0.1809837332409774
0.929246968537472       0.17153179790989115   0.1803513141228603
0.9294133840922627      0.17121377956905615   0.1799727549601302
0.9295807834275898      0.1708939951815575    0.17988118668324143
0.9297408318209993      0.17058836210949024   0.17994461786790306
0.9299060313561721      0.17027300568613243   0.17995803361922047
0.9300722146718812      0.16995588949693913   0.17973532360798172
0.9302393817681267      0.16963701766803024   0.1792154235531989
0.9304075326449086      0.1693163944097562    0.1785010955097628
0.930568332579773       0.16900990693005596   0.17783726330791533
0.9307342836564007      0.16869372553564552   0.17736519211828872
0.9309012185135646      0.16837579910473155   0.17718810880655125
0.9310691371512649      0.16805613197847138   0.17722731418201973
0.9312422069307285      0.1677268002865114    0.17727802696621284
0.9314079257682746      0.16741159343524212   0.1771241516493197
0.931574628386357       0.16709465249050437   0.17667276221802694
0.9317423147849757      0.1667759819263755    0.175988853070861
0.9319026502416772      0.16647141499991666   0.17530202305480602
0.9320681368401418      0.1661572024152713    0.1747651575348526
0.9322346072191428      0.16584126659945647   0.1745169169990989
0.93240206137868        0.16552361215621092   0.17451949749546264
0.9325746666799805      0.16519634416465417   0.1745926357019798
0.9327399210393634      0.1648831662395225    0.17450759504856758
0.9329061591792829      0.16456827629044044   0.17413563728866102
0.9330733810997387      0.16425167905291405   0.1734985773306776
0.9332415868007307      0.16393337935031688   0.17276653998680563
0.933414943643486       0.16360550223281975   0.17215584057808503
0.9335809495443239      0.1632916922890996    0.17185764149036045
0.9337479392256981      0.16297616898033854   0.17182839270287176
0.9339159126876087      0.1626588702168712    0.17190733586891754
0.9340765352076019      0.1623556171660127    0.17187485566228644
0.9342423088693583      0.1620428051357741    0.17157577363279847
0.9344090663116511      0.16172830957690595   0.1709915719889538
0.93457680753448        0.16141213603478516   0.17026279631321928
0.9347496998990723      0.1610864419745224    0.16960137174678522
0.9349152413217472      0.16077477771250553   0.16923109465510675
0.9350817665249583      0.1604614435084941    0.1691474944890953
0.9352492755087058      0.16014644506840814   0.16922237666197606
0.9354094335505361      0.15984544750937732   0.1692380823790607
0.9355747427341294      0.15953495335788812   0.16901772249241484
0.9357410356982592      0.1592228027454374    0.16850071876767994
0.9359083124429252      0.1589090015345392    0.16779059676957653
0.9360765729681275      0.15859355569505534   0.16710120873625245
0.9362374825514126      0.1582920817925586    0.16665502507373497
0.9364035432764607      0.1579811548609939    0.1664886903324401
0.9365705877820452      0.15766859121276033   0.16653681993132288
0.9367386160681661      0.15735439697548012   0.16659579123893795
0.9369117954960501      0.15703079393816724   0.16644176981537395
0.9370776239820167      0.1567211417926064    0.16599137719102222
0.9372444362485197      0.1564098672448661    0.16531115803526572
0.9374122322955591      0.1560969765828942    0.16459785730158408
0.937572677400681       0.15579800160208293   0.1640905212666762
0.9377382736475661      0.15548964385310943   0.16385624153989042
0.9379048536749877      0.15517967789577042   0.16386995196572116
0.9380724174829453      0.15486811017438226   0.16395017244527174
0.9382451324326664      0.15454720724476395   0.16386651226816817
0.93841049644047        0.15424019577287965   0.16349341943120804
0.9385768442288099      0.1539315907253434    0.1628586228704089
0.9387441757976862      0.15362139870539668   0.16213346816607008
0.9389041564246452      0.15332505913733252   0.16156691136294804
0.9390692881933673      0.15301941149526693   0.1612554491699552
0.9392354037426258      0.1527121847804488    0.16121868788350463
0.9394025030724205      0.15240338575110005   0.16130593145441077
0.9395705861827516      0.15209302127797705   0.16129188106196296
0.9397313183511653      0.15179646953385412   0.16102326402528808
0.9398972016613422      0.15149066154374807   0.16046303158894593
0.9400640687520554      0.1511832415926606    0.15974466043088092
0.9402319196233051      0.15087420935105444   0.1590895049114324
0.9404049216363178      0.1505559713976343    0.1586825519405659
0.9405705727074132      0.15025152312523912   0.15859141964045262
0.940737207559045       0.1499455338576687    0.15867097225813864
0.9409048261912132      0.14963801130707946   0.15870368639748736
0.9410650938814638      0.14934423422181994   0.1585093928701248
0.9412305127134777      0.14904128324980248   0.1580169529035404
0.941396915326028       0.14873680848700685   0.15732162352936768
0.9415643017191144      0.14843081783267204   0.15663164256428014
0.9417368392539642      0.14811571208900137   0.15614928408899803
0.9419020258468966      0.14781432111819073   0.15599060536217604
0.9420681962203653      0.14751142409120735   0.15604669860988052
0.9422353503743703      0.14720702909743946   0.15611653554143418
0.9424034883089119      0.1469011443612854    0.1559849127676232
0.9425767773852164      0.14658620830541388   0.15552947539730294
0.9427427155196035      0.1462849392437302    0.1548594130469899
0.9429096374345272      0.14598219044907956   0.1541548124105694
0.9430775431299869      0.14567797033742313   0.15363393933887243
0.9432380978835293      0.14538736539284575   0.15342000194848318
0.943403803778835       0.14508774350034476   0.15344320102665826
0.9435704934546771      0.14478665994096335   0.15353334513529251
0.9437381669110554      0.14448412331747032   0.1534676484548218
0.943910991509197       0.14417263551386766   0.1530913026274511
0.9440764651654212      0.14387472598293763   0.15246609738117176
0.9442429226021816      0.1435753733987351    0.1517515625463574
0.9444103638194784      0.14327458655390862   0.15117125548976332
0.9445704540948578      0.1429873216181957    0.15088562197794889
0.9447356955120004      0.1426911411909607    0.15086087110462634
0.9449019207096794      0.14239353614535027   0.15095815697434736
0.9450691296878948      0.14209451545915314   0.15095463437340909
0.9452373224466464      0.14179408825046821   0.15067701379419235
0.9453981642634806      0.1415071250886658    0.15013544882002372
0.9455641572220781      0.14121131657462463   0.14942838318646473
0.945731133961212       0.14091411134601162   0.14878559676805622
0.945899094480882       0.14061551870746575   0.14839541339539322
0.9460722061423155      0.14030815343782058   0.1483084784568615
0.9462379668618315      0.14001420913459078   0.14839930673114435
0.9464047113618836      0.13971877632325447   0.14844134050763255
0.9465724396424723      0.13942196517081837   0.14823958623054662
0.9467328169811435      0.13913851730869073   0.14776171138608152
0.9468983454615779      0.13884633278040628   0.1470772900366954
0.9470648577225487      0.13855279161500522   0.14640185720077722
0.9472323537640559      0.13825790404655258   0.14594206149539726
0.9474050009473262      0.13795435842364237   0.14578774915911902
0.947570297188679       0.1376641314550052    0.1458575270263965
0.9477365772105684      0.13737256984219237   0.14593647711323532
0.9479038410129939      0.13707968404177454   0.1458131356475097
0.9480720885959557      0.13678548467472348   0.14538300749210084
0.9482454873206809      0.13648271103275508   0.14469419091435015
0.9484115351034887      0.13619318855751927   0.14400568334781128
0.9485785666668327      0.13590236448011914   0.1435030850401664
0.9487465820107132      0.13561024964569213   0.14330175001214415
0.9489072464126762      0.13533131456099715   0.1433433153277586
0.9490730619564023      0.1350438490312351    0.1434433743694012
0.9492398612806648      0.1347551042625215    0.14338431665842105
0.9494076443854638      0.13446509131856849   0.14302919894316116
0.9495805786320258      0.13416663476141652   0.14238628164441167
0.9497461619366705      0.13388130600375803   0.14168811817837684
0.9499127290218516      0.1335947210365395    0.1411290530245131
0.9500802798875689      0.1333068911456617    0.14085682238309835
0.9502404798113688      0.13303211232679601   0.1408562840557967
0.9504058308769321      0.13274893526511647   0.1409655869155806
0.9505721657230317      0.13246452478812804   0.14096790690671518
0.9507394843496675      0.13217889239834318   0.14069606611916938
0.9509119541180666      0.13188495321592317   0.1401149557848124
0.9510770729445484      0.1316040087454227    0.1394211281558435
0.9512431755515662      0.131321854328764     0.13881069261680817
0.9514102619391206      0.13103850168867487   0.13846056252353994
0.9515783321072113      0.13075396271963338   0.13840588757558586
0.9517515534170652      0.13046121260505425   0.13851744099885327
0.9519174237850017      0.13018137424224022   0.13855677993038973
0.9520842779334746      0.1299003617233077    0.13834340529980002
0.9522521158624837      0.12961818716461063   0.13782911918391405
0.9524126028495754      0.12934883864794236   0.13716897284420065
0.9525782409784305      0.12907128555299177   0.1365244477633953
0.9527448628878217      0.1287925070255353    0.13610769124783745
0.9529124685777492      0.12851259047627395   0.13599283855419575
0.9530852254094402      0.12822460848769832   0.13608830865786667
0.9532506312992137      0.12794939562982682   0.136168509592546
0.9534170209695234      0.12767305843397253   0.1360327714389452
0.9535843944203696      0.1273956099903225    0.1355931458669764
0.9537444169292983      0.1271308407161054    0.13496461509955357
0.9539095905799901      0.12685805852964685   0.1342957922914036
0.9540757480112184      0.12658417825162505   0.13381135856343246
0.9542428892229831      0.12630921321912125   0.1336245494725867
0.954411014215284       0.12603317696528057   0.13368409445793406
0.9545717882656675      0.12576972694094996   0.13379383157837635
0.9547377134578143      0.1254983699679545    0.13374818479175163
0.9549046224304973      0.12522595515340557   0.13341032470072234
0.9550725151837167      0.12495249627922304   0.13280569549592686
0.9552455590786995      0.12467124106236337   0.1320944831204923
0.9554112520317648      0.12440250247865772   0.13155750807308822
0.9555779287653663      0.12413273375497727   0.13130428946372794
0.9557455892795043      0.12386194892615343   0.13132347793362942
0.9559058988517248      0.12360358055280292   0.13144398219203962
0.9560713595657084      0.12333747178203434   0.13145804237511233
0.9562378040602285      0.12307036027423279   0.1312006698627671
0.956405232335285       0.12280226031076026   0.13065557788642873
0.9565778117521045      0.1225265358138365    0.12995049870985134
0.9567430402270067      0.1222631531538882    0.12936531050560618
0.9569092524824454      0.12199879595848075   0.12903891513198923
0.9570764485184202      0.1217334787596313    0.12900317253400467
0.9572446283349314      0.12146721629254353   0.12912706265478527
0.957417959293206       0.12119344915192977   0.12918354795681675
0.9575839393095629      0.12093191552286581   0.12898208867121333
0.9577509031064562      0.12066945078277967   0.12848519324978402
0.9579188506838859      0.12040606991942887   0.12781266142609404
0.9580794473193983      0.12015481208244058   0.1272126636217366
0.9582451950966737      0.11989610896518461   0.1268241360213993
0.9584119266544856      0.11963650330624347   0.12673272248686845
0.9585796419928339      0.11937601033953585   0.12684238546793283
0.9587525084729451      0.1191081973951738    0.1269406872583102
0.9589180240111391      0.11885234392941911   0.12681444682971033
0.9590845233298695      0.1185955971400245    0.12638935966034112
0.9592520064291361      0.11833799286645985   0.12574772110967825
0.9594121385864853      0.1180923176697328    0.1251243283764255
0.9595774218855979      0.11783938213571106   0.12467213382453707
0.9597436889652466      0.11758560395646908   0.12451388641436849
0.9599109398254317      0.11733099932634879   0.12459435110340464
0.9600791744661533      0.11707558466548963   0.1247221774628223
0.9602400581649575      0.11683197943913275   0.12468257544904109
0.9604060930055247      0.116581244259827     0.12435568286338065
0.9605731116266284      0.11632971413797356   0.12377051481854567
0.9607411140282683      0.1160774057679236    0.12310300542516144
0.9609142675716715      0.1158181042455661    0.12257050755006978
0.9610800701731572      0.11557052221036689   0.12235100847544106
0.9612468565551793      0.11532217764867403   0.12239557316228976
0.9614146267177378      0.11507308753363768   0.12253772264453777
0.961575045938379       0.11483559024177667   0.12255623215570727
0.9617406163007831      0.11459116690425851   0.12230675158657033
0.9619071704437238      0.11434601308795748   0.1217781413405117
0.9620747083672008      0.11410014603774454   0.12111705102685266
0.9622473974324408      0.11384749554003727   0.12053721249605202
0.9624127355557636      0.1136063423787936    0.12024961488317837
0.9625790574596227      0.11336449170527275   0.12024473528006811
0.9627463631440181      0.11312196104064166   0.1203882765033245
0.962906317886496       0.11289079443727758   0.1204599878363592
0.9630714237707374      0.1126529134076269    0.1202936370546989
0.9632375134355149      0.11241436744751439   0.11983574451838365
0.9634045868808288      0.11217517434734321   0.1191963784111533
0.9635726441066791      0.11193535213259419   0.11858993247423572
0.9637333503906119      0.11170675512947872   0.11822849855010628
0.963899207816308       0.11147158881109354   0.118146551944078
0.9640660490225403      0.1112358086887953    0.11826971359729252
0.964233874009309       0.11099943305938677   0.1183933051829106
0.9644068501378409      0.11075664113538672   0.11830058215266469
0.9645724753244554      0.110524969662559     0.11791297236972656
0.9647390842916064      0.11029271866640027   0.11730697182311701
0.9649066770392936      0.11005990671961896   0.11668048748869522
0.9650669188450633      0.10983806393394129   0.11626238214199901
0.9652323117925965      0.10960981131681563   0.11611790265036835
0.9653986885206658      0.1093810129478161    0.11621238155415355
0.9655660490292715      0.10915168798811606   0.11636254451712767
0.9657385606796403      0.1089161770970895    0.11634465119084175
0.9659037213880918      0.10869153620737736   0.11603768147239085
0.9660698658770797      0.10846638565596618   0.11548018581658112
0.9662369941466038      0.10824074489834136   0.11484554613077329
0.9664051061966643      0.10801463364422226   0.11435477248196232
0.966578369388488       0.10778249934441544   0.11415167520051313
0.9667442816383943      0.1075610797713401    0.11422224970835045
0.966911177668837       0.10733920691375795   0.11438434860180192
0.967079057479816       0.1071169007779542    0.11441720188302176
0.9672395863488776      0.10690515353011579   0.11418663086535982
0.9674052663597024      0.10668746168144032   0.11368318856034718
0.9675719301510636      0.10646935302728616   0.11305549318000922
0.967739577722961       0.10625084786291591   0.11252296254108549
0.9679123764366218      0.10602656911710714   0.11225422622697057
0.968077824208365       0.10581273048554114   0.11228053229417544
0.9682442557606445      0.10559851255479276   0.11244636431821008
0.9684116710934606      0.1053839359140061    0.11253494464345244
0.9685717354843592      0.10517963651244822   0.1123825681525377
0.968736951017021       0.1049696440101181    0.11194636173143943
0.968903150330219       0.10475930925469945   0.11133935570423562
0.9690703334239534      0.10454865312165111   0.11077188531368393
0.9692385002982242      0.10433769674912495   0.11043746385042724
0.9693993162305775      0.1041368474204593    0.11039781623501815
0.9695652833046942      0.1039304771062999    0.1105477710421191
0.9697322341593471      0.1037238232792151    0.11069017760511353
0.9699001687945363      0.10351690736590659   0.11062191957483872
0.9700732545714887      0.10330465269724418   0.1102411624330698
0.9702389894065239      0.10310237632310242   0.10966541743190665
0.9704057080220952      0.10289985535549445   0.10907937390180926
0.9705734104182029      0.10269711151458535   0.10868967096691111
0.9707337618723932      0.10250417085844692   0.10859520730156708
0.9708992644683467      0.10230597601994372   0.10872163940571987
0.9710657508448367      0.10210757501835352   0.10889236401470742
0.9712332210018629      0.10190898985972462   0.10889485026946845
0.9714058423006524      0.10170532032581892   0.10859475964387313
0.9715711126575244      0.10151130295040185   0.10806403765225883
0.9717373667949327      0.10131711892665697   0.10746965879701274
0.9719046047128775      0.10112279070566574   0.10702562815047388
0.9720728264113586      0.10092834101502174   0.1068669698373642
0.9722461992516029      0.10072901452290713   0.10697442176373433
0.9724122211499298      0.10053916953297887   0.10715937612003905
0.9725792268287929      0.10034922131185725   0.10720758852793738
0.9727472162881925      0.10015919289030835   0.10697581868430592
0.9729078548056745      0.09997845876234134   0.10650455816631753
0.97307364446492        0.09979293735580441   0.10591576279190389
0.9732404179047016      0.09960735316436092   0.10543268228372192
0.9734081751250196      0.0994217295141137    0.10521567850181973
0.9735810834871008      0.09923151896921893   0.10528284989730433
0.9737466409072647      0.09905045855494386   0.10547592524844056
0.9739131821079648      0.09886937691818687   0.10558006309598303
0.9740807070892012      0.09868829768545252   0.10542715845117971
0.9742408811285204      0.09851617417680072   0.10501640573206499
0.9744062063096026      0.09833955537658225   0.10444522718681272
0.9745725152712214      0.09816295637844487   0.10392833511447694
0.9747398080133763      0.09798640110148099   0.10364827481864948
0.9749080845360676      0.09780991375002891   0.10365909648511498
0.9750690101168416      0.0976421783926609    0.10384017625505011
0.9752350868393787      0.09747014716148825   0.1040007466778209
0.9754021473424521      0.09729820153432375   0.10394500150146944
0.9755701916260618      0.09712636601040157   0.1035966508534385
0.9757433870514349      0.09695044662104543   0.1030290116254357
0.9759092315348905      0.09678312471838706   0.10249341844780169
0.9760760597988825      0.09661593144709314   0.10216232200502487
0.9762438718434108      0.09644889160636694   0.10212281440581927
0.9764043329460216      0.09629024215899716   0.10228822176344629
0.9765699451903957      0.09612760469227638   0.10248064836571041
0.9767365412153062      0.09596513831659616   0.10249449488624403
0.976904121020753       0.09580286812330138   0.10222078842360599
0.9770768519679629      0.09563682816243682   0.101696015362341
0.9772422319732555      0.09547901811389234   0.1011500531098224
0.9774085957590845      0.095321422772828     0.10076755812237621
0.9775759433254496      0.09516406752778984   0.10066852548961845
0.9777359399498975      0.09501473859622463   0.1008057076603501
0.9779010877161085      0.09486173875869745   0.10102006364674425
0.978067219262856       0.09470899663463837   0.10110227369477037
0.9782343345901398      0.09455653785055151   0.1009119742853784
0.9784024336979598      0.09440438832667007   0.1004583425644353
0.9785631818638625      0.09426002887127084   0.09992702368536859
0.9787290811715283      0.09411221325284296   0.09948600098907767
0.9788959642597306      0.09396472466883009   0.0993049407808905
0.9790638311284691      0.0938175893278733    0.09939870211198079
0.9792368491389709      0.09366722749161      0.09963229495300487
0.9794025162075553      0.09352448470101424   0.09976835535488845
0.979569167056676       0.0933821138171966    0.09965343430002369
0.979736801686333       0.09324014134222186   0.09926041216908409
0.9798970853740726      0.09310556096127062   0.09874583926989042
0.9800625202035754      0.09296785594872402   0.09827475414744782
0.9802289388136145      0.09283056710042312   0.09803641846744714
0.9803963412041901      0.09269372120411741   0.0980828437841865
0.9805688947365288      0.09255398645299644   0.09831302324351959
0.9807340973269502      0.0924214669251151    0.09849771605651332
0.9809002836979078      0.09228940900897062   0.09846252514904308
0.9810674538494017      0.09215783978299956   0.09814226154565689
0.9812356077814322      0.09202678662800595   0.09763063902541391
0.9814089128552257      0.09189307825926298   0.09712285465415732
0.981574866987102       0.09176633958111441   0.0968523764694302
0.9817418048995145      0.09164013592857304   0.09686750366847717
0.9819097265924632      0.09151449497543263   0.09708528523083668
0.9820702973434947      0.09139558408204754   0.09729828818814552
0.9822360192362893      0.09127412407782957   0.09733117764245282
0.9824027249096203      0.09115324479011158   0.09708136679750379
0.9825704143634876      0.09103297417792482   0.09660860760300541
0.9827432549591182      0.09091040184009436   0.09608976537337692
0.9829087446128313      0.0907943723585439    0.09577119846555295
0.9830752180470808      0.09067897050264055   0.09573334840515914
0.9832426752618666      0.09056422452166352   0.09592924956253085
0.9834027815347349      0.0904557747997619    0.09616881776565149
0.9835680389493665      0.09034513146328062   0.09626999160095172
0.9837342801445343      0.09023516199915074   0.09609955837613471
0.9839015051202387      0.09012591432684297   0.09567893437495172
0.9840697138764793      0.09001744376341775   0.09517141785495473
0.9842305716908025      0.08991500677633625   0.0948072208129791
0.984396580646889       0.08981061959338987   0.0946936701142445
0.9845635733835116      0.0897069814119299    0.09484374525138468
0.9847315499006708      0.08960412106472979   0.0951108455397109
0.9849046775595931      0.08949956847194104   0.09527951245471042
0.985070454276598       0.08940085073203667   0.09518133670205199
0.9852372147741392      0.08930292922334221   0.09481637781711899
0.9854049590522167      0.08920583304994488   0.09432191472368595
0.985565352388377       0.08911431208865438   0.09392784508692494
0.9857308968663003      0.0890212097061258    0.09376178723215893
0.9858974251247601      0.08892895009569785   0.09387197210060973
0.9860649371637561      0.08883756262547103   0.09414143905261571
0.9862376003445155      0.08874485733702565   0.09436339736513166
0.9864029125833573      0.08865752312029014   0.09434285260466774
0.9865692086027356      0.08857107943494856   0.09404524749849678
0.9867364884026502      0.08848555591783323   0.09357578624159765
0.9869047519831009      0.08840098251521374   0.09313994220314958
0.987078166705315       0.08831534988117586   0.09292868242870599
0.9872442304856117      0.08823480739997792   0.09301387677211816
0.9874112780464447      0.08815523370605813   0.09328281609135901
0.987579309387814       0.08807665901658995   0.09353415202449092
0.9877399897872661      0.08800290279246158   0.0935766090946316
0.9879058213284813      0.08792820218809731   0.09334470901988547
0.9880726366502328      0.08785451827326426   0.09290881375637969
0.9882404357525206      0.08778188152872705   0.09246062445316562
0.9884133859965717      0.0877085750354791    0.09220189717129472
0.9885789852987052      0.08763987304949657   0.09224100935375758
0.9887455683813753      0.08757223690033229   0.09249611093459402
0.9889131352445817      0.08750569733706874   0.09278185116180275
0.9890733511658706      0.08744348414354391   0.09289209086013674
0.9892387182289227      0.08738071790746203   0.09273523965245889
0.9894050690725111      0.08731906592179897   0.09234505321133649
0.989572403696636       0.08725855919666224   0.09189294694422355
0.989744889462524       0.08719777965247041   0.09158647338597657
0.9899100242864947      0.0871411071587161    0.09157063478454293
0.9900761428910015      0.08708559858542075   0.091799178926246
0.9902432452760448      0.08703128520843068   0.09210974790972865
0.9904113314416245      0.08697819862263832   0.09228991971783397
0.9905845687489673      0.08692511316919563   0.09219034441273091
0.9907504551143926      0.08687583381368538   0.09184011994381408
0.9909173252603545      0.08682780019168386   0.09139658130299222
0.9910851791868527      0.0867810441659923    0.09107202753743773
0.9912456821714334      0.08673780404624666   0.09101333010875293
0.9914113362977773      0.08669468579403526   0.09121195898270511
0.9915779742046575      0.08665286305091326   0.0915353164852446
0.9917455958920742      0.08661236794028526   0.09177260056643861
0.9919183687212539      0.08657228595591801   0.09175247794603993
0.9920837906085163      0.08653549073404286   0.09146153142535723
0.992250196276315       0.08650004207988049   0.09103375874221462
0.9924175857246502      0.08646597238216078   0.0906778644902537
0.9925776242310679      0.08643489287255902   0.09056679092642927
0.9927428138792488      0.08640434905803072   0.09072398797312291
0.992908987307966       0.08637520209134583   0.09104882063843994
0.9930761445172196      0.08634748461927448   0.09133853405383686
0.9932442855070096      0.08632122961577599   0.09140525898445763
0.993405075554882       0.08629765173268729   0.09120053036346186
0.9935710167445178      0.08627489058566928   0.09080624069620571
0.9937379417146899      0.08625361006811952   0.09041946784521318
0.9939058504653981      0.08623384341418949   0.0902374344367055
0.9940789103578698      0.08621519576405685   0.09035328488752944
0.994244619308424       0.08619898629112337   0.09067137866676876
0.9944113120395145      0.08618430989201727   0.09100002398493763
0.9945789885511414      0.08617120006581208   0.0911385107032961
0.9947393141208509      0.08616021982480106   0.09100296139098404
0.9949047908323235      0.08615048517325122   0.09064870717535183
0.9950712513243325      0.0861423352335952    0.09025434647959722
0.9952386955968779      0.08613580376253117   0.09002686508707773
0.9954112910111864      0.08613082567261153   0.09009058425404419
0.9955765354835775      0.08612773292472992   0.09039004684300138
0.995742763736505       0.08612627784772621   0.09074987383703767
0.995909975769969       0.08612649446063993   0.09096116358610373
0.996078171583969       0.08612841711784544   0.09089381706471517
0.9962515185397325      0.08613219268053937   0.09056424106915796
0.9964175145535785      0.08613751967355013   0.09017528985128623
0.9965844943479607      0.08614457219801709   0.0899260061614895
0.9967524579228795      0.08615338487282732   0.08995449403647228
0.9969130705558806      0.08616342826415009   0.09022718413195573
0.9970788343306451      0.086175455197567     0.09060670335253501
0.997245581885946       0.08618926066973168   0.09087817389832112
0.9974133132217831      0.08620487955673374   0.09088658395754504
0.9975861956993834      0.08622280087383091   0.09061392849762905
0.9977517272350664      0.08624169878498371   0.09023490756617017
0.9979182425512857      0.08626242958971403   0.08995214772446296
0.9980857416480413      0.08628502842603904   0.08992867505781924
0.9982458898028796      0.08630827683512621   0.09016857112801517
0.998411189099481       0.08633396012007899   0.0905578863347547
0.9985774721766187      0.08636152980231077   0.0908868808675908
0.998744739034293       0.08639102127447651   0.09097760020374784
0.9989129896725033      0.08642247027271424   0.09078260657867407
0.9990738893687963      0.08645422328035698   0.09043611935512533
0.9992399402068526      0.0864887174041657    0.09011681871364853
0.9994069748254453      0.08652518769171703   0.09002140310916552
0.9995749932245741      0.08656367013594014   0.09021927928397654
0.9997481627654663      0.0866052236544064    0.09062635827294521
0.9999169757370227      0.0866475845570756    0.09100549469235805
0.1739119343953373      0.9999993372651459    0.9999993323084118
0.17508278704535857     0.9999992802069243    0.9999992851383384
0.175245949777998       0.9999992718855881    0.9999992687992815
0.17640521635842432     0.9999992105681689    0.9999992180070242
0.1767378476897537      0.9999991920872375    0.999999182350709
0.1777405992910932      0.9999991337293137    0.9999991430422251
0.17790283306740654     0.9999991239012265    0.9999991255145417
0.1780723016660965      0.999999113516452     0.9999991048236014
0.1790722663984578      0.9999990497050688    0.9999990610450313
0.17923403569660812     0.9999990389630073    0.9999990435578745
0.1794030398171351      0.9999990276124628    0.9999990210650229
0.17956677667635812     0.9999990164896179    0.9999990009866289
0.18039605031929135     0.9999989582065187    0.9999989714674052
0.18056985722295923     0.9999989455688588    0.9999989533091613
0.18073839686532317     0.9999989331713527    0.9999989293501049
0.18090166924638312     0.9999989210259201    0.9999989062719927
0.18190196292400604     0.9999988437781289    0.9999988549696295
0.18223284599110384     0.9999988172917526    0.9999988040723323
0.18324493856404256     0.9999987324923524    0.9999987457461754
0.18340629820824011     0.9999987184263562    0.9999987211149972
0.18356864163297404     0.9999987041182655    0.999998692482423
0.18373821988008454     0.9999986890032178    0.9999986664785598
0.18457794733693456     0.9999986115393121    0.9999986272987845
0.18473884250296907     0.9999985961872537    0.9999986028550446
0.18490072144953995     0.9999985805718312    0.9999985720026282
0.18506983521848747     0.9999985640750804    0.9999985415519322
0.18589890556206856     0.9999984804242077    0.9999984984934714
0.18606350361116694     0.9999984632551935    0.9999984750272024
0.1862374201632554      0.9999984449074248    0.9999984401448497
0.18640606945403987     0.9999984269104774    0.9999984054881628
0.1865694514835204      0.999998409281404     0.9999983801709653
0.18723281740680575     0.9999983357034758    0.9999983552760568
0.1873969509777411      0.999998316994042     0.9999983334145588
0.1875704030516665      0.9999982970010062    0.9999982975218857
0.18773858786428793     0.9999982773959011    0.9999982588033075
0.1879015054156054      0.9999982581971733    0.9999982278610069
0.18857968287114613     0.9999981767217002    0.9999981968834427
0.1887433519639184      0.9999981565316696    0.9999981760406713
0.18891425587906732     0.9999981352135981    0.9999981395579961
0.18907572517168542     0.9999981148486945    0.9999980987538103
0.18923817824483985     0.9999980941378632    0.9999980622525472
0.19007770240233773     0.9999979834670983    0.9999980058822939
0.19024814183932365     0.9999979602340402    0.9999979698637763
0.19040914665377873     0.9999979380438824    0.9999979261046368
0.19057113524877012     0.9999979154772225    0.9999978837710186
0.1907403586661382      0.9999978916419199    0.9999978525127293
0.19140833701545285     0.9999977949004177    0.9999978197892817
0.19157831197427572     0.9999977695930039    0.9999977855840627
0.19174301967179463     0.9999977447972546    0.9999977385912313
0.19191704587230363     0.9999977183042673    0.9999976872317503
0.1920858048115086      0.9999976923213955    0.9999976496360985
0.19274359052571738     0.9999975882509258    0.9999976148235505
0.1929131010063772      0.9999975606976479    0.999997582921194
0.19307734422573308     0.9999975337081691    0.9999975346891903
0.19325090594807903     0.9999975048718133    0.9999974776171116
0.19341920040912097     0.9999974765980619    0.9999974319296897
0.19407512821067763     0.9999973634119274    0.9999973903421885
0.19424417421317441     0.9999973334570187    0.9999973621034239
0.19440795295436725     0.9999973043611526    0.9999973141135239
0.1945789665179367      0.9999972736821303    0.9999972528379757
0.19474054545897534     0.9999972443808239    0.9999972000303716
0.19490310818055034     0.9999972145888063    0.9999971646019962
0.19541128479278735     0.9999971193982675    0.9999971435717893
0.1955798663171211      0.9999970871172186    0.9999971191258079
0.19574318058015092     0.9999970555045816    0.9999970721673431
0.19591372966555737     0.9999970221301522    0.9999970070105858
0.19607484412843296     0.9999969902591734    0.9999969463450242
0.1962369423718449      0.9999969578538905    0.9999969015702382
0.19691184259576355     0.9999968191926341    0.9999968537376593
0.19707469238063033     0.9999967848119465    0.9999968095653149
0.19740959433381317     0.9999967129534284    0.9999966720588633
0.19758373018274272     0.9999966749677164    0.999996613940341
0.19824843777155543     0.9999965259651797    0.9999965619709092
0.19841082307825916     0.9999964885812419    0.9999965211092504
0.19858044320733953     0.9999964491116459    0.9999964523053383
0.19874479607511591     0.9999964104544773    0.9999963753596094
0.19891846744588243     0.9999963691594143    0.9999963058665722
0.19908687155534494     0.9999963286749676    0.9999962660056051
0.19958131712204302     0.9999962072581938    0.9999962429286016
0.19974323795058374     0.9999961666570073    0.9999962070971794
0.19991239360150106     0.999996123792817     0.9999961391361646
0.20007628199111444     0.9999960818220981    0.9999960563318813
0.20024740520310447     0.9999960375311107    0.9999959756496909
0.20040909379256366     0.9999959952395178    0.9999959239584797
0.20090631328599956     0.9999958633134501    0.9999958940499329
0.20108027172005782     0.9999958162620477    0.9999958621270784
0.20124896289281213     0.9999957701342944    0.9999957960057422
0.2014123868042625      0.9999957249709746    0.9999957086043406
0.20158304553808948     0.999995677302716     0.9999956167740437
0.20174426964938563     0.9999956317900408    0.9999955522258701
0.2022400957083324      0.9999954888472241    0.9999955109393059
0.20241358966422762     0.9999954377511956    0.9999954858217242
0.2025818163588189      0.9999953876627385    0.999995423976048
0.20291497004777015     0.9999952868662132    0.9999952314547695
0.20307572968090326     0.9999952374560269    0.9999951524273389
0.20323747309457268     0.9999951872295919    0.9999951066247926
0.20374110810247964     0.9999950274790371    0.9999950750486731
0.2039026192770676      0.9999949751550784    0.9999950234447585
0.20406511423219187     0.9999949219679376    0.999994935275875
0.20423484400969277     0.9999948658240877    0.999994823662413
0.2043993065258897      0.9999948108434207    0.9999947257875831
0.20457308754507675     0.999994752122258     0.9999946593036949
0.20507532911849463     0.9999945787451585    0.9999946237264388
0.20523637581491955     0.9999945219790858    0.9999945792952576
0.20539840629188078     0.999994464283515     0.999994493473861
0.20556767159121866     0.9999944033820877    0.9999943750840501
0.2057316696292526      0.9999943437567463    0.9999942624720187
0.20590498617027658     0.9999942800753978    0.9999941774682954
0.2064058343092054      0.9999940921355024    0.9999941305048295
0.20657058388869412     0.9999940290249568    0.9999940936928093
0.20674465197117295     0.9999939616415244    0.9999940043747297
0.2069134527923478      0.9999938959703247    0.999993879734307
0.20707698635221866     0.9999938318479509    0.9999937541393856
0.20724775473446622     0.9999937641870289    0.9999936533994546
0.20740908849418288     0.999993699598251     0.9999936043321977
0.20774095839706558     0.999993564671822     0.9999935908917548
0.20790524349839126     0.9999934968362348    0.9999935627253054
0.20807884710270705     0.9999934243912909    0.9999934802510039
0.20824718344571885     0.9999933533889099    0.9999933530053361
0.20841025252742668     0.9999932838913241    0.9999932142645103
0.2085805564315112      0.9999932105493158    0.9999930927746938
0.2087414257130648      0.999993140548893     0.9999930251332376
0.2092361872827841      0.9999929207917082    0.9999929833083177
0.2094072427283234      0.999992843219677     0.9999929124376716
0.20956886355133186     0.9999927691609358    0.9999927926088009
0.2097314681548767      0.9999926938938154    0.9999926433373512
0.20990130758079814     0.9999926144586976    0.9999924982783749
0.21006587974541563     0.9999925366811011    0.999992403569464
0.21057174996432643     0.9999922925579974    0.9999923488692904
0.21074234093170274     0.9999922084900857    0.9999922884683288
0.21090349727654817     0.9999921282511367    0.9999921724031654
0.21106563740192996     0.9999920467107288    0.9999920153427162
0.2112350123496884      0.9999919606548185    0.9999918502869554
0.21139912003614283     0.9999918764126786    0.9999917314852603
0.21157254622558738     0.9999917864563169    0.9999916713290481
0.21190359682056453     0.9999916120536043    0.999991656735784
0.2120737233097778      0.9999915210407633    0.9999916093340144
0.21223858253768707     0.9999914319368347    0.9999914978744118
0.21258167073818182     0.9999912436017449    0.9999911396552302
0.21274531394647322     0.9999911523754077    0.9999909993906332
0.2129161919771413      0.9999910561428323    0.9999909208922854
0.21324006257395206     0.9999908717576592    0.9999909011626674
0.2134097245850023      0.9999907740402734    0.9999908655243704
0.2135741193347485      0.999990678385249     0.9999907636102029
0.2137478325874848      0.9999905762577181    0.9999905860061993
0.21391627857891718     0.9999904761847357    0.999990385805889
0.21407945730904554     0.9999903782519348    0.9999902187926912
0.21424987086155056     0.9999902749269665    0.9999901129775947
0.21474201003492246     0.9999899704057503    0.9999900551407724
0.21490594030650567     0.9999898669147873    0.999989967055309
0.2150771054004655      0.9999897577416367    0.9999897954753816
0.21523883587189452     0.9999896535293166    0.9999895892690233
0.21540155012385986     0.9999895476366938    0.9999893922908686
0.2155714991982019      0.9999894359042648    0.9999892495981046
0.21573618101123992     0.9999893265212508    0.9999891902845586
0.21607891438199217     0.9999890953151106    0.999989169538909
0.21624238017541234     0.9999889833266005    0.9999890951867181
0.2164130807912092      0.999988865184103     0.9999889297845325
0.21657434678447515     0.9999887524371763    0.9999887137091201
0.21673659655827748     0.9999886378803413    0.9999884914806054
0.21690608115445642     0.9999885170026512    0.9999883150220599
0.21707029848933143     0.9999883986890822    0.9999882288578958
0.21741210290375762     0.9999881486201668    0.9999882050793074
0.21757510421901474     0.9999880275325047    0.9999881474680834
0.21774534035664855     0.9999878997913405    0.9999879944488133
0.21806792716739076     0.9999876541050025    0.999987528087237
0.21823694728540666     0.9999875234611497    0.9999873142283611
0.21840070014211863     0.9999873956182616    0.9999871937846779
0.218733240877765       0.9999871321115142    0.999987155175541
0.21889577771485907     0.999987001399299     0.9999871194275993
0.21906554937432982     0.9999868635100364    0.9999869906482926
0.2192300537724966      0.9999867285638163    0.9999867702494587
0.21940387667365346     0.999986584967722     0.9999864849018353
0.21957243231350634     0.9999864448657374    0.9999862349133728
0.21973572069205527     0.9999863077884611    0.9999860777612254
0.22022940483030662     0.9999858851057762    0.9999859904947008
0.22039871201161437     0.9999857372449406    0.9999858816894975
0.22056275193161812     0.9999855925454652    0.9999856691078822
0.22073611035461194     0.9999854380713089    0.999985368135209
0.2209042015163018      0.9999852867493381    0.9999850812189642
0.22106702541668768     0.9999851387075963    0.99998488089151
0.22123708413945026     0.9999849825370242    0.9999847844917126
0.22157598556535746     0.9999846665249433    0.9999847576553098
0.22174482826850217     0.999984506675057     0.9999846624708816
0.22190840371034287     0.9999843502631577    0.999984454658287
0.22224058961624676     0.999984027875378     0.9999838353711143
0.2224029490384696      0.9999838679608156    0.9999835914998765
0.22257254328306914     0.9999836992567352    0.9999834560593924
0.22291051575265025     0.9999833579271293    0.9999834205178386
0.22307889397763192     0.9999831852951913    0.9999833480674392
0.22324200494130958     0.999983016407165     0.9999831576689929
0.2234123507273639      0.9999828382725866    0.9999828425742726
0.2235732618908874      0.9999826683425205    0.9999825041067982
0.22373515683494719     0.999982495730666     0.9999822125818013
0.22390428660138367     0.999982313630272     0.9999820281947039
0.22423924643402526     0.9999819475691093    0.9999819672285958
0.2244009091390036      0.9999817682932663    0.9999819240529464
0.22456355562451824     0.9999815861978338    0.9999817660490071
0.22473343693240952     0.9999813941361627    0.9999814630673419
0.22489805097899684     0.9999812061956175    0.9999810921337265
0.22507198352857422     0.9999810056388665    0.9999807264450503
0.2252406488168477      0.9999808091995835    0.9999804904205724
0.22540404684381715     0.9999806170434415    0.9999803979455082
0.2257358779199785      0.999980222241256     0.9999803639555149
0.2258980599273301      0.9999800272280709    0.9999802319302572
0.2260674767570584      0.9999798215282439    0.9999799417362443
0.2262316263254827      0.9999796202668034    0.9999795539722957
0.22640509439689704     0.9999794054650063    0.999979139503285
0.22657329520700747     0.999979195088617     0.9999788434259441
0.2267362287558139      0.9999789893114893    0.999978704601783
0.2270712982368761      0.9999785599021849    0.9999786674339396
0.22724551784974528     0.999978333272223     0.9999785491128909
0.2274144702013105      0.9999781112705378    0.9999782664972419
0.22774907520420967     0.9999776650568701    0.9999774112455216
0.22791056049431674     0.9999774465558671    0.9999770707094608
0.22807302956496012     0.9999772246274731    0.9999768802540526
0.22840717008969624     0.9999767615108156    0.9999768245820663
0.22858092522440238     0.9999765170858809    0.9999767365487107
0.22874941309780455     0.9999762776884756    0.999976479060566
0.22891263370990278     0.9999760435183506    0.9999760727746833
0.22908308914437764     0.9999757965766783    0.999975580513819
0.22924410995632166     0.9999755610390282    0.9999751769529365
0.22940611454880203     0.9999753218234188    0.9999749225904718
0.22973932611721204     0.9999748226642927    0.9999748254447325
0.22991261677375519     0.9999745592295877    0.99997476917406
0.23008064016899432     0.9999743012638824    0.9999745474993782
0.23024339630292953     0.9999740489821595    0.9999741516173003
0.23041338725924132     0.9999737829405464    0.9999736273759748
0.23057811095424918     0.9999735226409225    0.999973148148814
0.23075215315224712     0.9999732449189116    0.9999728050589921
0.2309209280889411      0.9999729729335103    0.9999726716090532
0.23108443576433113     0.9999727069103995    0.9999726594192234
0.2312551782620978      0.9999724264417662    0.9999726237866463
0.23141648613733362     0.9999721589410098    0.9999724411182639
0.23157877779310576     0.9999718873091795    0.9999720617651572
0.23174830427125456     0.9999716008703549    0.9999715147192417
0.2319125634880994      0.9999713212000497    0.9999709744531611
0.2320861412079343      0.9999710240770936    0.9999705486054004
0.23225445166646524     0.9999707331362385    0.9999703510215172
0.23258777288329585     0.9999701486095616    0.9999703014661715
0.23274861628036864     0.9999698625258585    0.9999701601921112
0.23291044345797773     0.9999695720158474    0.9999698116928929
0.2330795054579635      0.9999692656229298    0.9999692558951065
0.2332433001966453      0.9999689659272779    0.9999686591169633
0.2334143297577037      0.9999686499729808    0.9999681466144685
0.23357592469623134     0.9999683485853942    0.9999678699822737
0.2339083169567358      0.999967719792844     0.9999677830540612
0.23407286323687243     0.9999674040541402    0.9999676907673548
0.23424672801999913     0.9999670671832614    0.9999673614292479
0.23441532554182185     0.9999667372978429    0.9999668061745233
0.2345786558023406      0.9999664146674729    0.9999661622536626
0.23474922088523598     0.9999660745120524    0.9999655628864155
0.23491035134560057     0.9999657501104406    0.9999652003719132
0.23507246558650147     0.9999654207004425    0.9999650606069016
0.235241814649779       0.9999650733184601    0.9999650504488956
0.23540589645175258     0.9999647335256608    0.9999649950271295
0.2355792967567162      0.9999643709662582    0.9999647161489202
0.23574742980037594     0.9999640159867231    0.9999641793112977
0.23591029558273166     0.9999636688774158    0.9999634999594965
0.23608039618746401     0.9999633029051584    0.9999628121712023
0.23624522953089244     0.9999629448810557    0.9999623400548006
0.23641938137731092     0.9999625629676381    0.9999621209397094
0.23675188328623598     0.9999618232668301    0.9999620616714231
0.2369227354324232      0.9999614377477737    0.9999618224527996
0.23708415295607954     0.9999610700935094    0.9999613251278158
0.23741619038684159     0.9999603032559076    0.9999598540489123
0.23758055925210697     0.9999599183300875    0.9999592780600296
0.2377542466203624      0.9999595077158129    0.9999589662821661
0.23808581957296138     0.9999587127008623    0.9999588927145423
0.23825620724098556     0.9999583001616033    0.9999587103176968
0.23841716028647886     0.9999579074118119    0.9999582554686033
0.23857909711250852     0.9999575086712975    0.999957545775592
0.23874826876091482     0.9999570882329011    0.9999567058260139
0.23891217314801716     0.9999566770635553    0.9999560164092649
0.2390833123574961      0.9999562436919331    0.9999555911213357
0.23924501694444425     0.9999558303706103    0.999955466151178
0.23940770531192873     0.9999554107331978    0.999955462218311
0.23957762850178982     0.9999549683252326    0.9999553520708362
0.23974228443034698     0.9999545355885219    0.9999549609184721
0.23991625886189422     0.999954073997974     0.9999542127349904
0.24008496603213747     0.9999536220606199    0.9999533130052458
0.24024840594107677     0.9999531801377528    0.9999525157813355
0.24041908067239268     0.9999527143116032    0.9999519687729803
0.2405803207811778      0.9999522701254073    0.9999517666809692
0.24074254467049924     0.9999518191639327    0.9999517491059654
0.24091200338219732     0.9999513436986387    0.9999516855050705
0.24107619483259146     0.9999508786946144    0.9999513585249001
0.24124970478597565     0.9999503826414773    0.9999506414139533
0.2414179474780559      0.9999498970364923    0.9999496984114814
0.24158092290883215     0.99994942226904      0.9999487926946896
0.2417511331619851      0.9999489218006252    0.9999481041913832
0.24191190879260716     0.9999484446972755    0.9999477967341731
0.24224266243730064     0.9999474496687808    0.9999477158606431
0.24240638940953174     0.9999469503163068    0.999947465723731
0.24257735120413945     0.9999464240395206    0.9999468176859524
0.2429013893288296      0.9999454127795955    0.999944888624361
0.24307113510381945     0.9999448757680941    0.9999440271709822
0.24323561361750537     0.9999443506043232    0.9999435571295466
0.24357794038955338     0.9999432422302802    0.9999434158403004
0.24374120288362144     0.9999427062423536    0.9999432340370284
0.2439117002000661      0.9999421413612023    0.9999426515160963
0.24407276289397997     0.9999416028774774    0.9999417344804924
0.24423480936843017     0.9999410563037578    0.9999406460246145
0.24440409066525698     0.999940480288398     0.999939635890326
0.24456810470077986     0.9999399199298991    0.9999390140908024
0.24474143723929281     0.9999393222402134    0.999938779437058
0.2449095025165018      0.9999387372692685    0.9999387692442301
0.2450723005324068      0.9999381654716771    0.9999386574873832
0.24524233337068843     0.9999375627902956    0.9999381644589718
0.2454070989476661      0.999936973394208     0.9999372535436104
0.2455811830276339      0.9999363448517418    0.9999360135400337
0.2457499998462977      0.9999357295697491    0.9999348758025433
0.24591354940365756     0.9999351280288399    0.9999341192135192
0.24608433378339403     0.9999344940943516    0.9999337872472461
0.24624568354059967     0.9999338897037799    0.9999337542779981
0.24640801707834165     0.9999332762112647    0.9999336949352358
0.2465775854384603      0.9999326295244588    0.999933289079412
0.24674188653727497     0.9999319971718746    0.9999324254666736
0.24691550613907973     0.9999313227469244    0.999931140137509
0.24708385847958048     0.9999306626380013    0.9999298621336865
0.2472469435587773      0.999930017363368     0.9999289272311361
0.24741726346035076     0.9999293372983048    0.9999284415547586
0.2477400177989723      0.9999280311164647    0.9999283319071395
0.24790912168092794     0.9999273375270109    0.9999280278679106
0.24807295830157955     0.9999266594242325    0.9999272446455105
0.24824402974460782     0.9999259448996268    0.9999259660620696
0.24840566656510527     0.9999252636524708    0.9999246100431305
0.24856828716613905     0.9999245722021413    0.999923460668353
0.24873814258954943     0.9999238434525405    0.9999227528230883
0.24890273075165587     0.9999231308798228    0.9999225432994815
0.2490766374167524      0.9999223710413291    0.9999225354681632
0.24924527682054498     0.9999216273708305    0.9999223197019418
0.24940864896303355     0.9999209004500774    0.9999216228645904
0.24957925592789879     0.9999201344811859    0.9999203567651409
0.24974042827023318     0.9999194043874301    0.9999189042648917
0.24990258439310392     0.999918663433571     0.9999175727550346
0.2500719753383513      0.9999178825113185    0.999916656588001
0.2502360990222947      0.9999171190882086    0.9999163111312774
0.2504095412092282      0.9999163050099369    0.9999162840056253
0.25057771613485774     0.9999155084284882    0.9999161564779546
0.25074062379918327     0.9999147317771155    0.9999155740160435
0.2509107662858855      0.9999139154683638    0.9999143624828575
0.2510756415112837      0.9999131173084526    0.9999128016573603
0.251249835239672       0.9999122663492799    0.9999111915852418
0.25141876170675637     0.9999114335053353    0.999910092752432
0.25158242091253674     0.9999106194116053    0.9999096177807907
0.25175331494069375     0.999909761677919     0.9999095512682613
0.25191477434632        0.9999089440516975    0.9999094867364613
0.25207721753248247     0.9999081142762704    0.9999090131247624
0.25224689554102164     0.9999072398010733    0.9999078722710111
0.2524113062882568      0.9999063848585173    0.9999062679296683
0.25258503553848205     0.9999054732449628    0.999904480112429
0.2527534975274034      0.9999045811392422    0.9999031416793116
0.2529166922550207      0.99990370922638      0.9999024669632096
0.2532481167324779      0.999901914915178     0.9999022913977491
0.2534100954404774      0.9999010263479491    0.9999019426986812
0.25357930897085357     0.9999000898515936    0.9999009154313809
0.25374325523992575     0.9998991744092603    0.9998993110451156
0.25391652001198795     0.9998981982015579    0.9998973655142169
0.25408451752274624     0.9998972430318038    0.9998957673647347
0.25424724777220054     0.9998963096346316    0.999894843193702
0.2544172128440315      0.9998953260820023    0.9998945376858436
0.2545819106545585      0.9998943645049536    0.9998945303141018
0.25475592696807553     0.9998933393552184    0.9998942597597237
0.2549246760202887      0.9998923361686706    0.9998933129358568
0.2550881578111978      0.9998913557150026    0.9998917042550624
0.2552588744244836      0.9998903227896339    0.9998896520482189
0.2554201564152386      0.9998893383610278    0.9998878850323759
0.2555824221865298      0.9998883394449676    0.9998867000636993
0.25575192278019776     0.9998872868381715    0.9998862087554733
0.2559161561125617      0.999886257954799     0.9998861719736953
0.2560897079479157      0.9998851610098054    0.9998860144888888
0.2562579925219658      0.9998840877845706    0.9998852206914194
0.2564210098347119      0.9998830390983433    0.9998836824205882
0.2565912619698346      0.999881934291521     0.9998815386767654
0.25675207948242657     0.9998808816561969    0.999879535567276
0.25691388077555477     0.9998798136445635    0.999878050146708
0.25708291689105967     0.9998786928190654    0.9998773070714151
0.25724668574526055     0.9998775978884451    0.99987718489464
0.25741768942183807     0.9998764445385825    0.9998771242112996
0.2575792584758848      0.9998753453051376    0.9998765603781047
0.25774181131046786     0.9998742299649127    0.9998751934674015
0.25807611936308333     0.9998719061182691    0.9998707602764852
0.25824995826172914     0.9998706815715862    0.9998688161253261
0.25841852989907105     0.9998694834480686    0.9998677944798053
0.25858183427510895     0.9998683126465512    0.9998675400592535
0.2587523734735234      0.9998670792588424    0.9998675242754315
0.2589134780494071      0.9998659039641601    0.9998671169724731
0.25907556640582713     0.9998647114677777    0.9998658997512921
0.25924488958462377     0.9998634549180521    0.9998637670851285
0.2594089455021165      0.9998622268191114    0.9998613135356523
0.2595823199225993      0.9998609174907115    0.9998590324045041
0.25975042708177815     0.9998596365912805    0.9998576690831443
0.259913266979653       0.9998583850850089    0.9998572098955575
0.26008334169990455     0.9998570666006764    0.9998571900330807
0.2602481491588521      0.9998557777753695    0.9998569270339903
0.2604222751207897      0.9998544040328782    0.999855764845823
0.2605911338214233      0.9998530599324416    0.9998536384782528
0.260754725260753       0.9998517464844113    0.9998510396428685
0.26092555152245933     0.999850363015213     0.9998484997056527
0.2610869431616348      0.9998490446690798    0.9998468657465277
0.2612493185813466      0.9998477071393569    0.9998461603260077
0.2614189288234351      0.999846297986635     0.9998460792512514
0.2615832718042196      0.9998449207855133    0.9998459396982172
0.26175693328799415     0.9998434527696654    0.999844982869862
0.2619253275104647      0.9998420166999458    0.9998429648737003
0.26208845447163137     0.9998406136519923    0.9998402738856609
0.26225881625517466     0.9998391357882519    0.9998374252175226
0.26241974341618707     0.9998377278711533    0.9998354050216304
0.26258165435773584     0.9998362996006895    0.9998343737789067
0.26291467862428275     0.9998333244996389    0.9998341104067361
0.26308579194928083     0.9998317760967166    0.9998334078392456
0.26324747065174814     0.999830302586853     0.9998316770764312
0.26341013313475176     0.9998288120270554    0.9998290022506641
0.263580030440132       0.999827241904927     0.9998258699574822
0.2637446604842082      0.9998257074167463    0.9998233342496198
0.26391860903127456     0.9998240719993893    0.9998217960360383
0.2640872903170369      0.9998224721719935    0.9998213818799668
0.2642507043414954      0.9998209091060004    0.9998213729748228
0.2644213531883305      0.9998192628591198    0.9998208751811823
0.2645825674126347      0.9998176943988292    0.9998193408518349
0.26474476541747527     0.9998161032892502    0.9998167079437908
0.26491419824469253     0.999814427085348     0.9998133671224959
0.2650783638106058      0.9998127891131229    0.9998104311286758
0.2652518478795091      0.9998110431988024    0.9998084298463423
0.26542006468710844     0.9998093354871963    0.9998077243467843
0.26558301423340386     0.9998076672350902    0.999807688189178
0.2657531986020759      0.9998059100744009    0.9998073943095932
0.2659139483482171      0.9998042362935659    0.9998061170047855
0.2660756818748946      0.9998025384046608    0.999803615099609
0.26624465022394883     0.999800749596595     0.999800136683028
0.26640835131169904     0.9997990018525927    0.9997968059592458
0.2665792872218259      0.9997971613084029    0.9997942923283201
0.26674078850942196     0.9997954076381658    0.9997931880032801
0.2669032735775543      0.9997936287515305    0.9997930025596556
0.26707299346806324     0.9997917549767362    0.9997929036387248
0.2672374460972683      0.9997899239570792    0.9997919515082004
0.26741121722946337     0.9997879725942843    0.9997894886828556
0.2677429577099417      0.9997841995119249    0.9997822645267943
0.2679134291419055      0.9997822359952017    0.9997792319744916
0.26807446595133855     0.9997803656347598    0.9997776863556275
0.2682364865413079      0.9997784685313094    0.9997772819746026
0.26840574195365385     0.9997764701930574    0.9997772603202147
0.26856973010469576     0.9997745178285776    0.9997765844353187
0.26874303675872785     0.9997724370553808    0.999774397573787
0.26891107615145593     0.9997704022630564    0.9997708723818369
0.2690738482828801      0.9997684149373216    0.9997669079728659
0.26924385523668093     0.9997663220213762    0.9997633234333569
0.2694085949291777      0.9997642770133787    0.999761201011502
0.2695826531246646      0.9997621037990151    0.9997604696581733
0.26975144405884754     0.9997599807937798    0.9997604534390339
0.2699149677317266      0.99975790695093      0.9997599567991082
0.27008572622698224     0.9997557232515708    0.999758003403412
0.27024705009970695     0.9997536430722809    0.9997546639017636
0.2704093577529681      0.9997515332709765    0.9997504774550475
0.27057890022860587     0.9997493111365473    0.9997463843263845
0.2707431754429396      0.9997471400667269    0.999743690619012
0.27091676916026347     0.9997448264625386    0.9997425327137388
0.27108509561628336     0.9997425638852564    0.9997424421053309
0.27124815481099934     0.9997403539612332    0.9997421718378379
0.27141844882809196     0.9997380267673236    0.9997405686987066
0.27157930822265364     0.9997358103311527    0.9997374167316252
0.27174115139775173     0.9997335623842194    0.9997330993418739
0.27191022939522647     0.999731194565053     0.9997285141510887
0.27207404013139713     0.999728881475179     0.9997251726920158
0.27224508568994454     0.9997264460785087    0.9997234557658311
0.272569291342514       0.9997217728977909    0.9997230825504952
0.2727391208814436      0.9997192948252998    0.9997219651470295
0.2729036831590691      0.9997168736700917    0.9997191194814193
0.2730775639396848      0.9997142939301814    0.999714457036402
0.2732461774589965      0.99971177109725      0.9997094521939316
0.2734095237170041      0.9997093069845516    0.99970548369748
0.2735801047973885      0.9997067124761788    0.9997031509325258
0.27374125125524196     0.9997042413764261    0.9997025424362415
0.27390338149363186     0.9997017353544823    0.999702533043894
0.2740727465543984      0.999699096108788     0.9997017627736732
0.27423684435386086     0.9996965179449571    0.9996992693009357
0.2744102606563135      0.9996937707573769    0.9996946908783847
0.27457840969746217     0.9996910846606313    0.9996893376068384
0.2747412914773069      0.9996884615773676    0.9996847114568375
0.2749114080795283      0.999685699633209     0.9996816345423699
0.2750762574204456      0.9996830012735107    0.999680557893729
0.2752504252643531      0.9996801267691815    0.9996805059514308
0.2754193258469566      0.9996773158666579    0.9996799540550296
0.27558295916825604     0.9996745705677543    0.999677707654466
0.2757538273119322      0.9996716820263013    0.9996732481038779
0.2759152608330775      0.9996689390845235    0.9996678304012269
0.2760776781347592      0.999666157642482     0.9996626053326628
0.27624733025881754     0.9996632287791298    0.9996587730194977
0.2764117151215718      0.9996603677351056    0.9996571454711222
0.27658541848731627     0.9996573195817365    0.9996569401537587
0.2767538545917567      0.9996543391997592    0.9996566578097461
0.2769170234348931      0.999651428679689     0.9996548401075916
0.27708742710040624     0.999648364394472     0.9996506588898864
0.2772483961433885      0.9996454463948965    0.9996451015136826
0.27741034896690714     0.9996424874692955    0.9996393055869227
0.27757953661280244     0.9996393714346733    0.9996346271231745
0.27774345699739367     0.9996363279312337    0.9996322821704855
0.27807633278879873     0.9996300725471636    0.9996316936117103
0.2782390371537722      0.9996269781348266    0.9996304586051912
0.2784089763411222      0.9996237200127631    0.999626839606071
0.27857364826716835     0.9996205372347952    0.9996212217474378
0.2787476386962045      0.9996171467233056    0.99961446415782
0.27891636186393676     0.9996138315431219    0.9996089752276219
0.27907981777036495     0.9996105940245156    0.9996058623403382
0.2792505084991699      0.9996071858755897    0.9996048870386923
0.2794117646054439      0.9996039402726015    0.9996048790410087
0.2795740044922544      0.999600649362181     0.9996040500118702
0.2797434792014415      0.9995971841879978    0.9996009031661528
0.2799076866493245      0.9995937997110339    0.9995954296953845
0.28008121260019775     0.9995901940882537    0.9995882712109011
0.28024947128976696     0.9995866691746278    0.9995819402339823
0.2804124627180321      0.9995832274407459    0.9995779114413478
0.280582688968674       0.9995796042008414    0.999576285766257
0.280743480596785       0.9995761546319969    0.9995761882751587
0.2809052560054324      0.9995726571690562    0.9995757452676555
0.2810742662364565      0.9995689744070205    0.9995731910513261
0.2812380092061765      0.9995653780811646    0.9995680524799809
0.2814089869982732      0.9995615928923549    0.999560750288047
0.28157053016783906     0.9995579882658137    0.9995538292551641
0.2817330571179413      0.999554333754365     0.9995486311200509
0.28190281889042007     0.999550486456558     0.9995459856602109
0.28206731340159497     0.9995467341894284    0.9995455329047662
0.2822411264157599      0.9995427446573815    0.9995453561372319
0.2824096721686209      0.9995388445522079    0.9995433524326467
0.28257295066017785     0.9995350365410228    0.9995386148394094
0.28290454266551435     0.9995272119218737    0.9995236502904541
0.2830666051374535      0.9995233428074699    0.9995174461907723
0.28323590243176927     0.9995192691561783    0.9995138081951179
0.2833999324647811      0.999515291006406     0.9995128573725444
0.283573281000783       0.9995110531854513    0.9995128261356487
0.28374136227548097     0.9995069108377003    0.9995114251563665
0.2839041762888749      0.9995028668088135    0.9995072681618956
0.28407422512464553     0.9994986097451964    0.9995000059573087
0.2842390066991122      0.9994944518131776    0.9994916922558683
0.28441310677656895     0.9994900234861311    0.9994840284321175
0.2845819395927217      0.999485694275785     0.9994795173272751
0.2847455051475706      0.9994814671552158    0.9994780572446227
0.284916305524796       0.9994770181754558    0.9994780192839017
0.2850776712794907      0.9994727819745644    0.9994771159916082
0.28524002081472155     0.999468487391725     0.9994735461489176
0.28540960517232916     0.9994639663172018    0.9994665321700712
0.28557392226863276     0.99945955120754      0.9994578197599548
0.2857475578679265      0.9994548486010001    0.9994491148079506
0.2859159262059162      0.9994502519758507    0.9994433997108827
0.28607902728260204     0.9994457644835228    0.9994410929313755
0.28624936318166444     0.9994410412661907    0.9994408607285443
0.28641026445819606     0.9994365450213075    0.9994404059042608
0.2865721495152639      0.9994319871001651    0.9994375544601579
0.28674126939470845     0.9994271886186237    0.9994310297055139
0.286905122012849       0.999422503425637     0.9994221226317263
0.2870762094533662      0.9994175731288482    0.9994125365584088
0.2872378622713526      0.9994128785950822    0.999405628873469
0.2874004988698754      0.9994081198491095    0.9994020301486432
0.2877349744503702      0.9993982197773128    0.9994011302578343
0.28790889711295564     0.9993930110998325    0.9993988735073268
0.28807755251423717     0.999387920113608     0.9993929043989291
0.2882409406542147      0.9993829502750666    0.9993839614406895
0.28841156361656883     0.9993777335095813    0.9993735688451282
0.2885727519563922      0.9993727691500476    0.9993654175908685
0.28873492407675194     0.999367737117373     0.9993605916558247
0.28890433101948826     0.9993624402591436    0.9993590727511981
0.2890684707009207      0.9993572684816321    0.9993590644503089
0.2892419288853431      0.9993517604000479    0.9993575454682123
0.28941011980846165     0.999346377366596     0.9993523529421585
0.2895730434702761      0.999341122988246     0.9993436368761083
0.28974320195446734     0.9993355930251974    0.999332605569577
0.28990809317735455     0.9993301927499714    0.9993229624228176
0.2900823029032318      0.999324442592519     0.9993164626891892
0.2902512453678051      0.9993188221072189    0.9993142388206888
0.29041492057107443     0.9993133350616311    0.9993141609322802
0.2905858305967204      0.9993075612605585    0.9993131353733653
0.29074730599983556     0.9993020643953264    0.999308799834693
0.2909097651834871      0.9992964927791875    0.999300436369662
0.2910794591895152      0.9992906285445089    0.9992889631803906
0.29124388593423944     0.9992849026648576    0.9992781376750035
0.29141763118195363     0.9992788052541767    0.9992700736993653
0.2915861091683639      0.9992728462224323    0.999266700323764
0.2917493198934702      0.9992670295617966    0.9992663123145641
0.2919197654409531      0.9992609085969372    0.9992658258442828
0.29208077636590524     0.9992550825538671    0.9992623825183723
0.29224277107139374     0.9992491775986593    0.9992546538208008
0.29257596286582        0.9992368944825       0.9992310803076105
0.29274924363537114     0.9992304325656673    0.9992212816362646
0.2929172571436184      0.9992241183518632    0.9992164171891299
0.2930800033905616      0.9992179560559247    0.9992153884319303
0.2932499844598815      0.9992114711411333    0.9992152771763416
0.29341469826789746     0.9992051394297743    0.9992127076105334
0.29358873057890345     0.9991983981220967    0.9992050682404003
0.29375749562860554     0.9991918100983195    0.9991932770178543
0.2939209934170036      0.9991853797363018    0.999180465359896
0.29409172602777833     0.9991786141420049    0.9991693245244548
0.29425302401602227     0.9991721745509013    0.999163204041826
0.2944153057848026      0.9991656484724566    0.999161295338385
0.29458482237595945     0.9991587865026421    0.9991612782943355
0.2947490717058124      0.9991520994797172    0.9991595667532529
0.2949226395386554      0.9991449792940544    0.9991529393201029
0.2950909401101944      0.9991380220124326    0.9991414510940981
0.29525397342042947     0.9991312321785002    0.9991278956815877
0.29542424155304114     0.9991240878121842    0.9991150658300625
0.29558507506312204     0.9991172890158824    0.999107136199585
0.2957468923537392      0.9991103989641258    0.9991039574392445
0.2959159444667331      0.9991031472732277    0.9991037236092117
0.29607972931842297     0.9990960689349149    0.9991028216994026
0.2962507489924895      0.9990886222932441    0.9990975547686631
0.2964123340440252      0.9990815338765536    0.9990873162314136
0.2965749028760972      0.9990743503886436    0.9990734345006047
0.29674470653054585     0.9990667912330967    0.9990588722622339
0.2969092429236906      0.999059411623252     0.9990484436689515
0.29708309781982534     0.9990515549104543    0.9990432373797378
0.2972516854546562      0.9990438777786811    0.9990424789537378
0.29741500582818303     0.9990363852487534    0.9990421232073774
0.2975855610240865      0.9990285023738564    0.9990380130279757
0.2977466815974592      0.9990210003544094    0.9990286153504507
0.29807812512765375     0.9990053977205833    0.998998787391921
0.29824219704263544     0.99899758862557      0.9989863455392536
0.29841558746060715     0.9989892739822238    0.9989791338626517
0.29858371061727496     0.9989811506590905    0.9989774231401056
0.29874656651263876     0.9989732239547844    0.9989773939013029
0.2989166572303792      0.9989648839083959    0.9989745395616645
0.2990814806868157      0.9989567421007497    0.9989660459065506
0.2992556226462423      0.998948075384465     0.9989510946558408
0.29942449734436494     0.9989396070128401    0.9989341850618271
0.2995881047811836      0.9989313424880992    0.998920078317699
0.29975894704037886     0.9989226487859623    0.9989111758367225
0.29992035467704325     0.9989143750234165    0.9989084394515241
0.300082746094244       0.9989059914876625    0.9989084053100918
0.3002523723338214      0.9988971705100282    0.9989065785186283
0.3004167313120949      0.99888856074907      0.9988993144704149
0.3005904087933584      0.9988793953906759    0.9988848975363841
0.300758819013318       0.9988704414038229    0.9988671594397196
0.30092196197197363     0.998861720458751     0.9988511187010348
0.30109233975300587     0.998852551032242     0.9988398203932378
0.3012532829115073      0.9988438264801739    0.9988354433358518
0.30141520985054504     0.9988349865137189    0.9988350226264292
0.3015843716119594      0.9988256846081712    0.9988341322854962
0.30174826611206984     0.9988166066116428    0.9988283578107562
0.3019193954345569      0.9988070583618683    0.9988151708025388
0.3020810901345131      0.9987979708143312    0.9987977714496398
0.3022437686150057      0.9987887631077517    0.9987797938397117
0.3024136819178749      0.9987790759651711    0.9987654892480411
0.3025783279594402      0.998769620486934     0.9987585138459885
0.3027522925039955      0.9987595559542711    0.998757096214061
0.3029209897872468      0.9987497231390471    0.9987567985484284
0.3030844198091942      0.9987401283130252    0.9987523744133656
0.30325508465351825     0.9987300357499427    0.9987403509810149
0.3035785288776409      0.9987107019668697    0.9987035206484502
0.30374797770234707     0.998700464444834     0.9986866884377905
0.3039121592657493      0.9986904732121998    0.998677291122313
0.3040856593321416      0.9986798374571657    0.998674481534543
0.30425389213723        0.9986694481051451    0.9986744780083682
0.30441685768101434     0.9986593117711567    0.9986714920352767
0.30458705804717534     0.9986486490356471    0.9986610723073109
0.30474782379080545     0.9986385052299034    0.9986441482146609
0.3049095733149719      0.9986282281980992    0.9986236457523333
0.30507855766151504     0.9986174147977457    0.998604233313825
0.30524227474675425     0.9986068632557435    0.9985919660175562
0.3054132266543701      0.9985957659616189    0.9985871200530364
0.30557474393945505     0.9985852060753143    0.9985869016702189
0.3057372450050764      0.9985745078554368    0.9985854980936619
0.30590698089307433     0.998563253593211     0.9985774922561231
0.3060714495197684      0.9985522703923172    0.9985615542375361
0.30624523664945247     0.9985405807868599    0.9985388526256354
0.30641375651783254     0.9985291624429119    0.9985171714918899
0.3065770091249087      0.9985180225658757    0.9985020535169978
0.3067474965543615      0.9985063061385226    0.99849484271705
0.3069085493612834      0.9984951598776285    0.9984938956350872
0.3070705859487417      0.9984838696826207    0.9984933744265012
0.3072398573585766      0.99847201588485      0.9984872305887682
0.30740386150710763     0.9984604494135663    0.998472717615984
0.3077452395488457      0.9984361138201592    0.9984261364656936
0.3079080276777588      0.9984243845586596    0.9984079022556549
0.3080780506290486      0.9984120472573232    0.9983976799010231
0.30824280631903433     0.9984000070337032    0.998395346238435
0.3084168805120102      0.9983871941716723    0.9983952216700864
0.3085856874436821      0.9983746784671452    0.9983903310084994
0.30874922711405        0.9983624676991321    0.9983768675617469
0.30892000160679456     0.9983496262435954    0.9983542595727242
0.30908134147700833     0.9983374087477828    0.9983298064397684
0.3092436651277584      0.9983250324071103    0.9983086724348083
0.3094132236008851      0.9983120135328686    0.9982952944466252
0.3095775148127078      0.9982993099106577    0.998291108684954
0.3097511245275206      0.9982857897586156    0.9982910753759017
0.30991946698102946     0.9982725850124439    0.9982879028158124
0.3100825421732344      0.9982597038711507    0.9982763989427785
0.31025285218781595     0.9982461565859861    0.9982546283796833
0.3104137275798666      0.9982332703965101    0.9982290031122607
0.31057558675245367     0.998220217243877     0.9982049577134217
0.31074468074741735     0.9982064856236934    0.9981879049020973
0.3109085074810771      0.9981930885847933    0.9981811100144317
0.31107956903711353     0.9981790014732421    0.9981805149055802
0.31124119597061906     0.998165598364051     0.9981791534345558
0.3114038066846609      0.9981520219965424    0.998170512832681
0.31157365222107936     0.9981377428364785    0.9981508589265768
0.31173823049619387     0.9981238096787932    0.9981242142842538
0.3119121272742985      0.9981089834576975    0.9980954964105634
0.31208075679109915     0.99809450349883      0.9980748499621327
0.3122441190465959      0.9980803787241911    0.9980651268637907
0.31241471612446925     0.9980655258075385    0.9980633615538783
0.31257587857981173     0.998051397479223     0.9980629450787752
0.3127380248156905      0.9980370874348066    0.998056453904953
0.31290740587394594     0.9980220360514664    0.9980387265677176
0.3130715196708974      0.9980073519336806    0.9980121918170355
0.313244951970839       0.9979917256243998    0.9979811355443017
0.3134131170094766      0.9979764813093628    0.9979566109566534
0.3135760147868103      0.9979616272019103    0.9979432362228423
0.3137461473865206      0.9979460064977977    0.9979394537821961
0.31391101272492694     0.9979307645413478    0.9979394875802519
0.31408519656632333     0.9979145481989954    0.997934318342381
0.3142541131464157      0.9978987107240015    0.9979179663625614
0.31441776246520414     0.9978832617006025    0.997891508374671
0.31458864660636926     0.9978670182563578    0.9978590084534175
0.3147500961250035      0.9978515663741867    0.9978320867984836
0.31491252942417414     0.9978359164911378    0.9978149552845587
0.3150821975457214      0.9978194576414914    0.997808633252852
0.3152465984059647      0.9978033999844726    0.9978085186943452
0.3154203177691981      0.997786314018553     0.9978053427867919
0.3155887698711275      0.9977696294546622    0.9977915008021125
0.31575195471175294     0.9977533563857643    0.9977662217136473
0.315922374374755       0.9977362453601388    0.9977324280849204
0.31608335941522625     0.9977199716491163    0.9977020653836406
0.31624532823623386     0.9977034900249194    0.9976806145815862
0.3164145318796181      0.997686155319726     0.9976708217947944
0.31657846826169833     0.9976692456510803    0.997669784047092
0.3167517231467687      0.9976512514888077    0.9976683655687117
0.31691971077053505     0.9976336826614448    0.9976575087973404
0.31708243113299744     0.9976165497662478    0.9976342897165181
0.3172523863178365      0.9975985336443068    0.9976000553627027
0.31741707424137156     0.9975809567953927    0.9975657341705285
0.31759108066789676     0.9975622573074024    0.9975383161108767
0.31775981983311796     0.9975439974309009    0.9975253812002784
0.31792329173703515     0.9975261881388605    0.9975231012361634
0.318093998463329       0.9975074645131868    0.997522613470066
0.318255270567092       0.9974896566577439    0.9975145473271658
0.3184175264513914      0.9974716228010271    0.9974936185473013
0.31858701715806736     0.9974526584235747    0.9974596050058882
0.3187512406034394      0.9974341595170882    0.9974226996674022
0.31892478255180157     0.9974144776858526    0.9973904679315434
0.3190930572388597      0.9973952617529596    0.9973729146864074
0.31925606466461387     0.9973765231951274    0.997368199471336
0.31942630691274465     0.9973568218555098    0.9973682562258913
0.31958711453834465     0.9973380886672812    0.9973629203079983
0.31974890594448097     0.9973191488012895    0.9973450420711636
0.3199179321729939      0.9972992322176725    0.9973122929779766
0.3200816911402029      0.9972798074116445    0.9972734942595266
0.3202526849297886      0.9972593882027626    0.9972367669149733
0.3204142440968434      0.9972399669735438    0.9972141421104874
0.3205767870444346      0.9972203005550861    0.9972051991922253
0.32074656481440245     0.9971996220089027    0.9972046750586756
0.3209110753230663      0.9971794508749657    0.9972021051757953
0.3210849043347202      0.9971579928355759    0.9971866136694233
0.3212534660850701      0.997137042493531     0.9971556190284809
0.32141676057411606     0.9971166121848994    0.9971156989479678
0.3215872898855387      0.9970951344235065    0.9970748086815123
0.32174838457443045     0.9970747106180908    0.9970469698095725
0.3219104630438586      0.997054029626703     0.9970337381395192
0.3220797763356634      0.9970322827642514    0.9970316506704044
0.3222438223661642      0.99701107252947      0.99703076883695
0.3224171868996551      0.9969885069277077    0.9970189526317919
0.32258528417184196     0.996966478345364     0.9969907733021791
0.32291817901598446     0.996922418519356     0.9969061364981997
0.32308297658794005     0.9969003913765817    0.9968719630157611
0.32325709266288577     0.9968769624081057    0.996852970518038
0.3234259414765275      0.9968540879405107    0.9968490545100063
0.32358952302886523     0.9968317813661383    0.9968489234309451
0.3237603394035797      0.9968083343447088    0.9968398082099553
0.32392172115576323     0.9967860371435541    0.9968152503387213
0.3240840866884831      0.9967634608561117    0.9967758249453851
0.32425368704357965     0.9967397244275672    0.9967283497894791
0.3244180201373722      0.9967165741008812    0.9966888260841725
0.3245916717341549      0.9966919485645668    0.9966638899781312
0.32476005606963354     0.9966679096767178    0.9966565868900974
0.32492317314380825     0.9966444714335126    0.9966567141424846
0.32509352504035965     0.9966198337999945    0.9966508988537207
0.32525444231438017     0.9965964099155723    0.9966300841050435
0.3254163433689371      0.9965726942453316    0.9965924535169358
0.32558547924587067     0.996547758804504     0.9965430277967373
0.3257493478615002      0.99652344316714      0.9964981891361406
0.3259204512995064      0.9964979012929951    0.9964666168102053
0.32608212011498183     0.9964736347036978    0.9964544701641784
0.3262447727109935      0.996449066377534     0.9964534392227902
0.3264146601293818      0.9964232392568175    0.9964510681700134
0.3265792802864662      0.9963980501266361    0.9964349698950049
0.32675321894654064     0.9963712600178011    0.9963976211117328
0.32692189034531116     0.9963451082877613    0.9963471631750377
0.3270852944827777      0.9963196099123329    0.9962977962625196
0.3272559334426208      0.9962928100144597    0.996259601235532
0.3274171377799332      0.9962673289934152    0.9962421459559062
0.3275793258977818      0.9962415318431141    0.9962389466925922
0.3277487488380071      0.9962144109080746    0.9962383361974539
0.32791290451692845     0.996187963535298     0.9962263620298167
0.32808637869883983     0.9961598323760235    0.9961927975568263
0.3284175252787508      0.9961056079810868    0.996089267616828
0.3285876997604309      0.996077472513374     0.9960439772779113
0.3287526069808071      0.9960500318191092    0.9960195627414977
0.32892683270417333     0.9960208511753471    0.9960129964455029
0.32909579116623555     0.9959923658772935    0.996013055434495
0.32925948236699387     0.995964592155337     0.9960038030245135
0.32943040839012877     0.9959354045327161    0.9959735079771747
0.3295918997907329      0.9959076521258255    0.9959258417394546
0.3297543749718733      0.9958795573034691    0.9958695645401634
0.3299240849753904      0.9958500248028802    0.9958178215481004
0.33008852771760355     0.9958212260035417    0.9957865097155475
0.33026228896280674     0.9957905986003154    0.9957752909388718
0.330430782946706       0.9957607055970151    0.9957753102343319
0.3305940096693013      0.9957315639400371    0.9957696631702079
0.3307644712142732      0.9957009372188649    0.9957441865889936
0.33092549813671435     0.9956718231878621    0.9956988669069271
0.33108750883969174     0.9956423515147725    0.9956407120571317
0.33125675436504576     0.9956113701862733    0.995582641868984
0.3314207326290959      0.9955811634559619    0.9955434403540655
0.331594029396136       0.9955490362856251    0.9955258415716733
0.33176205890187227     0.9955176845375096    0.9955243829211117
0.3319248211463045      0.9954871258677147    0.9955219244484296
0.3320948182131134      0.995455008530881     0.9955021096903207
0.33225954801861834     0.9954237215884627    0.9954593797895411
0.33243359632711333     0.995390462213399     0.9953957574102141
0.3326023773743043      0.9953580010388074    0.9953327473146151
0.3327658911601914      0.9953263561280103    0.99528704596547
0.33293663976845506     0.995293103169863     0.9952638122127426
0.33309795375418794     0.9952614913751642    0.9952600314378904
0.33326025152045713     0.9952294933332594    0.9952595193188043
0.33342978410910296     0.9951958605308318    0.9952447652138737
0.3335940494364449      0.9951630684551565    0.9952064463481302
0.33376763326677683     0.9951281964779504    0.9951433725080763
0.33393594983580477     0.9950941659042668    0.9950755546041339
0.3340989991435288      0.995060995637775     0.9950217365366911
0.3342692832736294      0.9950261373714095    0.9949901160533443
0.33443013278119926     0.9949930064962733    0.9949820242709174
0.3345919660693054      0.9949594719710011    0.9949823559076267
0.3347610341797882      0.994924221777561     0.9949728266878202
0.3349248350289671      0.9948898576766066    0.9949402929958514
0.33509587070052255     0.9948537517765975    0.9948805660529512
0.33525747174954723     0.9948194261872829    0.9948119102751192
0.33542005657910823     0.9947846833128452    0.9947491163930855
0.33558987623104586     0.9947481701591756    0.9947062417352994
0.33575442862167953     0.9947125697966251    0.9946905195462601
0.33592829951530323     0.9946747171795641    0.9946898985972062
0.336096903147623       0.9946377782447773    0.9946844439135949
0.3362602395186388      0.9946017733578443    0.9946575196219039
0.3364308107120313      0.9945639414357932    0.9946012228265823
0.3365919472828929      0.9945279830284851    0.9945307986907628
0.33675406763429083     0.9944915892467971    0.9944611909593487
0.3369234228080655      0.9944533389959581    0.994408610696166
0.3370875107205361      0.9944160508126171    0.9943851675754297
0.3372609171359968      0.9943764003801275    0.99438187110521
0.3374290562901536      0.9943377130532906    0.9943798574879252
0.3375919281830063      0.9943000100297539    0.9943593680760181
0.33776203489823575     0.9942603918753322    0.9943084276282844
0.3379268743521612      0.9942217650502428    0.9942361015155522
0.33810103230907673     0.9941807016306069    0.9941552701534364
0.3382699230046883      0.9941406306477918    0.9940948098034509
0.3384335464389959      0.99410158398801      0.9940645854905228
0.33860440469568015     0.9940605961084061    0.994058056041847
0.33876582832983354     0.9940216369893794    0.9940579261199901
0.33892823574452324     0.9939822092089519    0.994043053907854
0.33909787798158963     0.9939407760280438    0.9939978108869272
0.339262252957352       0.9939003853090387    0.9939267974222978
0.33943594643610453     0.9938574424180163    0.9938408330215002
0.33960437265355303     0.993815542874675     0.9937706340888339
0.33976753160969764     0.9937747092796072    0.9937305968611704
0.3399379253882188      0.9937318069170824    0.9937180679481676
0.3400988845442092      0.9936910365517018    0.9937185210848223
0.34026082748073594     0.9936497770305973    0.9937093263621882
0.3404300052396393      0.9936064158397051    0.9936713467301846
0.3405939157372387      0.9935641515387545    0.9936038202423099
0.3407650610572147      0.9935197544759669    0.9935155035032299
0.34092677175465985     0.9934775527036446    0.9934379745546237
0.34108946623264136     0.9934348456828475    0.9933849686709995
0.34125939553299955     0.9933899720520359    0.9933624101889352
0.34142405757205374     0.9933462273174496    0.9933609916702805
0.34159803811409806     0.9932997253291433    0.9933561960013017
0.34176675139483836     0.9932543533718371    0.9933251006233806
0.3419301974142748      0.9932101357440046    0.9932619436644285
0.34210087825608776     0.9931636840550455    0.993172003037532
0.3422621244753699      0.9931195389955703    0.9930867415740459
0.3424243544751884      0.9930748673962169    0.9930227508246205
0.34259381929738353     0.9930279267826091    0.9929903398756688
0.3427580168582747      0.9929821740352255    0.9929851917368073
0.34293153292215595     0.9929335333127486    0.9929838990145069
0.3430997817247332      0.9928860817616131    0.9929605744392217
0.3432627632660066      0.9928398446442647    0.9929038969998595
0.34359376137077563     0.992745115680265     0.9927227290831344
0.34375552689243105     0.9926984143458443    0.9926470837673649
0.34392452723646316     0.9926493382902764    0.9926025183027075
0.34408826031919126     0.9926015117221162    0.9925909717472066
0.34425922822429605     0.9925512762906888    0.9925914572946177
0.34442076150686995     0.9925035343283583    0.9925776647033292
0.3445832785699803      0.992455227077508     0.9925314093068557
0.3447530304554672      0.9924044990672463    0.9924471989867163
0.3449175150796501      0.9923550765366296    0.9923482118198773
0.34509131820682315     0.9923025430351837    0.9922546500483314
0.3452598540726923      0.9922512947630527    0.9921977528257304
0.3454231226772574      0.9922013585382233    0.9921782183864749
0.34559362610419914     0.9921489036998516    0.9921780550535123
0.34575469490861        0.9920990629435087    0.9921702870059568
0.3459167474935573      0.9920486335296208    0.9921324750192448
0.34608603490088113     0.9919956466966734    0.9920536771107655
0.34625005504690104     0.9919440087785288    0.9919524298948087
0.346423393695911       0.991889114901721     0.9918482483457257
0.3465914650836171      0.9918355712882669    0.9917773941740883
0.3467542692100192      0.9917834058601429    0.9917470829326669
0.34692430815879793     0.9917286056726182    0.9917435949968352
0.3470890798462727      0.9916751929105694    0.9917405740258259
0.34726317003673757     0.9916184261765935    0.9917075800912508
0.3474319929658984      0.9915630483084723    0.9916329314939405
0.34759554863375525     0.9915090880797673    0.9915302533965963
0.3477663391239888      0.9914524135762848    0.9914192298985155
0.34792769499169146     0.9913985611145844    0.9913389187405824
0.3480900346399305      0.991344076195674     0.9912969554740862
0.3482596091105462      0.9912868358344968    0.9912879231148436
0.3484239163198579      0.991231052969374     0.9912878668933189
0.34859754203215976     0.9911717620312288    0.9912639724699275
0.34876590048315753     0.9911139302244434    0.9911976880509114
0.34892899167285135     0.9910575874253604    0.9910968104000806
0.34909931768492186     0.9909984071431314    0.990978410658772
0.3492602090744615      0.9909421862484656    0.9908847200932216
0.34942208424453747     0.9908853077004905    0.9908286703915101
0.34959119423699014     0.9908255492495651    0.9908108742365918
0.3497550369681388      0.990767321398308     0.9908116445125322
0.34992611452166417     0.9907061734449595    0.9907973078973682
0.35008775745265863     0.9906480687011238    0.9907452023245391
0.35025038416418947     0.9905892862983444    0.9906511542108105
0.35042024569809693     0.9905275402695747    0.990528285097874
0.3505848399707004      0.9904673676787503    0.9904177550434252
0.350758752746294       0.9904034218271075    0.9903405125755714
0.35092739826058356     0.9903410513346576    0.9903122726927366
0.3510907765135692      0.9902803338752438    0.9903113292036894
0.3512613895889315      0.9902165680524285    0.9903038301673805
0.3514225680417629      0.9901559890625946    0.9902616385696589
0.3515847302751307      0.9900947060176346    0.9901744012031142
0.3517541273308752      0.9900303290318295    0.9900498577362635
0.35191825712531566     0.9899676014097699    0.989928383676464
0.35209170542274626     0.9899009336501742    0.9898343361972629
0.3522598864588728      0.9898359169722695    0.9897923865810712
0.3524228002336954      0.98977258423132      0.9897866648465707
0.35259294883089465     0.989706067030953     0.9897846323026647
0.3527578301667899      0.9896412447422084    0.9897521169124553
0.3529320300056753      0.9895723675670474    0.9896655460946945
0.3532646278995341      0.9894397372442838    0.9894102533985124
0.35343552803818823     0.9893710099096578    0.9893037532072982
0.35359699355431146     0.9893057143996942    0.9892503514167756
0.353759442850971       0.9892396641950876    0.9892375444971394
0.3539291269700072      0.9891702884426675    0.9892381796307763
0.35409354382773944     0.9891026902945286    0.989215413865906
0.3542672791884618      0.9890308570397554    0.989139476404463
0.35443574728788013     0.9889608034320592    0.9890169749183064
0.3545989481259945      0.9888925645406138    0.9888799354204457
0.3547693837864856      0.988820904258035     0.9887572938685727
0.3549303848244458      0.9887528372519477    0.9886872917786651
0.35509236964294233     0.9886839864777689    0.9886638334004936
0.3552615892838155      0.9886116648041776    0.988664443825447
0.3554255416633847      0.9885412068508307    0.9886512460681625
0.35559881254594394     0.9884663280429336    0.9885885057462311
0.3557668161671993      0.9883933152162938    0.988472892579153
0.35592955252715064     0.9883222046736342    0.9883314566326211
0.35609952370947867     0.9882475245640017    0.9881931414285778
0.3562642276305027      0.9881747591469862    0.9881025624384613
0.3564382500545168      0.9880974476943531    0.9880645581423803
0.356607005217227       0.9880220533023473    0.9880630215263358
0.3567704931186332      0.9879486132568099    0.9880555571517298
0.35694121584241606     0.98787150246939      0.9880033333937154
0.35710250394366805     0.9877982564593188    0.987899083905688
0.35726477582545635     0.9877241928934979    0.9877561345106571
0.35743428252962134     0.9876464341085702    0.9876049558198704
0.3575985219724823      0.9875706803111886    0.9874958491473539
0.3577720799183334      0.9874901864785524    0.9874409246542071
0.35794037060288053     0.9874116998143739    0.9874333421668943
0.3581033940261237      0.9873352587077108    0.9874317337271352
0.3582736522717435      0.987254991491631     0.9873931184419635
0.35843447589483246     0.9871787633569273    0.9873003173060749
0.35876532552445967     0.9870206888563255    0.9869978456330153
0.3589291004891576      0.98694181050901      0.9868692743815561
0.35910011027623223     0.9868590002319545    0.9867940696023283
0.35926168544077597     0.9867803367682412    0.9867753844309672
0.3594242443858561      0.9867007789165043    0.9867768513426098
0.35959403815331287     0.986617233419138     0.9867544617546455
0.35975856465946565     0.9865358421757967    0.986676800248662
0.3599324096686085      0.9864493711630856    0.9865324195444628
0.3601009874164474      0.9863650569454341    0.9863632368406382
0.3602642979029823      0.9862829403670315    0.9862174601324614
0.3604348432118939      0.986196725117114     0.986121498476812
0.3605959538982746      0.9861148450262591    0.9860895431871952
0.36075804836519165     0.9860320373456728    0.9860895269384872
0.3609273776544854      0.9859450737062273    0.986077611568677
0.36109143968247515     0.9858603650199844    0.9860147533927428
0.36126482021345496     0.9857703613934657    0.9858802405571077
0.3614329334831309      0.985682615504002     0.9857073817531548
0.3615957794915028      0.9855971695677986    0.9855452686976915
0.3617658603222514      0.9855074535060186    0.9854259869097515
0.361930673891696       0.9854200519475994    0.9853756519745586
0.36210480596413064     0.9853272105618929    0.9853707123899066
0.36227367077526135     0.9852366865839611    0.98536477784276
0.36243726832508805     0.9851485233346592    0.9853122907098275
0.36260810069729144     0.9850559729769953    0.9851874691855569
0.36276949844696393     0.9849680738461588    0.9850199142153928
0.3629318799771728      0.9848791860160864    0.9848447045576907
0.36310149632975836     0.9847858507987948    0.9847035827957689
0.3632658454210399      0.9846949374102554    0.9846337145124616
0.3634395130153115      0.9845983574818862    0.9846195880338454
0.3636079133482792      0.9845042416148213    0.9846193695436837
0.36377104641994285     0.9844125996031655    0.9845816546525963
0.3639414143139832      0.9843163916427361    0.9844715949611311
0.36410234758549265     0.9842250386831342    0.9843078278941166
0.36426426463753847     0.9841326617673491    0.984122181113041
0.364433416511961       0.9840356564432459    0.9839584647182107
0.3645973011250795      0.9839411816468313    0.9838651105934177
0.3647684205605747      0.9838420191741912    0.9838363859413144
0.364930105373539       0.9837478366657876    0.9838382845870035
0.3650927739670396      0.9836526012605818    0.9838177703822246
0.36526267738291684     0.9835526141748645    0.983730137938155
0.3654273135374902      0.9834552217858081    0.9835737766160414
0.36560126819505356     0.9833517745467327    0.9833672725817202
0.36576995559131303     0.9832509251517164    0.9831838909350552
0.3659333757262685      0.9831527208067555    0.9830673230849721
0.36610403068360065     0.9830496372436782    0.9830212380159429
0.3662652510184019      0.9829517517258088    0.9830204233403314
0.36642745513373953     0.9828527757962804    0.9830106008988412
0.36659689407145385     0.9827488549246787    0.9829406567327202
0.36676106574786416     0.9826476456568599    0.9827964364444738
0.3669345559272645      0.982540134369415     0.982587660005644
0.3671027788453609      0.9824353380990835    0.9823859253955837
0.36726573450215333     0.9823333055536696    0.9822436392628084
0.3674359249813224      0.9822261970912144    0.98217483787677
0.36759668083796065     0.9821245119967633    0.9821662724997616
0.36775842047513524     0.9820216984645421    0.9821647173219215
0.3679273949346865      0.9819137415876545    0.9821141664663692
0.3680911021329338      0.9818086173526415    0.9819870678613994
0.3684235515516508      0.981593514292538     0.9815751879773753
0.3685860427302802      0.9814875838846456    0.9814026607799078
0.36875576873128624     0.9813763770541225    0.9813015014809972
0.3689202274709883      0.9812680735556724    0.9812761363680754
0.3690940047136804      0.9811530452175756    0.9812782470325493
0.36926251469506866     0.9810409240216212    0.9812440815311543
0.3694257574151529      0.9809317613798194    0.9811345322807061
0.36959623495761373     0.9808171848361371    0.9809396873223005
0.3697572778775438      0.9807084138368496    0.9807208474270819
0.36991930457801014     0.9805984760021057    0.9805249946757025
0.37008856610085317     0.980483055093479     0.9803951783168586
0.3702525603623922      0.9803706641871374    0.9803508297669333
0.37042587312692127     0.9802512836260677    0.9803514497835236
0.37059391863014646     0.9801349367998982    0.9803329448257384
0.37075669687206764     0.9800216765099794    0.9802445042047939
0.37092670993636545     0.9799027912857958    0.9800634735974122
0.37109145573935937     0.9797870109913571    0.9798344592248113
0.3712655200453433      0.9796640609382874    0.9796031483325973
0.37143431709002334     0.979544219692286     0.979450136099003
0.37159784687339936     0.9794275413754006    0.9793884932211224
0.37176861147915197     0.9793050928991301    0.9793842658256917
0.37192994146237385     0.9791888366870081    0.9793766796391246
0.372092255226132       0.9790713076632751    0.9793076352730731
0.3722618038122668      0.9789479337480554    0.9791425337911505
0.37242608513709763     0.9788277994655753    0.9789137238831289
0.3725996849649185      0.978700213977587     0.9786631315002948
0.3727680175314355      0.9785758722858272    0.9784800459049454
0.3729310828366485      0.9784548301273214    0.978392056421482
0.3731013829642381      0.9783277941566171    0.9783761228241088
0.37326224846929695     0.9782072087779939    0.9783764981270522
0.37342409775489205     0.9780853081236794    0.9783282330067242
0.37359318186286383     0.9779573370936963    0.9781845838302039
0.3737569987095316      0.9778327448419608    0.9779624726100213
0.373928050378576       0.9777020093817175    0.9776998742125003
0.37408966742508964     0.9775778812560777    0.9774913151225476
0.3742522682521396      0.9774524035695695    0.977366653727143
0.37442210390156616     0.9773207044026124    0.9773278154090329
0.37458667228968884     0.977192465197138     0.9773308159412977
0.37476055918080153     0.9770562942299302    0.9772995071927385
0.3749291788106103      0.9769235878173552    0.9771771756456699
0.375092531179115       0.9767944046247943    0.9769654151491223
0.3752631183699964      0.9766588439877466    0.9766937531801453
0.37542427093834696     0.9765301629893981    0.9764596655587389
0.37558640728723386     0.9764000885922584    0.9763033242039398
0.3757557784584974      0.9762635566498293    0.9762402347145367
0.375919882368457       0.9761306316493934    0.9762390241135178
0.3760933047814067      0.9759894916901616    0.9762241667872152
0.37626145993305243     0.9758519767648994    0.9761269022847998
0.3764243478233942      0.9757181327007132    0.9759321584819182
0.3765944705361125      0.975577672489525     0.9756578061583889
0.3767593259875269      0.97544090435468      0.975394308966873
0.3769334999419314      0.9752957004350568    0.9751950842611669
0.3771024066350319      0.975154193341267     0.9751106629711148
0.3772660460668285      0.9750164446310029    0.9751024985287157
0.37743692032100173     0.9748719158569391    0.9750971201184118
0.37759835995264407     0.9747347172101357    0.9750235210256687
0.37776078336482277     0.9745960430409474    0.9748474905924687
0.37809483257262955     0.9743088122745168    0.9742936134635077
0.378268542048871       0.9741583665011008    0.9740598308149544
0.37843698426380845     0.9740117719086103    0.9739433569485703
0.378600159217442       0.9738690917569387    0.9739202339670412
0.3787705689934522      0.9737193795155531    0.9739224765902722
0.3789315441469315      0.9735772912130891    0.9738719814711776
0.37909350308094714     0.9734336805215011    0.9737203607958251
0.3792626968373395      0.9732829515958422    0.9734587011461823
0.3794266233324278      0.9731362275357232    0.973162957860656
0.3795977846498928      0.9729823032119851    0.9728967793424463
0.379759511344827       0.9728361808319905    0.9727444339277054
0.37992222182029745     0.9726884980388997    0.9726938651169484
0.38009216711814453     0.9725335270029141    0.9726967131052768
0.3802568451546877      0.9723826533423033    0.9726703643886568
0.3804308416942209      0.9722224849627363    0.9725370639586559
0.38059957097245023     0.9720664194668523    0.9722892606888026
0.3807630329893755      0.9719145232284544    0.971985903827383
0.38093372982867746     0.9717551634467544    0.9716904355685617
0.3810949920454486      0.9716039142827521    0.971502235312542
0.3812572380427561      0.9714510563782177    0.9714238068642873
0.3814267188624402      0.9712906450244112    0.9714194859435294
0.3815909324208203      0.9711344981138347    0.9714091319744009
0.3817644644821905      0.9709687167510185    0.97130512766842
0.38193272928225674     0.9708072056388629    0.9710787636882705
0.382095726821019       0.9706500328359772    0.9707748437400852
0.3822659591821579      0.9704851280886089    0.9704529966108666
0.382426756920766       0.9703286706470162    0.9702258033506225
0.3825885384399104      0.9701705556082061    0.9701119152926112
0.3827575547814315      0.9700046168928151    0.9700917823525256
0.38292130386164863     0.9698431133954869    0.9700928554625601
0.3830922877642424      0.9696736984841865    0.9700218184443734
0.38325383704430527     0.9695129010289473    0.9698358438940136
0.38341637010490454     0.9693504056530423    0.9695442853215933
0.38358613798788044     0.9691799048651336    0.9692012432583383
0.3837506386095523      0.9690139388709801    0.9689235405751696
0.3839244577342143      0.9688377610048694    0.9687564356354912
0.3840930095975723      0.9686661240200788    0.9687146381794354
0.3842562941996264      0.9684990990883263    0.9687196732898126
0.3844268136240572      0.9683238814471333    0.9686742150308484
0.384587898425957       0.9681576125854856    0.9685169400975694
0.3847499670083932      0.9679895947240317    0.9682405126064783
0.38491927041320606     0.9678132881692131    0.9678865044892394
0.38508330655671485     0.9676416960021251    0.9675742253551418
0.38525666120321383     0.9674595289780339    0.967361675021267
0.3854247485884088      0.9672820827378084    0.9672889346442292
0.3855875687122999      0.9671094301902438    0.9672904087178322
0.3857576236585676      0.966928297680064     0.9672685445455773
0.3859224113435312      0.9667519852269619    0.967139851445542
0.38609651753148505     0.966564854733817     0.9668603700343736
0.38626535645813487     0.9663825508632293    0.9664998896001675
0.38642892812348073     0.966205148042711     0.9661622407243794
0.38659973461120317     0.9660190701319175    0.9659155513054516
0.38676110647639483     0.965842490370746     0.9658148926981613
0.3869234621221228      0.9656640669748585    0.965805577500081
0.3870930525902274      0.9654768685718207    0.965799698748524
0.3872573757970281      0.9652946785320917    0.9657020271991561
0.38743101750681885     0.9651012922948461    0.9654501364775014
0.3875993919553057      0.9649129212947917    0.9650923417281699
0.3877624991424885      0.9647296416945349    0.9647291020872816
0.38793284115204796     0.9645373879929429    0.9644367049775457
0.38809374853907663     0.96435498767094      0.9642950672468197
0.38825563970664156     0.9641706905227859    0.9642656117814914
0.38842476569658313     0.9639773173923355    0.964269877381762
0.3885886244252208      0.9637891483353153    0.9642045720386304
0.38875971797623504     0.9635918107795028    0.9639921549238782
0.3889213769047185      0.9634045407298021    0.9636630990296047
0.3890840196137383      0.963215330211118     0.9632809352101799
0.3892538971451347      0.9630168428499678    0.9629369389704229
0.3894185074152272      0.9628236688618393    0.9627360990746355
0.38959243618830974     0.9626186573138834    0.9626717023924736
0.38976109770008827     0.9624189664280157    0.962677613996123
0.3899244919505629      0.9622246756011776    0.962638324408067
0.3900951210234141      0.9620209008376461    0.9624606980386308
0.3902563154737345      0.9618275641361071    0.9621505665888094
0.39041849370459125     0.9616322320143869    0.9617598067192228
0.3905879067578246      0.9614273100299597    0.9613779881108317
0.3907520525497541      0.9612279030057487    0.961128522107587
0.3909255168446736      0.9610162570717649    0.9610250879927583
0.39109371387828906     0.9608101336967134    0.9610237251494992
0.39125664365060064     0.9606096140180419    0.9610076875848939
0.39142680824528886     0.9603992934047201    0.9608699175916093
0.39159170557867307     0.9601946062017686    0.960579570462228
0.3917659214150474      0.9599774115911008    0.9601565663285131
0.39193486999011773     0.9597658581660894    0.9597473901216574
0.39209855130388416     0.9595600286354545    0.9594599452425756
0.39226946744002716     0.9593441831113078    0.959323287812859
0.3924309489536394      0.9591393880850547    0.959309099970906
0.39259341424778793     0.9589324956205509    0.959307983119695
0.3927631143643131      0.9587154772357367    0.9592054572336182
0.3929275472195344      0.9585043027112246    0.9589465074437685
0.39326978267465296     0.9580619572251081    0.9580947430945845
0.39343299951025634     0.9578496468468708    0.9577592030123128
0.3936034511682363      0.9576269915171015    0.9575727201470281
0.3937644682036855      0.9574157814318858    0.9575342269859805
0.393926469019671       0.9572024168199056    0.9575414095017004
0.3940957046580331      0.9569785953633583    0.9574750275775452
0.39425967303509135     0.9567608330009573    0.9572557119680455
0.3944329599151396      0.9565297219628095    0.9568591527757944
0.39460097953388384     0.9563046784103212    0.9564044299947404
0.3947637318913242      0.956085788512913     0.9560209313501705
0.3949337190711411      0.9558562100824237    0.9557761507788481
0.39509843898965413     0.9556328155919749    0.9557008124889378
0.3952724774111572      0.9553957882083676    0.9557073223012537
0.3954412485713563      0.9551649553147394    0.9556638717980029
0.3956047524702514      0.9549404047037399    0.955473717743194
0.3957754911915232      0.9547049470700149    0.9550995907008711
0.39593679529026415     0.9544815866116328    0.9546520791167095
0.39609908316954145     0.9542559657223082    0.954228865055242
0.3962686058711954      0.9540193220767861    0.9539281160544671
0.3964328613115453      0.9537890886612274    0.9538103465538048
0.39660643525488537     0.9535447829034147    0.9538064060867759
0.3967747419369214      0.9533068963390138    0.953789243190524
0.39693778135765356     0.9530755184466909    0.953642026573792
0.3971080556007623      0.952832887193837     0.9533026387734279
0.39726889522134023     0.952602772125031     0.9528566305891423
0.3974307186224545      0.9523703377581999    0.9523986900300507
0.3975997768459454      0.952126532392203     0.9520378631627514
0.3977635678081324      0.9518893662989318    0.9518663972414719
0.39793459359269595     0.9516407167126784    0.9518385092503479
0.39809618475472874     0.9514048356878876    0.9518422008649423
0.39825875969729774     0.9511665865464651    0.9517482933812902
0.3984285694622435      0.9509167342507444    0.9514649081003014
0.3985931119658852      0.950673654147083     0.9510273955496896
0.39876697297251706     0.9504157592717764    0.9505075228206854
0.3989355667178449      0.9501646460290221    0.9500933063199073
0.39909889320186887     0.9499204071463959    0.949867065721027
0.3992694545082694      0.9496643266918313    0.9498065224083312
0.39943058119213914     0.9494214493165312    0.9498162647198009
0.3995926916565451      0.9491761434347443    0.9497589538838879
0.3997620369433278      0.9489188744172814    0.9495223863203899
0.3999261149688065      0.9486686153297227    0.9491099234230811
0.4000995114972753      0.9484030794162578    0.9485748416302633
0.4002676407644401      0.9481445631709358    0.9481083597843147
0.400430502770301       0.9478931609894184    0.9478192619022509
0.4006005995985385      0.9476295534464185    0.9477122120295651
0.4007654291654721      0.9473730952352577    0.9477170287567572
0.40093957723539564     0.9471010518069631    0.9476830367047048
0.4011084580440153      0.946836158670649     0.9474799815537223
0.40127207159133094     0.946578499207549     0.9470883042645646
0.4014429199610232      0.9463083869092259    0.9465521427896775
0.4016043337081847      0.9460521942273827    0.9460635226872964
0.4017667312358826      0.9457934607433067    0.9457152564451753
0.401936363585957       0.9455221493355458    0.9455567783577057
0.40210072867472757     0.9452582353907892    0.9455457111085359
0.40227441226648813     0.9449782581594133    0.945537297126799
0.4024428285969448      0.9447056888740062    0.9453829858381739
0.4026059776660974      0.944440625243043     0.9450313054115599
0.4029373108266252      0.9438992221451414    0.9439774732297069
0.40309924387616003     0.9436331075867737    0.9435666063145514
0.40326841174807143     0.9433540390195345    0.943344560303577
0.40343231235867894     0.9430826197519766    0.9433035979436956
0.403603447791663       0.9427981249858788    0.9433120368803422
0.4037651486021163      0.9425282845161058    0.9432179609825314
0.4039278331931059      0.9422557903855208    0.9429284992944084
0.4040977526064722      0.9419700919165559    0.942428253554908
0.4042624047585345      0.941692188681081     0.9418651001229685
0.4044363754135869      0.9413974202992917    0.9413580953957289
0.40460507880733526     0.9411104580896884    0.9410725805100608
0.4047685149397797      0.9408314028634536    0.9409934139409097
0.40493918589460076     0.9405388866044426    0.9410049756667589
0.405100422226891       0.9402614983155396    0.9409497394843334
0.40526264233971754     0.9399813934662087    0.9407108772698611
0.4054320972749208      0.9396876969832542    0.9402429772006784
0.40559628494882005     0.9394020558753975    0.9396691781017136
0.40576979112570943     0.9390990521475334    0.9391065900922192
0.40593803004129475     0.9388041150103758    0.9387492288908186
0.40610100169557617     0.9385173468216246    0.93861762457846
0.40627120817223417     0.9382167267777946    0.9386189258683396
0.4064319800263614      0.9379317158335337    0.9385981534544303
0.40659373566102486     0.9376439249158285    0.9384162566201507
0.4067627261180651      0.9373421497990758    0.9379956463013923
0.4069264493138013      0.9370486944563188    0.9374261075661185
0.40709740733191413     0.936741128098706     0.9368219095161018
0.4072589307274962      0.9364494598584292    0.9363967079694995
0.4074214379036146      0.9361549275226438    0.9361867177949306
0.40759117990210963     0.9358461252534809    0.9361543105186876
0.4077556546393007      0.9355457979845409    0.9361604236003126
0.4079294478794818      0.9352272686736899    0.9360254416329287
0.40809797385835894     0.9349172265102293    0.9356547709583181
0.40843172611588197     0.9342998137385723    0.9344636430776089
0.408592785033301       0.9340002487584678    0.933972753043165
0.40875482773125627     0.9336977870901072    0.9336926127350733
0.4089241052515883      0.9333806762363       0.9336178608592202
0.4090881155106163      0.9330723146162884    0.9336321299999899
0.40926144427263444     0.9327452347532671    0.9335499334529663
0.4094295057733485      0.9324269165141116    0.9332406985790477
0.4095923000127587      0.9321174668603128    0.9327185648237175
0.40976232907454546     0.9317930985865283    0.9320617616704131
0.40992709087502827     0.9314776399156377    0.931495046282152
0.4101011711785012      0.9311431195672368    0.9311234431484212
0.4102699842206701      0.9308175213866289    0.9310091085483194
0.41043353000153515     0.9305009538265933    0.9310205160506616
0.41060431060477676     0.9301691944547789    0.9309729838634101
0.4107656565854875      0.9298546460718533    0.9307266211418574
0.41092798634673466     0.9295370828037842    0.9302432884648267
0.4110975509303584      0.9292041890653921    0.9295811969490455
0.41126184825267814     0.928880486772404     0.9289619456891176
0.41143546407798803     0.9285371933565446    0.9285094890753728
0.4116038126419939      0.9282031041937485    0.9283317483973986
0.4117668939446959      0.9278783290858832    0.9283261735926589
0.41193721006977446     0.9275379472440504    0.928313936266263
0.41209809157232224     0.9272152939793668    0.9281294814380397
0.4122599568554063      0.926889560971739     0.9276998957697243
0.41242905696086707     0.9265480810553859    0.9270495397429535
0.41259288980502384     0.9262160781420911    0.9263866330556232
0.41276395747155725     0.9258681937899614    0.9258538329491379
0.4129255905155598      0.9255383483080193    0.92559864939813
0.41308820734009877     0.9252053681116235    0.9255503779881543
0.4132580589870143      0.9248563643915502    0.9255637510334859
0.41342264337262585     0.9245170028380596    0.9254477247621968
0.41359654626122755     0.9241571607625922    0.9250547711026647
0.41376518188852524     0.9238068946505021    0.9244275834618114
0.413928550254519       0.9234663994258635    0.9237361984501352
0.4140991534428894      0.9231095948528171    0.9231303874092858
0.4142603220087289      0.9227713657146519    0.9227979116880011
0.4144224743551048      0.9224299364995623    0.9227005332871268
0.4145918615238573      0.9220720554746241    0.9227171369015698
0.4147559814313058      0.9217241141227182    0.9226523867407831
0.4149294198417445      0.9213551434582846    0.9223304071534009
0.4150975909908791      0.9209961266007988    0.9217443734538803
0.41526049487870986     0.9206471775919773    0.921038977704005
0.4154306335889172      0.9202814929338304    0.9203636713700499
0.41559550503782056     0.9199259205758876    0.919936752941595
0.41576969498971406     0.9195489568327377    0.9197745456169104
0.4159386176803035      0.919182119687376     0.9197838272424764
0.41610227310958897     0.9188255245441768    0.9197499156038975
0.4162731633612511      0.9184519064934302    0.9194838637535645
0.4164346189903824      0.9180977323129972    0.9189602891263515
0.4165970584000501      0.9177402383736261    0.9182551511754189
0.4167667326320944      0.9173655761236511    0.9175256992705114
0.41693113960283473     0.9170013283879922    0.9170162993266193
0.4171048650765652      0.9166151325603855    0.9167785286726003
0.41727332328899164     0.916239365738941     0.9167634558494312
0.41743651424011413     0.9158741443980543    0.9167620349229941
0.4176069400136132      0.915491465059007     0.916566417646439
0.41776793116458155     0.9151287803436522    0.9161035684637241
0.4180991158499674      0.9143790365914001    0.9146436549347496
0.41826305834254474     0.914006082093501     0.9140487005538458
0.4184363193381121      0.9136106180928111    0.91371889967947
0.41860431307237556     0.913225888453276     0.913659199973928
0.418767039545335       0.912852010707811     0.9136772410409266
0.4189370008406711      0.9124602362747318    0.9135531440601561
0.4191016948747033      0.9120793596392822    0.9131519599848453
0.4192757074117255      0.9116756008653047    0.912440931257616
0.4194444526874437      0.9112827547391112    0.9116417435718763
0.419607930701858       0.9109009401706357    0.910984429322545
0.4197786435386489      0.9105009333718134    0.9105829870787998
0.41993992175290895     0.9101217658056793    0.9104761144757555
0.42010218374770536     0.9097390385379205    0.9104959112447414
0.4202716805648784      0.9093379655734504    0.9104281525199959
0.42043591012074755     0.908948106504895     0.9101010310178708
0.4206094581796067      0.908534788370666     0.9094396141076438
0.42077773897716186     0.9081327000877487    0.9086269528807374
0.4209407525134131      0.9077419618715854    0.9078990685277247
0.421111000872041       0.9073325826764653    0.9073978355092563
0.421271814608138       0.9069446694662302    0.9072185021135316
0.4214336121247714      0.9065531853172425    0.9072216609380098
0.4216026444637814      0.906142911194125     0.9072024333973161
0.4217664095414875      0.9057441684076766    0.906957442309539
0.4219374094415702      0.9053264921152302    0.906375615086541
0.42209897471912206     0.9049306221611166    0.9056042016683953
0.42226152377721027     0.9045311263599182    0.9048109244279762
0.4224313076576751      0.9041125465765438    0.9041886516962919
0.42259582427683595     0.9037056811533153    0.9038982938751159
0.4227696593989869      0.9032744090580066    0.9038594440413444
0.4229382272598339      0.9028548675538678    0.903869890623221
0.4231015278593769      0.9024471787275725    0.903696149771197
0.4232720632812965      0.9020201061392569    0.903187728115893
0.4235952486606106      0.9012070423150738    0.9016114655825445
0.4237645680629124      0.9007791304639513    0.9008976864891703
0.42392862020391026     0.9003632554277551    0.9005101714731967
0.42410199084789824     0.8999223927505232    0.9004122941977072
0.42427009423058215     0.8994935832488457    0.9004369701034353
0.4244329303519621      0.8990769497080845    0.9003338539211411
0.4246030012957187      0.8986404797932884    0.899914181531182
0.42476780497817135     0.8982162345883888    0.8991952580639392
0.4249419271636141      0.8977666169770545    0.8982800277607164
0.4251107820877529      0.8973292405845392    0.8975010479325438
0.42527436975058774     0.8969042293837632    0.8970371106521756
0.42544519223579924     0.8964590774956136    0.8968843808260668
0.42560658009847985     0.8960372487474318    0.8969053544080448
0.42576895174169677     0.8956116091489242    0.8968561559183449
0.4259385582072903      0.8951656753255446    0.896519547535026
0.4261028974115799      0.8947322937213816    0.8958583428767534
0.42627655511885965     0.8942728319538441    0.8949387716510873
0.42644494556483536     0.893825924857105     0.8940872072270873
0.4266080687495072      0.8933917164538275    0.8935207635368206
0.42677842675655564     0.8929369039617552    0.8932806979149339
0.4269393501410732      0.8925060155771942    0.8932786115734707
0.42710125730612714     0.8920712530730905    0.8932740783146469
0.4272703992935577      0.8916157340910725    0.8930278052219547
0.42743427401968426     0.8911731043126052    0.8924446098065767
0.42776705849415986     0.8902703164075404    0.8906751339735188
0.4279297172006686      0.8898271338509458    0.8899861864672386
0.42809961072955394     0.8893628941227941    0.8896196035445139
0.42826423699713534     0.8889117355193031    0.8895603956366156
0.42843818176770687     0.8884336358081103    0.8895832775005538
0.4286068592769744      0.8879686351996952    0.8894137183736492
0.428770269524938       0.8875168600552344    0.8889095390752928
0.4289409145952783      0.8870437218424722    0.8880573371776072
0.42910212504308765     0.8865954642804769    0.8871417012171141
0.4292643192714333      0.8861432172498512    0.8863623844657338
0.42943374832215564     0.8856694533227105    0.8858867301160324
0.429597910111574       0.8852091071960283    0.8857606270388837
0.4297713904039825      0.8847212271377748    0.8857899893396202
0.42993960343508697     0.884246782910813     0.8856951494966554
0.43010254920488755     0.8837859011348252    0.8852843801786959
0.43027272979706477     0.8833031960025365    0.8844927366020278
0.430437643127938       0.8828341044948853    0.8835383964219662
0.43061187496180126     0.8823370874144559    0.8826236580131784
0.4307808395343605      0.8818537021139504    0.8820642799454228
0.43094453684561584     0.8813840761529264    0.8818787151788706
0.43111546897924785     0.8808923180603845    0.8818987677146758
0.43127696649034897     0.8804264100053123    0.8818581978695224
0.4314394477819865      0.8799563956488211    0.8815330860054987
0.4316091638960007      0.8794640938828882    0.8808102915626184
0.43177361274871084     0.8789857456329557    0.879860093469574
0.4319473801044111      0.8784788734019416    0.8788729902360416
0.4321158801988073      0.8779859728948782    0.8782016643577709
0.4322791130318996      0.877507171929207     0.877922721571439
0.43244958068736855     0.8770057221106137    0.8779105945318926
0.4326106137203066      0.876530634099779     0.8779144585892219
0.4327726305337811      0.8760513776471912    0.877681229005025
0.43294188216963225     0.875549363758138     0.8770485825897212
0.4332791694217166      0.8745448091856352    0.8750826636075106
0.43344720503794987     0.874042288079257     0.8742952628034951
0.43360997339287916     0.8735542159718771    0.8739025509112239
0.4337799765701851      0.8730430805256499    0.8738308463588865
0.43394471248618705     0.8725464469706811    0.8738609627711751
0.43411876690517914     0.8720202931977873    0.8736869914314059
0.43428755406286723     0.871508660987794     0.8731229932874504
0.4344510739592513      0.8710116793555527    0.8722309165282628
0.434621828678012       0.8704913269527429    0.8711665998089504
0.43478314877424196     0.8699984279807164    0.8703186740111956
0.4349454526510082      0.8695012510065584    0.8698145089624192
0.4351149913501511      0.8689805492452424    0.8696678409839864
0.43527926278799        0.868474696394079     0.8697018985548449
0.43545285272881906     0.8679387275246029    0.8696079671951846
0.4356211754083441      0.8674176271999633    0.8691475732916504
0.43578423082656514     0.8669115241809022    0.8683198746937023
0.4359545210671628      0.8663815895577257    0.8672411906273689
0.4361153766852297      0.8658797240224066    0.8663037934676967
0.4362772160838329      0.865373523266172     0.8656757995428668
0.43644629030481275     0.8648433372070674    0.8654279631257971
0.43661009726448863     0.8643283463867892    0.8654411026393305
0.4367811390465412      0.8637892214750776    0.8654182122573274
0.4369427462060629      0.8632785310336812    0.8650979586571136
0.4371053371461209      0.8627634535061849    0.8643799738021714
0.43727516290855556     0.862224087406945     0.8633246700009319
0.43743972140968623     0.8617001154717847    0.8622810580321278
0.43761359841380704     0.8611450449775679    0.8614698382649475
0.43778220815662383     0.8606053883396576    0.8611152261276702
0.43794555063813667     0.8600812747004948    0.8610867081089256
0.4381161279420262      0.8595325647590281    0.861105871245893
0.43827727062338484     0.8590129067800858    0.8608811998800024
0.4384393970852799      0.858488804323583     0.8602597675803841
0.4386087583695516      0.857939951402445     0.859245694715154
0.4387728523925193      0.8574067224111787    0.8581556359508349
0.43894626491847705     0.8568417359889882    0.85722316583356
0.4391144101831308      0.8562925175175495    0.8567411311976465
0.4392772881864806      0.8557591964854836    0.8566435551113306
0.4394474010122071      0.8552008132982506    0.8566839323003932
0.43961224657662956     0.8546583820139578    0.8565456392708061
0.43978641064404217     0.8540838582541429    0.8559706051101635
0.43995530745015077     0.8535253076350412    0.8549950876373339
0.4401189369949554      0.8529828599358793    0.8538807108401637
0.44028980136213675     0.8524150462144263    0.8528735595400553
0.4404512311067872      0.8518772890603243    0.8522998251341437
0.44061364463197394     0.8513349841416945    0.8521201039764147
0.4407832929795374      0.8507671616228702    0.8521560441093773
0.4409476740657968      0.850215643687215     0.8520921312880799
0.4411213736550464      0.8496314443805371    0.851631404440792
0.44128980598299195     0.8490635707577493    0.8507333000515984
0.44145297104963355     0.8485121517725206    0.8496143568174962
0.44162337093865184     0.8479349132471945    0.8485141241624293
0.44178433620513924     0.8473883505811043    0.8478109560037883
0.44194628525216295     0.8468371887094303    0.8475236614585941
0.44211546912156335     0.846260056689894     0.8475279649479135
0.44227938572965975     0.8456995796728433    0.8475259101034134
0.44245053716013283     0.8451129862872104    0.8471960527473514
0.442612253968075       0.8445574351667219    0.8464450059448845
0.4427749545565536      0.8439972368183288    0.8453613092117077
0.44294488996740883     0.8434107711651191    0.8441785269069109
0.44310955811696007     0.8428411613724948    0.8433039035334108
0.44328354476950144     0.8422379044906495    0.8428568366533129
0.4434522641607388      0.841651524493666     0.8428089222987173
0.4436157162906722      0.8410821498022055    0.8428425300500231
0.4437864032429823      0.8404862071531376    0.842616749739326
0.4439476555727615      0.839921923532522     0.8419697018084316
0.444109891683077       0.8393529415003309    0.8409356038062611
0.4442793626157692      0.8387572415437711    0.8397112866344407
0.4444435662871574      0.8381787464532283    0.8387208376556026
0.44461708846153575     0.8375660216135496    0.8381330169992406
0.44478534337461006     0.8369705225278036    0.8380037903761667
0.44494833102638043     0.8363923523184789    0.8380514269475831
0.4451185535005275      0.8357869991011937    0.8379268278375117
0.44527934135214364     0.8352139291489888    0.8374017757737277
0.4454411129842961      0.8346361104019762    0.8364471695347346
0.4456101194388253      0.8340311204696625    0.8352088284884811
0.44577385863205043     0.8334436910925139    0.8341109891930426
0.4459448326476523      0.8328289478017922    0.833372032581357
0.44610637204072323     0.832246853447522     0.8331265511639078
0.44626889521433055     0.831659966649793     0.833152190733855
0.44643865321031456     0.8310456195711884    0.8331280526921446
0.44660314394499456     0.830449034978923     0.832744624211801
0.44677695318266464     0.8298172652292765    0.831830642620421
0.44694549515903076     0.829203280318426     0.8306049673462134
0.44710876987409287     0.8286072072509654    0.8294277826067398
0.44727927941153167     0.8279833823617414    0.8285485148358106
0.4474403543264396      0.8273928193051829    0.8281834409024075
0.44760241302188386     0.826797419553104     0.8281615962349018
0.44777170653970483     0.8261741232792719    0.8281894512487786
0.4479357327962218      0.825568938520057     0.8279252095568015
0.4481090775557289      0.8249280036110287    0.8271354979170652
0.4484399652908311      0.8237006602429201    0.8247179785856982
0.4486100103501069      0.8230679334062632    0.823696009146247
0.4487747881480787      0.8224535198120178    0.8231806122337821
0.4489488844490406      0.8218029863941333    0.8230892094401921
0.44911771348869844     0.8211707883483854    0.8231385773281888
0.4492812752670524      0.820557051148237     0.8229531413852317
0.44945207186778297     0.8199148413422302    0.8222697580763109
0.44961343384598274     0.8193068641110952    0.8211864533316452
0.4497757796047188      0.8186939633324173    0.8199198806416509
0.44994536018583153     0.8180524470941635    0.8187804147016331
0.45010967350564035     0.8174295890434747    0.8181238764668579
0.4502833053284392      0.8167700533123269    0.8179334376266705
0.450451669889934       0.816129197655846     0.8179875300160563
0.45061476719012494     0.8155071461208827    0.8178973727785012
0.45078509931269245     0.8148561945801774    0.8173506535247
0.4509459968127292      0.8142400743349628    0.8163571764326433
0.4511078780933022      0.8136189875447641    0.8150885823052169
0.4512769941962519      0.8129687623309407    0.8138427916768528
0.45144084303789767     0.8123374441019588    0.8130318188251406
0.45161192670192        0.8116769400047407    0.8127106997857921
0.45177357574341154     0.8110516330037141    0.8127322473177012
0.45193620856543937     0.8104213191945279    0.8127334894472179
0.4521060762098439      0.8097616812263727    0.8123600523981818
0.4522706765929444      0.8091212477220288    0.8114836454765328
0.45244459547903504     0.8084432234714776    0.8101525376749718
0.45261324710382167     0.8077844269883158    0.8088310347276925
0.4527766314673044      0.8071449810141371    0.8078833251468592
0.4529472506531637      0.8064759354475336    0.8074262841840413
0.45310843521649224     0.805842682511622     0.8073924285761352
0.4532706035603571      0.8052043869469296    0.8074374447504548
0.45344000672659857     0.8045363558806885    0.8071888009603653
0.4536041426315361      0.803887870822003     0.8064409696490394
0.45377759703946363     0.8032012612377204    0.8051717307982829
0.4539457841860873      0.8025342206127326    0.8037970492799233
0.4541087040714069      0.8018868696396305    0.8027123491353318
0.45427885877910323     0.801209512466634     0.8020953091985331
0.45444374622549555     0.8005518982096387    0.801973361268537
0.454617952174878       0.7998558116767347    0.8020369570796769
0.4547868908629564      0.7991794909891045    0.8018687808756743
0.45495056228973096     0.7985230563230438    0.8012161638794935
0.4551214685388821      0.7978363461997499    0.800021308520813
0.45528294016550236     0.7971863654592307    0.7986739979460056
0.45544539557265895     0.7965312703000468    0.7974788921170788
0.45561508580219223     0.7958457668697128    0.7967107654972635
0.4557795087704215      0.7951803410501963    0.7964837399843818
0.4559532502416409      0.7944759211304795    0.7965401916443787
0.4561217244515563      0.793791601399162     0.7964686619688114
0.4562849314001678      0.7931274999852903    0.7959534910732838
0.45645537317115586     0.7924327257733624    0.7948645434150732
0.4566163803196131      0.7917752543755133    0.7935222356140617
0.45677837124860665     0.7911126346936892    0.7922291052912446
0.4569475969999769      0.790419211762212     0.7912976546762701
0.4571115554900431      0.7897461955167623    0.7909349430752481
0.4572848324830995      0.7890336725370082    0.7909495057690277
0.45745284221485183     0.78834157847156      0.7909578732838398
0.4576155846853003      0.7876698510609328    0.7905896617319831
0.4577855619781253      0.7869670452736796    0.7896392531071832
0.4579502720096464      0.7862848434120084    0.7883020388054273
0.45812430054415754     0.7855627929950173    0.7868409797532332
0.4582930618173648      0.7848613703702042    0.7857925409547725
0.45845655582926803     0.784180691591984     0.7853194361779425
0.4586272846635479      0.783468688624188     0.7852811474094842
0.4587885788752969      0.7827949051543697    0.7853325803080097
0.4589508568675823      0.782115910487626     0.7850883720942029
0.4591203696822443      0.7814054673643783    0.7842771000286132
0.4592846152356023      0.780715955660123     0.7830057805841728
0.45945817929195043     0.7799861038958062    0.7814947201392899
0.45962647608699464     0.7792772069582231    0.7803019176547157
0.45978950562073484     0.7785893782443231    0.7796701867638365
0.45995976997685173     0.7778698535872526    0.7795341610839072
0.4601205997104377      0.7771891021530511    0.7796054444879379
0.4602824132245601      0.7765031140903624    0.77948026899485
0.46045146156105904     0.7757853088753619    0.778829755173073
0.46061524263625403     0.7750887552756819    0.7776609811416386
0.4607862585338257      0.7743602653747551    0.7761527207304391
0.4609478398088665      0.7736708724851387    0.7748633687352939
0.46111040486444366     0.7729762139828555    0.7740264166236029
0.4612802047423975      0.7722494992148555    0.7737283738362768
0.46144473735904734     0.7715442181988567    0.7737787159104128
0.4616185884786873      0.7707978097421291    0.7737480295447094
0.4617871723370232      0.7700728579247895    0.7732437016659987
0.46195048893405516     0.7693694725816712    0.7721893213096603
0.4621210403534638      0.768633792060478     0.770700246353682
0.46228215715034154     0.767937745294605     0.7693190796419842
0.46244425772775566     0.7672364091652153    0.7683241384740486
0.46261359312754646     0.7665026609477424    0.7678756187002721
0.46277766126603326     0.7657906571638908    0.7678715726817728
0.4629510479075102      0.7650370629425712    0.7679118289135163
0.4631191672876831      0.764305235586855     0.7675612866959385
0.46328201940655206     0.7635952822958607    0.7666514027921888
0.46361692602773924     0.7621320088374501    0.7637368551110286
0.46379106421067096     0.7613693736957635    0.7625279886635806
0.46395993513229866     0.7606285724220132    0.7619611254498698
0.4641235387926224      0.7599098293998043    0.7618986901487188
0.46429437727532286     0.7591582053603873    0.7619704806275102
0.4644557811354924      0.7584470640617124    0.7617476776085108
0.4646181687761983      0.757730585808449     0.7609790725288252
0.46478779123928093     0.7569811165589879    0.7596296045248706
0.46495214644105953     0.7562538813727463    0.7581117858275166
0.46512582014582826     0.755484306622818     0.7567598865320769
0.4652942265892929      0.7547369893451205    0.7560238266691115
0.46545736577145363     0.7540120332822292    0.7558563029811914
0.46562773977599103     0.7532538672161591    0.7559391439231911
0.46578867915799754     0.7525366942346982    0.7558340174463214
0.4659506023205404      0.75181416917293      0.7552258069752257
0.46611976030546        0.7510583278299237    0.7539950834230906
0.46628365102907554     0.7503250184431524    0.7524771753115749
0.4664547765750678      0.7495582876439887    0.7510109566229712
0.46661646749852914     0.748832846812038     0.7501042793248327
0.4667791422025268      0.7481020328263872    0.7497683952369204
0.46694905172890117     0.7473376929182398    0.7498139082330373
0.4671136939939715      0.7465960538666412    0.7498183698956957
0.46728765476203193     0.7458113803007868    0.7493396745184713
0.46745634826878846     0.7450494303386578    0.7482365498339225
0.46761977451424097     0.744310303195713     0.7467493755900481
0.4677904355820702      0.7435374420189821    0.7451937586290821
0.4679516620273685      0.7428063595675884    0.7441292517039867
0.46811387225320317     0.7420698909706885    0.7436421952324282
0.4682833173014145      0.741299587526668     0.7436209933992662
0.4684474950883218      0.7405522707362537    0.7436843142034213
0.4686209913782192      0.7397615161221642    0.743363689275649
0.46878922040681265     0.7389937703547408    0.7424181536071075
0.4691223787637683      0.7374704529569817    0.7393812557184815
0.46928314073090355     0.73673402852795      0.7381588525019169
0.4694448864785752      0.7359922060994415    0.7374945390290213
0.4696138670486235      0.7352162509295098    0.7373654949756003
0.46977758035736783     0.7344635594333022    0.7374557924433269
0.46994852848848884     0.7336766389167426    0.7372928768782216
0.47011004199707895     0.7329321343666231    0.7365790751936744
0.4702725392862054      0.7321821713177876    0.7352946433793935
0.4704422713977085      0.7313978782262532    0.7336559533761379
0.4706067362479076      0.7306370124467986    0.7322375501869959
0.4707805196010968      0.7298320663428238    0.7313180680344612
0.4709490356929821      0.7290505704454059    0.7310634794022568
0.47111228452356335     0.7282926166020278    0.7311437736185997
0.4712827681765213      0.7275001468816792    0.7310931559425414
0.4714438172069484      0.7267506688351475    0.7305426156469875
0.4716058500179118      0.7259957697033496    0.7293897236660163
0.47177511765125185     0.7252062656198757    0.7277751384904879
0.4719391180232879      0.7244404579789021    0.7262519273568216
0.47211243689831406     0.7236302098891678    0.7251416868306653
0.4722804885120363      0.7228436804960816    0.7247258715186127
0.47244327286445453     0.7220809579621116    0.7247575238349165
0.47261329203924946     0.7212834531268       0.724796872095768
0.47277804395274037     0.7205097983285648    0.7243999510157895
0.47295211436922135     0.719691473393425     0.7232957446474477
0.4731209175243984      0.7188970205986341    0.721718051761833
0.4732844534182714      0.7181265263835669    0.720136288040747
0.4734552241345211      0.717321077047113     0.7189035384760847
0.47361656022823995     0.7165593157796794    0.7183591813973885
0.47377888010249514     0.715792118659038     0.7183147463223175
0.473948434799127       0.7149898838708538    0.7184030523482525
0.4741127222344549      0.7142117537969255    0.7181527737660269
0.47428632817277283     0.7133886186647391    0.7172204993591893
0.4744546668497869      0.7125896097100486    0.7157277660257706
0.4746177382654969      0.711814809690735     0.7141003940069152
0.47478804450358364     0.7110048076590485    0.7127065273672017
0.4749489161191395      0.7102389060664237    0.7119822623458014
0.4751107715152316      0.7094675679373444    0.7118230378257214
0.47527986173370046     0.7086609494697893    0.7119262089100765
0.4754436846908653      0.7078786802799463    0.7118141646825188
0.4756147424704068      0.70706105230772      0.7110906570675717
0.47577636562741743     0.7062877625159255    0.7097916630188639
0.4759389725649644      0.7055090269898717    0.7081620064268307
0.4761088143248881      0.7046948566476202    0.7066020027067721
0.4762733888235078      0.7039051627511083    0.7056313427203043
0.47644728182511753     0.7030698226437252    0.7052975171012383
0.4766159075654233      0.7022589950504705    0.7053791877226712
0.4767792660444251      0.7014727578159492    0.7053675423043994
0.4769498593458036      0.7006509306742197    0.7048180802194601
0.47711101802465117     0.6998738375495509    0.7036577091655543
0.4772731604840351      0.6990913024399765    0.702064592517811
0.47744253776579576     0.6982731074891365    0.700407690277247
0.4776066477862524      0.6974796373206472    0.6992588132120903
0.4777800763096991      0.696640348274958     0.6987495896406585
0.47794823757184185     0.6958258051114774    0.6987700806786143
0.4781111315726806      0.6950360816065957    0.6988358695327727
0.478281260395896       0.6942105590529553    0.6984667760947555
0.47844612195780745     0.6934098942183068    0.6974437228326035
0.47862030202270894     0.692563229279493     0.695791512173336
0.47878921482630654     0.6917414429727095    0.6940837581448539
0.47895286036860013     0.6909446070264017    0.6928139397925079
0.47912374073327035     0.6901118381798271    0.6921720479774368
0.4792851864754098      0.6893243915903101    0.6921171532792761
0.47944761599808555     0.6885315076643073    0.6922214709289545
0.479617280343138       0.6877026282879333    0.6919999489670499
0.47978167742688643     0.6868988243590073    0.6911457627474927
0.47995539301362494     0.6860487609184679    0.6895960356255334
0.4801238413390595      0.6852237931304479    0.6878525581412189
0.48028702240319004     0.684423988659129     0.6864347578714957
0.4804574382896972      0.6835880642365976    0.6856010424959225
0.4806184195536736      0.6827978042402857    0.6854271904091392
0.48078038459818634     0.6820021171931954    0.6855370996565111
0.48094958446507574     0.6811702522985634    0.6854528765821172
0.48111351707066113     0.6803636694129722    0.6847875463315244
0.4812867681792366      0.6795105857600133    0.683382879185267
0.4814547520265081      0.6786828036185778    0.6816432931207125
0.4816174686124756      0.6778803866206439    0.6800932041348381
0.48178742002081976     0.6770416765503566    0.6790505077254646
0.48195210416786        0.676228365464102     0.6787162439773554
0.48212610681789025     0.6753684030841615    0.6788041380656256
0.4822948422066166      0.6745338589974863    0.6788004795211877
0.48245831033403896     0.6737247947480772    0.6782692303054287
0.482629013283838       0.6728792724127413    0.6770049742029585
0.48279028161110615     0.6720798937547223    0.6753629583995898
0.4829525337189106      0.6712751010426988    0.6737227388465252
0.4831220206490917      0.6704338521217744    0.672497754042903
0.4832862403179689      0.6696181964877379    0.6719958570072354
0.4834597784898361      0.6687556738387412    0.6720160304147879
0.48362804940039944     0.6679187638468945    0.6720927041670127
0.48379105304965875     0.6671075241970698    0.6717359613665211
0.48396129152129475     0.6662597301662109    0.6706546502913955
0.4842838830000413      0.6646516969245482    0.6673897491247728
0.4844529054520595      0.6638083832624171    0.6659836518534362
0.48461666064277364     0.6629908443078526    0.6652849146166961
0.4847876506558645      0.6621366604725984    0.6651892068093033
0.48494920604642444     0.6613291194698335    0.6653096158236452
0.4851117452175207      0.660516187556011     0.6651478708343891
0.4852815192109936      0.659666569361145     0.6643154697715036
0.4854460259431626      0.6588428270029851    0.6628584972693354
0.4856198511783216      0.65797191217545      0.6610089928509939
0.48578840915217675     0.657126891593236     0.6594606445123686
0.48595169986472786     0.6563078163713437    0.6585781016825037
0.48612222539965566     0.6554519725049242    0.6583474699231905
0.48628331631205257     0.654643035882426     0.6584646297871737
0.4864453910049858      0.6538287289597196    0.658426096276838
0.48661470052029565     0.652977616932035     0.6577889829720592
0.4867787427743016      0.6521525445997672    0.6564822122386545
0.4869521035312976      0.651280140197473     0.654657393952382
0.48712019702698967     0.6504337931516193    0.6529847767045885
0.48728302326137773     0.6496135502937326    0.6519049144472427
0.4874530843181425      0.6487564298164348    0.6514995555383806
0.48761787811360324     0.647925441007103     0.651575374242691
0.48779199041205407     0.6470470240081401    0.6516181657608703
0.4879608354492009      0.6461947561150208    0.6511163211687689
0.4881244132250438      0.6453686818127758    0.6499302264469105
0.48829522582326335     0.644505663479954     0.6481708513274204
0.488456603798952       0.6436899352209166    0.6464855582646559
0.48861896555517703     0.6428688689028115    0.6452370522531572
0.48878856213377875     0.6420108300985492    0.6446555347016343
0.48895289145107645     0.6411790260965191    0.6446549907291591
0.48912653927136424     0.6402996618389295    0.6447655406427684
0.4892949198303481      0.639446593873068     0.6444393871829072
0.489458033128028       0.63861986255511      0.6434341669189483
0.4896283812480845      0.6377561010650292    0.6417689960941412
0.4897892947456102      0.6369398451267407    0.6400328319992517
0.4899511920236722      0.6361182771818882    0.6386187320037566
0.4901203241241109      0.6352596561789087    0.6378335660500133
0.49028418896324555     0.6344274500827974    0.6377139196986658
0.49045528862475685     0.6335581655351039    0.6378539570526909
0.49061695366373737     0.6327365046444012    0.6377191556745482
0.4907796024832542      0.631909545323093     0.6369563416560032
0.49094948612514766     0.6310454878722965    0.6354611552066282
0.4911141025057372      0.6302079200956942    0.633673452823694
0.4912880373893168      0.629322624419446     0.6319973629337223
0.49145670501159244     0.6284638344989286    0.6310255878145096
0.49162010537256406     0.627631583846596     0.6307711380917996
0.4917907405559123      0.6267621959743269    0.630898579751088
0.4919519411167298      0.6259406126774606    0.6308817183413005
0.4921141254580836      0.6251137608441676    0.6303051352518961
0.492283544621814       0.6242497571684127    0.6289717135208325
0.49244769652424053     0.6234123605696393    0.6272216160223708
0.4926211669296571      0.6225271603261868    0.6254324833188031
0.49278937007376966     0.6216685824745505    0.6242620686461631
0.49295230595657824     0.6208366561405394    0.6238368435141184
0.49312247666176345     0.6199675489553412    0.6239086112198347
0.49328738010564477     0.6191251137488992    0.6239834906077596
0.4934616020525161      0.6182348343187394    0.6235238763040217
0.49363055673808354     0.6173712419374754    0.6223166547234341
0.49379424416234696     0.6165343632115823    0.6206158472617003
0.493965166408987       0.6156602811597992    0.6187889519886444
0.4941266540330963      0.6148342513530751    0.6175122129332764
0.4942891254377419      0.6140030027945259    0.6169185013653788
0.4944588316647641      0.6131345446262229    0.616905271425406
0.4946232706304824      0.6122928573756129    0.617034550069583
0.49479702809919074     0.6114032823396698    0.6167489933738769
0.4949655183065951      0.6105404925252292    0.6157298509704066
0.4951287412526955      0.609704511819418     0.6141279092783333
0.4952991990211726      0.6088313114978523    0.612257270522978
0.49546022216711877     0.6080062900114989    0.6108214902412079
0.4956222290936013      0.6071760844332472    0.6100348030501792
0.49579147084246056     0.606308655675441     0.609894468555492
0.4959554453300158      0.6054680845359283    0.6100432460243683
0.4961287383205611      0.6045796017084056    0.6099224449601672
0.4962967640498024      0.603717990009007     0.6091156205817649
0.4964595225177398      0.6028832673092803    0.6076558549527283
0.4966295158080538      0.6020113189053125    0.6057835089548862
0.4967942418370638      0.6011662752552756    0.6041695752135464
0.4969682863690639      0.6002733132518363    0.6031445629199063
0.49713706363976007     0.599407269261096     0.6028955997121623
0.4973005736491523      0.5985681585986583    0.603033395883882
0.4974713184809212      0.5976918263258592    0.6030153586505782
0.4976326286901591      0.5968638339410965    0.6024090264376277
0.49779492267993347     0.5960307162372708    0.601103143663121
0.49796445149208446     0.595160383911461     0.5992701605391001
0.4981287130429314      0.5943170246807147    0.5975535922358606
0.4983022930967685      0.5934257550354947    0.5963293455140033
0.4984706058893016      0.5925614709864224    0.595909320890762
0.4986336514205308      0.5917241835200042    0.5959967412078169
0.4988039317741366      0.5908496954706618    0.5960800373541784
0.49896477750521157     0.5900236202480137    0.595665688722969
0.4991266070168229      0.5891924590061417    0.5945478372465621
0.4992956713508109      0.5883241091134687    0.5927998729316328
0.49945946842349487     0.5874827887586916    0.5910109372214695
0.49963050031855544     0.5866042880968407    0.5895987742174422
0.4997920975910852      0.5857742349868271    0.5889746636740946
0.49995467864415133     0.5849391209795299    0.5889436565951406
0.5001244945195941      0.5840668419520438    0.5890974980431855
0.5002890431337328      0.5832216217345976    0.588887083632329
0.5004629102508618      0.582328545195059     0.5879077615565056
0.5006315101066865      0.5814625389608467    0.5862710180646369
0.5007948427012073      0.5806236071512153    0.5844547167186452
0.5009654101181049      0.5797475408828714    0.5828860953703165
0.5011265429124716      0.5789199623453646    0.5820742629314087
0.5012886594873747      0.5780873654396529    0.5819164631271143
0.5014580108846544      0.5772176971783117    0.5820789555898715
0.50162209502063        0.5763751359155522    0.5820138866425533
0.5017954976595959      0.5754847801485727    0.5812510789560682
0.5019636330372577      0.5746215296638599    0.5797676055453032
0.5021265011536156      0.573785384128603     0.5779631970303992
0.50229660409235        0.5729121658967988    0.5762550305746083
0.5024614397697805      0.5720660605113941    0.5752227787936469
0.5026355939502011      0.5711722074106793    0.5749137355955846
0.5028044808693177      0.5703054771774753    0.5750585874212517
0.5029681005271305      0.5694658669471252    0.5750784741019838
0.5031389550073198      0.5685892297079399    0.5744802591256929
0.5033003748649783      0.5677610979351183    0.5731941376001927
0.503462778503173       0.5669280195270754    0.5714375126890942
0.5036324169637445      0.5660579417246568    0.5696365995034616
0.503796788163012       0.5652149950545503    0.5684244093417603
0.5039704778652696      0.5643243892106156    0.5679375560407424
0.5041389003062231      0.5634609238001913    0.5680240561728686
0.5043020554858727      0.5626245918511397    0.5681314253216526
0.504472445487899       0.5617513156272954    0.5677295772887523
0.5046334008673945      0.56092653106994      0.5666296433320562
0.5049645140098348      0.5592302475124594    0.5630993194598555
0.5051284207309392      0.5583907878748757    0.5617078379876379
0.5052995622744204      0.5575144444615912    0.5610151717839279
0.5054612691953707      0.5566865763327519    0.5609859607160153
0.5056239598968573      0.5558538380669814    0.5611509417091789
0.5057938854207206      0.5549842516428857    0.5609706892237974
0.5059585436832799      0.5541418043544365    0.5600905421358007
0.5061325204488292      0.5532518831347519    0.5584368163143947
0.5063012299530746      0.5523891093817731    0.5565553536313135
0.506464672196016       0.5515534695011051    0.5550246115945119
0.5066353492613341      0.5506810525140641    0.5541398502533362
0.5067965917041214      0.5498570654184888    0.5539866857236337
0.5069588179274449      0.5490282566996175    0.5541540118292695
0.5071282789731452      0.548162711172898     0.5541151157074122
0.5072924727575414      0.5473242940694286    0.5534362354639294
0.5074659850449277      0.5464385407409709    0.5519458115685222
0.50763423007101        0.5455799492569114    0.5500844330644281
0.5077972078357884      0.5447485503383374    0.548430583176355
0.5079674204229435      0.5438804988920515    0.5473384265837754
0.5081281983875677      0.5430608056312095    0.5470236205397653
0.5082899601327282      0.5422363399560675    0.547151762490578
0.5084589567002654      0.541375265283851     0.5472304440294646
0.5086226860064986      0.5405412918410376    0.5467622877410016
0.5087936501351085      0.539670748737817     0.5454954595379468
0.5089551796411874      0.5388485154964778    0.5437763497414458
0.5091176929278027      0.5380215429740772    0.5420158524265768
0.5092874410367948      0.5371580474246137    0.5406763770295511
0.5094519218844829      0.5363216358587864    0.5401226092519221
0.5096257212351609      0.5354381534277631    0.5401669375487556
0.5097942533245351      0.5345817607902593    0.5403158837618387
0.5099575181526053      0.5337524337355469    0.5400235916570402
0.5101280178030522      0.5328866766473778    0.538955507448064
0.5102890828309682      0.5320691315682026    0.5373378926858974
0.5104511316394205      0.5312468967511772    0.5355399323998218
0.5106204152702495      0.5303882827843764    0.5340335530034562
0.5107844316397745      0.529556711288328     0.5332834622035427
0.5109577665122895      0.5286782491628127    0.5332036317193323
0.5111258341235007      0.5278268346687258    0.5333875292095014
0.5112886344734078      0.527002439898581     0.5332629863544321
0.5114586696456916      0.5261417674557112    0.5324184357829171
0.5116234375566715      0.5253081105656704    0.5309010687656658
0.5117975239706414      0.5244276898636236    0.5289683106944726
0.5119663431233074      0.5235742896238748    0.527362796218765
0.5121298950146693      0.5227478794921802    0.52647386132722
0.5123006817284079      0.5218852994606097    0.5262856526459692
0.5124620338196156      0.5210707379701559    0.526460635673165
0.5126243696913597      0.5202515751941522    0.5264608388146624
0.5127939403854805      0.5193963008209393    0.5258170085922433
0.5129582438182974      0.5185679840577595    0.5244590512546063
0.5131318657541042      0.5176931123480945    0.5225629062921404
0.5133002204286071      0.51684520250905      0.5208440698129709
0.513463307841806       0.5160242204934853    0.5197658766902284
0.5136336300773816      0.5151672431267074    0.5194083547814639
0.5137945176904263      0.5143581393260986    0.519541472194573
0.5139563890840073      0.5135445734189509    0.5196439517917871
0.5141254952999651      0.512695136936755     0.5192084979126191
0.5142893342546189      0.5118725844004197    0.5180447772084187
0.5144604080316493      0.511014161353848     0.5162551237309118
0.5146220471861488      0.5102035081095486    0.514510256586401
0.5147846701211847      0.509388344938066     0.5132069186169212
0.5149545278785973      0.5085373748254334    0.5126069623002307
0.5151191183747059      0.5077132440842091    0.5126370557401293
0.5152930273738046      0.5068429409481552    0.5128139643317792
0.5154616691115993      0.5059994793184582    0.5125503194923572
0.51562504358809        0.5051828189227382    0.5115739681414853
0.5157956528869574      0.5043304782997673    0.5098956244662551
0.5159568275632939      0.5035257305153803    0.5081202700633238
0.5161189860201667      0.5027165244840395    0.5066686409560153
0.5162883792994163      0.5018717054470472    0.5058739145416191
0.5164525053173619      0.501053638022559     0.5057836977677052
0.5166259498382975      0.5001896437820306    0.5059840208992201
0.5167941270979292      0.4993524027293684    0.5058836346239478
0.5169570370962568      0.49854187141224127   0.5051159698049107
0.5171271819169612      0.4976958610961276    0.5035912606331943
0.5172920594763615      0.49687654899496814   0.5017778101230692
0.5174662555387519      0.49601147889452757   0.5001056910153558
0.5176351843398385      0.4951731083516529    0.49917707222247143
0.517798845879621       0.4943613916894784    0.49898740882344844
0.5179697422417802      0.49351433574294035   0.49917721228524686
0.5181312039814084      0.49271455712864143   0.4991899137764505
0.5182936495015731      0.49191041365067495   0.4986074197506894
0.5184633298441144      0.4910710046369627    0.49724556347031734
0.5186277429253516      0.4902581916240516    0.49547536583567736
0.518801474509579       0.48939989137657824   0.4937014247315498
0.5189699388325025      0.488568187913216     0.49258805038173326
0.519133135894122       0.4877630323805222    0.49224043965500763
0.5193035677781181      0.48692275831047305   0.4923839230927245
0.5194645650395834      0.4861295435013824    0.49249430758521046
0.519626546081585       0.48533201914199536   0.49210347775587954
0.5197957619459633      0.48449945360928176   0.4909375444191242
0.5199597105490377      0.4836933731288227    0.4892516602066343
0.520132977655102       0.4828420986138788    0.487404865397359
0.5203009774998625      0.48201748427037533   0.4861042904370142
0.5204637100833189      0.4812192966812795    0.48556783908668266
0.520633677489152       0.48038622838530937   0.4856196566404899
0.5207983776336811      0.47957957029388987   0.48579701758883315
0.5209723962812003      0.47872791112855495   0.4855345862416629
0.5211411476674156      0.477902660584664     0.4845170085899863
0.5213046317923269      0.47710376401252147   0.4829162555563387
0.5214753507396148      0.4762701420392605    0.4810687674321192
0.5216366350643719      0.4754831830223801    0.4796845309005075
0.5217989031696653      0.47469200912465626   0.47897125549555986
0.5219684060973353      0.4738661906813141    0.478908282350188
0.5221326417637013      0.47306665222145994   0.47910952863011
0.5223061959330575      0.47222241367263457   0.4790090597521602
0.5224744828411098      0.47140445307431683   0.4781981707201825
0.522637502487858       0.47061271322776443   0.4767370024251629
0.5228077569569829      0.46978648998182987   0.47489262262935367
0.5229685768035769      0.4690066688127581    0.4733770209198508
0.5231303804307073      0.4682226854804472    0.4724704314457118
0.5232994188802143      0.46740430250146137   0.4722530463177144
0.5234631900684172      0.46661206227492336   0.4724383859501244
0.523634196078997       0.46578550149746106   0.47248286300304404
0.5237957674670458      0.4650051826775346    0.4719380569930647
0.523958322635631       0.46422074299119426   0.47068805891173515
0.5241281126265929      0.4634020689683222    0.468903828026713
0.5242926353562507      0.4626094569293813    0.46722767079355365
0.5244664765888987      0.4617726664965714    0.46604531813940336
0.5246350505602427      0.46096193536453345   0.4656698715213628
0.5247983572702828      0.46017720190088945   0.465805324358387
0.5249688988026995      0.45935840591178645   0.4659418111194884
0.5251300057125853      0.45858556984334536   0.4655782947067619
0.5252920964030074      0.4578086677075675    0.46451178793337156
0.5254614219158062      0.4569977921540914    0.46281968982453053
0.525625480167301       0.4562128294934553    0.46108884527405697
0.525798856921786       0.45538402141468026   0.4597293168505946
0.525966966414967       0.4545811230996961    0.4591669237772049
0.526129808646844       0.45380407039219195   0.459212340922591
0.5262998857010976      0.452993221859145     0.4594089628029087
0.5264646954940473      0.45220828824345416   0.4592097570736701
0.526638823789987       0.451379860026483     0.45823687127921625
0.5268076848246228      0.450577244929482     0.45663292424351415
0.5269712785979546      0.4498003760983349    0.4548908517242909
0.5271421071936631      0.44898989968509967   0.45343797638945327
0.5273035011668407      0.44822489014038047   0.4527443649021271
0.5274658789205546      0.4474559115590068    0.45268167522720526
0.5276354914966452      0.44665341679433523   0.45289574050201253
0.5277998368114318      0.4458765729603753    0.45283787709364565
0.5279735006292086      0.4450564647613842    0.45206870736632127
0.5281418971856813      0.4442620019822447    0.45060668739762233
0.52830502648085        0.4434931159301687    0.4488751770326363
0.5284753905983954      0.442690895821802     0.4472955086865006
0.52863632009341        0.4419338234899114    0.4464204144153917
0.5287982333689609      0.4411728329077064    0.4462147935959087
0.5289673814668885      0.4403786017519367    0.44640841641454615
0.529131262303512       0.4396098490946531    0.44647386392292054
0.5293023779625123      0.4388079448884271    0.44593236739492526
0.5294640589989817      0.4380509951578916    0.44471608378863503
0.5296267238159875      0.4372901683961368    0.4430467526586154
0.5297966234553699      0.4364962855530816    0.44134758731305684
0.5299612558334483      0.43572778072408364   0.4402303562910928
0.5301352067145167      0.4349165989522868    0.43982548676461874
0.5303038903342812      0.43413078918977205   0.439967653580606
0.5304673066927418      0.4333702793778679    0.4401133313187753
0.530637957873579       0.4325769037071862    0.43975251260338427
0.5307991744318854      0.4318281473552364    0.4387100442742233
0.5309613747707281      0.4310755663372798    0.437128774933245
0.5311308099319474      0.4302902170092496    0.4353813830163951
0.5312949778318629      0.4295300642569757    0.43410891385486683
0.5314684642347683      0.4287276032147673    0.43352055558946473
0.5316366833763698      0.42795033242041663   0.43357333071672166
0.5317996352566673      0.4271981779953853    0.43377119427922983
0.5319698219593414      0.42641344964454847   0.4335875389487662
0.5321305740394848      0.42567299668045233   0.43274246350711476
0.5322923099001645      0.42492877203137064   0.4312880491573759
0.5324612805832208      0.42415207306645564   0.4295277381772082
0.5326249840049732      0.42340038423220605   0.42811029459627875
0.5327959222491021      0.42261649802496365   0.42732151654249406
0.5329574258707004      0.4218767103720031    0.42723082834409104
0.5331199132728348      0.4211331939726529    0.42744014926451485
0.533289635497346       0.42035740667501764   0.4274422162306939
0.5334540904605531      0.4196065106610053    0.4268223872574618
0.5336278639267503      0.41881394012960105   0.4254303436503847
0.5337963701316435      0.418046252168027     0.4236965750500647
0.5339596090752329      0.4173033691276899    0.4221774807522131
0.5341300828411989      0.41652841183155453   0.4212104364936437
0.534291121984634       0.4157971432921339    0.4209798407192225
0.5344531449086055      0.4150621932748227    0.4211571177363166
0.5346224026549535      0.41429526892565005   0.42126669249122595
0.5347863931399978      0.41355303477409766   0.4208378172188005
0.5349597021280319      0.41276950719784694   0.41963152714435326
0.5351277438547621      0.4120106609118105    0.41796299169102474
0.5352905183201884      0.4112764171885315    0.41636533603418374
0.5354605276079913      0.4105103982345807    0.41521515255918096
0.5356252696344903      0.409768951818088     0.41481216430734374
0.5357993301639793      0.40898646569734326   0.41494305491457234
0.5359681234321644      0.40822854307275763   0.41510726195666525
0.5361316494390455      0.4074951040670929    0.4148049364300056
0.5363024102683032      0.40673009183210174   0.41375880060349185
0.5364637364750302      0.4060081707183396    0.41223206731232115
0.5366260464622934      0.4052826565795951    0.4105977829236789
0.5367955912719333      0.40452567182821186   0.409301530585924
0.5369598688202692      0.4037930525620771    0.40873369824665334
0.5371334648715952      0.403019785070077     0.40877004102306663
0.5373017936616171      0.4022708738285252    0.40898029697356586
0.5374648551903353      0.40154623788264354   0.4088354540387427
0.53763515154143        0.40079033578063455   0.40798727050283784
0.5377960132699939      0.40007714358483326   0.40658146250011223
0.5379578587790941      0.39936040703131676   0.40494145026660805
0.538126939110571       0.3986125081508335    0.4035090905244654
0.5382907521807438      0.39788876473647866   0.40275832892644636
0.5384618000732933      0.39713395858585193   0.40265927955992276
0.538623413343312       0.3964216336368532    0.40287134645273065
0.538786010393867       0.39570580542723344   0.40289212041903005
0.5389558422667986      0.39495906052255286   0.402294669018358
0.5391204068784264      0.3942365264951824    0.40103683692586967
0.5392942899930442      0.3934740127553752    0.399308426631566
0.539462905846358       0.39273551430629655   0.3977819114126872
0.5396262544383679      0.39202094724430314   0.39687244954832895
0.5397968378527542      0.39127563779257724   0.396633635440504
0.5399579866446098      0.39057240111909675   0.3968164293296497
0.5401201192170019      0.38986570662599745   0.39693188638309096
0.5402894866117706      0.38912837288272517   0.39651877351763165
0.5404535867452351      0.3884148434571682    0.3954300328794654
0.54062700538169        0.3876617310609251    0.39377268169126023
0.5407951567568408      0.386932411411824     0.3921733345744027
0.5409580408706876      0.3862268002644641    0.3911008593059473
0.5411281598069111      0.3854907553500688    0.39069307623112415
0.5412930114818306      0.38477838522109614   0.3908160378286462
0.5414671816597402      0.3840266945858447    0.3909970530489738
0.5416360845763457      0.38329866707217664   0.3907039655189441
0.5417997202316475      0.38259421770471075   0.3897424667505561
0.5419705907093257      0.3818595418641885    0.3881804003192702
0.5421320265644731      0.38116629477111114   0.38661010493827463
0.5422944462001569      0.3804696705168258    0.3854086445292786
0.5424641006582174      0.37974292422494094   0.3848423464968462
0.5426284878549739      0.3790396273665147    0.38487880305181843
0.5428021935547204      0.37829741173788134   0.38509928661270226
0.542970631993163       0.3775786338141026    0.3849580304815116
0.5431338031703015      0.3768832082786298    0.3841754072002748
0.5433042091698168      0.3761578687624625    0.3827341840845478
0.5434651805468012      0.37547355247308556   0.3811568207192721
0.5436271357043219      0.3747859021652616    0.37983436471124715
0.5437963256842193      0.37406844268323886   0.37909216487041636
0.5439602484028128      0.37337420633950613   0.3790085958273437
0.5441314059437827      0.3726502618990755    0.37923226270081173
0.5442931288622219      0.3719670990017341    0.3792454262744356
0.5444558355611976      0.371280639949426     0.3786903281262709
0.5446257770825498      0.37056457852907176   0.37743448956146575
0.5447904513425978      0.3698716098945342    0.37585078593265375
0.5449644441056363      0.3691403898471035    0.3743233095279143
0.5451331696073707      0.3684322506992496    0.37342947544773264
0.5452966278478011      0.367747236337112     0.37322127143821643
0.545467320910608       0.3670329112160628    0.37341720845103926
0.5456285793508843      0.36635894394937735   0.3735218594267416
0.5457908215716968      0.3656817232013323    0.37313260949901306
0.5459602986148859      0.3649752223970815    0.37203993877713326
0.5461245083967713      0.36429157492994624   0.3705213857577339
0.5462980366816466      0.3635700899634061    0.3689272165881825
0.5464662977052179      0.3628714446302086    0.3678765918738445
0.5466292914674853      0.36219555190473957   0.367517907298804
0.5467995200521292      0.36149058488186203   0.36765134270468514
0.5469603140142423      0.36082555879362926   0.36782493882741535
0.5471220917568919      0.3601573161283061    0.3676014107212051
0.5472911043219181      0.35946010204272455   0.3666967332920453
0.5476258297517391      0.35808203854768145   0.36366627829148496
0.5477873752553071      0.3574182647927604    0.36248949175638423
0.5479499045394114      0.35675130833214813   0.3619241529123492
0.5481196686458922      0.3560555834393003    0.36193357137444027
0.5482841654910693      0.35538234179580397   0.36215026359424096
0.5484579808392364      0.3546719212003465    0.3620681046954917
0.5486265289260994      0.353983970152023     0.3613365748620113
0.5487898097516587      0.3533184016433591    0.3600357257846942
0.5489603253995945      0.35262427041441113   0.35842609553290594
0.5491214064249994      0.3519694161857047    0.35714221410025654
0.5492834712309407      0.3513114161424633    0.35642044738201584
0.5494527708592585      0.35062495623597584   0.35630812809480666
0.5496168032262725      0.3499607444710546    0.3565186542497593
0.5497901540962765      0.3492597524172146    0.356560454786924
0.5499582377049766      0.3485809948577634    0.356014700767954
0.5501210540523727      0.3479243851285068    0.3548608710941926
0.5502911052221455      0.34723952043855666   0.3532885348396929
0.5504558891306143      0.34657676753723854   0.35188390712614886
0.5506299915420731      0.3458774966134413    0.3509728698291358
0.5507988266922281      0.3452003237778388    0.35076735641977824
0.5509623945810791      0.3445451620426011    0.3509545974188194
0.5511331972923066      0.3438619510081875    0.35106597759598557
0.5512945653810033      0.3432173505981547    0.35068223349507044
0.5514569172502365      0.3425696753734822    0.34967500487798103
0.5516265039418462      0.3418942352439344    0.3481676031955331
0.5517908233721518      0.34124066681169785   0.34670826788848425
0.5519644613054477      0.34055098731441596   0.3456497248737699
0.5521328319774397      0.33988316289197296   0.3453002446076839
0.5522959353881276      0.3392371063703089    0.3454307012371059
0.552466273621192       0.33856331176993065   0.3456095946147393
0.5526271772317258      0.3379276987540797    0.3453806793405968
0.5527890646227958      0.33728904358576345   0.3445418279546986
0.5531220417883852      0.33597808736121054   0.34164357130711304
0.5532931315629046      0.33530588330486927   0.340451647875366
0.5534547867148931      0.3346716128289037    0.3399358209740933
0.553617425647418       0.3340343300705997    0.3399551185548515
0.5537872994023194      0.3333696053820697    0.3401743396387671
0.553951905895917       0.33272637487554696   0.3401082089763881
0.5541258308925047      0.3320476740317408    0.3393963767561228
0.5542944886277883      0.33139045229587377   0.33809859169623996
0.5544578791017678      0.3307546234596832    0.3366135924085799
0.5546285043981243      0.330091550951051     0.33531694640392895
0.5547896950719498      0.3294659961832341    0.334657420230458
0.5549518695263116      0.3288374595714332    0.3345694661138464
0.55512127880305        0.32818177773603413   0.33478354361832796
0.5552854208184845      0.3275473534110141    0.3348260387591877
0.5554588813369091      0.3268778429663722    0.3342917314853454
0.5556270745940297      0.32622957511589784   0.33313386753172963
0.5557900005898465      0.3256024645810705    0.3316816631489703
0.5559601614080398      0.32494840492446675   0.3302949139471313
0.556125054964929       0.3243154659606574    0.3294699538821346
0.5562992670248084      0.3236476923569081    0.32926539144855405
0.5564682118233839      0.3230010245137707    0.3294587546365276
0.5566318893606553      0.322375377198818     0.329560396126705
0.5568028017203035      0.3217229766690475    0.32915390910819253
0.5569642794574208      0.321107434868827     0.328161518374494
0.5571267409750744      0.32048897088002054   0.3267656323475653
0.5572964373151046      0.31984385094863965   0.3253258526816932
0.557460866393831       0.3192196176402851    0.3243720282580414
0.5576346139755474      0.31856092936186625   0.32403041335145705
0.5578030942959598      0.31792317935315123   0.32417253431749404
0.557966307355068       0.31730628864425103   0.3243363428178682
0.5581367552365533      0.31666294097151926   0.3240827526688188
0.5582977684955075      0.31605603596684895   0.3232438657665836
0.5584597655349981      0.31544623727856896   0.3219327311228261
0.5586289973968652      0.31481007585169907   0.3204619870968783
0.5587929619974286      0.314194562092215     0.3193811669418968
0.5589662451009818      0.31354497210444754   0.31888165211783176
0.5591342609432313      0.31291601398050767   0.3189407403594158
0.5592970095241766      0.31230760418071624   0.3191415954353152
0.5594669929274987      0.31167301931965863   0.3190353281106439
0.5596317090695166      0.31105894598011646   0.3183424810328919
0.5598057437145247      0.31041103748159005   0.31703033766919764
0.559974511098229       0.30978362471455145   0.31556218201200603
0.5601380112206292      0.3091766244766879    0.31441001739480584
0.5603087461654062      0.30854363590905604   0.3138054339408887
0.5604700464876522      0.30794644180760405   0.3137805608031727
0.5606323305904346      0.30734640428416476   0.31398680043070365
0.5608018495155935      0.30672047057622776   0.3139914584281248
0.5609661011794487      0.30611481688304576   0.31345523235393197
0.5611396713462937      0.3054756899132069    0.31227162875494036
0.5613079742518348      0.30485682740598      0.3108237849685627
0.561471009896072       0.3042581475444642    0.3095838386117772
0.5616412803626858      0.30363375502830586   0.30882869075005753
0.5618021162067688      0.3030447599807343    0.3086882312250454
0.5619639358313883      0.3024529453156965    0.3088717257646895
0.5621329902783843      0.30183550854227276   0.30897154142332517
0.5622967774640761      0.30123812407957845   0.3085986797115264
0.562467799472145       0.30061520607346776   0.30758666514041666
0.5626293868576828      0.3000274523576066    0.3062473405951514
0.5627919580237569      0.29943690358505975   0.30492552805652745
0.5629617640122078      0.29882091133984073   0.30398586192240373
0.5631263027393545      0.29822484151054013   0.3036753786958967
0.5633001599694913      0.2975958835383833    0.3038128230772796
0.5634687499383244      0.2969868326806193    0.3039723280181933
0.5636320726458534      0.29639760883832994   0.30373484421525837
0.563802630175759       0.29578312091690373   0.3028720050790429
0.5639637530831338      0.29520340847489246   0.30160740915557527
0.5641258597710448      0.29462099937765457   0.30025748490332016
0.5642952012813327      0.29401343388891005   0.2991968809583511
0.5644592755303164      0.2934255640471993    0.2987481739053691
0.5646326682822903      0.29280515770385807   0.29880517010041047
0.5648007937729602      0.292204431135364     0.29900197682887203
0.564963652002326       0.2916233053747243    0.29889602105371865
0.5651337450540688      0.29101718167173823   0.29819993028509606
0.5652985708445073      0.29043062326155755   0.29700426173183003
0.5654727151379361      0.2898117520846274    0.29555249862893823
0.5656415921700608      0.28921243036396266   0.29442503874115733
0.5658052019408815      0.28863257968009176   0.29388352162594455
0.5659760465340788      0.2880279035052964    0.2938705439207528
0.5661374565047455      0.2874573831999992    0.29406902443086674
0.5662998502559483      0.2868841325435218    0.294062551796474
0.5664694788295279      0.2862861410384371    0.29351510625375615
0.5666338401418034      0.285707494253716     0.29243169880301045
0.566807519957069       0.2850968689135192    0.291000006322651
0.5669759325110308      0.2845055727469784    0.2897871021793641
0.5671390778036886      0.283933529014895     0.28911240766132623
0.567309457918723       0.2833369130080243    0.28898907502656723
0.5674704034112266      0.2827740786999962    0.2891727091018022
0.5676323326842663      0.2822085324499406    0.28925344732308655
0.567801496779683       0.28161849663291666   0.2888621573563468
0.5679653936137954      0.28104758957735815   0.287914682084383
0.5681365252702846      0.2804522741591912    0.2865492244487559
0.568298222304243       0.27989052141393966   0.2853048676567378
0.5684609031187375      0.27932607714356444   0.2844642622153293
0.5686308187556091      0.27873730646948774   0.2841777800333818
0.5687954671311763      0.2781675416303538    0.28431102795449126
0.5689694340097339      0.2775663348350286    0.2844644137901093
0.5691381336269874      0.27698411865241224   0.2842028241305112
0.5693015659829368      0.2764208185650242    0.2833860118558001
0.5694722331612629      0.27583335529965874   0.282089326670403
0.5696334657170583      0.2752790911270855    0.28081308800324706
0.5697956820533899      0.2747221527899461    0.27986376728413465
0.5699651332120983      0.2741411313013204    0.27944716989215573
0.5701293171095025      0.27357890562265896   0.27951245730315843
0.5703028195098969      0.27298556483308334   0.2797029341322598
0.5704710546489875      0.2724110337724345    0.2795692675312559
0.5706340225267739      0.271855209299361     0.27889934389528515
0.570804225226937       0.2712754623549677    0.27769845664653703
0.5709649933045693      0.27072855627184156   0.2764116531360545
0.5711267451627378      0.2701789923484129    0.27535928339721644
0.5712957318432832      0.2696055835053447    0.2747977392965104
0.5714594512625244      0.26905076286452934   0.274767596669524
0.5716304055041423      0.2684721736073436    0.2749659730002931
0.5717919251232295      0.26792621667759864   0.27496425202347496
0.5719544285228528      0.26737761999471976   0.2744817924920557
0.5721241667448531      0.26680533137901935   0.27342862125737527
0.5722886377055492      0.26625151346173526   0.2721299998494253
0.5724624271692355      0.2656670764834352    0.27090684634697326
0.5726309493716177      0.2651010952772134    0.27021974961013123
0.5727942043126959      0.2645534992621006    0.2700894699719191
0.5729646940761508      0.26398236402940944   0.27026939690088164
0.5731257492170749      0.2634435157959748    0.2703463551349352
0.5732877881385353      0.26290204218000346   0.27000094863576335
0.5734570618823724      0.26233710396730386   0.26907896766357076
0.5736210683649055      0.26179043626390575   0.2678287421201299
0.5737943933504286      0.26121344544917907   0.26654574622220867
0.5739624510746479      0.26065471052526795   0.2657280147904079
0.5741252415375631      0.26011416269171816   0.26547618086775776
0.5742952668228549      0.2595502988707664    0.2656096173326557
0.5744600248468428      0.2590045905832096    0.2657478560629303
0.5746341013738208      0.2584287506132119    0.26548948934403244
0.574802910639495       0.25787105160853835   0.2646648263695903
0.5749664526438649      0.2573314254395703    0.26346575955925816
0.5751372294706116      0.25676863009838635   0.2621763470889004
0.5752985716748276      0.2562375843751083    0.2612971601294564
0.5754608976595796      0.255703943658157     0.26093129785097136
0.5756304584667085      0.25514720543121855   0.26100199677586544
0.5757947520125333      0.25460842909729114   0.26117323810276655
0.5759683640613483      0.25403980560675254   0.26103574141700414
0.5761367088488594      0.25348912974887017   0.260349740386893
0.5762997863750663      0.25295633517291105   0.25923344484291033
0.57647009872365        0.25240058580057406   0.2579302654191142
0.5766309764497028      0.2518762491307396    0.2569530113096439
0.576792837956292       0.25134932392357184   0.2564592691223464
0.5769619342852579      0.2507995128628094    0.2564416320019977
0.5771257633529197      0.25026747572313957   0.2566233950075581
0.5772968272429582      0.24971262153029683   0.2566020711861442
0.5774584565104659      0.24918900361944848   0.2561043025935435
0.5776210695585098      0.24866281902837362   0.25511656424592744
0.5777909174289304      0.24811388637585696   0.2538280708515252
0.577955498038047       0.2475826207638773    0.25272342768482914
0.5781293971501538      0.24702196134865775   0.25205087089054246
0.5782980290009567      0.2464789552255946    0.25194072051881866
0.5784613935904555      0.2459535384922002    0.2521092889761396
0.5786319930023309      0.24540551109116618   0.25216764158094995
0.5787931577916756      0.2448884068758924    0.25179653840885263
0.5789553063615565      0.24436874803331762   0.2509225002472354
0.579124689753814       0.24382654554218988   0.24967596338198664
0.5792888058847676      0.24330182826008906   0.2485167129360615
0.5794622405187114      0.24274798343599768   0.24771966465632542
0.5796304078913512      0.24221161036727956   0.2474963242789584
0.5797933080026869      0.24169264685480357   0.24762794788144757
0.5799634429363993      0.2411512731604656    0.24775079096436278
0.5801283106088078      0.24062728030785607   0.24749631518408585
0.5803024967842063      0.2400743319207651    0.24666512002003982
0.5804714156983009      0.2395387509731198    0.24546347306064906
0.5806350673510914      0.23902047593014195   0.24428470518244622
0.5808059538262587      0.23847992240049654   0.24342036660687832
0.5809674056788952      0.237969805707463     0.243112721289611
0.5811298413120678      0.23745716022651478   0.24319318376052154
0.5812995117676172      0.23692230042017398   0.24335170488569705
0.5814639149618626      0.236404645949768     0.24320445106975358
0.5816376366590981      0.23585829117197493   0.24250381904997315
0.5818060910950298      0.2353291286224272    0.2413757643981772
0.5819692782696573      0.23481709846813958   0.2401799003194587
0.5821397002666615      0.23428298176071918   0.23921646681881342
0.582300687641135       0.23377900829251308   0.23879265035739658
0.5824626587961446      0.2332725163649489    0.23879970533260456
0.5826318647735309      0.23274400009010274   0.23897424439798629
0.5827958034896132      0.23223251617014357   0.23892915679296597
0.5829690607086858      0.2316925030250373    0.23837221375136372
0.5831370506664544      0.2311695143112431    0.23734178388843022
0.5832997733629188      0.23066349216624948   0.23615018255611708
0.58346973088176        0.2301355673146085    0.23509539210104202
0.5836344211392972      0.22962458260439322   0.23453676538517929
0.5838084298998245      0.229085302615273     0.2344643593671314
0.5839771713990478      0.22856295065653015   0.23463491262721897
0.5841406456369671      0.22805746939461474   0.2346484190243209
0.5843113546972631      0.22753020952841888   0.23419876717825358
0.5844726291350284      0.22703264417142996   0.23329573474851034
0.5846348873533298      0.22653258557941625   0.23213047557930913
0.584804380394008       0.22601080903237847   0.23101419712469876
0.5849686061733821      0.225505810041627     0.23034428978268298
0.5851421504557464      0.22497275627872212   0.23016920930567975
0.5853104274768066      0.22445646820986537   0.23031660494322956
0.5854734372365629      0.22395688994062193   0.23039750494319752
0.5856436818186959      0.22343571450001726   0.23007802166762098
0.5858044917782981      0.22294396002120725   0.22929037422833362
0.5859662855184364      0.222449723519325     0.22817394748589714
0.5861353140809515      0.2219339488406277    0.2270108715995468
0.5862990753821626      0.22143479308506248   0.22622713629947513
0.5864700715057505      0.2209141572518265    0.22593011615980443
0.5866316330068075      0.22042278267631568   0.22601910459464328
0.5867941782884007      0.21992893947910144   0.22615715616708998
0.5869639583923707      0.2194136744826601    0.22599055759665823
0.5871284712350366      0.21891493826441943   0.22533755836880545
0.5873023025806927      0.21838853124795363   0.22421406404678662
0.5874708666650448      0.217878641434254     0.22303123573226855
0.5876341634880928      0.21738521489164417   0.22215833516473985
0.5878046951335175      0.21687048260919192   0.22174894140036464
0.5879657921564114      0.21638474739438857   0.22177540330020457
0.5881278729598418      0.21589655384409323   0.2219315426054009
0.5882971885856487      0.21538711117951126   0.2218644917772072
0.5884612369501516      0.2148940439032394    0.22133724564169224
0.5886346038176447      0.2143735307432209    0.22030520581654153
0.5888027034238336      0.2138693816429974    0.21912152224192002
0.5889655357687187      0.21338154409808582   0.21816470124342646
0.5891356029359804      0.21287251005881086   0.2176310382979404
0.5893004028419381      0.21237969660576955   0.21757550790117372
0.5894745212508858      0.2118595762712124    0.21773992290290786
0.5896433723985297      0.21135573743662872   0.21772992890496684
0.5898069562848696      0.21086812856237375   0.21728773781090258
0.5899777749935862      0.21035949177289562   0.21634334490956883
0.590139159079772       0.20987945136047742   0.21522071952357993
0.590301526946494       0.2093969772703582    0.21420604990287365
0.5904711296355926      0.20889353149097867   0.21356588222058967
0.5906354650633874      0.20840623303642045   0.21342229467125653
0.590809118994172       0.20789184921101056   0.21356711506152729
0.5909775056636528      0.20739360273282978   0.2136257268095866
0.5911406250718296      0.20691144306525858   0.21330009972610356
0.5913109793023832      0.20640842369133822   0.21246535885672574
0.591471898910406       0.20593375477605363   0.21138027104587567
0.5916338022989649      0.20545666537059235   0.21031900594164005
0.5918029405099006      0.20495877132894488   0.2095682328434775
0.5919668114595322      0.20447688304034947   0.2093187423486973
0.5921379172315405      0.20397424409916523   0.2094182127888778
0.592299588381018       0.20349981146327836   0.20953286521820264
0.5924622433110318      0.20302297249433748   0.20934716514703758
0.5926321330634223      0.20252543753206456   0.2086643635751441
0.5927967555545088      0.20204382766624757   0.2076235568195719
0.5929706965485855      0.20153548912721683   0.2064529887092449
0.593139370281358       0.20104306589643486   0.20561455022182268
0.5933027767528267      0.20056650873452392   0.20526674179257223
0.5934734180466721      0.2000693644392743    0.20531036381324025
0.5936346247179864      0.19960018671398053   0.20544948403406454
0.5937968151698372      0.19912861523058994   0.2053558151448466
0.5939662404440647      0.1986365101185419    0.20479222163254765
0.5941303984569881      0.19816019199543683   0.2038281477454372
0.5943038749729017      0.1976573561655346    0.20265258222583718
0.5944720842275113      0.19717029770783334   0.20172940744131176
0.5946350262208169      0.19669896837797368   0.2012714718974282
0.5948052030364991      0.1962072119825421    0.20123771000988816
0.5949701125908775      0.19573116307953198   0.201386126930344
0.5951443406482457      0.1952287335150261    0.20135397770372515
0.5953133014443102      0.19474200184600343   0.20087040454063973
0.5954769949790706      0.19427076146045877   0.19996803171130983
0.5956479233362076      0.19377918156337964   0.19881911810014768
0.5958094170708138      0.19331520393794774   0.1978770883155997
0.5959718945859565      0.19284885923447634   0.19732411632436164
0.5961416069234757      0.1923622406733511    0.19721029865530706
0.5963060519996909      0.19189120380204025   0.19734671239847412
0.5964798155788962      0.19139398664368487   0.19738374012285204
0.5966483118967976      0.19091234322090467   0.1970127731233275
0.596811540953395       0.19044622683469883   0.1962044087645486
0.5969820048323691      0.1899599447723541    0.19508470982233794
0.5971430340888122      0.1895010401220209    0.19409159924152114
0.5973050471257917      0.1890397857452953    0.19343810239703954
0.5974742949851479      0.18855841950987393   0.19322649203728984
0.5976382755832002      0.18809250728622634   0.1933310097642019
0.5978115746842425      0.18760062442600892   0.19342506847415364
0.5979796065239809      0.18712418769431527   0.19317064973393572
0.5981423711024152      0.18666315081292273   0.1924734107481663
0.5983123705032262      0.18618210927674506   0.19140536643987816
0.5984771026427333      0.18571644788140876   0.19035112676650334
0.5986511532852303      0.18522495325724306   0.18956661133244196
0.5988199366664235      0.18474883087681518   0.18928636094492504
0.5989834527863127      0.18428803443510425   0.18935842090558258
0.5991542037285785      0.18380734057057835   0.18947818818253584
0.5993155200483135      0.18335366696103694   0.18931755697306757
0.5994778201485849      0.18289767744958324   0.18872439787090797
0.5996473550712329      0.18242184385798108   0.18772015272184783
0.5998116227325768      0.18196126377858804   0.18665126584910469
0.5999852088969109      0.18147505828473096   0.18577966633992407
0.600153527799941       0.18100409847694207   0.18539582077809902
0.6003165794416672      0.18054833847067897   0.18540859522952777
0.6004868659057699      0.180072840639432     0.185548209841755
0.6006477177473419      0.17962414210753427   0.18547705217760435
0.6008095533694502      0.17917314467969847   0.18499925896839836
0.6009786238139351      0.17870246227969455   0.18407954221626635
0.6011424269971162      0.17824690791606637   0.18301404823084125
0.6013134650026737      0.1777717199902902    0.18207046340212946
0.6014750683857004      0.17732320088616474   0.18157656875066608
0.6016376555492635      0.17687234127017565   0.18149163716108418
0.6018074775352033      0.1764017495435548    0.18162721277920613
0.6019720322598392      0.1759462216284298    0.1816471288626463
0.6021459054874649      0.17546539858046817   0.18125875883846224
0.6023145114537869      0.17499963337195656   0.18042624016721517
0.6024778501588048      0.17454888109466893   0.17938287196103925
0.6026484236861994      0.17407864972562592   0.17838506692424574
0.6028095625910631      0.17363488420329676   0.17779696440237824
0.6029716852764633      0.17318885775284146   0.1776276778622665
0.6031410427842401      0.17272340784542534   0.1777378221687806
0.6033051330307128      0.17227290313376337   0.17781189019499885
0.6034785417801756      0.1717973169808505    0.17753646026539419
0.6036466832683345      0.1713366700040962    0.17680817908316923
0.6038095574951894      0.17089091703897388   0.1758062693988388
0.6039796665444209      0.1704258514592919    0.17476630904885102
0.6041445083323485      0.16997566195846558   0.17406896095093635
0.6043186686232662      0.16950053253870448   0.1738060418964522
0.6044875616528799      0.16904027312370856   0.17388927062064513
0.6046511874211897      0.16859483811262174   0.17398903864472587
0.604822048011876       0.16813020241637747   0.17379028760901208
0.6049834739800315      0.16769168770320156   0.17317552175543793
0.6051458837287234      0.16725095656277558   0.1722262828434715
0.6053155282997918      0.16679108120929032   0.17116747469898191
0.6054799056095563      0.16634596146288422   0.17039147300560203
0.6056536014223111      0.16587611871681607   0.17003052395818233
0.6058220299737618      0.16542102545622012   0.17006238517908578
0.6059851912639084      0.16498063582433098   0.17018574483274337
0.6061555873764317      0.16452121448543905   0.17007918938004998
0.6063165488664242      0.16408769651040794   0.1695707365238425
0.606478494136953       0.16365198604134717   0.16869259047997778
0.6066476742298585      0.16319730061960583   0.1676309193471827
0.60681158706146        0.16275724962683205   0.166779956730447
0.6069827347154382      0.1622982783639171    0.16631235705320563
0.6071444477468855      0.16186508062236143   0.16626512204330224
0.6073071445588691      0.1614297112358265    0.16639338734904244
0.6074770761932294      0.1609754788867598    0.1663878320943695
0.6076417405662857      0.1605358111108877    0.1660012838703282
0.6078157234423323      0.1600717813565296    0.16515012265052614
0.6079844390570747      0.15962214033866842   0.16410386881826078
0.6081478874105133      0.15918696325915888   0.16320050141941467
0.6083185705863283      0.15873302946095755   0.16263971039277333
0.6084798191396126      0.15830466322497075   0.1625193614465953
0.6086420514734332      0.15787415164947632   0.16263171354701697
0.6088115186296305      0.1574249440867166    0.16268424959652236
0.6089757185245237      0.15699019001647804   0.16239956193972976
0.6091492369224072      0.1565312905046733    0.161646489372227
0.6093174880589867      0.15608684008551577   0.1606333822405228
0.609480471934262       0.15565679218599027   0.15968669990138706
0.6096506906319141      0.155208169313253     0.15902813381018602
0.6098114747070353      0.1547848964131564    0.15881900928950368
0.6099732425626929      0.15435950989671762   0.1588964183804789
0.6101422452407271      0.15391561025106595   0.15899433536594282
0.6103059806574573      0.15348604554258075   0.15881321066224155
0.6104769508965643      0.15303802677319298   0.15818485608186073
0.6106384865131403      0.1526152267029082    0.157267650127464
0.6108010059102529      0.1521903391504749    0.15628729323955526
0.6109707601297418      0.15174706051821607   0.15551251932976914
0.6111352470879269      0.1513180479409642    0.15517594866790038
0.6113090525491021      0.15086527973450273   0.15519936157574596
0.6114775907489732      0.1504267729099959    0.15532045674511216
0.6116408616875406      0.15000247903354638   0.1552220463581706
0.6118113674484844      0.1495599195667269    0.15469875812217546
0.6119724385868974      0.1491423523268188    0.15384645633605898
0.6121344935058468      0.14872273020037086   0.15286397886568925
0.6123037832471728      0.14828490657562277   0.15201778479846917
0.6124678057271948      0.14786122547066288   0.15158490019707022
0.6126411467102071      0.1474140319360484    0.1515386516566496
0.6128092204319153      0.14698097607478683   0.1516666729200888
0.6129720268923194      0.14656200847976683   0.15164614170743437
0.6131420681751003      0.14612496709907435   0.15123762548440298
0.6133068421965773      0.14570199572017833   0.15044519186351119
0.6134809347210441      0.14525567443603613   0.1494032205795173
0.6136497599842071      0.14482341820503059   0.14852076867190733
0.6138133179860661      0.1444051767163269    0.14802639352571684
0.6139841108103018      0.1439689914451932    0.1479260116600449
0.6141454690120066      0.1435573964068546    0.14804189232740508
0.6143078109942479      0.1431436205008822    0.14807739485060945
0.6144773877988656      0.14271195864547445   0.14776826412955896
0.6146416973421794      0.142294247137489     0.14706180337446498
0.6148153253884834      0.14185342858643454   0.1460499912524249
0.6149836861734833      0.1414265571731589    0.1451223883816807
0.6151467796971792      0.14101358216528753   0.1445399905209254
0.6153171080432519      0.14058285891207184   0.14435517940359474
0.6154780017667938      0.14017653211999456   0.14444574053176076
0.6156398792708718      0.1397682508868968    0.14452733565353787
0.6158089915973266      0.13934229243938487   0.14432067221768494
0.6159728366624775      0.1389301591285392    0.14371516396537157
0.6161439165500049      0.1385004160235441    0.1427661870059658
0.6163055618150015      0.13809492690149      0.1418371533293889
0.6164681908605345      0.13768751612837277   0.1411476594252949
0.6166380547284441      0.13726256833011202   0.14084245217855057
0.6168026513350497      0.1368513724340519    0.1408798742452886
0.6169765664446454      0.13641751429100446   0.1409975788314608
0.6171452142929371      0.13599740439536065   0.14087380361032634
0.6173085948799248      0.1355909889801991    0.14036231460762325
0.6174792102892893      0.13516718154298835   0.1394737370393903
0.6176403910761229      0.13476737956672913   0.1385368572856578
0.617802555643493       0.13436569838216483   0.1377810220137397
0.6179719550332394      0.1339466998719591    0.1373843725996844
0.618136087161682       0.13354131993467744   0.1373639823944396
0.6183095377931147      0.13311355876876504   0.13749277016563682
0.6184777211632434      0.1326994122228464    0.13744892204578038
0.6186406372720681      0.1322988248466275    0.13704121860056792
0.6188107882032695      0.13188106900057858   0.13623145611781473
0.6189756718731669      0.13147685298233122   0.1352790172345806
0.6191498740460544      0.13105044472806748   0.13441914483162143
0.6193188089576379      0.13063757219763883   0.13396495029252353
0.6194824766079174      0.130238178563022     0.13390079962763712
0.6196533790805737      0.12982176882415386   0.13402544373839564
0.6198148469306991      0.12942894874154234   0.13403641406731387
0.6199772985613607      0.1290343276836933    0.13371931494066733
0.6201469850143991      0.12862276891554444   0.13299012071016747
0.6203114042061335      0.12822460845575143   0.1320612959719601
0.6204851419008579      0.1278044374761893    0.13115597509336996
0.6206536123342784      0.12739757229778736   0.13061682063073882
0.6208168155063949      0.1270040501187969    0.13048083601208255
0.6209872535008881      0.12659374007503135   0.13058759838485368
0.6211482568728505      0.12620676244733695   0.13064923900390657
0.6213102440253493      0.12581803079217765   0.13042699194338503
0.6214794660002245      0.1254125944644227    0.12979340294385242
0.6216434207137959      0.12502042157912038   0.1289055315014246
0.6218166939303572      0.12460665120522843   0.12796586865968185
0.6219846998856147      0.12420614124007398   0.12733911513112614
0.6221474385795682      0.12381883164355564   0.12711705646732607
0.6223174120958983      0.12341498317951066   0.12718664331704252
0.6224821183509245      0.12302431511088283   0.1272869568319643
0.6226561431089407      0.12261225971602582   0.12712576638425133
0.622824900605653       0.12221338148862118   0.12656227057760583
0.6229883908410614      0.12182761858956492   0.12571483352769766
0.6231591158988463      0.1214254872219203    0.12477713527106685
0.6233204063341005      0.12104624080183352   0.12411361780775282
0.6234826805498909      0.12066533361489336   0.12381233092291401
0.623652189588058       0.12026814612820041   0.12383515059822867
0.6238164313649212      0.11988398811473477   0.1239549298058526
0.6239899916447744      0.11947877300997091   0.12387689277797051
0.6241582846633236      0.11908658374051145   0.1234133148282223
0.6243213104205688      0.1187073561484008    0.12263010941960684
0.6244915710001908      0.1183120239655891    0.12169040066918774
0.6246523969572819      0.11793928155092052   0.12096341126451837
0.6248142066949094      0.11756493196059645   0.12057394176688914
0.6249832512549134      0.11717456858136449   0.12053247444915588
0.6251470285536135      0.11679707737546113   0.12065591935744124
0.6253180406746902      0.11640365846130094   0.12065712153947714
0.625479618173236       0.1160326492698489    0.12032741280869477
0.6256421794523184      0.11566007551877483   0.1196416294985072
0.6258119755537772      0.1152716671932201    0.11872171635583749
0.6259765043939322      0.114896038903506     0.11791128494373035
0.6261503517370771      0.11449992139365825   0.1173959006935345
0.6263189318189182      0.11411657967923111   0.11728882715201662
0.6264822446394552      0.11374594549474702   0.1174005133694641
0.6266527922823688      0.11335966232481712   0.11745595119201355
0.6268139053027517      0.11299532392553001   0.11721663992301809
0.6269760021036709      0.11262947204804645   0.11661574120211639
0.6271453337269668      0.1122480613677318    0.11573287626871158
0.6273093980889587      0.11187926936530049   0.11489095843905768
0.6274827809539407      0.1114903419209414    0.11429216157709174
0.6276508965576186      0.11111402987455793   0.11410551186115993
0.6278137448999926      0.11075026300775578   0.1141888463452401
0.6279838280647434      0.11037113224882025   0.11428756472557283
0.6281486439681901      0.1100045235190573    0.11413184162767108
0.628322778374627       0.10961802632578936   0.11356642013204715
0.6284916455197598      0.10924404764818758   0.11272091627777288
0.6286552454035886      0.10888251509952944   0.11187130270651666
0.6288260801097941      0.1085058191083725    0.1112304096173755
0.6289874801934687      0.10815070432921277   0.1109871474481767
0.6291498640576798      0.10779419108400366   0.11103342583011586
0.6293194827442674      0.1074226182181332    0.1111558293746045
0.6294838341695511      0.10706339137600829   0.11107634476745855
0.6296575040978248      0.10668466454153888   0.11060737303205331
0.6298259067647947      0.10631827971946045   0.10982104397875764
0.6299890421704605      0.10596416170613401   0.10896706655941539
0.6301594123985028      0.10559519097314066   0.10826226715723064
0.6303203480040145      0.10524745592691476   0.10793933026489465
0.6304822673900624      0.10489838626673388   0.10793316952045166
0.6306514215984871      0.10453457090787592   0.10806523580175208
0.6308153085456077      0.10418291781987052   0.10805803300183166
0.6309864303151049      0.10381662042481211   0.1077013257632344
0.6311481174620714      0.1034713477579212    0.10702572929893038
0.6313107883895742      0.10312479112158188   0.10618478095444706
0.6314806941394536      0.10276370001082953   0.10541055679480489
0.631645332628029       0.10241466346013557   0.1049767135660672
0.6318192896195947      0.10204679577447913   0.10489884691415448
0.6319879793498562      0.10169097790815225   0.10502568112767957
0.6321514018188138      0.10134712968477955   0.10507147055898985
0.632322059110148       0.10098896584922394   0.1048055671684439
0.6324832817789514      0.10065145748527533   0.10420637522931112
0.6326454882282913      0.10031273133797353   0.10339524260864115
0.6328149295000076      0.09995980260599693   0.10258824089377226
0.6329791035104201      0.09961869724601105   0.10208142066118842
0.6331525960238226      0.09925911245125386   0.10193077937517643
0.6333208212759212      0.09891138377648237   0.10203732342136525
0.6334837792667157      0.09857542862651433   0.10212772383225668
0.6336539720798868      0.0982254932648476    0.10195562298195811
0.6338147302705273      0.09789583810959115   0.10144567038155092
0.633976472241704       0.09756503391957691   0.10068081287698745
0.6341454490352574      0.097220367227082     0.09985296068068973
0.6343091585675068      0.09688735935664602   0.09927318104765298
0.6344801029221327      0.09654060048602475   0.09903773084288979
0.634641612654228       0.09621389015075377   0.09909838048461445
0.6348041061668596      0.0958860866032651    0.09922380007739534
0.6349738345018677      0.09554465287940019   0.0991594257114382
0.6351382955755719      0.09521475976071446   0.09875308789136715
0.6353120751522663      0.09486718964143305   0.09799750012521997
0.6354805874676566      0.09453115498714934   0.09716784528402013
0.635643832521743       0.09420656768823407   0.09653385478006579
0.635814312398206       0.09386859098668017   0.09622245806908501
0.6359753576521381      0.0935502571642925    0.09623738204564923
0.6361373866866066      0.09323090314474698   0.09637247896273922
0.6363066505434516      0.09289828409244273   0.09637795251548047
0.6364706471389928      0.09257698926384053   0.0960656000025747
0.6366439622375241      0.09223848406205906   0.09538592333771634
0.6368120100747514      0.09191129739621301   0.09456893949318217
0.6369747906506747      0.09159533806128298   0.09388621311341203
0.6371448060489746      0.09126636155429084   0.09349222192133208
0.6373095541859707      0.09094858119778507   0.09344868752244329
0.6374836208259568      0.09061390493375333   0.09358990556406675
0.637652420204639       0.09029041886387677   0.09363629809234358
0.6378159523220172      0.08997802931563868   0.09338815769485737
0.6379867192617719      0.0896528767490989    0.0927786936483095
0.6381480515789959      0.08934668494166624   0.09201577566471632
0.6383103676767562      0.08903960719910715   0.0913034352627081
0.6384799185968931      0.08871989752728464   0.09084135619542628
0.638644202255726       0.08841115274157862   0.09073704404971224
0.6388178044175492      0.08808600459955908   0.0908623012735897
0.6389861393180682      0.08777181484002877   0.0909558594582444
0.6391492069572834      0.08746848669850489   0.090793258112535
0.6393195094188752      0.08715276197203216   0.09027065051201011
0.639480377257936       0.08685554466916881   0.08954697596400911
0.6396422288775333      0.08655751814563865   0.08881444946865842
0.6398113153195071      0.08624725447733475   0.08828331081873808
0.6399751345001771      0.0859477174744462    0.0881070098524893
0.6401461885032237      0.08563607151972048   0.08819797515274662
0.6403078078837394      0.08534266989182944   0.08832624842481553
0.6404704110447915      0.0850485212282298    0.08826361130719039
0.6406402490282201      0.08474240192764318   0.0878570256535732
0.640804819750345       0.08444686975012142   0.08717973005612129
0.6409787089754598      0.08413577766183414   0.08638611677955424
0.6411473309392707      0.08383526565567598   0.08580354365831154
0.6413106856417775      0.08354523141582815   0.08556279400459672
0.641481275166661       0.08324350215262397   0.08561455746379612
0.6416424300690138      0.0829595441299331    0.08575869590274929
0.6418045687519028      0.08267491948617033   0.08576194738826164
0.6419739422571685      0.08237874176317515   0.0854459156428928
0.6421380485011302      0.0820928972304042    0.0848343205988304
0.6423114732480821      0.08179202693103282   0.08405192504239514
0.6424796307337299      0.08150148221946923   0.0834217871994127
0.6426425209580738      0.08122115760993365   0.08311001352273996
0.6428126460047944      0.08092956278958768   0.08310837991473852
0.6429775037902108      0.08064815121196378   0.08325894318555098
0.6431516800786174      0.08035207303156995   0.08331205675482084
0.6433205891057201      0.08006617020613663   0.0830578987308661
0.6434842308715187      0.07979033464879406   0.08249921930129651
0.6436551074596941      0.07950351837461277   0.08174780820131179
0.6438165494253386      0.07923368215289431   0.0811156405597128
0.6439789751715195      0.07896332756903479   0.08074535580740524
0.644148635740077       0.07868214118338202   0.08069072644968246
0.6443130290473305      0.07841086866674297   0.08083262355195066
0.6444867408575741      0.07812549058552859   0.08093462032933782
0.6446551854065137      0.07785001792438484   0.08076418982641419
0.6448183626941493      0.07758433943002087   0.08028148034894471
0.6449887748041616      0.07730812587318686   0.07956495240431989
0.6451497522916431      0.07704837647040362   0.07891051560257786
0.6453117135596609      0.07678819290315182   0.07847968647261191
0.6454809096500553      0.07651763278115734   0.07836112465496436
0.6456448384791458      0.07625674318118246   0.0784808135476068
0.645816002130613       0.07598561772227064   0.07862163229211833
0.6459777311595492      0.07573064081084221   0.07854879003863739
0.6461404439690218      0.07547529751135695   0.07816842684847371
0.6463103916008711      0.07520987364474974   0.07751071605356202
0.6464750719714163      0.07495392121975003   0.07682519371865916
0.6466490708449517      0.07468482169233238   0.07630190860084746
0.6468178024571831      0.07442518396028797   0.07612606299694727
0.6469812668081105      0.07417489096369786   0.07621728397928684
0.6471519659814147      0.07391482682239277   0.07638061047278752
0.6473132305321879      0.07367036696581238   0.07637325410238487
0.6474754788634975      0.07342562623327359   0.07607342437062078
0.6476449620171838      0.07317127307682343   0.0754753495843996
0.647809177909566       0.07292609688350347   0.0747970455371595
0.6479827123049383      0.07266837325032903   0.07422847326017668
0.6481509794390067      0.07241981640178802   0.07398821723054516
0.648313979311771       0.07218030627964407   0.0740387628190253
0.648484214006912       0.07193150001308421   0.07421325469558271
0.6486450140795224      0.07169773931111253   0.07426798575792819
0.6488067979326689      0.07146378426027684   0.07405470345060929
0.6489758166081921      0.07122069518226215   0.07352969195299763
0.6491395680224114      0.07098648034005074   0.07287177320785683
0.6493105542590072      0.07074328600155065   0.07227008205380957
0.6494721058730724      0.07051479920536592   0.07196218893240272
0.6496346412676737      0.07028618860669701   0.07194608051517368
0.6498044114846517      0.0700487639278073    0.07211347674170551
0.6499689144403258      0.06982003723890047   0.07223142315915806
0.6501427358989899      0.06957978281488601   0.07209392758202127
0.6503112900963499      0.069348215389535     0.07164138608575037
0.6504745770324062      0.06912520936609096   0.07101534003605225
0.6506450987908391      0.06889371886269546   0.0703935875132699
0.6508061859267411      0.06867635074627228   0.07003245808402529
0.6509682568431794      0.06845894795208127   0.06996332789009531
0.6511375625819944      0.0682332294857182    0.0701130472974231
0.6513016010595054      0.0680158914621175    0.07026731548845395
0.6514749580400064      0.06778766447291912   0.07021120128753465
0.6516430477592036      0.0675678064078021    0.06984131716826032
0.6518058702170967      0.06735624543721061   0.06926013388028786
0.6519759274973664      0.06713676511411577   0.06862787237629107
0.6521407175163324      0.06692547413356598   0.06820668879407593
0.6523148260382883      0.0667037269902561    0.06808523598870901
0.6524836672989403      0.06649015637394742   0.06822010021560727
0.6526472412982882      0.0662846307843086    0.06839359280138728
0.6528180501200127      0.06607147086814798   0.06839213429209394
0.6529794243192064      0.06587145501766038   0.06810383369432915
0.6531417822989366      0.06567156685284688   0.06757314130016796
0.6533113751010434      0.06546421808503106   0.06694568572672649
0.6534757006418461      0.0652647229349582    0.06648569339130043
0.6536493446856391      0.06505543100651427   0.06630692372796784
0.653817721468128       0.06485397941917402   0.06640973940673213
0.6539808309893129      0.06466023408522266   0.06659940771904943
0.6541511753328746      0.06445937429785809   0.06666291413691759
0.6543120850539054      0.06427103096295242   0.06645453231168846
0.6544739785554724      0.06408290416467152   0.06598584373879074
0.6546431068794161      0.0638878395609316    0.06537433384551437
0.654806967942056       0.06370028572981938   0.06487958348345438
0.6549780638270724      0.06350596297234504   0.06463890118989214
0.6551397250895579      0.06332377851913494   0.06468865181473483
0.6553023701325797      0.06314188418475332   0.06488074382134443
0.6554722499979784      0.0629534006509454    0.06501289807793112
0.655636862602073       0.0627722288816595    0.06489662956167187
0.6558107937091576      0.06258237487861347   0.06446895881693472
0.6559794575549384      0.06239981863435858   0.06388426995525985
0.656142854139415       0.06222442109546789   0.06337028659193408
0.6563134855462684      0.06204279244890701   0.06307972748874996
0.6564746823305909      0.061872650789115635  0.0630869706734785
0.6566368628954498      0.06170289042669852   0.06327091682784318
0.6568062782826853      0.061527081603790806  0.0634440335848163
0.6569704264086169      0.061358228410696505  0.06340132223804135
0.6571438930375385      0.061181386966726356  0.06305088418167872
0.6573120924051561      0.0610114866900937    0.06250434943641879
0.6574750245114699      0.06084838596740957   0.06197788075805213
0.6576451914401602      0.06067960072108022   0.06163604510029102
0.6578100911075464      0.060517562233516814  0.061594214506767664
0.657984309277923       0.06034799879820622   0.06177853316225587
0.6581532601869955      0.06018534261112775   0.061975145799776886
0.658316943834764       0.060029274034123274  0.0619805823079819
0.6584878623049091      0.05986789725418753   0.06169249911381384
0.6586493461525235      0.05971692371900823   0.06120540657727668
0.6588118137806741      0.059566499482012605  0.06067972914645366
0.6589815162312014      0.05941095318452917   0.060300835536187564
0.6591459514204249      0.059261774347619624  0.06021299605375992
0.6593197051126383      0.0591057914204482    0.06037388071303314
0.6594881915435478      0.05895615924825853   0.06059406902467905
0.6596514107131533      0.05881273104412981   0.06066151926720769
0.6598218647051353      0.05866455164640263   0.0604510193557046
0.6599828840745865      0.05852608434111258   0.06001576371050695
0.6601448872245742      0.058388254738862326  0.05949916112724791
0.6603141251969386      0.05824586229961827   0.05908510638385016
0.6604780959079988      0.05810945660923771   0.05894532902780884
0.6606513851220492      0.05796696556730681   0.059071524895641186
0.6608194070747957      0.057830444218694904  0.059305663233718345
0.6609821617662381      0.05769974382862486   0.05943232857447177
0.6611521512800571      0.05756485590595659   0.05930531401615529
0.6613168735325723      0.05743573204007278   0.05892104243373167
0.6614909142880775      0.0573010011526303    0.05838595638347956
0.6616596877822787      0.05717201701901512   0.05796025197732879
0.6618231940151761      0.05704862878773678   0.05779240981883747
0.66199393507045        0.05692143366395484   0.057893299444824194
0.662155241503193       0.056802821055812545  0.058123856492738143
0.6623175317164725      0.05668501169817769   0.05829584908908392
0.6624870567521284      0.0565635887157348    0.05824189995884457
0.6626513145264805      0.0564475381968505    0.057921643818707076
0.6628248908038227      0.05632661833634505   0.05741565551037754
0.6629931998198609      0.05621105328669252   0.056971211330753654
0.6631562415745951      0.05610069025096077   0.05675763367937523
0.6633265181517061      0.05598709866158302   0.056816149736083336
0.6634873601062861      0.05588136965040292   0.05704148912633735
0.6636491858414024      0.05577653483718261   0.05725340472008828
0.6638182463988955      0.05566866692562639   0.05727483117880974
0.6639820396950846      0.05556577452737864   0.057027668597317234
0.6641530678136502      0.055460036569126306  0.05656866837114652
0.6643146613096851      0.05536180925221079   0.05612596586987194
0.6644772385862564      0.05526469598569161   0.055853185558241585
0.6646470506852042      0.05516494391663071   0.05584636279129827
0.664811595522848       0.05506992830145604   0.056054613518230594
0.664985458863482       0.054971290370734865  0.05631441875225751
0.6651540549428121      0.054877368662871566  0.05639879855913731
0.6653173837608382      0.05478800577668418   0.05621985719896616
0.6654879474012407      0.054696393163832194  0.055807291761667974
0.6656490764191126      0.054611453580934916  0.05536760169986884
0.6658111892175207      0.054527571522348495  0.0550621913325522
0.6659805368383056      0.05444163635524727   0.05500923534752086
0.6661446171977864      0.0543600236408154    0.05519367413392939
0.6663180160602573      0.05427554263569033   0.055474090303731495
0.6664861476614243      0.05419536363716739   0.055620670628115085
0.6666490120012873      0.05411932806964613   0.055516354627751036
0.6668191111635269      0.05404163112831566   0.05516102962077782
0.6669839430644626      0.05396801519108847   0.05472240439486631
0.6671580934683884      0.053892030912410134  0.05437457935574844
0.6673269766110101      0.053820106983544835  0.054297849434818767
0.6674905924923278      0.05375208312794562   0.05446784439081881
0.6676614431960224      0.05368279388625591   0.0547566722170547
0.667822859277186       0.0536189683591922    0.05494273887567266
0.667985259138886       0.05355636133217481   0.05490481411454667
0.6681548938229624      0.05349268901258365   0.05460748843197058
0.6683192612457352      0.05343267599332133   0.05418988607259354
0.6684929471714979      0.053371062056871275  0.05382142473367256
0.6686613658359566      0.05331308655156017   0.05370205439826004
0.6688245172391115      0.05325858802807156   0.053840309622791356
0.668994903464643       0.053203422854158175  0.05413339957032563
0.6691558550676435      0.05315295650489381   0.05436485193104002
0.6693177904511804      0.053103795707037785  0.05439709877516421
0.6694869606570939      0.0530541696561233    0.05416710774799547
0.6696508636017036      0.05300777796793185   0.05378003915754767
0.6698220013686897      0.052961114877027975  0.0533994405705667
0.6699837045131453      0.05291869369499491   0.053232304211994166
0.6701463914381369      0.05287765333330699   0.053317401581681974
0.6703163131855054      0.05283654513019895   0.05359850072723796
0.6704809676715698      0.052798426123179046  0.053879656583785664
0.6706549406606244      0.05276019196572619   0.05398814693474242
0.6708236463883748      0.0527249823720762    0.053822066840082505
0.6709870848548214      0.05269256622953583   0.05347126642308507
0.6711577581436445      0.05266049578250667   0.0530875982628676
0.6713189968099369      0.052631870222238926  0.05288755670379855
0.6714812192567657      0.052604710361946604  0.05293483440605044
0.6716506765259711      0.05257809725166867   0.05320141818065466
0.6718148665338723      0.05255402580993351   0.05350932039394782
0.6719883750447639      0.052530423010414704  0.053682952644801774
0.6721566162943513      0.05250933820390791   0.05358832612608716
0.6723195902826349      0.05249060576111994   0.05328330658624898
0.672489799093295       0.05247282061092399   0.05290339287029667
0.6726505732814244      0.05245769139201906   0.05267031471708031
0.67281233125009        0.052444107928395475  0.05267293379413463
0.6729813240411324      0.05243167312126099   0.05291428548818785
0.6731450495708707      0.052421338418622704  0.053240648805004495
0.6733160099229858      0.05241234683645518   0.05347648761853022
0.67347753565257        0.05240554170366835   0.05346879659398318
0.6736400451626905      0.05240035335707147   0.05323007484974091
0.6738097894951875      0.0523967110006246    0.05286459279333707
0.6739742665663808      0.052394914662227274  0.05258816917814853
0.674148062140564       0.05239487098130296   0.05254041592804964
0.6743165904534432      0.0523966494211515    0.052755368867018246
0.6744798515050185      0.052400082853012106  0.053090950482680595
0.6746503473789704      0.052405466419095215  0.0533768793216434
0.6748114086303916      0.05241224030748382   0.05343681075125262
0.6749734536623492      0.05242071173361859   0.05325727874880283
0.6751427335166833      0.05243133647197155   0.0529177725211735
0.6753067461097133      0.05244336164036256   0.05262463208729154
0.6754800772057337      0.052457922687829964  0.052532364644086636
0.6756481410404498      0.052473860206484174  0.05271114691996979
0.6758109376138621      0.052491006772218955  0.05304646617600517
0.675980969009651       0.05251071143197359   0.05337804919825599
0.6761457331441361      0.052531557383117426  0.053509188193370895
0.676319815781611       0.052555456544105136  0.053374354039008254
0.676488631157782       0.05258047287580031   0.05306183061591298
0.6766521792726492      0.05260643773900563   0.05276772970623585
0.6768229622098929      0.05263542319206138   0.05265346282868579
0.6769843105246058      0.0526647268491062    0.0527980688641867
0.6771466426198551      0.0526958820481453    0.053127236362779766
0.677316209537481       0.05273021738998691   0.05349282709414709
0.677480509193803       0.052765232251499226  0.05368676020353941
0.6776541273531149      0.05280410034146295   0.053621127706763264
0.6778224782511229      0.05284362142033033   0.05334766239945706
0.677985561887827       0.052883625988641575  0.05305055149642896
0.6781558803469077      0.05292721205041693   0.0528993083334019
0.6783167641834575      0.05297007897427917   0.05300370002810976
0.6784786318005439      0.053014869971468634  0.05331670234547309
0.6786477342400068      0.053063443208433377  0.053709418783476626
0.6788115694181656      0.05311223836147147   0.053966718284721694
0.6789826394187011      0.05316501028953241   0.05398009294879515
0.6791442747967058      0.05321658186782858   0.05376985959229398
0.6793068939552468      0.05327014392803102   0.053476840723933286
0.6794767479361645      0.05332788432214121   0.05327897376914715
0.6796413346557781      0.05338558399126406   0.053327031121303874
0.6798152398783819      0.05344842158644751   0.05363577753918596
0.6799838778396818      0.05351119191397262   0.0540459750243424
0.6801472485396777      0.05357372540776494   0.054355599465577435
0.6803178540620503      0.053640838740166216  0.054439105827429675
0.6804790249618919      0.053705939650423674  0.05428075192682289
0.6806411796422698      0.05377310359631377   0.054004464110845356
0.6808105691450246      0.05384504821105309   0.05378370117629362
0.6809746913864752      0.05391649430418542   0.05378973188425946
0.681148132130916       0.05399385643025124   0.05406684617863686
0.6813163056140528      0.05407069358813124   0.05448571769701073
0.6814792118358854      0.05414683681695456   0.054845292789907275
0.6816493528800951      0.0542281607884536    0.05500450818350929
0.6818142266630005      0.05430872068820183   0.05490308850578576
0.6819884189488962      0.054395708646188166  0.05463056331522124
0.6821573439734878      0.0544819059781516    0.054404028050952916
0.6823210017367753      0.05456714305340677   0.054388851604164005
0.6824918943224396      0.05465796252528886   0.05464347456237443
0.6826533522855731      0.05474547075545283   0.05505052992877786
0.6828157940292429      0.05483518154137673   0.055449245673427905
0.6829854705952894      0.05493067563567733   0.055675165123545194
0.6831498799000317      0.05502509955130414   0.05563567686864167
0.6833236077077642      0.05512685903481584   0.05539617356532975
0.683492068254193       0.055227359587982534  0.05515871118558917
0.6836552615393174      0.055326431230811574  0.055105085193046216
0.6838256896468187      0.055431693659400934  0.055320655821472685
0.6839866831317891      0.05553281555520247   0.055720194354116695
0.6841486603972957      0.05563620780323057   0.05615290738799726
0.6843178724851793      0.0557459869800042    0.056448158218550516
0.6844818173117586      0.05585407160707818   0.05647975471194407
0.6846550806413282      0.055970141000276676  0.05628473584316038
0.6848230767095937      0.056084487251575395  0.05604279629439945
0.6849858055165552      0.056196941869668976  0.05594880104349184
0.6851557691458936      0.0563161741343871    0.05611538479298385
0.6853204655139278      0.05643344309236701   0.05650693436367703
0.6854944803849522      0.05655919758010079   0.056997597299210744
0.6856632279946726      0.056682960115765665  0.05734110109363422
0.6858267083430889      0.056804562083626615  0.05742579817803186
0.6859974235138819      0.05693333258750729   0.05727283349764393
0.6861587040621442      0.05705666245301566   0.057045417124053616
0.6863209683909427      0.057182386571243624  0.05692176874607347
0.686490467542118       0.05731547372462014   0.057044421585532065
0.6866546994319891      0.057446136741940636  0.05741221529518539
0.6868282498248504      0.05758604282653723   0.05792158245301508
0.6869965329564078      0.05772349638053002   0.05832491450858759
0.6871595488266611      0.057858330275823905  0.05848305054384836
0.6873297995192911      0.058000914638589064  0.058388133501698004
0.6874906155893903      0.058137253773807594  0.05817450351938479
0.6876524154400256      0.05827604955724208   0.058022832932873074
0.687821450113038       0.05842278857316654   0.05809436479120602
0.6879852175247461      0.058566646701972536  0.05842614434569813
0.6881562197588309      0.058718635532488433  0.05893617607684222
0.688317787370385       0.05886390392719548   0.059383437980316
0.6884803387624752      0.059011688349222366  0.05963389358935838
0.6886501249769421      0.05916779633305044   0.05962247272814614
0.688814643930105       0.05932076192612213   0.05943350506474185
0.6889884813862582      0.05948420882893691   0.059248511248396724
0.6891570515811074      0.05964448514339577   0.05927715064018994
0.6893203545146525      0.05980142510036595   0.05957265479070966
0.6894908922705743      0.05996732889308024   0.06008016031996565
0.6896519954039653      0.0601257024182357    0.06056797741499596
0.6898140823178925      0.06028665320037098   0.06088810091160059
0.6899834040541966      0.060456510487899684  0.060948525971232036
0.6901474585291965      0.06062276023044439   0.06079674108167884
0.6903208315071865      0.060800243728765     0.06060096481123799
0.6904889372238727      0.060974089575752635  0.06058299793768264
0.6906517756792547      0.06114413304175039   0.06083129667718793
0.6908218489570135      0.061323457012488564  0.0613230273844885
0.6909866549734682      0.06149890685336737   0.06185625620551157
0.6911607794929131      0.06168607049132886   0.062261972897423135
0.6913296367510542      0.06186932989388655   0.062376426245098995
0.691493226747891       0.06204852073740691   0.062258266168602264
0.6916640515671046      0.06223736488966529   0.06206667025322783
0.6918254417637875      0.06241739923832512   0.06201822836680679
0.6919878157410064      0.06260011720230235   0.06222293142450725
0.6921574245406021      0.06279267296114263   0.06269318468465741
0.6923217660788938      0.06298090004600759   0.06324886486812828
0.6924954261201758      0.06318156342154439   0.06372030660830276
0.6926638189001537      0.06337786841272172   0.0639113030668454
0.6928269444188275      0.06356965300601876   0.06384507146383323
0.6929973047598781      0.06377164285341026   0.06366071534558476
0.6931582304783979      0.06396403880710139   0.06357702484666236
0.6933201399774538      0.06415916994441512   0.06372965804936188
0.6934892842988865      0.06436468815013373   0.0641647785781463
0.6936531613590151      0.06456542947704252   0.06473192918081846
0.6938242732415205      0.06477673536036477   0.065261228542455
0.6939859505014951      0.06497798596061496   0.06553503001229669
0.6941486115420059      0.06518202334365003   0.06554615585093049
0.6943185074048934      0.06539680649927697   0.06538524964180806
0.6944831360064769      0.06560655678863359   0.06526223978621716
0.6946570831110506      0.0658299158951738    0.06536759012849738
0.6948257629543204      0.06604821281974835   0.06576592508569097
0.694989175536286       0.06626128825726425   0.06633462642966284
0.6951598229406284      0.06648547040439277   0.06691254688032612
0.69532103572244        0.06669882608137116   0.06725912860633877
0.6954832322847877      0.06691501882752734   0.06733797783935945
0.6956526636695122      0.06714258409277664   0.06720996739362844
0.6958168277929326      0.06736481412216933   0.06707020476485745
0.6959903104193433      0.06760136385244554   0.06712247115996531
0.69615852578445        0.06783240119608684   0.06747138529133055
0.6963214738882526      0.06805776840848846   0.06802846885951797
0.6964916568144319      0.06829478177353916   0.06864769045725594
0.6966524051180805      0.06852019117041035   0.06907001826975993
0.6968141372022651      0.06874848250014146   0.0692280163910917
0.6969831041088266      0.0689885923686165    0.06914759639551284
0.697146803754084       0.06922277941878052   0.06900014244971973
0.6973177382217182      0.0694689539243887    0.06899592795162601
0.6974792380668215      0.06970307458651717   0.06926328593919287
0.6976417216924611      0.06994012195592028   0.06978289572328568
0.6978114401404774      0.07018932795081884   0.07043178813068048
0.6979758913271896      0.07043235961844836   0.0709495652548771
0.698149661016892       0.07069082708389755   0.07121237020439364
0.6983181634452904      0.07094308986392854   0.07118575807886844
0.6984813986123848      0.07118899413051992   0.07104368361271007
0.6986518686018559      0.07144739793265727   0.07099861327880068
0.6988129039687963      0.0716929993611576    0.07121052269123825
0.6989749231162727      0.07194156768756718   0.07169391279038016
0.699144177086126       0.07220280320941773   0.07235415854055881
0.6993081637946752      0.07245743350100413   0.07293267585818358
0.6994814690062147      0.07272816024933877   0.07328471186418961
0.6996495069564501      0.07299225193018889   0.07332657581437568
0.6998122776453815      0.07324955773994574   0.07320254732432875
0.6999822831566895      0.07351986467258284   0.07312067246949877
0.7001470214066936      0.07378331828029926   0.0732774181148602
0.7003210781596878      0.07406329771311305   0.0737600212895311
0.700489867651378       0.07433639404028129   0.07442558881630618
0.7006533898817642      0.07460245738810249   0.07504474222141029
0.7008241469345271      0.07488185324990185   0.0754586631999656
0.7009854693647593      0.07514727372705969   0.07556076660282761
0.7011477755755275      0.07541574284728579   0.07546541019033717
0.7013173166086726      0.0756977073696149    0.07536146426356015
0.7014815903805136      0.07597239822439393   0.0754632700526167
0.7016551826553448      0.07626425713509209   0.07589168324967285
0.7018235076688719      0.0765488131182189    0.07654616329461418
0.701986565421095       0.07682607719797799   0.07720762838930734
0.7021568579956949      0.07711720832247675   0.07770493638251334
0.702317715947764       0.07739363613123162   0.07788623979266752
0.7024795576803693      0.0776731487638913    0.07783422536293472
0.7026486342353513      0.07796664593472466   0.07771748718552911
0.7028124435290293      0.07825244728551452   0.07776212144435517
0.702983487645084       0.0785523873717809    0.0781163307526146
0.703145097138608       0.07883720195895833   0.07871355834728494
0.703307690412668       0.07912513761265601   0.07940663764045974
0.7034775185091048      0.07942736687710267   0.07999905811975597
0.7036420793442377      0.07972166230131751   0.08028993956092485
0.7038159586823605      0.08003415792760435   0.08029754858680449
0.7039845707591794      0.08033868937503638   0.08018226126404678
0.7041479155746944      0.08063511385184756   0.08018278356701307
0.704318495212586       0.0809461403707307    0.08047438875119671
0.704479640227947       0.08124134249050367   0.08103531227472698
0.704641769023844       0.08153969417116093   0.08174079063126917
0.7048111326421178      0.08185279853417116   0.08239885900560044
0.7049752289990876      0.08215756370101539   0.08277660202210976
0.7051486438590475      0.0824811267910738    0.08285544117452952
0.7053167914577034      0.08279632097858929   0.08275490304505558
0.7054796717950553      0.08310300712775112   0.08271485225094041
0.705649786954784       0.08342474596037454   0.0829350936634992
0.7058146348532086      0.0837379128245389    0.08346002852703843
0.7059888012546234      0.08407026410559919   0.08422101751005506
0.706157700394734       0.084394013817234     0.08492245103138649
0.7063213322735409      0.08470902421468393   0.08536282059163923
0.7064921989747244      0.08503938357954755   0.0854993679906635
0.7066536310533769      0.08535283147119518   0.08542535582249795
0.7068160469125657      0.08566948939341401   0.08535950100913854
0.7069856975941313      0.08600164083177919   0.08551836901561231
0.7071500810143929      0.08632482842903454   0.08598850357871377
0.7073237829376446      0.08666777464397214   0.08673659264136996
0.7074922175995924      0.08700172804879737   0.08748418916818088
0.7076553850002361      0.08732655455135939   0.08800701479450261
0.7078257872232564      0.08766716243272846   0.08822798874217923
0.707986754823746       0.08799020162491177   0.08819350769271335
0.7081487062047718      0.08831650279513882   0.08811064285146555
0.7083178924081743      0.0886588734988154    0.08820522933987657
0.7084818113502729      0.08899189223715553   0.0886068633340224
0.7086529651147481      0.08934097689934979   0.08931375060589401
0.7088146842566924      0.08967209769622454   0.09006819477459857
0.7089773871791731      0.09000648218444372   0.09068613973494685
0.7091473249240305      0.09035706857883173   0.09102154926515382
0.7093119954075838      0.09069808289982738   0.09105393159294971
0.7094859843941272      0.09105977369490792   0.09096315352878391
0.7096547061193668      0.09141186272465965   0.09100737255375559
0.7098181605833023      0.09175422077677527   0.09134495842806274
0.7099888498696145      0.09211305151619753   0.09201214344063856
0.710150104533396       0.09245328173299788   0.09278139113719575
0.7103123429777136      0.09279679228513937   0.09346463037935171
0.7104818162444079      0.0931569066626818    0.09389138422945284
0.7106460222497983      0.0935070765930834    0.09399137719757118
0.7108195467581786      0.0938784477602075    0.09391398678256035
0.7109878040052551      0.09423984551443439   0.09391106175075227
0.7111507939910275      0.0945911448309125    0.09417554796398887
0.7113210187991768      0.09495930910645571   0.09478528202930987
0.7114818089847951      0.09530825640304451   0.09555337370629102
0.7116435829509498      0.09566049918812491   0.09629494929807056
0.7118125917394811      0.09602973326371365   0.0968195512989801
0.7119763332667083      0.09638866202006119   0.09700368798262653
0.7121473096163122      0.09676470729580215   0.09695951742525322
0.7123088513433853      0.09712117757070467   0.09691519300507849
0.7124713768509947      0.09748096667201808   0.09708586092598295
0.7126411371809809      0.09785799625326795   0.09760329288786206
0.712805630249663       0.09822451611706977   0.09836289216146445
0.7129794418213353      0.09861306509929575   0.09920809264770393
0.7131479861317035      0.09899107626706523   0.09981753772128528
0.7133112631807679      0.09935843053702852   0.10008538800310775
0.7134817750522087      0.09974327212127486   0.1000863240243224
0.7136428523011187      0.10010795057894374   0.10002498617724363
0.7138049133305651      0.10047596021617891   0.10013195660837797
0.7139742091823882      0.10086157632703911   0.10057344487370223
0.7141382377729073      0.10123633772419167   0.10129573542793197
0.7143115848664165      0.10163360658039043   0.10216915222232652
0.7144796646986218      0.1020200718001061    0.10286290838458621
0.714642477269523       0.10239560220836266   0.10322666415988203
0.7148125246628009      0.10278898165399696   0.10329204828849708
0.714977304794775       0.10317130300099318   0.10322709684302761
0.7151514034297388      0.1035764440869421    0.10329357320804686
0.7153202348033989      0.10397049859929124   0.10368266993493114
0.715483798915755       0.10435335311286145   0.10437364770344748
0.7156545978504877      0.10475428540172904   0.10524904260436417
0.7158159621626895      0.10513413855386262   0.10597831977418436
0.7159783102554278      0.10551734912905723   0.10643054717999788
0.7161478931705427      0.10591874702007946   0.10656681942473978
0.7163122088243535      0.10630875393964664   0.10651528138944748
0.7164858429811545      0.10672202388958807   0.10653224749957806
0.7166542098766515      0.10712387519636808   0.10684211995937554
0.7168173095108445      0.10751419905681772   0.10747375020498709
0.7169876439674142      0.10792292832943819   0.1083454209333245
0.7171485438014531      0.10831003734486058   0.10913337654880359
0.7173104274160282      0.10870050691067756   0.10968022098631287
0.71747954585298        0.10910948612996263   0.10990456004306265
0.717643397028628       0.10950675420217935   0.10988459278762133
0.7178144830266524      0.10992263612181233   0.10986100345729355
0.717976134402146       0.11031658539851688   0.11006740607316227
0.718138769558176       0.11071390880917592   0.11060548801514453
0.7183086395365826      0.11112994746793364   0.11144306408857797
0.7184732422536853      0.11153409429604082   0.11230141956500089
0.7186471634737781      0.11196219324039865   0.11298935419156851
0.718815817432567       0.11237837366794551   0.11330068709522216
0.7189792041300517      0.11278253351296183   0.11332451797109203
0.7191498256499131      0.11320561048867833   0.11328164601704788
0.7193110125472437      0.11360624583075996   0.1134211235599924
0.7194731832251107      0.11401025526075638   0.1138826108010995
0.7196425887253544      0.11443327792902727   0.11467803650754413
0.719806726964294       0.11484410659670864   0.1155598645599596
0.7199801837062239      0.11527927861266495   0.11633314593078994
0.7201483731868495      0.11570223073876823   0.11674402865671539
0.7203112954061714      0.11611286549123265   0.1168303116383268
0.7204814524478698      0.11654270481778464   0.11678371595609748
0.7206463422282642      0.11696017802886823   0.11686433637579248
0.7208205505116487      0.11740235835939056   0.11729054538704886
0.7209894915337294      0.11783214919670462   0.11805084826857994
0.721153165294506       0.11824944832098014   0.1189413328651522
0.7213240738776593      0.11868614101053532   0.11976164018498468
0.7214855478382818      0.11909961084978966   0.12023859216582988
0.7216480055794404      0.11951646062069016   0.12039303256369575
0.7218176981429758      0.1199527901297513    0.12036006639034742
0.7219821234452073      0.12037646275205191   0.12039268329682914
0.7221558672504287      0.1208250881292748    0.12073435060482046
0.7223243437943463      0.12126103074859      0.12143016316507747
0.7224875530769599      0.12168419986048927   0.12231398315668511
0.7226579971819501      0.12212702024760083   0.12319507637555384
0.7228190066644095      0.12254616118883939   0.1237662609428042
0.7229809999274053      0.1229686732755503    0.124005261546648
0.7231502280127775      0.12341091712673163   0.12400468480810814
0.7233141888368458      0.12384023010403834   0.12399892568461025
0.7234874681639043      0.12429482824387866   0.12425073988415045
0.7236554802296589      0.12473647049605052   0.12486390322407699
0.7238182250341094      0.1251650710341996    0.1257201734531167
0.7239882046609366      0.12561356354624814   0.1266484368451917
0.7241529170264598      0.12604896970467183   0.1273288130481685
0.7243269478949732      0.1265098697782743    0.12767146690098494
0.7244957115021825      0.12695765871343645   0.12770581889318153
0.724659207848088       0.12739225318299302   0.12768442328636456
0.7248299390163699      0.1278468933034701    0.12787361800336605
0.7249912355621211      0.12827716863660027   0.12839281786409643
0.7251535158884086      0.12871080650340413   0.1292085949415611
0.7253230310370727      0.12916456154962513   0.13016204008929202
0.7254872789244329      0.12960497609445593   0.13092341700271937
0.7256608453147833      0.13007118237261658   0.1313678614432483
0.7258291444438296      0.13052402422152065   0.13146305823126647
0.7259921763115719      0.13096342323235957   0.13143367360095312
0.7261624430016909      0.13142308178480425   0.1315503757442839
0.726323275069279       0.13185797807884503   0.13197977568433064
0.7264850909174034      0.1322962228930021    0.13273473880947037
0.7266541415879045      0.13275479314637356   0.13369429134480798
0.7268179249971016      0.1331997822642149    0.13453021186630035
0.7269889432286755      0.13366519959726675   0.13508223630036803
0.7271505268377184      0.13410567845829613   0.13526144521932795
0.7273130942272978      0.1345495047814616    0.13524776532723579
0.7274828964392537      0.13501378938516048   0.13529063302098643
0.7276474313899056      0.13546435412305258   0.13561719615576928
0.7278212848435477      0.1359411594396615    0.13634572277141982
0.7279898710358857      0.13640422128372945   0.13729273405140502
0.7281531899669198      0.13685346846776236   0.13818192556303763
0.7283237437203306      0.13732329618424252   0.13883144995869107
0.7284848628512106      0.13776776565579785   0.13909456874803014
0.7286469657626268      0.13821556217345318   0.1391140949639181
0.7288163034964197      0.13868399470057713   0.13911940939859158
0.7289803739689087      0.1391384842017208    0.13936021129199175
0.7291537629443876      0.13961945118362745   0.13999868972176593
0.7293218846585627      0.1400864524336159    0.14091180156796088
0.7294847391114339      0.14053942192870128   0.1418406752722332
0.7296548283866816      0.14101313780802066   0.14258882993838434
0.7298196504006254      0.14147278489700438   0.14295597021390763
0.7299937909175591      0.14195905469654518   0.1430233372862434
0.730162664173189       0.14243123326066326   0.14301396118263737
0.7303262701675148      0.14288925733657426   0.14319999517417623
0.7304971109842174      0.1433681309566091    0.14376219999673548
0.7306585171783891      0.14382111146293305   0.14460170560629024
0.7308209071530973      0.14427738861066963   0.14554932491512224
0.7309905319501819      0.1447545611995311    0.14637677279990527
0.7311548894859626      0.14521746304917574   0.14683942680896064
0.7313285655247335      0.14570718753401787   0.14697254839331708
0.7314969743022003      0.14618261973492108   0.146956863096962
0.7316601158183631      0.1466437015585154    0.1470741951749339
0.7318304921569027      0.14712577011956776   0.14753969650140686
0.7319914338729114      0.14758164388399608   0.14831582737652535
0.7321533593694564      0.14804078832767498   0.1492639354082858
0.732322519688378       0.1485209595040666    0.1501633837826412
0.7324864127459958      0.14898667216996744   0.15072996306599667
0.7326575406259901      0.1494734540453508    0.150948803706279
0.7328192338834536      0.14993387210753634   0.15094891865861867
0.7329819109214534      0.15039754962316723   0.15099708651809954
0.7331518227818299      0.15088233268871806   0.1513393395619049
0.7333164673809024      0.15135259722161354   0.1520367926134712
0.733490430482965       0.1518499818656632    0.153033742935783
0.7336591263237235      0.152332781171634     0.15398309347327577
0.7338225549031782      0.15280094475541067   0.15463988397633818
0.7339932183050095      0.1532902876401323    0.15494666493162018
0.73415444708431        0.15375299848654111   0.15497933536589778
0.7343166596441468      0.15421893808347498   0.1549929705032826
0.7344861070263603      0.15470608655602588   0.1552471189645453
0.7346502871472698      0.15517850223003365   0.15585610743949388
0.7349920171337648      0.1561630704887735    0.15779642632717475
0.7351549812350564      0.15663318038232496   0.15854374252757347
0.7353251801587247      0.1571245579228845    0.15895356518780587
0.7354859444598623      0.15758906318298424   0.15903943202738846
0.735647692541536       0.15805676378162928   0.15903464599338335
0.7358166754455864      0.15854575554010628   0.15920305357488912
0.7359803910883329      0.15901986071356763   0.15970979865565854
0.736151341553456       0.15951528368753937   0.16058657689147116
0.7363128573960482      0.15998370216586913   0.1615508167158814
0.7364753570191768      0.16045529839850922   0.16239544140547577
0.736645091464682       0.16094823207734812   0.16293968621326954
0.7368095586488833      0.16142619464984986   0.1631168271380489
0.7369833443360747      0.16193157920388188   0.16311699115728712
0.737151862761962       0.1624219738839014    0.16322090929828162
0.7373151139265454      0.16289734182518678   0.16363428542716454
0.7374855999135055      0.16339408621276666   0.16443607346757122
0.7376466512779347      0.1638636250579116    0.16539077843010885
0.7378086864229003      0.1643363043797254    0.16629376437249838
0.7379779563902424      0.1648303737390168    0.16694043436174202
0.7381419590962807      0.1653093401730527    0.16720378294828356
0.7383152803053089      0.16581580414512848   0.16722979734612292
0.7384833342530333      0.16630714734280755   0.16727992608675854
0.7386461209394537      0.16678333815559138   0.16759471089520897
0.7388161424482507      0.16728094593110765   0.16830182436126
0.7389808966957437      0.16776337816065962   0.16924971195701494
0.7391549694462268      0.1682733465247977    0.1702591561949155
0.739323774935406       0.16876812178970707   0.17097478703311955
0.7394873131632812      0.16924767972960592   0.17130557422356055
0.7396580862135329      0.1697486925270038    0.17136369202687216
0.7398194246412539      0.17022222688599187   0.17138282509232766
0.7399817468495112      0.1706988392024928    0.17161545508674672
0.7401513038801453      0.17119689177061098   0.17222920654342816
0.7404892019217955      0.172189998064951     0.17416489720554676
0.7406575429328115      0.1726850349630026    0.17496723541171305
0.7408206166825236      0.17316474538477247   0.17539482249886268
0.7409909252546125      0.17366590244100116   0.17551157948230875
0.7411517992041705      0.17413944435137918   0.17551508681340236
0.7413136569342649      0.17461602045923447   0.17566893053671415
0.7414827494867358      0.1751140403958683    0.17617683416934865
0.7416465747779027      0.17559667825593817   0.1770078693510751
0.7418176348914464      0.1761007611592013    0.1780355947965369
0.7419792603824591      0.17657715827372336   0.17889294180426293
0.7421418696540083      0.1770565627882568    0.17944558329200525
0.742311713747934       0.17755740549012713   0.17965885131956796
0.7424762905805559      0.1780428149514246    0.1796712127809311
0.7426501859161677      0.1785558073969741    0.1797672996219055
0.7428188139904757      0.17905335100165692   0.18018009474126723
0.7429821748034796      0.17953543078848017   0.1809352882251065
0.7431527704388601      0.18003893497252124   0.1819456768763849
0.7433139314517099      0.18051465665634348   0.18285471764149977
0.7434760762450959      0.18099333803696047   0.18350010184719942
0.7436454558608587      0.18149343088488978   0.18380139162134249
0.7438095682153175      0.18197801711928496   0.18384269076046245
0.7439829990727664      0.18249015959612763   0.18389029846142618
0.7441511626689112      0.1829867807825537    0.1842039378805271
0.7443140590037521      0.18346787097631015   0.18486604820131464
0.7444841901609697      0.18397034673059062   0.18583627342153733
0.7446490540568832      0.18445727722548558   0.18680480789605775
0.744823236455787       0.18497173502942127   0.1875772849176517
0.7449921515933866      0.18547063321303225   0.1879452701018965
0.7451557994696824      0.1859539652595946    0.1880185717190459
0.7453266821683547      0.18645864872217782   0.1880431456083048
0.7454881302444962      0.18693544601757817   0.1882718203981196
0.7456505621011742      0.18741512069696706   0.18884393906294675
0.7458202287802287      0.1879161136303785    0.18975999694695003
0.7459846281979793      0.1884015038610615    0.19074397693178521
0.7461583461187199      0.18891435440687865   0.19159681732915032
0.7463267967781566      0.18941159623067494   0.192060631309435
0.7464899801762893      0.18989322813805054   0.19219115858467714
0.7466603983967985      0.1903961416992225    0.1922031358079273
0.7468213819947771      0.1908711394534972    0.19235617770151398
0.7469833493732919      0.19134896142978522   0.19282783211394885
0.7471525515741835      0.19184803616130375   0.19366876031092492
0.747316486513771       0.1923314796841402    0.1946486546024486
0.7474897399563487      0.19284229463662145   0.19557346798039693
0.7476577261376223      0.1933374659971108    0.1961417489481905
0.7478204450575919      0.1938169977414742    0.19634945527072248
0.7479903987999382      0.19431772451730317   0.19637053431346035
0.7481550852809805      0.19480280320842508   0.19646321766127392
0.748329090265013       0.19531518362066824   0.19688055224261405
0.7484978279877414      0.19581190388615244   0.19765774800705035
0.748661298449166       0.19629297118642935   0.19861948631409382
0.7488320037329671      0.19679516751683784   0.19957251445163424
0.7489932743942374      0.19726944986435507   0.20020010387856105
0.749155528836044       0.1977464638822515    0.20048775750853784
0.7493250181002271      0.1982445679176366    0.20053704028396582
0.7494892401031065      0.1987270104895339    0.20058565002883066
0.7496627806089758      0.19923662686389312   0.20090392947923652
0.7498310538535413      0.19973057061818528   0.2015866613959596
0.7499940598368028      0.20020885409892222   0.20250830003147288
0.7501643006424408      0.20070814973658327   0.2034964460330556
0.7503251068255481      0.20117956584379693   0.20421118711720448
0.7504868967891917      0.2016536549203648    0.2045927932125008
0.750655921575212       0.202148711248461     0.20469222071814458
0.7508196790999283      0.20262810693150057   0.2047135212217787
0.7509906714470211      0.20312843025756466   0.20493049369393657
0.7511522291715833      0.20360090481630802   0.2054718432402485
0.7513147706766817      0.2040760117829474    0.20631866406745328
0.7514845470041568      0.2045719972903037    0.20732212293663732
0.7516490560703278      0.2050523264153958    0.20814812454466652
0.7518228836394891      0.20555956899614253   0.2086673905316454
0.7519914439473463      0.2060511441875236    0.20882732819450348
0.7521547369938996      0.20652701564827977   0.20884321601180958
0.7523252648628295      0.2070236653462198    0.2089872853318132
0.7524863581092285      0.20749254445948276   0.20943129053082665
0.7526484351361639      0.20796399372594715   0.21020216274423975
0.7528177469854758      0.20845616707136175   0.21119147819698192
0.7529817915734839      0.20893271026163998   0.212073908459602
0.753155154664482       0.20943597551707235   0.21269307200156073
0.7533232504941763      0.2099236020912558    0.21293197849866635
0.7534860790625665      0.21039561587604916   0.21296230822560952
0.7536561424533332      0.21088824570538334   0.21304466994254742
0.7538209385827962      0.21136526370639788   0.2134000359809437
0.7539950532152492      0.2118688695418134    0.21414977275281752
0.7541639005863982      0.21235685534737203   0.21511373090545294
0.7543274806962431      0.21282925020087493   0.216022819017548
0.7544982956284647      0.21332214485739792   0.21670202412868786
0.7546596759381556      0.21378744016126233   0.21700205196559305
0.7548220400283828      0.2142551978939556    0.21706270357835422
0.7549916389409866      0.21474339148468602   0.2171069843478134
0.7551559705922865      0.21521602005828383   0.21737433681459273
0.7553296207465765      0.21571501118941547   0.2180261889436911
0.7556611192712445      0.2166663103747684    0.2198747403110721
0.7558314697253031      0.21715449912543133   0.22063615637547102
0.7559923855568309      0.21761522744446532   0.2210243358309068
0.756154285168895       0.2180783519794231    0.22113458405198874
0.7563234196033357      0.21856171530886595   0.22115764302180232
0.7564872867764726      0.21902957393941422   0.22134087738414834
0.7566583887719861      0.21951760775126328   0.22187386037468312
0.7568200561449687      0.21997827347912216   0.2226826688844088
0.7569827072984877      0.22044128755263392   0.22361870257704422
0.7571525932743832      0.22092440309920824   0.22446747197751854
0.757317211988975       0.22139205223110278   0.2249785667123655
0.7574911492065567      0.2218856450081478    0.2251704885235239
0.7576598191628344      0.22236376521543713   0.22519281231212224
0.7578232218578081      0.22282645492264325   0.22531302732852598
0.7579938593751585      0.22330909915022729   0.2257475407583656
0.7581550622699782      0.2237645527582609    0.2264803744347378
0.7583172489453341      0.2242222453387208    0.22739546724987786
0.7584866704430667      0.22469974545761934   0.2282915718802799
0.7586508246794953      0.22516186502088592   0.2288887145474174
0.758824297418914       0.22564963870121063   0.22916029688639214
0.7589925028970286      0.22612202744076665   0.22920092395075595
0.7591554411138394      0.22657907846911018   0.2292690785651809
0.7593256141530268      0.22705584472735976   0.2296018777087224
0.7594905199309101      0.2275172837164789    0.23026323630137494
0.7596647442117836      0.22800418148146817   0.23121134389294207
0.7598337012313532      0.2284757479105498    0.23212648660238905
0.7599973909896187      0.22893203330551756   0.23277786548696586
0.760168315570261       0.22940787045063604   0.23310793553868528
0.7603298055283724      0.22985685923210317   0.2331745494001723
0.7604922792670201      0.23030800562625897   0.23321187237632274
0.7606619878280445      0.23077861656656395   0.23346056986227734
0.7608264291277649      0.23123400636819086   0.23403086372059084
0.7611686814718818      0.23217984547370343   0.23585390268909318
0.7613319067519843      0.23262998322647288   0.2365740809684242
0.7615023668544634      0.23309941168891554   0.23699057914660446
0.7616633923344118      0.23354223196770577   0.23710455412059991
0.7618254015948965      0.23398713802255058   0.2371263379192077
0.7619946456777579      0.23445124244092758   0.23729574463594522
0.7621586224993152      0.23490024426527342   0.237765666110858
0.7623319178238627      0.23537405113399756   0.23858374778218763
0.762499945887106       0.235832752923713     0.2395098209806512
0.7626627066890456      0.2362764089280434    0.2402911176687724
0.7628327023133618      0.23673907919250198   0.24080161679515272
0.7629974306763738      0.23718671913183095   0.2409829195750597
0.7631714775423761      0.23765893226660442   0.24100979948312226
0.7633402571470744      0.238116112917367     0.2411311943119096
0.7635037694904687      0.23855832349814177   0.24152817711458807
0.7636745166562398      0.23901935622758383   0.2422695504855973
0.7638358291994799      0.23945421056609753   0.24314050636204318
0.7639981255232563      0.23989102095770065   0.24396061936199617
0.7641676566694093      0.2403465519771012    0.24455113370799214
0.7643319205542586      0.24078719153526826   0.24480222484257444
0.7645055029420977      0.2412520319732054    0.24484826659403622
0.764673818068633       0.24170184372146805   0.24491908624452388
0.7648368659338642      0.24213682037950182   0.24522391255768985
0.7650071486214722      0.24259031009588433   0.24587262474307495
0.7651679966865493      0.24301792949226653   0.24670346029085125
0.7653298285321628      0.24344742944250036   0.24754869764774529
0.7654988952001529      0.2438953379385259    0.24821806785696923
0.765662694606839       0.24432851417958132   0.2485514676141493
0.7658337288359016      0.24478000085616425   0.2486371153798115
0.7659953284424335      0.24520580540295336   0.24866885660666527
0.7661579118295018      0.2456334350859783    0.24887091132539427
0.7663277300389467      0.24607926701407967   0.2493978290714406
0.7664922809870875      0.24651045924145967   0.25018015489312556
0.7666661504382186      0.24696519640706885   0.25109543481723345
0.7668347526280457      0.24740529482193188   0.2518238126765557
0.7669980875565687      0.24783082878962595   0.2522328849871896
0.7671686573074685      0.24827434937835402   0.2523687299037078
0.7673297924358374      0.24869252355336383   0.252390045426821
0.7674919113447426      0.24911244747190991   0.252524993199297
0.7676612650760244      0.24955024529736758   0.25295439087869576
0.7678253515460024      0.24997357748807927   0.25366288505291257
0.7679987565189703      0.2504200367282459    0.25456235612089373
0.7681668942306343      0.25085203197883765   0.25534124241672035
0.7683297646809943      0.2512696414858634    0.2558309472488912
0.7684998699537311      0.2517049002402648    0.25603347907147483
0.7686647079651638      0.2521257974843043    0.2560627859507091
0.7688388644795866      0.252569537654173     0.25615986963996823
0.7690077537327055      0.2529989182683293    0.25651985164203633
0.7691713757245202      0.25341402048117423   0.2571656494575543
0.7693422325387118      0.2538465398382618    0.2580260299457104
0.7695036547303724      0.2542542908319215    0.25880170327351215
0.7696660607025694      0.2546636541790404    0.25935784836211356
0.7698357014971431      0.2550903136820809    0.2596276163554633
0.7700000750304128      0.2555028040664621    0.25967913275440996
0.7701737670666725      0.25593768756188895   0.2597346986121428
0.7703421918416282      0.2563584045881433    0.260008056168804
0.7705053493552801      0.2567650402378064    0.2605675759200102
0.7706757416913086      0.2571887309161061    0.2613807917188052
0.7708366994048061      0.25758798623492407   0.2621726041828423
0.7709986408988401      0.257988646981132     0.2627928481659308
0.7711678172152507      0.2584062302099465    0.2631401456296237
0.7713317262703574      0.2588098551765111    0.2632310405594622
0.7715028701478406      0.2592302860124306    0.2632598387684262
0.7716645794027931      0.25962658750340634   0.26343551430406276
0.7718272724382819      0.26002436184097966   0.26388445229565183
0.7719972002961473      0.2604388156028708    0.2646214756414325
0.7721618608927089      0.26083943422920947   0.26543293619107994
0.7723358399922604      0.26126166292025954   0.26615823755865076
0.772504551830508       0.2616700616931073    0.2665747543676293
0.7726679964074517      0.26206472254818947   0.2667121468665824
0.7728386758067718      0.262475810861291     0.26673515585282553
0.7729999205835612      0.26286319312376405   0.2668514883457844
0.773162149140887       0.26325197137695894   0.267213779793479
0.7733316125205895      0.2636570468767222    0.26787540947367217
0.7734958086389878      0.2640485128050472    0.2686639683545017
0.7736693232603764      0.26446110048382476   0.2694277808308598
0.773837570620461       0.26486008438995795   0.26991683214420004
0.7740005507192416      0.26524555981125364   0.2701139539408356
0.7741707656403987      0.2656470729858827    0.2701481047289141
0.7743315459390251      0.26602531925653183   0.27021529905331054
0.7744933100181878      0.26640488354008      0.2704897394845735
0.7746623089197271      0.26680035163381377   0.27106262692448957
0.7748260405599626      0.26718244476196784   0.27181015468404734
0.7749970070225747      0.2675803146216344    0.27258968028614644
0.7751585388626558      0.2679551851121753    0.27313759801913534
0.7753210544832734      0.2683313114027358    0.2734197072630913
0.7754908049262674      0.2687230771424003    0.2734913630158572
0.7756552881079577      0.2691016052102274    0.27352305012697015
0.775829089792638       0.26950041666335467   0.2737321105340592
0.7759976242160144      0.26988599710154726   0.2742195294439633
0.7763313933625358      0.27064626410854825   0.27569581307347146
0.7764924607244539      0.2710115482497017    0.2762938123303773
0.7766545118669084      0.27137800746777146   0.27664261427762527
0.7768237978317396      0.27175969156200896   0.2767581848865317
0.7769878165352668      0.27212838824033414   0.2767780862491481
0.7771611537417841      0.27251670004586964   0.27691711684944115
0.7773292236869973      0.27289195218909934   0.27731268318707564
0.7774920263709068      0.2732543368169304    0.27793950028528425
0.7776620638771927      0.2736316584754699    0.27870631680555896
0.7778268341221746      0.27399615060254845   0.27935999412991047
0.7780009228701468      0.2743800320919966    0.27979326226867235
0.778169744356815       0.2747510934388875    0.2799453844370745
0.7783332985821791      0.27510944209241445   0.2799668110456833
0.7785040876299198      0.27548244574321407   0.28006113191918103
0.7786654420551299      0.27583371762537884   0.2803688997923738
0.7788277802608761      0.27618602321947755   0.2809247492417531
0.7789973532889991      0.27655283935648173   0.2816604461348726
0.779161659055818       0.27690709759259186   0.2823373278831699
0.7793352833256271      0.27728019860994874   0.2828327816156903
0.7795036403341322      0.2776407515667658    0.2830416253116108
0.7796667300813334      0.2779888664526211    0.2830780601080756
0.779837054650911       0.27835120404011576   0.283130861460807
0.7799979445979579      0.2786923226975349    0.2833607278070403
0.7801598183255413      0.2790343978982277    0.2838357316259018
0.7803289268755011      0.27939054847021905   0.2845242763540162
0.7804927681641569      0.2797344197614643    0.2852114932322459
0.7806638442751896      0.2800922258547609    0.2857609548705157
0.7808254857636914      0.2804291228610693    0.28603306046213994
0.7809881110327295      0.2807669131863439    0.28610723967601165
0.7811579711241441      0.28111848812557566   0.2861308761322684
0.7813225639542549      0.28145794583344264   0.28628403076036624
0.7814964752873558      0.2818153187030302    0.2867039190850182
0.7816651193591526      0.2821605848777601    0.28733648869897566
0.7818284961696457      0.28249385909014657   0.2880162138722169
0.7819991078025152      0.28284061866816046   0.2886058360228532
0.7821602848128539      0.28316700463656175   0.28893488445698
0.782322445603729       0.2834942045311469    0.28904912799352955
0.7824918412169806      0.28383473673132414   0.2890672082458893
0.7826559695689284      0.284163443012063     0.28916519601800184
0.7828294164238662      0.28450948450915303   0.28949966802600446
0.7829975960175001      0.2848437110705787    0.2900636817191184
0.7831605083498299      0.28516623994286555   0.2907217182675933
0.7833306555045365      0.28550178299376533   0.29134260142707274
0.783495535397939       0.28582545223729433   0.2917373665611011
0.7836697337943317      0.28616605672769313   0.29190284278242956
0.7838386649294202      0.28649502725516324   0.2919254456857719
0.784002328803205       0.2868124841270159    0.29199123579758113
0.7841732274993662      0.2871426518489766    0.29225973762807567
0.7843346915729967      0.2874533485626168    0.29274010069913453
0.7844971394271637      0.2877647162549929    0.2933638966331699
0.7846668221037072      0.2880886395539949    0.2939975550785288
0.7848312375189466      0.2884012266870681    0.29443945490822504
0.7850049714371763      0.288730157306998     0.2946572242306298
0.7851734380941018      0.28904776539437493   0.2946970897706768
0.7853366374897235      0.28935417299446714   0.2947318608515421
0.7855070717077217      0.28967282689335283   0.29492880909760577
0.7856680713031893      0.2899725839549901    0.2953351929657142
0.7859992728775735      0.2905853800334196    0.29654634979550276
0.78616322381465        0.2908868023592168    0.2970313804496403
0.7863364932547166      0.2912039687339222    0.2973079847147991
0.7865044954334792      0.2915101287883973    0.2973786308337365
0.7866672303509379      0.2918054062887173    0.29739519098605727
0.7868372000907731      0.29211245965690774   0.29752620091601756
0.7870019025693045      0.29240867762562545   0.29786474573864435
0.7871759235508258      0.2927202405028367    0.2984302810109218
0.7873446772710431      0.2930209821346994    0.2990492748575833
0.7875081637299566      0.2933110281102471    0.2995533152415155
0.7876788850112466      0.2936125321246814    0.29986629737253334
0.7878401716700059      0.29389607869877166   0.2999654697225236
0.7880024421093016      0.29418008222772063   0.2999785820993215
0.7881719473709738      0.2944753823128804    0.30006089706494715
0.788336185371342       0.29476017238888375   0.30032746035058594
0.7885097418747005      0.2950596914212041    0.30082653748870997
0.7886780311167547      0.2953487149070273    0.30142072247906526
0.7888410530975052      0.2956273701540834    0.3019457356841485
0.7890113099006322      0.2959170002118968    0.30230960103501175
0.7891721320812285      0.2961892722054221    0.30245185820804865
0.789333938042361       0.2964619243773019    0.30247424579213966
0.7895029788258703      0.29674538794297434   0.3025178895925906
0.7896667523480755      0.29701855744546574   0.30271356814673156
0.7898377606926574      0.2973022486648127    0.30313089397648724
0.7899993344147085      0.29756895655629995   0.3036636558581688
0.7901618919172957      0.297835980822794     0.30420211608068565
0.7903316842422599      0.29811348731459125   0.30462375662853913
0.7904962093059198      0.2983810168662781    0.3048290635126045
0.79067005287257        0.29866223399143543   0.30487760756984555
0.7908386291779161      0.2989334912184581    0.3048987247013536
0.7910019382219582      0.2991949197622459    0.3050371507179337
0.7911724820883772      0.29946650684143344   0.30538360421712274
0.7914956843566896      0.29997720391713645   0.3063988542342773
0.7916650122034905      0.3002426704694234    0.3068515135227081
0.7918290727889876      0.30049850564560565   0.3071031964917489
0.7920024518774746      0.30076740108822414   0.3071835062359252
0.7921705637046577      0.30102668236379704   0.3071937562546606
0.7923334082705369      0.30127648159674686   0.3072810486541113
0.7925034876587926      0.30153595012013606   0.3075538478930095
0.7926682997857446      0.3017859887496505    0.3079946408701931
0.7928424304156865      0.3020486718594858    0.3085423529645274
0.7930112937843243      0.30230194245376774   0.309004646376588
0.7931748898916584      0.30254593403178415   0.3092843618744213
0.793345720821369       0.30279926571932264   0.3093901796545839
0.7935071171285488      0.30303724386698533   0.3094002759137633
0.793669497216265       0.3032753350361182    0.3094512171071039
0.7938391121263577      0.30352260005538056   0.30966034190744374
0.7940034597751466      0.3037607868647483    0.31004151296145277
0.7941771259269255      0.3040109802020171    0.310557128745711
0.7943455248174003      0.3042521128746868    0.3110295312373981
0.7945086564465713      0.3044843192902635    0.3113463904299216
0.7946790228981189      0.30472536832167557   0.31149153251676354
0.7948399547271356      0.30495170091116697   0.31151251306622957
0.7950018703366888      0.30517807475181585   0.31153606974973375
0.7951710207686185      0.30541312394120745   0.3116841752943815
0.7953349039392443      0.3056394491096304    0.312000165574249
0.7956677053027181      0.3060947920468519    0.31292253501442785
0.7958303724537261      0.3063152745727568    0.31328021841132747
0.7960002744271106      0.30654389596640974   0.3134785517338704
0.7961649091391914      0.30676397180556364   0.31352635833527764
0.796338862354262       0.30699497980167517   0.313535822124246
0.7965075483080286      0.30721749634374285   0.3136359887270713
0.7966709670004914      0.30743165882935924   0.3138920211842659
0.7968416205153308      0.30765382511964046   0.3143089175552856
0.7970028394076394      0.307862321617573     0.31475067049251343
0.7971650420804841      0.30807072958840465   0.31512666735161526
0.7973344795757058      0.30828697503613034   0.31536249166371655
0.7974986498096233      0.30849507616358696   0.31543794239555445
0.797672138546531       0.30871346776020514   0.3154434745390442
0.7978403600221347      0.3089237350188745    0.31550218412820685
0.7980033142364342      0.30912601545894863   0.31569768071914783
0.7981735032731108      0.30933580201897065   0.31605784711794516
0.7983342576872563      0.30953257445540977   0.3164747386552585
0.7984959958819381      0.3097291932221257    0.3168612687252571
0.7986649688989966      0.3099331516717383    0.3171335455322754
0.798828674654751       0.31012933323123254   0.3172432961386828
0.7989996152328822      0.3103326936449106    0.31725642653656655
0.7991611211884826      0.31052342984400705   0.31727955385682627
0.7993236109246192      0.3107139542844577    0.31740966605326904
0.7994933354831326      0.3109114899944058    0.31769965089964464
0.7996577927803419      0.3111014601468688    0.31808815508191884
0.7998315685805413      0.3113006588121142    0.3185033806646636
0.8000000771194369      0.3114923120559793    0.3188007900241726
0.8001633183970283      0.3116765584357616    0.31894222335385586
0.8003337944969964      0.31186748223595206   0.3189708085020607
0.8004948359744337      0.31204644245709434   0.31897729917167505
0.8006568612324072      0.31222512529622426   0.3190620444882666
0.8008261213127577      0.3124103178625079    0.3192927546417481
0.8011634254538403      0.3127748938375285    0.3200385128049064
0.8013314695145728      0.31295429741103387   0.3203557858026606
0.8014942463140011      0.3131266644838127    0.3205306457757671
0.8016642579358062      0.31330520719931565   0.3205829460031073
0.8018290022963075      0.3134767697839124    0.3205828546023729
0.8020030651597987      0.3136564868139616    0.3206394055928883
0.8021718607619859      0.3138291768363597    0.320824815315496
0.8023353891028693      0.31399489793370466   0.321127633553703
0.8025061522661291      0.31416645001495286   0.3215012477215786
0.8026674808068581      0.3143271155619252    0.3218124783691121
0.8028297931281236      0.3144873802588648    0.3220132016142189
0.8029993402717657      0.31465331004790836   0.3220905085369082
0.8031636201541039      0.3148126442123455    0.3220926347732068
0.8033372185394321      0.3149794757068307    0.3221192890184558
0.8035055496634562      0.31513973375720916   0.3222533944251916
0.8036686135261766      0.31529355834876144   0.3225060870785625
0.8038389122112735      0.31545271756814747   0.3228488755760724
0.8039997762738393      0.3156016611522692    0.3231605879205329
0.8041616241169418      0.3157501450841293    0.32338464866573363
0.8043307067824208      0.31590379847636646   0.3234909417256443
0.804494522186596       0.3160512348358639    0.32350370976795345
0.8046655724131477      0.31620368051782805   0.3235090075542211
0.8048271880171685      0.3163463080725838    0.32358899510935096
0.8049897874017258      0.31648842186538045   0.3237828212507808
0.8053241885542894      0.31677633677041583   0.3243948722211395
0.8054980740029094      0.31692373407876384   0.32465427425831356
0.8056666921902254      0.3170651537337702    0.3247853393242123
0.8058300431162375      0.31720073560404877   0.3248131175466144
0.8060006288646261      0.3173408308648753    0.3248089326639628
0.8061617799904839      0.317471778356374     0.324854537908426
0.8063239148968782      0.31760215334590797   0.3250004776528918
0.806493284625649       0.3177368766965349    0.32525662039653674
0.8066573870931157      0.31786597852085985   0.32554834090906315
0.8068308080635727      0.31800088023307826   0.3258153584561829
0.8069989617727258      0.318130182486191     0.325970899215929
0.8071618482205747      0.3182540245427436    0.32601892979928326
0.8073319694908003      0.318381886207625     0.32601376267296756
0.807496823499722       0.31850434569898145   0.326033031765024
0.8076709960116337      0.31863218424508233   0.3261476197370558
0.8078399012622417      0.3187546427670399    0.3263644309602208
0.8080035392515454      0.31887186101359927   0.3266303102980217
0.8081744120632259      0.31899276898354134   0.326888715439545
0.8083358502523756      0.31910560030381063   0.32705273132642304
0.8084982722220615      0.3192176157873607    0.3271201964373468
0.8086679290141241      0.3193330872433058    0.3271208685655606
0.8088323185448827      0.31944354206472925   0.3271219479816427
0.8090060265786315      0.3195587281322426    0.3271954554093783
0.8091744673510763      0.3196689213208347    0.3273666260890472
0.809337640862217       0.3197742612977369    0.3276006258856475
0.8095080491957345      0.31988279449566787   0.3278502315465454
0.8096690229067212      0.31998393346959225   0.32802712947704843
0.809830980398244       0.3200843333591422    0.32811555221706695
0.8100001727121435      0.32018776543557664   0.32812833289633964
0.8101640977647391      0.3202865626505935    0.32811874519703166
0.8103352576397114      0.3203882354360257    0.32815535330337764
0.8104969828921529      0.32048291114120114   0.32827293713786776
0.8106596919251305      0.3205767982114219    0.3284665718175529
0.810829635780485       0.32067339974698705   0.3286995965794467
0.8109943123745353      0.3207655849918864    0.3288909974106624
0.8111683074715759      0.3208614670201837    0.3290071911539784
0.8113370353073124      0.3209529563464903    0.32903407544080526
0.8115004958817451      0.3210401914864196    0.3290216601038846
0.8116711912785544      0.3211298201926946    0.3290327743482034
0.8118324520528326      0.32121311878350656   0.3291117214966935
0.8119946966076474      0.32129557725417734   0.32926638194282076
0.8121641759848387      0.321380269644859     0.3294738696556422
0.8123283881007262      0.32146092445092594   0.32966231500922866
0.8125019187196036      0.3215446541440338    0.32979345802445353
0.8126701820771771      0.3216243696312385    0.32983747527943713
0.8128331781734468      0.3217002081835367    0.3298286147918756
0.8130034090920929      0.32177796264855124   0.3298210460122468
0.8131642053882082      0.3218500479553153    0.32986454637136164
0.81332598546486        0.3219212421249392    0.32997853219344786
0.8134950003638883      0.32199419401121654   0.33015437235545214
0.8136587480016126      0.32206348360674675   0.3303329245045016
0.8138297304617136      0.32213437705133185   0.3304735132236742
0.8139912782992837      0.3221999916685009    0.3305359581155619
0.8141538099173904      0.3222646665324599    0.33053894561068925
0.8143235763578736      0.32233078698926      0.33052042185461167
0.8144880755370527      0.32239346000578695   0.3305321912197687
0.814661893219222       0.3224581839329395    0.33061230910762507
0.8148304436400872      0.322519323625068     0.3307535989100908
0.8149937267996485      0.32257718116607964   0.33091450156318003
0.8151642447815866      0.32263616338553064   0.3310568071669994
0.8153253281409936      0.32269053357342253   0.3311328463758201
0.8154873952809372      0.32274391526172364   0.3311482503475551
0.8156566972432572      0.3227982671812787    0.3311281243673022
0.8158207319442734      0.3228495529526293    0.3311187664649541
0.8159940851482796      0.322902283162859     0.3311613261617647
0.8161621710909819      0.32295197161530126   0.33126512427474974
0.8163249897723801      0.322998753339998     0.3314022913477248
0.8164950432761551      0.32304619765637227   0.3315400528194336
0.816659829518626       0.32309079383387385   0.3316286551792015
0.8168339342640871      0.32313643968045097   0.33165742015903915
0.8170027717482442      0.32317926178496625   0.33163940003762266
0.8171663419710973      0.32321939491519514   0.33161756101072354
0.817337147016327       0.3232598837951145    0.3316314483060319
0.8174985174390259      0.32329680582361126   0.3316977321467439
0.8176608716422613      0.3233326501929492    0.33180691842718235
0.8178304606678732      0.32336869832887466   0.3319321734942335
0.8179947824321812      0.3234022703831341    0.33202536609040995
0.8181684226994792      0.32343629744810704   0.33206750392001727
0.8183367957054731      0.3234678725243593    0.33205700420697276
0.8184999014501633      0.3234971286004481    0.33202719225477134
0.81867024201723        0.3235262857195946    0.33201489943796303
0.8188311479617658      0.3235525190501082    0.3320469434047277
0.8189930376868382      0.3235776311883944    0.33212440661780196
0.819162162234287       0.3236024944462005    0.3322308352053094
0.8193260195204319      0.3236252487976465    0.3323235225408446
0.8194971116289536      0.32364760800526904   0.33237758935910305
0.8196587691149442      0.3236674220571587    0.3323798352925212
0.8198214103814712      0.32368607175717845   0.3323496749337744
0.8199912864703749      0.32370417678772656   0.3323167612355084
0.8201558952979746      0.32372038270317727   0.33231539282523204
0.8203298226285645      0.32373607740921434   0.33236247753523857
0.8204984826978503      0.32374989679659555   0.332444759878803
0.8206618755058321      0.32376197181823596   0.3325297863468076
0.8208325031361906      0.32377320434934803   0.33259055968254037
0.8209936961440183      0.32378246728903987   0.3326038706277827
0.8211558729323825      0.32379045281933033   0.33257812519938873
0.8213252845431231      0.3237974434758151    0.3325342397678497
0.8214894288925597      0.3238029022379561    0.3325068521097383
0.8216628917449864      0.3238072673996575    0.3325187736001116
0.8218310873361092      0.32381012529819725   0.33257221882736676
0.821994015665928       0.3238116054315168    0.3326436117890892
0.8221641788181235      0.3238118002618207    0.33270685388650406
0.8223290747090151      0.32381067448090584   0.3327315092484062
0.8225032891028967      0.32380808200042055   0.33271085494515634
0.8226722362354743      0.3238041935156727    0.3326614154478341
0.822835916106748       0.32379913796945364   0.33261510586076315
0.8230068308003984      0.3237925082402113    0.3325971418330668
0.8231683108715178      0.3237849789908386    0.33261984200942973
0.8233307747231735      0.3237761652803164    0.33267204587644966
0.823500473397206       0.3237656348960035    0.33273116325651325
0.8236649048099345      0.32375414302313044   0.3327638244345213
0.8238386547256531      0.3237406245969565    0.3327536132340893
0.8240071373800677      0.32372616891150224   0.3327044941812392
0.8241703527731783      0.32371090271108327   0.33264322909386507
0.8243408029886655      0.32369363603669893   0.33259599549552044
0.824501818581622       0.3236760851042908    0.3325872871881938
0.8246638179551147      0.32365721339648346   0.3326151017534036
0.8248330521509841      0.32363620126074094   0.33266418260466063
0.8249970190855496      0.32361458072437027   0.33270188228564135
0.8251703045231051      0.3235903836499578    0.332703907999739
0.8253383226993566      0.32356560202908485   0.33266096309836524
0.8255010736143042      0.3235403604228758    0.33259144373399874
0.8256710593516285      0.323512699632807     0.3325193539470146
0.8258357778276487      0.32348463395762267   0.33247825750063104
0.826009814806659       0.32345363358758095   0.3324794169047159
0.8261785845243654      0.3234222521592994    0.3325136068679564
0.8263420869807678      0.32339061365966665   0.3325503074125308
0.8265128242595469      0.3233562788144657    0.33256052156981386
0.8266741269157951      0.32332262684896784   0.3325269236103195
0.8268364133525796      0.3232875811592865    0.3324551714084306
0.8270059346117408      0.3232497024971061    0.3323647649598863
0.8271701886095981      0.3232117646755052    0.33229421044391044
0.8273437611104453      0.3231702641786294    0.3322628230582503
0.8275120663499886      0.32312872146631705   0.33227586220934724
0.827675104328228       0.32308726867343385   0.33230773996590796
0.8278453771288441      0.3230427078700792    0.3323266655701873
0.8280062153069293      0.32299942843509355   0.33230481765551473
0.8281680372655509      0.32295472240628537   0.33223720854640115
0.8283370940465491      0.3229067756695098    0.33213514334495475
0.8285008835662432      0.32285911508459597   0.3320382209095311
0.828671907908314       0.3228080836511496    0.3319726536487729
0.828833497627854       0.3227586821342822    0.3319575401563742
0.8289960711279303      0.32270782030973405   0.3319769337483813
0.8291658794503833      0.32265345599214035   0.33200170719809746
0.8293304205115324      0.32259957302392134   0.3319947139035291
0.8295042800756716      0.322541352855391     0.3319322755862034
0.8296728723785066      0.3224836378688873    0.33182489805883575
0.8298361974200379      0.3224265469801923    0.33170718178446934
0.8300067572839456      0.32236569141462307   0.3316094365048361
0.8301678825253225      0.32230704510194064   0.33156503055251474
0.8303299915472357      0.3222469089014729    0.3315663747480461
0.8304993353915258      0.3221828792570338    0.33158934956700586
0.8306634119745118      0.3221196649774311    0.3315932697435209
0.8308368070604879      0.32205160540014627   0.33154409890304354
0.83100493488516        0.3219843845926942    0.33143860822415466
0.831167795448528       0.32191811899045525   0.33130594147986764
0.8313378908342728      0.32184770348527875   0.3311776895719464
0.8315027189587136      0.321778295552978     0.3310992991439192
0.8316768655861444      0.3217037124377746    0.33107747209060756
0.8318457449522714      0.3216301602478407    0.3310939527116399
0.8320093570570943      0.3215577546108285    0.3311037248784996
0.8321802039842939      0.3214809448020173    0.3310658792477574
0.8323416162889626      0.32140725074439663   0.3309692851372251
0.8325040123741678      0.3213320062091133    0.33082780682332674
0.8326736432817496      0.3212522327030395    0.3306740178891529
0.8328380069280273      0.32117379200320917   0.33056296034870547
0.8330116890772952      0.321089683346679     0.3305116832010314
0.8331801039652591      0.32100693042131667   0.330515392435149
0.833343251591919       0.3209256463765808    0.3305298158565639
0.8335136340409556      0.3208395625623757    0.33050721700594843
0.8336745818674614      0.3207571085985145    0.33042246983326673
0.8338365134745034      0.3206730769836943    0.3302793802585406
0.834005679903922       0.32058414384015366   0.33010504971260063
0.8341695790720369      0.3204968647153356    0.32996114489573564
0.8343407130625282      0.32040456495402053   0.3298755261677261
0.8345024124304887      0.32031626029213695   0.3298580954340279
0.8346650955789856      0.3202263491521236    0.3298715936296104
0.8348350135498591      0.3201312974948728    0.32986639249551875
0.8349996642594286      0.32003808219846      0.32979969936174286
0.8351736334719884      0.31993840746379026   0.3296518987284264
0.8353423354232441      0.31984059205244153   0.32946330801269613
0.8355057701131957      0.3197447458368546    0.32929048941340944
0.8356764396255241      0.319643520360411     0.329169888875916
0.8358376745153215      0.31954682704122406   0.3291279255959025
0.8359998931856554      0.31944850376138706   0.3291337131224711
0.8361693466783658      0.3193446844239295    0.32913839936125505
0.8363335329097723      0.31924301240638636   0.32908921975985783
0.836507037644169       0.31913441827444855   0.3289530531087378
0.8366752751172616      0.31902799396501463   0.3287574229179504
0.8368382453290504      0.3189238466891934    0.32855897479546803
0.8370084503632156      0.31881397075028983   0.32840105434798306
0.83716922077485        0.31870915118685517   0.32832809122232265
0.8373309749670208      0.31860267913470985   0.32831885916827935
0.8374999639815683      0.31849036468356384   0.32832914694985893
0.8376636857348119      0.3183805012033355    0.3282990930771446
0.837834642310432       0.31826468339816205   0.32818379872425574
0.8379961642635214      0.3181542282123879    0.3279988762141678
0.8381586699971469      0.31804209414932605   0.3277806250948889
0.8383284105531492      0.31792389332008725   0.3275821761163236
0.8384928838478476      0.31780831608276106   0.3274661551561075
0.8386666756455359      0.3176850770450217    0.3274313876143234
0.8388352001819204      0.3175644835749271    0.3274408776574596
0.8389984574570009      0.3174466392776172    0.32742635542529985
0.839168949554458       0.31732250428824704   0.3273311421867884
0.8393300070293842      0.31720423894578703   0.32715426850298407
0.8394920482848469      0.31708427389897886   0.32692445320867514
0.839661324362686       0.31695790877636987   0.3266943445454375
0.8398253331792213      0.3168344427841426    0.3265403376167021
0.8399986604987466      0.31670286905785716   0.3264750537152696
0.8401667205569681      0.3165742356006874    0.32647667598359054
0.8403295133538855      0.31644864348052226   0.32647541807901104
0.8404995409731796      0.316316432662842     0.3264043457875182
0.8406643013311696      0.31618730977438547   0.32623837326041794
0.8408383801921498      0.3160498097080467    0.32598474198138006
0.841007191791826       0.3159154192660743    0.3257302131156198
0.8411707361301983      0.31578423845686227   0.32554454121503595
0.8413415152909471      0.3156462245918276    0.3254514088081302
0.8415028598291652      0.3155148716671613    0.325440541622058
0.8416651881479196      0.31538177602465245   0.3254466202427697
0.8418347512890506      0.31524174285061474   0.32539723979623597
0.8419990471688776      0.3151050826248015    0.3252505183069207
0.8421726615516949      0.3149596297975986    0.3249991089319414
0.8423410086732082      0.3148175711406631    0.32472276441219666
0.8425040885334173      0.3146790039290087    0.3245000373404893
0.8426744032160032      0.3145332911287373    0.324367914414171
0.8428352832760582      0.31439471642369693   0.32433582027768515
0.8429971471166495      0.3142543818474261    0.3243436292143035
0.8431662457796175      0.3141068003478078    0.32431607116082567
0.8433300771812815      0.31396286930861445   0.32419482612929934
0.8435011434053222      0.3138115913559892    0.32395882484026917
0.8436627750068321      0.31366772963051603   0.32367929511709914
0.8438253903888784      0.3135220862880838    0.32341795442050825
0.8439952405933011      0.31336899612584923   0.3232357590808828
0.8441598235364199      0.3132197139654716    0.3231680571466365
0.8443337249825289      0.313060978314777     0.32316944023197947
0.8445023591673338      0.312906071335115     0.3231582190144804
0.8446657260908348      0.3127550865188237    0.3230616743993213
0.8448363278367125      0.31259645593001556   0.3228424524193611
0.8449974949600594      0.3124457006882961    0.3225575288321333
0.8451596458639424      0.31229314845979383   0.3222681872411698
0.8453290315902022      0.3121328536898856    0.32204375922836476
0.8454931500551581      0.3119766343281192    0.3219406111508368
0.8456665870231039      0.31181057591781447   0.3219273298959914
0.8458347567297458      0.31164861311535386   0.32192868452263196
0.8459976591750837      0.31149083670211886   0.3218596900093671
0.8461677964427984      0.31132512503534737   0.3216657974397151
0.8465068549586097      0.3109920715773655    0.3210427965747146
0.8466757762067064      0.31082474989714354   0.3207851195629981
0.8468394301934992      0.3106617667403635    0.32065147490379414
0.8470103190026687      0.31049065857317454   0.3206216484506097
0.8471717731893073      0.31032813684322663   0.3206283296771574
0.8473342111564823      0.31016378428219155   0.3205827186913147
0.8475038839460338      0.3099912145507629    0.320415371496264
0.8476682894742814      0.3098231307622676    0.32013619631943785
0.847842013505519       0.30964459160875973   0.3197852469134631
0.8480104702754527      0.3094705580399088    0.3194889172213324
0.8481736597840825      0.3093011170218008    0.3193133183421521
0.8483440841150889      0.30912327531065015   0.319255164512383
0.8485050738235644      0.3089544478869607    0.31926102157227065
0.8486670473125764      0.30878377702780935   0.3192378035585276
0.8488362556239649      0.30860461642548115   0.31910284076219503
0.8490001966740495      0.3084301915328486    0.31884319728606186
0.8491734562271243      0.30824495562915527   0.3184842316342332
0.8493414485188948      0.30806447461103975   0.31815197940798196
0.8495041735493615      0.3078888327569669    0.317929869523632
0.8496741334022049      0.30770452383005614   0.3178335579393376
0.8498388259937443      0.30752509371032577   0.31782983280292676
0.8500128370882738      0.3073346237098144    0.31782052393424254
0.8501815809214992      0.3071490515994968    0.31770879042535044
0.8503450574934207      0.3069684605246175    0.3174655137234323
0.850515768887719       0.3067790280496279    0.3171081084699751
0.8506770456594863      0.30659927093869005   0.31676195695849463
0.85083930621179        0.30641764187382425   0.31649707311079134
0.8510088015864703      0.3062270869706526    0.31635990085613486
0.8511730296998467      0.30604165043215187   0.31633872218380277
0.851346576316213       0.30584483617269803   0.3163418797182765
0.8515148556712755      0.30565315889529876   0.31626264742071236
0.8516778677650341      0.30546669906009044   0.31604972147204224
0.8518481146811692      0.30527114524559135   0.3157015882492222
0.8520089269747735      0.30508566347344007   0.3153345406845953
0.8521707230489142      0.30489829989948347   0.3150264460583073
0.8523397539454315      0.30470177392573633   0.31484085555093105
0.8525035175806448      0.3045106101032408    0.3147918887204215
0.8526745160382347      0.3043101912262306    0.31479955998643866
0.8528360798732938      0.3041200731289487    0.3147577486733475
0.8529986274888892      0.30392805793887545   0.3145900223567057
0.8531684099268614      0.30372670750406877   0.3142692069963911
0.8533329251035295      0.30353083780583245   0.31388187286503666
0.8535067587831877      0.30332305786029284   0.31350798496687643
0.853675325201542       0.30312077649990626   0.31327554217852027
0.8538386243585923      0.30292407034834984   0.31319378475819143
0.8540091583380193      0.30271786927196526   0.3131964423113356
0.8541702576949154      0.30252234696858565   0.3131775798582152
0.8543323408323478      0.3023249186840705    0.31304566654075155
0.8545016587921569      0.3021179182689872    0.3127526128558861
0.854665709490662       0.3019166200052103    0.31236461667541965
0.8548390786921571      0.3017031020483422    0.311956747268615
0.8550071806323484      0.3014953036437056    0.3116735666763366
0.8551700153112356      0.30129329883460443   0.31154958942234195
0.8553400848124996      0.30108156815903375   0.3115367844419601
0.8555048870524595      0.3008756663391377    0.3115350259542498
0.8556790077954095      0.30065734620651513   0.31142482228875557
0.8558478612770556      0.30044487223538885   0.31115359723179914
0.8561822685401164      0.3000218858257216    0.3103423537298632
0.8563436549603042      0.29981671510702146   0.31002848526272453
0.8565060251610284      0.29960961715595535   0.3098611637341069
0.8566756301841293      0.29939256986328383   0.30982539763075595
0.8568399679459262      0.2991815629242281    0.30983299436706474
0.8570136242107131      0.2989578455212066    0.3097583834481125
0.8571820132141962      0.29874018533016994   0.309524444771813
0.8573451349563752      0.29852865268575773   0.30915463994823833
0.8575154915209309      0.29830702588834984   0.30871091524249106
0.8576764134629556      0.2980970077456548    0.30835355039009077
0.8578383191855168      0.2978850561467281    0.3081360338318939
0.8580074597304547      0.2976629406265281    0.30806629532447993
0.8581713330140885      0.29744706987657976   0.30807465717420235
0.858342441120099       0.29722096584892466   0.3080353946510557
0.8585041146035787      0.2970066720776523    0.307858188025628
0.8586667718675947      0.2967904727865194    0.30752271259918884
0.8588366639539874      0.29656397807362783   0.30707375407038023
0.8590012887790761      0.2963438419903942    0.3066629432383904
0.8591752321071549      0.2961105387657805    0.3063713870567892
0.8593439081739297      0.2958836100965509    0.3062635673626831
0.8595073169794005      0.2956631225016595    0.30626372379080896
0.859677960607248       0.2954321981285097    0.3062486848523105
0.8598391696125646      0.29521341064019135   0.30611079628178994
0.8600013623984175      0.29499267243105004   0.30580778513966905
0.8601707900066472      0.29476143141501204   0.30536375532475607
0.8603349503535729      0.29453674231774585   0.3049235154971498
0.8605084292034886      0.294298620541504     0.30457812975023696
0.8606766407921004      0.2940670661916471    0.30442255563325626
0.8608395851194082      0.2938421433787525    0.3044039464230491
0.8610097642690926      0.2936065862851869    0.30440702465566016
0.861174676157473       0.29337769160226135   0.30430616969044477
0.8613489065488436      0.2931351942197618    0.3040133244061897
0.8616815655476728      0.29267029595440097   0.3031173327123202
0.8618524962388121      0.29243045435547504   0.3027365117644262
0.8620139923074205      0.2922032545644348    0.3025442305549121
0.8621764721565652      0.29197408868128005   0.30250020805963074
0.8623461868280866      0.2917340980974082    0.30251114688873565
0.862510634238304       0.29150095387387304   0.3024455481649135
0.8626844001515115      0.2912539576999118    0.30219616054061293
0.8628528988034151      0.2910138228828929    0.3017803049260711
0.8630161301940147      0.29078061004086003   0.30130796304125124
0.8631865964069909      0.29053644942856743   0.3008812645156297
0.8633476279974363      0.2903052313107516    0.30063656373649156
0.863509643368418       0.2900720436452465    0.30055639850868515
0.8636788935617764      0.2898278491592062    0.3005657538351367
0.8638428764938308      0.28959067837356434   0.30053316410344655
0.8640161779288753      0.28933941714416284   0.3003339727426803
0.8641842121026159      0.28909519419694935   0.2999516645118323
0.8643469790150524      0.28885806776237805   0.2994766326109581
0.8645169807498656      0.288609816935331     0.2990083427479192
0.8646817152233748      0.28836869104855156   0.2987008247553507
0.8648557681998741      0.28811334701635244   0.298574008153106
0.8650245539150695      0.2878651777522439    0.2985756751329856
0.8651880723689609      0.28762420281785195   0.2985622290279457
0.8653588256452289      0.2873719907621394    0.29840179218663326
0.8655201442989661      0.2871331767851707    0.2980657402075267
0.8656824467332396      0.2868923819958464    0.2975976871140036
0.8658519839898897      0.2866402944001534    0.29709935384939423
0.8660162539852359      0.28639549642140905   0.29673939736012295
0.8661898424835721      0.28613623459351367   0.29656076919293023
0.8663581637206045      0.2858842759569303    0.2965424594419587
0.8665212176963328      0.2856396752027222    0.2965477057492904
0.8666915064944378      0.2853836707684392    0.2964345291667367
0.8668523606700119      0.28514133578057793   0.29614331993840165
0.8670141986261224      0.28489701704002135   0.2956944519925253
0.8671832714046095      0.28464124144654995   0.29517482073469864
0.8673470769217926      0.28439291557621865   0.2947623558001558
0.8675181172613524      0.2841330798280154    0.2945241615096324
0.8676797229783814      0.28388707043843653   0.29447087806222266
0.8678423124759467      0.28363906955342516   0.29448537431741995
0.8680121367958886      0.2833795069400186    0.29442574878635275
0.8681766938545266      0.28312748449015573   0.2941892758888086
0.8683505694161546      0.2828606475487599    0.2937398360455759
0.8685191777164788      0.2826013638810314    0.293210505165843
0.868682518755499       0.28234968500407615   0.2927555147678835
0.8688530946168957      0.28208634116329934   0.2924599111137631
0.8690142358557617      0.2818370799131839    0.2923677870001155
0.8691763608751639      0.2815858257224243    0.2923777967458297
0.8693457207169428      0.2813228572266364    0.29235175156656695
0.8695098132974177      0.2810675805185862    0.2921660481642774
0.8696832243808827      0.2807972892583978    0.29175743698827133
0.8698513682030439      0.28053470251855656   0.291230031759533
0.870014244763901       0.2802798696445057    0.290737347740859
0.8701843561471347      0.2800132242295381    0.29037970537742236
0.8703492002690645      0.2797543570631708    0.2902359925337155
0.8705233628939844      0.27948034801246485   0.2902329146387562
0.8706922582576002      0.27921412985179106   0.29022567357579754
0.8708558863599121      0.27895575089486074   0.2900749382220893
0.8710267492846007      0.2786854638702382    0.2897044062100673
0.8711881775867584      0.27842970438616566   0.28920548879620167
0.8713505896694524      0.2781719565322036    0.2886866973496761
0.8715202365745232      0.27790225753190784   0.2882749026822458
0.8716846162182899      0.27764047694549365   0.2880792357132844
0.8718583143650467      0.277363371874931     0.28805120385801936
0.8720267452504995      0.277094196901097     0.2880618740726028
0.8721899088746483      0.2768329980185524    0.28795788318098664
0.8723603073211739      0.2765597557385536    0.28763860889001613
0.8725212711451685      0.2763012113997764    0.2871622366210387
0.8726832187496996      0.2760406665520632    0.2866260946911522
0.8728524011766072      0.2757680342058467    0.28616062371423695
0.8730163163422109      0.2755034555861846    0.2859045863960923
0.8731874663301913      0.2752267453924524    0.28583795847826626
0.8733491816956408      0.27496486519270463   0.2858560066158314
0.8735118808416267      0.2747009787014514    0.2858044432951882
0.8736818148099892      0.2744249177307815    0.285556191364516
0.8738464815170477      0.2741569868521013    0.285111218245069
0.8740204667270962      0.2738734399973172    0.284528898776086
0.8741891846758408      0.273598034502287     0.28402079978321865
0.8743526353632814      0.2733308135239427    0.2837086305154445
0.8745233208730987      0.2730513326438606    0.28359855009267954
0.8746845717603853      0.272786896946982     0.2836112654692093
0.8748468064282081      0.27252045491326476   0.2835918789115058
0.8750162759184076      0.27224171230249095   0.28339819900568713
0.875180478147303       0.27197122751286457   0.28299617864602716
0.8753539988791886      0.27168496120350266   0.2824206971922646
0.8755222523497701      0.27140696364732886   0.28187521275404825
0.8756852385590478      0.27113727607090665   0.2815028334771076
0.8758554595907021      0.27085520693850657   0.281337521108777
0.8760162459998255      0.2705883888666701    0.28133260657935283
0.8761780161894853      0.27031956509581556   0.2813384486387339
0.8763470212015216      0.2700383213059303    0.28120208333089947
0.8765107589522542      0.2697654577631777    0.2808543870316592
0.8768432694759416      0.2692101829879853    0.27975385812968145
0.8770057912070561      0.2689382207586659    0.27931120477500365
0.8771755477605474      0.2686537626551028    0.2790679698929099
0.8773400370527347      0.2683777621615109    0.27902426884695897
0.877513844847912       0.2680857887276048    0.2790443186664731
0.8776823853817853      0.2678022724788285    0.27895523047524795
0.8778456586543547      0.26752725154406526   0.2786605789630808
0.8780161667493007      0.26723966260188314   0.2781448236374578
0.8781772402217161      0.26696763056605644   0.27757858811578756
0.8783392974746677      0.2666935898397163    0.2770873668579253
0.8785085895499959      0.2664069450454377    0.2767799904233806
0.87867261436402        0.2661288602459451    0.27669385970302063
0.8788459576810344      0.2658345957222039    0.27671466860957344
0.8790140337367447      0.2655489008008825    0.276670117629913
0.879176842531151       0.2652718119250311    0.2764356407781776
0.8793468861479341      0.2649820471894939    0.275964601907578
0.8795116625034132      0.2647009066655711    0.27538399822224147
0.8796857573618823      0.26440349328442814   0.27481730011527666
0.8798545849590473      0.2641147136501678    0.2744634053309563
0.8800181452949086      0.26383460339351195   0.2743416048529038
0.8801889404531466      0.26354174761629273   0.2743547415927076
0.8803503009888536      0.2632647375126354    0.27434114205461724
0.8805126453050969      0.2629857154189951    0.2741603039635344
0.8806822244437169      0.2626939147600968    0.27373723435424135
0.8808465363210329      0.26241084434651446   0.27316811149501075
0.881020166701339       0.2621113660739675    0.2725669283071209
0.8811885298203411      0.26182062728897276   0.2721510533925296
0.8813516256780393      0.2615386619380412    0.27197401728138737
0.881521956358114       0.26124385223727253   0.27196562088777854
0.8816828524156581      0.2609650581866442    0.2719765399947295
0.8818447322537384      0.2606842536545148    0.271851453676032
0.8820138469141954      0.26039057365760726   0.2714875913663019
0.8823487765348779      0.2598079701773549    0.2703256124450583
0.8825104241338766      0.25952632835592887   0.2698588503727702
0.8826730555134118      0.25924267347628005   0.2696046027008053
0.8828429217153236      0.2589460817700026    0.2695513978671427
0.8830075206559314      0.2586583787803658    0.26957656194889235
0.8831814380995293      0.25835406069243005   0.26949691089655053
0.8833500882818233      0.2580586401990636    0.26918972061578816
0.8835134712028132      0.2577721487568186    0.2686791420410521
0.8836840889461797      0.2574726989397941    0.2680509758814358
0.8838452720670155      0.25718954529166155   0.2675366682002292
0.8840074389683875      0.25690438124397763   0.26722035344239853
0.8841768406921362      0.2566061945657914    0.26712117426473886
0.8843409751545811      0.2563169883679585    0.2671442787078362
0.8845144281200159      0.2560110533962088    0.26710895538635326
0.8846826138241468      0.2557141069348319    0.2668654069749226
0.8848455322669737      0.25542617864670153   0.26640095748602205
0.8850156855321774      0.25512517007316543   0.2657763084639623
0.885180571536077       0.254833194622796     0.2652071773546963
0.8853547760429666      0.2545244155230436    0.2648101588087223
0.8855237132885524      0.2542246775124747    0.26467370232856513
0.885687383272834       0.2539340095417709    0.26468826563113246
0.8858582880794925      0.2536302054969115    0.26467942755806734
0.88601975826362        0.25334290452860336   0.2644948809712265
0.8861822122282839      0.2530535919170608    0.26407863820929733
0.8863519010153245      0.25275111688660346   0.26346875696769306
0.886516322541061       0.2524577616780564    0.2628696549634234
0.8866900625697877      0.2521474946997588    0.26240964410179995
0.8868585353372103      0.2518463552563948    0.26221526021074837
0.8870217408433291      0.25155437084516225   0.26220746552415625
0.8871921811718244      0.25124917157109805   0.2622237083050277
0.8873531868777889      0.2509606134850808    0.2620956125807583
0.8875151763642897      0.25067004590388015   0.26173804823185426
0.8876844006731672      0.2503662387801124    0.2611571948750503
0.8878483577207408      0.2500716341836498    0.2605377922979636
0.8880216332713043      0.24976001619042137   0.26001472048548196
0.888189641560564       0.24945760818643828   0.25975204914678207
0.8883523825885196      0.2491644362268046    0.25970738913913316
0.888522358438852       0.24885797559900558   0.25973713900220174
0.8886870670278803      0.24856076446471168   0.25965957165367304
0.8888610941198987      0.24824647597666302   0.25933077590797265
0.8890298539506132      0.24794144437777252   0.25877668481775296
0.8891933465200237      0.24764569499673095   0.25814986171744947
0.8893640739118107      0.2473366097216204    0.2575912810647573
0.889525366681067       0.24704437339886282   0.2572815333591452
0.8896876432308597      0.24675012929105722   0.2571938217266395
0.889857154603029       0.24644252859085164   0.2572229658595527
0.8900213987138943      0.246144307302545     0.2571886306744969
0.8901949613277498      0.2458289199695407    0.2569263905449515
0.8903632566803013      0.24552286395264958   0.25641965626688645
0.8905262847715487      0.24522616295559022   0.2557955479117544
0.8906965476851727      0.24491606145329342   0.25519104535614046
0.890857375976266       0.24462292560281557   0.25481430851668185
0.8910191880478956      0.24432778462131713   0.25467067123861514
0.8911882349419018      0.24401922188710282   0.25468499089820695
0.8913520145746042      0.24372005513070566   0.2546869457598817
0.8915230290296832      0.24340744515053608   0.2544996371962122
0.8916846088622312      0.24311186908395477   0.2540777407437979
0.8918471724753156      0.24281428665174273   0.25347706356608773
0.8920169709107766      0.24250324056712155   0.25283184782878065
0.8921815020849337      0.24220163054076801   0.2523658920442546
0.8923553517620809      0.24188271240028092   0.25214028387053733
0.8925239341779242      0.24157323670473108   0.2521292417100688
0.8926872493324634      0.2412732252866111    0.2521529559326015
0.8928577993093794      0.24095970955226875   0.2520247555425198
0.8930189146637644      0.24066333788603966   0.25166239331477774
0.8931810137986858      0.24036496263249174   0.251092595742689
0.8933503477559837      0.2400530640743739    0.25042977952238715
0.8935144144519778      0.2397506678953447    0.2499061065834541
0.8936877996509619      0.23943088480139374   0.2496092725160612
0.893855917588642       0.23912061025583914   0.2495582870490894
0.8940187682650184      0.23881986484341555   0.2495926864116059
0.8941888537637712      0.2385055583332774    0.24952181687893812
0.89435367200122        0.23820079179514872   0.24921689798119312
0.894527808741659       0.23787858902785486   0.24864058846428685
0.8946966782207941      0.23756593234283058   0.24797250715109215
0.8948602804386252      0.23726284168411552   0.24741236369035238
0.8950311174788328      0.2369461538079327    0.24706627038059756
0.8951925198965097      0.23664677475064788   0.2469776967946489
0.8953549060947228      0.23634539533921153   0.2470098120571882
0.8955245271153127      0.23603040200189107   0.24698211725988642
0.8956888808745985      0.23572500990956574   0.24674071251694657
0.8958625531368745      0.23540211158787933   0.24621611800661114
0.8960309581378465      0.2350888204041887    0.24555326958342363
0.8961940958775145      0.2347851727903285    0.24495127712662743
0.8963644684395591      0.2344679007469776    0.24453502103886493
0.8965254063790729      0.23416803018094773   0.2443908643617266
0.8966873280991231      0.23386616321311182   0.2444073925969355
0.8968564846415499      0.2335506348960043    0.24441575540539578
0.8970203739226725      0.23324476372862385   0.2442416362341923
0.8971914980261722      0.23292521509918418   0.2437894261299572
0.8973531875071408      0.2326231213544547    0.24317407308917405
0.8975158607686458      0.2323190311589406    0.2425351818854664
0.8976857688525273      0.23200124843211287   0.24203442499951308
0.897850409675105       0.23169315453015213   0.2418071243744494
0.8980243690006726      0.23136745037469228   0.24179147847331145
0.8981930610649365      0.23105144033450936   0.24182120205574503
0.8983564858678962      0.23074514124281267   0.24170330189337122
0.8985271454932326      0.230425119628298     0.24131453852918872
0.8986883704960382      0.23012263851991688   0.24073029373201035
0.8988505792793802      0.22981816459842713   0.24007627671845164
0.8990200228850986      0.22949995436062423   0.2395174180532459
0.8991841992295133      0.22919148499294578   0.23922270122273606
0.899357694076918       0.22886534725122      0.23916403343484052
0.8995259216630188      0.2285489554772797    0.23920408686760253
0.8996888819878155      0.2282423253987123    0.2391407293795377
0.899859077134989       0.2279219315553828    0.2388245114632666
0.9000198376596316      0.22761915850649783   0.23828590154295662
0.9001815819648105      0.22731439667148404   0.23762978394107337
0.9003505610923659      0.226995858543585     0.23701738180481383
0.9005142729586175      0.2266871102931689    0.23664819119774794
0.9006852196472457      0.226364572822393     0.23653071371172513
0.9008467317133431      0.22605970165940992   0.2365637029376893
0.9010092275599769      0.2257528427425185    0.23655576290629896
0.9011789582289873      0.2254321828351594    0.23633200775724889
0.9013434216366936      0.22512134016870838   0.23585105962072478
0.9015172035473902      0.22479274350948278   0.23516005589885677
0.9016857181967826      0.22447396894371863   0.23450976171500684
0.9018489655848712      0.2241650305243647    0.2340751111636458
0.9020194477953363      0.2238422676956984    0.23389646385156138
0.9021804953832707      0.2235372434758985    0.23391030114530748
0.9023425267517413      0.22323023631733208   0.23393229895316642
0.9025117929425888      0.2229094089350907    0.2337760690880964
0.9026757918721321      0.22259844775572263   0.23336020431699112
0.9028491093046656      0.22226968813730377   0.23269801550356586
0.9030171594758949      0.22195079410611881   0.23201947882758675
0.9031799423858204      0.22164177856034906   0.23151817991910972
0.9033499601181225      0.22131890761081072   0.23126638074404796
0.9035147105891208      0.22100592242549644   0.2312465729661306
0.903688779563109       0.22067511067832732   0.23128586886413918
0.9038575812757932      0.22035418913501043   0.23117361599071956
0.9040211157271736      0.2200431701385457    0.23080624038684933
0.9041918850009305      0.21971827582968578   0.23018083193940683
0.9043532196521566      0.21941122383436995   0.22951483350072813
0.9045155380839193      0.21910219543555454   0.22896001515854558
0.9046850913380584      0.21877928290650972   0.2286392254202031
0.9048493773308934      0.21846629569745984   0.2285766591984726
0.9050229818267188      0.21813544334459675   0.22862281831963208
0.90519131906124        0.2178145205714698    0.22856532745011024
0.9053543890344573      0.21750353869036276   0.2282677086804895
0.9055246938300512      0.21717865514035478   0.22769210018020486
0.9056855640031144      0.21687167277073993   0.22702511063186254
0.9058474179567138      0.21656271931558418   0.22642112994202232
0.90601650673269        0.2162398565526313    0.22602449119264142
0.906180328247362       0.21592695583312546   0.2259047198220153
0.9063513845844109      0.2156001376599808    0.22594188562888828
0.9065130062989288      0.21529125359920992   0.22593822893458956
0.9066756117939829      0.21498040081453137   0.22572935602541777
0.906845452111414       0.2146556236696471    0.22523189634662497
0.9070100251675408      0.21434082892472567   0.2245669087237772
0.9071839167266579      0.21400811500845066   0.2238757383541575
0.9073525410244709      0.21368538752032057   0.22341309293066453
0.9075158980609799      0.2133726561908589    0.22323463604136612
0.9076864899198658      0.2130459867685194    0.22325058878242124
0.9078476471562207      0.21273730278069639   0.22327655410575117
0.9080097881731118      0.21242665613119077   0.22313169197796437
0.9081791640123797      0.21210206565976308   0.22270170781011753
0.9083432725903435      0.2117874900832876    0.22206648251994912
0.9085166996712974      0.2114549687251173    0.2213498053076903
0.9086848594909476      0.21113246529240356   0.22081991647744734
0.9088477520492936      0.21081998598114696   0.2205717188199466
0.9090178794300163      0.21049355130073583   0.22055060999626466
0.909182739549435       0.21017714979720076   0.2205954560550984
0.9093569181718437      0.20984278697088635   0.2204917640768514
0.9095258295329487      0.2095184611530143    0.22011102280865896
0.9096894736327495      0.2092041804885329    0.2195035239650313
0.9098603525549271      0.2088759346693691    0.21878566525077886
0.9100217968545739      0.20856574707202344   0.21822725284399705
0.9101842249347567      0.20825360680898694   0.2179139845116331
0.9103538878373165      0.2079274976589725    0.2178487404489899
0.9105182834785721      0.2076114500302287    0.2178985234623608
0.9106919976228179      0.20727742211363395   0.21784921138558194
0.9108604445057598      0.20695345936909645   0.21753989732986784
0.9110236241273975      0.20663956890715457   0.21698081621740553
0.9111940385714121      0.2063117021501323    0.21626284144028007
0.9113550183928958      0.20600193245137166   0.21565587664624705
0.9115169819949157      0.20569021740248708   0.21527104114602968
0.9116861804193124      0.20536452350849582   0.21514727536302092
0.9118501115824049      0.2050489166374068    0.215188089539789
0.9120233612484876      0.20471531512761268   0.21518686398017636
0.9121913436532665      0.2043918041121349    0.21495492583056525
0.9123540587967413      0.2040783896754929    0.21445871307140787
0.9125240087625928      0.2037509913435344    0.2137568840894641
0.9126886914671402      0.20343369407626283   0.21309469608383017
0.9128626926746777      0.2030983948995924    0.21261740823126812
0.9130314266209114      0.2027732002011719    0.21244770622173487
0.9131948933058409      0.20245811547180997   0.21247380512291394
0.9133655948131472      0.2021290433360434    0.21249946634319666
0.9135268616979226      0.20181812059684864   0.21233561280183266
0.9136891123632342      0.20150526477344177   0.21190264400163814
0.9138585978509228      0.2011784209493068    0.21122913942454694
0.9140228160773072      0.20086169936097145   0.21054185483844604
0.9141963528066818      0.2005269690605786    0.20999603137686176
0.9143646222747523      0.2002023642302635    0.2097566954203457
0.9145276244815188      0.19988788933584783   0.20975043873842758
0.9146978615106622      0.19955942528475346   0.20979912233007036
0.9148586639172747      0.19924913679639164   0.20969989985997725
0.9150204501044233      0.1989369053704447    0.2093406706496521
0.9151894711139488      0.19861067316412487   0.20871189774517934
0.9153532248621701      0.19829458269999026   0.20801223545032285
0.9155242134327681      0.19796450250658515   0.2074060827355076
0.9156857673808354      0.19765261350291033   0.20708998723354716
0.9158483051094389      0.19733880540717005   0.20702449283269508
0.9160180776604192      0.19701100937928906   0.20708047776807442
0.9161825829500954      0.19669336545239777   0.207047417245797
0.9163564067427616      0.19635771079101436   0.20674475120643657
0.9165249632741241      0.19603221148163874   0.2061668412810681
0.9166882525441824      0.1957168703961455    0.20546976447439722
0.9168587766366174      0.19538754498713412   0.20481486014720676
0.9170198661065216      0.1950764298019812    0.204429082313326
0.917181939356962       0.1947634057312208    0.20430753802211404
0.9173512474297794      0.1944364004170019    0.20435265337984998
0.9175152882412925      0.19411956191786073   0.20436270117329453
0.9176886475557958      0.1937847196822847    0.20413835283900053
0.9178567396089952      0.1934600473177388    0.2036256080525053
0.9180195644008904      0.1931455465819364    0.2029461213829338
0.9181896240151624      0.1928170707705771    0.20225076707167758
0.9183544163681303      0.19249876948143002   0.2017834168640851
0.9185285272240885      0.19216247145776183   0.20159518415313502
0.9186973708187427      0.19183635094527357   0.20162483890503058
0.9188609471520928      0.1915204090270152    0.20165797406468525
0.9190317583078196      0.19119049994549922   0.20149026443847107
0.9191931348410156      0.19087882074546564   0.20105363152558228
0.9193554951547479      0.19056525029660112   0.20040243025837845
0.9195250902908569      0.19023771792454824   0.19968349827199153
0.9196894181656619      0.18992037000571574   0.19915363510528142
0.9198630645434569      0.1895850406866681    0.1988943583986538
0.9200314436599482      0.1892598986148378    0.19889065144324586
0.9201945555151352      0.1889449437602496    0.19894459418007812
0.920364902192699       0.18861603745102457   0.19884389193775673
0.9205258142477319      0.18830536615539398   0.1984796718932249
0.9206877100833011      0.1879928150462774    0.19787184407232672
0.9208568407412472      0.1876663189930449    0.19714184074792857
0.9210207041378891      0.18735001426474956   0.19655171129766472
0.9211918023569077      0.18701976806485396   0.19621372428007028
0.9213534659533955      0.18670770502351947   0.1961562420834944
0.9215161133304195      0.1863937697155223    0.19621648886218793
0.9216859955298202      0.18606590038019308   0.19618863524011199
0.9218506104679169      0.18574822797267973   0.1959087765549098
0.9220245439090039      0.18541260799245052   0.19531617788688668
0.9221932100887869      0.18508718796036308   0.1945897132829783
0.9223566090072657      0.184771966184669     0.19395481941610715
0.9225272427481213      0.1844428265937895    0.19354510537262415
0.922688441866446       0.1841319241351797    0.19343393969519473
0.922850624765307       0.18381916354429903   0.19348401669601498
0.9230200424865447      0.18349249474225562   0.1934996584029432
0.9231841929464784      0.18317602646370792   0.19329295878807928
0.9233576619094022      0.1828416420347611    0.19276774051693601
0.9235258636110222      0.1825174609239029    0.1920595174106272
0.923688798051338       0.18220348015174134   0.1913872836460693
0.9238589673140305      0.181875610337134     0.19090192412069862
0.924023869315419       0.18155794216186633   0.1907227593712595
0.9241980898197977      0.18122238132200696   0.19075582759344406
0.9243670430628725      0.1808970248343678    0.19079473457127705
0.9245307290446432      0.1805818689089385    0.1906366109462628
0.9247016498487906      0.18025284520118365   0.19017006245804338
0.9248631360304072      0.17994204293118665   0.18951249525873978
0.9250256059925599      0.1796294073742933    0.18881854011788451
0.9251953107770894      0.179302916172899     0.1882706383148297
0.9253597483003149      0.17898662459556325   0.1880257288047734
0.9255335043265306      0.17865248155430066   0.18802435635628456
0.9257019930914423      0.17832854062435677   0.18808508987266911
0.9258652145950499      0.17801479672940138   0.18799045064616487
0.9260356709210342      0.17768722140310245   0.18759906611465005
0.9261966926244878      0.17737784938978204   0.1869825736072929
0.9263586981084775      0.17706665965719456   0.18627807101563487
0.9265279384148439      0.17674165209828455   0.1856694254330405
0.9266919114599064      0.17642683861219052   0.1853496653157429
0.9268652030079589      0.1760942205930206    0.1852977121191374
0.9270332272947077      0.1757717989101088    0.18536704929639772
0.9271959843201523      0.17545956721155434   0.1853326357146529
0.9273659761679736      0.17513354482784987   0.18502536849394083
0.9275307007544908      0.17481767430635556   0.1844487731373879
0.9277047438439983      0.17448398656590697   0.1836939101632339
0.9278735196722017      0.1741604939796126    0.18304951470648637
0.9280370282391011      0.173847189662391     0.18267774279831478
0.9282077716283772      0.173520121720252     0.1825836333821181
0.9283690803951226      0.17321122138595668   0.1826466025017073
0.9285313729424041      0.17290053227359103   0.18265646414648587
0.9287009003120623      0.17257609689094228   0.1824240222607691
0.9288651604204166      0.17226184505710618   0.18190886389730326
0.929038739031761       0.17192987826425665   0.18117036584482654
0.9292070503818015      0.17160809764871027   0.18048506889548604
0.9293700944705379      0.1712964947849576    0.1800417938103511
0.929540373381651       0.17097118026567135   0.1798815185876319
0.9297012176702333      0.17066400044550006   0.17992549570104333
0.9298630457393517      0.17035505156229377   0.17997126563250213
0.930032108630847       0.17003241013466508   0.17981654442902223
0.9301959042610382      0.1697199393147546    0.17937587617181974
0.9303669347136061      0.16939379329167345   0.17868034336476676
0.9305285305436433      0.16908575934163156   0.17798855959147625
0.9306911101542166      0.16877597040313047   0.1774600465626371
0.9308609245871666      0.16845252666051114   0.17720531953560664
0.9310254717588127      0.16813924495464247   0.17720644083694584
0.9311993374334488      0.16780836244075847   0.17727814176567736
0.931367935846781       0.1674876442229957    0.17718725830421012
0.9315312669988092      0.16717707935669826   0.17681777910981428
0.9317018329732141      0.16685290036157113   0.17616568888558481
0.9318629643250882      0.16654678851199997   0.17546357570052637
0.9320250794574985      0.16623894257294622   0.17487916732798423
0.9321944294122855      0.1659175046583753    0.17454952020961456
0.9323585121057685      0.16560620910280863   0.17450300728007184
0.9325319133022416      0.16527739157254062   0.17458148004894658
0.9327000472374107      0.1649587183926171    0.174551511050144
0.9328629139112758      0.16465017709245916   0.17426154025693508
0.9330330154075177      0.16432808799358664   0.17366899982751755
0.9331978496424556      0.16401612866580706   0.17295268207639317
0.9333720023803835      0.1636867026368099    0.17228220750382314
0.9335408878570075      0.16336740826574267   0.1719016786321861
0.9337045060723275      0.1630582320662345    0.17181699553226124
0.9338753591100241      0.16273545961013622   0.17189066162830077
0.9340367775251899      0.16243066442708962   0.171902685708976
0.9341991797208922      0.16212417270241253   0.1716831105812381
0.934368816738971       0.1618042021010253    0.1711530148217601
0.9345331864957457      0.16149433958486287   0.17045359494233028
0.9347068747555105      0.16116709783294814   0.16974421203566736
0.9348752957539714      0.16084996665356238   0.16929243344894793
0.9350384494911284      0.16054293121590738   0.1691463259077588
0.935208838050662       0.1602224698841527    0.1692012309816174
0.9353697919876649      0.15991993272176272   0.1692501043986809
0.9355317297052039      0.1596157249906878    0.1691039476047382
0.9357009022451197      0.15929812005581145   0.16864929981525295
0.9358648075237315      0.15899059511312857   0.16798246116211016
0.93603594762472        0.15866969933276398   0.16725295136274182
0.9361976531031777      0.15836668719532784   0.16673957059742567
0.9363603423621716      0.1580620234250394    0.16650521672843496
0.9365302664435421      0.15774401909728694   0.16651534067278173
0.9366949232636088      0.1574360772462364    0.16659257170441794
0.9368688985866653      0.15711092982949154   0.1665090057373226
0.9370376066484181      0.1567958469270456    0.16612607434393487
0.9372010474488668      0.15649081063935374   0.16550025485895695
0.9373717230716923      0.15617249415942275   0.16476008197467965
0.9375329640719868      0.15587198472008365   0.16419212074882591
0.9376951888528178      0.15556985106284688   0.16388863149485938
0.9378646484560254      0.1552544696097386    0.163852055054443
0.9380288407979288      0.15494911433345981   0.1639359874591082
0.9382023516428224      0.15462667048238482   0.16391353565628833
0.9383705952264121      0.15431425452200861   0.163610944098557
0.9385335715486978      0.1540118467113317    0.16304090718417732
0.9387037826933602      0.15369625569444278   0.16230402107338107
0.9388645592154918      0.1533983862348364    0.16168613194252035
0.9390263195181596      0.15309892060918176   0.16130704335965362
0.9391953146432042      0.15278630627129286   0.1612084290003923
0.9393590425069447      0.15248367674768404   0.1612837808593649
0.9395300051930618      0.15216793050481764   0.1613159007927611
0.9396915332566482      0.1518698513481453    0.16111679614436872
0.9398540451007709      0.1515701968706413    0.16063133551308104
0.9400237917672701      0.15125743522765792   0.15992073741060905
0.9401882711724656      0.1509545455904378    0.15924101172139377
0.9403620690806509      0.1506347723791918    0.15875221357718633
0.9405305997275324      0.15032496490065936   0.15859091416671806
0.9406938631131099      0.1500251007220899    0.15864534166696428
0.9408643613210641      0.14971222517770935   0.15871222874247365
0.9410254249064874      0.14941692531292466   0.15858376098976465
0.9411874722724471      0.14912008187909057   0.1581707708135298
0.9413567544607835      0.14881026705366943   0.1574974573118134
0.9415207693878157      0.14851036951088825   0.1567972361747942
0.941694102817838       0.1481937330652211    0.15623906945129212
0.9418621689865565      0.1478870157977138    0.15600358308360135
0.9420249678939709      0.14759019307213223   0.156021431618239
0.9421950016237621      0.14728047865526694   0.1561109623160557
0.9423597680922493      0.1469806530282309    0.15604726713113906
0.9425338530637266      0.14666418865657724   0.15567151728311196
0.9427026707738998      0.14635761481367257   0.1550326029978959
0.9428662212227692      0.1460609053554836    0.15432799952640996
0.943037006494015       0.14575138777944371   0.15373361180807504
0.94319835714273        0.14545926921379915   0.15344693037410123
0.9433606915719814      0.14516566690299812   0.15342177016859
0.9435302608236096      0.14485930093051932   0.15351790521941755
0.9436945628139337      0.14456276765480125   0.15351008248211873
0.943868183307248       0.14424975760076977   0.15321525153792417
0.9440365365392582      0.14394658159517915   0.15263298253445648
0.9441996225099644      0.1436532112928378    0.15193289458959408
0.9443699433030472      0.14334716595073999   0.15128830111105387
0.9445308294735992      0.14305839478493634   0.15092945488987689
0.9446926994246876      0.142768175627504     0.15084735809923233
0.9448618041981527      0.14246532839286663   0.15093539382337093
0.9450256417103138      0.14217225170996287   0.15097694639005702
0.9451967140448515      0.14186659073078434   0.15077323017078337
0.9453583517568583      0.14157812535403652   0.15028971046400563
0.9455209732494017      0.1412882388899239    0.14961507653523226
0.9456908295643216      0.1409858170569629    0.14892276045500305
0.9458554186179374      0.1406931284315686    0.14846649935769454
0.9460293261745434      0.14038425161240645   0.14830488154702404
0.9461979664698454      0.14008510881021288   0.14837312986086035
0.9463613395038435      0.1397955883153457    0.14844757585930807
0.9465319473602182      0.13949358547850615   0.14831677743131097
0.9466931205940621      0.13920864344885334   0.14790303133571406
0.9468552776084422      0.13892231840254832   0.147263470193746
0.947024669445199       0.13862360366552234   0.14655142953634787
0.947188794020652       0.13833455615456794   0.14603272555462804
0.9473622370990948      0.13802950609810283   0.1457978439611253
0.9475304129162339      0.13773412472273142   0.14583119495053387
0.9476933214720689      0.13744837798106757   0.14592792364655588
0.9478634648502805      0.13715034737017087   0.14586959602223323
0.9480283409671881      0.13686194253763076   0.1455234572108056
0.948202535587086       0.1365576684153264    0.1448774802217428
0.9483714629456799      0.136263021338625     0.14416175054633124
0.9485351230429696      0.13597796546301985   0.14360652650761616
0.9487060179626361      0.13568073615185963   0.14332261270433827
0.9488674782597717      0.1354003209183013    0.1433198234345282
0.9490299223374437      0.13511859735375797   0.14342436793613683
0.9491996012374923      0.13482475872148003   0.14342278654077062
0.9493640128762368      0.13454046618690835   0.14315149829104387
0.9495377430179717      0.13424051838126388   0.14256304927875668
0.9497062058984024      0.13395011748694113   0.14185149747451495
0.9498694015175293      0.13366922508261503   0.14125020944996086
0.9500398319590326      0.13337633395306733   0.14089375573739532
0.9502008277780052      0.13310008576902196   0.1408391193005308
0.9503628073775141      0.1328225735621842    0.1409385637514057
0.9505320217993998      0.1325331239310014    0.14098805123273844
0.9506959689599814      0.1322531336316717    0.14079710894635508
0.9508692346235532      0.13195771354318664   0.14028101354619815
0.951037233025821       0.1316717531502596    0.13958901935189394
0.9511999641667847      0.13139521147692726   0.13894938908424817
0.951369930130125       0.13110685457896482   0.1385162838979967
0.9515346288321616      0.13082790466324992   0.1383964636666554
0.951708646037188       0.13053367918094655   0.13848681818435216
0.9518773959809106      0.130248860852371     0.13856510637231773
0.9520408786633292      0.12997340687674555   0.13842879032024902
0.9522115961681245      0.12968626455371174   0.13797662360713708
0.9523728790503888      0.12941546516822264   0.13733741150700535
0.9525351457131896      0.12914347059040648   0.13667639598889883
0.9527046471983669      0.1288597467732364    0.13618048849566577
0.9528688814222402      0.128585335698499     0.1359960678906885
0.9530424341491037      0.1282958895479395    0.1360560074739384
0.9532107196146633      0.1280157568805931    0.13616247371770857
0.9533737378189189      0.1277448926827079    0.13609649053850087
0.9535439908455511      0.12746253721517264   0.13572529785994464
0.9537048092496525      0.12719632945224477   0.13513016306385958
0.9538666114342901      0.1269289879632677    0.13445935194174663
0.9540356484413044      0.1266502265309749    0.13390257648533868
0.9541994181870148      0.1263806756586247    0.13364426663689055
0.9543704227551018      0.1260997716380505    0.13365597453096062
0.9545319927006579      0.12583489007983364   0.13377353866933117
0.9546945464267504      0.12556891368342393   0.1337851583543199
0.9548643349752196      0.12529165814229282   0.13351995224068156
0.9550288562623848      0.1250235522575254    0.13298036725141735
0.9552026960525402      0.1247408516782627    0.13226517841975494
0.9553712685813915      0.12446730076590426   0.13166409932894155
0.9555345738489388      0.12420284952545661   0.13134030546599948
0.9557051139388628      0.12392726663608399   0.13130084039949036
0.9558662194062559      0.12366748154856821   0.13141614549017117
0.9560283086541854      0.12340665500221779   0.13147597245632348
0.9561976327244915      0.12313477414934466   0.13129122670756557
0.9563616895334937      0.12287192744713347   0.13081900143826194
0.956535064845486       0.12259477190527451   0.13012560582492452
0.9567031728961742      0.12232665001191821   0.1294872373168545
0.9568660136855587      0.12206750886296501   0.1290939771061133
0.9570360892973195      0.12179746757170673   0.12898986104639776
0.9572008976477765      0.12153639111490847   0.1290916245897534
0.9573750245012236      0.12126120062686914   0.12918927525812
0.9575438840933667      0.12099497424690558   0.12905875836221475
0.9577074764242057      0.12073765695458516   0.1286384558837171
0.9578783035774215      0.12046959910418896   0.1279795015973736
0.9580396961081066      0.12021694956950357   0.1273461363101547
0.9582020724193278      0.11996335565656399   0.1268963443491346
0.9583716835529257      0.11969910459739083   0.1267301167154727
0.9585360274252197      0.11944368942451286   0.12680539666287888
0.9587096898005036      0.11917446949460919   0.12693128759440458
0.9588780849144837      0.11891402790519666   0.12687171760154894
0.9590412127671597      0.11866232055605613   0.12652664559956028
0.9592115754422125      0.1184001186043342    0.12591221067439914
0.9593725034947344      0.11815306903508922   0.12526810472629227
0.9595344153277927      0.11790513266368877   0.12476293754424451
0.9597035619832275      0.11764678981677432   0.12452531247942454
0.9598674413773584      0.11739715126400192   0.12455959908434504
0.960038555593866       0.11713718904818998   0.12469973996229856
0.9602002351878427      0.1168922188567722    0.1247156480057265
0.9603628985623558      0.11664640803285872   0.12446958504081103
0.9605327967592455      0.11639036468538767   0.12392739963883233
0.9606974276948312      0.11614294640380077   0.12327111911686726
0.960871377133407       0.11588226316344746   0.122676744519691
0.9610400593106788      0.11563020384247406   0.12237622111991434
0.9612034742266466      0.11538670541407138   0.12236590633969269
0.9613741239649912      0.11513315609716125   0.12250686840589849
0.961535339080805       0.11489431339455704   0.12257202117259182
0.961697537977155       0.11465469262127333   0.12240098803731787
0.9618669716958816      0.11440511577996311   0.12192522695249654
0.9620311381533044      0.11416401550981549   0.12128901795068488
0.9622046231137171      0.11391000229479172   0.12065903068629812
0.9623728408128259      0.11366446393851705   0.12029108455297552
0.9625357912506307      0.11342733420576342   0.12022394501581085
0.9627059765108122      0.11318043739267239   0.12035162624471128
0.9628667271484629      0.11294794631131665   0.12045878107766572
0.96302846156665        0.11271474082279895   0.12036570469121032
0.9631974308072135      0.11247186672225742   0.11996918898556111
0.9633611327864732      0.11223731187529472   0.11936861377615397
0.9635320695881096      0.11199318125555947   0.11872080496796845
0.963693571767215       0.11176327116095316   0.11829201857780294
0.9638560577268569      0.11153269623090131   0.11814187378857673
0.9640257785088753      0.11129264705102995   0.11823119549430393
0.9641902320295898      0.11106082432337351   0.11837410546593681
0.9643640040532943      0.11081670094440389   0.11835261212416119
0.9645325088156949      0.11058080149209012   0.11803169186828937
0.9646957463167916      0.11035305418583172   0.11747552463504664
0.9648662186402648      0.11011603495666615   0.11682084196019
0.9650272563412072      0.10989290890818036   0.11634176838029812
0.965189277822686       0.1096691236222816    0.11612762299362768
0.9653585341265415      0.10943615776731448   0.11617607583069428
0.9655225231690929      0.10921124909375114   0.11633107412878069
0.9656958307146344      0.10897442875403038   0.11637546830553706
0.9658638709988719      0.10874566314365845   0.11613848099385103
0.9660266440218055      0.1085248774953814    0.11564109925244626
0.9661966518671158      0.10829513356024996   0.11499324164294632
0.9663613924511221      0.10807334574642001   0.11445778094131605
0.9665354515381184      0.10783991385696359   0.11417149306921259
0.9667042433638108      0.10761443536242543   0.11418885978713833
0.9668677679281992      0.10739683293064997   0.11434598877438025
0.9670385273149643      0.10717049017183035   0.11443153326499263
0.9671998520791986      0.10695749010030935   0.11427003386557712
0.9673621606239691      0.10674401622109239   0.1138338791052501
0.9675317039911163      0.1065219153051943    0.11320639568064293
0.9676959800969596      0.1063075850329868    0.11264044095224685
0.9678695747057928      0.10608203309319188   0.1122901033077978
0.9680379020533222      0.1058642484677844    0.11225420621737836
0.9682009621395475      0.10565415040973898   0.11240164560720493
0.9683712570481496      0.10543565109812617   0.1125320428515374
0.9685321173342208      0.10523012532489937   0.112446310613026
0.9686939614008283      0.10502419829134704   0.11208303165638399
0.9688630402898125      0.10480998717502146   0.11149053240751493
0.9690268519174927      0.10460335183345866   0.1109030553677143
0.9691978983675495      0.10438854317499226   0.11049034112005318
0.9693595101950755      0.10418648186584421   0.11038480786606064
0.9695221058031378      0.1039840763416013    0.1105004007782094
0.9696919362335769      0.10377361800042613   0.11066860231300021
0.9698564994027119      0.1035706208929182    0.11066770470416774
0.970030381074837       0.10335713273669894   0.11036282181201014
0.9701989954856581      0.103151101610089     0.1098134942308665
0.9703623426351753      0.10295244080134523   0.10921992331851323
0.9705329246070691      0.1027459673064176    0.10875756000595975
0.970694071956432       0.10255184354771074   0.10859397427862218
0.9708562030863312      0.10235745092136175   0.10867570114962433
0.9710255690386073      0.10215536998407451   0.108859240745556
0.9711896677295793      0.10196053945543136   0.10891981002088741
0.9713630849235413      0.10175567379404643   0.10869817420387898
0.9715312348561994      0.10155802759399622   0.1082057445570055
0.9716941175275535      0.10136753820790236   0.10761765629843525
0.9718642350212844      0.10116960720816481   0.10710929360999374
0.9720290852537112      0.1009788031435282    0.10687896963805235
0.9722032539891281      0.10077828642849149   0.10693058614380516
0.9723721554632411      0.10058489186789901   0.10711955370992872
0.97253578967605        0.10039852680532793   0.10721850269682918
0.9727066587112356      0.10020497565650853   0.10705953491894928
0.9728680931238903      0.10002310519264783   0.10663767731817049
0.9730305113170814      0.09984110524260889   0.10606710128711616
0.9732001643326492      0.09965205144539478   0.10552881274175427
0.973364550086913       0.09946989858889      0.10524259016547544
0.9735382543441669      0.0992785282928505    0.10524509947500219
0.9737066913401168      0.09909405353925717   0.1054293801084107
0.9738698610747627      0.0989163781780985    0.10557285120314448
0.9740402656317853      0.09873191249755991   0.10549182716559183
0.9742012355662769      0.09855868522848266   0.10513773139631621
0.9743631892813049      0.09838540901039398   0.1045972058221803
0.9745323778187097      0.09820547867905009   0.10403654018658579
0.9746962990948105      0.09803221368527476   0.10369225240118358
0.9748674551932879      0.09785242368618564   0.10363328454151383
0.9750291766692344      0.09768360229389933   0.10378962534595088
0.9751918819257173      0.09751479596762627   0.10397339047654794
0.9753618220045768      0.09733960435108246   0.10398496647379558
0.9755264948221323      0.09717094005517349   0.10371304482115012
0.9757004861426779      0.09699391003701188   0.10317827664870888
0.9758692102019196      0.09682340103403622   0.1026100079340474
0.9760326669998574      0.09665931071143746   0.10222100963462473
0.9762033586201717      0.09648911350296399   0.10210704278566529
0.9763646156179552      0.0963294131118059    0.1022374279211922
0.976526856396275       0.09616981122659159   0.1024404820455027
0.9766963319969716      0.096004246132272     0.1025157624118874
0.9768605403363642      0.09584495620941835   0.10231954790463248
0.9770340671787467      0.0956778401153193    0.10183948806463917
0.9772023267598254      0.09551699227515777   0.10127369796952673
0.9773653190796001      0.09536230689537702   0.10084199054425765
0.9775355462217514      0.09520194462888272   0.10066562467395314
0.9776963387413719      0.09505159854693661   0.10075776968925629
0.9778581150415286      0.09490143911187915   0.100969001283607
0.9780271261640622      0.09474575081698848   0.10110418457520201
0.9781908700252917      0.09459607553466286   0.10099000990459825
0.9783618487088979      0.09444101143652925   0.10058511731121286
0.9785233927699731      0.09429565763592418   0.10005653660416443
0.9786859206115848      0.09415055452344769   0.09957983113652488
0.978855683275573       0.09400021357528031   0.09932118372992141
0.9790201786782573      0.09385573245465616   0.09935350579755935
0.9791939925839317      0.09370434996095264   0.09957412007061518
0.9793625392283022      0.0935588193716122    0.09975411606657882
0.9795258186113687      0.09341902870133097   0.09971181323722693
0.9796963328168118      0.09327430104905565   0.09937581692014885
0.979857412399724       0.09313876549290868   0.09887566452514991
0.9800194757631726      0.09300356775210011   0.09838026835425925
0.9801887739489978      0.09286358788500673   0.09806705401184412
0.9803528048735191      0.09272918937572375   0.0980468296219364
0.9805261543010304      0.09258847255465498   0.09825033687597876
0.9806942364672379      0.09245332856837796   0.09846776077780378
0.9808570513721413      0.09232364204599534   0.09849919737930217
0.9810271010994214      0.0921894819986344    0.0982429450730419
0.9811918835653974      0.09206073893968598   0.09777132915662042
0.9813659845343636      0.09192606946747676   0.09723378448129323
0.9815348182420259      0.09179680815353738   0.09689167054843148
0.9816983846883842      0.09167283683189931   0.09683805658439874
0.9818691859571191      0.09154470770439471   0.0970239328024897
0.9820305526033232      0.09142490488186908   0.09725608989018207
0.9821929030300636      0.09130560059536538   0.09734848728458094
0.9823624882791806      0.09118230082437394   0.09716681788034434
0.9825268062669936      0.09106412303510676   0.09674368298654058
0.9827004427577968      0.09094063069662314   0.09620811756614173
0.982868811987296       0.09082225091020252   0.09582393268885356
0.9830319139554913      0.09070886212807258   0.09571598602623628
0.9832022507460632      0.09059180131806935   0.09586882968447248
0.9833631529141041      0.09048250280604442   0.0961156844335218
0.9835250388626815      0.09037379400478193   0.0962669720550611
0.9836941596336355      0.09026157933094513   0.09616699813617911
0.9838580131432855      0.090154184820845     0.09580491477884796
0.9840291014753122      0.09004350650941416   0.09528973049025426
0.984190755184808       0.08994024462640549   0.09487719957661757
0.9843533926748402      0.08983764611466707   0.09469498395343982
0.9845232649872491      0.08973187157561453   0.09478936882596253
0.9846878700383539      0.08963073418525047   0.09504228059417397
0.9848617935924489      0.08952532774578437   0.09525929031442151
0.9850304498852398      0.08942454756605843   0.09523159592602633
0.9851938389167267      0.08932826582476394   0.09493128609038855
0.9853644627705904      0.0892291445999317    0.09444158314881437
0.9855256520019232      0.08913684483907526   0.09400826958326425
0.9856878250137924      0.08904530036784339   0.09377700334167195
0.9858572328480382      0.08895108890013774   0.09382418962066705
0.98602137342098        0.08886119214700851   0.0940672978588249
0.9861948324969119      0.08876767856970422   0.09432600715703604
0.9863630243115399      0.08867846826114374   0.09437389701863683
0.9865259488648639      0.08859343026482236   0.094145737248708
0.9866961082405645      0.0885060689845937    0.0936942904202028
0.9868610003549612      0.08842283269494021   0.09323762339972935
0.9870352109723478      0.08833641656796877   0.09295190456111696
0.9872041543284306      0.0882541136807102    0.09297054729110561
0.9873678304232093      0.08817579034000633   0.09320566976633993
0.9875387413403647      0.08809549438109571   0.09348698624325665
0.9877002176349894      0.08802103321903737   0.09359031862969822
0.9878626777101503      0.08794749762350078   0.09343017791442028
0.988032372607688       0.08787216899002027   0.0930233056966436
0.9881968002439215      0.08780062720033383   0.09256571397615078
0.9883705463831453      0.0877265851620302    0.09223845753001374
0.9885390252610649      0.08765631766007594   0.09220691383002755
0.9887022368776807      0.08758968786969173   0.09241811921689105
0.9888726833166731      0.08752162286631968   0.09272211801992443
0.9890336951331347      0.08745875456818542   0.09288723358258857
0.9891956907301326      0.08739690756549189   0.09280274417503497
0.989364921149507       0.08733380847522064   0.09245270302820759
0.9895288843075777      0.08727414946550274   0.0920039502146381
0.9897021659686384      0.08721268367702024   0.09163739046000298
0.989870180368395       0.0871546451099299    0.09154849408757856
0.9900329275068477      0.08709989383683664   0.09172363327219006
0.990202909467677       0.0870442553928524    0.09203910529096923
0.9903676241672025      0.0869918531898387    0.09226597270153654
0.9905416573697179      0.08693810834045984   0.09224411078208132
0.9907104233109294      0.0868875865620425    0.09194061547578211
0.9908739219908369      0.08684014515431991   0.09150870572852977
0.9910446554931212      0.08679218853094828   0.09112916292561872
0.9912059543728745      0.08674837296585741   0.09100233912127995
0.9913682370331642      0.08670575631204624   0.09114074125179618
0.9915377545158306      0.08666281466559958   0.09145715366631117
0.9917020047371929      0.0866227467738581    0.09173052273186012
0.9918755734615453      0.08658205708600418   0.09178673584657636
0.9920438749245937      0.0865442274314313    0.09155100255655577
0.9922069091263382      0.08650911201082798   0.0911466870592891
0.9923771781504594      0.08647405063503749   0.0907459833628398
0.9925380125520498      0.08644244923828506   0.09056901062035744
0.9926998307341764      0.08641214627986771   0.09066035242892137
0.9928688837386797      0.08638209109515436   0.09096546985569484
0.9930326694818791      0.08635453865150525   0.09127863890694651
0.9932036900474551      0.08632741900027267   0.09141581306903364
0.9933652759905003      0.08630334838625954   0.0912731174362228
0.9935278457140817      0.08628065816950628   0.0909167596266038
0.9936976502600399      0.08625859804934906   0.09050025707368682
0.993862187544694       0.08623882509145234   0.09025737234464169
0.9940360433323382      0.08621965133637474   0.09029782478395743
0.9942046318586785      0.08620275016227105   0.09058555453360775
0.9943679531237147      0.08618796991240553   0.09092590097786109
0.9945385092111276      0.08617421288368009   0.09113032600434195
0.9946996306760099      0.08616279585346745   0.09106017959274545
0.9948617359214282      0.08615286149377686   0.09075317732357695
0.9950310759892234      0.08614415116031905   0.09034123951606969
0.9951951487957145      0.08613734130057901   0.09006070775134993
0.9953685401051956      0.08613189252167588   0.09004625330459617
0.9955366641533727      0.08612832908894158   0.09030447627565973
0.995699520940246       0.08612649627907759   0.09066280859000092
0.995869612549496       0.086126287625372     0.09093309820417723
0.9960344368974419      0.08612775243615022   0.09093961534727978
0.996208579748378       0.08613108750870342   0.0906621014699547
0.99637745533801        0.08613608059694706   0.09026376999932764
0.9965410636663381      0.0861425741361817    0.08996729010235925
0.9967119068170428      0.08615109916403266   0.0899201863739505
0.9968733153452167      0.08616079491069852   0.09014365797746798
0.9970357076539269      0.08617216349148535   0.09051010150677913
0.9972053347850138      0.08618577150321377   0.09083294580011851
0.9973696946547967      0.08620065047961852   0.09091320191998
0.9975433730275697      0.08621818912771889   0.09070150283009006
0.9977117841390387      0.08623698274574149   0.090325411145239
0.9978749279892037      0.08625687070599779   0.09000526653005445
0.9980453066617454      0.08627941242550448   0.08990660990594061
0.9982062507117564      0.08630237285503643   0.09008996103896724
0.9983681785423035      0.08632711218834041   0.0904538041382909
0.9985373411952274      0.08635471668483782   0.09082428296417987
0.9987012365868473      0.08638318121691169   0.0909825347948852
0.9988723668008437      0.08641471304208152   0.09085197513512205
0.9990340623923093      0.08644621055229823   0.0905265079899643
0.9991967417643114      0.08647957486681457   0.0901844296519192
0.9993666559586901      0.08651622153753619   0.09001749286098835
0.9995313028917647      0.08655348974389869   0.09014228288391796
0.9997052683278296      0.08659475161862387   0.09051764452772924
0.9998754636888811      0.08663699791931632   0.09092639792746081
0.17399880519981992     0.9999993331881407    0.9999993247619844
0.17533258834339907     0.9999992674292383    0.9999992601574467
0.17681849486450113     0.9999991875420754    0.999999174754841
0.1781527166017624      0.9999991085460407    0.9999990958426491
0.17915146531450582     0.999999044460882     0.9999990532777758
0.17948322251371948     0.9999990221811325    0.9999990106564713
0.17965370230505096     0.9999990105338172    0.9999989928167442
0.1804833517187116      0.9999989518773598    0.999998963491877
0.18081834732282603     0.999998927240757     0.999998917619977
0.1809883626359945      0.9999989145224408    0.9999988962018098
0.1823193071416337      0.9999988102725438    0.9999987919056754
0.18332601633372766     0.9999987254439224    0.9999987342662173
0.1836548705444223      0.9999986964541487    0.9999986782764224
0.18381868963996073     0.9999986817696066    0.9999986575018238
0.1846587928675381      0.9999986038462002    0.999998616186126
0.18498671812190673     0.9999985722065495    0.9999985557226886
0.18515007273928213     0.9999985561819338    0.9999985301640152
0.18597951885359065     0.9999984720391959    0.9999984885301397
0.18632318459654065     0.999998435780639     0.9999984219591447
0.18648607473575302     0.9999984183018251    0.9999983916543888
0.18665619969734198     0.9999983998428996    0.9999983717994959
0.18731319845924632     0.9999983265662515    0.9999983463206248
0.1874840749622901      0.9999983069802192    0.9999983165975159
0.18781836090691956     0.9999982680208288    0.999998242301207
0.18798802139034548     0.9999982479181863    0.9999982165475604
0.18865983168450512     0.9999981668620468    0.9999981885992985
0.1888302437093859      0.9999981457232899    0.9999981590151432
0.18915526597523552     0.9999981047361477    0.9999980796767706
0.1893244619804984      0.9999980830463778    0.9999980478229116
0.1901643619087237      0.9999979716872176    0.9999979895077383
0.1904884552182473      0.9999979270257382    0.9999979044342702
0.1906571867453472      0.9999979033902451    0.9999978657265434
0.1908206510111431      0.9999978802386488    0.9999978439420572
0.1914947642827573      0.9999977820678724    0.9999978047379255
0.19183043071963546     0.9999977315281753    0.9999977120310743
0.19200286512979914     0.9999977051272327    0.9999976659305547
0.19216586491743204     0.9999976798937031    0.9999976384286808
0.1928297855539403      0.9999975742783852    0.9999976012682612
0.19333649296649302     0.9999974905316812    0.9999974523251511
0.19349902827596288     0.9999974630784411    0.9999974169305818
0.19416109099981904     0.9999973482200356    0.9999973788943295
0.1943243778507437      0.9999973192307186    0.999997340759625
0.19466015393604233     0.9999972589977749    0.9999972247268807
0.19482014108673573     0.9999972298331123    0.9999971799984604
0.19498944674041918     0.9999971986373961    0.9999971545731176
0.19549701534284725     0.9999971030263799    0.999997134190928
0.1956598377156089      0.9999970716792687    0.9999970987897677
0.1959946848445815      0.9999970061577987    0.9999969753329522
0.19615420751711182     0.9999969744363217    0.9999969216996873
0.19632304869263223     0.9999969405004706    0.999996887213212
0.19699158175516984     0.9999968024036868    0.9999968352384173
0.19749706020586427     0.999996693927413     0.9999966397528817
0.19766960426444852     0.9999966560767317    0.9999965955597988
0.1983279446918802      0.9999965077102178    0.9999965454653037
0.19849707293069238     0.9999964685653007    0.9999964887328979
0.1988320297080855      0.999996389769886     0.9999963375958556
0.1990041092885067      0.9999963486256817    0.9999962817658558
0.19916675424639707     0.9999963093178417    0.9999962573399477
0.19966059180328627     0.999996187432579     0.9999962291471465
0.19982925556393544     0.9999961449176363    0.9999961759459792
0.2001632833850025      0.9999960593638865    0.9999960130308276
0.20032864744542037     0.9999960163352016    0.9999959459474941
0.20048874424453428     0.9999959742468036    0.9999959096604869
0.200993690450615       0.9999958397453061    0.9999958824840651
0.201166057094328       0.9999957928664792    0.9999958326176894
0.20149915595906903     0.999995700799588     0.9999956602610416
0.20166405554132386     0.9999956544926181    0.9999955805877013
0.20182368786227473     0.9999956091978347    0.999995532266324
0.2023272406338663      0.9999954632528004    0.9999955026918539
0.20249914279941628     0.9999954123455363    0.9999954591396889
0.202995747811923       0.9999952621021485    0.9999951882001995
0.20315491565471083     0.9999952129309955    0.9999951254334946
0.20332340200048876     0.9999951603345765    0.9999950961989834
0.20382226163735995     0.9999950012552637    0.999995054013949
0.2039821810216026      0.9999949491818253    0.9999949841035288
0.2043153895347641      0.9999948389688593    0.9999947726676306
0.20448659498306954     0.9999947814290576    0.9999946867162662
0.2046587842119113      0.9999947229266796    0.9999946430697542
0.2051562504142934      0.9999945502937198    0.9999946065438136
0.20531570532037302     0.999994493804897     0.9999945420347114
0.2056479848772085      0.9999943742587644    0.9999943176365664
0.2058187258473509      0.9999943118560383    0.9999942139779312
0.20599045059802962     0.9999942484188021    0.9999941534680363
0.20648652336592263     0.9999940613065151    0.9999941178817855
0.20665801587751986     0.999993995269891     0.9999940552454633
0.20699353383925612     0.999993864651798     0.9999938166053562
0.2071638103312355      0.9999937975371224    0.9999936972495688
0.20732881956191085     0.9999937318146623    0.9999936230106911
0.20748856153128228     0.9999936675417361    0.999993595340289
0.2078214152147013      0.9999935315371785    0.9999935822587572
0.20799244324813548     0.9999934605460518    0.9999935285461572
0.20816445506210599     0.9999933883762766    0.9999934198839423
0.20849684426736198     0.9999932466979414    0.9999931472440186
0.2086613890198743      0.9999931754640065    0.9999930524208234
0.20882066651108272     0.9999931058083353    0.9999930096487493
0.2093231547934468      0.9999928814563406    0.9999929549218894
0.20948845108741393     0.9999928061013528    0.9999928577609944
0.20981782765573043     0.9999926536079438    0.9999925654589888
0.20998190793007976     0.9999925764661797    0.9999924448764316
0.21015322302680572     0.9999924950774175    0.9999923764463312
0.21065848523590763     0.9999922499256294    0.9999923260780149
0.21082331705171176     0.9999921682725674    0.9999922368936961
0.21115176466370222     0.9999920030637406    0.999991928484095
0.21131538045988849     0.9999919195056156    0.9999917850816153
0.21148623107845144     0.9999918313481557    0.999991692978507
0.2116580654775507      0.9999917417426862    0.9999916627379445
0.21199009985306422     0.999991565895493     0.9999916404112181
0.2121544671907053      0.9999914775121683    0.9999915629870036
0.2126618066093004      0.9999911990417621    0.999991064013516
0.2128321927497003      0.9999911035730146    0.999990950387812
0.2129973116287962      0.9999910101087276    0.9999909068308714
0.21332633336737022     0.9999908221962696    0.999990890354252
0.21349023622684826     0.9999907273137333    0.999990824745611
0.213661373908703       0.9999906272231833    0.9999906815023482
0.21399618221095423     0.9999904283522996    0.9999902977198425
0.2141661038731911      0.9999903258512605    0.9999901550398105
0.21433075827412396     0.9999902255046281    0.9999900895928729
0.21482228943768694     0.9999899198543449    0.9999900217068057
0.21499296264137863     0.9999898115529089    0.9999898888088292
0.21531850726485008     0.9999896018118216    0.9999894880623639
0.21548796444892393     0.9999894909690618    0.999989309470757
0.21565215437169377     0.9999893824700379    0.9999892114842825
0.21582357911684028     0.9999892680219251    0.9999891830224485
0.21615896154567513     0.9999890406152521    0.9999891430443081
0.2163291702712038      0.9999889234124616    0.9999890218449262
0.2166537859383492      0.9999886964902194    0.9999886017125917
0.21682277864425997     0.9999885765701692    0.9999883915932737
0.21698650408886683     0.9999884592076914    0.9999882622572809
0.2171574643558503      0.9999883354083166    0.9999882134683861
0.21749191782835905     0.9999880894776463    0.9999881864713425
0.21766166207572468     0.9999879627455438    0.9999880820056529
0.21815387701429173     0.9999875878359581    0.9999874099530027
0.21831713798073554     0.9999874610126317    0.9999872434724173
0.21848763376955596     0.9999873272370258    0.9999871659155655
0.2188128235632849      0.999987068269429     0.9999871459156426
0.2189821033324875      0.99998693145946      0.9999870674143787
0.21914611584038607     0.9999867975846067    0.9999868927211986
0.2194895942814729      0.9999865138964019    0.9999863496060136
0.21965239076975368     0.9999863779095465    0.9999861454351735
0.21982242208041108     0.9999862344581713    0.9999860348091181
0.22031549820885354     0.9999858101054865    0.9999859491166083
0.2204790462385891      0.9999856665600094    0.9999857899032255
0.22082159572334992     0.9999853613047907    0.9999852160846319
0.22098392773346762     0.9999852144417516    0.999984970441502
0.221153494565962       0.9999850594990846    0.9999848180079357
0.2213177941371524      0.9999849078593848    0.9999847718941142
0.22166184670482286     0.9999845854388852    0.9999847234526941
0.22182493025639538     0.9999844302723546    0.9999845748093062
0.22232008359433106     0.9999839497722117    0.999983703639599
0.2224891859486624      0.9999837823899319    0.9999835067881652
0.22265302104168977     0.9999836186022039    0.9999834320583726
0.22299614465303413     0.9999832703511331    0.9999833972096519
0.2231587637264436      0.9999831028011791    0.9999832706827462
0.22332861762222977     0.9999829260593628    0.9999830087791596
0.22365252362989016     0.9999825840406573    0.9999823502462876
0.22382116150605846     0.9999824033578489    0.9999821014701865
0.22398453212092279     0.9999822265918963    0.9999819874214908
0.22432047573410074     0.9999818577031443    0.9999819562179522
0.2244805466487338      0.9999816793503584    0.9999818636172096
0.22464993606635691     0.9999814887784856    0.9999816274532337
0.22498541520137183     0.9999811057133564    0.9999808979278123
0.225157755960604       0.9999809059840201    0.9999805882054147
0.22532066209730528     0.9999807153324942    0.9999804299638557
0.22549080305638322     0.9999805142726557    0.9999803883604972
0.2258152831906272      0.9999801269929309    0.9999803165884142
0.2259842081300873      0.9999799228850863    0.9999801029685456
0.22631875830877618     0.9999795126462067    0.9999793384423102
0.2264906345898453      0.9999792987353558    0.9999789707120409
0.22665307624838354     0.9999790945748005    0.9999787574550133
0.22682275272929847     0.9999788792336682    0.9999786816894856
0.227158805990897       0.9999784463591087    0.9999786279018438
0.22733143381342094     0.99997822065448      0.9999784264143631
0.22783021579684953     0.9999775555263977    0.999977223736512
0.2279901092966113      0.9999773381582262    0.9999769572431856
0.22815932129936317     0.9999771058938189    0.9999768396234398
0.2284944456046356      0.999976639048908     0.9999767989453437
0.2286666089489965      0.9999763956358894    0.9999766289159642
0.22882933767082653     0.9999761633006261    0.9999762954094921
0.22916399749793598     0.9999756785018953    0.9999753630366491
0.22932342651953472     0.9999754442021314    0.9999750305979974
0.22949217404412356     0.9999751938283705    0.9999748566273184
0.22982636939306994     0.9999746906728432    0.9999748130122269
0.22999806825926777     0.9999744283498739    0.999974681204995
0.23016033250293483     0.9999741780326624    0.9999743733878864
0.23049406337371814     0.9999736557634505    0.9999733802998119
0.23066553000083445     0.9999733834927204    0.9999729494614714
0.23083798040848716     0.9999731069356552    0.9999727149273622
0.23100099619360898     0.9999728429764486    0.9999726588931641
0.23117124680110748     0.9999725646538444    0.9999726542097103
0.23133623014730204     0.9999722923399067    0.9999725555622817
0.23149594623219255     0.9999720262609915    0.9999722789102916
0.23182874814664986     0.9999714639785712    0.9999712416584711
0.23199975029560316     0.9999711723264324    0.9999707343660751
0.23217173622509282     0.9999708764684104    0.9999704218341038
0.2323342875320516      0.9999705941481       0.9999703216169584
0.23266859252941857     0.9999700051898366    0.9999702533235119
0.23282784413614604     0.9999697206340764    0.9999700164445618
0.23333025476506752     0.9999688056778412    0.9999683748782779
0.23349552517455385     0.9999684988862934    0.999967980094961
0.23365552832273617     0.9999681990877382    0.9999678098183734
0.233988904363777       0.999967565530293     0.9999677586944428
0.23416019357602208     0.9999672352671833    0.999967559252796
0.23433246656880352     0.9999668998214593    0.9999671028920526
0.2346653781316813      0.999966242133125     0.9999658368051725
0.2348301840630046      0.9999659118835018    0.9999653513950659
0.23498972273302388     0.9999655892118325    0.9999651071823685
0.23515857990603325     0.9999652444755267    0.9999650489955133
0.23532216981773868     0.9999649073101039    0.9999650403566868
0.2354929945518207      0.9999645518614056    0.9999648902062281
0.2356648030664391      0.9999641908609297    0.999964472985765
0.23599678567299087     0.9999634832350749    0.9999631345068642
0.23616112712615112     0.9999631279725146    0.9999625495579675
0.236332703401688       0.9999627535222524    0.9999621960157188
0.2365052634577612      0.9999623732390287    0.9999620960129324
0.23683874914722264     0.9999616277224341    0.9999619750725047
0.23700384214183767     0.9999612534317331    0.9999616048409152
0.23749668908644717     0.9999601151834981    0.9999595413916162
0.237667800883821       0.9999597125806305    0.9999590823721963
0.23783989646173118     0.9999593037590755    0.9999589149679272
0.23817245319486652     0.9999585031382358    0.9999588334418978
0.2383370817113185      0.9999581032582002    0.9999585181719697
0.23882853522143888     0.9999568873492597    0.999956340490848
0.23899918254064967     0.9999564572517231    0.9999557581755666
0.2391645625985565      0.999956036482822     0.9999555012387136
0.23932467539515936     0.9999556253786672    0.9999554621545999
0.2394941066947523      0.9999551863077163    0.9999554344113945
0.2396582707330413      0.9999547568866465    0.9999552015800622
0.2398296695937069      0.9999543043004501    0.9999546220495699
0.24016500025358        0.9999534061630865    0.9999528989898924
0.24033518309462776     0.9999529438517433    0.9999521934368834
0.24050009867437155     0.9999524916236091    0.9999518331704402
0.2406597469928114      0.9999520498424076    0.9999517475701097
0.2408287138142413      0.9999515779547717    0.9999517392170402
0.24099241337436728     0.9999511165039994    0.9999515673922416
0.24116334775686987     0.9999506301312545    0.9999510432204367
0.2414977494604169      0.9999496651025349    0.9999492377691644
0.24166746782330167     0.9999491683943975    0.9999483982580718
0.24183191892488243     0.9999486826015741    0.9999479090426657
0.24199110276515923     0.9999482081151612    0.9999477492349653
0.24232284019038908     0.9999472057003913    0.9999476339713718
0.24249331009472863     0.9999466833685544    0.9999471869887994
0.24298770200421754     0.9999451403473506    0.999944410220034
0.2431516886276353      0.9999446191631367    0.9999437483858
0.24332291007342968     0.999944069937241     0.9999434543634653
0.2436578859035603      0.9999429803700649    0.999943364218682
0.2438278913297368      0.9999424196904494    0.999942993493689
0.24399262949460937     0.9999418713813818    0.9999422267551239
0.2443208898047366      0.9999407639889341    0.9999400972907064
0.24448441194999132     0.9999402064970058    0.9999392795965374
0.24465516891762265     0.9999396204220083    0.9999388496765131
0.24482690966579032     0.9999390254165093    0.9999387670657905
0.24498921579142718     0.999938457928732     0.9999387457453843
0.24515875673944065     0.9999378597297042    0.9999384651287427
0.24532303042615017     0.9999372747877099    0.9999377662251531
0.24566703122485883     0.9999360326749598    0.9999354047206092
0.24583008889195052     0.9999354356734521    0.9999344515964588
0.24600038138141883     0.9999348064589546    0.9999338965234873
0.24616540660958316     0.9999341910764906    0.9999337536383387
0.24632516457644354     0.9999335900100148    0.9999337487944867
0.24649424104629403     0.9999329481274324    0.9999335472420172
0.2466580502548405      0.9999323205467455    0.999932922616765
0.24700112209722314     0.999930987806424     0.9999304696629265
0.24716371528615177     0.9999303473890424    0.9999293515447045
0.24733354329745705     0.9999296723723108    0.999928617413734
0.2474981040474584      0.9999290122929636    0.9999283665055477
0.24782600952784317     0.9999276792146574    0.9999282329802962
0.2479893542582266      0.9999270062079816    0.999927708052421
0.24815993381098672     0.9999262969785754    0.9999266384410475
0.24848529113259504     0.9999249258568879    0.9999240005681427
0.24865465466573727     0.9999242024880328    0.9999230307899605
0.24881875093757555     0.9999234952572783    0.9999226031502966
0.24899008203179046     0.9999227501143602    0.9999225387877195
0.24916239690654174     0.9999219937025503    0.9999224761148285
0.24932527715876213     0.9999212722106584    0.9999220470940338
0.2494953922333592      0.9999205118802028    0.9999210366965824
0.24981982059864144     0.9999190424186852    0.9999182129071162
0.24998871965362063     0.9999182672203075    0.9999170334548078
0.25015235144729586     0.9999175094782622    0.999916427907576
0.2503232180633478      0.9999167111236356    0.999916280132739
0.250495068459936       0.9999159007918539    0.9999162601034551
0.25065748423399337     0.9999151280942735    0.9999159419886419
0.2508271348304274      0.999914317643518     0.9999150291656402
0.25116313632306414     0.999912690876903     0.9999119564790444
0.2513357382611072      0.999911843770432     0.9999105573801014
0.2514989055766194      0.9999110357355175    0.9999097912117664
0.2516693077145083      0.9999101842995938    0.999909553519279
0.2518344425910932      0.9999093517335665    0.9999095473288564
0.2519943102063741      0.9999085386769564    0.9999093249596874
0.2521634963246451      0.9999076706114544    0.9999085153601504
0.25267070632083577     0.9999050205724718    0.9999037262228
0.2528334091581849      0.9999041551401489    0.9999027325204193
0.25300334681791076     0.99990324314837      0.9999023384609325
0.2533274203534505      0.9999014808391411    0.999902186151076
0.2534961419935585      0.9999005511880069    0.9999015118271468
0.2536595963723625      0.9998996425429736    0.9999001881450297
0.25400195855526014     0.9998977134978702    0.9998964866240676
0.25416419691444625     0.9998967870114653    0.9998952288010815
0.25433367009600905     0.9998958106363109    0.9998946195600463
0.25449787601626783     0.9998948561855621    0.9998945302626094
0.2546693167589033      0.9998938507677381    0.9998944629847606
0.25484174128207515     0.9998928303227044    0.9998938760526791
0.2550047311827161      0.9998918571108009    0.9998925973175552
0.25533991336744744     0.999889829193224     0.9998887112718136
0.25549960356785706     0.9998888503473715    0.999887214996002
0.2556686122712568      0.9998878053734325    0.9998863648310926
0.25583235371335256     0.9998867840663893    0.999886167248282
0.256003329977825       0.9998857082154025    0.999886147766258
0.2561752900228338      0.9998846163971987    0.9998857113930059
0.2563378154453117      0.9998835754028209    0.999884555579001
0.25667206867371695     0.9998814064722267    0.9998804936872857
0.2568312943959635      0.9998803599013247    0.999878719091348
0.25699983862120024     0.9998792446937359    0.9998775709521858
0.25716311558513294     0.9998781577952882    0.9998772010296292
0.25733362737144233     0.9998770127913817    0.9998771876631226
0.2574988718964477      0.999875893375405     0.9998769304124995
0.2576588491601492      0.9998748003863344    0.9998759941596449
0.25816343675999254     0.9998712924335029    0.9998696871034463
0.25833568386829314     0.9998700735964801    0.999868183130123
0.25849849635406286     0.9998689113821176    0.9998676023282745
0.2586685436622092      0.9998676869137143    0.9998675439914224
0.25883332370905154     0.9998664899450707    0.9998674019678636
0.25899283649459        0.9998653213794118    0.999866633228205
0.2591616677831185      0.9998640738962862    0.9998649047872803
0.2594960306599442      0.9998615706292184    0.9998600834593527
0.25966781329008176     0.99986026747267      0.9998582172985623
0.25983016129768843     0.9998590251197408    0.9998573544255095
0.2599997441276718      0.9998577161391075    0.9998571844896176
0.26016405969635115     0.9998564367504744    0.9998571340716009
0.2603356100874072      0.99985508931759      0.9998564798812328
0.26050814425899954     0.9998537219938133    0.9998547912212356
0.26084157817949916     0.9998510446187965    0.9998496775499381
0.26100664528963335     0.9998497019672035    0.9998475619019718
0.26116644513846354     0.9998483911921496    0.9998464114224912
0.26133556349028386     0.9998469921424463    0.9998460765005036
0.26149941458080017     0.9998456249702666    0.9998460661178576
0.26167050049369317     0.9998441850541051    0.9998455967317181
0.2618425701871225      0.9998427240098866    0.9998440850907703
0.262175075151296       0.9998398638432707    0.9998387770311674
0.26233967778326717     0.9998384297980825    0.9998362953765304
0.2624990131539343      0.9998370300834711    0.9998347739086214
0.2626676670275916      0.9998355360344537    0.9998341891121708
0.2630016750746748      0.9998325389544659    0.9998338806634671
0.26316702924810076     0.9998310362502646    0.9998326745411957
0.263660659729143       0.9998264919844275    0.9998245207004305
0.26383203270532773     0.99982488778206      0.9998224003090485
0.2640043894620488      0.9998232601520749    0.999821483023916
0.2641673115962391      0.9998217083925314    0.9998213813795558
0.26433746855280593     0.9998200738852295    0.9998212352031832
0.2645023582480689      0.9998184763706699    0.9998202511236307
0.26466198068202784     0.9998169170260931    0.9998181669890719
0.264994595294622       0.9998136266322774    0.9998118384221478
0.26516550379264375     0.9998119140812625    0.9998092561350658
0.2653373960712018      0.9998101765557668    0.9998079390532828
0.26549985372722906     0.9998085203492407    0.9998076777520525
0.2656695462056329      0.9998067756910355    0.9998076326785161
0.2658339714227328      0.9998050707511991    0.9998069038639251
0.2659931293785287      0.9998034067901403    0.9998050337474609
0.26632481503479677     0.999799895536575     0.9997984405606297
0.2664952590546555      0.9997980680714605    0.9997953627427122
0.2666604358132102      0.9997962819458206    0.9997935876863907
0.26682034531046095     0.9997945384783089    0.9997930246758273
0.2669895733107018      0.9997926779779887    0.9997930075323415
0.2671535340496386      0.9997908601364512    0.9997925892523581
0.2673247296109521      0.9997889459457172    0.9997909044856881
0.26782963321381664     0.9997832032730811    0.999780565251078
0.2679943454942083      0.9997812980836762    0.9997782867573737
0.2681537905132961      0.9997794387474596    0.9997773803864236
0.26832255403537386     0.9997774544802188    0.999777289709271
0.2684860502961477      0.9997755160812669    0.9997770679778843
0.26865678137929816     0.9997734749155508    0.9997756945004963
0.26882849624298494     0.9997714043558722    0.9997727404837609
0.2691602915476735      0.9997673529654758    0.9997649470555443
0.2693245393499022      0.9997653225332018    0.9997620932702292
0.2694960219745075      0.9997631865129314    0.9997606731886687
0.26966848837964913     0.9997610264168318    0.9997604501229242
0.26983152016226        0.9997589673612977    0.9997603359591439
0.2700017867672474      0.9997567990098414    0.9997591786911478
0.2701667861109309      0.9997546801199939    0.9997564827477528
0.27049556877868        0.9997504056778308    0.9997482833534473
0.2706593521027456      0.9997482500994515    0.9997448689055188
0.2708303702491879      0.9997459804603656    0.999742911139767
0.27100237217616646     0.9997436781897341    0.9997424301601683
0.27116493948061426     0.9997414840100067    0.9997424072243803
0.27133474160743865     0.9997391731517519    0.999741574509553
0.27149927647295913     0.9997369152816102    0.999739173984998
0.2718271301843821      0.9997323607960047    0.9997306927442603
0.2719904490302847      0.9997300641731856    0.9997266864637939
0.2721610026985639      0.9997276458586414    0.9997240735202666
0.27232628910553913     0.9997252826296517    0.9997231982577611
0.2726556458998718      0.999720515455343     0.9997227169699702
0.27281971628722923     0.9997181115106586    0.9997207995910729
0.2729910214969633      0.9997155806656925    0.9997169327610804
0.2733261648549732      0.9997105669497124    0.9997073305707885
0.27349625404508937     0.9997079905550662    0.9997040476414359
0.27366107597390155     0.9997054732725148    0.9997026962062525
0.2738206306414098      0.9997030169160348    0.999702542553808
0.2739895038119081      0.9997003960367877    0.9997023223951291
0.2741531097211025      0.9996978361029443    0.9997007866448262
0.27432395045267355     0.9996951409625747    0.9996971767334373
0.27465816485435746     0.9996898028731026    0.99968692021542
0.2748277895663106      0.9996870600927161    0.999682882582202
0.27499214701695984     0.9996843807529674    0.9996809044331703
0.2751637392899857      0.9996815604965038    0.9996804887850465
0.27533631534354785     0.9996787002370466    0.999680386656444
0.2754994567745792      0.9996759742200652    0.9996791093575542
0.2756698330279872      0.9996731042029781    0.9996756814164403
0.27599478375089126     0.9996675799732145    0.9996651526237647
0.27616394398468136     0.9996646713702138    0.9996603884802381
0.27632783695716756     0.9996618304506398    0.9996577334910631
0.2764989647520304      0.9996588398896536    0.9996569378349385
0.2766710763274295      0.9996558069611045    0.9996569170451494
0.2768337532802978      0.9996529168901055    0.9996560220841584
0.27700366505554275     0.9996498738138732    0.9996529977867882
0.2774963825777478      0.9996409061296491    0.9996366688551332
0.27765981107207094     0.9996378840036124    0.99963320185315
0.27783047438877073     0.9996347024305596    0.9996318434370776
0.2781559992382583      0.9996285604457741    0.9996313204061275
0.2783254465353402      0.9996253248162982    0.9996289380960085
0.27848962657111814     0.9996221643706338    0.9996242880173472
0.2788334400679636      0.9996154642163801    0.9996114359242563
0.27899640408412374     0.9996122493605248    0.9996071500946391
0.2791666029226605      0.9996088647041081    0.9996051353511454
0.27933153449989323     0.9996055582131194    0.9996048787850895
0.2794911988158221      0.9996023322186949    0.9996046760675713
0.27966018163474093     0.999598890861459     0.9996027886664852
0.2798238971923559      0.9995955300260023    0.9995984791771978
0.28016678173287535     0.9995884050170816    0.9995848529180595
0.2803292812708724      0.9995849872553053    0.9995796542153994
0.28049901563124613     0.9995813888563404    0.9995767922423113
0.2806634827303158      0.9995778741862422    0.9995761643184109
0.2808226825680816      0.9995744457128615    0.9995761280813605
0.28099120090883745     0.9995707881157866    0.9995747889071378
0.28115445198828937     0.9995672167644254    0.9995709850583765
0.28149015653064247     0.9995597851484339    0.9995571507326968
0.281650107909863       0.999556202426392     0.9995509808017282
0.28181937779207367     0.9995523813321722    0.9995469330757404
0.2819833804129804      0.9995486498767726    0.9995455962374699
0.2821546178562637      0.9995447343919188    0.9995455544602181
0.2823268390800834      0.9995407651769155    0.9995446693243898
0.28248962568137226     0.9995369835389498    0.9995413846094733
0.28298388816845677     0.9995253213175936    0.9995203309843442
0.2831526935725044      0.9995212754149744    0.9995152056374561
0.28331623171524806     0.9995173248209024    0.9995131000350624
0.28348700468036836     0.9995131667091344    0.9995128696730828
0.283658761426025       0.9995089506404393    0.9995124146585714
0.2838210835491508      0.9995049345805431    0.9995097723711563
0.2839906404946533      0.9995007065045676    0.9995038766772111
0.28432645468542683     0.9994922320730361    0.9994875490680841
0.28449896297253835     0.9994878262672061    0.9994813190183588
0.284662036637119       0.999483628349984     0.9994785029350672
0.2848323451240763      0.9994792096316196    0.9994779980677562
0.28499738634972965     0.9994748936466895    0.9994778084853512
0.285157160314079       0.9994706833691727    0.9994757664148693
0.28532625278141843     0.9994661929688868    0.9994703531049984
0.2856611380158659      0.9994571939101103    0.9994532128061239
0.2858331818248144      0.9994525155089223    0.9994457825831204
0.28599579101123196     0.9994480588874641    0.9994419042652729
0.28616563502002623     0.9994433676487181    0.999440838074383
0.28633021176751655     0.999438786243996     0.9994408142435485
0.28648952125370286     0.9994343178210523    0.9994394052402701
0.28665814924287925     0.9994295517304085    0.9994346770792348
0.28699210552100063     0.9994200016830426    0.9994171050431344
0.28715743380994574     0.9994152187062364    0.9994086876587389
0.28731749483758684     0.9994105530273943    0.9994034385814552
0.28748687436821807     0.9994055778903236    0.9994013630248856
0.28782233372924926     0.9993956087463478    0.9994004264951145
0.2879946646014895      0.9993904270900484    0.9993963290224366
0.28849255573406685     0.999375243685599     0.9993691258869475
0.2886521522835449      0.99937031014151      0.9993625894851776
0.2888210673360131      0.9993650488403837    0.9993594607958209
0.2889847151271773      0.9993599123746586    0.999359060801544
0.2891555977407182      0.999354507308691     0.9993586800776568
0.2893274641347954      0.9993490280157598    0.9993554305712058
0.2894898959063418      0.9993438094705133    0.9993484656549746
0.28982396183288384     0.9993329532177567    0.9993275751774577
0.2899955959878795      0.9993273102975428    0.9993191328877186
0.2901682139234115      0.9993215899071327    0.9993149000600938
0.29033139723641266     0.999316140254604     0.9993141159558175
0.29050181537179043     0.9993104051860179    0.9993139619447367
0.2906669662458643      0.9993048043746666    0.9993114605779628
0.29082684985863416     0.9992993415887984    0.9993051643716618
0.2911599868288502      0.9992878296963007    0.9992834270361096
0.29133115650568286     0.9992818460553621    0.9992735188693699
0.29150330996305185     0.9992757805511275    0.9992678419453818
0.29166602879788994     0.999270003389817     0.9992663140305266
0.29183598245510467     0.9992639233262598    0.9992662970150608
0.2920006688510155      0.9992579865238049    0.9992645916995686
0.2921600879856223      0.9992521969695486    0.9992591295128548
0.2926630011981819      0.9992336550410367    0.9992256180021268
0.29283469017738784     0.9992272273636614    0.9992182189983596
0.2929969445340629      0.9992211067113703    0.9992155940821309
0.29316643371311457     0.9992146649041999    0.9992154415222849
0.29333065563086236     0.9992083759767348    0.9992145022044622
0.2935021123709868      0.9992017599818315    0.999209545616648
0.29367455289164757     0.9991950541701995    0.99919947143763
0.294007799510284       0.9991819463760276    0.999174306594884
0.29417277296948663     0.9991753842893767    0.999165657739954
0.29433247916738525     0.9991689852367011    0.9991618393887095
0.294501503868274       0.9991621627077812    0.9991612681216107
0.2946652613078588      0.9991555177818774    0.9991608627125333
0.2948362535698202      0.9991485299959344    0.9991569541144683
0.2950082296123179      0.9991414477018788    0.9991476193867996
0.29534054727462833     0.9991276064027275    0.9991209589000191
0.2955050562556679      0.9991206777224624    0.9991104663677425
0.2956642979754035      0.99911392201342      0.9991050403789318
0.2958328581981292      0.9991067182241399    0.9991036666110477
0.2959961511595509      0.9990996874469956    0.9991036071214062
0.29616667894334925     0.9990922900494501    0.999100850074146
0.2963319394658437      0.999085067036445     0.9990930113496856
0.29666124449121456     0.9990705138879795    0.9990657656039358
0.2968252889940911      0.9990631838140757    0.9990531381146576
0.2969965683193443      0.9990554729074742    0.999045094035725
0.29716883142513384     0.9990476579968688    0.9990425324542329
0.2973316599083925      0.9990402156662268    0.9990425183212238
0.2975017232140278      0.9990323847516078    0.9990407169337797
0.2976665192583592      0.9990247395612043    0.9990339733204461
0.2981584753521175      0.999001580514217     0.9989921132120447
0.2983292901992076      0.9989934202142824    0.9989819080776295
0.2985010888268341      0.998985150315833     0.9989777846230793
0.29866345283192974     0.9989772764964034    0.9989774762422352
0.298833051659402       0.9989689911732289    0.9989765604035031
0.29899738322557035     0.9989609036808873    0.9989711589669774
0.29916894961411533     0.9989523972605542    0.9989591446936139
0.29950461532974715     0.998935567339942     0.9989267417949105
0.29967496569867424     0.9989269305293578    0.9989147064031587
0.29984004880629733     0.9989184988348844    0.9989092309872835
0.2999998646526165      0.9989102777413694    0.9989083408141861
0.30016899900192573     0.9989015143151327    0.9989080031605301
0.30033286608993104     0.9988929616676813    0.9989038315807798
0.300503968000313       0.9988839657596579    0.9988928343705652
0.3008387047596187      0.9988661764585748    0.9988588730509855
0.30100859065038277     0.9988570667840758    0.9988445080272251
0.3011732092798429      0.9988481748472828    0.9988369218912314
0.30133256064799907     0.9988395063323667    0.9988349800744398
0.30150123051914524     0.9988302649715333    0.9988349583625558
0.3016646331289875      0.9988212470922897    0.9988321079152599
0.3018352705612064      0.9988117610593119    0.9988225355837009
0.3021607436417323      0.998793470531165     0.9987887181281078
0.3023301650543333      0.9987838464112966    0.9987716986419318
0.3024943192056304      0.9987744535054504    0.9987612122205106
0.30266570817930416     0.998764574721498     0.9987572897358168
0.30283808093351416     0.9987545646410848    0.9987571797463627
0.3030010190651934      0.9987450332091535    0.9987553966377182
0.30317119201924925     0.9987350062360968    0.9987472458469945
0.303664693077887       0.9987055056137555    0.9986942118032227
0.3038283827510211      0.9986955802886247    0.9986811504925764
0.3039993072465318      0.9986851409196761    0.998675164034725
0.30417121552257886     0.9986745633073907    0.9986745174112045
0.30433368917609505     0.9986644937029515    0.9986736928342513
0.30450339765198786     0.9986538999884965    0.9986672484394022
0.30466783886657667     0.9986435608100701    0.998653282583667
0.3049955052761365      0.9986227391881922    0.9986131541956497
0.30515873047110753     0.9986122569469117    0.9985972551200122
0.3053291904884552      0.9986012313215463    0.9985886050075902
0.30549438324449885     0.9985904691419606    0.9985868002751487
0.3056543087392386      0.9985799772479237    0.9985867129014868
0.3058235527369684      0.9985687954715401    0.9985824884688291
0.30598752947339425     0.9985578842090382    0.9985706299694399
0.3063309363715355      0.9985347843073603    0.9985273637282038
0.3064936970883435      0.9985237172050908    0.9985088045520225
0.30666369262752813     0.9985120760159952    0.9984973324906311
0.30682842090540874     0.9985007149919032    0.9984939437692671
0.30698788192198545     0.9984896413066092    0.9984939633013458
0.3071566614415522      0.998477852603191     0.9984912713836928
0.307320173699815       0.9984663615969446    0.9984811959011665
0.30782494788027515     0.9984303807686427    0.9984163069398269
0.3079944789412967      0.9984181225471731    0.998401521839868
0.3081587427410143      0.998406160841856     0.9983958561009953
0.3083302413631086      0.9983935831217307    0.998395457981956
0.3085027237657392      0.9983808407493662    0.9983937074991887
0.30866577154583896     0.9983687095080792    0.9983848965287344
0.30883605414831533     0.9983559502970164    0.9983662126237284
0.30916081756935626     0.9983313597094886    0.9983186580327824
0.3093298841522148      0.9983184240671938    0.9983006089447817
0.3094936834737694      0.9983058030699932    0.9982923501270964
0.30966471761770054     0.9982925312118778    0.9982909991241853
0.30983673554216806     0.9982790861496116    0.9982903021558333
0.3099993188441048      0.998266288642171     0.9982834845539843
0.3101691369684182      0.9982528278255418    0.9982664038568684
0.31049297143313304     0.9982268908477222    0.9982165943245846
0.31066157353782853     0.9982132466822582    0.9981950016825459
0.3108249083812201      0.9981999364461598    0.9981834780669576
0.31099547804698835     0.9981859390659183    0.9981804063610161
0.31116078045145257     0.9981722782411766    0.9981803902922063
0.3113208155946129      0.9981589624444099    0.9981761162270424
0.31149016924076317     0.998144774018073     0.9981618300256516
0.3118255768328325      0.9981163760572456    0.9981093068279989
0.31199788182059185     0.9981016325347566    0.9980837410212405
0.3121607521858204      0.998087598839759     0.9980688300273095
0.3123308573734256      0.9980728400480476    0.998063538676981
0.31249569529972676     0.9980584385501894    0.9980634907855931
0.31265526596472404     0.9980444032273654    0.9980608939558948
0.3128241551327113      0.9980294469700614    0.9980489001719722
0.31333047427805083     0.9979839791584576    0.9979675092686041
0.31349288016511634     0.9979692204384875    0.9979486812006265
0.3136625208745585      0.9979536983468228    0.9979403145361447
0.3138268943226966      0.9979385543063021    0.9979395384830766
0.3139985025932115      0.9979226338670235    0.9979380904758257
0.31417109464426257     0.9979065082351191    0.9979275611729175
0.31433425207278287     0.9978911583535769    0.997906063136236
0.31466976931327273     0.9978592670520395    0.9978445324276188
0.3148296270415617      0.997843916901324     0.9978222509952209
0.3149988032728408      0.9978275617328525    0.9978104754143012
0.3151627122428159      0.9978116070055298    0.9978083470116502
0.3153338560351677      0.997794833082706     0.9978079240420493
0.3155059836080558      0.9977778435297237    0.9977999153744843
0.31566867655841313     0.9977616746169502    0.9977804246669867
0.316003264842577       0.9977280816720198    0.9977164316560316
0.31616265809270294     0.9977119160065538    0.9976901115110908
0.316331369845819       0.9976946901368694    0.9976741183721088
0.31649481433763105     0.9976778884977682    0.9976697399009423
0.31666549365181984     0.9976602231257491    0.997669780211332
0.3168371567465449      0.99764233151748      0.9976644307138548
0.31699938521873916     0.9976253078787299    0.9976476657285221
0.31733304454657696     0.9975899398224787    0.9975827653125992
0.3175044754022205      0.997571580785416     0.9975503271490216
0.3176768900384004      0.9975529871670615    0.9975300477667595
0.3178398700520494      0.9975352911595875    0.9975234260482265
0.31801008488807514     0.9975166845558058    0.9975232963164044
0.3181750324627968      0.9974985311420771    0.9975199608908064
0.3183347127762146      0.9974808418385146    0.9975059859995634
0.3185037115926224      0.9974619956994897    0.9974774754805699
0.3188384095252068      0.9974242906310834    0.9974050228993336
0.31901035968322367     0.9974047215265808    0.9973796962682195
0.31917287521870963     0.9973861015599453    0.9973694268031928
0.31934262557657234     0.997366522689046     0.9973682841161522
0.319507108673131       0.9973474239613924    0.9973668053679123
0.3196663245083857      0.9973288306034899    0.9973559517665315
0.3198348588466305      0.9973090376900052    0.997329904488004
0.3201686278228888      0.9972694433204093    0.9972536317415321
0.32033386246090234     0.9972496454247467    0.9972236252872402
0.3204938298376119      0.9972303536795665    0.9972082131790241
0.32066311571731154     0.9972098034361906    0.9972044515513396
0.3208271343357072      0.9971897596802525    0.9972043671442832
0.3209983877764796      0.9971686913265198    0.997196408636776
0.3211706249977882      0.9971473563047607    0.9971726678549039
0.3215034650177204      0.9971057101859042    0.9970940287527086
0.3216682351775709      0.99708488842816      0.9970590672842589
0.3218277380761174      0.9970646018675843    0.9970386888812851
0.32199655947765404     0.997042989543862     0.9970317545278315
0.32216011361788666     0.9970219128576872    0.9970318245881243
0.322330902580496       0.9969997573233287    0.9970268340531536
0.3225026753236416      0.9969773222925998    0.9970066988244538
0.3229988920689352      0.9969116481894346    0.996887678614269
0.32317043257299927     0.9968886434757668    0.9968601247972741
0.3233429568575997      0.9968653491578425    0.9968497228303005
0.3235060465196693      0.9968431824894122    0.9968493116585032
0.3236763710041155      0.9968198799238267    0.9968461365721154
0.3238414282272578      0.996797148447259     0.99682955933471
0.3240012181890961      0.9967750013837442    0.9967973971244275
0.32433416785744884     0.9967284053702976    0.9967074729009341
0.3245052438833499      0.9967042258247601    0.9966738519386377
0.32467730368978726     0.9966797433518656    0.9966584957930318
0.32483992887369384     0.9966564514333958    0.996656528300633
0.325009788879977       0.9966319648425606    0.9966553421235052
0.32517438162495627     0.9966080822870708    0.9966426248597114
0.3253337071086315      0.9965848176790729    0.9966134928250614
0.3256657278206583      0.9965358704477275    0.9965198712532491
0.3258363393683964      0.9965104719803385    0.9964797082229097
0.32600168365483045     0.9964857272937826    0.9964585479840601
0.3261617606799605      0.9964616245324163    0.996453244530636
0.32633115620808073     0.9964359551912175    0.9964533287126563
0.3264952844748969      0.9964109227087175    0.9964454363934396
0.3266666475640897      0.9963846163397199    0.9964187184703801
0.32700190668101725     0.9963326422901793    0.9963221594018182
0.32717205375059233     0.9963060059319067    0.9962760102750091
0.32733693355886334     0.9962800264412967    0.9962485818510269
0.32749654610583034     0.9962547187325127    0.9962393768183121
0.32766547715578753     0.9962277630548562    0.9962392675901476
0.32782914094444066     0.9962014797301756    0.9962345893351795
0.3280000395554704      0.9961738569275964    0.996212299949797
0.3285040523074839      0.9960913252023254    0.9960640758751796
0.3286684676375919      0.9960640543887748    0.9960294331981961
0.3288401177900765      0.9960353992314658    0.9960143976035499
0.3290127517230975      0.9960063888448635    0.996013284889666
0.3291759510335876      0.9959787867938971    0.9960105020877927
0.3293463851664543      0.9959497763397765    0.9959913088004495
0.32951155203801713     0.9959214813575545    0.9959514084736547
0.32984066976152493     0.9958645644008755    0.9958413780523885
0.33000462061346986     0.9958359431768287    0.995799780680498
0.3301758062877914      0.9958058674853778    0.9957784924975175
0.3303479757426494      0.9957754205985766    0.9957749849652499
0.3305107105749765      0.99574645830776      0.995774199082525
0.33068068022968033     0.9957160166495124    0.9957597103263819
0.3308453826230801      0.9956863304645115    0.9957237075675001
0.3311735713902618      0.9956266220679191    0.9956097469595216
0.3313370577640437      0.9955966007282722    0.9955607397360772
0.3315077789602022      0.9955650523489669    0.9955317358162249
0.33167948393689717     0.9955331165717877    0.9955241215694716
0.3318417542910612      0.9955027450840828    0.9955243598658697
0.332011259467602       0.9954708208235631    0.9955147914275015
0.33217549738283875     0.9954397053358053    0.9954839989533698
0.3325194266386019      0.9953739804269456    0.9953626541515839
0.33268244853422074     0.9953425290897455    0.9953077161068873
0.3328527052522162      0.9953094758696985    0.9952721131518835
0.3330176947089078      0.9952772431858871    0.9952604662110444
0.33317741690429536     0.9952458490187422    0.9952604445460265
0.3333464576026731      0.9952124179794712    0.9952547775349984
0.3335102310397468      0.9951798261979364    0.9952291079935748
0.33368123929919713     0.9951455808141654    0.9951768843754802
0.3340157887566397      0.9950779489423499    0.9950467874362294
0.3341855809964721      0.9950432994454833    0.9950022589360267
0.33435010597500064     0.9950095147994816    0.9949840026957227
0.33450936369222517     0.9949766138032681    0.9949822041106344
0.3346779399124399      0.9949415744016766    0.9949799070589297
0.3348412488713505      0.9949074195335378    0.9949601785444039
0.3350117926526378      0.9948715294570084    0.9949126782288491
0.33533707843130056     0.9948024411496525    0.9947792704780928
0.3355064061929701      0.994766145874719     0.9947238106296246
0.3356704666933356      0.9947307617731852    0.994695715782331
0.33584176201607774     0.9946935872726601    0.9946893943906016
0.3360140411193562      0.9946559610215133    0.9946890659439486
0.33617688560010384     0.994620174508014     0.9946745814317921
0.33634696490322813     0.9945825677836563    0.9946321995383857
0.3366713217255648      0.9945101916772887    0.9944953495810496
0.3368401850090712      0.9944721686766246    0.9944310122965551
0.3370037810312737      0.9944351060358034    0.9943938415563047
0.33717461187585285     0.9943961661084331    0.9943818628909611
0.33734642650096824     0.9943567551970994    0.9943821483362711
0.33750880650355286     0.9943192798291309    0.9943730157231342
0.3376784213285141      0.9942798963860294    0.9943372449101414
0.3381869174447756      0.9941603553109831    0.9941211905085681
0.338350048988815       0.9941215336970477    0.9940764143375792
0.33852041535523103     0.9940807764697789    0.9940590303982961
0.3386855144603432      0.9940410490698931    0.9940586129522793
0.3388453463041513      0.9940023614012763    0.9940535807097508
0.3390144966509495      0.9939611728463704    0.9940241432336058
0.3391783797364437      0.993921024980461     0.9939656249992905
0.33952159933272186     0.9938361664120462    0.9938021517456053
0.3396842663985982      0.9937955780940166    0.9937471896782428
0.3398541682868512      0.9937529286256077    0.9937212157119845
0.3400188029138003      0.9937113505824878    0.9937181451558526
0.3401781702794454      0.9936708664291481    0.9937165049789541
0.3403468561480807      0.9936277604904893    0.9936942695107125
0.3405102747554119      0.9935857495645715    0.9936415670498346
0.3408463143535236      0.9934985804422131    0.9934744038679222
0.3410064332606235      0.9934566729167148    0.993408146143452
0.34117587067071353     0.9934120628983821    0.9933696643999329
0.34134004081949954     0.9933685798306581    0.9933604867856602
0.3415114457906622      0.9933229062267929    0.9933607373071012
0.3416838345423613      0.9932766864215237    0.9933446340705195
0.34184678867152946     0.9932327329256526    0.9932980190554758
0.34218189931331516     0.9931415343595271    0.9931276571296525
0.34234155374225206     0.9930976996895585    0.9930516648653073
0.342510526674179       0.9930510337127176    0.9930018984240807
0.342674232344802       0.993005553552285     0.9929857124005688
0.34284517283780164     0.9929577796432972    0.9929858378814043
0.34301709711133765     0.9929094369161683    0.9929760689969449
0.3431795867623428      0.9928634748515717    0.9929371286781911
0.34367295839857626     0.9927222850629077    0.9926823032732202
0.3438414668523402      0.9926734948297777    0.9926196586858445
0.3440047080448001      0.9926259519309554    0.99259384516601
0.3441751840596367      0.9925760087758176    0.9925914122342727
0.3443403928131693      0.9925273216217244    0.9925877359415827
0.344500334305398       0.992479916259104     0.9925596074124368
0.3446695943006167      0.9924294600917314    0.9924924129750153
0.345004814590823       0.9923287295589328    0.9922979557691265
0.34517702592765076     0.9922765189454714    0.9922207672214693
0.34533980264194775     0.9922268779410218    0.9921843989861155
0.3455098141786213      0.9921747273178906    0.9921772792804576
0.34567455845399087     0.9921238953009459    0.992176749279571
0.3458340354680565      0.992074408532148     0.9921564070926165
0.3460028309851122      0.9920217286327353    0.992097016221212
0.3463371223189924      0.9919164773121268    0.9918977028561482
0.3465088691776571      0.9918619236139381    0.9918073118283237
0.34667118141379105     0.9918100656720503    0.9917580650223985
0.34684072847230163     0.9917555822727723    0.9917432686381865
0.3470050082695082      0.991702483974105     0.9917439726167357
0.34717652288909145     0.9916467226902721    0.9917292578775349
0.34734902128921097     0.9915903055007175    0.9916747653354466
0.34768238366676507     0.9914803149487175    0.9914721976993517
0.34784741500542643     0.9914253921125521    0.9913745192757668
0.3480071790827839      0.9913719226274698    0.9913133829452664
0.3481762616631314      0.9913150118971985    0.9912892428160629
0.34834007698217495     0.9912595562189029    0.9912887893014223
0.34851112712359517     0.9912013159481081    0.9912807185209935
0.34868316104555164     0.9911423934363726    0.9912359088219814
0.34918016132727797     0.9909701964308203    0.9909274518976263
0.3493394609264724      0.9909143787204174    0.9908519591913175
0.3495080790286569      0.990854962927552     0.9908153997045647
0.34967142986953736     0.9907970751721117    0.9908109680446601
0.34984201553279454     0.9907362774473928    0.990808360367236
0.3500073339347477      0.9906770180730052    0.9907766057227916
0.35016738507539696     0.9906193275145514    0.9907036902712578
0.3505008571013715      0.9904981123073935    0.9904710459373586
0.3506721943060835      0.9904352956210876    0.9903726924332172
0.35084451529133176     0.9903717486650616    0.9903210205505649
0.35100740165404926     0.9903113610169297    0.9903103853753813
0.3511775228391434      0.990247958899215     0.9903106462288138
0.3513423767629335      0.9901861702974456    0.9902880955819168
0.35150196342541973     0.9901260265938214    0.9902242436858929
0.3518345064950683      0.989999652918297     0.9899882486869963
0.3520053792216173      0.9899341633411916    0.9898750746448771
0.3521772357287025      0.9898679147728312    0.989807065328511
0.352339657613257       0.9898049493538656    0.9897868947162993
0.35250931432018806     0.9897388103651703    0.9897875712571687
0.35267370376581514     0.9896743635270833    0.9897741512120961
0.3528453280338189      0.9896066991923559    0.9897154430868439
0.3533515177567542      0.9894048437363985    0.9893504537131351
0.35351665874383614     0.9893382453306848    0.9892709045575354
0.3536765324696141      0.9892734195217805    0.9892402603023611
0.35384572469838216     0.9892044367939344    0.9892385663225376
0.3540096496658462      0.9891372287747932    0.9892319464660306
0.3541808094556869      0.9890666611273434    0.9891845783578596
0.35435295302606395     0.9889952806389355    0.9890818446860656
0.3546856057441331      0.988856179746471     0.9888126326770713
0.354850282253052       0.9887867479773976    0.9887157766699814
0.35500969150066697     0.9887191745762288    0.9886708517065201
0.35517841925127197     0.988647260948165     0.9886633524421905
0.35534187974057296     0.9885772080113935    0.9886620257611278
0.35551257505225065     0.9885036490609106    0.9886272397960251
0.3556842541444647      0.9884292465008022    0.9885355976542006
0.3560159779062077      0.9882842842497254    0.9882572832767752
0.35618018993696354     0.9882119358510117    0.9881420733137702
0.35635163679009607     0.9881359818670196    0.9880767122845125
0.35652406742376497     0.9880591594074134    0.9880618320083872
0.356687063434903       0.9879861396782812    0.9880626433040921
0.35685729426841767     0.98790946138755      0.9880363253404832
0.35702225784062835     0.9878347470088634    0.9879572698277993
0.3573509689654319      0.9876847067778375    0.9876766642785555
0.3575147165180247      0.9876093854332483    0.9875450782983836
0.3576856988929942      0.9875303057325147    0.9874605577293782
0.35785766504850003     0.9874503254014937    0.9874335511489366
0.358020196581475       0.9873743203419015    0.9874345302358285
0.35818996293682664     0.9872945018228172    0.9874189945274946
0.3583544620308743      0.9872167385597214    0.9873535372263229
0.3588455272737815      0.9869821138428023    0.9869290972226268
0.35901604517058794     0.9868997654140993    0.9868228629396556
0.3591812958060904      0.9868195261044965    0.9867797615385349
0.3593412791802889      0.9867414350302972    0.9867761669038875
0.3595105810574775      0.9866583549141769    0.9867712449693661
0.3596746156733621      0.986577425644636     0.9867242878532495
0.3598458851116234      0.986492469215935     0.9866109149388685
0.3601809569266877      0.9863249001082365    0.986286955183377
0.3603510103453311      0.9862391638509677    0.9861602111198496
0.3605157965026705      0.9861556356805097    0.9860992751144189
0.360675315398706       0.98607435606095      0.9860879878495334
0.3608441527977316      0.9859878749421002    0.9860880473930657
0.36100772293545313     0.985903645220642     0.9860547219211891
0.3611785278955514      0.9858152188771023    0.9859552000134016
0.3615126707542897      0.9856408324015944    0.9856244013837339
0.36168225969477014     0.9855516127056633    0.9854762847948044
0.36184658137394654     0.9854647038536638    0.9853936968999797
0.36201813787549963     0.9853734831974429    0.9853697429088141
0.36219067815758904     0.9852812372741581    0.9853712595573941
0.36235378381714756     0.9851935704448574    0.9853468436049716
0.36252412429908276     0.9851015305937892    0.9852575554214348
0.3630181279413586      0.9848317884442246    0.9847650789523833
0.363181985142372       0.9847413853335051    0.98466089474647
0.36335307716576204     0.9846464920087102    0.984621299321125
0.3635251529696884      0.9845505542305304    0.9846213632244656
0.3636877941510839      0.9844594261831584    0.9846080637759751
0.36385767015485604     0.9843637467569502    0.9845352394399485
0.3643502803626428      0.984083397499029     0.9840320147739952
0.3645136730854931      0.9839894512230408    0.9839037521019376
0.3646843006307201      0.9838908322067041    0.9838434205303193
0.3648496609146431      0.9837947554264944    0.9838369599677523
0.36500975393726215     0.9837012661441388    0.9838340425907731
0.36517916546287127     0.983601826433506     0.983783133448685
0.36534330972717644     0.983504977435765     0.9836611730868295
0.36568705168107635     0.9833005549805727    0.9832683021820751
0.3658499799257636      0.9832028980869808    0.9831176663814302
0.3660201429928276      0.9831003773324876    0.9830352791993715
0.36618503879858755     0.9830005138259351    0.9830185988208177
0.3663446673430436      0.9829033543042409    0.9830200390035995
0.36651361439048974     0.9828000001544662    0.9829848508992314
0.36667729417663186     0.9826993534622804    0.9828789409387857
0.36702010717420575     0.9824869070006482    0.9824810394391731
0.36718257094073        0.9823854410916365    0.9823074035546545
0.3673522695296309      0.9822789150402755    0.9821987107423465
0.3675167008572279      0.9821751653968103    0.9821662711743243
0.36767586492352083     0.9820742403598758    0.9821681025861615
0.3678443474928039      0.9819668698017232    0.9821481780636194
0.368007562800783       0.9818623275782756    0.9820618604364927
0.36850311140793834     0.9815417134367835    0.9814829532361813
0.3686723455186762      0.9814311087323327    0.9813402836489077
0.3688363123681101      0.9813234029386305    0.9812821500591256
0.36900751403992066     0.9812103716651775    0.9812777182869685
0.3691796994922676      0.9810960980622293    0.9812690238634739
0.3693424503220836      0.9809875372513255    0.9812009429516996
0.36951243597427635     0.9808735791009944    0.981043792648657
0.36983660549474984     0.9806546557504947    0.9806184679636676
0.3700053751273247      0.9805398572016888    0.9804477133445048
0.3701688774985955      0.9804280841535765    0.9803647466962617
0.370339614692243       0.9803107774515549    0.9803490738548702
0.3705113356664269      0.9801921866582467    0.9803485887700997
0.3706736220180799      0.9800795486034412    0.9803002539025808
0.3708431431921096      0.979961302700817     0.9801626771415514
0.37117888583993763     0.9797253347761151    0.9797123129894086
0.37135135835557637     0.9796031935255521    0.9795140378325962
0.3715143962486842      0.9794871543517126    0.9794100279488338
0.3716846689641687      0.9793653625112115    0.9793823557760584
0.37184967441834926     0.9792467477418763    0.9793849750546834
0.3720094126112258      0.9791313635899499    0.9793531590555508
0.3721784693070924      0.9790086505874113    0.9792353171270356
0.3725132829985944      0.9787637962094955    0.9787841334409084
0.37268529103607007     0.9786370569442226    0.9785592911683865
0.37284786445101487     0.9785166753458202    0.9784258326473853
0.37301767268833635     0.9783903178767718    0.9783774023205354
0.3731822136643539      0.9782672746047613    0.9783786699673555
0.37334148737906736     0.9781476006477016    0.9783622488936038
0.37351007959677096     0.9780203123900941    0.9782678582106762
0.37367340455317055     0.9778963979372967    0.9780835655515501
0.3740092568494191      0.9776397131217995    0.9775864807783834
0.3741692821055875      0.9775165178505655    0.9774183755000893
0.3743386258647459      0.977385519166179     0.9773374198923825
0.3745027023626004      0.977257975345017     0.9773286124959709
0.3746740136828315      0.9771241543371493    0.9773242615998887
0.3748463087835989      0.9769888892760659    0.9772504681299461
0.3750091692618355      0.9768604059863031    0.9770832254603405
0.37534409260175794     0.9765942608844521    0.9765690893168638
0.37550365337976327     0.9764665544844194    0.9763710930033274
0.3756725326607587      0.9763307455703518    0.9762597264690175
0.3758361446804501      0.9761985383270493    0.9762362966308137
0.37600699152251815     0.9760598189477825    0.9762385505085325
0.3761788221451226      0.9759196400728731    0.976187794536642
0.37634121814519617     0.9757865183125873    0.9760431072423367
0.37667521252879266     0.975510767869928     0.975523559863467
0.37684681091231553     0.9753680613396956    0.9752805799822086
0.3770193930763747      0.9752238262558108    0.9751392915064455
0.37718254061790313     0.9750868186255665    0.9751013235997337
0.3773529229818081      0.9749430506829644    0.9751047549305092
0.3775180380844092      0.9748030574072677    0.9750718285001083
0.3776778859257063      0.9746668996532734    0.974950196500708
0.3781820852583366      0.9742333375873448    0.974163507096566
0.3783542029442328      0.9740839037643994    0.973986443622965
0.37851688600759814     0.9739419883212325    0.9739240774015885
0.3786868038933401      0.9737930610127182    0.97392350873937
0.37885145451777813     0.9736480649784303    0.973907643662742
0.37901083788091217     0.9735070627294473    0.9738114942734002
0.37917953974703633     0.9733571233296008    0.9735977262784747
0.37951364377905333     0.973058063253991     0.9730170426535468
0.37967904594494617     0.9729089655136961    0.9728064224598653
0.37983918084953505     0.9727639537785642    0.9727087847221684
0.380008634257114       0.9726097918901431    0.9726934573954588
0.38017282040338896     0.9724597215414865    0.9726925190663581
0.3803442413720406      0.9723023002694637    0.9726196630955313
0.3805166461212286      0.9721432124112664    0.9724237716770869
0.3808498211969195      0.9718335938891496    0.9718272276145024
0.3810147588846493      0.9716792504633832    0.9715820456279592
0.3811744293110752      0.971529159638595     0.9714509741244137
0.3813434182404911      0.9713695822564837    0.9714165197966412
0.3815071399086031      0.9712142632996478    0.9714207274607053
0.38167809639909167     0.971051326802724     0.9713727436291205
0.38185003667011663     0.9708866730801098    0.9712047914544705
0.38218228278948146     0.9705662821216627    0.9706056466612709
0.3823467559990482      0.9704065987669918    0.970325699149353
0.38250596194731107     0.9702513485737598    0.970155575265727
0.382674486398564       0.9700862689721201    0.9700931290885791
0.38283774358851297     0.9699256184322669    0.9700955241562728
0.3830082356008385      0.9697570785731406    0.9700708934129484
0.3831734603518602      0.9695929925404957    0.9699435963766256
0.38366670256568924     0.9690987156449351    0.9690533940235314
0.38383794611946964     0.9689255508448857    0.9688216930406067
0.38401017345378635     0.9687505747068819    0.9687236504120147
0.3841729661655723      0.9685844284732731    0.9687171967593174
0.3843429936997348      0.9684101121118173    0.9687088484726046
0.3845077539725934      0.968240426978473     0.968611059670451
0.384667246984148       0.9680754433452508    0.9683941745607566
0.38499960275193335     0.9677293504021024    0.9677239790185755
0.3851703818275507      0.9675503007262064    0.9674492499959401
0.38534214468370437     0.9673693862499331    0.9673103017725008
0.38550447291732726     0.9671976384329505    0.9672861504686301
0.3856740359733267      0.9670174337172622    0.9672887903829475
0.3858383317680223      0.9668420424308001    0.9672221307534958
0.3860098623850945      0.9666581013267952    0.9670163747584285
0.38634545655778063     0.966295774563352     0.9663268890126082
0.38651577115523494     0.9661106465109893    0.966019222735599
0.3866808184913853      0.9659304396775105    0.9658497027467139
0.38684059856623165     0.9657552275227589    0.9658035069462743
0.38700969714406813     0.9655689838968926    0.965809008255254
0.3871735284606006      0.9653877418186388    0.9657677079479234
0.38734459459950976     0.9651976534264038    0.9655945225943431
0.38784910993516125     0.9646319978374627    0.9645640194126531
0.38801369279314857     0.9644458334003837    0.9643486583734835
0.38817300838983193     0.9642648562401818    0.9642703399111774
0.38834164248950537     0.9640724651559479    0.9642703350042311
0.3885050093278748      0.9638852685134216    0.9642520403454178
0.3886756109886209      0.9636889295060482    0.9641165521077099
0.3888409453880631      0.9634978138103252    0.9638389465407694
0.3891703981673295      0.9631145142596325    0.963092413655619
0.3893345165471538      0.9629223376807088    0.9628199178771484
0.38950586974935475     0.9627208100587845    0.9626880482856954
0.389678206732092       0.9625172166186126    0.9626734513898421
0.3898411090922984      0.9623239284867815    0.962670314656999
0.3900112462748815      0.962121181397519     0.9625685050216797
0.3901761161961606      0.9619238560494358    0.9623198497176013
0.39050464001910096     0.9615281414793293    0.9615549773858115
0.3906682939207622      0.9613297598331086    0.9612372301702293
0.3908391826448001      0.9611217126676651    0.9610577530879466
0.39101105514937434     0.9609115423462241    0.9610188129064939
0.3911734930314178      0.9607120533311765    0.9610246904581173
0.39134316573583783     0.9605027888437443    0.9609576949017288
0.39150757117895385     0.9602991502976459    0.960746058911193
0.3918518354904756      0.9598699460587146    0.9599402522793947
0.3920150249139738      0.9596651705504493    0.9595884660132591
0.39218544915984865     0.9594504049926675    0.9593701289831174
0.3923506061444196      0.9592413860223462    0.9593075582574762
0.3925104958676865      0.9590381952052279    0.959314042223638
0.39267970409394354     0.9588222618788741    0.9592746987558959
0.3928436450588966      0.9586121647554946    0.9590988015634676
0.3933497053594275      0.9579581039837937    0.9579138010628941
0.3935196651271393      0.9577365578467107    0.957642295145489
0.3936843576335472      0.9575209713572473    0.9575409800276591
0.39384378287865107     0.9573114277714196    0.957539144671513
0.3940125266267451      0.9570887205917478    0.9575242005958442
0.39417600311353507     0.9568720646623469    0.9573888858441636
0.39434671442270175     0.9566448718985393    0.9570735508921673
0.39468066997957685     0.9561976116132841    0.9562026327538441
0.39485016526912564     0.9559691770978659    0.9558735444734229
0.3950143932973705      0.9557469133215793    0.9557224440783618
0.39518585614799195     0.9555138878986136    0.9557014359049993
0.39535830277914974     0.9552785231663117    0.9556996492796057
0.39552131478777675     0.9550551086479728    0.9555923866489923
0.3956915616187804      0.9548208144707528    0.9553030656185777
0.39601625349687575     0.9543712324659303    0.9544339463538504
0.3961852843082615      0.9541357573618653    0.954053193873595
0.39634904785834324     0.9539066842884578    0.9538506407843375
0.3965200462308016      0.9536665056717124    0.9538005825412312
0.3966920283837964      0.9534239271339058    0.9538083314787521
0.3968545759142603      0.9531937149616292    0.953738207254603
0.3970243582671009      0.9529522772601902    0.9534921223080688
0.3973481211888702      0.952489090650335     0.9526256108133919
0.39751668752209296     0.9522464839126422    0.9521938360579703
0.3976799865940117      0.9520105080674024    0.9519317101418525
0.39785052048830716     0.9517630770400853    0.9518392266124851
0.39801578712129865     0.9515223106052226    0.9518451708939925
0.39817578649298613     0.9512882984661093    0.9518149609715879
0.3983451043676637      0.9510396698810426    0.9516292696006129
0.39850915498103723     0.9507978049580328    0.9512656761193212
0.398852709633074       0.9502881852308339    0.9502777009995482
0.3990155442268297      0.950045166984881     0.9499589810006867
0.3991856136429621      0.9497903359447949    0.9498187492145738
0.39935041579779057     0.9495424051294984    0.9498116302522217
0.399509950691315       0.9493014658789781    0.9498042069253243
0.3996788040878295      0.9490454514195005    0.9496645876109383
0.39984239022304        0.9487964381791344    0.9493394809235304
0.4001850159187507      0.9482717363926064    0.9483218882249176
0.4003473860343434      0.9480215855134692    0.9479428965130267
0.40051699097231275     0.9477592585130006    0.9477430053277136
0.40068132864897815     0.947504071980795     0.9477094310215238
0.4008529011480202      0.9472365915951682    0.94771513878905
0.40102545742759854     0.9469664831521396    0.9476058317792101
0.401188579084646       0.9467101075166545    0.9473099509508481
0.4015240247821903      0.9461797801256672    0.9462953993371287
0.4016838467390065      0.9459256360881659    0.945869706638417
0.4018529871988128      0.9456556366235088    0.9456096413627385
0.40201686039731516     0.9453930255179916    0.9455411617868796
0.40218796841819415     0.9451177468583819    0.9455518496535786
0.40236006021960946     0.9448397767179306    0.9454837889018239
0.402522717398494       0.9445760212613379    0.9452353232180415
0.40301659161836545     0.9437690595383297    0.9437548424635741
0.4031852676000087      0.943491334242129     0.9434262326407096
0.40334867632034804     0.9432212489738221    0.9433090460597043
0.403519319863064       0.9429381187734795    0.9433120386952772
0.403684696144476       0.9426626658812602    0.9432851866666114
0.403844805164584       0.9423949879875891    0.943102717957613
0.4040142326876821      0.9421106591578159    0.9426939434042098
0.40434978803364696     0.9415442764445553    0.941589052356048
0.4045221668983541      0.9412516279089005    0.941184204053572
0.40468511114053035     0.9409739383116912    0.9410139725727759
0.40485529020508326     0.9406828186806865    0.9409976651904316
0.4050202020083322      0.9403996345471873    0.9409944287791244
0.40517984655027717     0.9401244851183849    0.940859738432093
0.40534880959521224     0.939832190328509     0.9404972693652959
0.405683435984851       0.939250007056534     0.9393697333334132
0.4058553503713951      0.9389491989336896    0.9388961012636781
0.4060178301354083      0.9386638282587215    0.9386608965426906
0.4061875447217982      0.9383646371558156    0.9386101515926566
0.4063519920468841      0.9380736440162812    0.9386211858436415
0.406511172110666       0.9377909495490167    0.9385352126475994
0.40667967067743804     0.9374906084298253    0.9382305724568433
0.40701336811075073     0.9368924671212799    0.9371084614848396
0.4071785669772915      0.9365947061482022    0.936583301175993
0.40733849858252824     0.9363053929256929    0.9362664371110732
0.4075077486907551      0.9359980519968522    0.9361541724628749
0.40767173153767805     0.9356991760514242    0.9361636264204076
0.40784294920697756     0.9353859565298106    0.9361202886967406
0.4080151506568134      0.9350697424677942    0.9358676765765099
0.4085126535221778      0.9341494231008862    0.9341946353325865
0.4086721206492515      0.9338522982681934    0.9338064842742156
0.40884090627931535     0.9335366808237452    0.9336327526663722
0.4090044246480752      0.9332298024495935    0.9336249280494521
0.40917517783921165     0.9329081777906109    0.9336145489709625
0.4093469148108845      0.9325834936622882    0.9334247659238867
0.4095092171600265      0.9322755323878396    0.9330076892613824
0.4098430242417598      0.9316387370051437    0.9317648963303717
0.41001452897435103     0.9313097718842392    0.9312740119114606
0.41018701748747866     0.9309776912356348    0.9310395421608934
0.4103500713780755      0.9306626395067701    0.9310100166886086
0.410520360091049       0.9303324289754054    0.9310159016500701
0.4106853815427185      0.9300112802061132    0.9308790756440474
0.41084513573308395     0.9296992998134681    0.9305167630889178
0.4111780138584912      0.9290458006643576    0.929263555964286
0.4113490541129194      0.9287082115335761    0.9287010578523285
0.41152107814788397     0.928367440618611     0.9283890233245246
0.41168366756031777     0.9280442139267677    0.9283178299497138
0.4118534917951282      0.927705413883057     0.928333573968827
0.4120180487686347      0.9273759594731801    0.9282498069808486
0.41217733848083715     0.9270559586203531    0.9279493592499107
0.41267986342618357     0.9260393639285056    0.9260855234507187
0.4128451719411449      0.9257025984410514    0.9256944566866908
0.41300521319480216     0.925375451356464     0.9255560531791471
0.41317457295144955     0.9250280623756507    0.925562030999009
0.413338665446793       0.9246903048200744    0.9255330782093933
0.413509992764513       0.9243364216815204    0.9252878272230977
0.4136823038627694      0.9239791903227168    0.9247586726204916
0.4140152916365972      0.9232851431292503    0.9234016590460338
0.4141801356733955      0.9229397856922976    0.9229307499565801
0.41433971244888973     0.9226043430106139    0.9227257746833505
0.4145086077273741      0.9222481096616276    0.9227064339925078
0.4146722357445545      0.9219018049495264    0.922707125435298
0.41484309858411145     0.921538945925592     0.9225289112962016
0.4150149452042048      0.9211727154495655    0.9220611089965554
0.41534700402170655     0.9204613988968385    0.9206746078453446
0.4155113835803418      0.9201074910866478    0.9201203070341983
0.4156829979613536      0.9197367438878046    0.9198221938192137
0.4158555961229018      0.9193625682426633    0.9197712230082737
0.41601875966191915     0.9190076424669362    0.9197850963479046
0.4161891580233131      0.9186357285138039    0.9196516975420314
0.4163542891234031      0.9182740902426099    0.9192513971039663
0.4166833353039653      0.9175498872311679    0.9178693060249521
0.4168472503844375      0.9171873367219399    0.917243344158045
0.41701840028728626     0.9168075132736786    0.9168594021582626
0.4171905339706714      0.9164241951646098    0.9167553344794034
0.41735323303152577     0.9160606758670804    0.916774363576754
0.4175231669147567      0.9156797335962569    0.9166974749376513
0.4176878335366837      0.9153093708334197    0.9163678391478424
0.41817940136322895     0.9141965448056312    0.9143225907085216
0.41835008678791474     0.9138076102761353    0.9138424670339552
0.4185217559931369      0.9134151145436642    0.9136653277241085
0.41868399057582817     0.9130429716858752    0.9136719675734206
0.41885345998089607     0.9126529674567939    0.9136448118566592
0.4190176621246601      0.9122738494931772    0.9133945163759846
0.4191890990908007      0.9118767280844594    0.9128235961511352
0.4195245059616237      0.9110959360643691    0.9112932981902854
0.41969472690814646     0.9106977285464711    0.9107393925324085
0.4198596805933652      0.9103105873579932    0.9105022100082799
0.42001936701728004     0.9099345281028132    0.9104823250451103
0.4201883719441849      0.9095352587945928    0.9104868120754946
0.42035210960978586     0.9091471915201155    0.9103062285060868
0.42052308209776346     0.9087406716702187    0.9098041184270342
0.4208575600122604      0.9079415236206908    0.9082496746154666
0.4210273164806201      0.9075339761438235    0.9076034794815793
0.4211918056876758      0.9071378144370043    0.9072752869145941
0.4213510276334276      0.9067531563812609    0.9072083668519881
0.4215195680821694      0.9063447177431841    0.9072299780548997
0.4216828412696073      0.9059477985029943    0.9071192739173193
0.42185334927942186     0.9055319818326597    0.9067015648069787
0.42217856351513905     0.9047351696700776    0.9052038753996275
0.4223478555053357      0.904318453321158     0.9044569837593943
0.42251188023422837     0.9039134392794236    0.9040075642479589
0.42268313978549776     0.9034892327410357    0.9038558373522505
0.4228553831173034      0.9030612194849738    0.9038752599455359
0.4230181918265783      0.9026553862843258    0.9038187071361552
0.42318823535822975     0.9022302055908308    0.9034809743994854
0.4236813481496545      0.9009896146283709    0.9012153625107417
0.42384490840038425     0.9005756231621748    0.9006668864333048
0.42401570347349055     0.9001419875859193    0.9004297194356754
0.4241874823271332      0.8997044803578491    0.900426277379264
0.42434982655824505     0.8992897346275239    0.9004154280701571
0.42451940561173346     0.898855188983206     0.9001654651673404
0.42468371740391797     0.8984328558034864    0.8995937872958658
0.4250277944135765      0.8975443667865654    0.8978547796831067
0.4251908901861432      0.8971212718994833    0.8972321711458084
0.4253612207810865      0.8966780731436517    0.8969233957900985
0.4255262841147258      0.8962472764319508    0.8968888013210143
0.4256860801870612      0.8958290033303246    0.8969044038051563
0.4258551947623866      0.8953850264820499    0.8967297308955509
0.42601904207640806     0.8949535895312755    0.8962331473461033
0.4263621901297405      0.8940457249867586    0.8944836336259862
0.42652482142414416     0.8936134653220217    0.8937691466695523
0.42669468754092443     0.893160638258875     0.893356728520944
0.4268592863964007      0.8927205477595015    0.8932648359593375
0.42701861799057306     0.8922933163962893    0.8932918956361976
0.42718726808773544     0.8918397858373847    0.8931911481040282
0.4273506509235938      0.8913991318443896    0.8927842323088501
0.42784670211438713     0.8900534754601281    0.8903012063037619
0.4280161037530043      0.889591251509902     0.8897530224328299
0.4281802381303176      0.8891420959543124    0.8895644538596638
0.42835160733000743     0.8886717722222579    0.8895796211549395
0.42852396031023365     0.8881973360252455    0.8895354119633372
0.4286868786679291      0.8877475674098517    0.8892115007890691
0.42885703184800117     0.8872764727630601    0.8885076619230668
0.4291815364242334      0.8863741983064913    0.8867284830925426
0.42935047358468753     0.8859024809632292    0.8860719833836211
0.42951414348383776     0.8854441686613211    0.8857912387427469
0.4296850482053646      0.8849642282521177    0.8857725380812072
0.42985693670742775     0.8844801136225257    0.8857739165776686
0.43001939058696015     0.8840212691478202    0.8855395770641409
0.4301890792888692      0.8835406384216222    0.8849213405804339
0.43052515699245597     0.8825846431048808    0.88304828979799
0.43069779703597394     0.8820914472642571    0.8822903857469226
0.43086100245696113     0.8816238863655617    0.8819347696236551
0.4310314427003249      0.8811342307213244    0.8818783328923057
0.43119661568238477     0.8806583730185458    0.8819022228465191
0.4313565214031406      0.8801964381750983    0.8817436661380879
0.4315257456268866      0.8797062418992521    0.881211052430993
0.4318608943741473      0.8787313309494843    0.879342242546305
0.43203306993950225     0.8782283828520713    0.878484497305308
0.4321958108823264      0.8777516779990009    0.8780211953269006
0.4323657866475271      0.8772524138535666    0.8778956616274373
0.43253049515142394     0.8767671608635811    0.8779280783514452
0.43268993639401676     0.8762961510590849    0.8778418929362004
0.4328586961395997      0.8757962733817862    0.8774097096485608
0.43336462701772627     0.8742894136309114    0.8746393476035391
0.4335269034823874      0.8738034670421495    0.8740553969621562
0.43369641476942516     0.8732944941821649    0.8738347001310472
0.433860658795159       0.8728000099313442    0.8738516663503905
0.4340321376432694      0.8722823504671318    0.873817234170764
0.4342046002719162      0.8717602851595214    0.8734527012013231
0.4343676282780321      0.8712654553679818    0.8727187579467527
0.4347028866737133      0.8702438191369553    0.8707056165860815
0.434862614979598       0.8697551627521607    0.8700227464839024
0.4350316617884727      0.8692366521398739    0.8697000128517358
0.4351954413360435      0.8687329771818288    0.8696804193316773
0.43536645570599086     0.868205665787376     0.869690964506161
0.4355384538564746      0.8676738915837032    0.8694272695407552
0.43570101738442746     0.8671699689886564    0.8687826330330135
0.4360353468237826      0.866129571982227     0.8667426936338838
0.4361946106515042      0.8656320549268548    0.8659479049365015
0.4363631929822159      0.8651040885188261    0.8655022417432249
0.43652650805162363     0.8645913041199862    0.8654191275422771
0.43669705794340796     0.8640544229950362    0.8654548345946856
0.4368623405738884      0.8635327770343356    0.865304886419559
0.4370223559430648      0.863026492769477     0.8647940740696783
0.4373557564260938      0.8619676331356326    0.8627965142949514
0.437527057859333       0.861421493198553     0.8618210174381257
0.4376993430731085      0.8608707808158949    0.861236776273396
0.4378621936643532      0.8603489022678141    0.8610761778071742
0.4380322790779745      0.8598024647075022    0.8611115105218419
0.4381970972302919      0.8592716099926478    0.8610367914823467
0.43835664812130526     0.8587564643981517    0.8606283345674367
0.43852551751530877     0.8582098859132878    0.859777895830654
0.4388599566030845      0.8571231150598916    0.8576406502389202
0.43903177733869697     0.856562595280914     0.8569219243999286
0.43919416345177864     0.8560315373905283    0.856658482866774
0.4393637843872369      0.8554754541234457    0.8566668974579739
0.43952813806139124     0.8549353090848901    0.8566558239268947
0.4396997265579222      0.854369990964121     0.8563182880649238
0.4398722988349895      0.8537999939883382    0.8555153628090729
0.4402058089664391      0.8526943447774402    0.8533287290786434
0.4403709141820483      0.8521449994001234    0.8525322655487375
0.4405307521363535      0.8516119245291974    0.852169218591931
0.4406999085936487      0.8510464274788122    0.8521296556274809
0.44086379778964        0.8504972214249937    0.8521565668691077
0.44103492180800796     0.8499223875496883    0.8519220060235604
0.4412070296069122      0.8493428242204615    0.8512237003905375
0.44153961078203574     0.8482188299437617    0.8490244424613701
0.4417042515194819      0.8476604366121053    0.8481090138830703
0.44186362499562404     0.8471186644275946    0.8476209657394276
0.4420323169747562      0.8465438832663482    0.8475046372962987
0.44219574169258447     0.8459857436083804    0.8475485799967999
0.44236640123278936     0.8454015242836936    0.8474148921788628
0.44253179351169025     0.8448340006913655    0.8468709247498909
0.44286136204987425     0.843699208340652     0.8447436660142871
0.44302553830915736     0.8431319599988603    0.8437008807961652
0.4431969493908171      0.8425383345970979    0.8430166964487241
0.44336934425301316     0.84193988105226      0.8428004705717018
0.44353230449267844     0.8413728692206387    0.8428366657043056
0.4437024995547203      0.8407793258987288    0.8427792493184008
0.44386742735545826     0.8402028283495355    0.8423473117666139
0.4440270878948922      0.8396435019380967    0.841499873931333
0.44435977871843624     0.8384740939357888    0.8391834270972995
0.44453072532193294     0.8378711586672634    0.8383602710552955
0.44470265570596595     0.8372633452279135    0.8380246592717739
0.4448651514674682      0.8366875876975031    0.8380263554502589
0.445034882051347       0.8360847281706867    0.8380314483300644
0.4451993453739219      0.8354991997532948    0.8377199148477534
0.4453585414351928      0.8349311965413817    0.836980117056523
0.4456903033024108      0.8337436126125778    0.8346377471487867
0.44586078542774443     0.8331313159565088    0.8336719274631245
0.4460260002917741      0.8325366212474582    0.8331994164219139
0.44618594789449983     0.8319596536520343    0.8331228773959927
0.4463552140002156      0.8313477529016985    0.833169820062539
0.4465192128446275      0.8307536009166373    0.8329977115738941
0.446690446511416       0.8301318818398503    0.8323470752516502
0.44702544678353473     0.8289115548157789    0.8300051708766543
0.44719546443070535     0.8282901981213447    0.8289204559864233
0.44736021481657195     0.8276867942023798    0.8283098885766776
0.44751969794113466     0.82710146681537      0.8281448716471933
0.4476884995686874      0.8264806384264411    0.8281934021862123
0.4478520339349362      0.8258779079838444    0.8281127398406083
0.4480228031235617      0.8252471754203735    0.8275956834441611
0.44852642760836203     0.8233791079929516    0.8241444410803634
0.4486907135160657      0.8227671703266487    0.823381572759892
0.44886223424614596     0.8221269424349716    0.8230918004150342
0.4490347387567626      0.8214816615119721    0.8231224778596624
0.4491978086448483      0.8208704003883133    0.8230956496450165
0.4493681133553107      0.8202307016127031    0.8226736652529342
0.44953315080446915     0.8196095041497093    0.8217669401083697
0.44986200968316825     0.8183679247122427    0.8192947102439501
0.45002583111270883     0.817747563936159     0.8183956482468439
0.45019688736462604     0.8170984834389984    0.8179743783747673
0.4503689273970797      0.8164443083752788    0.8179555321028695
0.45053153280700237     0.8158247542559068    0.8179825209804714
0.4507013730393017      0.8151763346824297    0.8176877197288216
0.4508659460102971      0.8145467592408976    0.8169020169119975
0.45119387593267013     0.8132885515107937    0.8144204984169171
0.45135723288404767     0.8126597519107978    0.8133840649692454
0.45152782465780184     0.8120018005093442    0.812806597250607
0.4516931491702521      0.8113628960419663    0.8127012825310548
0.45185320642139837     0.8107431598912044    0.8127581573688505
0.4520225821755347      0.8100860729933032    0.8126078460191051
0.45218669066836703     0.8094481769306192    0.8119925014390114
0.45253036107932143     0.8081083616803713    0.8094576339227798
0.4526932535525359      0.8074714520133679    0.8083093828636618
0.45286338084812705     0.8068049762408218    0.8075829557850641
0.4530282408824143      0.8061578915302124    0.8073772230604287
0.4531878336553975      0.8055303172429918    0.8074276238357729
0.4533567449313709      0.8048648543433107    0.8073690381232642
0.45352038894604024     0.8042189240424211    0.8068883479089947
0.4536912677830862      0.8035431586380621    0.8058485683627512
0.45402555839572        0.8022173907674501    0.8032154218620929
0.45419522121314815     0.8015426201202548    0.8023270849726658
0.4543596167692723      0.8008875790206986    0.8019900445441864
0.4545312471477731      0.8002024326284964    0.8020080772064939
0.4547038613068103      0.799512044819839     0.8020036159600984
0.4548670408433166      0.7988581821124889    0.8016164446059844
0.4550374552021995      0.7981740766874679    0.800661574381832
0.45536248213605357     0.7968657600758791    0.7980468183033418
0.45553168047531867     0.7961828596623681    0.797016345439655
0.45569561155327976     0.7955200237109822    0.7965445975571964
0.45586677745361753     0.7948266821276189    0.7964998051888285
0.4560389271344917      0.7941280673650387    0.7965457129443495
0.45620164219283493     0.7934665546121444    0.7962829829980475
0.4563715920735548      0.7927744012254809    0.7954626162028626
0.4566956900510828      0.7914509816029575    0.7928583422871877
0.45686442391218485     0.7907601778135075    0.7916862043455553
0.4570278905119829      0.790089767779539     0.7910570767119391
0.4571985919341576      0.7893884588548628    0.7909138599566734
0.45737027713686873     0.7886818457113243    0.7909823207394063
0.45753252771704894     0.7880128182621658    0.7908402499717389
0.4577020131196058      0.7873126491321741    0.7901770715594212
0.4580376842244883      0.7859223278078488    0.7875448174790778
0.4582101209686542      0.7852062481721555    0.7862423057578926
0.4583731230902893      0.7845281902785921    0.7854936511574855
0.458543360034301       0.7838188390771053    0.7852611017243122
0.4587083297170087      0.7831302711373643    0.7853218579557634
0.4588680321384125      0.7824625990581748    0.7852694994304631
0.4590370530628063      0.781754804731859     0.7847508268833655
0.4592008067258962      0.7810679288275072    0.7836990175447188
0.4595437674773656      0.7796257375404495    0.780830515772222
0.45970630512083765     0.7789405421270648    0.7799223341318148
0.45987607758668636     0.7782236809067546    0.7795488188037945
0.46004058279123106     0.7775279264718445    0.7795704455402673
0.4601998207344718      0.776853388821092     0.7795912294561718
0.46036837718070256     0.7761382440731902    0.7792257156207539
0.4605316663656294      0.7754443382270254    0.7783134590091538
0.4608674471189324      0.7740140035639432    0.7754635811185563
0.461027436603628       0.7733308802663119    0.7743840355188181
0.46119674459131366     0.7726068400508604    0.7738090853169307
0.4613607853176953      0.7719042209519662    0.7737346302947699
0.46153206086645365     0.7711694571679173    0.7738039081393717
0.46170432019574825     0.7704292846374406    0.7735659096209072
0.4618671449025121      0.7697285586234974    0.7727917880860801
0.46220199669948897     0.7682841783880042    0.7699769137847952
0.4623615217060215      0.7675945007156861    0.7687662119339331
0.46253036521554414     0.7668634378816669    0.7680226000411035
0.46269394146376275     0.7661541071290555    0.7678407211012956
0.46286475253435805     0.7654122780946057    0.7679169007781691
0.4630365473854897      0.7646650202184069    0.7678024859488333
0.4631989076140905      0.7639577373616709    0.767186407229654
0.4637043930667914      0.7617491303004277    0.7630595639964584
0.4638769394593778      0.7609927945867717    0.7621624249847502
0.4640400512294334      0.7602767353244382    0.7618878389375491
0.4642103978218657      0.7595278226685807    0.7619477177410174
0.46437547715299393     0.7588010057445082    0.7619159952101845
0.4645352892228183      0.7580963880534464    0.7614449659766342
0.4647044197956327      0.7573496259128267    0.7603471021697077
0.4650393812410302      0.7558674725926285    0.7573726336640627
0.4652114631554536      0.75510439175588      0.7563059661368278
0.46537411044734617     0.7543821266512504    0.7558872199315283
0.4655439925616154      0.7536266782642962    0.7558942391451308
0.4657086074145806      0.7528936275365851    0.7559336738334116
0.4658679550062419      0.7521830748185078    0.75560991119753
0.4660366211008933      0.7514299468812795    0.7546660391005807
0.4663706535899647      0.7499353358549639    0.751686745553287
0.46653601998438476     0.7491939008444278    0.7504831047412918
0.46669611911750086     0.7484751329639134    0.7498734473469667
0.46686553675360704     0.7477135164180785    0.7497662301610217
0.4670296871284092      0.7469745888666399    0.7498495521552595
0.46720107232558805     0.746202057825236     0.7496594463383823
0.46737344130330327     0.7454240292355793    0.7488530850371508
0.4677065448360486      0.743917481436895     0.7459276562104219
0.46787144675230563     0.7431702111064769    0.7445910086720531
0.46803108140725874     0.7424458945862074    0.7438181071349993
0.4682000345652019      0.7416783197358487    0.7435906125821207
0.468363720461841       0.7409337200206505    0.7436707230463608
0.4685346411808568      0.7401552105861593    0.7435974935028428
0.4687065456804089      0.7393711950373741    0.742963213667729
0.4692031576949222      0.7371005284366364    0.7387070876303621
0.4693623278717123      0.7363709601834237    0.737757377269314
0.4695308165514924      0.7355977382793663    0.7373730395768378
0.46969403796996856     0.7348477688716977    0.7374113874536099
0.4698644942108214      0.7340635945250163    0.7374338910954276
0.4700296831903702      0.7333026825936361    0.737011160349038
0.47018960490861506     0.7325650431153575    0.7360094738692563
0.4705228180897809      0.7310253558862218    0.732912652821245
0.47069402587208853     0.7302328204286321    0.7316873640698474
0.4708662174349325      0.729434757603407     0.731121193476982
0.4710289743752456      0.7286795288242872    0.7310887763181575
0.4711989661379354      0.7278898069008181    0.7311659202107792
0.47136369063932115     0.7271236605654781    0.730891903735055
0.47152314787940297     0.7263811783542267    0.7300469470672794
0.47185543210424274     0.724831340288021     0.7269946138353219
0.4720261754083873      0.7240335923725617    0.7256100148410286
0.47219790249306826     0.7232303176552275    0.724853206593402
0.4723601949552183      0.7224703213818318    0.7247107963743634
0.47252972223974504     0.7216755645084073    0.7248088843354209
0.47269398226296777     0.7209046468430699    0.72467638668996
0.4728654771085672      0.7200988818372044    0.7239318281602003
0.47320099973830776     0.7185198177100828    0.7209200441549984
0.4733712785642893      0.7177171214630517    0.7194325413635392
0.4735362901289668      0.7169384156821372    0.7185534621946823
0.47369603443234043     0.7161837836158615    0.7182929604104628
0.4738650972387041      0.7153842963277212    0.7183746081926926
0.4740288927837638      0.7146089033210963    0.7183463250980443
0.4741999231512002      0.7137984107104652    0.7177756847767721
0.47437193729917293     0.712982384368155     0.7165145647401673
0.4747043311724333      0.7114030661576218    0.7133248501382944
0.47486887825894786     0.7106200557473071    0.7122633209260429
0.47502815808415844     0.7098613646261926    0.7118465766424379
0.4751967564123591      0.7090574925564075    0.7118718326929057
0.47536008747925573     0.7082779597247865    0.7119253482696221
0.47553065336852907     0.7074630876965552    0.7115361552913686
0.4756959519964984      0.7066725951892195    0.7105015564201336
0.4760253332328193      0.705095140077398     0.707319104394831
0.4761894158411708      0.7043081972722876    0.7060448532711546
0.476360733271899       0.7034856848172383    0.7053840429273472
0.4765330344831635      0.7026575825944036    0.7053185571161878
0.4766959010718971      0.7018740801161727    0.7054145525024029
0.47686600248300737     0.7010550059731502    0.7051750662382806
0.47703083663281365     0.7002605520159532    0.7043075104062353
0.4773592889128085      0.6986753444700822    0.7011881628205197
0.47752290704299694     0.6978846110560702    0.6997684229465836
0.4776937599955621      0.6970581647688361    0.6989145804891133
0.4778655967286635      0.6962261938852541    0.6987219667055129
0.4780279988392341      0.6954392009704462    0.6988274070373403
0.47819763577218133     0.6946164261115254    0.698728000188262
0.47836200544382457     0.6938184994293428    0.6980485328300705
0.4788693518644262      0.6913513160900866    0.69339089163633
0.4790397403388283      0.6905212955522875    0.6923971536667274
0.47920486155192643     0.689716253338253     0.6920965544094527
0.47936471550372056     0.688936258163444     0.6921779004964316
0.4795338879585048      0.6881101204502015    0.6921790727369151
0.4796977931519851      0.6873090490798284    0.691667383124499
0.479868933167842       0.686471934570288     0.6904317074887463
0.48020374613809763     0.6848322307498756    0.6870976055831505
0.4803736701343367      0.6839990478957353    0.685918335087158
0.4805383268692718      0.683191055013725     0.6854529046860066
0.48069771634290287     0.6824083168499215    0.685473369567648
0.4808664243195241      0.681579186883        0.6855510630665973
0.4810298650348413      0.6807753302547851    0.6852123110542
0.4812005405725352      0.679935251486043     0.6841602101630684
0.4815344245864647      0.678289980428764     0.6808376736847287
0.48170388410454074     0.6774540044885249    0.6794725806586571
0.4818680763613128      0.6766434190496331    0.6788115068208596
0.4820395034404614      0.675796498200037     0.678734928666758
0.4822119143001465      0.6749440852166353    0.6788478402876204
0.48237489053730065     0.6741377401499669    0.6786229157534172
0.4825451015968315      0.6732949955077405    0.677710684458312
0.48286972193198124     0.6716859259377059    0.6745279354364782
0.4830387169718942      0.6708474026913015    0.6730150442812053
0.4832024447505032      0.6700344645060547    0.6721691506401976
0.4833734073514888      0.6691850318380459    0.6719619882780115
0.48354535373301083     0.6683301273131388    0.6720834954293767
0.48370786549200195     0.6675215980711113    0.6719937828413786
0.48387761207336977     0.6666765265978805    0.6712775808611031
0.48436983401394346     0.6642229230559051    0.6665998954800412
0.4845330973143894      0.6634080915598662    0.66555429589903
0.4847035954372121      0.6625566266258016    0.665175408981316
0.48486882629873074     0.661730959763984     0.66525548731549
0.48502878989894543     0.66093114365265      0.6652928624407567
0.4851980720021502      0.6600842368809374    0.6648193342396445
0.485362086844051       0.6592631983878062    0.663667042190582
0.48570556995314224     0.6575422445940695    0.660158775125602
0.48586836877542516     0.6567258660349732    0.6589408737527481
0.4860384024200848      0.6558727290672678    0.6583865054280122
0.4862031688034404      0.655045559434531     0.6583949975095446
0.48636266792549204     0.6542444055991521    0.6584949075461625
0.48653148555053377     0.6533959929984752    0.6581949219147114
0.48669503591427155     0.6525736129965951    0.6572253311776214
0.4870375900670367      0.6508497722773073    0.6537597275068977
0.48719992441115656     0.6500322145380159    0.6523717938810604
0.48736949357765313     0.6491777882182957    0.6516131358231135
0.4875337954828457      0.6483494866822195    0.6515063040522128
0.48770533221041495     0.6474842807927366    0.6516391647146639
0.4878778527185205      0.6466136744558842    0.6514520573722381
0.48804093860409525     0.645790280062226     0.650617843221415
0.488376312758694       0.6440958333675416    0.6472942543870599
0.4885360989440374      0.6432879723152699    0.64579635847285
0.4887052036323709      0.6424326141149344    0.644848830724154
0.48886904105940043     0.6416035067652226    0.6446088893786357
0.48904011330880665     0.6407373785281675    0.6447340429256364
0.4892121693387492      0.6398658795210528    0.6446805232953284
0.4893747907461609      0.6390418148382142    0.6440348329222496
0.4895446469759493      0.6381807269780135    0.642643347375739
0.4898685576516141      0.6365376556120285    0.6392727424683177
0.49003719786178457     0.6356817003501771    0.6381234412253144
0.49020057081065105     0.6348521539496875    0.6377132504878472
0.4903711785818942      0.6339855353339828    0.6377868539769537
0.4905365190918334      0.6331453498018423    0.6378479756833035
0.49069659234046864     0.6323316328710773    0.6374348913220917
0.490865984092094       0.6314702333140765    0.6362705141233046
0.4912014678951133      0.6297632869037935    0.6327685346792201
0.49137381098834765     0.6288859356057224    0.6314076439817842
0.4915367194590511      0.628056329199329     0.6308278200339672
0.4917068627521312      0.6271895905740605    0.6308187099588971
0.49187173878390733     0.6263494079695495    0.6309384462042275
0.49203134755437955     0.6255358125870203    0.6306858614751328
0.49220027482784184     0.6246744501275114    0.6297125323515029
0.4925348296745351      0.622967763641127     0.6262774654202852
0.4927067082896064      0.6220905532450702    0.624746154789308
0.4928691522821468      0.6212612549432291    0.6239713040735942
0.49303883109706387     0.6203947795661189    0.6238362575279642
0.49320324265067694     0.6195549710465447    0.6239791722616752
0.4933748890266667      0.6186779708443924    0.6238471521627788
0.49354751918319284     0.617795706601059     0.6230011178118241
0.49388114507356        0.6160899861991816    0.6196562283232059
0.49404630816862793     0.6152452541326188    0.6180691018337813
0.49420620400239196     0.6144272748148433    0.6171335956583953
0.49437541833914606     0.6135614301552291    0.6168582447465947
0.4945393654145962      0.6127223517644869    0.616985484274884
0.4947105473124229      0.61184605661903      0.6169757498853761
0.49488271299078596     0.6109645366241396    0.61632676298102
0.4950454440466182      0.6101311551129738    0.6150008915870562
0.495380108541732       0.6084167799190358    0.6114677041778319
0.4955395398973329      0.6075998434509762    0.6103460689071234
0.49570828975592396     0.6067350091699635    0.6098942170615305
0.495871772353211       0.605897028334708     0.6099651036740681
0.49604248977287474     0.6050218207998928    0.6100529076748734
0.4962141909730748      0.6041414290067317    0.6096093485437839
0.496376457550744       0.6033092886255891    0.6084703996202634
0.4967101930895317      0.6015974583977072    0.6049358021789855
0.4968816620506502      0.6007177668447192    0.6035511771270073
0.49705411479230505     0.5998329160279908    0.6029393681598737
0.4972171329114291      0.598996376163347     0.6029472470766901
0.49738738585292974     0.5981226144554995    0.6030807149621346
0.4975523715331265      0.5972757971862604    0.6027986785377922
0.4977120899520192      0.596455937771957     0.6018491085013211
0.4980448965344808      0.5947473673064403    0.5983878753539988
0.4982159010174363      0.5938693547615452    0.5968401102859239
0.4983878892809281      0.5929862267154811    0.5960270676915552
0.4985504429218891      0.5921514899045207    0.5959183411624022
0.4987202313852267      0.5912795510693075    0.5960773573823046
0.4988847525872604      0.5904346185324731    0.5959537206947056
0.4990440065279901      0.5896167014865149    0.5952052623099328
0.4993758841541257      0.5879121101082427    0.5919009012900953
0.49954642415891815     0.587036145141381     0.5902064877227481
0.49971169690240663     0.5861872187279563    0.5891960522360818
0.4998717023845911      0.5853653375621048    0.5889050652813478
0.5000410263697657      0.5844955858038906    0.5890349521150751
0.5002050830936363      0.5836528900811493    0.5890676780696757
0.5003763746398835      0.582773037840524     0.5885000401118896
0.5005486499666671      0.5818881444802103    0.5871389528342255
0.5008815661975492      0.5801781753065511    0.5835827891978631
0.5010463744628746      0.5793317038161417    0.582387423086853
0.5012059154668961      0.5785123168471737    0.5819292092586188
0.5013747749739075      0.5776451259405975    0.5819921171846887
0.5015383672196151      0.5768050666380695    0.5821067041332879
0.5017091942876993      0.5759279072469273    0.5817351188627373
0.5018810051363198      0.5750457549395918    0.5805742091059618
0.5022129924108758      0.573341374619726     0.5770365092701139
0.5023773361980381      0.572497756742784     0.5756562132667012
0.5025489148075771      0.5716170802758839    0.5749785657403939
0.5027214771976525      0.5707314420428399    0.5749665201696801
0.5028846049651969      0.5698943111776589    0.5751168053434249
0.5030549675551181      0.5690201480195076    0.5748731250630263
0.5032200628837353      0.5681731100932825    0.5739144428756192
0.5035490375213518      0.5664855811748669    0.5704803738786424
0.5037129168303511      0.5656450988524157    0.5689552661878392
0.5038840309617271      0.5647676346949485    0.5680817143110732
0.5040561288736394      0.5638852584073712    0.5679422116059752
0.5042187921630208      0.5630513826556278    0.5681082791305647
0.504388690274779       0.5621805561989092    0.5680184173355387
0.5045533211252331      0.5613368671090048    0.5672640730909053
0.5050447816373599      0.5588191317316844    0.5623398617051599
0.5052154312905729      0.5579452215857549    0.5612542105738086
0.5053808136824819      0.5570984527764297    0.5609452359949911
0.5055409288130869      0.556278814527288     0.5610753820222418
0.5057103624466819      0.5554116530237502    0.5611359870010244
0.5058745288189731      0.5545716298051271    0.5606349371240728
0.5060459300136408      0.5536947817770511    0.5593325271917797
0.5063812653415182      0.5519798835986279    0.5557392569323097
0.506551450516568       0.5511098740100776    0.5544722329898825
0.5067163684303141      0.5502670001801072    0.5539949452975944
0.506876019082756       0.5494512473300521    0.5540593373132996
0.5070449882381881      0.5485881008642018    0.5541963264829947
0.5072086901323162      0.5477520824098856    0.5538765254437611
0.5073796268488209      0.5468793539162011    0.5527738448989602
0.5077140332203721      0.5451728197430358    0.5492231448917076
0.507883753917259       0.5443071491630088    0.5477757083694498
0.508048207352842       0.543468593562331     0.5471014306909301
0.5082073935271209      0.5426571338670341    0.5470582030840009
0.5083758982043898      0.5417984323852499    0.5472355071882382
0.5085391356203549      0.5409668323504451    0.5470894354023417
0.5087096078586966      0.5400986526546461    0.5462130236150358
0.509034750551468       0.5384435734089312    0.5428847187934236
0.5092040067701917      0.5375824334561761    0.5412448094907809
0.5093679957276117      0.5367483779244149    0.5403131860133764
0.5095392195074082      0.5358778309156393    0.5400916199788803
0.5097114270677411      0.535002601836326     0.5402683358196912
0.5098742000055431      0.5341756229931889    0.5402524766153203
0.5100442077657217      0.5333122030763954    0.5395808490410897
0.5102089482645965      0.5324758468251174    0.5381951305392119
0.510537213242728       0.5308102446108538    0.5346956297228052
0.5107007377219849      0.5299810033207155    0.5335714288664584
0.5108714970236183      0.5291154175363493    0.5331727873199774
0.511043240105788       0.5282452043195931    0.5333023139652998
0.5112055485654271      0.5274231319273797    0.5333937908303559
0.5113750918474428      0.5265647697915423    0.5329355225645609
0.5115393678681545      0.5257334242705911    0.5317450778122436
0.5118833733348673      0.5239936641296711    0.52808373031188
0.5120464333359612      0.5231695573950255    0.5268332533097555
0.5122167281594316      0.522309267930664     0.5262990156897323
0.512381755721598       0.5214759647835051    0.5263624410390562
0.5125415160224606      0.5206696169110377    0.5265143326117017
0.5127105948263131      0.5198166251763751    0.5262329370025642
0.5128744063688618      0.5189905925123881    0.52523253231079
0.5132174828792487      0.5172618539580784    0.5216366855644614
0.5133800784021795      0.5164431480644461    0.5202256314889716
0.5135499087474868      0.5155884329603195    0.5194939448220857
0.5137144718314903      0.5147606406981067    0.5194461532006432
0.5138737676541897      0.5139597497493167    0.5196284529179898
0.5140423819798793      0.5131125674440793    0.5195153029897813
0.5142057290442649      0.5122922712397155    0.5187274045708476
0.5145416255564854      0.5106067866777252    0.5153502684607163
0.5147016729206396      0.5098043230488742    0.5137913081620857
0.5148710387877841      0.5089555878967509    0.5128029974964684
0.5150351373936245      0.5081336944721144    0.5125697780339843
0.5152064708218416      0.5072760382230724    0.5127471667146702
0.5153787880305949      0.5064139494151058    0.512763307837079
0.5155416706168175      0.505599519145315     0.5121649546452106
0.5157117880254167      0.5047493934925729    0.5107815841932049
0.5160362210587032      0.5031294825606504    0.5073397201516243
0.5162051224476845      0.5022868713930588    0.506163534382249
0.516368756575362       0.5014710145374032    0.5057635085306746
0.516539625525416       0.5006195914624452    0.5058844293134624
0.5167114782560064      0.49976379188797454   0.5060031021111492
0.5168738963640659      0.49895546394509277   0.505601067746452
0.517043549294502       0.4981116409615587    0.5044167096413115
0.517379555455143       0.4964419670783032    0.5008697339891667
0.5175521597271883      0.49558508118162475   0.49953351026027454
0.5177153293767026      0.4947755474357974    0.49900975354296717
0.5178857338485936      0.49393065767554134   0.499069334499798
0.5180508710591807      0.4931124125583095    0.49923644336337164
0.5182107410084636      0.49232076623447396   0.4989951832676999
0.5183799294607367      0.491483516194714     0.4980014506570738
0.5187150066650517      0.4898270005863573    0.49453310227284686
0.5188871464589337      0.48897686048394684   0.4930395425875232
0.5190498516302852      0.48817385984832035   0.4923343070116892
0.5192197916240131      0.48733572306040945   0.49227964737535873
0.519384464356437       0.4865241239407643    0.49247625410440854
0.5195438698275571      0.48573901354419124   0.4923869889619634
0.5197125938016671      0.4849085780903556    0.4916050888092346
0.5200467420496562      0.48326569744235615   0.4882930107854204
0.5202184173653753      0.48242264949812824   0.4866562177073564
0.5203806580585636      0.48162658819667903   0.48575184330908117
0.5205501335741285      0.4807956281165378    0.48554138911607914
0.5207143418283895      0.4799910824072385    0.48572627476722113
0.520885784905027       0.47915171088054564   0.48575342721379455
0.521058211762201       0.47830816702859824   0.4851146480131595
0.5212212039968441      0.47751137654491604   0.4837872348575019
0.5215563908495797      0.4758746485160185    0.4803015683166281
0.5217160833839914      0.4750957416827273    0.47924342377206014
0.5218850944213933      0.47427200453186474   0.47887160567202464
0.5220488381974911      0.473474551806228     0.4790084535549785
0.5222198167959657      0.47264251156239523   0.47913374239855655
0.5223917791749766      0.4718063543431104    0.47869537651447736
0.5225543069314568      0.47101669393487083   0.47755019658628517
0.5228885648278662      0.46939457459872574   0.47407267996014224
0.523047792884115       0.4686227685649914    0.47284267058379525
0.5232163394433538      0.4678064395270946    0.4722791332777491
0.5233796187412887      0.46701625795643187   0.4723259914171627
0.5235501328616001      0.46619173588533164   0.4725164487218775
0.5237153797206077      0.4653933422964555    0.4722974238020939
0.5238753593183113      0.4646210184464207    0.4714065721087369
0.5242086882583947      0.46301380222904676   0.46804179710846744
0.524379953920161       0.462189054125588     0.46653496454712484
0.5245522033624638      0.46136028949068625   0.4657650527615711
0.5247150181822356      0.46057758699221635   0.4657018921153773
0.5248850678243842      0.4597608013102004    0.4659127751766997
0.5250498502052288      0.4589699982043615    0.4658410508780933
0.5252093653247693      0.45820511645135503   0.4651428893334205
0.5255417653085265      0.4566132915727565    0.46194858705316605
0.5257125664921298      0.4557964277645464    0.46031691116153134
0.5258843514562694      0.45497560569437      0.45934821397531395
0.5260467017978784      0.45420055583205693   0.45913858188602263
0.5262162869618638      0.45339168878392005   0.45933027667367365
0.5263806048645453      0.4526086514362468    0.45938716323802076
0.5265521575896035      0.45179208240667174   0.45882446406662936
0.526724694095198       0.45097161608984676   0.4574798060073982
0.5270581326837018      0.4493882112016507    0.45407437828058944
0.5272232021278381      0.4486054228126661    0.45299973984649233
0.5273830043106705      0.4478482969028365    0.45265060848500493
0.527552124996493       0.4470477566219425    0.45279010772285183
0.5277159784210115      0.4462728732700081    0.45293096176724323
0.5278870666679065      0.4454645388134238    0.4525549779567527
0.5280591386953379      0.44465234564693595   0.4513970282818093
0.5283916483275157      0.4430851281239862    0.44800866629746017
0.5285562532934891      0.4423103997921871    0.44676747113322085
0.5287155909981583      0.44156116201758916   0.4462456826245545
0.5288842472058177      0.44076885923815146   0.44629321490933976
0.5290476361521731      0.44000204115259856   0.4464885332158432
0.5292182599209052      0.4392020490974706    0.4462955292676347
0.5293836164283334      0.4384275168131044    0.44539941518816195
0.5297131134235717      0.4368863991793984    0.4421408581464113
0.529877253911382       0.43611980779830073   0.44071410527095406
0.5300486292215689      0.4353202283570905    0.43993188382250015
0.530220988312292       0.43451688709235614   0.43986123228578067
0.5303839127804844      0.4337582854589492    0.4400717079482847
0.5305540720710534      0.4329667950359609    0.4400197449003092
0.5307189641003185      0.4322005852422785    0.4393127207677211
0.5310475321392308      0.430676115789943     0.4362162334083799
0.531211208148878       0.4299178497865991    0.4346807822630152
0.5313821189809019      0.42912688501773405   0.4337132620130109
0.5315540135934621      0.42833221328790827   0.4334945015077231
0.5317164735834914      0.4275819403421361    0.4336853363462539
0.5318861683958974      0.42679907003435963   0.43375671027935425
0.5320505959469994      0.4260412961864012    0.4332499041359314
0.5322097562367976      0.42530854585682776   0.4320887706855306
0.5325414465610698      0.4237838708188192    0.428768778473645
0.5327118929149307      0.42300170919443736   0.42761042170247265
0.5328770720074876      0.4222446851314607    0.4272113208567647
0.5330369838387405      0.42151256827002576   0.4273257422300655
0.5332062141729834      0.42073861250255473   0.42750213974362616
0.5333701772459224      0.41998955533641164   0.4272289736343131
0.5335413751412381      0.41920829807610166   0.4262045043920061
0.5338763038704111      0.417682382772778     0.4229014409546893
0.5340462857461089      0.41690923662744217   0.4215898907779862
0.5342110003605027      0.41616087327224993   0.4210204739397774
0.5343704477135927      0.41543721633677344   0.4210423318662655
0.5345392135696726      0.41467209946250977   0.42125738786307004
0.5347027121644485      0.41393168022066995   0.4211418719408389
0.5348734455816011      0.4131593583509973    0.420322724795341
0.5352074453544481      0.41165104256113266   0.41714584904073004
0.5353769627519829      0.41088681052264364   0.41569154135928094
0.5355412128882138      0.4101471585662572    0.41493266258568556
0.5357126978468211      0.4093758041315576    0.41483493553995204
0.5358851665859649      0.40860093043080786   0.41505873949541644
0.5360482007025777      0.40786928188826016   0.4150386302981591
0.5362184696415674      0.40710603571378706   0.4143657583748954
0.5365432057356347      0.4056528472327874    0.41141268831663286
0.5367122586550064      0.40489762482049946   0.4098590869703321
0.5368760443130741      0.4041667755550803    0.4089349482319869
0.5370470647935186      0.4034045293280006    0.40869319118023084
0.5372190690544992      0.40263881361881576   0.40889139858234463
0.5373816386929491      0.40191594367800665   0.40897872559501247
0.5375514431537757      0.40116178288046833   0.4084995663892785
0.5377159803532983      0.4004318742176728    0.4073369013654718
0.5380438387327255      0.3989799767992544    0.4041465166688754
0.5382071599126304      0.3982579787517569    0.40305277066420103
0.5383777159149116      0.39750489291770275   0.40263629738241946
0.538543004655889       0.39677593905909053   0.40275902985948936
0.5387030261355623      0.39607103644426983   0.40293477843643877
0.5388723661182259      0.3953259673513438    0.40268206086392944
0.5390364388395854      0.3946050875042254    0.40175146573260667
0.5393800377075941      0.3930983446832547    0.398479306619443
0.5395428944093358      0.3923854989608571    0.3972509676297704
0.5397129859334542      0.39164188607239475   0.3966696862460962
0.5398778101962685      0.390922179435565     0.39670314633111464
0.5400373671977787      0.39022629682560256   0.39691188998978083
0.5402062427022792      0.38949065730481297   0.3968101371523149
0.5403698509454757      0.3887788304987064    0.39606586188693654
0.5407125208571584      0.3872907134283568    0.3929229983336839
0.5408749130807371      0.3865868022579222    0.3915689911073527
0.5410445401266923      0.3858524324515684    0.3908049422375671
0.5412088999113437      0.3851417458993       0.3907138734282865
0.5413804945183718      0.3844007007032744    0.39093749514294795
0.541553072905936       0.38365636032764683   0.39093048091705485
0.5417162166709695      0.3829535918665566    0.3903163755443608
0.5420517065844856      0.38151110450400055   0.3873711162777027
0.5422115506492878      0.38082510657051527   0.38595094766785826
0.5423807132170801      0.380100013734144     0.3850289287112795
0.5425446085235686      0.3793983788374966    0.38480779370838936
0.5427157386524334      0.3786666973056575    0.38500232126372624
0.5428878525618347      0.3779317640947269    0.38509985540079655
0.5430505318487051      0.3772379978535405    0.38466006327529056
0.5432204459579522      0.37651429245078005   0.38350310411178146
0.5435444723925345      0.37513677848033694   0.3804497641117231
0.5437131704821636      0.37442095075443754   0.37936505071902454
0.5438766013104889      0.37372835471018595   0.37898693417958407
0.5440472669611909      0.3730060251692004    0.3791155925928889
0.5442126653505888      0.3723068929758509    0.3792919860393959
0.5443727964786826      0.37163087380443943   0.37905730777457347
0.5445422461097666      0.37091642489963483   0.37812668759339796
0.5448778456717034      0.3695042038609146    0.3750301337194467
0.5450502466443965      0.36878016120227564   0.37377983729950137
0.5452132129945588      0.3680966617720711    0.3732550745491412
0.5453834141670976      0.3673839293897035    0.37329794946682004
0.5455483480783325      0.3666941580917502    0.37350986184426427
0.5457080147282634      0.36602726250935463   0.37341028336452264
0.5458769998811844      0.36532235339397695   0.3726593962459963
0.5462116704867952      0.36392905486611543   0.3696837627893245
0.5463836069813253      0.3632146734634852    0.3683101353190868
0.5465461088533244      0.36254038010418876   0.3676223351329882
0.5467158455477004      0.36183698750409043   0.3675463570504728
0.5468803149807722      0.36115632040734263   0.3677633527228488
0.54703951715254        0.3604982937867828    0.36778664462031835
0.547208037827298       0.35980265307790515   0.3672286086395371
0.547371291240752       0.35912963957426014   0.36604879388111167
0.5477070004511093      0.35774841007088454   0.36300846535824327
0.5478669541643322      0.35709159643446475   0.3621303299978051
0.5480362263805448      0.35639742784925      0.36187113911071245
0.5482002313354537      0.35572575156490965   0.36204502833995433
0.5483714711127392      0.3550253818113196    0.36217909394106107
0.548543694670561       0.35432195320621346   0.3617844365131816
0.548706483605852       0.35365794868267597   0.3607569648463225
0.5490412638598832      0.3522951197584311    0.35772526523299536
0.5492007530949429      0.3516471534430318    0.35670733898058815
0.5493695608329927      0.3509622315385152    0.35629441013505836
0.5495331013097384      0.35029956679089114   0.35640045643555357
0.5497038766088608      0.34960851684654204   0.35659550076026525
0.5498756356885196      0.34891444445261227   0.35636927094671916
0.5500379601456475      0.34825938037074294   0.3555168876298382
0.5503718114433527      0.34691481344540426   0.3525557911409794
0.55054333828393        0.3462254103870374    0.3513353611871611
0.5507158489050437      0.34553301946155845   0.350793727172917
0.5508789249036264      0.3448793862118809    0.3508377067111455
0.551049235724586       0.3441976787306791    0.35105463132568354
0.5512142792842414      0.3435379552545143    0.3509492954806784
0.5513740555825928      0.34290013078675113   0.3502623842642006
0.5517069779239718      0.341574046319759     0.3474220570293871
0.5518780402863861      0.3408941241453669    0.34609221452371725
0.5520500864293367      0.34021124949942955   0.3453912645885208
0.5522126979497566      0.3395667042158285    0.34532884211562415
0.5523825442925528      0.3388943965012413    0.345549145716976
0.5525471233740453      0.33824382866748376   0.3455637207951535
0.5527064351942337      0.33761491575388586   0.345047299301743
0.5528750655174122      0.33695010911886436   0.3438720989045817
0.5532090264635379      0.33563621071548394   0.3409652409712381
0.5533743570864851      0.33498708163046426   0.34011406208512635
0.5535344204481284      0.3343594706247841    0.339894224622223
0.5537038023127618      0.33369621802180305   0.3400728781784173
0.5538679169160912      0.33305446889321766   0.3402022925931611
0.5540392663417971      0.3323853504450807    0.33984146266545406
0.5542115995480394      0.3317133388910347    0.3387943897451853
0.5545446315378391      0.3304173758247596    0.335893123022625
0.5547094976826232      0.32977712559290273   0.33490757219863887
0.5548690965661036      0.329158156514937     0.33455386374526125
0.5550380139525738      0.32850393191689425   0.3346687011954458
0.5552016640777402      0.32787097402215853   0.3348527273460005
0.5553725490252832      0.3272109427250943    0.33464623033222285
0.5555444177533625      0.3265480474624351    0.33377073141493696
0.5558765207868361      0.325269785926783     0.330929290500005
0.5560409224534573      0.3246382985801494    0.32981213875547144
0.5562125589424551      0.32397993386921053   0.32929035117997246
0.5563851792119892      0.32331873475494716   0.32934382484749364
0.5565483648589924      0.32269453887893107   0.3295477666009383
0.5567187853283724      0.3220435660014851    0.329436482013025
0.5568839385364484      0.3214135862218251    0.3287236529553111
0.5572130289329826      0.320160824347688     0.32599990242435045
0.5573769661214407      0.31953802794018615   0.32478551483474083
0.5575481381322754      0.3188886465588666    0.32411771829211
0.5577202939236465      0.3182364594525154    0.3240689158522068
0.5578830150924867      0.31762100144615063   0.32427849076587645
0.5580529710837037      0.31697906708733986   0.3242832373298381
0.5582176598136166      0.31635788752693733   0.32373386480275074
0.5585458212538247      0.3151226331167616    0.32116682054662016
0.5587092939641198      0.31450854299603515   0.31986781269249015
0.5588800014967916      0.31386815917000904   0.31904417488236647
0.5590516928099996      0.3132249943191745    0.3188644217895411
0.5592139495006767      0.3126180088848806    0.319049626086379
0.5593834410137306      0.31198482557423096   0.31915327138067184
0.5595476652654805      0.3113721630220718    0.3187741485600502
0.5597191243396071      0.31073339410806494   0.31773967443073253
0.5600545754264019      0.30948628086364577   0.3149403940789665
0.5602248184809107      0.3088546811448655    0.31401860922690317
0.5603897942741154      0.3082434664693688    0.31374038278608135
0.5605495028060162      0.30765255592362106   0.31387853968508816
0.5607185298409072      0.3070280121625412    0.314043770702639
0.560882289614494       0.306423757504558     0.31380585989079685
0.5610532842104574      0.3057936741892992    0.3129272399977568
0.5613878063409263      0.30456357755540997   0.3101703621302227
0.561557584917272       0.3039405625310617    0.30911800219660046
0.5617220962323137      0.3033377031108933    0.3086972682153638
0.5618813402860514      0.30275491973838653   0.3087608240074741
0.5620499028427792      0.30213886155608694   0.30896340881289386
0.562213198138203       0.30154286462524865   0.3088619946557334
0.5623837282560036      0.30092131194924293   0.30815741553313525
0.5627089867076926      0.2997382034058958    0.305570153720802
0.5628783008058753      0.29912357762504405   0.3043724346514174
0.563042347642754       0.29852888317418885   0.3037622797675891
0.5632136293020094      0.29790881127584984   0.30370885226852845
0.5633858947418009      0.29728605198828195   0.3039216600271089
0.5635487255590617      0.2966982043373804    0.30392313854467845
0.5637187911986992      0.2960850707945769    0.30337389848362695
0.5640431206940622      0.29491815635053276   0.30092970876139863
0.5642119703140818      0.29431194631450636   0.299650946649276
0.5643755526727974      0.2937254408835892    0.2989029635887511
0.5645463698538897      0.2931138277921222    0.2987280944488977
0.5647181708155182      0.29249954753494595   0.2989180522935588
0.564880537154616       0.29191978759404635   0.2990086244678531
0.5650501383160904      0.2913150085422785    0.2986219036196324
0.5652144722162609      0.2907298036957452    0.29766274095164474
0.5655585934418914      0.28950687951923687   0.2949188833699806
0.565721711322444       0.2889283838572348    0.29408577391536034
0.5658920640253733      0.28832504171324447   0.29382185470718003
0.5660571494669985      0.28774114414211965   0.2939707458352398
0.5662169676473197      0.2871766150163715    0.2941150763413112
0.5663861043306311      0.28657995932184394   0.2938631162483531
0.5665499737526385      0.2860026570796259    0.2930413802807087
0.566893166021943       0.28479606614202474   0.2903337989383109
0.5670558194243325      0.2842253684167665    0.2893843296569448
0.5672257076490987      0.2836300778971365    0.28898578134863767
0.567390328612561       0.2830540138875294    0.28906871008134166
0.5675496823147193      0.282497101904787     0.28925000011721086
0.5677183545198676      0.28190839335519197   0.2891298770575182
0.567881759463712       0.28133882211982386   0.28846207604368396
0.56821777173485        0.28016992521140655   0.285892047017161
0.5683778769784631      0.2796140572632415    0.2848261863214323
0.5685473007250663      0.2790266038924368    0.28424646368997053
0.5687114572103655      0.27845816486302394   0.28421356349774757
0.5688828485180415      0.27786545975776233   0.2844171459784779
0.5690552236062536      0.2772701609591395    0.2844013589899484
0.5692181640719349      0.27670818820109666   0.2838707471975484
0.569553247386747       0.27555476743146      0.2814279725762765
0.5697128881521969      0.27500632100479283   0.2802877568228265
0.5698818474206371      0.27442660879992253   0.2795767470263055
0.5700455394277733      0.273865700732416     0.2794387018502563
0.5702164662572862      0.27328076843343513   0.27962260610780254
0.5703883768673352      0.27269328804892534   0.27969698564232687
0.5705508528548535      0.272138782720226     0.2793112281403539
0.5707205636647485      0.271560335607047     0.27833572242352045
0.5710441835006264      0.2704594155022282    0.2758448898259595
0.5712126782909035      0.26988730856649623   0.2749979271971672
0.5713759058198766      0.2693337980549544    0.27473238616250983
0.5715463681712265      0.2687564998183687    0.2748667670846897
0.5717115632612721      0.2681977649969344    0.27501146810726995
0.571871491090014       0.2676575234184098    0.27479736123597603
0.572040737421746       0.2670865279503544    0.27400568073082515
0.5723759303849787      0.2659578593884269    0.27146841312374853
0.5725481280583196      0.26537915913844184   0.2704836375011129
0.5727108911091296      0.2648328662698909    0.2700983213494958
0.5728808889823165      0.2642630155780265    0.2701658354857615
0.5730456195941992      0.2637115258663786    0.2703439418883679
0.5732050829447779      0.26317832871732444   0.270243170032984
0.5733738647983468      0.26261467717238296   0.2695981116231985
0.5737081288052535      0.2615005215286226    0.2671504274144406
0.5738798620004313      0.2609292025559899    0.2660606402030454
0.5740421605730783      0.2603899512037211    0.2655415531973428
0.5742116939681021      0.2598273657574973    0.2655152161052799
0.5743759601018217      0.25928294376271466   0.2657024805204968
0.5745474610579182      0.2587152602137315    0.26568929237862376
0.5747199457945509      0.258145056363043     0.26513884721229286
0.5750532808451314      0.25704519333230025   0.2627868379393019
0.5752182985203058      0.2565017176562672    0.2616756218827284
0.5753780489341764      0.2559762253294557    0.2610521797169447
0.5755471178510372      0.2554207595745535    0.2609275695122985
0.5757109195065938      0.25488326343949175   0.2610989753647126
0.5758819559845272      0.2543227227260332    0.2611675658171217
0.5760539762429968      0.25375967245946235   0.2607579949878215
0.5762165618789357      0.25322815998097215   0.2598433867590146
0.5765509355342627      0.25213704609021176   0.2573880106679398
0.5767102214699702      0.25161819619378717   0.2566446653429944
0.5768788259086679      0.251069651995552     0.2564003865831699
0.5770421630860616      0.2505388892182021    0.25653055787312234
0.577212735085832       0.2499852921611741    0.2566640608444114
0.5773780398242984      0.24944944641657424   0.25641796068920397
0.5775380773014607      0.24893128839436707   0.2556717793219726
0.5778715220004615      0.2478536160643312    0.2532455157254819
0.5780428455416868      0.24730092004305468   0.25231025267540474
0.5782151528634483      0.24674573959530796   0.2519377122238903
0.5783780255626789      0.24622159139167413   0.2520118094987254
0.5785481330842863      0.24567481577494218   0.25217992252733035
0.5787129733445895      0.24514560749057315   0.2520446759148477
0.5788725463435889      0.24463390462696574   0.2514255531014887
0.5792050620862638      0.24356949990147672   0.24907762583579826
0.5793759211493259      0.2430235496503241    0.2480453113998738
0.5795477639929243      0.24247512466447924   0.24754300891909958
0.5797101722139919      0.2419574241524116    0.24753759972208472
0.5798798152574363      0.24141729719045485   0.2477188531864531
0.5800441910395765      0.2408945582747159    0.24768809744912845
0.5802158016440935      0.24034945806958485   0.2471484197411986
0.580551555791669       0.23928487775183777   0.24486345102383295
0.5807219503765682      0.23874556400205538   0.24378041461133645
0.5808870777001632      0.23822353482067976   0.24320454196179306
0.5810469377624543      0.2377187306101542    0.2431195918608948
0.5812161163277356      0.23718511274178344   0.243290564427387
0.5813800276317127      0.23666870723087197   0.24333530685328905
0.5815511737580668      0.23613013485082993   0.24292471935251939
0.581723303664957       0.23558911014368414   0.2419680740885312
0.5820559290560525      0.23454544943049013   0.23963268425898582
0.5822205919014845      0.23402967961377927   0.23894076433515204
0.5823799874856126      0.23353096346954832   0.23875458613294026
0.5825487015727309      0.23300368463719998   0.2388922972309999
0.582712148398545       0.23249344714869763   0.23899951887025025
0.5828828300467359      0.23196118965230575   0.23872202121141697
0.5830544954754631      0.23142645169308468   0.23789566453801442
0.5833861919102326      0.23039498151154533   0.2355663875528325
0.5835503902775016      0.22988523419531715   0.2347564424686498
0.5837218234671473      0.22935363014503385   0.2344453572731972
0.5838942404373293      0.22881959546005773   0.23454631637576578
0.5840572227849803      0.22831535306999354   0.23468217030794433
0.5842274399550083      0.22778931695596866   0.23448914019810507
0.5843923898637321      0.22728013187955623   0.23379486388432835
0.584721073661562       0.22626719118248143   0.23152551941789623
0.5848848075506679      0.22576342413744932   0.23062347475648406
0.5850557762621507      0.2252379838949852    0.23019467370437616
0.5852277287541696      0.22471012274991287   0.23022751118825424
0.5853902466236577      0.2242117779090837    0.23038672031777757
0.5855599993155225      0.22369182118069458   0.2303007439256983
0.5857244847460832      0.22318855505500676   0.2297372823890361
0.5860522395875871      0.22218737110964748   0.22755714231263388
0.58621550899853        0.22168944210618358   0.22656939375644322
0.5863860132318497      0.2211700184775777    0.22601104777734668
0.5865512502038652      0.22066719471086615   0.2259468337222811
0.586711219914577       0.22018091714499097   0.22610310662045502
0.5868805081282789      0.21966686611136638   0.22613057117988528
0.5870445290806766      0.2191693501331813    0.22573269177005637
0.5872157848554509      0.218650455397367     0.22481142192882128
0.5873880244107617      0.21812916152515857   0.2235983067507142
0.5875508293435416      0.21763695675399689   0.22255227230787109
0.5877208690986981      0.21712343209858756   0.22188314496164782
0.5878856415925509      0.21662635211498293   0.22172567727065862
0.5880451468250996      0.21614566454844936   0.22185732663494773
0.5882139705606382      0.21563743126887175   0.2219476727131414
0.588377527034873       0.21514557950142527   0.22166876932227378
0.5885483183314844      0.21463252045662332   0.22086655002416622
0.588882433863249       0.21363045008321943   0.2186096694051875
0.5890520091402425      0.21312267949342523   0.21782641180018386
0.5892163171559321      0.21263107995276717   0.21755738786601922
0.5893878599939983      0.21211837599418776   0.2176543395581109
0.5895603866126009      0.21160329336677808   0.21777878203059298
0.5897234786086726      0.21111689481429224   0.21757476156251776
0.589893805427121       0.21060945444072385   0.21685926401198793
0.5902186572801058      0.20964316177846695   0.21468894145950712
0.5903877680789362      0.20914091338335966   0.2138157839845824
0.5905516116164629      0.208654818007104     0.21344340940351048
0.590722689976366       0.20814779223929739   0.21347957539756116
0.5908947521168055      0.20763839995955946   0.21363118272503714
0.591057379634714       0.20715744413303008   0.21352490785741796
0.5912272419749995      0.20665561497589494   0.21293324914899375
0.5915511648716583      0.20570011724502074   0.21083568648015968
0.5917198111923259      0.20520341438902276   0.20987681304298436
0.5918831902516892      0.2047227228731789    0.20938948347585623
0.5920538041334295      0.2042212671770855    0.20934235793642791
0.5922191507538657      0.2037358003869982    0.20949560481540253
0.5923792301129979      0.20326627370116498   0.2094946367604524
0.5925486279751202      0.20276992297964114   0.20906252159281052
0.5928841239991335      0.20178842702449112   0.20701905348909144
0.5930564732028647      0.20128501008286004   0.20597173275149322
0.5932193877840652      0.20080964351919808   0.20538559700586417
0.5933895371876423      0.2003136765758623    0.20525275766763976
0.5935544193299156      0.19983355917604592   0.205390168341376
0.5937140342108846      0.1993692434989954    0.2054500054960239
0.593882967594844       0.19887831673495426   0.20513294512246974
0.5940466337174993      0.19840318241539823   0.2043592431767346
0.5943894193880994      0.19740959530856603   0.2021357328644159
0.5945518694911369      0.1969394505810931    0.2014460898200254
0.594721554416551       0.19644886582100377   0.20120943153358373
0.5948859720806612      0.19597399398280455   0.2013097994541758
0.595057624567148       0.19547873409752936   0.20141672057903906
0.5952302608341711      0.19498115754184717   0.20117116269669383
0.5953934624786632      0.19451118177518148   0.20047206635423012
0.595729068151097       0.19354599251278307   0.19830850857628524
0.595888970095358       0.19308681244726564   0.19754859216342574
0.596058190542609       0.19260135888325947   0.19721521629686847
0.5962221437285562      0.19213149199698099   0.19726671201240978
0.5963933317368799      0.19164139082087409   0.19740278342092146
0.59656550352574        0.19114898762394392   0.19725621729906218
0.5967282406920692      0.19068404094911645   0.1966659804734313
0.597062917408177       0.18972930212964337   0.19455714152011913
0.5972223548742748      0.18927515538884684   0.1937169758806063
0.5973911108433629      0.18879494564302796   0.19327437592099575
0.597554599551147       0.18833019485195826   0.1932571281488569
0.5977253230813075      0.18784537141205584   0.19340467780193157
0.5978970303920046      0.1873582627141256    0.19335279401180594
0.5980593030801709      0.18689838699428227   0.19288255123415224
0.5983930508399526      0.1859539853135862    0.1908693996888225
0.5985645259115682      0.18546951158625993   0.18989814349943246
0.5987369847637201      0.18498276927842588   0.1893659307766144
0.598900008993341       0.18452312567275408   0.18929491242036228
0.5990702680453387      0.1840435716298911    0.18943716883326103
0.5992352598360323      0.18357932874940572   0.18944600213904203
0.599394984365422       0.1831303518727956    0.18908244676701805
0.5995640273978018      0.1826556580187225    0.1882480983338849
0.5998988137623302      0.18171698192685307   0.18616113713892277
0.6000708081363191      0.18123548926883457   0.1855245370542348
0.600233367887777       0.18078087364246317   0.18536720720712835
0.6004031624616115      0.1803065072956091    0.18548625267228022
0.6005676897741422      0.17984732610828394   0.1855544674305981
0.6007269498253689      0.1794032853352697    0.1852988898379878
0.6008955283795856      0.17893373433699558   0.18457420842668015
0.601229385787788       0.17800525166814127   0.18249192100708625
0.6013946646417734      0.17754630048179249   0.1817668978012702
0.6015546762344548      0.17710241789861222   0.18149130415532402
0.6017240063301265      0.17663299438401114   0.18155281742634077
0.6018880691644941      0.17617859455374957   0.1816708379731264
0.6020593668212384      0.17570464528116259   0.18151498251219836
0.602231648258519       0.17522847807570674   0.1808838761762119
0.6025645767103952      0.17430973413879464   0.17884191794847792
0.6027293910862176      0.17385561554087828   0.17803633386851384
0.6028889382007361      0.17341645249345658   0.17766568436280616
0.6030578038182446      0.17295211437223607   0.17766570677700355
0.6032214021744493      0.17250272590875332   0.17779889204826832
0.6033922353530305      0.17203395411383843   0.17773279602697786
0.6035640523121482      0.17156298826378633   0.1772193737968929
0.6038960518076981      0.17065438568775815   0.1752542685650292
0.6040604017053576      0.17020530197058326   0.17437288369059115
0.6042319864253937      0.169736946158447     0.17387790506613376
0.604404554925966       0.1692664175951903    0.17382417421260088
0.6045676888040077      0.16882208679158495   0.17395647461111505
0.6047380575044259      0.16835854176052026   0.17394207904844958
0.6049031589435401      0.16790980757700258   0.17353303584020768
0.6052321458021507      0.1670170534250144    0.17167435744437665
0.605396031221647       0.16657302746411976   0.17074075646941628
0.6055671514635199      0.166109898020777     0.17014908938555423
0.6057392554859293      0.1656446200266089    0.1700141008059197
0.605901924885808       0.16520532260545556   0.17013164950361873
0.6060718291080631      0.16474697986511241   0.17017940391686961
0.6062364660690143      0.1643033271135214    0.16987655847011018
0.6063958357686615      0.1638743193886611    0.16917612075320548
0.6067279449126322      0.16298174191276593   0.16717450921620955
0.6068986006763422      0.16252383865568137   0.1664822945299202
0.6070639891787482      0.16208055605987032   0.1662500106822883
0.6072241104198502      0.16165184873210137   0.16632432137729558
0.6073935501639423      0.16119868296262485   0.16642843521431472
0.6075577226467305      0.16076008636934405   0.16625335518914652
0.6077291299518953      0.16030266814789615   0.16562193221313598
0.6080644775007669      0.15940898089772065   0.16362902435317386
0.6082346687863138      0.1589561025567981    0.16285704022259154
0.6083995928105568      0.15851773127018798   0.1625356172727793
0.6085592495734957      0.15809382207061987   0.16256188894637363
0.6087282248394249      0.15764566711104738   0.16268763551125165
0.60889193284405        0.15721196985422328   0.16259704839573189
0.6090628756710517      0.15675961999493163   0.1620746640148195
0.6093972942635972      0.15587620393631824   0.1601460200120146
0.6095670210709812      0.15542862027340873   0.15929696319905454
0.6097314806170611      0.1549954263391677    0.15887541417251205
0.6098906729018371      0.15457657621003257   0.15883533718753656
0.610059183689603       0.1541337129245352    0.15896522635808008
0.6102224272160651      0.15370518889195717   0.15895333798591396
0.6103929055649039      0.15325819592979592   0.15854951648486715
0.6107180604786695      0.1524071289026489    0.15677582639950424
0.6108873228078904      0.15196487318525043   0.15584527068600318
0.6110513178758072      0.15153688792247694   0.15529462625672635
0.6112225477661009      0.15109055659148038   0.15515311064683698
0.6113947614369307      0.15064221370747702   0.1552705551965933
0.6115575404852298      0.15021894409625294   0.15531453406878112
0.6117275543559055      0.14977739375342106   0.15501305194075865
0.6118923009652772      0.1493500429996602    0.1543030868458615
0.6122205781644029      0.14850002670058965   0.15239254313148878
0.6123841087541567      0.14807735633679714   0.15175262439060977
0.6125548741662872      0.1476365303000545    0.15151932767518217
0.6127266233589541      0.14719372663962413   0.1516030134342945
0.6128889379290903      0.1467757667367752    0.15169174803893215
0.613058487321603       0.14633971763708176   0.15149502660572584
0.6132227694528116      0.1459177431298495    0.15089027419494452
0.6135667871405186      0.14503579075272255   0.14891874490725937
0.6137298532521095      0.14461854304129626   0.14822665059674697
0.613900154186077       0.14418333585944884   0.14792990107029572
0.6140651878587405      0.1437621294879286    0.1479759204891236
0.6142249542701002      0.14335474034359594   0.1480871695271041
0.6143940391844498      0.14292405365287192   0.14797463387861534
0.6145578568374954      0.14250732155341433   0.1474675899470133
0.6149009455688764      0.14163627154705444   0.14555089476913094
0.6150635472023042      0.1412242714509067    0.1447878102915369
0.6152333836581085      0.14079450653163356   0.1443958887556761
0.6153979528526091      0.14037862500958054   0.14438373328062684
0.6155572547858057      0.1399765766327955    0.14450499980017129
0.6157258752219923      0.13955157217923675   0.1444726686514247
0.6158892283968749      0.13914039728842606   0.14407288116112651
0.6162251371300895      0.13829660569064944   0.14228431942293532
0.6163851906047408      0.13789537594953905   0.141455572064419
0.6165545625823824      0.13747136517307715   0.14093856199531776
0.6167186672987198      0.137061110851384     0.1408325404927041
0.6168900068374339      0.1366333711761017    0.1409507302202945
0.6170623301566842      0.13620379776415226   0.14098009652414906
0.6172252188534039      0.13579831935313516   0.14067353045731265
0.6173953423725002      0.13537543145828637   0.13994298013722073
0.6177197876267808      0.1345706440851883    0.13812841800914014
0.6178886951262592      0.13415256071489143   0.137524760078829
0.6180523353644336      0.13374810065215637   0.13733942656464454
0.6182232104249847      0.13332637634560499   0.13743013017528244
0.6183950692660721      0.13290286305750476   0.13750927652455225
0.6185574934846286      0.13250319098484212   0.13729953159920494
0.6187271525255618      0.13208633251264884   0.13666909015687878
0.619063170907197       0.13126259082968875   0.1348087864632856
0.6192357812897392      0.13084041042638653   0.13413435670171223
0.6193989570497506      0.13044191438548672   0.13389499804676072
0.6195693676321385      0.1300263838868394    0.13395885922677164
0.6197345109532227      0.12962431741575856   0.13406111972436222
0.6198943870130028      0.12923565875297044   0.13392996355516357
0.620063581575773       0.1288249747596589    0.13339310670269056
0.620398671001082       0.12801352171913444   0.13157632517076748
0.6205708169054611      0.12759744637751455   0.13082997228704238
0.6207335281873095      0.12720479920794672   0.13050729152240576
0.6209034742915346      0.12679534487463381   0.13052060746795605
0.6210681531344556      0.12639921934205997   0.13064114275327765
0.6212275647160728      0.12601636529068547   0.13058629775674435
0.6213962948006799      0.1256117789330688    0.13015339446598073
0.6217304552696629      0.12481249700215927   0.12841289144141915
0.621902136695879       0.12440287973999363   0.1275990820545385
0.6220643834995643      0.12401641905770927   0.1271841446019079
0.6222338651256263      0.12361339956312255   0.12712861092447456
0.6223980794903843      0.12322356548906413   0.12725051260466025
0.6225695286775189      0.12281725306158817   0.12725508497160382
0.6227419616451899      0.12240933048948269   0.1268892961472365
0.6230751931578469      0.12162307169111372   0.12522476065505458
0.6232401590640597      0.12123484768051121   0.12440323318203002
0.6233998577089686      0.12085966218971635   0.12391846881728844
0.6235688748568675      0.12046327636244498   0.12379281292502817
0.6237326247434625      0.12007992528419834   0.12390003072202815
0.6239036094524341      0.11968035636408436   0.12395886639669575
0.624075577941942       0.11927923328588366   0.12369234831409927
0.6242381118089191      0.1189008066166637    0.12306089893105035
0.6245723819263227      0.11812464728113872   0.12129043588259991
0.6247316160930685      0.11775592221211338   0.12072508389118078
0.6249001687628044      0.11736633279450853   0.12051494005009968
0.6250634541712363      0.11698962107450633   0.1205898680377097
0.6252339744020449      0.11659695994962016   0.1206909096365951
0.6253992273715494      0.11621715428828687   0.12053835856013866
0.62555921307975        0.11585013871452188   0.12002987931685893
0.6258925542408276      0.11508761155168339   0.11829744217093761
0.626063826013091       0.11469697230932106   0.11759907571491122
0.6262360815658907      0.1143048810726605    0.11729884867836628
0.6263989024961596      0.11393499834945163   0.11733254939705028
0.6265689582488051      0.11354944430000129   0.11745537860193482
0.6267337467401466      0.11317650686931835   0.11738039792278039
0.6268932679701842      0.11281611280591762   0.1169643830074956
0.6272256801749356      0.11206736162818837   0.11530111998602656
0.627396487469036       0.11168380852174437   0.11454040062141116
0.6275682785436727      0.11129886436947005   0.11415091737216343
0.6277306349957785      0.11093581894381147   0.11412725038423313
0.6279002262702611      0.11055738618570249   0.11425666186166751
0.6280645502834397      0.11019148128172568   0.11425442226981215
0.6282361091189949      0.10981028361707924   0.11389857709092992
0.6285717597286471      0.10906691032343266   0.11229099781270625
0.6287421025445844      0.10869088680815299   0.11150033779139251
0.6289071780992177      0.10832729148474886   0.11106364770967586
0.6290669863925471      0.10797605249764604   0.11098435612240884
0.6292361131888666      0.10760514508180412   0.1111049410578661
0.6293999727238822      0.10724658981917437   0.11115526052655098
0.6295710670812743      0.10687304842889138   0.1108928568149965
0.6297431452192028      0.10649823413532844   0.110236728450427
0.6300756670723747      0.10577644916941176   0.10856927398359527
0.630240278148845       0.10542036584259563   0.10805484321097351
0.6303996219640113      0.10507645587950049   0.10790479324928393
0.6305682842821678      0.10471327294845072   0.10800041192605102
0.6307316793390203      0.10436225871955042   0.10809379624926144
0.6309023092182494      0.10399657462516954   0.10792727167676847
0.6310676718361745      0.10364303334653904   0.10739574056861742
0.6313971810524069      0.10294107321877519   0.10576051920731698
0.6315613276507142      0.10259264947559814   0.10515174614045558
0.6317327090713981      0.10222976909834249   0.1048953045051311
0.6319050742726184      0.101865737525731     0.10495554811880002
0.6320680048513079      0.1015224947463816    0.10507383337836601
0.6322381702523739      0.10116491001378984   0.1049850695783308
0.632403068392136       0.10081927456038083   0.10454227580899192
0.6327316486520425      0.10013315217245014   0.10296186914678929
0.6328953307721867      0.09979265986568053   0.10229634030289093
0.6330662477147077      0.09943795742990856   0.10195949000042068
0.6332381484377649      0.09908215618548415   0.10196896167509768
0.6334006145382913      0.09874677233509889   0.10209977786577416
0.6335703154611944      0.09839738090261631   0.10208561962603768
0.6337347491227934      0.09805974233923195   0.10174015551271252
0.6340624004263737      0.09738964417799514   0.10024560319746256
0.634225618068355       0.09705717952423477   0.09952958957268379
0.6343960705327129      0.0967109357621401    0.09910570317296684
0.6345612557357667      0.09637632949244383   0.09904485658349704
0.6347211736775167      0.09605327641788872   0.0991676992488172
0.6348904101222567      0.09571234991478539   0.09923018256826868
0.6350543793056926      0.09538297129356667   0.09900523238449456
0.6352255833115054      0.0950400481135378    0.09840425678407926
0.6355605242616726      0.09437209684661929   0.09682287146661973
0.6357305122478675      0.09403459752000684   0.0963271455209979
0.6358952329727583      0.09370852321059514   0.09620098846052645
0.6360546864363452      0.0933937865802467    0.09630364999929864
0.6362234584029222      0.09306163702607635   0.09640819073573352
0.6363869631081951      0.09274081930700025   0.09626941447310147
0.6365577026358448      0.09240682476689263   0.09576101951375429
0.636891714629686       0.09175646998302162   0.09420643368636991
0.6370612381377178      0.09142793260042237   0.09363841036005777
0.6372254943844455      0.09111059921098158   0.09343535717782175
0.6373969854535501      0.09078033913013502   0.09351271090666724
0.637569460303191       0.09044927162824407   0.09364165645592012
0.637732500530301       0.09013732076684118   0.09355789715702535
0.6379027755797876      0.08981257628983853   0.093117337785837
0.6382275238948489      0.08919621201028297   0.09164506808429154
0.6383965829247177      0.08887690243676835   0.0910232802295161
0.6385603746932824      0.08856856592641515   0.09075131921988056
0.6387314012842239      0.08824769055828649   0.09078498895303802
0.6389034116557017      0.08792608764251028   0.09093081746987777
0.6390659874046487      0.08762315949449569   0.09091671345816128
0.6392357979759723      0.0873078200613567    0.09057002719717512
0.6395596173347076      0.08670950880998479   0.08917372714202451
0.6397282118864133      0.08639960558299938   0.0885029629176929
0.639891539176815       0.08610043746447679   0.08815562562538402
0.6400621012895934      0.08578912809118734   0.0881306402100414
0.6402273961410678      0.0854885198502784    0.08827309848063021
0.6403874237312384      0.08519851458799728   0.08833105813874989
0.6405567698243989      0.08489272354660177   0.08810190537179317
0.6407208486562554      0.08459752831017378   0.08755008243649354
0.6408921623104887      0.0842904606692455    0.08677309907411163
0.6410644597452583      0.08398281308539043   0.0860529498756749
0.6412273225574969      0.08369310715628252   0.0856435265551658
0.6413974201921123      0.0833916730116185    0.08556046995782454
0.6415622505654237      0.08310069006703727   0.0856906233062641
0.6417218136774312      0.08282005705515336   0.08579079822711845
0.6418906952924287      0.08252416614028456   0.08564633061543014
0.6420543096461222      0.08223861744040745   0.08517639040169747
0.642396991778799       0.08164412048424276   0.0837003861797438
0.6425593901128748      0.08136408307584386   0.0832274150862513
0.6427290232693271      0.08107274138333583   0.0830750133029698
0.6428933891644755      0.08079159220265116   0.08317826973427578
0.6430649898820004      0.08049927608873143   0.08331565313433432
0.6432375743800618      0.08020653395216525   0.0832269090197813
0.6434007242555924      0.07993095202200985   0.08281693837707621
0.6437362263901026      0.07936779592209915   0.08140633602243481
0.6438960765654019      0.07910116936611254   0.08089420394758694
0.6440652452436912      0.07882019330526865   0.08067971674658277
0.6442291466606767      0.07854914072508155   0.08074969289000965
0.6444002829000386      0.07826736227281804   0.08090702243057929
0.6445724029199369      0.07798524526814703   0.08088996695685041
0.6447350883173043      0.07771977793932308   0.08056366236696642
0.6450696614954887      0.07717746646661167   0.07921865691469006
0.6452290471926249      0.0769208479563235    0.07866223151166918
0.6453977513927511      0.07665044882744987   0.0783786888057135
0.6455611883315734      0.07638972140114651   0.07840243599703745
0.6457318600927724      0.0761187363428703    0.07856710424731629
0.6458972645926675      0.07585735539020436   0.07862034033047799
0.6460574018312584      0.07560546585324683   0.07840109842062683
0.6462268575728395      0.07534017410836949   0.07785605769949097
0.6463910460531166      0.07508436408224477   0.07716554023435752
0.6465624693557703      0.07481858411130735   0.07652359495363527
0.6467348764389604      0.07455262516602526   0.07617051723284114
0.6468978488996198      0.07430246592412475   0.07614817512338723
0.6470680561826556      0.07404249823984735   0.07630794522630624
0.6472329962043876      0.07379184396686492   0.07640822843820916
0.6473926689648156      0.07355038747110833   0.07626549624212964
0.6475616602282337      0.07329612250499691   0.07579680039946392
0.6478963430548385      0.07279646841068622   0.0744780350931729
0.6480682856598655      0.07254180237029119   0.07406409452676846
0.6482307936423617      0.07230238239543375   0.07398543921569016
0.6484005364472346      0.07205362814577615   0.07412737416817801
0.6485650119908035      0.07181388832931468   0.07426693303559811
0.6487242202730685      0.07158304439522158   0.07420197389071152
0.6488927470583236      0.07133999907976733   0.07382013208686797
0.6492265009286023      0.07086266035838258   0.07253987998020267
0.6493917280136261      0.07062832301386414   0.07207741477371446
0.649551687837346       0.07040270582954608   0.07192162433268111
0.6497209661640557      0.0701652885434964    0.07202137987053385
0.6498849772294616      0.06993658041266383   0.07219105816033303
0.6500562231172442      0.0696991761757646    0.07220467066360792
0.650228452785563       0.06946184769631866   0.07189939688840609
0.6505612776995157      0.06900733098517867   0.07067963159212282
0.6507260403063764      0.06878433793609567   0.07017644061433055
0.6508855356519331      0.0685697483318515    0.06996350827751527
0.6510543495004799      0.0683439915332929    0.07002305076849903
0.6512178960877228      0.06812662666397175   0.07020192931058382
0.6513886774973422      0.06790106657982714   0.07027803841337701
0.651560442687498       0.0676756754484523    0.07006107473606168
0.6518923386451246      0.06724446505544285   0.0689264131747158
0.6520566367738223      0.06703310970584492   0.06838760081071897
0.6522281697248966      0.06681390186069566   0.06810382828764831
0.6524006864565073      0.06659493957434652   0.06813361875787047
0.6525637685655871      0.066389341435617     0.06831397047618488
0.6527340854970436      0.06617606791498003   0.06842775004143567
0.652899135167196       0.06597080307578744   0.06828328981963366
0.6530589175760444      0.06577341730085919   0.06786582907908772
0.653228018487883       0.06556594718685307   0.0672484470979643
0.6533918521384177      0.06536634294405032   0.06669051980759069
0.6535629206113289      0.06515940233209189   0.06635351999498457
0.6537349728647766      0.06495279880702007   0.06633389057751106
0.6538975904956934      0.0647589362010922    0.06650512130358768
0.6540674429489868      0.06455791736878254   0.06666177506821466
0.6542320281409763      0.06436456760764543   0.06659426013005021
0.6543913460716617      0.0641787547855071    0.06625131052565403
0.6547233516777089      0.0637958153328969    0.06510734768035364
0.6548939556724571      0.06360129582921613   0.06471652372532787
0.6550592924059013      0.06341424923622868   0.06463548669538623
0.6552193618780416      0.06323454107079586   0.06477698600599763
0.6553887498531721      0.06304585294699741   0.06497075667846725
0.6555528705669986      0.06286448934661523   0.06499247307275052
0.6557242261032016      0.06267666512176245   0.06471631655265335
0.6560594701141496      0.062313749948854025  0.06361297292913291
0.6562296096307347      0.06213187795297357   0.06318362472740657
0.6563944818860159      0.061957125964335266  0.06305223725188522
0.6565540868799933      0.06178935698785016   0.06316568755598859
0.6567230103769606      0.0616132969842681    0.06337575487486427
0.656886666612624       0.06144420564681683   0.06345727339781689
0.6570575576706641      0.06126919656329236   0.06326250393505535
0.6573918727252859      0.060931441720511764  0.06223288070311356
0.657561547763708       0.060762366103087016  0.061768065683071804
0.6577259555408261      0.06060005051315628   0.06158078195377107
0.657897598140321       0.06043218335284329   0.06166897687968846
0.6580702245203522      0.060265083841212305  0.06189126662536739
0.6582334162778527      0.06010872967122107   0.06200977069676867
0.6584038428577296      0.059947023038404164  0.06187133217346326
0.6585690021763027      0.05979185665359815   0.06146457854275976
0.6587288942335716      0.05964308808917432   0.06093970201624098
0.6588981047938308      0.05948720516029702   0.06045451123788287
0.659062048092786       0.05933770357576935   0.06022216846397745
0.6592332262141178      0.059183213519725164  0.0602707908320919
0.659405388115986       0.05902949686482716   0.06049253910997551
0.6595681153953233      0.058885738638135524  0.060655483752337924
0.6597380774970374      0.05873718441378441   0.06059243727032886
0.6599027723374473      0.058594792184707076  0.06025358534690259
0.6602309459986494      0.05831564358101793   0.05926054937909445
0.6603944248194416      0.05817887039231374   0.058980995717198725
0.6605651384626103      0.05803766957974559   0.058980773958738464
0.6607368358863155      0.05789733323354472   0.05919064360741972
0.6608990986874899      0.05776625807273923   0.05939147403939045
0.6610685963110406      0.057630950036681644  0.059405367781411866
0.6612328266732876      0.05750142035509651   0.059143613502719135
0.6615767408230712      0.057235203150106904  0.05814481983414328
0.6617397551657003      0.05711140173101368   0.05784316242744943
0.661910004330706       0.056983747431355286  0.05781389128398661
0.6620749862344077      0.05686164581983297   0.058004384481812596
0.6622347008768056      0.05674494857754597   0.0582272885335477
0.6624037340221935      0.056623059569268594  0.05830469159754901
0.6625674999062774      0.056506557533494754  0.05811446921380508
0.6629104850997348      0.05626763964334973   0.05717022996741476
0.6630730349642009      0.056156817721313555  0.05683332308442518
0.6632428196510436      0.05604272071976407   0.05675453554522135
0.6634073370765823      0.05593378188731979   0.05691913086371335
0.6635665872408172      0.05582985101760423   0.057159081516672657
0.663735155908042       0.05572147085575218   0.05729816350263334
0.6638984573139629      0.05561808092531413   0.057185490097081494
0.6640689935422606      0.05551179826556574   0.056809192712219535
0.6642342625092541      0.05541044666658754   0.05633471706150126
0.6643942642149436      0.05531406277155775   0.05596234467838225
0.6645635844236233      0.05521375917831446   0.05581421348363615
0.6647276373709989      0.0551182074280494    0.05593083520841011
0.6648989251407513      0.05502015755080412   0.0561945075079292
0.6650711966910401      0.054923314690299964  0.056388244657962994
0.6652340336187981      0.054833409659907975  0.05634395524451954
0.6654041053689325      0.0547412078462927    0.056029857034229344
0.6655689098577631      0.05465351849193756   0.05558008563705636
0.6657284470852896      0.05457018715213168   0.0551913426261648
0.6658973028158062      0.054483657336571234  0.054999100504576015
0.6660608912850188      0.05440146549093216   0.055077680187159694
0.6662317145766081      0.05431736218536065   0.05533752266853648
0.6664035216487338      0.05423455282785608   0.055575612259792156
0.6665658940983287      0.05415793212432709   0.05560269280781151
0.6667355013703002      0.05407960266799775   0.05536015088473567
0.6670714162140118      0.05392961894314187   0.05452030710834985
0.6672439748275922      0.05385523880613817   0.05429965446032196
0.6674070988186418      0.053786592040569084  0.0543569142761223
0.6675774576320681      0.05371663199476884   0.054613524959465934
0.6677425491841904      0.05365052478085282   0.05487178568822168
0.6679023734750089      0.05358811313526028   0.05495670800001757
0.6680715162688172      0.05352376450552652   0.054781469388704214
0.6684065021562029      0.053401496334984026  0.05398218823636152
0.6685785962916203      0.05334136077334344   0.053725639729087074
0.6687412558045069      0.05328619567717761   0.05374249801943041
0.6689111501397702      0.0532303157145516    0.05398235562915982
0.6690757772137295      0.053177865110494105  0.05426682969284279
0.6692351370263848      0.053128685468271944  0.054411789391996465
0.6694038153420303      0.05307833913327991   0.05431094271790874
0.6697378722730897      0.052983826929672545  0.05357014094724339
0.6699032508885039      0.05293959700606919   0.0532848634987449
0.670063362242614       0.05289839325092322   0.05324364690692256
0.6702327920997142      0.05285652640629335   0.053446046759420114
0.6703969546955104      0.052817664860129315  0.05374763212778349
0.6705683521136834      0.052778952910551624  0.053967657875443095
0.6707407333123926      0.05274206474545392   0.05393626964969591
0.670903679888571       0.05270890026592385   0.0536646305458632
0.6710738612871261      0.05267603306264454   0.05326499520845512
0.6712387754243772      0.0526459092321267    0.052958839003139135
0.6713984223003242      0.05261836688829715   0.05287914050720069
0.6715673876792614      0.052590953144092185  0.05305134275430604
0.6717310857968946      0.05256609777784929   0.053358430063526696
0.6719020187369045      0.05254193453544378   0.05362675212270819
0.6720739354574508      0.05251947845253511   0.053668594800650615
0.672236417555466       0.05249995760704009   0.05345779194456013
0.672406134475858       0.052481335556027034  0.05308485094738795
0.6725705841349461      0.052465015688848086  0.05276129024102767
0.6727297665327301      0.052450835835645124  0.05263950168348372
0.6728982674335042      0.05243756025910505   0.05277195953343081
0.6730615010729744      0.05242640152999082   0.05307451607818044
0.6732319695348212      0.05241653706506276   0.05338565991801804
0.6733971707353641      0.05240872216654961   0.053504230130767834
0.6735571046746031      0.05240279347455571   0.05337582989666363
0.6738903422977571      0.0523956180616727    0.05270824598887372
0.6740615623010586      0.052394654470243655  0.052527437661711175
0.6742337660848966      0.0523955512976709    0.05262258924743393
0.6745665392298876      0.05240259048019105   0.05325701303124952
0.6747312759522675      0.052408665023718846  0.05343784523446409
0.6748907454133433      0.05241618027694931   0.05337573247435327
0.6750595333774092      0.05242588774171938   0.05309284447685424
0.6752230540801711      0.05243701253259719   0.052757899998389285
0.6753938096053098      0.052450437553349175  0.05254299128116741
0.6755655489109849      0.052465804121748276  0.05259307938013655
0.6757278535941289      0.052482045803653794  0.052865308693430584
0.6758973930996496      0.05250079656787696   0.053230363335686794
0.6760616653438665      0.052520705640458694  0.05347302502191794
0.6762331724104598      0.052543320811847254  0.05347457915104956
0.6764056632575897      0.05256795156589228   0.053227050490483395
0.6765687194821886      0.05259297500339168   0.05290471816136712
0.6767390105291641      0.05262091493184732   0.05267669720606388
0.6769040343148358      0.052649940079800894  0.05269642566202976
0.6770637908392034      0.05267977127470516   0.052944769328876144
0.6772328658665612      0.052713112503131855  0.05332336259287544
0.6773966736326149      0.05274715084913455   0.05361625814422195
0.6775677162210452      0.05278451555258971   0.053688735165144524
0.6777397425900119      0.052823973460464094  0.05349863278424547
0.6779023343364479      0.052862998710386196  0.053194128466138385
0.6780721609052605      0.05290555680315051   0.05294352177276454
0.678236720212769       0.05294854568740747   0.05292073433410976
0.6783960122589736      0.052991799731106064  0.053137947069532475
0.6785646228081683      0.05303934283658741   0.05352077995491233
0.678727966096059       0.053087125392369997  0.053860591281929944
0.6788985442063264      0.05313883587831375   0.054008651267268566
0.6790638550552898      0.05319071536848165   0.053895063515855504
0.6792238986429491      0.053242597556432146  0.05362531948429195
0.6793932607335986      0.053299274338422784  0.05335098580151736
0.6795573555629442      0.05335592800089847   0.05326933967967337
0.6797286852146665      0.05341690626996643   0.05345535999183412
0.6800638774566525      0.05354160146335818   0.05421927578666543
0.6802339910887569      0.053607617316926225  0.05443318238817862
0.6803988374595573      0.0536733436160907    0.05438357396161692
0.6807273141815402      0.053809459630648766  0.05387129993268986
0.6808909445327227      0.05387982362413918   0.05375408634122672
0.6810618097062819      0.05395511403503451   0.0538983902307329
0.6813960729919419      0.05410776670852912   0.054679036553668896
0.6815657221458833      0.05418795729066151   0.05496041887300883
0.6817301040385206      0.05426740141387598   0.054983291062889385
0.6819017207535347      0.054352172608738356  0.05477546567745146
0.682074321249085       0.054439315633765084  0.05449749226037776
0.6822374871221044      0.05452343413101179   0.05436486281450504
0.6824078878175006      0.05461308633611127   0.05448648532782044
0.6827328874243808      0.05478918597491393   0.05525870866519163
0.6829020721001591      0.05488351063970678   0.05559663641995105
0.6830659895146334      0.054976646663572576  0.055686509557877316
0.6832371417514844      0.055075973248538745  0.05553011789597949
0.6834092777688716      0.05517774364070086   0.05526253985718683
0.683571979163728       0.05527566142187561   0.055102610865602954
0.6837419153809612      0.05537972215331382   0.055180555102293666
0.6839065843368901      0.055482299819298614  0.05550623273345301
0.6840659860315152      0.05558322863565447   0.055939876429642627
0.6842347062291305      0.05569180569262277   0.05633285068483479
0.6843981591654418      0.05579870617273741   0.0564959634565615
0.6845688469241298      0.0559121364594741    0.0564022225001841
0.6847405184633539      0.056028072075234366  0.056154739712500276
0.6849027553800473      0.056139341474924984  0.05596990158227122
0.6850722271191174      0.056257341106826134  0.055997824125179264
0.6852364315968835      0.05637339542587811   0.056286176532961474
0.6854078708970264      0.05649637034346089   0.05675749460444028
0.6855802939777054      0.056621911638824504  0.05719986510001046
0.6857432824358536      0.056742297922418104  0.0574155100980597
0.6859135057163785      0.05686980530398535   0.05737144542168968
0.6860784617355994      0.05699509837620118   0.057156737703606436
0.6862381504935162      0.05711801268326953   0.056961337684339725
0.6864071577544233      0.05724983632112665   0.0569480198952819
0.6865708977540264      0.057379253657966506  0.057199484282060814
0.6869138311785221      0.057655725025210894  0.05815029134715707
0.6870763551585073      0.05778931220409885   0.058435385399134496
0.6872461139608692      0.05793060283315368   0.05846265537321534
0.6874106055019269      0.0580692205371443    0.05828463535025353
0.6875698297816808      0.05820500232277776   0.05808110653632532
0.6877383725644248      0.058350447217238574  0.05802367427327771
0.6879016480858648      0.05849302890167846   0.058227989573403055
0.6882374015121941      0.05879142537396497   0.05917724070714547
0.6883973773334029      0.058936059153987336  0.05953775574467345
0.6885666716576018      0.05909084328788688   0.0596605741316776
0.6887306987204965      0.059242502728704356  0.059541465344828716
0.688901960605768       0.059402626229101016  0.05932437688897747
0.6890742062715758      0.05956549678283436   0.0592289751593897
0.6892370173148528      0.059721129587092636  0.05939036792044359
0.6894070631805065      0.0598855519646859    0.05981603443555451
0.6895718417848561      0.06004670729193674   0.06033670443065131
0.6897313531279017      0.06020430223550423   0.06075437446099405
0.6899001829739375      0.060372806404108095  0.0609537374910786
0.6900637455586693      0.06053772106502042   0.06089117365889746
0.6902345429657779      0.06071167952947278   0.06068846726315203
0.6904063241534226      0.06088843989606525   0.06056033782011321
0.6905686707185366      0.06105714889723515   0.06067109091331268
0.6907382521060272      0.06123509296427572   0.061060647191222145
0.6909025662322137      0.061409180910972     0.06159027555151071
0.6910741151807771      0.06159268624293467   0.06209185500658994
0.6912466479098767      0.061779046697313565  0.06235621650235246
0.6914097460164454      0.06195687632122834   0.062338945245599804
0.691580078945391       0.062144313956113015  0.06215555141403749
0.6917451446130323      0.06232762937816073   0.06201574287371402
0.6919049430193698      0.06250666262761578   0.062084038437049495
0.6920740599286974      0.06269781319486571   0.06243650413257966
0.6924089940471212      0.06308146522373574   0.0635134856453078
0.6925810622980577      0.06328118191765798   0.06385425667479279
0.6927436959264635      0.06347157976652662   0.06390382447684667
0.6929135643772459      0.06367213822181934   0.0637525719773671
0.6930781655667243      0.06386812336575189   0.06359606365213089
0.6932374994948987      0.064059377503076     0.06361732022786594
0.6934061519260633      0.06426346564596312   0.06392069045135623
0.693740157088161       0.06467264325433805   0.06502282348806784
0.6939055098190942      0.06487766237404313   0.0654324617319131
0.6940655952887234      0.06507769447094634   0.06557037961655415
0.6942349992613428      0.06529102195477446   0.06547423978187321
0.694399135972658       0.06549933405967225   0.06530776131500926
0.6945705075063502      0.06571852449275321   0.06527652842876343
0.6947428628205785      0.06594071848398027   0.0655364948117394
0.6950759390263503      0.06637505734828157   0.0666448690514899
0.6952408272791204      0.06659248462886222   0.06711891853580568
0.6954004482705867      0.06680448328079659   0.06733041899160365
0.6955693877650431      0.06703048138061676   0.06728903039247672
0.6957330599981953      0.06725122085948897   0.0671299421138158
0.6959039670537244      0.06748341280718756   0.06706026425212352
0.6960758578897897      0.06771865480413318   0.06726341196002614
0.6962383141033242      0.0679425611915613    0.06772653799215504
0.6964080051392354      0.06817807111421831   0.06835313565592238
0.6965724289138425      0.06840785821732644   0.06888944422307512
0.6967315854271456      0.06863176925192656   0.06918205781707912
0.696900060443439       0.06887037853508891   0.06920968752998266
0.6970632681984281      0.06910308240178448   0.06907054401675701
0.6972337107757942      0.06934773139227803   0.06896668619724601
0.6973988860918562      0.0695864050903573    0.06909462466434942
0.6975587941466141      0.06981895109077739   0.06949306568949179
0.6978919800008063      0.07030816092804754   0.07071122837074698
0.6980631741196273      0.0705619716930593    0.07112271603253484
0.6982353520189843      0.07081891364040657   0.07122655502108462
0.6983980952958105      0.07106331497149176   0.07111760805947248
0.6985680733950135      0.07132017436169186   0.07099403713387695
0.6987327842329123      0.07157062374201555   0.07106922790417831
0.6988922278095073      0.07181451368916672   0.07141804536407503
0.6992244847073735      0.07232731356204719   0.07265824003589114
0.6993952143480313      0.07259320974451698   0.07315012912509172
0.6995669277692254      0.07286227310032074   0.07333853630198722
0.6997292065678886      0.07311805676607248   0.07327422017779378
0.6998987201889286      0.07338680092732905   0.07314026871175033
0.7000629665486645      0.07364870869177513   0.07316078540400581
0.7002344477307771      0.0739237398417894    0.07348246526887098
0.7005699430335439      0.074466499965388     0.07474565951614294
0.7007402081960388      0.07474431207041095   0.07529374438445724
0.7009052060972295      0.07501504090318972   0.0755431286002935
0.7010649367371162      0.07527854090330569   0.07552851090537589
0.7012339858799932      0.0755589246910825    0.07539807278061236
0.7013977677615659      0.07583205141492942   0.07537679963397871
0.7015687844655156      0.07611879324973372   0.07563782005993895
0.7019033508119564      0.07668438010627422   0.07688069536354164
0.7020731514962881      0.07697390990927439   0.07749622029000007
0.7022376849193157      0.07725593315699264   0.07783168628597822
0.7023969510810395      0.07753030656652145   0.0778814856419113
0.7025655357457534      0.07782220672455949   0.07776779036405831
0.7027288531491632      0.07810642772995413   0.07770845533303394
0.7028994053749498      0.07840474885707345   0.07790041792614065
0.7030646903394322      0.07869532391697931   0.07839246481604435
0.7033940442487794      0.07927862562868619   0.07973797069195039
0.7035581131936441      0.07957132306048939   0.0801804032190133
0.7037294169608854      0.0798784284049949    0.08032444851595989
0.7039017045086631      0.08018883951846886   0.08023878913640803
0.7040645574339097      0.08048366981874074   0.08015588871172717
0.7042346451815333      0.08079306457996625   0.08028869605680208
0.7043994656678528      0.08109430330661231   0.08072735080182261
0.704727890620874       0.08169872462011508   0.08210005588511582
0.7048914950875755      0.08200187860273903   0.08262238121517271
0.705062334376654       0.08231989648332871   0.08285232000260333
0.7052341574462686      0.08264124355845143   0.08281194886587828
0.7053965458933523      0.08294632235143329   0.08271437295230939
0.7055661691628128      0.08326641727166777   0.0827854027921261
0.7057305251709691      0.08357795626409087   0.08315729076046696
0.7060746906125717      0.08423472069261818   0.08459815283105908
0.7062378306011103      0.08454810715925834   0.08517570107904787
0.7064082054120255      0.08487680648245666   0.08547062104207809
0.706573312961637       0.08519671896439318   0.08547512572263552
0.7067331532499441      0.08550771114020017   0.08537800783964886
0.7069023120412417      0.08583820705563268   0.08540097755472252
0.7070662035712351      0.08615975477795337   0.0857105576896271
0.7074094400565114      0.08683743321680783   0.08713041810657102
0.707572115566887       0.08716062379730415   0.08777525763163799
0.7077420258996392      0.08749956066957089   0.08816092003365719
0.7079066689710876      0.08782932428607396   0.08823025201905854
0.7080660447812318      0.08814978475422405   0.08814485638314154
0.7082347390943662      0.08849043018624618   0.0881232167932368
0.7083981661461964      0.08882179769383082   0.08836173156454417
0.7085688280204036      0.0891691968900298    0.08893934899399063
0.7088943499849056      0.08983566915310043   0.09040062134003729
0.7090637958394947      0.0901845758246609    0.09090018118101258
0.7092279744327801      0.09052392588598371   0.09106584435864384
0.7093993878484419      0.0908795787977709    0.09100777437718366
0.7095717850446401      0.09123865724548338   0.09095463113506794
0.7097347476183075      0.09157935622222738   0.09113183294193394
0.7099049450143515      0.09193649483286989   0.09165136578031503
0.7102295380225276      0.09262131708190104   0.09314068075247334
0.7103985193989537      0.09297974507168881   0.0937249883070101
0.710562233514076       0.09332824362707771   0.0939739133660607
0.7107331824515748      0.09369344342023253   0.09396052804843857
0.71090511516961        0.09406207959379219   0.09388761178202963
0.7110676132651141      0.09441171273209843   0.09400043521309626
0.7112373461829953      0.09477817894882945   0.09444737945682963
0.7115610102348453      0.09548056216137177   0.09593459881435193
0.7117295271331083      0.09584810391687996   0.09660320928991364
0.7118927767700676      0.09620535491888058   0.09694760338990382
0.7120632612294033      0.09657969111898058   0.09699813976918259
0.7122284784274351      0.09694367810365989   0.0969223466358323
0.7123884283641629      0.0972971967661948    0.0969617967366608
0.7125576968038809      0.09767252313705113   0.09730596544305341
0.7128929339830855      0.09841951796037138   0.09879967906530851
0.7130651537644124      0.09880514797431877   0.09955639458044611
0.7132279389232086      0.0991708187868741    0.09998917934627932
0.7133979589043813      0.09955394676394443   0.1001103075724483
0.71356271162425        0.09992637529209636   0.10004784563700711
0.7137221970828147      0.10028798935398373   0.10004434736803877
0.7138910010443698      0.10067189772032888   0.10031149381345644
0.7140545377446207      0.10104496533563988   0.1008996156204456
0.7143970645704121      0.101829970738205     0.10255541488147649
0.7145593852510452      0.10220381252789981   0.103082790389932
0.7147289407540549      0.10259547469298923   0.10329169886393268
0.7148932289957608      0.10297609285481177   0.10326063810544583
0.7150647520598432      0.10337464598549881   0.10322705320036911
0.715237258904462       0.10377668814332869   0.10344601883009821
0.7154003311265498      0.10415784588721695   0.10398925924922695
0.7157356779541749      0.10494501941912701   0.10564173509141146
0.7158954504760315      0.10532163457428781   0.1062412465738715
0.7160645415008782      0.10572131462948253   0.10653718584662938
0.716228365264421       0.10610961731116039   0.1065498884270336
0.7163994238503404      0.1065161893443256    0.1064972711746536
0.7165714662167961      0.1069262487740738    0.10664523302107083
0.716734073960721       0.10731487482477767   0.10711467594582631
0.7170684918320199      0.10811731687006039   0.10876094616727942
0.7172277998757135      0.10850108185660513   0.10944073664755231
0.7173964264223971      0.10890834348948075   0.10983632277022051
0.7175597857077769      0.10930390707961687   0.10991147639729898
0.7177303798155332      0.10971805849357984   0.10985531398908803
0.7178957066619855      0.11012045928208385   0.1099263234046797
0.7180557662471339      0.11051100672246436   0.11028974538486454
0.7183892551621068      0.11132775823596892   0.1118741207828103
0.7185606008113179      0.1117489859275846    0.11268877013305988
0.7187329302410654      0.11217370795428815   0.11319249681901218
0.7188958250482822      0.11257616386355736   0.11333606025620649
0.7190659546778755      0.11299751217204283   0.1132937509966917
0.7192308170461648      0.11340680190497586   0.11331725698120727
0.7193904121531501      0.11380393485145522   0.1136042273068352
0.7195593257631256      0.1142252359103687    0.11425669765471494
0.7198938532828453      0.11506256039016442   0.11598446535221046
0.7200657182344297      0.11549425334160755   0.11658830697180139
0.7202281485634834      0.11590318551133531   0.11681678161316782
0.7203978137149136      0.11633129880559515   0.11680715562407994
0.7205622116050399      0.11674705820708178   0.11679273056881559
0.7207338443175428      0.11718214718056423   0.11702842846384959
0.7209064608105822      0.11762079723578406   0.11764226661499655
0.7212400593739756      0.11847135289839748   0.11938894197187813
0.7214052088055568      0.11889378793851632   0.12004396695457652
0.721565090975834       0.11930360277769617   0.12034981010525546
0.7217342916491012      0.1197382109334436    0.12038537461961647
0.7218982250610644      0.12016017414780004   0.1203520211313958
0.7220693932954043      0.1206016824226569    0.12051551284158366
0.7222415453102806      0.12104667243791437   0.12104794472453544
0.7225742149173481      0.12190923708514947   0.12278491078958353
0.7227388998707661      0.12233752596865974   0.1235229863381369
0.7228983175628803      0.1227529193002691    0.12392314419955285
0.7230670537579844      0.1231934481361229    0.12402314342559187
0.7232305226917846      0.12362105751335334   0.1239858662300349
0.7234012264479615      0.12406845983050843   0.12407988924523049
0.7235729139846747      0.12451932829452772   0.12451809793806817
0.723904654635416       0.12539300951000804   0.1262058041105815
0.7240688751106711      0.12582671109284868   0.12702091301271645
0.7242403304083028      0.1262803634830747    0.1275498975831927
0.7244127694864708      0.12673748173033525   0.12771311230636528
0.7245757739421081      0.12717037932647235   0.1276854486334984
0.724746013220122       0.1276233036601228    0.12773966773038525
0.7249109852368318      0.1280630007575368    0.1280912822639942
0.7254034692477257      0.12938015574870199   0.13056915167548624
0.7255744600671943      0.12983904514143538   0.13119719621454243
0.7257464346671993      0.1303013813120962    0.1314479974846518
0.7259089746446736      0.13073909101428158   0.13144758247152513
0.7260787494445244      0.13119704271984814   0.1314582299594605
0.7262432569830712      0.13164152039949079   0.13172229930858403
0.7264024972603141      0.13207244910242533   0.13231552174226388
0.726734347559476       0.1329726211085716    0.13413096764807875
0.7269048739007816      0.13343629763335738   0.13485970960854496
0.7270701329807833      0.13388644101786065   0.13520869151479156
0.727230124799481       0.1343229063340153    0.13526494146403842
0.7273994351211688      0.13478549406290666   0.1352446542022547
0.7275634781815525      0.13523437201289695   0.13540628882091318
0.727734756064313       0.13570375580393157   0.13593784942971446
0.7280698447683757      0.13662412794480575   0.13774819562500648
0.7282399066315182      0.1370922622127882    0.13855866087041213
0.7284047012333569      0.13754655240348473   0.1390041704180188
0.7285642285738916      0.13798693073282436   0.13912256856220448
0.7287330744174163      0.1384536786321467    0.13910064335055314
0.728896652999637       0.13890649303605604   0.1391965043236187
0.7290674664042345      0.13937998666191692   0.13963172884740088
0.7295712235369508      0.14078021067527682   0.1422630743037295
0.7297355536606264      0.1412381867423363    0.14281405185251905
0.7299071186066786      0.14171694981733757   0.14301846966338247
0.730079667333267       0.14219909420625088   0.14300895954715795
0.7302427814373248      0.14265545591517872   0.14306767352164987
0.7304131303637592      0.14313265435868525   0.14343630870652097
0.7309071593395327      0.14451995315818406   0.14600643159051374
0.7310710249850452      0.14498119754131186   0.1466496548855336
0.7312421254529344      0.14546337364620004   0.14694259228797268
0.73141420970136        0.14594889978727854   0.146964372428491
0.7315768593272546      0.146408332810857     0.14698291897621624
0.7317467437755258      0.14688874120413398   0.14726099072974422
0.7319113609624933      0.14735477388220117   0.1478955986090183
0.7322393793168103      0.14828489584437043   0.14975008292968786
0.7324027804841597      0.1487489654933376    0.15048620484081196
0.732573416473886       0.14923409384788025   0.15088333908258902
0.7327387852023082      0.14970473910497153   0.15095925824745018
0.7328988866694264      0.15016084898096382   0.15095046319666802
0.7330683066395347      0.15064398767558268   0.15112359578438495
0.733232459348339       0.15111258672172423   0.15163884267174996
0.7335762181912373      0.15209544418416587   0.15353743776725215
0.7337391548804237      0.15256198037780214   0.1543475358634629
0.7339093263919869      0.15304968695536236   0.1548422627584037
0.7340742306422461      0.15352273454147256   0.1549815428619307
0.7342338676312014      0.1539810757576523    0.15497209761960745
0.7344028231231465      0.15446659743600508   0.15507863214303647
0.7345665113537879      0.15493739378963647   0.1555002471778621
0.7350718134513835      0.15639321565298045   0.1582007183005599
0.7352415204847835      0.1568829756759699    0.15880127538334302
0.7354059602568798      0.15735791628583395   0.1590231281229108
0.735565132767672       0.15781799527951465   0.1590325428780754
0.7357336237814542      0.15830537902469696   0.15908285548252868
0.7358968475339325      0.158777883091873     0.1594048154407412
0.7360673061087875      0.1592716984379829    0.16012067261660207
0.7363924214745854      0.1602145677494149    0.16199486784851877
0.7365616640298225      0.16070590348421035   0.16272183026401182
0.7367256393237556      0.16118227475137253   0.16306279056964756
0.7368968494400653      0.16168000082541642   0.16312287770932934
0.7370690433369114      0.16218092648966567   0.16313939137501954
0.7372318026112266      0.16265471256553865   0.16337765921890365
0.7374017967079185      0.1631498700456578    0.16400078772998228
0.7377259831173903      0.16409501345168415   0.165855881138962
0.7378947611944644      0.16458750557824975   0.16667023604960024
0.7380582720102344      0.16506490085192516   0.1671107755299594
0.7382290176483811      0.16556369910343935   0.16723327831393583
0.7384007470670642      0.16606565241319488   0.16723364486452755
0.7385630418632163      0.16654028151622863   0.16739111268675685
0.7387325714817452      0.1670363240072716    0.1679081414659767
0.7390683310185716      0.16801949656361512   0.16977537390061154
0.7392408119787095      0.1685249253897393    0.1706680136067267
0.7394038583163165      0.16900292913670129   0.17118056120883884
0.7395741394763001      0.1695023824172837    0.17135811147015326
0.7397391533749798      0.1699866037129397    0.1713607264630396
0.7398989000123555      0.1704555598857604    0.17145789090409927
0.7400679651527213      0.17095206992925296   0.17187884374948584
0.7405748122151965      0.17244172870114546   0.17461272654413973
0.7407373940746405      0.17291991166976914   0.17522222500866952
0.7409072107564612      0.1734195401808831    0.17548558629136338
0.7410717601769778      0.17390382725583778   0.17550990698206737
0.7412310423361905      0.17437275195143126   0.17555759177454205
0.7413996429983933      0.17486925332535108   0.17587777449189845
0.7415629763992921      0.17535037732785075   0.17654989013137556
0.741898845584539       0.17634011895716958   0.17849568802269694
0.7420588792852065      0.1768118775502436    0.1792088342102592
0.7422282314888642      0.17731121642678202   0.17959410572811177
0.7423923164312178      0.17779512648490509   0.17967319125131798
0.7425636361959481      0.1783004729212949    0.1796901630058622
0.7427359397412148      0.1788088167733506    0.17992905238103293
0.7428988086639505      0.17928940715917435   0.1805108767527911
0.7432337488928713      0.18027796320458092   0.1824243408431365
0.7433933181153758      0.18074901415297756   0.18321363758116474
0.7435622058408704      0.18124762910662254   0.1836981205993312
0.743725826305061       0.18173074130012937   0.1838388509394866
0.7438966815916283      0.1822352587162708    0.1838468201768076
0.7440685206587319      0.18274271856615484   0.18400493925213068
0.7442309251033046      0.18322234473102472   0.18448525154916007
0.7445649363758993      0.18420883213026318   0.18632462055346338
0.7447365432039214      0.1847156813364827    0.18723876221772093
0.7449091338124798      0.18522543735324104   0.18781132577573575
0.7450722897985074      0.1857073219646532    0.1880041930891179
0.7452426806069116      0.18621056180717432   0.18801900042944053
0.7454078041540118      0.186698225688195     0.18812038934570882
0.745567660439808       0.18717030914241317   0.18850727130077996
0.7457368352285945      0.18766987728265433   0.1892800748286686
0.7460718851059359      0.18865911117374823   0.1912123476026114
0.7462440112363313      0.18916723303022948   0.19188103381913765
0.7464067027441958      0.18964744477352968   0.19215471231087916
0.746576629074437       0.1901489434116302    0.19219515723364164
0.7467412881433741      0.19063482410520552   0.1922483451389908
0.7469006799510074      0.19110508756113548   0.19254229457231978
0.7470693902616308      0.19160275758106127   0.19321865767160917
0.7474035111826461      0.19258807488764232   0.19514512697558434
0.7475751728348785      0.1930941388052071    0.19591025970750053
0.74773739986458        0.1935722783579075    0.19627988499163276
0.7479068617166581      0.1940716196257724    0.1963687909309949
0.7480710563074322      0.19455531491534586   0.19639055756658402
0.748242485720583       0.19506018438867448   0.19662256595184363
0.7484148989142703      0.19556780102169263   0.19723505122405555
0.7487480908789595      0.1965483261599631    0.1991285274053119
0.7489130370111885      0.1970334986452324    0.1999300993610415
0.7490727158821135      0.19750302159216884   0.20038143011780646
0.7492417132560286      0.19799977060655188   0.20053058177621214
0.7494054433686397      0.198480859706727     0.200543814663731
0.7495764083036275      0.1989830136353428    0.20069894795897053
0.7497483570191517      0.19948785069645367   0.20120645894139566
0.7500806200275149      0.20046275264052135   0.20302565682554913
0.7502451016815808      0.2009450501686942    0.2038937646317215
0.7504043160743428      0.20141169755227867   0.20444108978959952
0.7505728489700949      0.20190543022756247   0.204670066610246
0.750736114604543       0.20238350325453924   0.20469522402151794
0.7509066150613678      0.2028825140393996    0.20478435817267135
0.7510718482568886      0.20336586125214254   0.20515892823162926
0.7512318141911054      0.20383356196749186   0.2058573632596459
0.7515651158042151      0.20480727333380613   0.20775935631858028
0.7517363678024948      0.20530714722858315   0.20845956080765396
0.7519086035813107      0.2058095928847471    0.20878225593548738
0.7520714047375958      0.2062842027039177    0.20883648493302856
0.7522414407162575      0.20677957321278304   0.20888391887011096
0.7524062094336152      0.20725929912755736   0.20916718870620887
0.7525657108896691      0.2077234027551177    0.2097739056324601
0.7528980835464527      0.2086895812630457    0.21164884808476195
0.7530688710665693      0.20918554323041114   0.21243390020660738
0.7532406423672222      0.20968400871139273   0.2128539347249694
0.7534029790453443      0.21015476365114716   0.21295747771661017
0.7535725505458429      0.21064614742761864   0.21298039535911917
0.7537368547850376      0.21112191911505918   0.2131756545406885
0.753908393846609       0.21161826759052849   0.21373226443056098
0.7542440049082935      0.21258823027376686   0.21557813862391195
0.754414327950247       0.21307990274024294   0.21641332216978953
0.7545793837308965      0.21355598584109242   0.21689240628418877
0.7547391722502421      0.21401650981263287   0.21705012676439628
0.7549082792725778      0.21450349153966142   0.2170701320337331
0.7550721190336094      0.21497490745864184   0.21719876649303205
0.7552431936170177      0.21546671585131083   0.2176543268849515
0.7557477342861668      0.21691458799065855   0.22030214670952922
0.7559123255886533      0.21738605439314623   0.22087284405731872
0.7560716496298359      0.21784202062819571   0.2211041159487367
0.7562402921740085      0.21832420711841344   0.22114105949322047
0.756403667456877       0.21879088759304688   0.22121403282342833
0.7565742775621224      0.21927775938534355   0.22156440375447145
0.7567396204060637      0.21974913053009368   0.22225137007241913
0.7570690900743285      0.22068700319405288   0.22408223220751128
0.757233216898652       0.22115349918782623   0.22476165584355365
0.7574045785453521      0.22164004620768712   0.22511206743139925
0.7575769239725886      0.22212885147274758   0.2251854308704255
0.7577398347772941      0.22259039858825388   0.22522480130873324
0.7579099804043764      0.2230719177289129    0.2254875116415807
0.7580748587701547      0.2235380120034201    0.22608187139882774
0.7584033994820935      0.2244651230539304    0.22787522131830285
0.7585670618282538      0.2249261258234737    0.22862576190636616
0.758737958996791       0.22540694558584876   0.22906732568745514
0.7589098399458644      0.22588994763992748   0.22919373841956847
0.7590722862724069      0.22634589176096168   0.22921534172943608
0.7592419674213261      0.2268215696871426    0.229394877535029
0.7594063813089413      0.22728191770404868   0.22988625926334036
0.7597506625094614      0.22824405880549622   0.23169467517362663
0.7599138603774589      0.22869926329312296   0.23248449096518517
0.7600842930678329      0.22917403905801356   0.23298910535303113
0.760249458496903       0.2296335424488004    0.23315978104583926
0.7604093566646691      0.2300778243266644    0.2331815424290954
0.7605785733354253      0.23054738384991508   0.23329931885015998
0.7607425227448776      0.23100171910687453   0.23369841432278274
0.7612486083789058      0.2324003425213042    0.23624082874143645
0.7614185765911169      0.232868747295152     0.23683085125287665
0.7615832775420239      0.23332199259087516   0.23707291366289385
0.7617427112316271      0.2337601336707089    0.23711246296122931
0.7619114634242202      0.23422322435152126   0.23717968367758566
0.7620749483555094      0.2346712088273473    0.23748411853678852
0.7622456681091753      0.23513832745337876   0.23814367802710135
0.7625796405550487      0.23605006824917849   0.23992017737560611
0.7627491442890967      0.23651175327408247   0.24059532868548156
0.7629133807618407      0.2369584040736317    0.24092274714097975
0.7630848520569613      0.23742400203689068   0.24100134716903132
0.7632573071326183      0.23789151426749042   0.24104411830423372
0.7634203275857445      0.23833274596379878   0.24128501244374162
0.7635905828612473      0.2387928228071752    0.2418690404766197
0.7639152916283409      0.2396681662806915    0.24356301550748702
0.7640843308842259      0.24012275152179222   0.24430327685226202
0.7642481028788068      0.24056244091852572   0.24471042302975282
0.7644191096957644      0.24102078009293704   0.24483926250692198
0.7645911002932584      0.24148088992595476   0.24486476460924297
0.7647536562682214      0.24191492676665075   0.2450305575180095
0.7649234470655613      0.24236750056650594   0.24551381019353155
0.765247226876329       0.2432282979519491    0.24713022637363
0.7654158016540509      0.24367529967617618   0.24792728818491078
0.7655791091704688      0.2441075640053253    0.24842021557946795
0.7657496515092634      0.2445581635416291    0.24861724478954242
0.765914926586754       0.24499404639941216   0.2486443755833643
0.7660749344029405      0.24541528284099337   0.24873550291828686
0.7662442607221173      0.24586023692567083   0.2490965038779902
0.7667518913210252      0.24718911037568975   0.25149880940448494
0.7669147343592801      0.24761377058209139   0.2520635625271263
0.7670848122199116      0.2480564432885505    0.2523306332246605
0.7672496228192393      0.24848456840755437   0.2523785679292928
0.7674091661572628      0.24889822007617074   0.2524296299979415
0.7675780279982766      0.2493351793042011    0.25270146042340247
0.7677416225779863      0.2497576670671161    0.25327138358417867
0.7680842651626953      0.2506398449956305    0.25498460817826407
0.7682466437227872      0.25105661956533387   0.2556193714791188
0.7684162571052557      0.2514910703590166    0.2559680726280199
0.7685806032264204      0.2519111535410381    0.2560552490123124
0.7687521841699615      0.25234880401764226   0.25608734181246884
0.7689247488940391      0.25278800433440474   0.25630272462212306
0.7690878789955857      0.2532023021379386    0.25680342178177695
0.7694233415821283      0.25405152833126804   0.25843398366505216
0.7695831719834438      0.2544548330926266    0.25910980645928205
0.7697523208877493      0.25488072548707197   0.25953245043241097
0.7699162025307508      0.25529244223603403   0.25966738533270967
0.770087318996129       0.25572136988476035   0.2596909223907646
0.7702594192420434      0.2561517633469292    0.2598368181652782
0.7704220848654271      0.25655763501747486   0.26024676548081527
0.7709159844187959      0.2577842599773802    0.26250797671388393
0.7710846688449384      0.25820111755152625   0.2630087324919732
0.771248086009777       0.2586040095366038    0.26320596369765015
0.771418737996992       0.25902373678904206   0.2632389384497769
0.7715841227229032      0.2594295279450798    0.2633191105779917
0.7717442401875103      0.2598214709923971    0.26361882664900776
0.7719136761551076      0.26023522952453115   0.2642330470683204
0.7722492483900709      0.26105165104301814   0.2658277160558206
0.7724216356992772      0.26146947669567394   0.2664093010748794
0.7725845883859527      0.2618634433767424    0.2666682730159643
0.7727547758950049      0.2622738678364272    0.2667250522985265
0.7729196961427529      0.2626705776152024    0.2667700963949839
0.773079349129197       0.263053663930372     0.2669933436375376
0.7732483206186312      0.2634580839188983    0.2675193222701468
0.7735829638972684      0.26425589383455206   0.2690709913934334
0.7737548867283117      0.26466413984537035   0.26971414312792313
0.7739173749368242      0.26504896045776566   0.2700436243945023
0.7740870979677131      0.2654498500583043    0.2701406936652914
0.7742515537372983      0.26583725547242226   0.2701650214602674
0.7744107422455793      0.266211271117817     0.27031669636738925
0.7745792492568505      0.2666061239679589    0.27074667904707345
0.7749129635791616      0.26738487194746235   0.2722209965246139
0.7750781708902015      0.2677688003730363    0.2728964037562697
0.7752381109399374      0.2681394758030155    0.2733090939999756
0.7754073694926635      0.26853065851432595   0.27347470159626786
0.7755713607840855      0.2689085950374663    0.27349831126370433
0.7757425868978841      0.2693020735729928    0.27359457555011596
0.7759147967922192      0.26969664202576354   0.2739441327686491
0.776412324991081       0.27082993922401644   0.27602333863185363
0.7765718005626541      0.2711910983854961    0.2764984175318649
0.776740594637217       0.27157224109000666   0.27672573780976084
0.776904121450476       0.27194038751031074   0.27676681393174707
0.7770748830861118      0.27232363252603203   0.2768203541106593
0.7772466285022838      0.2727076856363911    0.27708247539573827
0.7774089392959249      0.27306952856635724   0.2775954560382342
0.7777427632666565      0.27381031606344086   0.27904908507201265
0.777914276443747       0.2741891263552811    0.2796157634485042
0.7780867734013739      0.27456887590502793   0.27989806630165787
0.77824983573647        0.2749267141608875    0.2799599104066789
0.7784201328939425      0.2752992414218443    0.2799929265769682
0.7785851627901113      0.27565908505433945   0.2801839536432861
0.7787449254249759      0.2760063511231937    0.28061363539785705
0.7790778204393813      0.27672647723431737   0.2820084293749909
0.7792488691383088      0.2770946638972562    0.28262186380207266
0.7794209016177727      0.27746371002245657   0.2829696561600902
0.7795834994747056      0.27781135338650365   0.28306997070819884
0.7797533321540153      0.27817325416528654   0.28309005680744265
0.7799178975720209      0.2785227467592589    0.28321730018784474
0.7800771957287225      0.2788599397040069    0.28356368345642374
0.7804091617868019      0.2795590921732634    0.2848703271345778
0.7805797460075663      0.2799164939116815    0.2855215985149615
0.7807450629670267      0.28026164688266714   0.2859280874608033
0.7809051126651834      0.28059466179803944   0.28608699024630235
0.7810744808663298      0.28094583965091574   0.2861135001037953
0.7812385818061724      0.2812848899656491    0.2861803478855396
0.7814099175683917      0.2816376176749409    0.2864598269201173
0.7819152417739734      0.2826703279458674    0.2883419084018161
0.7820800942552708      0.28300476323253143   0.288801505871134
0.7822396794752643      0.28332735107029794   0.2890126683730823
0.7824085831982479      0.28366752686510743   0.28905957251808784
0.7825722196599274      0.2839958661614633    0.2890948713939782
0.7827430909439835      0.2843374284409791    0.2892990216628154
0.7829149460085761      0.2846796200721505    0.28976202175017396
0.7832470217150762      0.2853370232228408    0.29105728304084044
0.7834114097182105      0.2856604643859903    0.29156638799504303
0.7835830325437216      0.2859967071132921    0.2918486940188655
0.783755639149769       0.2863335102348505    0.29192026203378296
0.7839188111332855      0.28665064054971556   0.2919416226486521
0.7840892179391787      0.2869805185950932    0.29209629956983135
0.7842543574837679      0.2872989171700529    0.29247617298297024
0.7845834205533284      0.287929594090845     0.2936999772134599
0.7847473440782997      0.288241885895476     0.29424181302071506
0.7849185024256478      0.2885666219501968    0.2945793017880034
0.7850906445535322      0.2888918420486165    0.29468924857835405
0.7852533520588856      0.2891979605672329    0.2947039331016296
0.7854232943866156      0.28951636274598896   0.2948038010829246
0.7855879694530419      0.2898235991856654    0.29510659001984096
0.7860795626130845      0.2907331513826913    0.2968079018575765
0.7862502564822695      0.2910462924456587    0.29720173166276326
0.7864219341319909      0.29135984065062875   0.2973610237088589
0.7865841771591814      0.29165486682741953   0.2973829275510104
0.7867536550087486      0.2919617067094944    0.29743820080307765
0.7869178655970117      0.29225769934312684   0.29766400355875744
0.7870893110076513      0.292565353390209     0.29812863817317087
0.7874247347674728      0.2931631755004683    0.299316559279716
0.7875949641584947      0.2934644989052214    0.29974248717358104
0.7877599262882127      0.2937551624841271    0.29993522868508293
0.7879196211566266      0.29403528975336596   0.299973637122546
0.7880886345280307      0.2943304151276932    0.3000018557432358
0.7882523806381307      0.29461501845733373   0.300164575290938
0.7884233615706076      0.2949108024552871    0.3005543985090664
0.7887578563741029      0.2954853238170802    0.3016932538292047
0.7889276212869617      0.29577481232723907   0.3021591208439768
0.7890921189385167      0.29605396898602293   0.3024028880559537
0.7892513493287676      0.29632291897760993   0.30247007171385665
0.7894198982220086      0.2966062468844128    0.3024837358323782
0.7895831798539457      0.29687938268427005   0.3025892300185729
0.7897536963082594      0.29716297261916963   0.30290013570708546
0.7900789274329749      0.29769986359135325   0.30393611553186467
0.7902482278676708      0.2979772663628517    0.3044413764884221
0.7904122610410627      0.2982446793271741    0.3047491647327286
0.7905835290368313      0.29852245744146383   0.3048674554987955
0.790755780813136       0.29880035658486775   0.30488118131875175
0.79091859796691        0.29906167341406964   0.30494521369231375
0.7910886499430607      0.29933318793123737   0.30518697146276
0.7915817880679831      0.30011237429178617   0.3066498522654342
0.791745356763212       0.3003681283429476    0.30699957145486134
0.7919161602808173      0.3006337596032658    0.3071618251986021
0.7920879475789593      0.3008994402412471    0.3071873996495929
0.7922503002545702      0.30114916321823665   0.30721934068442314
0.7924198877525578      0.30140859488762517   0.3073939207689934
0.7925842079892416      0.30165858356321545   0.30775220631262973
0.7929283018878985      0.30217764720200013   0.3087947001272074
0.7930914061049643      0.3024215937528202    0.30916534141739954
0.7932617451444067      0.30267492023115944   0.3093581146451805
0.7934268169225454      0.3029190071441329    0.30939831873894624
0.7935866214393799      0.3031539854561885    0.30941233051552997
0.7937557444592045      0.30340124971595894   0.30953339685146886
0.7939196002177251      0.30363942256668874   0.3098273898430941
0.7942627651600559      0.3041337894560933    0.3108097941935128
0.7944254048989587      0.30436598683595273   0.3112064632744805
0.7945952794602382      0.30460706689861267   0.3114421079451675
0.7947598867602137      0.3048392604510415    0.3115095689675009
0.7949192267988852      0.30506269933460395   0.31151595851881825
0.7950878853405466      0.3052977839962098    0.31158997309424946
0.7952512766209043      0.30552413133961115   0.31181812188804797
0.7957473531451951      0.30620291929687493   0.3131160746970742
0.7959167632283114      0.3064317260399055    0.31340404637859837
0.7960809060501239      0.30665185574014436   0.3135146001377401
0.7962522836943129      0.3068802001050355    0.3135273835281698
0.7964246451190384      0.307108321508787     0.31356866725466853
0.796587571921233       0.3073225413278484    0.31374063442010036
0.796757733545804       0.30754480505088594   0.3140897999562364
0.7970822550110346      0.30796453025042975   0.3149496355158705
0.797251200615988       0.30818087609660333   0.31526889267034824
0.7974148789596374      0.3083890636786874    0.3154147827769499
0.7975857921256635      0.3086049683971359    0.315442132524808
0.7977576890722258      0.3088205848766263    0.3154592997444913
0.7979201513962573      0.30902295523834555   0.31557818299782725
0.7980898485426655      0.30923287184870357   0.3158639709178466
0.79841344105157        0.30962900477385097   0.31667469724359837
0.7985819221783603      0.30983309623221233   0.31702030343687976
0.7987451360438467      0.31002939684264297   0.31720451570297753
0.7989155847317098      0.3102329167090671    0.3172562415456658
0.7990807661582687      0.3104287018146319    0.31725979560134226
0.7992406803235237      0.31061688767104667   0.31732580085655604
0.7994099129917689      0.31081458581995797   0.3175383079443307
0.799745078628028       0.31120171292456705   0.31830406047925597
0.7999172626378821      0.3113983085131709    0.3186726964188424
0.8000800120252054      0.3115827067724278    0.3188878970117733
0.8002499962349054      0.3117738228831203    0.3189666567129128
0.8004147131833013      0.31195757430149207   0.31896990229652766
0.8005741628703933      0.31213409664821135   0.31900413303458264
0.8007429310604754      0.31231948450389685   0.31916044162716867
0.8012488872720995      0.31286631791199954   0.32021464599523514
0.8014111721812598      0.31303886969534306   0.32045899046760656
0.8015806919127968      0.31321763758047383   0.3205701579790112
0.8017449443830298      0.3133894111778103    0.3205827848301907
0.8019164316756393      0.3135672382565279    0.3205975953402742
0.8020889027487853      0.3137445262402572    0.32071564820307863
0.8022519391994005      0.3139105047604481    0.3209608438482724
0.80258721448408        0.3142473501188296    0.32166828683020166
0.8027469512344638      0.31440575645044805   0.32192717201129123
0.8029160064878377      0.3145719429864863    0.32206718270720897
0.8030797944799077      0.3147315193060023    0.3220950181066496
0.8032508172943542      0.3148966402620667    0.32209631654525156
0.8034228238893373      0.3150611614222392    0.322170978201917
0.8035853958617893      0.315215230275439     0.3223635110402809
0.8039197421901427      0.31552772758682107   0.3230125861233543
0.8040790144623635      0.31567452832551446   0.32328477928962784
0.8042476052375744      0.31582846761946554   0.323454319694822
0.8044109287514813      0.315976175352821     0.3235039207454246
0.8045814870877648      0.3161289326133812    0.32350117727565375
0.8047467781627444      0.31627551667929465   0.3235362337970232
0.8049068019764201      0.3164160647964255    0.32366959104515297
0.8054115292261856      0.31685057076885087   0.324539010732236
0.8055838228844604      0.31699583756631855   0.3247364315984262
0.8057466819202044      0.31713172011435287   0.3248079186356906
0.8059167757783248      0.3172721560833569    0.3248094793322515
0.8060816023751414      0.31740679791331305   0.32482150779515745
0.806241161710654       0.31753578224762036   0.32491225239361016
0.8064100395491565      0.3176708478786144    0.32512019314256857
0.8067444955259306      0.31793393558452065   0.32569297313747686
0.8069163247060422      0.3180668232863215    0.3259090737508674
0.8070787192636231      0.31819099520166366   0.32600506357186937
0.8072483486435806      0.3183192265819627    0.3260183963692047
0.807412710762234       0.3184420412195577    0.32601546728919995
0.8075843077032643      0.3185687552170434    0.32607604720926275
0.8077568884248307      0.3186946436395789    0.3262465820457403
0.8080904154452788      0.3189335243380137    0.326769177047333
0.808255529105387       0.3190496328823705    0.3269832731160108
0.8084153755041914      0.31916064862897664   0.3270971147062476
0.8085845404059859      0.31927651907996907   0.3271254028843828
0.8087484380464763      0.31938735798598783   0.32711671976108114
0.8089195705093435      0.31950159541110734   0.3271463328543075
0.8090916867527469      0.31961495113312316   0.32727067713329244
0.8094242848168689      0.31982963344769055   0.32773168055671686
0.8095889339988142      0.3199337821745712    0.3279492910698532
0.8097483159194554      0.3200332584312874    0.3280818245299244
0.8099170163430869      0.32013711502149933   0.32812924779022407
0.8100804495054141      0.3202363219728449    0.32812208842040463
0.8102511174901184      0.3203384437940007    0.32812768868108183
0.8104165182135185      0.3204359753348234    0.3282036259193808
0.8105766516756145      0.3205290529156041    0.32836043180107266
0.810910288344483       0.3207187235395567    0.32880120205924446
0.811081707870642       0.32081394040474015   0.32896179614966103
0.8112541111773373      0.32090817565376256   0.32902946016310597
0.8114170798615017      0.32099584613319043   0.32902905034086766
0.8115872833680426      0.32108594912439326   0.32902034097253063
0.8117522196132798      0.3211718421103818    0.32906261010441984
0.811911888597213       0.32125366018090934   0.32917917202979946
0.8122445963097553      0.3214199418435469    0.3295713546290874
0.8124155513577511      0.32150317416106744   0.32973912815059636
0.8125874901862834      0.3215853751190461    0.32982524015996345
0.8127499943922849      0.32166167438073173   0.3298365434552523
0.8129197334206628      0.32173992814863944   0.3298207308117654
0.8130842051877369      0.3218143487651856    0.3298346048066304
0.8132434096935071      0.3218850700967118    0.32991174757802494
0.8135751884497233      0.32202829620363216   0.330244218163983
0.8137456790195561      0.3220997133056715    0.33041299781875555
0.813910902328085       0.32216751194845633   0.3305137247792343
0.81407085837531        0.3222318260984473    0.330543009092449
0.8142401329255249      0.32229847046634413   0.33052883841061803
0.814404140214436       0.3223616528493743    0.33051978376069036
0.8145753823257236      0.32242616552203907   0.33056290556025697
0.8147476082175475      0.32248945588609346   0.3306780982667683
0.8150804255785105      0.3226073538879363    0.33099312005850307
0.8152451844088764      0.32266364646486206   0.3311031410466946
0.8154046759779383      0.32271683456714173   0.3311469931584521
0.8155734860499901      0.32277173372801626   0.33113979114226905
0.8157370288607382      0.32282355201769675   0.3311194709104819
0.8159078064938627      0.3228762276746364    0.33113179270532483
0.8160795679075237      0.32292773000837177   0.33120718050257963
0.8164114563121607      0.32302305788947927   0.3314760085629682
0.8165757506643635      0.32306820896492916   0.33159067250632124
0.8167472798389428      0.3231139100448721    0.33165108106348573
0.8169197927940586      0.32315839331161      0.33165161542096433
0.8170828711266436      0.32319908101381556   0.33162659605328215
0.8172531842816051      0.32324016174123615   0.3316179591783533
0.8174182301752628      0.3232785973259597    0.33165824918340747
0.8175780088076166      0.32331451899831115   0.33174746519954657
0.817910935817          0.32338530665152754   0.33198333285884163
0.8180820005134164      0.32341954786659255   0.33205429171185674
0.8182540489903691      0.32345252956051795   0.3320670311320598
0.8184166628447911      0.32348236177556744   0.3320425075981236
0.8185865115215896      0.32351213180822835   0.33201625947564245
0.8187510929370843      0.32353962610233467   0.33202493750740353
0.8189104070912749      0.3235649741435658    0.3320800402331339
0.8192424051443323      0.32361380153104113   0.33227969965282983
0.8194130053625857      0.32363679514645016   0.3323577126340726
0.8195783383195352      0.32365772306794655   0.3323841890740041
0.8197384040151805      0.32367671453785546   0.33236742201511454
0.8199077882138162      0.32369545307500946   0.3323307793047459
0.8200719051511477      0.32371227818731707   0.332310875098822
0.8202432569108558      0.3237284493307049    0.3323328805984764
0.8204155924511003      0.3237432772137823    0.33240147006722776
0.820578493368814       0.3237559710462315    0.3324878302161951
0.8207486291089043      0.3237678585696594    0.33256599343249216
0.8209134975876908      0.3237780443701973    0.33260287349901313
0.8210730988051733      0.3237865352723264    0.3325950870788381
0.8212420185256457      0.3237941799309186    0.33255587632499206
0.8214056709848143      0.32380027830706226   0.33251717021340904
0.8215765582663594      0.3238052749463297    0.33250684944065756
0.8217484293284409      0.32380888979634054   0.3325416809763755
0.8220805370299189      0.3238118768374397    0.3326792210792378
0.8222449410305421      0.32381141040563105   0.3327244063839571
0.8224165798535422      0.3238095523551554    0.332726597749809
0.8225892024570784      0.3238062735728201    0.33268753467459633
0.822752390438084       0.323801876085595     0.3326366479179229
0.8229228132414661      0.32379593954102187   0.3326008068212374
0.8230879687835444      0.32378887861175026   0.33260386388015895
0.8232478570643185      0.3237808186778086    0.3326430187003678
0.8234170638480829      0.3237709796701072    0.33270355620430797
0.8235810033705432      0.3237601650650973    0.33275167226280944
0.82375217771538        0.32374752933049383   0.332764726827989
0.8239243358407533      0.32373343880764993   0.332732253702218
0.8240870593435958      0.32371884857594013   0.33267434326103923
0.8242570176688149      0.3237022923781739    0.33261529440851517
0.8244217087327301      0.3236849676369869    0.3325866843489946
0.8245811325353413      0.32366699757916545   0.33259703126473744
0.8249133498852397      0.32362576836585943   0.3326856916761743
0.8250840597519137      0.323602599584708     0.33270884464541056
0.8252557533991238      0.32357794273543483   0.33268713112296955
0.8255875062708593      0.32352646120175604   0.3325526480842992
0.8257517328566115      0.32349910907610896   0.3324943984946878
0.8259231942647403      0.32346923572153985   0.3324732151723748
0.8260956394534054      0.32343783732733533   0.33249429826945376
0.8262586500195396      0.3234069110090945    0.3325331150834273
0.8264288954080503      0.32337332204991415   0.3325607356184835
0.8265938735352573      0.32333951715195147   0.332549131123509
0.8267535844011602      0.32330561684905784   0.3324954533468425
0.8270863758776423      0.3232312748065583    0.33232629308353484
0.827257372807608       0.32319109353918846   0.33227217004731624
0.8274293535181101      0.32314929621308675   0.33226508246815767
0.8275918996060811      0.3231085723117843    0.33229103670337795
0.8277616805164292      0.32306477325677346   0.33232136194626744
0.8279261941654731      0.3230211051348552    0.33232147766949266
0.828085440553213       0.3229776865881397    0.3322769343980173
0.828417303073369       0.32288358424711927   0.3320852432641117
0.8285878355251717      0.322833331143537     0.3319988052371324
0.8287531007156703      0.32278340501626834   0.331959518819408
0.8289130986448651      0.32273392377101356   0.33196426239306565
0.8290824150770499      0.3226803352274497    0.3319914327314815
0.8292464642479307      0.3226272145579684    0.3320038935880419
0.8294177482411883      0.3225704946134436    0.3319707012842919
0.8295900160149821      0.3225121571546995    0.3318818290336629
0.8299229171398848      0.3223957630665035    0.3316520593994917
0.8300877178522204      0.3223363637581567    0.3315807178440367
0.8302472513032519      0.3222777440304587    0.3315610235141243
0.8304161032572739      0.32221450396630175   0.33157782812955544
0.8305796879499917      0.3221520660580822    0.33159588441113597
0.830750507465086       0.32208564001983375   0.331576587974137
0.8309223107607169      0.3220175703301893    0.33149659757872213
0.8312542829292935      0.32188246901974543   0.33123665340006714
0.831418619163466       0.3218138534503027    0.3311322217503236
0.8315901902200153      0.3217409940706277    0.33108116658202813
0.831762745057101       0.32166645980481495   0.3310837490719626
0.8319258652716557      0.32159484437915503   0.33110223379303194
0.8320962203085872      0.32151885573176264   0.3310924758219556
0.8322613080842146      0.32144405265878506   0.33102447057348033
0.8325902676158516      0.32129159243793287   0.3307472696543836
0.8327541393718614      0.3212139573985554    0.33061279274968425
0.8329252459502476      0.32113170157488663   0.33052852798368054
0.8330973363091703      0.3210477462760217    0.33050913564584006
0.833259992045562       0.3209672657312372    0.33052409773634056
0.8334298826043303      0.32088203792046927   0.3305257776576983
0.8335945059017947      0.32079826486568164   0.33047286919375046
0.8337538619379552      0.3207161019576505    0.33035810432688545
0.8340859437549524      0.3205415388946251    0.33002866488786575
0.8342565858551756      0.3204500872008871    0.3299082039213415
0.8344219606940948      0.32036032791694996   0.329860348785586
0.83458206827171        0.32027237021299654   0.32986368631789004
0.8347514943523154      0.3201781635255468    0.3298747823719479
0.8349156531716168      0.3200857805445363    0.3298429119578559
0.8350870468132947      0.3199881686713988    0.32973433627643794
0.8354223670351928      0.31979379075526626   0.32937413139612887
0.835592544657253       0.3196934240330868    0.32921945496493255
0.8357574550180091      0.31959506403662236   0.3291406399773397
0.8359170981174614      0.31949881731319796   0.3291273941970772
0.8360860597199036      0.31939585363217576   0.3291400740667488
0.836249754061042       0.3192950249026642    0.32912337012484844
0.836420683224557       0.31918861398786746   0.32903152055037693
0.8367550744901289      0.318977127141735     0.3286576917856254
0.836924787634026       0.31886812017435445   0.3284693878210561
0.8370892335166191      0.31876142691570925   0.32835482604434857
0.8372484121379082      0.3186571513097405    0.3283177073929454
0.8374169092621877      0.3185457027323283    0.32832555715436373
0.837580139125163       0.3184366930413301    0.3283226035398801
0.8377506038105149      0.31832175705417204   0.32825217447710187
0.837915801234563       0.3182093084693503    0.3280981904791181
0.8382449800630412      0.31798212812339866   0.32767205325341303
0.8384089614674712      0.31786741752830266   0.32751444703927557
0.8385801776942781      0.31774655719198047   0.3274391649373857
0.8387523777016213      0.3176238838764651    0.32743504522697453
0.8389151430864334      0.31750690348158106   0.3274405521885352
0.8390851432936225      0.31738365967790844   0.3273901989524085
0.8392498762395074      0.3172632006242516    0.32725167172381675
0.8395781261116595      0.31702014963072866   0.32680170623896954
0.8397416430379265      0.3168975746805366    0.32660792825756985
0.8399123947865703      0.31676849257018286   0.32649575189586544
0.8400841303157505      0.3166375802920723    0.3264716907339102
0.8402464312223997      0.3165128616258642    0.3264807269870427
0.8404159669514255      0.3163815507319567    0.32645130669000005
0.8405802354191475      0.31625331607711227   0.3263348006021283
0.840924225779881       0.31598159736329434   0.32585132515726506
0.841087278227985       0.31585130176922216   0.32562854127053575
0.8412575654984658      0.3157141992503755    0.32548417753877923
0.8414225855076425      0.31558034098955184   0.3254395963929882
0.8415823382555152      0.31544982385711884   0.32544581824181573
0.8417514095063782      0.31531069858957766   0.32543266227920226
0.841915213495937       0.3151749348739621    0.32533838731120707
0.8420862523078725      0.31503215637551607   0.3251335781876482
0.8424208628702856      0.31474983691367897   0.32460408800574103
0.8425906856626033      0.3146050429613117    0.3244186856000256
0.842755241193617       0.31446377420118876   0.3243427922325382
0.8429145294633268      0.31432612483064704   0.3243390261183927
0.8430831362360265      0.3141794587399383    0.3243390228926434
0.8432464757474224      0.3140364320050349    0.32427032810848594
0.8434170500811949      0.31388608338497515   0.3240868727713668
0.8437423969648281      0.31359653111501073   0.32354413967254636
0.8439117552789828      0.31344436676642945   0.32331089850885303
0.8440758463318334      0.3132959994074888    0.32319034533359786
0.8442471722070608      0.31314011128455477   0.32316435142925404
0.8444194818628245      0.3129823225147971    0.3231709657228952
0.8445823568960572      0.31283224937895093   0.32312452859215973
0.8447524667516666      0.31267455473360434   0.3229644978258505
0.8450768846789738      0.3123711202185215    0.32241111032326897
0.8452457785149654      0.3122117578740474    0.3221401838106829
0.845409405089653       0.31205646021695327   0.32197917401528686
0.8455802664867174      0.31189334836135113   0.3219259461179146
0.8457521116643179      0.3117283242448706    0.3219322946908042
0.8459145222193877      0.31157146640723377   0.321907630036304
0.8460841675968341      0.3114066968221893    0.32177715278927077
0.846592755370551       0.31090709994055554   0.320898609185499
0.8467559174670757      0.31074504502258565   0.3207044990330213
0.846926314385977       0.3105748882448978    0.3206258790129435
0.8470914440435743      0.31040910142282213   0.32062631883309833
0.8472513064398677      0.3102477713246785    0.3206173239764193
0.8474204873391511      0.3100761493850215    0.3205146699254025
0.8475844009771305      0.30990900305504476   0.32029037140196226
0.847927681678379       0.30955619884069785   0.3196232375027662
0.8480903792967405      0.3093876920573128    0.31938708367764057
0.8482603117374787      0.3092108069729363    0.3192702543575725
0.848424976916913       0.30903854500107975   0.31925607194401245
0.8485843748350432      0.30887099026242637   0.3192587446496174
0.8487530912561637      0.3086927826903133    0.31918643422573567
0.8489165404159801      0.3085193007836863    0.31899004501641776
0.8492588921609026      0.30815327592356745   0.31830672750585803
0.849421125301101       0.30797857382348026   0.31802753235974585
0.8495905932636763      0.30779522628976735   0.3178646800300258
0.8497547939649475      0.307616747618022     0.3178257829950491
0.8499262294885953      0.3074295368924265    0.31783299223078776
0.8500986487927795      0.3072403608189557    0.31778072113246386
0.8502616334744328      0.30706071787386935   0.31760572223402955
0.850596805221189       0.306688802174914     0.31692891078200675
0.850756490202611       0.30651044036152664   0.31661743989843827
0.8509254936870232      0.306320851190936     0.316409225780589
0.8510892299101314      0.3061363707286173    0.3163400777952945
0.8512602009556162      0.305942901711776     0.31634441760778376
0.8514321557816373      0.30574746026200084   0.3163173043784675
0.8515946759851276      0.3055619531767314    0.3161759609943839
0.8517644310109946      0.30536737263915403   0.3158844935107119
0.8520881392788167      0.30499402725623936   0.3151706756002703
0.8522566782850659      0.30479845840930214   0.31491282167795437
0.852419950030011       0.30460825478496734   0.3148039534552252
0.8525904565973328      0.30440881633222394   0.3147953607323733
0.8527556959033507      0.3042147556634278    0.3147905975348978
0.8529156679480644      0.3040261493609582    0.31469384680851903
0.8530849584957684      0.3038257757563998    0.31444283499735015
0.853420239890945       0.30342657673900164   0.31368203909964537
0.8535924817802578      0.30322028731302597   0.31337025922517037
0.85375528904704        0.3030245455095709    0.3132198316673585
0.8539253311361988      0.30281932827921254   0.3131905340731233
0.8540901059640537      0.3026197135024987    0.31319626970477693
0.8542496135306045      0.30242577522011044   0.3131309083345981
0.8544184396001453      0.30221975505850895   0.31291527746664
0.8547527920389959      0.30180947160196314   0.3121514194443683
0.8549245694501458      0.3015975172864231    0.3117939072585458
0.8550869122387649      0.30139648051862056   0.31159486926389834
0.8552564898497605      0.3011857366597022    0.31153397532790716
0.8554208001994524      0.30098081281157496   0.3115420366684558
0.8555923453715208      0.30076610674833976   0.31149946942000395
0.8557648743241256      0.30054939056987195   0.31130715849464724
0.8562633596977967      0.2999188785292792    0.31016835991427655
0.8564231543276392      0.2997154008430669    0.30992711411079865
0.8565922674604719      0.2994993429172408    0.3098299844562787
0.8567561133320006      0.2992893167775017    0.3098313870411045
0.856927194025906       0.29906928712130443   0.30981324162469637
0.8570992585003476      0.2988472437790691    0.3096610112412976
0.8572618883522586      0.2986366885746245    0.3093570265072584
0.8575963504395295      0.29820157808194275   0.30851729636672787
0.8577556805912091      0.2979933199769913    0.3082269072425468
0.8579243292458787      0.29777219592370985   0.30808381464661266
0.8580877106392445      0.29755730834839567   0.308068197086375
0.8582583268549868      0.2973322048339054    0.30806877131678584
0.8584236758094251      0.297113371076625     0.3079666898321415
0.8585837575025596      0.2969008910211439    0.30771141412530156
0.8589172906335046      0.2964562456147516    0.30686181092579107
0.8590886583907018      0.296226747164003     0.3064934334867766
0.8592610099284353      0.2959952221431752    0.3062969193782924
0.859423926843638       0.2957757200298538    0.3062573389250815
0.8595940785812173      0.29554579788139157   0.3062665795421324
0.8597589630574927      0.2953223407113943    0.306199123150432
0.8599185802724639      0.2951054127363209    0.30598226166640946
0.8602511844470829      0.29465147232214683   0.30514125177482765
0.8604220877261171      0.2944172221230944    0.30472829886430425
0.8605939747856874      0.2941809427325592    0.3044770313466484
0.8607564272227272      0.29395700767859617   0.3044027975718578
0.8609261144821434      0.29372245441283595   0.3044114823612041
0.8610905344802556      0.29349455581377054   0.3043767127821367
0.8612621893007446      0.29325597473418735   0.3041837293180306
0.8617684706811355      0.2925484362219099    0.3029039776284862
0.8619336422207026      0.292316366870645     0.3026182630745985
0.8620935464989657      0.29209112200794285   0.3025083786582962
0.8622627692802189      0.29185213645499286   0.3025072723179239
0.8624267248001682      0.29161998977734943   0.30249576238542264
0.8625979151424941      0.2913769722275737    0.3023459820602636
0.8627700892653564      0.2911319151802742    0.30200160161080675
0.8631028030883958      0.2906565453558161    0.3010742108035576
0.8632675101498         0.29042033812489054   0.30073613776108604
0.8634269499499         0.29019113338311675   0.30057954335240505
0.8635957082529903      0.289947944711677     0.3005575637106071
0.8637591992947765      0.28971177251254054   0.300563138008751
0.8639299251589394      0.28946454933408305   0.30045878831176065
0.8641016348036386      0.28921528650690465   0.3001601108810117
0.864433419670352       0.2887319142748788    0.2992258440084265
0.864597662253593       0.28849179096476596   0.29883491222574043
0.8647691396592108      0.2882405001480977    0.29861368520619025
0.8649416008453648      0.28798721769926483   0.2985675723920159
0.8651046274089881      0.2877472420300315    0.2985795969601736
0.8652748887949879      0.287496043413545     0.29850507333849546
0.8654398829196838      0.28725205976464907   0.2982543175341264
0.8657686551494577      0.2867642685754785    0.2973359751939135
0.8659324332545357      0.2865204740559632    0.29690167146077723
0.8661034461819903      0.28626534512161483   0.29662378044211135
0.8662754428899814      0.2860081692500503    0.29653929395320744
0.8664380049754415      0.2857645689830839    0.296551090528829
0.8666078018832784      0.285509578596957     0.2965125952512422
0.8667723315298113      0.2852619660266191    0.29631137163271504
0.8672634884301739      0.2845196976699604    0.29495403867111725
0.8674340368794655      0.28426087947435597   0.29461452786603726
0.8675993180674533      0.28400953067228196   0.29448116080003267
0.8677593319941369      0.2837657030245547    0.29447786762689576
0.8679286644238107      0.28350715468012644   0.294473424958395
0.8680927295921804      0.2832561400589932    0.29433509366153543
0.8682640295829269      0.282993524875885     0.29398478576971954
0.8685991625029618      0.2824781825777217    0.29297206911963586
0.8687692464740904      0.2822158561517057    0.2925782459972185
0.868934063183915       0.2819611534829474    0.2923939622784116
0.8690936126324357      0.2817141241937252    0.2923676774564037
0.8692624805839464      0.28145216983097954   0.29237886060538665
0.8694260812741532      0.28119790104945713   0.29228548444564245
0.8695969167867366      0.2809318809596177    0.2919857205364116
0.8699311207504453      0.28040998140922735   0.2909773770520623
0.8701007402434109      0.2801443528932493    0.2905294923652273
0.8702650924750726      0.27988649651105924   0.2902856684051773
0.8704366795291107      0.27961679155492314   0.290223965925274
0.8706092503636853      0.2793450303554232    0.2902402860390184
0.870772386575729       0.2790876590535764    0.29017563834285953
0.8709427576101494      0.2788183913871313    0.2899129613136735
0.8712676978950783      0.27830356079251445   0.28894438956605556
0.8714368529098808      0.27803487790142273   0.2884531838720119
0.8716007406633793      0.27777410767912997   0.288153507781797
0.8717718632392546      0.2775013516150631    0.2880494461873942
0.8719439695956661      0.2772265414451698    0.2880626416202631
0.8721066413295469      0.27696635022073823   0.28803251371961813
0.8722765478858041      0.2766941268897398    0.2878237542590746
0.8727692497510464      0.2759020872560887    0.2863684253086321
0.872932673026382       0.2756385191355814    0.286008673460075
0.8731033311240941      0.2753628304495983    0.2858503396700622
0.8732687219605023      0.27509521217148963   0.2858465040973401
0.8734288455356066      0.27483570773195753   0.28584811202237337
0.873598287613701       0.27456066529539946   0.2857069834034181
0.8737624624304914      0.2742937474940017    0.28535967441747606
0.8741062654893615      0.2737334413488093    0.2842531931870113
0.874269224286534       0.2734672306718149    0.28384144051318116
0.8744394179060831      0.27318877055730756   0.2836285000500879
0.8746043442643283      0.2729185110150246    0.28359941165533276
0.8747640033612696      0.2726564935748449    0.283615101817531
0.8749329809612009      0.27237876875783334   0.2835213839539271
0.8750966912998281      0.27210929651293025   0.2832258517259639
0.8754395654023724      0.27154363516405083   0.2821307431065535
0.8756020597213818      0.27127495735130214   0.2816675729782608
0.875771788862768       0.27099390790671724   0.2813914504336252
0.8759362507428501      0.2707211838764743    0.2813242253714445
0.8760954453616283      0.2704568249045831    0.2813437533000184
0.8762639584833966      0.27017659783259546   0.28129496961641387
0.8764272043438608      0.26990474596488323   0.28105866992569645
0.8769228446084717      0.26907706844533286   0.2795149657462821
0.8770921092716948      0.26879362862220635   0.2791578958400735
0.877256106673614       0.26851863305346285   0.2790289324021528
0.8774273388979096      0.2682311582787103    0.2790374391253724
0.8775995549027417      0.2679416565992277    0.27902193749483944
0.8777623362850429      0.2676676462782602    0.2788388792926816
0.8779323524897208      0.26738107721793797   0.278419055489828
0.8782565831151647      0.2668335041956945    0.27731939073253187
0.8784253833002248      0.2665478767001085    0.27690096168930467
0.8785889162239808      0.26627080457145574   0.2767159189802763
0.8787596839701135      0.2659811012735232    0.2767009214442609
0.8789314354967825      0.2656893465514537    0.2767107007232531
0.8790937524009208      0.2654132679316223    0.27658320673370546
0.8792633041274356      0.26512452224849314   0.27622129875418594
0.8795991078802341      0.2645515679967392    0.27508170765528883
0.8797716109483579      0.2642566854264379    0.27460756675359443
0.8799346793939509      0.2639775871948848    0.27437930444903325
0.8801049826619206      0.26368575198309374   0.27434036371109677
0.8802700186685863      0.2634025995406258    0.27436134334443957
0.8804297874139481      0.2631281641225591    0.2742794301079696
0.8805988746622999      0.2628373813939735    0.273973461807688
0.8809337494587722      0.26226046419484755   0.2728539550999674
0.8811057880487331      0.26196355275022204   0.27232724989010887
0.881268392016163       0.2616825984792315    0.2720372558871625
0.8814382308059697      0.2613888079275811    0.2719558551610055
0.8816028023344724      0.2611038036598911    0.27197937348583784
0.8817621066016712      0.2608276182859655    0.2719398316401862
0.8819307293718599      0.26053495480481736   0.2716969484557616
0.8824299982819637      0.25966649306709133   0.2700681521318888
0.8825900540906171      0.2593874784555612    0.269705462122623
0.8827594284022607      0.2590919041981442    0.2695568275652371
0.8829235354526004      0.2588052145361315    0.269565102492162
0.8830948773253167      0.2585055653054467    0.26956099453656385
0.8832672029785693      0.25820386751507424   0.2693721489946178
0.8834300940092912      0.25791838686301105   0.26896064261685826
0.883765078454184       0.25733045808388133   0.26777319342010947
0.8839246697846743      0.25704996282477716   0.26735273035293
0.8840935796181549      0.2567527917962105    0.26714476264926457
0.8842572221903315      0.25646459743573735   0.2671279479474725
0.8844280995848848      0.2561633586188838    0.267146094506446
0.8845999607599744      0.25586007511575326   0.2670163097599494
0.8847623873125331      0.2555731565370453    0.2666628803907424
0.8850964428011001      0.25498220232416835   0.2654820401452565
0.8852680717371081      0.2546781384623525    0.2649748424290722
0.8854406844536525      0.2543720281666286    0.2647130043187776
0.8856038625476661      0.25408237104966247   0.26467218434544143
0.8857742754640563      0.2537795847046942    0.26469893641290393
0.8859394211191425      0.2534858794085011    0.26461458742351757
0.8860992995129249      0.25320128296357625   0.2643186747097821
0.8864324260451656      0.25260747990884513   0.26316492243931505
0.8866035905030107      0.2523019541447278    0.2626077285506478
0.886775738741392       0.2519943855914523    0.26228030405918107
0.8869384523572426      0.25170341079512126   0.26219732143544344
0.8871084007954697      0.251399227575566     0.26222496427637293
0.8872730819723929      0.2511042099294096    0.26218490169776504
0.8874324958880121      0.2508183842393664    0.26195004549857326
0.8876012283066215      0.2505155908600669    0.2614615491934637
0.8879353934436088      0.2499151440217381    0.26024773066089735
0.8881070772038272      0.2496062528254199    0.2598489546798991
0.8882693263415147      0.24931408898854923   0.2597106749243702
0.8884388103015788      0.24900864251981272   0.25972486010750245
0.8886030270003389      0.24871244217270022   0.25972227267942705
0.8887744785214757      0.24840293512051875   0.2595294381148809
0.8889469138231489      0.2480913891137288    0.2590715973427376
0.8892801500038103      0.2474885773710328    0.25784241983559125
0.8894451182440253      0.24718979855490572   0.2574054375021475
0.8896048192229362      0.2469003360993426    0.25721492702977056
0.8897738387048374      0.24659374604127196   0.2572046267328379
0.8899375909254346      0.246296507175523     0.2572248678197903
0.8901085779684084      0.24598592203520964   0.25709164886924873
0.8902805487919185      0.24567330304412396   0.2566956660440725
0.8906128560162537      0.24506851962685      0.25546999674121074
0.8907773597783057      0.24476879455200745   0.25497233232675326
0.8909365962790536      0.24447845633081783   0.2547167464678504
0.8911051512827918      0.2441709039104951    0.2546671408069126
0.8912684390252259      0.24387274450335986   0.25469904133606197
0.8914389615900367      0.24356114722526304   0.2546238054147332
0.8916042168935434      0.2432589548529608    0.2543163131005987
0.8919335114809392      0.24265615410223812   0.25313859933966804
0.8920975507648281      0.24235555182829438   0.2525755603517424
0.8922688248710936      0.24204147036755622   0.252218117257209
0.8924410827578956      0.24172535843279946   0.2521180974671426
0.8926039060221667      0.24142635283213162   0.2521483628692742
0.8927739641088144      0.2411138479163459    0.252117003937261
0.8929387549341581      0.24081081580127914   0.2518731112910128
0.8934306953709537      0.239904997444542     0.2501487513596975
0.8936015049990562      0.23959006948222158   0.24972118798910278
0.893773298407695       0.23927311554929362   0.24956112374229306
0.8939356571938031      0.238973374203599     0.2495762189038703
0.8941052508022878      0.23866007641239861   0.2495815323909592
0.8942695771494684      0.23835631587610243   0.24940515144748485
0.8944411383190258      0.23803898035866303   0.24895379077342297
0.8947767935966826      0.2374175328123258    0.2476766806242774
0.894947138746622       0.23710185340763199   0.2472013146275969
0.8951122166352576      0.23679574797703093   0.2469973901305904
0.8952720272625891      0.2364992355844425    0.2469886770726989
0.8954411563929108      0.23618524864021223   0.24701577401650465
0.8956050182619284      0.23588086035812914   0.24689600214203233
0.8957761149533228      0.2355628453164626    0.24650795072038828
0.8961108412746533      0.23494014675202826   0.24524195456645312
0.8962807219464299      0.2346238784931503    0.2447056090115126
0.8964453353569024      0.23431724429125275   0.24443469500212656
0.8966046815060709      0.23402025976549348   0.2443880074155934
0.8967733461582295      0.23370573549749263   0.24442559650224857
0.896936743549084       0.2334008660062658    0.2443606788409534
0.8971073757623154      0.23308232306446      0.24404519499217556
0.8974328384048662      0.23247424722219825   0.24285203762264113
0.8976022545984795      0.23215746832343542   0.2422498401530704
0.8977664035307891      0.23185037592744498   0.24189079711337813
0.8979377872854752      0.2315295794299341    0.24177973259746907
0.8981101548206976      0.2312067687597113    0.24181457666796427
0.8982730877333891      0.23090146939886141   0.24179086121508067
0.8984432554684574      0.23058245121509763   0.24154096708118233
0.8989367408701325      0.22965637545566828   0.23976521186010408
0.8991004253242789      0.22934890545761147   0.2393397905972165
0.8992713446008019      0.22902768850401112   0.23916787424871372
0.8994432476578614      0.22870446254351476   0.23918554142240725
0.89960571609239        0.22839883005683248   0.23919633453205807
0.8997754193492953      0.22807943719622314   0.23901591051775126
0.8999398553448965      0.22776981246088707   0.2385778291034703
0.9002675113164812      0.227152431839995     0.2372967888757585
0.9004307312924646      0.22684468063080385   0.236803640238318
0.9006011860908247      0.2265231434129649    0.23655853526744983
0.9007663736278808      0.2262114023477122    0.2365405812470429
0.9009262939036329      0.22590947186685084   0.236576480864695
0.9010955326823751      0.22558981031943892   0.2364766284884602
0.9012595041998133      0.22527996398009226   0.23612653688846783
0.9016029006599795      0.22463064945128294   0.23481336367785902
0.9017656561577998      0.22432270554018333   0.23426553366906072
0.9019356464779967      0.2240009401227573    0.23395090300348673
0.9021003695368899      0.22368901677128933   0.23389069712681726
0.9022598253344789      0.22338694883908158   0.233931908058361
0.9024285996350581      0.2230671072479775    0.2338849289892186
0.9025921066743332      0.22275713965676072   0.2336050206456528
0.9029345741781732      0.2221075243424609    0.23234368545503925
0.9030968651978306      0.22179950068633328   0.23174560987006998
0.9032663910398646      0.221477624274515     0.23135440795352558
0.9034306496205946      0.2211656318030185    0.23123717507395297
0.9036021430237011      0.22083977606679717   0.23127290968836528
0.9037746202073441      0.2205119273018659    0.23125832079796782
0.9039376627684563      0.22020189928417463   0.23102734666346714
0.9041079401519452      0.21987799842823694   0.23051106344246725
0.9044326931350108      0.21925993197108193   0.22921847845265025
0.9046017544988818      0.21893801109497707   0.228760340317608
0.9047655486014488      0.21862601382601538   0.2285841053186802
0.9049365775263924      0.21830012543347085   0.22859916776203973
0.9051085902318725      0.21797225039209908   0.22861861150511667
0.9052711683148216      0.21766225671373798   0.22845353142728558
0.9054409812201473      0.2173383635526048    0.2280031281327699
0.905764805246887       0.21672042551885387   0.2267097073393091
0.9059334021325949      0.2163985513463515    0.22618344279655794
0.9060967317569988      0.21608663761461067   0.22593693435146397
0.9062672962037794      0.2157608084422623    0.22591513363740345
0.9064325933892561      0.21544494613432333   0.22595701723850542
0.9065926233134287      0.21513906084690831   0.2258683986158192
0.9067619717405915      0.2148152708699278    0.22550965533243955
0.9070973688946857      0.21417372264098442   0.2242037765121133
0.9072696686634574      0.21384400686607258   0.22360624407204446
0.9074325338096982      0.2135322593762814    0.22329386405524962
0.9076026337783158      0.21320657551894714   0.22322764548652704
0.9077674664856294      0.21289089230969396   0.22327489161789849
0.907927031931639       0.2125852188994217    0.22323556835953895
0.9080959158806388      0.21226161216174053   0.22294883640326987
0.9084303840784069      0.21162047701937575   0.22169775758057342
0.9086022193690155      0.21129096595964666   0.2210493334140897
0.9087646200370934      0.21097946845984183   0.22066457293578856
0.9089342555275479      0.21065401573215672   0.22053926685340108
0.9090986237566985      0.2103385947827021    0.22057761764868924
0.9092702268082258      0.21000921427930455   0.2205747449322571
0.9094428136402892      0.2096778685894811    0.22033493251171918
0.9096059658498219      0.20936456663663383   0.2198355236711584
0.9099414726523367      0.2087200840984083    0.2184796781446705
0.9101013251616381      0.20841292419467533   0.21803944763644478
0.9102704961739296      0.20808779311268177   0.21785391541037055
0.9104343999249171      0.2077727222648031    0.2178717023971191
0.9106055384982814      0.20744367921549128   0.21790018999874308
0.9107776608521818      0.2071126793613573    0.21772897742491473
0.9109403485835514      0.2067997638357001    0.2172928312503635
0.9112749264297402      0.20615605808932128   0.21593729177585766
0.9114343144608785      0.20584932576606266   0.2154336470845883
0.911603020995007       0.20552460587760943   0.21517657929272713
0.9117664602678315      0.20520997183549508   0.21515885599528553
0.9119371343630327      0.20488135649698683   0.21520749236371853
0.9121087922387701      0.20455079308579213   0.21510479340039995
0.9122710154919766      0.20423834963146098   0.21474267985755982
0.9126046643818393      0.20359559622001738   0.2134171753148981
0.9127760900184954      0.20326528369161168   0.21281748475538048
0.9129484994356876      0.20293302821166948   0.21249707078535598
0.913111474230349       0.20261891213074834   0.2124473546104543
0.9132816838473872      0.20229080913095518   0.21250135372077283
0.9134466262031211      0.2019728193002097    0.2124484494765901
0.9136063012975513      0.20166494763957812   0.21215683597290197
0.9139390212310878      0.2010233156396982    0.21088350965531044
0.9141099823895809      0.20069357121463238   0.2102335026707591
0.91428192732861        0.2003618933301556    0.20983788196257613
0.9144444376451084      0.20004838274981895   0.20973439523659426
0.9146141827839835      0.19972088321414577   0.2097824050013205
0.9147786606615547      0.1994035164842178    0.20977709632339506
0.9149378712778218      0.19909628320541622   0.20955864785655737
0.915106400397079       0.19877101362580232   0.2090455489834904
0.9154401589353622      0.19812676623752407   0.20767563944857134
0.9156053883543881      0.1978077922849607    0.2072124577800916
0.9157653505121099      0.19749896646867682   0.20703198121373448
0.915934631172822       0.19717212989516023   0.20705050509861922
0.9160986445722301      0.1968554445185181    0.20708740815296367
0.9162698927940149      0.19652477184490774   0.20693550636880031
0.9164421247963359      0.19619218260717866   0.20648030256748276
0.916774954378293       0.19554942860629893   0.2051138151108498
0.9169397193191557      0.19523122001331514   0.20458683573641326
0.9170992169987147      0.19492317458031377   0.20433961224510463
0.9172680331812637      0.19459712317801858   0.20432071386194597
0.9174315821025087      0.1942812378068252    0.20437484964070624
0.9176023658461305      0.19395137276423097   0.2042885225043242
0.9177741333702885      0.1936196029013684    0.20391115892724743
0.9181060339959195      0.19297852768811707   0.202576422548643
0.9182703344586193      0.1926611760033604    0.2019880020345231
0.9184418697436958      0.19232985144205392   0.2016512698762572
0.9186143888093086      0.19199662970672318   0.20159564506651484
0.9187774732523906      0.19168163523054754   0.20165387034372204
0.9189477925178493      0.19135267270040124   0.2016082610156938
0.9191128445220039      0.19103389074923066   0.2013042490557408
0.9194417325106954      0.19039870202527504   0.20002741096837168
0.9196055684952322      0.19008229786847963   0.19939324581897896
0.9197766393021458      0.18975193492328446   0.19898358920875517
0.9199486938895955      0.18941968738699988   0.19887148826555315
0.9201113138545145      0.1891056741301545    0.1989232397534617
0.9202811686418102      0.18877770821734713   0.19892504515446646
0.9204457561678017      0.18845993114790366   0.19869569704004725
0.9206050764324893      0.18815234249206889   0.1982053358866714
0.920937086706541       0.18751141765060664   0.19682666503099105
0.9211076930352915      0.1871821097693472    0.19634020216579826
0.9212730321027378      0.18686296515933065   0.19615911746225267
0.9214331039088803      0.18655398765391953   0.1961831128369936
0.921602494218013       0.186227052304172     0.19622729103483375
0.9217666172658415      0.1859103133473406    0.19608812536121478
0.9219379751360468      0.1855796454727825    0.19564368544354777
0.9222732238149991      0.18493282461690488   0.19425762040015532
0.9224433656655866      0.1846046140229667    0.19370753556920187
0.92260824025487        0.1842866030794511    0.19345953524318654
0.9227678475828494      0.183978789663826     0.1934492560392547
0.922936773413819       0.18365304727282813   0.19351080894215966
0.9231004319834843      0.18333750500932086   0.19343341903318087
0.9232713253755267      0.18300806072657266   0.1930658427218955
0.9236056450981529      0.1823637129384879    0.1917153665749861
0.9237753224705774      0.18203676432375854   0.1911037614943448
0.9239397325816976      0.18172001702680035   0.19077966746125208
0.9241113775151948      0.18138938763136161   0.19072276872465987
0.9242840062292281      0.18105692304995458   0.1907898588496242
0.9244472003207307      0.18074268538355454   0.19075045015254266
0.92461762923461        0.18041457743796255   0.1904372601525491
0.9249426852784564      0.17978896132065267   0.18916374160316254
0.9251118981727178      0.1794633832521341    0.18850643053726815
0.9252758438056751      0.17914800462487876   0.18811434776295763
0.9254470242610091      0.17881877837811708   0.1880018844951732
0.9256191884968794      0.17848773383964522   0.1880625878520973
0.9257819181102189      0.1781749002733993    0.18806835586165355
0.9259518825459352      0.17784823242013573   0.1878309565387638
0.9261165797203473      0.17753176194148565   0.1873116201455613
0.9264447580495538      0.17690138069530292   0.18593977741015805
0.926608239204348       0.1765874720107071    0.18547581971041188
0.9267789551815191      0.17625975459663415   0.1852941601178392
0.9269506549392263      0.17593023621708243   0.18533254878570363
0.9271129200744028      0.17561890669662855   0.18537480955438262
0.927282420031956       0.17529378327959785   0.1852161107653288
0.927446652728205       0.1749788497754064    0.18477172785656842
0.9277905715459931      0.17431946863724526   0.18334135492352693
0.9279535882226243      0.17400706041469774   0.1828308591993108
0.9281238397216323      0.17368088489629338   0.18259753177304489
0.9282888239593361      0.1733648978305368    0.18260891563922535
0.9284485409357361      0.1730590919213132    0.18267072104265752
0.9286175764151263      0.17273554577044378   0.1825767400815552
0.9287813446332124      0.17242218318936367   0.18220432062849976
0.9291243344946742      0.1717662213557603    0.18080322768549897
0.9292868866931425      0.17145550455367478   0.18023310236749251
0.9294566737139875      0.1711310719955113    0.17992402449027617
0.9296211934735285      0.1708168168761754    0.1798905158623978
0.9297804459717653      0.17051273039741568   0.17996069882689777
0.9299490169729925      0.17019096774257697   0.17992824365253945
0.9301123207129154      0.1698793758812573    0.17963621910593283
0.930448130576211       0.16923900296434544   0.1783233062195477
0.9306081346159027      0.16893406177742576   0.1776988773773303
0.9307774571585847      0.16861148934391496   0.17729123695017482
0.9309415124399625      0.16829907922426174   0.17718406071822965
0.9311128025437172      0.16797302815546394   0.1772490521451897
0.9312850764280081      0.1676452470935864    0.17726385379673673
0.931447915689768       0.16733555056100707   0.17704313043397799
0.9316179897739048      0.1670122350651896    0.1765115901116063
0.9319423361582662      0.16639604962148788   0.17515085651772916
0.9321111942227851      0.16607547196579495   0.17467162092181013
0.9322747850259999      0.16576503711089427   0.17449957902279678
0.9324456106515915      0.165441025417236     0.1745409240456299
0.9326174200577193      0.16511530673236427   0.17459295865893243
0.9327797948413161      0.16480762287815      0.17444639397861142
0.9329494044472899      0.16448638596539678   0.17399126647169766
0.9332853239590058      0.16385064103148375   0.17258998937721257
0.9334578849065884      0.16352431256839609   0.17204921395395373
0.9336210112316403      0.1632159857762165    0.1718299734001237
0.9337913723790687      0.1628941083034799    0.17184742634499153
0.9339564662651934      0.1625822907917292    0.17191678686411135
0.934116292890014       0.16228057962007103   0.1718308727899995
0.9342854380178245      0.16196144916510508   0.1714485399221342
0.9346204285732145      0.1613299446837105    0.17007791636543457
0.9347925250426341      0.16100579804190002   0.1694766895329216
0.9349551868895228      0.1606995992872489    0.1691872089594698
0.9351250835587883      0.16037996833234003   0.16915984880427112
0.9352897129667497      0.16007043131826135   0.16923968171580042
0.9354490751134072      0.1597709730645535    0.16921170649834924
0.935617755763055       0.15945419456401527   0.16891076679480577
0.9359518173621189      0.15882742142450088   0.16760044827124748
0.9361171983115351      0.1585174239489903    0.16696348659531537
0.9362773119996475      0.15821748795522      0.16658844070965825
0.9364467441907498      0.1579003000673242    0.16648730716637747
0.9366109091205483      0.1575931754678537    0.16655896506602957
0.9367823088727234      0.1572727311291772    0.1665845795058072
0.9369546924054348      0.15695067211931923   0.16635373540500598
0.9371176413156153      0.15664644904535976   0.16584293890378227
0.9374527415194259      0.1560214719907012    0.16444717507656711
0.9376123907293752      0.15572403109720512   0.1640065575311378
0.9377813584423145      0.15540945165175177   0.16384137058793075
0.93794505889395        0.1551048993936304    0.16389189136386378
0.938115994167962       0.15478712170431405   0.16395322974257975
0.9382879132225104      0.15446775929566797   0.16379933040777198
0.9384503976545279      0.15416615046280713   0.16336021264138584
0.9387845689020122      0.15354655577707368   0.16197135124497314
0.9389437536337986      0.1532517456847338    0.16146431606007244
0.939112256868575       0.15293991861017794   0.1612232749436506
0.9392754928420474      0.1526380775575058    0.16123647805658825
0.9394459636378962      0.15232311169029347   0.16132096505566823
0.9396111671724413      0.15201812696359543   0.16125114228994591
0.9397711034456824      0.15172310218753052   0.16091241528384626
0.9401043457368408      0.15110906308494698   0.15957263503756686
0.9402755680741446      0.15079389108077437   0.1589560861641894
0.9404477741919848      0.15047718793782192   0.1586333272715233
0.940610545687294       0.1501780967641283    0.15860228916320357
0.9407805520049801      0.14986598531643996   0.15869331302911088
0.9409452910613622      0.14956381358040896   0.15868032569220772
0.9411047628564402      0.1492715588061928    0.15841719888538208
0.9412735531545082      0.1489625032685548    0.15784907704079193
0.9414370761912725      0.14866336632722973   0.15714633330434088
0.9416078340504133      0.14835128563331743   0.15648143582356963
0.9417795756900904      0.14803771008623606   0.15608029407837698
0.9419418827072367      0.1477416431082323    0.15599066949905804
0.9421114245467597      0.14743267491667272   0.1560729379380707
0.9422756991249785      0.14713359696816392   0.156109916153917
0.9424472085255743      0.14682165636149488   0.15590089941043594
0.9426197017067062      0.14650824807877846   0.15537152925395317
0.9429530536462851      0.1459034965299688    0.15399478716437243
0.9431180797659588      0.145604571366466     0.15355280560247936
0.9432778386243286      0.14531547948596196   0.15340804529608465
0.9434469159856886      0.1450098413760331    0.15346935083074692
0.9436107260857446      0.14471403765374127   0.15353935227757187
0.9437817710081771      0.14440550115096934   0.15340434495272615
0.943953799711146       0.14409553499717193   0.15294929415624275
0.9442862226943989      0.14349755796904545   0.1515799189946527
0.9444507843359096      0.14320202690687237   0.1510718094055716
0.9446100787161164      0.1429162675909478    0.15085832131163626
0.9447786915993133      0.1426141294836489    0.15088298708190237
0.9449420372212061      0.14232176385459722   0.15097481745711408
0.9451126176654757      0.14201680286151022   0.1509130951459043
0.9452779308484413      0.14172160657094576   0.15056286921904344
0.9456073411947545      0.14113441816984718   0.14924646363936972
0.9457714383581024      0.14084242663555963   0.1486639242780473
0.9459427703438268      0.14053793384788998   0.14834571818957942
0.9461150861100875      0.14023207943217653   0.1483238276580783
0.9462779672538174      0.13994331183180705   0.14842252913355308
0.9464480832199238      0.13964198948743947   0.14841799238276737
0.9466129319247264      0.13935036699542028   0.1481438558632499
0.9469414133147136      0.13877037264734632   0.14689199373576015
0.9471050459998984      0.13848200196001764   0.14626554222714103
0.9472759135074598      0.1381812784867836    0.14587305137022982
0.9474477647955575      0.1378792365720858    0.14579241542531196
0.9476101814611242      0.13759416084610807   0.14588441379336609
0.9477798329490678      0.13729678858730776   0.14593082083432427
0.9479442171757072      0.13700904434089364   0.145738158163394
0.9481158362247233      0.13670905479056245   0.14522580776166363
0.9484516072612976      0.13612337965869054   0.14386107420114633
0.9486220102906959      0.13582679180279447   0.14342099383894794
0.9487871460587902      0.13553978802369435   0.14329601374989645
0.9489470145655806      0.1352623323108319    0.14337043332093213
0.9491162015753609      0.13496912930723123   0.14345119615810847
0.9492801213238374      0.1346854749136308    0.14332796871399398
0.9494512758946906      0.13438974618062477   0.14288805128079624
0.9497861179749387      0.13381251985689258   0.1415333657331782
0.949956056526174       0.1335202470165695    0.1410282510826145
0.9501207278161052      0.13323747471072153   0.14083671643829865
0.9502801318447325      0.13296416441760905   0.14088028072053713
0.9504488543763501      0.1326753272563589    0.14098508358503892
0.9506123096466635      0.13239595221514822   0.1409312000432714
0.9507829997393535      0.13210468262199715   0.14057458207673215
0.9511169128632757      0.13153629110845552   0.139258494223862
0.9512863869363479      0.13124852890241753   0.13869050436143046
0.9514505937481161      0.13097017663633975   0.1384228445897329
0.9516220353822611      0.1306800537025137    0.13842683228789263
0.9517944607969424      0.1303887780056231    0.13854320513056956
0.9519574515890927      0.1301139156592184    0.13853318515438082
0.9521276772036199      0.12982734975200325   0.13823697850193725
0.9524523266487621      0.1292822403145807    0.13700264128562864
0.9526213362436713      0.12899913393188367   0.1363887332822979
0.9527850785772765      0.12872529660797127   0.13605329788952683
0.9529560557332584      0.1284398799802304    0.13600422203935475
0.9531280166697766      0.12815336116738568   0.1361196082329093
0.953290542983764       0.12788306395600707   0.13616149406933722
0.9534603041201279      0.12760125923656734   0.13594777683094902
0.9536247979951881      0.12732871354161535   0.13544757449236297
0.9539525697256903      0.1267871644343484    0.13414542805025625
0.9541158475811324      0.12651816096228394   0.13373829669148649
0.9542863602589513      0.1262377874697581    0.1336222213323339
0.9544516056754662      0.12596661451959562   0.13371544798561125
0.954611583830677       0.12570459499536943   0.13380453941328432
0.9547808804888781      0.12542786321929456   0.13369072626010103
0.9549449098857751      0.12516028459541492   0.13328466827852953
0.9552884221048588      0.12460166769185639   0.13193387326377046
0.9554512354821381      0.12433773682693668   0.13146824128503606
0.9556212836817939      0.12406265662358164   0.13128751316030973
0.9557860646201457      0.12379666512779824   0.1313525746877414
0.9559455782971936      0.12353971236378847   0.13146521373036138
0.9561144104772314      0.12326832744256155   0.1314210673713756
0.9562779753959655      0.12300598048206118   0.13109280161574216
0.9566205586587231      0.12245833884687209   0.12978177507130004
0.9567829075578393      0.12219969055120562   0.12925920547056358
0.9569524912793321      0.12193012361366391   0.12900439687325405
0.9571168077395209      0.12166952551756352   0.1290261762479479
0.9572883590220864      0.1213980835060687    0.12915783687979535
0.9574608940851882      0.12112573846978447   0.12916015719258778
0.9576239945257591      0.12086889253088196   0.12888763266792513
0.9579593977903503      0.1203425778203061    0.12764883196022422
0.95811919853069        0.12009271045628923   0.12709318791639548
0.9582883177740198      0.11982890473913024   0.12677294501307235
0.9584521697560455      0.1195739392291958    0.12674753913135192
0.958623256560448       0.11930837523792544   0.12687854052510836
0.9587953271453866      0.11904196188208886   0.12693480801301493
0.9589579631077946      0.11879069731721603   0.12673917012680136
0.9591278338925792      0.11852891795782168   0.12623702519169255
0.9594517736782364      0.1180316038367247    0.12499211643081055
0.9596204284434031      0.1177736760864835    0.1246021447949595
0.9597838159472658      0.11752445707166556   0.12451695994841207
0.9599544382735052      0.1172648934885949    0.12463242343501776
0.9601197933384407      0.11701402074410915   0.12473389749020217
0.9602798811420722      0.11677177915445695   0.12463206579266857
0.9604492874486936      0.1165161268198359    0.12422377030711935
0.9607848003617054      0.11601191313212703   0.12294528028377984
0.960957158009936       0.11575399189651085   0.12248426165245954
0.9611200810356356      0.11551088135539653   0.12234183866902358
0.9612902388837119      0.11525769811940895   0.12243209906893404
0.9614551294704844      0.11501306127600908   0.12256073965921796
0.9616147527959529      0.11477690798819576   0.12252410093910174
0.9617836946244114      0.11452768961782013   0.12219301510803751
0.9621182785810971      0.1140363267094553    0.12095180762781006
0.9622901717511645      0.1137850374131942    0.12043385479971398
0.9624526302987012      0.11354826336939622   0.12022532661019823
0.9626223236686146      0.1133016995457449    0.12027597035797499
0.9627867497772239      0.11306352881023195   0.12042045935610086
0.9629459086245293      0.11283368520052875   0.12044660354940613
0.9631143859748248      0.11259113648947494   0.12020117012694705
0.9634480409751844      0.11211308907536968   0.11902668551869613
0.9636132186252486      0.11187756883150211   0.11847342369052363
0.9637731290140088      0.11165028338077888   0.11818275614592924
0.9639423579057591      0.11141053379118195   0.11816568115128659
0.9641063195362054      0.11117901622245406   0.11830823793953316
0.9642775159890283      0.11093809600067042   0.1183977038875207
0.9644496962223876      0.11069663387344429   0.11822804042624542
0.964612441833216       0.1104691838049868    0.11778118055399749
0.9649471354383223      0.11000382611138483   0.11655207004299667
0.9651065813489195      0.1097832540917598    0.1162004961751849
0.9652753457625068      0.10955055349115295   0.11612503014320524
0.9654388429147901      0.10932591583339303   0.11625204318931295
0.96560957488945        0.10909218325654026   0.11638244854501158
0.9657812906446461      0.10885798009107392   0.11629389799374736
0.9659435717773116      0.10863745707055167   0.11592185054332314
0.9662773364260918      0.10818640577368345   0.11470675552196505
0.9664488199422065      0.10795598003109833   0.11427223084159874
0.9666212872388575      0.10772514153441415   0.11414993594439451
0.9667843199129779      0.10750777379907697   0.11426116668829821
0.9669545874094748      0.10728163954312396   0.11441359264904463
0.9671195876446678      0.10706336278771368   0.11438572976673433
0.9672793206185567      0.10685286662176924   0.11408696833837041
0.9676121563110108      0.10641684219421768   0.11291054265126717
0.9677831753489625      0.10619417144594397   0.11242439234972679
0.9679551781674506      0.10597116401750421   0.1122377368587574
0.9681177463634077      0.1057612639918119    0.11231529548130902
0.9682875493817416      0.10554293556869214   0.11248561750575417
0.9684520851387716      0.10533227404842752   0.11252229973069856
0.9686113536344975      0.10512919919414235   0.11230177649727269
0.9689432603706255      0.10470868465628182   0.11119075799367321
0.9691138149304142      0.10449401739327303   0.11065729048035218
0.9692791022288989      0.10428690552208875   0.11040281902698158
0.9694391222660795      0.10408726628967539   0.11042257909745926
0.9696084608062504      0.10387694091329998   0.11059457584702852
0.9697725320851173      0.10367408374708222   0.11069897480793195
0.9699438381863608      0.10346325897393067   0.11055509529965288
0.9701161280681405      0.10325223576313027   0.11010516779238146
0.9702789833273896      0.10305370621830592   0.10951689691269313
0.9704490734090152      0.10284733510584046   0.10895282616660604
0.9706138962293368      0.10264831282700473   0.10863991472968487
0.9707734517883545      0.10245655330669096   0.10861018926474306
0.9709423258503622      0.10225456633268416   0.10877065995657241
0.9711059326510662      0.10205983710930411   0.10891527293421542
0.9712767742741466      0.10185750761949913   0.10884940951905395
0.9714485996777634      0.10165503209931981   0.10847476479327976
0.9717806160623119      0.10126676704634793   0.10733243438136797
0.9719449744044707      0.10107603321310857   0.10695937949870747
0.972116567569006       0.10087794850152848   0.10687341365789846
0.9722891445140777      0.10067981005689774   0.10702366804006666
0.9724522868366184      0.10049350617601434   0.10719088843078448
0.9726226639815359      0.10029998546978816   0.10717700420670302
0.9727877738651494      0.10011347113622639   0.10687592350757216
0.9731167776127584      0.09974483911225063   0.10577206002707353
0.973280671476754       0.09956271582120303   0.10535280375577925
0.9734518001631263      0.09937363234685768   0.10520852808686287
0.9736239126300348      0.09918457928546211   0.10532923038717208
0.9737865904744126      0.09900692443769651   0.10551705011702253
0.973956503141167       0.0988224475597045    0.10556919768695214
0.9741211485466174      0.09864474582075025   0.10534530834548159
0.9744492233379003      0.09829377359410484   0.10429726584211967
0.974612652723733       0.09812049691223675   0.10383461463652732
0.9747833169319422      0.09794066268302794   0.10362483753990634
0.9749487138788474      0.09776746877794916   0.10369587180500228
0.9751088435644487      0.09760081720266882   0.10388925163005411
0.97527829175304        0.09742557245616978   0.10401286723118222
0.9754424726803275      0.09725686355453811   0.1038873502624333
0.9756138884299914      0.09708186843347127   0.10346599690299028
0.9757862879601918      0.09690705724656243   0.10288051283425657
0.9759492528678614      0.09674291311644669   0.10238935708912213
0.9761194525979076      0.09657262858789073   0.10212423968765887
0.9762843850666498      0.09640873659925366   0.10215168463686426
0.976444050274088       0.0962511357584579    0.10234047073452522
0.9766130339845164      0.09608547445664704   0.10250821830679227
0.9767767504336408      0.09592609722454859   0.1024558344781016
0.9769477017051418      0.09576085875255334   0.10210548616041454
0.9772821371866857      0.09544111051451551   0.10103645383133551
0.9774518724385688      0.09528061726082104   0.10071317570268011
0.9776163404291479      0.09512626141472925   0.10068641981945362
0.9777755411584231      0.09497794500545988   0.10085820958162568
0.9779440603906884      0.09482211686173031   0.10106178305367605
0.9781073123616498      0.0946723110229891    0.10108386763935118
0.9782777991549878      0.09451708108162558   0.10081551255747538
0.9786029709577518      0.09422446845529528   0.0998039377931589
0.9787722417314719      0.09407395272482062   0.09941058807834638
0.9789362452438881      0.09392930637111907   0.09930539157515678
0.9791074835786808      0.09377952945576987   0.09945331917226835
0.97927970569401        0.09363018559813328   0.09968424294354795
0.9794424931868083      0.09349022041654656   0.09976739215118081
0.9796125155019832      0.09334528201646591   0.0995755671279328
0.9799367583484212      0.09307242655621935   0.09861954555826852
0.9801055646439782      0.09293222701502298   0.09818575471378205
0.9802691036782313      0.09279761874707994   0.0980230861562074
0.9804398775348611      0.09265833846285892   0.09813133567241251
0.9806116351720271      0.0925195830134402    0.09837303942311265
0.9807739581866625      0.09238967745277873   0.0985143397113248
0.9809435160236744      0.0922552611897024    0.09840584830808899
0.9811078065993822      0.0921262720931542    0.09803009070951793
0.9812793319974669      0.09199292214881812   0.09749110766741555
0.9814518411760877      0.09186017204613284   0.09702687827761003
0.981614915732178       0.09173594525940325   0.09683043652151391
0.9817852251106447      0.0916075226298623    0.09691119000220882
0.9819502672278074      0.09148435890280143   0.09714674659034377
0.9821100420836663      0.0913663372211816    0.09732916742314508
0.9822791354425151      0.09124273489482637   0.09729400766843188
0.9824429615400601      0.09112426509417756   0.09698250172289602
0.9826140224599815      0.09100191531851114   0.0964712314472377
0.9827860671604395      0.09088026005457905   0.09598381164869829
0.9829486772383665      0.09076656983244605   0.09573552794370581
0.9831185221386702      0.09064916860579358   0.09576715024766622
0.98328309977767        0.09053672620096426   0.09599287805424607
0.9834424101553657      0.09042912247010858   0.09621314286825527
0.9836110390360515      0.09031655833950479   0.0962539065971901
0.9837744006554333      0.09020882278742762   0.0960170393742164
0.9841103262776465      0.08999146161691825   0.0950611724706126
0.984270388196797       0.08988984664685037   0.09475321194361754
0.9844397686189377      0.08978368481451877   0.09471091840693892
0.9846038817797743      0.08968217142971674   0.09490516787592183
0.9847752297629877      0.08957760241285054   0.09517232707095692
0.9849475615267373      0.08947390056658874   0.09528181694179969
0.9851104586679562      0.08937723366729142   0.09511498112719839
0.9854454553338431      0.08918260377903259   0.0942075587356219
0.9856050527748307      0.08909185866197662   0.09386204274960044
0.9857739687188083      0.08899721271312039   0.09376603525709307
0.985937617401482       0.08890689311703978   0.0939293646973886
0.9861085009065322      0.08881402954378936   0.09421194928309615
0.986280368192119       0.0887221293611618    0.09438466641042045
0.9864428008551748      0.0886366593580135    0.09429507493335558
0.9866124683406072      0.08854882462582217   0.09393278775950302
0.9867768685647358      0.0884651267806143    0.0934594540689091
0.9869485036112408      0.08837923117993941   0.09305828500398279
0.9871211224382822      0.08829437878803416   0.09292516614121707
0.9872843066427928      0.08821558458670102   0.09306829699602183
0.9874547256696801      0.08813477548208033   0.09335864024021406
0.9876198774352634      0.08805790971994584   0.09356828772534515
0.9877797619395428      0.08798485533815815   0.09354605909406125
0.9879489649468122      0.08790900468987108   0.0932449908812694
0.9881129006927776      0.08783695312081781   0.09279370397158286
0.9882840712611197      0.08776323666283908   0.09236918558646905
0.9884562256099981      0.0876906623810359    0.0921850313171356
0.9886189453363456      0.08762351350614508   0.09228815180261797
0.9887888998850699      0.08755488626380263   0.09257617289768512
0.9889535871724902      0.08748985951654058   0.09283081051619525
0.9891130071986065      0.08742829825643728   0.09288044767596691
0.9892817457277129      0.08736462807156138   0.09265139969042056
0.9896159230856943      0.08724307166290353   0.09179261645930638
0.9897876129564097      0.08718297494732688   0.09155465045098493
0.9899498682045942      0.08712765584596577   0.09160754630178991
0.9902835810844126      0.08701840434464038   0.09217240888790582
0.9904550387160465      0.08696464924560202   0.09229457427347387
0.9906274801282167      0.08691221970374435   0.09211917704371646
0.990960728529872       0.08681555989249133   0.09129296275831923
0.9911257028805842      0.08676999130680557   0.09103135690617804
0.9912854099699923      0.08672732332085885   0.0910401897141312
0.9914544355623904      0.08668371937516813   0.09129187202949797
0.9916181938934846      0.08664300235319719   0.09160817882559062
0.9917891870469555      0.08660209621625355   0.09179680753667357
0.9919611639809626      0.08656261836903806   0.0916996331176378
0.9921237062924388      0.08652684437193577   0.09136387324672408
0.9922934834262919      0.08649107869367573   0.09092602039930626
0.9924579932988409      0.08645798723731776   0.09062477344730782
0.992617235910086       0.08642742622854015   0.09058118127748285
0.9927857970243211      0.08639665777905821   0.09079934481402711
0.9929490908772522      0.08636840557215622   0.09112995373229865
0.9931196195525601      0.08634053958421968   0.09138266601213023
0.993284880966564       0.08631513553852509   0.09137721957842171
0.9934448751192638      0.08629204693478494   0.09111666372276939
0.9936141877749538      0.08626923138127272   0.09069637822059375
0.9937782331693399      0.08624871675517173   0.09035118020269103
0.9939495133861024      0.08622897322397924   0.09023806041866589
0.9941217773834014      0.08621084796218788   0.09042280516004308
0.9942846067581694      0.08619531645678331   0.09075774798012935
0.9944546709553143      0.08618076074022493   0.09106067715432448
0.9946194678911551      0.0861682841478977    0.09112921952235656
0.994778997565692       0.08615773716982249   0.09093257118280773
0.9949478457432188      0.08614821906601565   0.09054153394145567
0.9951114266594417      0.08614061553325855   0.0901776596181121
0.9952822423980414      0.08613437958922937   0.09001319836338899
0.9954540419171772      0.08612986839052938   0.09015115120766898
0.9956164068137823      0.08612723232108638   0.09047917659061976
0.9957860065327641      0.08612617212491469   0.09082638196622775
0.9959503389904418      0.08612679975972554   0.09097229374038526
0.9961219062704962      0.08612919771372048   0.09083029528088676
0.996457573769147       0.08613905653225377   0.09009536247037517
0.9966279250295834      0.08614668550143625   0.08990431487030974
0.996793009028716       0.08615577131803788   0.09000470255700552
0.9971219610073634      0.08617886143601973   0.0906955851181656
0.9972858289868782      0.0861928498484881    0.09090739970348324
0.9974569317887695      0.08620922641563232   0.09084083529793491
0.9976290183711972      0.08622752645248398   0.09051845286226187
0.9977916703310941      0.08624651412041684   0.09015032523031945
0.9979615571133676      0.08626810554487664   0.08991733155023535
0.9981261766343372      0.08629074671710542   0.08996776827427089
0.9982855288940029      0.0863142793644403    0.09025594114001811
0.9984541996566585      0.08634092437680992   0.09065759660880394
0.9986176031580102      0.08636844445186162   0.09093496568265595
0.9987882414817386      0.08639898094979265   0.09095255397237813
0.9989536125441629      0.08643033208071697   0.09070296609399706
0.9991137163452832      0.0864623367775076    0.09034792931747938
0.9992831386493938      0.08649797879631191   0.09006503382639099
0.9994472936922003      0.08653425764861995   0.09004309102017742
0.9996186835573835      0.08657397272991528   0.09031016337735358
0.999791057203103       0.0866158137917927    0.09073375207499887
0.9999584877851642      0.08665828209108956   0.09107060915223611
1.666944490748458e-10   1.000000000001851     1.000000000001855
0.0001603274543882296   1.0000000000018556    1.000000000003555
0.0003341406062989621   1.0000000000018603    1.0000000000056366
0.0004964354550654196   1.0000000000018647    1.0000000000077092
0.0006555467231413446   1.0000000000018692    1.0000000000096905
0.0008281438554342214   1.0000000000018736    1.0000000000118514
0.0009892226845828235   1.0000000000018778    1.000000000013864
0.0011637873779483776   1.0000000000018823    1.000000000016046
0.001335168490623399    1.0000000000018865    1.0000000000181875
0.0014950313001541454   1.0000000000018903    1.0000000000201847
0.0016683799739018437   1.0000000000018943    1.0000000000223503
0.0018302103445052673   1.000000000001898     1.000000000024372
0.0019888571344181577   1.0000000000019016    1.0000000000263545
0.002160989788548       1.0000000000019054    1.0000000000285056
0.002321604139533568    1.000000000001909     1.0000000000305131
0.002495704354736088    1.0000000000019125    1.000000000032688
0.0026666209892480747   1.000000000001916     1.0000000000348233
0.002826019320615787    1.0000000000019191    1.000000000036814
0.0029989035162004507   1.0000000000019225    1.0000000000389733
0.0031602694086408398   1.0000000000019256    1.0000000000409883
0.003335121165298181    1.0000000000019287    1.0000000000431728
0.003506789341264989    1.0000000000019318    1.0000000000453173
0.0036669392140875226   1.0000000000019345    1.0000000000473181
0.003840574951127008    1.0000000000019373    1.0000000000494869
0.004002692385022219    1.0000000000019398    1.0000000000515112
0.004161626238226897    1.0000000000019422    1.0000000000534954
0.004334045955648527    1.0000000000019449    1.0000000000556473
0.0044949473699258825   1.000000000001947     1.0000000000576559
0.00466933464842019     1.0000000000019496    1.0000000000598337
0.0048405383462239646   1.0000000000019518    1.0000000000619722
0.005000223740883465    1.0000000000019538    1.000000000063967
0.005173394999759916    1.000000000001956     1.0000000000661302
0.0053350479554920924   1.0000000000019578    1.000000000068148
0.0054935173305337365   1.0000000000019595    1.0000000000701246
0.005665472569792333    1.0000000000019613    1.0000000000722695
0.005825909505906654    1.0000000000019629    1.0000000000742708
0.0059998323062379275   1.0000000000019647    1.0000000000764417
0.006162236803424925    1.000000000001966     1.00000000007847
0.006321457719921391    1.0000000000019673    1.0000000000804588
0.006494164500634808    1.0000000000019686    1.000000000082616
0.006655352978203951    1.0000000000019698    1.0000000000846276
0.0068300273199900455   1.0000000000019709    1.000000000086806
0.007001518081085608    1.000000000001972     1.0000000000889433
0.0071614905390368945   1.0000000000019729    1.0000000000909375
0.007334948861205133    1.0000000000019738    1.0000000000931009
0.007496888880229097    1.0000000000019744    1.0000000000951221
0.007655645318562529    1.000000000001975     1.000000000097105
0.007827887621112911    1.0000000000019755    1.0000000000992562
0.00798861162051902     1.000000000001976     1.0000000001012614
0.00816282148414208     1.0000000000019764    1.0000000001034333
0.008333847767074607    1.0000000000019766    1.0000000001055636
0.008493355746862859    1.0000000000019769    1.0000000001075504
0.008666349590868063    1.0000000000019769    1.0000000001097065
0.008827825131728992    1.0000000000019769    1.000000000111721
0.009002786536806872    1.0000000000019766    1.0000000001139053
0.00917456436119422     1.0000000000019764    1.00000000011605
0.009334823882437294    1.0000000000019762    1.0000000001180498
0.009508569267897319    1.0000000000019758    1.0000000001202147
0.00967079635021307     1.0000000000019753    1.000000000122234
0.009829839851838287    1.0000000000019746    1.0000000001242133
0.010002369217680458    1.000000000001974     1.0000000001263611
0.010163380280378352    1.0000000000019733    1.0000000001283684
0.010337877207293199    1.0000000000019724    1.000000000130546
0.010509190553517513    1.0000000000019713    1.0000000001326848
0.010668985596597551    1.0000000000019704    1.000000000134678
0.010842266503894542    1.000000000001969     1.000000000136837
0.011004029108047259    1.000000000001968     1.0000000001388498
0.011162608131509443    1.0000000000019666    1.0000000001408205
0.01133467301918858     1.000000000001965     1.0000000001429605
0.01149521960372344     1.0000000000019638    1.00000000014496
0.011669252052475253    1.000000000001962     1.0000000001471305
0.01183176619808279     1.0000000000019602    1.000000000149159
0.011991096762999795    1.0000000000019587    1.0000000001511467
0.012163913192133752    1.0000000000019567    1.0000000001532998
0.012325211318123433    1.0000000000019547    1.0000000001553055
0.012499995308330066    1.0000000000019524    1.0000000001574758
0.012671595717846167    1.0000000000019518    1.000000000159607
0.012831677824217994    1.0000000000019513    1.000000000161598
0.013005245794806773    1.0000000000019509    1.0000000001637608
0.013167295462251276    1.0000000000019502    1.0000000001657834
0.013326161549005247    1.0000000000019496    1.0000000001677654
0.01349851349997617     1.0000000000019487    1.0000000001699128
0.013659347147802817    1.0000000000019476    1.0000000001719116
0.013833666659846417    1.0000000000019464    1.000000000174074
0.014004802591199484    1.000000000001945     1.000000000176196
0.014164420219408277    1.0000000000019436    1.0000000001781777
0.01433752371183402     1.0000000000019418    1.0000000001803326
0.01449910890111549     1.00000000000194      1.0000000001823484
0.014657510509706427    1.000000000001938     1.0000000001843248
0.014829397982514316    1.0000000000019358    1.0000000001864664
0.01498976715217793     1.0000000000019333    1.000000000188459
0.015163622186058495    1.0000000000019307    1.0000000001906135
0.015325958916794785    1.0000000000019278    1.0000000001926228
0.015485112066840542    1.000000000001925     1.000000000194595
0.015657751081103254    1.0000000000019214    1.0000000001967402
0.01581887179222169     1.0000000000019178    1.0000000001987475
0.015993478367557077    1.0000000000019138    1.0000000002009262
0.01616490136220193     1.0000000000019096    1.000000000203063
0.01632480605370251     1.0000000000019054    1.0000000002050498
0.01649819660942004     1.0000000000019005    1.0000000002071967
0.016660068861993296    1.0000000000018956    1.0000000002091967
0.01681875753387602     1.0000000000018907    1.0000000002111584
0.016990932069975696    1.000000000001885     1.0000000002132932
0.017151588302931096    1.0000000000018792    1.0000000002152927
0.017325730400103448    1.0000000000018727    1.0000000002174652
0.01749668891658527     1.0000000000018658    1.000000000219596
0.017656129129922815    1.0000000000018594    1.0000000002215772
0.017829055207477313    1.0000000000018519    1.000000000223717
0.017990462981887535    1.0000000000018445    1.000000000225708
0.01816535662051471     1.000000000001836     1.0000000002278653
0.01833706667845135     1.0000000000018276    1.0000000002299896
0.018497258433243718    1.0000000000018192    1.0000000002319802
0.018670936052253038    1.0000000000018097    1.0000000002341451
0.018833095368118082    1.0000000000018001    1.0000000002361675
0.018992071103292592    1.0000000000017895    1.0000000002381433
0.019164532702684055    1.0000000000017775    1.0000000002402765
0.01932547599893124     1.000000000001766     1.0000000002422589
0.01949990515939538     1.000000000001753     1.000000000244404
0.019671150739168988    1.0000000000017397    1.0000000002465166
0.01983087801579832     1.000000000001727     1.0000000002484966
0.020004091156644605    1.0000000000017129    1.0000000002506537
0.020165785994346614    1.000000000001699     1.0000000002526699
0.020324297251358092    1.0000000000016853    1.0000000002546419
0.020496294372586522    1.00000000000167      1.000000000256769
0.020656773190670677    1.0000000000016551    1.0000000002587428
0.020830737872971784    1.0000000000016387    1.0000000002608767
0.020993184252128615    1.000000000001623     1.000000000262873
0.02115244705059491     1.000000000001607     1.0000000002648408
0.02132519571327816     1.0000000000015892    1.0000000002669878
0.021486426072817137    1.0000000000015723    1.000000000268998
0.021661142296573066    1.0000000000015534    1.0000000002711726
0.02183267493963846     1.0000000000015343    1.0000000002732947
0.02199268927955958     1.0000000000015161    1.0000000002752607
0.02216618948369765     1.0000000000014961    1.0000000002773828
0.022328171384691446    1.0000000000014768    1.0000000002793652
0.02248696970499471     1.0000000000014575    1.0000000002813199
0.022659253889514928    1.000000000001436     1.000000000283456
0.02282001977089087     1.0000000000014155    1.0000000002854585
0.022994271516483762    1.0000000000013929    1.0000000002876288
0.02316533968138612     1.0000000000013702    1.0000000002897471
0.023324889543144208    1.0000000000013485    1.0000000002917055
0.023497925269119247    1.0000000000013245    1.0000000002938165
0.02365944269195001     1.0000000000013016    1.0000000002957852
0.023834445978997726    1.0000000000012763    1.0000000002979292
0.024006265685354907    1.000000000001251     1.0000000003000535
0.024166567088567813    1.0000000000012268    1.0000000003020482
0.02434035435599767     1.0000000000012002    1.0000000003042138
0.024502623320283256    1.0000000000011748    1.0000000003062242
0.024661708703878307    1.0000000000011495    1.0000000003081768
0.02483427995169031     1.0000000000011213    1.0000000003102765
0.02499533289635804     1.000000000001095     1.0000000003122302
0.02516987170524272     1.000000000001067     1.000000000314357
0.025341226933436867    1.00000000000104      1.0000000003164666
0.02550106385848674     1.0000000000010145    1.000000000318452
0.025674386647753566    1.0000000000009863    1.000000000320612
0.025836191133876116    1.0000000000009595    1.000000000322619
0.025994812039308132    1.0000000000009326    1.000000000324567
0.0261669188089571      1.0000000000009028    1.000000000326657
0.026327507275461796    1.0000000000008744    1.0000000003285963
0.026501581606183443    1.000000000000843     1.0000000003307041
0.026672472356214557    1.0000000000008116    1.0000000003327953
0.026831844803101395    1.0000000000007816    1.0000000003347687
0.027004703114205185    1.0000000000007485    1.0000000003369218
0.0271660431221647      1.000000000000717     1.000000000338926
0.027340868994341166    1.000000000000682     1.000000000341075
0.0275125112858271      1.000000000000647     1.0000000003431544
0.02767263527416876     1.0000000000006135    1.000000000345078
0.027846245126727374    1.0000000000005766    1.0000000003471667
0.02800833667614171     1.0000000000005416    1.0000000003491372
0.028167244644865516    1.0000000000005063    1.0000000003510954
0.028339638477806274    1.0000000000004674    1.0000000003532403
0.028500514007602756    1.0000000000004303    1.0000000003552416
0.02867487540161619     1.0000000000003892    1.000000000357388
0.02884605321493909     1.0000000000003482    1.0000000003594605
0.029005712725117715    1.0000000000003089    1.0000000003613698
0.029178858099513292    1.0000000000002656    1.000000000363437
0.029340485170764596    1.0000000000002245    1.0000000003653857
0.029498928661325367    1.0000000000001832    1.0000000003673253
0.02967085801610309     1.0000000000001377    1.000000000369459
0.029831269067736536    1.0000000000000941    1.0000000003714562
0.030005165983586934    1.0000000000000462    1.0000000003736018
0.030167544596293058    1.0000000000000007    1.0000000003755702
0.03032673962830865     0.999999999999955     1.0000000003774687
0.030499420524541195    0.9999999999999045    1.0000000003795138
0.030660583117629463    0.9999999999998566    1.0000000003814353
0.030835231574934684    0.9999999999998035    1.0000000003835547
0.03100669645154937     0.9999999999997504    1.0000000003856735
0.031166643025019786    0.9999999999996999    1.000000000387665
0.03134007546270715     0.9999999999996443    1.0000000003898104
0.03150198959725024     0.9999999999995941    1.0000000003917757
0.03166072015110281     0.9999999999995439    1.000000000393664
0.03183293656917232     0.9999999999994883    1.0000000003956877
0.031993634684097556    0.9999999999994355    1.0000000003975824
0.03216781866323975     0.9999999999993769    1.0000000003996736
0.03233881906169141     0.9999999999993183    1.0000000004017737
0.03249830115699879     0.9999999999992625    1.000000000403757
0.03267126911652312     0.9999999999992008    1.000000000405902
0.03283271877290318     0.9999999999991419    1.0000000004078675
0.033007654293500194    0.9999999999990768    1.0000000004099445
0.03317940623340667     0.9999999999990117    1.0000000004119465
0.03333963987016887     0.9999999999989495    1.0000000004138145
0.03351335937114803     0.9999999999988808    1.0000000004158769
0.03367556056898291     0.9999999999988153    1.0000000004178529
0.03383457818612726     0.9999999999987499    1.0000000004198255
0.03400708166748856     0.9999999999986774    1.0000000004219691
0.034168066845705586    0.9999999999986084    1.0000000004239356
0.03434253788813956     0.9999999999985321    1.0000000004260063
0.034513825349883       0.9999999999984556    1.000000000427989
0.034673594508482175    0.9999999999983828    1.0000000004298268
0.034846849531298296    0.9999999999983022    1.0000000004318532
0.03500858625097014     0.9999999999982256    1.0000000004338008
0.03516713938995145     0.999999999998149     1.0000000004357572
0.03533917839314972     0.9999999999980642    1.0000000004378977
0.03549969909320371     0.9999999999979835    1.0000000004398673
0.03567370565747465     0.9999999999978944    1.0000000004419372
0.03583619391860132     0.9999999999978094    1.000000000443809
0.03599549859903746     0.9999999999977245    1.0000000004456175
0.036168289143690545    0.9999999999976307    1.0000000004475993
0.036329561385199355    0.9999999999975414    1.0000000004495073
0.03650431949092512     0.9999999999974427    1.000000000451644
0.03667589401596035     0.9999999999973439    1.000000000453778
0.03683595023785131     0.99999999999725      1.0000000004557499
0.03700949232395922     0.9999999999971462    1.0000000004578207
0.037171516106922854    0.9999999999970475    1.0000000004596796
0.037330356309195956    0.9999999999969489    1.0000000004614598
0.037502682375686006    0.99999999999684      1.0000000004633989
0.03766349013903179     0.9999999999967376    1.0000000004652654
0.03783778376659452     0.9999999999966265    1.00000000046737
0.03800889381346671     0.9999999999965152    1.0000000004694927
0.03816848555719463     0.9999999999964094    1.0000000004714673
0.03834156316513951     0.9999999999962925    1.0000000004735443
0.03850312246994011     0.9999999999961813    1.0000000004753973
0.038661498194050174    0.9999999999960703    1.0000000004771517
0.038833359782377196    0.9999999999959475    1.0000000004790446
0.03899370306755994     0.9999999999958308    1.0000000004808614
0.039167532216959636    0.9999999999957019    1.0000000004829233
0.039329843063215054    0.9999999999955792    1.0000000004849228
0.039488970328779946    0.9999999999954566    1.0000000004869
0.039661583458561786    0.9999999999953213    1.0000000004889922
0.03982267828519935     0.9999999999951925    1.0000000004908505
0.03999725897605387     0.9999999999950504    1.0000000004927692
0.040168656086217856    0.9999999999949082    1.000000000494616
0.040328534893237566    0.999999999994773     1.000000000496379
0.040501899564474225    0.9999999999946236    1.0000000004983909
0.040663745932566615    0.9999999999944817    1.000000000500363
0.04082240871996847     0.9999999999943401    1.0000000005023357
0.040994557371587276    0.9999999999941835    1.0000000005044378
0.041155187720061805    0.9999999999940349    1.0000000005063014
0.04132930393275329     0.9999999999938708    1.0000000005082033
0.04150023656475424     0.9999999999937066    1.0000000005100051
0.041659650893610914    0.9999999999935508    1.0000000005117078
0.041832551086684544    0.9999999999933787    1.0000000005136562
0.0419939329766139      0.9999999999932153    1.0000000005155887
0.0421688007307602      0.9999999999930349    1.0000000005177567
0.04234048490421597     0.9999999999928547    1.000000000519863
0.04250065077452747     0.9999999999926835    1.0000000005217273
0.04267430250905592     0.9999999999924947    1.0000000005236107
0.04283643594044009     0.9999999999923153    1.0000000005252814
0.04299538579113373     0.9999999999921365    1.0000000005269223
0.04316782150604433     0.9999999999919391    1.000000000528799
0.043328738917810646    0.9999999999917518    1.0000000005306826
0.04350314219379391     0.9999999999915453    1.00000000053283
0.04367436188908665     0.999999999991339     1.000000000534944
0.04383406328123512     0.9999999999911434    1.0000000005368224
0.04400725053760053     0.9999999999909337    1.0000000005387022
0.04416891949082167     0.9999999999907359    1.0000000005403367
0.04432740486335228     0.9999999999905387    1.0000000005419116
0.04449937610009984     0.999999999990321     1.0000000005437044
0.04465982903370312     0.9999999999901142    1.0000000005455214
0.04483376783152336     0.9999999999898863    1.0000000005476337
0.044996188326199324    0.9999999999896696    1.0000000005496494
0.04515542524018475     0.9999999999894535    1.000000000551554
0.04532814801838713     0.9999999999892151    1.0000000005534533
0.04548935249344524     0.9999999999889886    1.0000000005550673
0.0456640428327203      0.9999999999887389    1.0000000005567384
0.04583554959130482     0.9999999999884891    1.0000000005584402
0.04599553804674507     0.9999999999882522    1.00000000056018
0.04616901236640227     0.9999999999879907    1.0000000005622436
0.0463309683829152      0.9999999999877422    1.000000000564253
0.04648974081873759     0.9999999999874947    1.000000000566176
0.04666199911877694     0.9999999999872213    1.000000000568091
0.046822739115672016    0.9999999999869619    1.000000000569688
0.04699696497678404     0.9999999999866757    1.0000000005712921
0.047168007257205526    0.9999999999863897    1.000000000572893
0.047327531234482745    0.9999999999861184    1.000000000574536
0.04750054107597691     0.9999999999858191    1.0000000005765286
0.047662032614326805    0.999999999985535     1.0000000005785188
0.04783701001689365     0.9999999999852217    1.0000000005806577
0.048008803838769966    0.9999999999849087    1.0000000005825795
0.048169079357502004    0.9999999999846119    1.0000000005841547
0.04834284074045099     0.9999999999842845    1.000000000585691
0.04850508382025571     0.9999999999839736    1.000000000587116
0.04866414331936989     0.9999999999836638    1.0000000005886518
0.048836688682701024    0.9999999999833222    1.0000000005905565
0.04899771574288788     0.999999999982998     1.0000000005925136
0.04917222866729169     0.999999999982641     1.0000000005946659
0.04934355801100497     0.9999999999822843    1.0000000005966172
0.04950336905157397     0.9999999999819462    1.0000000005981942
0.04967666595635993     0.9999999999815735    1.0000000005996763
0.04983844455800161     0.9999999999812199    1.0000000006009964
0.04999703957895276     0.9999999999808679    1.0000000006024066
0.05016912046412086     0.9999999999804825    1.0000000006041938
0.050329683046144685    0.9999999999801269    1.000000000606094
0.05050373149238546     0.999999999979735     1.0000000006082503
0.050674596357935704    0.9999999999793435    1.0000000006102414
0.05083394292034167     0.9999999999789724    1.000000000611841
0.05100677534696459     0.999999999978563     1.0000000006132872
0.05116808947044324     0.9999999999781745    1.0000000006145044
0.051342889458138835    0.9999999999777461    1.0000000006159167
0.0515145058651439      0.9999999999773183    1.0000000006175855
0.051674603969004695    0.9999999999769122    1.000000000619421
0.051848187937082436    0.9999999999764646    1.000000000621567
0.0520102536020159      0.9999999999760394    1.00000000062349
0.05216913568625884     0.9999999999756158    1.0000000006251175
0.052341503634718727    0.9999999999751484    1.00000000062654
0.05250235328003434     0.9999999999747049    1.0000000006276613
0.052676688789566904    0.999999999974216     1.0000000006289094
0.05284784071840894     0.9999999999737275    1.000000000630411
0.05300747434410669     0.9999999999732642    1.0000000006321397
0.0531805938340214      0.9999999999727532    1.0000000006342538
0.053342195020791836    0.9999999999722683    1.0000000006362115
0.05350061262687174     0.9999999999717851    1.0000000006378869
0.05367251609716859     0.9999999999712521    1.0000000006393124
0.053832901264321165    0.9999999999707464    1.000000000640353
0.054006772295690696    0.9999999999701891    1.000000000641428
0.05416912502391595     0.9999999999696598    1.0000000006426553
0.05432829417145067     0.9999999999691326    1.000000000644215
0.05450094918320235     0.9999999999685513    1.0000000006462493
0.05466208589180975     0.9999999999679997    1.0000000006482366
0.0548367084646341      0.9999999999673921    1.0000000006501588
0.055008147456767915    0.9999999999667853    1.0000000006516006
0.05516806814575746     0.9999999999662101    1.0000000006525729
0.05534147469896396     0.9999999999655762    1.0000000006534768
0.05550336294902618     0.9999999999649746    1.0000000006544973
0.05566206761839786     0.9999999999643757    1.0000000006558765
0.055834258151986504    0.9999999999637155    1.0000000006578027
0.05599493038243087     0.9999999999630895    1.0000000006597887
0.056169088477092184    0.9999999999624001    1.0000000006617775
0.05634006299106297     0.9999999999617121    1.0000000006632679
0.05649951920188948     0.9999999999610729    1.0000000006642007
0.05667246127693294     0.9999999999603768    1.0000000006649419
0.056833885048832126    0.9999999999597161    1.0000000006657297
0.057008794684948266    0.9999999999589885    1.0000000006670275
0.05718052074037387     0.9999999999582616    1.0000000006688354
0.0573407284926552      0.9999999999575724    1.0000000006708005
0.05751442210915349     0.9999999999568127    1.0000000006728293
0.0576765974225075      0.9999999999560913    1.0000000006742882
0.05783558915517097     0.9999999999553729    1.0000000006751986
0.0580080667520514      0.9999999999545807    1.0000000006757876
0.05816902604578755     0.9999999999538288    1.0000000006763379
0.05834347120374066     0.9999999999530005    1.0000000006773566
0.05851473278100323     0.999999999952173     1.0000000006789695
0.05867447605512152     0.9999999999513886    1.0000000006808765
0.05884770519345677     0.9999999999505236    1.0000000006829604
0.059009416028647746    0.9999999999497027    1.0000000006845031
0.059167943283148186    0.9999999999488851    1.000000000685432
0.05933995640186558     0.9999999999479833    1.0000000006858987
0.0595004512174387      0.9999999999471281    1.0000000006862007
0.05967443189722877     0.9999999999461855    1.0000000006868868
0.059836894273874563    0.9999999999452907    1.0000000006881502
0.05999617306982983     0.9999999999443994    1.0000000006899223
0.060168937730002044    0.9999999999434168    1.0000000006920464
0.06033018408702998     0.9999999999424847    1.0000000006937264
0.06050491630827488     0.9999999999414579    1.00000000069483
0.06067646494882924     0.9999999999404328    1.0000000006951948
0.06083649528623932     0.999999999939461     1.0000000006952543
0.061010011487866354    0.9999999999383903    1.0000000006955871
0.06117200938634912     0.9999999999373743    1.0000000006965495
0.06133082370414135     0.9999999999363628    1.000000000698149
0.06150312388615053     0.999999999935248     1.0000000007002603
0.06166390576501543     0.999999999934191     1.0000000007020375
0.06183817350809729     0.999999999933027     1.0000000007032357
0.062009257670488614    0.9999999999318655    1.000000000703549
0.06216882352973566     0.9999999999307652    1.0000000007033807
0.06234187525319967     0.999999999929553     1.0000000007033205
0.06250340867351939     0.9999999999284037    1.000000000703903
0.06267842795805606     0.9999999999271415    1.000000000705425
0.0628502636619022      0.9999999999259157    1.0000000007074918
0.06301058106260407     0.9999999999247536    1.000000000709327
0.0631843843275229      0.9999999999234734    1.0000000007105818
0.06334666928929744     0.9999999999222586    1.0000000007108327
0.06350577067038145     0.999999999921049     1.0000000007104548
0.06367835791568241     0.9999999999197156    1.0000000007099945
0.0638394268578391      0.9999999999184509    1.0000000007101584
0.06401398166421274     0.9999999999170578    1.0000000007113412
0.06418535288989584     0.999999999915667     1.0000000007132936
0.06434520581243466     0.9999999999143486    1.0000000007152041
0.06451854459919044     0.9999999999128958    1.0000000007166079
0.06468036508280195     0.9999999999115171    1.0000000007168983
0.06483900198572293     0.9999999999101444    1.000000000716354
0.06501112475286086     0.999999999908631     1.0000000007154806
0.06517172921685452     0.9999999999071959    1.000000000715151
0.06534581954506512     0.9999999999056148    1.000000000715876
0.06551672629258519     0.9999999999040365    1.0000000007176173
0.06567611473696099     0.9999999999025411    1.00000000071957
0.06584898904555374     0.9999999999008928    1.000000000721163
0.06601034505100221     0.9999999998993294    1.000000000721574
0.06618518692066763     0.9999999998976078    1.000000000720827
0.06635684520964252     0.9999999998958892    1.0000000007195478
0.06651698519547314     0.9999999998942604    1.000000000718737
0.06669061104552071     0.9999999998924659    1.0000000007190006
0.066852718592424       0.9999999998907636    1.000000000720402
0.06701164255863677     0.9999999998890692    1.000000000722335
0.06718405238906648     0.9999999998872018    1.000000000724095
0.06734494391635192     0.9999999998854316    1.0000000007246512
0.0675193213078543      0.9999999998834826    1.0000000007238365
0.06769051511866617     0.9999999998815379    1.0000000007221663
0.06785019062633375     0.9999999998796958    1.0000000007207739
0.0680233519982183      0.999999999877667     1.0000000007203855
0.06818499506695856     0.9999999998757434    1.0000000007213754
0.06834345455500829     0.9999999998738296    1.0000000007232035
0.06851539990727497     0.9999999998717211    1.0000000007251293
0.06867582695639737     0.9999999998697235    1.00000000072591
0.06884973986973673     0.9999999998675249    1.000000000725142
0.06901213447993182     0.9999999998654678    1.0000000007232563
0.06917134550943636     0.9999999998634492    1.0000000007212295
0.06934404240315786     0.9999999998612252    1.0000000007199812
0.06950522099373509     0.9999999998591167    1.000000000720314
0.06967988544852927     0.9999999998567953    1.0000000007220606
0.06985136632263292     0.9999999998544787    1.0000000007240932
0.0700113288935923      0.9999999998522836    1.0000000007250973
0.07018477732876861     0.9999999998498655    1.000000000724426
0.07034670746080066     0.9999999998475718    1.000000000722294
0.07050545401214217     0.9999999998452889    1.0000000007196834
0.07067768642770064     0.9999999998427727    1.0000000007175958
0.07083840054011484     0.9999999998403875    1.0000000007172274
0.07101260051674597     0.9999999998377607    1.0000000007185716
0.07118361691268658     0.9999999998351394    1.000000000720644
0.07134311500548292     0.9999999998326561    1.000000000721922
0.07151609896249621     0.99999999982992      1.0000000007214818
0.07167756461636522     0.9999999998273256    1.000000000719223
0.07185251613445119     0.9999999998244694    1.0000000007157268
0.07202428407184662     0.9999999998216189    1.000000000712823
0.07218453370609777     0.9999999998189177    1.000000000711758
0.07235826920456588     0.999999999815943     1.0000000007126582
0.07252048639988971     0.9999999998131214    1.0000000007145828
0.07267952001452302     0.9999999998103132    1.0000000007160945
0.07285203949337328     0.9999999998072194    1.0000000007159164
0.07301304066907927     0.999999999804287     1.0000000007135987
0.07318752770900219     0.999999999801059     1.0000000007095293
0.07335883116823459     0.9999999997978388    1.0000000007056513
0.07351861632432272     0.9999999997947888    1.0000000007036127
0.0736918873446278      0.9999999997914304    1.0000000007037695
0.0738536400617886      0.9999999997882465    1.0000000007054994
0.07401220919825886     0.9999999997850793    1.0000000007072343
0.07418426419894608     0.9999999997815903    1.0000000007074428
0.07434480089648902     0.9999999997782854    1.0000000007052343
0.07451882345824892     0.999999999774648     1.0000000007007066
0.07468132771686455     0.9999999997711995    1.0000000006960268
0.07484064839478964     0.9999999997677691    1.0000000006927214
0.07501345493693168     0.9999999997639926    1.000000000691715
0.07517474317592944     0.9999999997604146    1.0000000006929544
0.07534951727914416     0.9999999997565677    1.0000000006950178
0.07552110780166835     0.9999999997527447    1.0000000006956025
0.07568118002104826     0.9999999997491233    1.0000000006935659
0.07585473810464512     0.9999999997451362    1.0000000006886687
0.07601677788509771     0.9999999997413556    1.0000000006830603
0.07617563408485976     0.9999999997375938    1.00000000067853
0.07634797614883877     0.9999999997334497    1.0000000006763035
0.0765087999096735      0.9999999997295222    1.000000000676901
0.0766831095347252      0.9999999997251988    1.0000000006789609
0.07685423557908634     0.9999999997208858    1.000000000679995
0.07701384332030321     0.9999999997168008    1.0000000006783214
0.07718693692573704     0.9999999997123017    1.0000000006732592
0.07734851222802659     0.9999999997080361    1.0000000006667802
0.07750690394962562     0.9999999997037922    1.0000000006608938
0.07767878153544158     0.9999999996991157    1.0000000006571348
0.07783914081811329     0.9999999996946849    1.0000000006567638
0.07801298596500193     0.9999999996898067    1.0000000006585943
0.0781753128087463      0.9999999996851803    1.0000000006600664
0.07833445607180015     0.9999999996805767    1.0000000006591188
0.07850708519907094     0.9999999996755061    1.0000000006543543
0.07866819602319747     0.9999999996707006    1.0000000006472394
0.07884279271154093     0.999999999665412     1.0000000006391871
0.07901420581919387     0.9999999996601366    1.0000000006337513
0.07917410062370253     0.9999999996551406    1.000000000632185
0.07934748129242815     0.9999999996496404    1.0000000006335354
0.07950934365800949     0.9999999996444264    1.0000000006352991
0.0796680224429003      0.99999999963924      1.0000000006349241
0.07984018709200806     0.9999999996335279    1.000000000630446
0.08000083343797154     0.9999999996281174    1.00000000062281
0.08017496564815198     0.9999999996221633    1.0000000006132548
0.08034591427764189     0.9999999996162269    1.0000000006058547
0.08050534460398752     0.999999999610608     1.0000000006026875
0.0806782607945501      0.9999999996044228    1.000000000603168
0.08083965868196841     0.999999999598563     1.0000000006050735
0.08101454243360366     0.9999999995921183    1.0000000006052274
0.08118624260454839     0.9999999995856933    1.0000000006008756
0.08134642447234884     0.9999999995796112    1.000000000592669
0.08152009220436625     0.9999999995729729    1.000000000581655
0.08168224163323938     0.9999999995667905    1.000000000572726
0.08184120748142197     0.9999999995606413    1.0000000005677323
0.08201365919382152     0.9999999995538699    1.000000000567022
0.08217459260307679     0.9999999995474544    1.0000000005688214
0.08234901187654901     0.9999999995403949    1.0000000005696386
0.0825202475693307      0.9999999995333544    1.000000000565933
0.08267996495896812     0.9999999995266879    1.0000000005575853
0.08285316821282249     0.999999999519348     1.0000000005452134
0.08301485316353258     0.9999999995123908    1.000000000534131
0.08317335453355214     0.9999999995054704    1.0000000005268748
0.08334534176778866     0.9999999994978469    1.0000000005244234
0.0835058106988809      0.9999999994906252    1.0000000005257665
0.08367976549419008     0.9999999994826763    1.0000000005272203
0.083842201986355       0.999999999475139     1.0000000005247471
0.08400145489782938     0.9999999994676402    1.0000000005169412
0.08417419367352072     0.9999999994593826    1.0000000005036502
0.08433541414606778     0.9999999994515575    1.0000000004902623
0.0845101204828318      0.9999999994429474    1.000000000479289
0.08468164323890527     0.9999999994343604    1.000000000474739
0.08484164769183447     0.9999999994262287    1.000000000475267
0.08501513800898063     0.9999999994172778    1.0000000004771064
0.0851771100229825      0.9999999994087935    1.0000000004755396
0.08533589845629386     0.9999999994003548    1.0000000004682181
0.08550817275382215     0.9999999993910622    1.0000000004541663
0.08566892874820618     0.9999999993822603    1.0000000004386964
0.08584317060680716     0.9999999993725758    1.0000000004246041
0.0860142288847176      0.9999999993629205    1.0000000004172862
0.08617376885948377     0.9999999993537818    1.0000000004164245
0.0863467946984669      0.9999999993437232    1.0000000004183691
0.08650830223430575     0.9999999993341939    1.0000000004178748
0.08668329563436154     0.9999999993237144    1.000000000410418
0.08685510545372681     0.9999999993132673    1.0000000003954825
0.0870153969699478      0.9999999993033775    1.0000000003779754
0.08718917435038574     0.9999999992924977    1.0000000003608847
0.08735143342767941     0.9999999992821891    1.000000000351139
0.08751050892428255     0.9999999992719406    1.0000000003484433
0.08768307028510264     0.9999999992606625    1.000000000350073
0.08784411334277845     0.999999999250159     1.0000000003504779
0.08801864226467121     0.9999999992386894    1.0000000003441094
0.08818998760587345     0.9999999992272549    1.000000000329032
0.0883498146439314      0.9999999992164308    1.0000000003096807
0.08852312754620631     0.9999999992045179    1.0000000002890501
0.08868492214533695     0.9999999991932291    1.0000000002756726
0.08884353316377705     0.9999999991820029    1.0000000002703544
0.0890156300464341      0.9999999991696406    1.0000000002710647
0.08917620862594688     0.9999999991579325    1.000000000272294
0.08935027306967662     0.9999999991450493    1.000000000267464
0.08952115393271581     0.9999999991322048    1.000000000252915
0.08968051649261073     0.9999999991200476    1.0000000002321436
0.0898533649167226      0.9999999991066635    1.0000000002078713
0.0900146950376902      0.999999999093983     1.0000000001901943
0.09018951102287476     0.999999999080034     1.000000000180868
0.09036114342736877     0.9999999990661251    1.0000000001803755
0.09052125752871852     0.9999999990529558    1.000000000181984
0.09069485749428521     0.9999999990384629    1.0000000001781992
0.09085693915670763     0.9999999990247275    1.000000000164841
0.09101583723843952     0.999999999011068     1.0000000001429785
0.09118822118438837     0.9999999989960292    1.0000000001152347
0.09134908682719294     0.9999999989817862    1.0000000000930573
0.09152343833421445     0.999999998966118     1.0000000000792606
0.09169460626054543     0.9999999989504992    1.0000000000764282
0.09185425588373214     0.9999999989357172    1.0000000000781633
0.0920273913711358      0.99999999891945      1.00000000007613
0.09218900855539519     0.9999999989040401    1.0000000000642058
0.09234744215896405     0.9999999988887205    1.00000000004197
0.09251936162674985     0.9999999988718551    1.0000000000110383
0.09267976279139138     0.9999999988558901    0.9999999999839079
0.09285364982024986     0.9999999988383296    0.9999999999645011
0.09301601854596407     0.9999999988216917    0.999999999958145
0.09317520369098775     0.9999999988051517    0.9999999999590897
0.09334787470022837     0.9999999987869522    0.9999999999591476
0.09350902740632473     0.9999999987697215    0.9999999999500144
0.09368366597663803     0.9999999987507786    0.999999999926123
0.09385512096626081     0.9999999987319044    0.999999999892368
0.0940150576527393      0.9999999987140896    0.9999999998602981
0.09418848020343476     0.9999999986948858    0.9999999998347824
0.09435038445098594     0.9999999986766942    0.9999999998241123
0.09450910511784658     0.9999999986586093    0.9999999998234536
0.09468131164892417     0.9999999986387021    0.9999999998246407
0.09484199987685749     0.9999999986198541    0.9999999998177206
0.09501617396900776     0.9999999985991224    0.9999999997949729
0.09518716448046749     0.999999998578459     0.9999999997591353
0.09534663668878296     0.999999998558906     0.9999999997220582
0.09551959476131537     0.9999999985373873    0.9999999996894092
0.09568103453070351     0.9999999985170044    0.9999999996729493
0.0958559601643086      0.9999999984945898    0.999999999669475
0.09602770221722316     0.9999999984722451    0.9999999996710863
0.09618792596699345     0.9999999984510926    0.9999999996656338
0.09636163558098068     0.9999999984278207    0.9999999996434584
0.09652382689182365     0.9999999984057691    0.9999999996075348
0.09668283462197608     0.9999999983838437    0.9999999995657185
0.09685532821634546     0.9999999983597104    0.9999999995256709
0.09701630350757057     0.9999999983368575    0.999999999502622
0.09719076466301263     0.999999998311724     0.9999999994949555
0.09736204223776415     0.999999998286674     0.9999999994964576
0.0975218015093714      0.9999999982629689    0.9999999994933934
0.0976950466451956      0.9999999982368876    0.999999999473707
0.09785677347787552     0.9999999982121835    0.9999999994373362
0.09801531672986492     0.9999999981876273    0.9999999993912456
0.09818734584607126     0.9999999981605984    0.9999999993431392
0.09834785665913334     0.9999999981350147    0.9999999993119413
0.09852185333641236     0.9999999981068792    0.9999999992981038
0.09868433171054711     0.9999999980802238    0.9999999992981077
0.09884362650399132     0.9999999980537279    0.9999999992977825
0.0990164071616525      0.9999999980245781    0.9999999992827371
0.0991776695161694      0.9999999979969816    0.9999999992484153
0.09935241773490323     0.9999999979666472    0.9999999991939648
0.09952398237294655     0.9999999979364252    0.9999999991378646
0.09968402870784558     0.9999999979078348    0.9999999990978944
0.09985756090696157     0.9999999978763967    0.9999999990766013
0.10001957480293329     0.9999999978466291    0.9999999990738747
0.10017840511821448     0.9999999978170522    0.9999999990749106
0.10035072129771261     0.9999999977848658    0.9999999990634457
0.10051151917406646     0.9999999977547355    0.9999999990310035
0.10068580291463727     0.9999999977216074    0.999999998973845
0.10085690307451756     0.9999999976885999    0.9999999989097137
0.10101648493125356     0.9999999976573747    0.9999999988596016
0.10118955265220651     0.9999999976230232    0.9999999988284726
0.1013511020700152      0.9999999975904927    0.9999999988209666
0.10150946790713335     0.9999999975581608    0.9999999988224209
0.10168131960846845     0.9999999975225726    0.9999999988151237
0.10184165300665927     0.9999999974888903    0.9999999987862184
0.10201547226906706     0.9999999974518444    0.9999999987282097
0.10217777322833056     0.9999999974167478    0.9999999986601983
0.10233689060690353     0.9999999973818589    0.9999999985981728
0.10250949384969345     0.999999997343467     0.999999998553218
0.10267057878933909     0.9999999973071176    0.9999999985370629
0.10284514959320169     0.9999999972671507    0.9999999985370341
0.10301653681637375     0.9999999972273234    0.9999999985331974
0.10317640573640155     0.9999999971896391    0.9999999985081397
0.10334976052064629     0.9999999971481861    0.999999998450328
0.10351159700174675     0.9999999971089263    0.9999999983763388
0.10367024990215669     0.9999999970699062    0.9999999983033823
0.10384238866678358     0.9999999970269652    0.999999998244786
0.1040030091282662      0.9999999969863235    0.9999999982189286
0.10417711545396575     0.9999999969416355    0.9999999982152434
0.10434803819897479     0.9999999968971153    0.9999999982145122
0.10450744264083954     0.9999999968550083    0.9999999981947074
0.10468033294692125     0.9999999968086907    0.9999999981395179
0.10484170494985869     0.9999999967648431    0.99999999806123
0.10501656281701308     0.9999999967166519    0.99999999796897
0.10518823710347693     0.9999999966686428    0.99999999789734
0.1053483930867965      0.9999999966232263    0.9999999978617787
0.10552203493433303     0.9999999965732923    0.9999999978534566
0.10568415847872528     0.9999999965260122    0.9999999978543215
0.10584309844242701     0.9999999964790367    0.9999999978394877
0.10601552427034568     0.9999999964273691    0.9999999977877838
0.10617643179512008     0.999999996378483     0.9999999977064586
0.10635082518411143     0.9999999963247613    0.9999999976025846
0.10652203499241225     0.999999996271265     0.9999999975144276
0.1066817264975688      0.999999996221648     0.9999999974644723
0.1068549038669423      0.9999999961671524    0.9999999974473552
0.10701656293317152     0.9999999961155601    0.9999999974483645
0.1071750384187102      0.9999999960642969    0.9999999974391425
0.10734699976846585     0.9999999960078901    0.9999999973934971
0.10750744281507721     0.9999999959545168    0.9999999973118013
0.10768137172590553     0.9999999958958325    0.9999999971975724
0.10784378233358957     0.9999999958402482    0.9999999970959228
0.10800300936058309     0.9999999957850053    0.9999999970260843
0.10817572225179355     0.9999999957242336    0.9999999969937208
0.10833691683985973     0.9999999956667055    0.9999999969914063
0.10851159729214287     0.9999999956034693    0.9999999969869158
0.10868309416373548     0.9999999955404668    0.9999999969476352
0.10884307273218381     0.9999999954808635    0.9999999968669013
0.10901653716484909     0.9999999954153155    0.9999999967437909
0.1091784832943701      0.9999999953532439    0.9999999966253635
0.10933724584320058     0.9999999952915606    0.9999999965361639
0.10950949425624801     0.9999999952236938    0.9999999964871762
0.10967022436615116     0.999999995159468     0.9999999964786805
0.10984444034027127     0.9999999950888624    0.9999999964779444
0.11001547273370084     0.9999999950185312    0.9999999964471226
0.11017498682398613     0.9999999949520179    0.9999999963707045
0.11034798677848838     0.9999999948788666    0.9999999962415606
0.11050946842984635     0.9999999948096215    0.9999999961064262
0.11068443594542128     0.9999999947335296    0.9999999959855932
0.11085621988030567     0.9999999946577327    0.99999999592026
0.11101648551204579     0.999999994586033     0.99999999590448
0.11119023700800286     0.9999999945072129    0.9999999959049856
0.11135247020081565     0.9999999944325852    0.9999999958822127
0.11151151981293791     0.9999999943584432    0.9999999958111166
0.11168405528927713     0.9999999942769064    0.9999999956778539
0.11184507246247208     0.9999999941997612    0.9999999955272106
0.11201957549988396     0.9999999941149951    0.9999999953811068
0.11219089495660532     0.9999999940305893    0.9999999952918107
0.1123506961101824      0.9999999939507872    0.9999999952626818
0.11252398312797644     0.9999999938630697    0.999999995262268
0.1126857518426262      0.9999999937800622    0.9999999952479034
0.11284433697658544     0.9999999936980845    0.999999995185596
0.11301640797476162     0.9999999936091443    0.9999999950525377
0.11317696066979352     0.999999993525011     0.9999999948883587
0.11335099922904238     0.9999999934325401    0.9999999947150741
0.11351351948514696     0.999999993344977     0.9999999946004641
0.11367285616056101     0.9999999932579753    0.999999994548814
0.11384567870019201     0.9999999931622988    0.9999999945414485
0.11400698293667874     0.9999999930717499    0.9999999945363569
0.11418177303738242     0.9999999929722486    0.9999999944806689
0.11435337955739557     0.9999999928731406    0.99999999434974
0.11451346777426444     0.9999999927793985    0.9999999941740364
0.11468704185535027     0.9999999926763369    0.9999999939741697
0.11484909763329182     0.9999999925787597    0.9999999938295066
0.11500796983054283     0.9999999924818124    0.9999999937538382
0.1151803278920108      0.9999999923751751    0.9999999937352222
0.11534116765033449     0.9999999922742753    0.9999999937344793
0.11551549327287514     0.9999999921633805    0.9999999936915815
0.11568663531472524     0.9999999920529374    0.9999999935680985
0.11584625905343109     0.9999999919485042    0.9999999933849467
0.11601936865635387     0.999999991833675     0.999999993158869
0.11618095995613238     0.9999999917249913    0.9999999929798028
0.11633936767522036     0.9999999916170306    0.9999999928729083
0.1165112612585253      0.9999999914982709    0.9999999928354264
0.11667163653868597     0.9999999913859422    0.999999992835462
0.11684549768306357     0.9999999912624826    0.9999999928070907
0.11700784052429691     0.999999991145601     0.9999999927046723
0.11716699978483971     0.9999999910294927    0.9999999925236723
0.11733964490959947     0.9999999909018268    0.9999999922750981
0.11750077173121495     0.9999999907810457    0.9999999920561388
0.11767538441704739     0.9999999906483558    0.9999999918951296
0.11784681352218929     0.9999999905162431    0.9999999918331202
0.1180067243241869      0.9999999903913435    0.9999999918289437
0.11818012099040148     0.9999999902540762    0.9999999918123946
0.11834199935347178     0.9999999901241858    0.9999999917240022
0.11850069413585156     0.9999999899952007    0.9999999915458212
0.11867287478244827     0.9999999898533882    0.9999999912785665
0.11883353712590072     0.9999999897192929    0.9999999910235389
0.11900768533357012     0.9999999895719918    0.9999999908163357
0.11917864996054899     0.9999999894271921    0.999999990719833
0.11933809628438358     0.999999989291187     0.9999999907043524
0.11951102847243512     0.9999999891416776    0.9999999906979887
0.1196724423573424      0.9999999890002187    0.9999999906282018
0.11984734210646661     0.999999988844829     0.9999999904369897
0.1200190582749003      0.9999999886900978    0.9999999901508696
0.1201792561401897      0.9999999885437777    0.9999999898618797
0.12035293986969607     0.9999999883829619    0.9999999896109493
0.12051510529605816     0.999999988230736     0.9999999894837248
0.12067408714172972     0.9999999880795258    0.9999999894504081
0.12084655485161823     0.9999999879132496    0.9999999894489047
0.12100750425836246     0.9999999877559472    0.9999999893965017
0.12118193952932364     0.9999999875831089    0.9999999892185728
0.1213531912195943      0.9999999874110088    0.9999999889224498
0.12151292460672068     0.9999999872482972    0.9999999885988735
0.121686143858064       0.9999999870694309    0.9999999882930138
0.12184784480626305     0.9999999869001593    0.9999999881170185
0.12200636217377157     0.9999999867320383    0.9999999880551962
0.12217836540549705     0.99999998654714      0.9999999880522957
0.12233885033407825     0.9999999863722725    0.9999999880190286
0.1225128211268764      0.9999999861801149    0.9999999878629773
0.12267527361653029     0.9999999859982122    0.999999987583667
0.12283454252549363     0.9999999858175329    0.9999999872316261
0.12300729729867392     0.9999999856189035    0.99999998686311
0.12316853376870994     0.999999985430999     0.9999999866203029
0.12334325610296291     0.9999999852246007    0.9999999865050202
0.12351479485652535     0.9999999850191196    0.9999999864927714
0.12367481530694352     0.999999984824867     0.999999986474488
0.12384832162157863     0.9999999846114096    0.9999999863411326
0.12401030963306947     0.9999999844094319    0.9999999860669372
0.12416911406386978     0.9999999842088741    0.9999999856911992
0.12434140435888705     0.9999999839883993    0.9999999852662882
0.12450217635076004     0.9999999837799266    0.9999999849590621
0.12467643420684997     0.9999999835509502    0.9999999847877652
0.12484750848224938     0.9999999833230718    0.9999999847545223
0.12500706445450452     0.9999999831077524    0.9999999847483435
0.12518010629097662     0.9999999828711651    0.9999999846443276
0.12534162982430444     0.9999999826476615    0.9999999843865897
0.12551663922184922     0.9999999824062392    0.999999983950535
0.12568846503870346     0.9999999821659126    0.9999999834722729
0.12584877255241342     0.9999999819387057    0.9999999831044992
0.12602256593034034     0.9999999816890736    0.999999982878467
0.12618484100512298     0.99999998145283      0.9999999828206059
0.12634393249921508     0.999999981218219     0.9999999828190381
0.12651650985752413     0.9999999809603117    0.9999999827417938
0.12667756891268891     0.9999999807163725    0.9999999825038312
0.12685211383207065     0.9999999804484191    0.999999982056551
0.12702347517076185     0.9999999801816707    0.999999981524805
0.12718331820630877     0.9999999799295178    0.9999999810816986
0.12735664710607264     0.9999999796524042    0.9999999807759671
0.12751845770269224     0.9999999793901978    0.9999999806732682
0.12767708471862133     0.9999999791298197    0.9999999806672663
0.12784919759876737     0.9999999788435276    0.9999999806188864
0.12800979217576913     0.9999999785728044    0.9999999804120492
0.12818387261698783     0.9999999782753821    0.9999999799682794
0.128354769477516       0.9999999779793357    0.9999999793902721
0.1285141480348999      0.9999999776995695    0.9999999788666593
0.12868701245650077     0.9999999773920738    0.9999999784636255
0.12884835857495736     0.9999999771012129    0.9999999782956221
0.1290231905576309      0.999999976781789     0.9999999782708084
0.1291948389596139      0.9999999764638277    0.9999999782396377
0.12935496905845262     0.9999999761632677    0.9999999780538557
0.1295285850215083      0.9999999758330536    0.9999999776083297
0.1296906826814197      0.9999999755206235    0.9999999770184526
0.12984959676064056     0.9999999752104212    0.9999999764169606
0.13002199670407838     0.9999999748694701    0.9999999759108404
0.13018287834437192     0.9999999745471007    0.9999999756649246
0.13035724584888242     0.9999999741930814    0.9999999756055195
0.13052842977270238     0.9999999738407925    0.9999999755932415
0.13068809539337806     0.9999999735079353    0.9999999754476976
0.1308612468782707      0.9999999731422539    0.9999999750280322
0.13102288006001905     0.9999999727964296    0.9999999744138971
0.13118132966107687     0.9999999724531863    0.9999999737364492
0.13135326512635165     0.9999999720759429    0.9999999731131348
0.13151368228848215     0.9999999717194353    0.9999999727658623
0.1316875853148296      0.9999999713312798    0.9999999726471165
0.13184997003803278     0.9999999709669971    0.999999972643384
0.13200917118054542     0.999999970605333     0.9999999725531672
0.13218185818727501     0.999999970207894     0.9999999721944908
0.13234302689086033     0.99999996983207      0.9999999715876046
0.1325176814586626      0.9999999694193847    0.9999999707666897
0.13268915244577437     0.9999999690086637    0.9999999700251341
0.13284910512974185     0.9999999686204968    0.9999999695658177
0.13302254367792626     0.9999999681940333    0.9999999693691767
0.1331844639229664      0.9999999677905901    0.9999999693548519
0.13334320058731602     0.9999999673900433    0.9999999693013468
0.1335154231158826      0.9999999669497531    0.9999999689930067
0.1336761273413049      0.9999999665334768    0.9999999683964561
0.13385031743094417     0.9999999660762657    0.9999999675145314
0.13402132393989288     0.9999999656212585    0.9999999666479983
0.13418081214569733     0.9999999651913346    0.9999999660537845
0.13435378621571872     0.9999999647189104    0.9999999657470634
0.13451524198259585     0.9999999642721027    0.9999999656994296
0.13469018361368992     0.9999999637815278    0.9999999656673841
0.13486194166409346     0.999999963293277     0.9999999653921796
0.13502218141135272     0.9999999628317984    0.9999999647967276
0.13519590702282894     0.9999999623248922    0.9999999638547062
0.13535811433116088     0.9999999618453348    0.9999999629168912
0.13551713805880228     0.9999999613692541    0.9999999621813254
0.13568964765066063     0.9999999608460798    0.9999999617460852
0.1358506389393747      0.9999999603514588    0.9999999616437706
0.13602511609230575     0.9999999598083723    0.9999999616317776
0.13619640966454624     0.9999999592679999    0.9999999614187713
0.13635618493364246     0.9999999587574668    0.999999960862563
0.13652944606695563     0.9999999581966781    0.9999999598889497
0.13669118889712453     0.999999957666372     0.9999999588388977
0.1368497481466029      0.9999999571400655    0.9999999579442329
0.1370217932602982      0.9999999565617107    0.9999999573443862
0.13718232007084924     0.9999999560151696    0.9999999571532284
0.13735633274561723     0.9999999554151041    0.9999999571417937
0.13751882711724095     0.999999954847539     0.9999999570134441
0.13767813790817415     0.9999999542842577    0.9999999565437513
0.13785093456332428     0.9999999536655668    0.999999955589656
0.13801221291533014     0.9999999530891273    0.9999999544464194
0.13818697713155295     0.9999999524572184    0.9999999532717044
0.13835855776708525     0.9999999518284992    0.9999999524921063
0.13851862009947327     0.9999999512344434    0.9999999521881637
0.13869216829607825     0.9999999505819894    0.999999952151471
0.13885419818953895     0.9999999499648901    0.9999999520758335
0.1390130445023091      0.9999999493523615    0.9999999516757004
0.13918537667929623     0.9999999486792567    0.9999999507428438
0.13934619055313907     0.9999999480429852    0.9999999495237634
0.13952049029119887     0.9999999473443433    0.9999999481680361
0.13969160644856812     0.9999999466492199    0.9999999471740318
0.1398512043027931      0.9999999459925251    0.9999999467139621
0.14002428802123504     0.999999945271099     0.9999999466152825
0.1401858534365327      0.999999944588896     0.9999999465841369
0.14034423527113982     0.9999999439117986    0.9999999462728895
0.1405161029699639      0.999999943167596     0.9999999453997387
0.1406764523656437      0.9999999424642909    0.9999999441353875
0.14085028762554044     0.9999999416919253    0.9999999426036733
0.14101260458229292     0.9999999409613176    0.9999999414133947
0.14117173795835486     0.9999999402361057    0.9999999407204697
0.14134435719863375     0.9999999394393183    0.999999940487793
0.14150545813576837     0.9999999386860917    0.9999999404787839
0.14168004493711994     0.9999999378592186    0.9999999402432003
0.14185144815778097     0.9999999370365813    0.999999939441012
0.14201133307529773     0.999999936259432     0.9999999381523961
0.14218470385703144     0.9999999354059308    0.9999999364626032
0.14234655633562088     0.999999934598881     0.9999999350379083
0.1425052252335198      0.9999999337979925    0.9999999341128435
0.14267737999563565     0.9999999329180413    0.9999999337190636
0.14283801645460723     0.9999999320865399    0.9999999336897207
0.14301213877779576     0.9999999311737443    0.9999999335454948
0.14318307752029377     0.9999999302658914    0.9999999328518838
0.1433424979596475      0.9999999294086154    0.9999999315826124
0.1435154042632182      0.9999999284671401    0.999999929761515
0.14367679226364463     0.9999999275773073    0.9999999280898184
0.14385166612828798     0.9999999266009252    0.9999999267939776
0.14402335641224082     0.9999999256298568    0.999999926237035
0.14418352839304938     0.9999999247171321    0.9999999261652134
0.1443571862380749      0.9999999237250533    0.9999999260745803
0.14451932577995613     0.9999999227869604    0.999999925503398
0.14467828174114683     0.9999999218560521    0.9999999242730019
0.14485072356655448     0.9999999208334105    0.9999999223471706
0.14501164708881786     0.999999919866938     0.9999999204403093
0.1451860564752982      0.9999999188060537    0.9999999188229215
0.145357282281088       0.9999999177507686    0.999999918009205
0.1455169897837335      0.9999999167540121    0.9999999178372485
0.14569018315059598     0.9999999156593145    0.9999999178015213
0.14585185821431418     0.999999914624319     0.9999999173610609
0.14601034969734183     0.9999999135972685    0.9999999162201874
0.14618232704458645     0.999999912468725     0.9999999142387691
0.14634278608868678     0.9999999114023669    0.9999999121075953
0.14651673099700407     0.9999999102315905    0.9999999101286522
0.1466791576021771      0.9999999091242742    0.9999999090175192
0.14683840062665957     0.9999999080253148    0.9999999086444609
0.147011129515359       0.9999999068181701    0.999999908621384
0.14717234010091415     0.9999999056771642    0.999999908354959
0.14734703655068626     0.9999999044248713    0.9999999072523857
0.14751854941976783     0.9999999031791853    0.9999999052456308
0.14767854398570512     0.9999999020025055    0.9999999029149559
0.14785202441585937     0.999999900710482     0.9999999005757028
0.14801398654286935     0.999999899488897     0.999999899114503
0.1481727650891888      0.9999998982767818    0.999999898509621
0.14834502949972522     0.9999998969452562    0.9999998984359201
0.14850577560711736     0.9999998956871474    0.9999998982828696
0.14868000757872643     0.99999989430628      0.9999998973461833
0.14885105596964499     0.999999892933043     0.9999998953816124
0.14901058605741926     0.9999998916363962    0.9999998928902998
0.1491836020094105      0.9999998902126288    0.9999998901758796
0.14934509965825746     0.9999998888670403    0.9999998882970575
0.14952008317132137     0.9999998873908021    0.9999998873221625
0.14969188310369475     0.9999998859227299    0.9999998871681444
0.14985216473292384     0.9999998845362164    0.9999998870783472
0.1500259322263699      0.9999998830144644    0.9999998862529155
0.15018818141667167     0.9999998815759645    0.9999998844224973
0.1503472470262829      0.9999998801490151    0.999999881804923
0.1505197985001111      0.9999998785977323    0.9999998787354392
0.15068083167079502     0.999999877138059     0.9999998764234171
0.1508553507056959      0.999999875536329     0.9999998750477619
0.15102668615990622     0.999999873943482     0.9999998747188972
0.15118650331097228     0.9999998724393012    0.9999998746882582
0.1513598063262553      0.9999998707878288    0.9999998740607078
0.15152159103839402     0.9999998692267417    0.9999998723661118
0.15168019216984222     0.9999998676779734    0.9999998696892747
0.15185227916550736     0.9999998659766413    0.9999998662852232
0.15201284785802824     0.9999998643693448    0.9999998634911762
0.15218690241476607     0.9999998626051422    0.9999998616063248
0.15235777339081336     0.9999998608507518    0.9999998609920194
0.15251712606371637     0.9999998591943223    0.999999860964068
0.15268996460083634     0.9999998573752868    0.9999998605530845
0.15285128483481203     0.9999998556561698    0.999999859073713
0.15302609093300468     0.9999998537698338    0.9999998560955027
0.1531977134505068      0.9999998518937807    0.9999998523797612
0.15335781766486467     0.9999998501218798    0.9999998491431588
0.15353140774343946     0.9999998481767208    0.9999998467764345
0.15369347951886997     0.9999998463378279    0.9999998458788603
0.15385236771360997     0.9999998445134423    0.9999998457797145
0.15402474177256692     0.999999842509741     0.999999845539524
0.1541855975283796      0.9999998406167118    0.999999844282955
0.15435993914840923     0.9999998385393793    0.9999998413774627
0.15453109718774832     0.9999998364737969    0.9999998374102056
0.15469073692394314     0.9999998345235857    0.9999998336709176
0.1548638625243549      0.9999998323825677    0.9999998306536584
0.1550254698216224      0.9999998303592726    0.9999998292853376
0.15518389353819936     0.9999998283524276    0.9999998290112032
0.15535580311899327     0.9999998261482667    0.9999998289108528
0.1555161943966429      0.999999824066686     0.9999998279359587
0.1556900715385095      0.9999998217824301    0.9999998252223884
0.15585243037723182     0.9999998196232616    0.9999998213229785
0.1560116056352636      0.9999998174815987    0.9999998170869644
0.15618426675751232     0.9999998151304197    0.9999998132623358
0.15634540957661677     0.9999998129094875    0.9999998111906688
0.15652003825993818     0.999999810473408     0.9999998105258514
0.15669148336256905     0.9999998080569326    0.999999810484448
0.15685141016205564     0.9999998058010433    0.9999998097656573
0.1570248228257592      0.9999998033250701    0.9999998072798487
0.15718671718631846     0.9999998009851344    0.999999803312701
0.1573454279661872      0.9999997986642147    0.9999997986597955
0.15751762461027288     0.9999997961154501    0.9999997941009559
0.1576783029512143      0.9999997937080446    0.9999997913303814
0.15785246715637266     0.9999997910664018    0.9999997901891448
0.15802344778084051     0.9999997884400553    0.9999997901123947
0.1581829101021641      0.999999785960799     0.9999997896650618
0.1583558582877046      0.9999997832389047    0.999999787521853
0.15851728817010083     0.9999997806669562    0.9999997836214988
0.15869220391671401     0.999999777845571     0.9999997780859623
0.1588639360826367      0.9999997750401197    0.9999997728540834
0.15902414994541508     0.999999772390809     0.9999997694307804
0.15919784967241044     0.9999997694831508    0.999999767805895
0.1593600310962615      0.9999997667347241    0.9999997676159917
0.15951902893942205     0.9999997640084085    0.9999997673737863
0.15969151264679954     0.9999997610148019    0.9999997655746582
0.15985247805103275     0.9999997581868859    0.9999997618034677
0.16002692931948292     0.9999997550843152    0.9999997559569412
0.16019819700724255     0.9999997519997579    0.9999997499720262
0.1603579463918579      0.9999997490877882    0.9999997456779737
0.16053118164069022     0.9999997458915525    0.9999997432903422
0.16069289858637825     0.9999997428713537    0.9999997428166739
0.16085143195137575     0.9999997398760628    0.9999997427299286
0.1610234511805902      0.9999997365868815    0.9999997413554307
0.16118395210666037     0.9999997334808769    0.9999997378677882
0.1613579388969475      0.9999997300730578    0.9999997318608748
0.16152040738409035     0.9999997268520987    0.9999997254701732
0.16167969229054266     0.9999997236575666    0.9999997201101581
0.16185246306121193     0.9999997201510968    0.9999997166052053
0.16201371552873692     0.999999716839081     0.9999997155483624
0.16218845386047887     0.9999997132067944    0.99999971548351
0.16236000861153027     0.9999997095964605    0.9999997144859472
0.1625200450594374      0.9999997061886402    0.9999997113187591
0.1626935673715615      0.9999997024497307    0.9999997052480313
0.1628555713805413      0.9999996989173764    0.9999996982630113
0.1630143918088306      0.9999996954380734    0.999999691941949
0.16318669810133682     0.9999996916412254    0.9999996873440821
0.16334748609069877     0.9999996880557559    0.9999996856091811
0.16352175994427767     0.9999996841226669    0.9999996854201252
0.16369285021716606     0.9999996802133381    0.9999996848132288
0.16385242218691018     0.9999996765236887    0.999999682110983
0.16402548002087125     0.9999996724741343    0.999999676181179
0.16418701955168805     0.9999996686483925    0.9999996687204561
0.16436204494672177     0.9999996644528025    0.9999996607157211
0.16453388676106498     0.999999660281822     0.9999996551068638
0.1646942102722639      0.9999996563436465    0.9999996527016533
0.1648680196476798      0.9999996520225765    0.9999996522931915
0.16503031071995142     0.9999996479387955    0.9999996519690814
0.1651894182115325      0.999999643888585     0.9999996497231725
0.16536201156733052     0.9999996394423775    0.9999996440178488
0.16552308661998427     0.9999996352428538    0.9999996361914214
0.16569764753685498     0.9999996306365452    0.9999996271366832
0.16586902487303515     0.999999626057765     0.9999996201892362
0.16602888390607104     0.9999996217357114    0.9999996167404033
0.1662022288033239      0.9999996169927788    0.9999996158391108
0.16636405539743246     0.9999996125115925    0.9999996157223892
0.1665226984108505      0.9999996080679683    0.9999996140303786
0.16669482728848548     0.9999996031893446    0.9999996087793455
0.16685543786297619     0.99999959858288      0.9999996007895928
0.16702953430168385     0.9999995935297897    0.9999995907477309
0.16720044715970098     0.9999995885079156    0.9999995823039552
0.16735984171457383     0.9999995837692839    0.9999995775208159
0.16753272213366363     0.9999995785688945    0.9999995758021797
0.16769408424960916     0.9999995736572992    0.9999995757297683
0.16786893222977164     0.9999995682717044    0.9999995743736355
0.1680405966292436      0.9999995629192604    0.9999995694180313
0.16820074272557126     0.9999995578673966    0.9999995612286395
0.16837437468611588     0.999999552325629     0.9999995502843857
0.16853648834351623     0.9999995470903498    0.9999995408923772
0.16869541842022606     0.9999995419000306    0.9999995346633489
0.16886783436115285     0.9999995362039193    0.9999995318999843
0.16902873199893537     0.9999995308264099    0.9999995316513914
0.1692031155009348      0.9999995249323188    0.9999995308366995
0.16937431542224374     0.9999995191426003    0.9999995265810974
0.1695339970404084      0.9999995136794133    0.9999995185773957
0.16970716452279        0.999999507685193     0.9999995069050668
0.16986881370202733     0.9999995020234443    0.9999994960413318
0.17002727930057412     0.9999994964103323    0.9999994880999581
0.17019923076333787     0.9999994902482345    0.9999994838902762
0.17035966392295734     0.9999994844311711    0.9999994831638538
0.17053358294679377     0.9999994780503366    0.9999994828065831
0.17069598366748592     0.999999472020986     0.9999994797188919
0.17085520080748753     0.9999994660423668    0.9999994724487878
0.1710279038117061      0.9999994594809383    0.9999994604848037
0.1711890885127804      0.9999994532845439    0.9999994481601715
0.17136375907807164     0.9999994464896643    0.999999437256491
0.17153524606267234     0.9999994397366793    0.9999994313937272
0.17169521474412877     0.9999994333632293    0.9999994299220751
0.17186866928980216     0.9999994263707953    0.9999994297833339
0.17203060553233127     0.9999994197651411    0.9999994274802232
0.17218935819416983     0.999999413215861     0.999999420851476
0.17236159672022536     0.9999994060270485    0.9999994087285767
0.1725223169431366      0.9999993992400757    0.9999993951963108
0.1726965230302648      0.9999993917966518    0.9999993821675188
0.17286754553670247     0.9999993844003139    0.9999993742239188
0.17302704973999586     0.9999993774218606    0.9999993715782857
0.1732000398075062      0.9999993697649173    0.9999993713889528
0.17336151157187227     0.9999993625338918    0.9999993699240388
0.1735364692004553      0.9999993546065464    0.9999993633379453
0.1737082432483478      0.999999346729004     0.999999350965837
0.17386849899309603     0.9999993392944643    0.9999993363166964
0.1740422406020612      0.999999331140423     0.9999993213616041
0.17420446390788208     0.9999993234379452    0.9999993118199038
0.17436350363301245     0.9999993158023817    0.9999993077219532
0.17453602922235978     0.9999993074241486    0.9999993071518622
0.17469703650856283     0.999999299515059     0.999999306332206
0.1748715296589828      0.9999992908442288    0.9999993007744534
0.17504283922871228     0.9999992822301884    0.9999992887516743
0.17520263049529747     0.9999992741039186    0.9999992732476535
0.17537590762609961     0.9999992651911883    0.9999992561283428
0.17553766645375748     0.9999992568181962    0.9999992441001437
0.17569624170072481     0.9999992485696125    0.9999992380322604
0.1758683028119091      0.9999992395170954    0.9999992366139697
0.1760288456199491      0.9999992309732174    0.9999992363036598
0.17620287429220607     0.9999992216041087    0.9999992320188823
0.17636538466131876     0.9999992127528455    0.9999992215341553
0.1765247114497409      0.9999992039779263    0.9999992057760703
0.17669752410238002     0.9999991943503244    0.99999918655616
0.17685881845187484     0.9999991852599637    0.9999991714692528
0.17703359866558663     0.9999991752943164    0.9999991619418371
0.17720519529860787     0.9999991653921882    0.9999991592879349
0.17736527362848484     0.9999991560480814    0.9999991591698717
0.17753883782257876     0.9999991457991014    0.9999991560246413
0.1777008837135284      0.9999991361185187    0.999999146476263
0.1778597460237875      0.9999991265221931    0.9999991305983633
0.17803209419826357     0.9999991159913209    0.999999109643606
0.17819292406959536     0.9999991060504622    0.9999990918179892
0.1783672398051441      0.9999990951506248    0.9999990792201758
0.17853837196000233     0.9999990843215564    0.9999990746844767
0.17869798581171628     0.9999990741055523    0.9999990743905098
0.17887108552764716     0.999999062898718     0.9999990724359609
0.17903266694043377     0.9999990523164836    0.9999990642505416
0.17919106477252986     0.9999990418280754    0.9999990487929893
0.1793629484688429      0.9999990303169611    0.999999026476726
0.17952331386201167     0.9999990194542648    0.9999990058269638
0.17969716511939737     0.999999007542618     0.9999989895795532
0.17985949807363882     0.999998996291684     0.9999989825384221
0.18001864744718973     0.999998985139775     0.9999989813147322
0.1801912826849576      0.9999989729054217    0.9999989805573202
0.1803523996195812      0.9999989613570728    0.9999989745022216
0.18052700241842173     0.9999989486987086    0.9999989586088041
0.18069842163657174     0.9999989361246608    0.9999989352824362
0.18085832255157747     0.999998924263434     0.9999989120152368
0.18103170933080015     0.9999989112564301    0.9999988920196307
0.18119357780687856     0.9999988989757006    0.9999988819860469
0.18135226270226643     0.9999988868061802    0.9999988793614558
0.18152443346187125     0.9999988734551895    0.9999988790898827
0.1816850859183318      0.9999988608579634    0.9999988746350058
0.1818592242390093      0.9999988471562913    0.999998860159277
0.18203017897899626     0.9999988335849643    0.9999988364425876
0.18218961541583895     0.9999988207859893    0.999998810756368
0.1823625377168986      0.9999988067475067    0.9999987866250836
0.18252394171481395     0.9999987934950036    0.9999987728020556
0.18269883157694627     0.999998778970663     0.9999987677848695
0.18287053785838805     0.9999987645421813    0.9999987675455307
0.18303072583668556     0.9999987509292121    0.9999987641336541
0.18320439967920002     0.9999987360020477    0.9999987506184349
0.1833665552185702      0.9999987219051341    0.9999987278019213
0.18352552717724988     0.9999987079335353    0.9999987000449785
0.18369798500014647     0.9999986926052372    0.9999986718872326
0.1838589245198988      0.9999986781379623    0.9999986539974481
0.18403334990386808     0.9999986622789914    0.9999986459850712
0.18420459170714684     0.9999986465259055    0.9999986452501162
0.18436431520728133     0.9999986316666879    0.9999986432609422
0.18453752457163275     0.9999986153701127    0.9999986318222828
0.18469921563283992     0.9999985999838356    0.9999986097046096
0.18485772311335655     0.9999985847362826    0.9999985803916084
0.18502971645809013     0.9999985680056551    0.9999985481364627
0.18519019149967944     0.9999985522193098    0.9999985254829337
0.1853641524054857      0.9999985349122497    0.9999985133626973
0.1855265950081477      0.9999985185670067    0.9999985112331005
0.18568585403011914     0.9999985023677714    0.999998510584837
0.18585859891630754     0.9999984845997492    0.999998502300197
0.18601982549935167     0.9999984678296726    0.9999984824179825
0.18619453794661275     0.9999984494512382    0.9999984492517284
0.1863660668131833      0.9999984311976027    0.9999984132260906
0.18652607737660956     0.9999984139802465    0.9999983857377587
0.18669957380425278     0.9999983951031717    0.9999983689633497
0.18686155192875173     0.9999983772816009    0.9999983646885572
0.18702034647256013     0.9999983596232584    0.9999983644752338
0.1871926268805855      0.9999983402539165    0.9999983584931844
0.18735338898546658     0.9999983219794154    0.9999983405673533
0.18752763695456462     0.9999983019517031    0.9999983071920228
0.18769870134297212     0.9999982820650524    0.9999982677346014
0.18785824742823534     0.9999982633148568    0.9999982349708961
0.18803127937771552     0.999998242796879     0.9999982123954053
0.18819279302405142     0.9999982235854036    0.9999982047772745
0.18836779253460428     0.9999982025366404    0.9999982042277997
0.1885396084644666      0.9999981816318934    0.9999981997030414
0.18869990609118464     0.9999981619125115    0.9999981830702012
0.18887368958211964     0.9999981402956564    0.9999981492482195
0.18903595476991036     0.9999981198849452    0.9999981087954012
0.18919503637701057     0.9999980996597784    0.9999980709368796
0.1893676038483277      0.9999980774767141    0.999998042214571
0.1895286530165006      0.9999980565432744    0.9999980304960364
0.18970318804889041     0.9999980336021859    0.9999980288320941
0.18987453950058972     0.9999980108188671    0.9999980262640694
0.19003437264914474     0.9999979893316251    0.9999980123400835
0.19020769166191673     0.9999979657717688    0.9999979797545374
0.19036949237154444     0.9999979435312154    0.9999979371851522
0.1905281095004816      0.9999979214948544    0.9999978941923309
0.19070021249363572     0.9999978973208068    0.9999978583407204
0.19086079718364557     0.9999978745141892    0.9999978411135793
0.19103486773787237     0.9999978495163084    0.9999978370205375
0.19120575471140863     0.9999978246934578    0.9999978360047385
0.19136512338180062     0.9999978012892404    0.9999978253847273
0.19153797791640956     0.9999977756240555    0.9999977953521182
0.19169931414787422     0.9999977514031564    0.9999977518007055
0.19187413624355584     0.9999977248649006    0.9999976992059513
0.19204577475854692     0.9999976985107395    0.9999976569229924
0.19220589497039373     0.9999976736552801    0.9999976344783793
0.19237950104645749     0.9999976464092298    0.9999976275630055
0.19254158881937697     0.9999976206891631    0.9999976272182521
0.1927004930116059      0.999997595207725     0.9999976195407483
0.1928728830680518      0.9999975672625505    0.9999975923982634
0.19303375482135343     0.9999975408992534    0.9999975486399095
0.193208112438872       0.9999975120120104    0.9999974913579224
0.19337928647570005     0.999997483331655     0.9999974411985211
0.1935389422093838      0.9999974562923392    0.9999974112613478
0.19371208380728452     0.9999974266511098    0.9999973992877469
0.19387370710204097     0.9999973986806254    0.9999973986151571
0.1940321468161069      0.9999973709764101    0.9999973939699794
0.19420407239438975     0.999997340592534     0.9999973709153668
0.19436447966952836     0.9999973121059186    0.999997328449702
0.1945383728088839      0.999997280995314     0.9999972675301942
0.19470074764509515     0.9999972516264918    0.9999972117678543
0.1948599389006159      0.9999972225309635    0.9999971718923479
0.1950326160203536      0.9999971906280086    0.9999971516171638
0.19519377483694703     0.9999971605277821    0.9999971485910013
0.19536841951775738     0.9999971275499654    0.9999971462145607
0.19553988061787722     0.9999970948060586    0.9999971271665398
0.19569982341485279     0.999997063930052     0.9999970866391785
0.1958732520760453      0.9999970300847629    0.9999970231010625
0.19603516243409355     0.9999969981399452    0.9999969602787275
0.19619388921145126     0.9999969664941376    0.9999969112845464
0.19636610185302592     0.9999969317871644    0.999996882469898
0.1965267961914563      0.9999968990482748    0.9999968757664291
0.19670097639410364     0.9999968631725686    0.9999968749268022
0.19687197301606044     0.9999968275544969    0.9999968606521626
0.19703145133487296     0.9999967939764968    0.9999968236919086
0.19720441551790244     0.9999967571631888    0.9999967592510975
0.19736586139778764     0.9999967224260045    0.9999966899349073
0.1975407931418898      0.9999966843738338    0.9999966258606843
0.19771254130530141     0.9999966465917325    0.9999965890976941
0.19787277116556876     0.9999966109623424    0.9999965783874081
0.19804648689005305     0.9999965719144208    0.9999965777930996
0.19820868431139307     0.9999965350575953    0.9999965677627443
0.19836769815204255     0.9999964985475664    0.9999965346763552
0.198540197856909       0.9999964585155502    0.9999964703806421
0.19870117925863115     0.9999964207533757    0.9999963955848008
0.19887564652457027     0.9999963793840546    0.9999963207410314
0.19904693020981884     0.9999963383164582    0.9999962726950938
0.19920669559192317     0.9999962996021002    0.9999962550246929
0.19937994683824442     0.9999962571702593    0.9999962533022982
0.1995416797814214      0.9999962171335189    0.9999962473462386
0.19970022914390786     0.9999961774821852    0.9999962195693354
0.19987226437061129     0.9999961340033107    0.9999961576059955
0.20003278129417043     0.9999960930048832    0.9999960787570085
0.2002067840819465      0.9999960480883224    0.9999959929798905
0.20036926856657833     0.9999960056963925    0.999995933884192
0.2005285694705196      0.9999959637108589    0.9999959050864883
0.20070135623867785     0.9999959180844522    0.9999958990604635
0.20086262470369182     0.9999958750484249    0.9999958970264978
0.20103737903292274     0.9999958279119392    0.9999958733910792
0.20120894978146311     0.999995781120615     0.9999958145739768
0.20136900222685922     0.9999957370065584    0.999995732914768
0.20154254053647228     0.9999956886635917    0.9999956371673958
0.20170456054294106     0.9999956430432382    0.9999955652608182
0.2018633969687193      0.9999955978586776    0.9999955252709479
0.2020357192587145      0.9999955483164343    0.9999955131999922
0.20219652324556542     0.9999955015910976    0.9999955125892013
0.2023708130966333      0.9999954504020542    0.999995495386421
0.20254191936701063     0.9999953995905501    0.999995442038565
0.2027015073342437      0.9999953516964221    0.9999953597763966
0.2028745811656937      0.9999952992004626    0.9999952550105266
0.20303613669399945     0.9999952496721772    0.9999951691370906
0.20319450864161465     0.9999952006219671    0.9999951152475056
0.2033663664534468      0.9999951468317483    0.9999950937846949
0.20352670596213468     0.9999950961128988    0.9999950926257408
0.20370053133503951     0.9999950405402771    0.9999950820623441
0.20386283840480007     0.999994988092572     0.9999950397889411
0.20402196189387012     0.9999949361459164    0.9999949615214927
0.2041945712471571      0.9999948792003879    0.9999948503519814
0.2043556622972998      0.9999948254897764    0.9999947491815133
0.20453023921165944     0.9999947666608531    0.9999946714464251
0.20470163254532858     0.9999947082695099    0.9999946385108062
0.20486150757585345     0.999994653229648     0.9999946344937267
0.20503486847059524     0.9999945929166377    0.9999946288779805
0.20519671106219278     0.9999945360136032    0.9999945941053507
0.20535537007309979     0.9999944796650778    0.999994519855259
0.20552751494822374     0.99999441788891      0.9999944042739538
0.20568814152020343     0.9999943596421099    0.9999942903302729
0.20586225395640007     0.9999942958404984    0.9999941940204758
0.20603318281190616     0.9999942325271222    0.9999941457614399
0.20619259336426798     0.9999941728692828    0.9999941356680463
0.20636548978084676     0.9999941074922168    0.9999941336766834
0.20652686789428126     0.9999940458340868    0.9999941077943261
0.20670173187193272     0.9999939783237569    0.9999940310490978
0.20687341226889364     0.9999939115710786    0.9999939111462772
0.20703357436271028     0.9999938489338109    0.9999937859387745
0.20720722232074387     0.9999937803118808    0.9999936730566916
0.2073693519756332      0.9999937155669253    0.9999936122331968
0.20752829804983197     0.9999936514536861    0.9999935935074838
0.2077007299882477      0.9999935811767311    0.9999935925201382
0.20786164362351917     0.9999935149072268    0.9999935744053573
0.20803604312300758     0.9999934423265842    0.9999935059156447
0.20820725904180545     0.9999933702962936    0.9999933860680587
0.20836695665745905     0.9999933024125763    0.9999932504568829
0.2085401401373296      0.9999932280255399    0.9999931175879396
0.20870180531405588     0.9999931578547007    0.9999930370833371
0.20886028691009165     0.9999930883733776    0.9999930055078262
0.20903225437034434     0.9999930121964131    0.9999930024528914
0.20919270352745276     0.9999929403793867    0.9999929920704795
0.20936663854877813     0.9999928617083769    0.9999929348223588
0.20952905526695922     0.9999927874715052    0.9999928260324069
0.2096882884044498      0.9999927139555492    0.9999926837121368
0.20986100740615732     0.9999926333835372    0.9999925294156944
0.21002220810472058     0.9999925573982829    0.9999924231567708
0.21019689466750077     0.9999924741906376    0.9999923683749269
0.21036839764959045     0.9999923916151907    0.9999923599669696
0.21052838232853585     0.9999923137883144    0.9999923550402237
0.2107018528716982      0.9999922285233063    0.9999923086420683
0.2108638051117163      0.9999921480880481    0.9999922058399628
0.21102257377104383     0.9999920684472632    0.9999920590177764
0.21119482829458833     0.9999919811530202    0.9999918869639174
0.21135556451498855     0.9999918988543978    0.9999917573849957
0.21152978659960572     0.9999918087250762    0.9999916801580594
0.21170082510353236     0.9999917192975537    0.9999916616342232
0.21186034530431472     0.9999916350410151    0.9999916602520581
0.21203335136931403     0.9999915427245224    0.9999916264103533
0.21219483913116907     0.9999914556671129    0.9999915337713333
0.21236981275724107     0.999991360364334     0.9999913690200473
0.21254160280262252     0.999991265800365     0.9999911812053922
0.2127018745448597      0.9999911766801849    0.999991030982623
0.21287563215131383     0.9999910790751251    0.9999909330348695
0.2130378714546237      0.9999909870066928    0.9999909038533871
0.213196927177243       0.9999908964406226    0.9999909027146011
0.21336946876407928     0.9999907973174691    0.9999908795597148
0.21353049204777128     0.9999907038640848    0.9999907977498419
0.21370500119568023     0.9999906015395664    0.9999906346946927
0.21387632676289864     0.9999905000133485    0.9999904327418254
0.21403613402697277     0.9999904043481792    0.9999902579127795
0.21420942715526387     0.9999902995464514    0.9999901308874622
0.21437120198041068     0.9999902007014664    0.9999900832638993
0.21452979322486698     0.999990102846052     0.9999900783711937
0.2147018703335402      0.9999899955883704    0.9999900655611922
0.2148624291390692      0.9999898944852582    0.9999899980710181
0.2150364738088151      0.9999897837611175    0.9999898424102294
0.21519900017541674     0.9999896792936305    0.9999896412646619
0.21535834296132786     0.9999895758583459    0.999989440483852
0.21553117161145594     0.9999894625227059    0.9999892758950955
0.21569248195843974     0.9999893556539463    0.9999891991897246
0.21586727816964046     0.999989238654906     0.9999891826620387
0.21603889080015068     0.9999891225650085    0.9999891767966228
0.21619898512751662     0.9999890131649927    0.9999891228250181
0.2163725653190995      0.9999888933363338    0.9999889765968022
0.21653462720753813     0.9999887803094054    0.9999887696992773
0.2166935055152862      0.9999886684151994    0.9999885475383121
0.21686586968725124     0.9999885457944823    0.9999883492878197
0.217026715556072       0.9999884302042595    0.9999882436361835
0.21720104728910972     0.9999883036423425    0.9999882110129085
0.2173721954414569      0.9999881780845528    0.9999882091277019
0.21753182529065979     0.9999880597992155    0.9999881705350171
0.21770494100407964     0.9999879302249157    0.9999880395668467
0.21786653841435522     0.9999878080445979    0.9999878327667562
0.21802495224394025     0.9999876871113483    0.9999875920422748
0.21819685193774224     0.9999875545728897    0.9999873577946137
0.21835723332839996     0.9999874296755531    0.9999872166113724
0.21853110058327463     0.9999872929126988    0.9999871593440773
0.21869344953500502     0.9999871639189909    0.9999871561081516
0.21885261490604488     0.9999870362345268    0.9999871357699909
0.2190252661413017      0.999986896354474     0.9999870309837672
0.21918639907341422     0.9999867645033101    0.9999868361534916
0.2193610178697437      0.999986620367101     0.9999865559425398
0.21953245308538266     0.9999864782239526    0.9999862878051393
0.21969236999787736     0.9999863443111583    0.9999861096333476
0.219865772774589       0.9999861976500527    0.999986022596175
0.22002765724815634     0.9999860593497414    0.999986011500312
0.22018635814103318     0.9999859224610366    0.9999860019016933
0.22035854489812698     0.9999857724599424    0.9999859177714849
0.2205192133520765      0.9999856310896618    0.9999857348305402
0.22069336767024295     0.9999854763072421    0.9999854454641393
0.22086433840771888     0.9999853227737018    0.9999851442124936
0.22102379084205054     0.99998517815812      0.9999849239628631
0.22119672914059915     0.999985019740575     0.9999847976266043
0.2213581491360035      0.9999848703848719    0.9999847703634789
0.22153305499562478     0.9999847069118546    0.9999847658870313
0.22170477727455554     0.9999845447392276    0.9999846956151669
0.221864981250342       0.999984391933408     0.9999845205899752
0.22203867109034545     0.9999842246008717    0.9999842224229203
0.2222008426272046      0.9999840667880643    0.9999839079533414
0.22235983058337322     0.9999839105813265    0.9999836462839142
0.2225323044037588      0.9999837394390417    0.9999834769016616
0.2226932599210001      0.9999835781295585    0.999983427043007
0.22286770130245834     0.9999834015480108    0.9999834246156074
0.22303895910322605     0.9999832263955118    0.9999833754951424
0.22319869860084948     0.9999830614070764    0.999983220480588
0.22337192396268987     0.9999828807109478    0.9999829248956927
0.223533631021386       0.9999827103452333    0.9999825862357802
0.22369215449939156     0.9999825417411052    0.9999822808120339
0.2238641638416141      0.9999823569949579    0.9999820593893162
0.22402465488069234     0.9999821829177896    0.9999819763604693
0.22419863178398755     0.9999819923411022    0.9999819679490372
0.22436109038413848     0.9999818126087399    0.9999819431940222
0.2245203654035989      0.9999816347225163    0.99998182131384
0.22469312628727625     0.9999814398832999    0.9999815456211407
0.22485436886780932     0.9999812562444274    0.99998119266009
0.22502909731255935     0.9999810552793673    0.9999808079840474
0.22520064217661886     0.9999808559691652    0.9999805330607894
0.2253606687375341      0.9999806682345732    0.9999804111139977
0.22553418116266627     0.9999804626924748    0.9999803883602684
0.22569617528465416     0.9999802696997254    0.999980377134371
0.22585498582595154     0.9999800792025205    0.9999802807487571
0.22602728223146587     0.999979870515843     0.9999800236443414
0.22618806033383593     0.9999796738713227    0.9999796620951903
0.22636232430042294     0.9999794586293963    0.9999792352653092
0.2265334046863194      0.999979245169402     0.9999789000925758
0.2266929667690716      0.9999790441412062    0.999978727736033
0.22686601471604076     0.9999788239848464    0.9999786785801263
0.22702754435986564     0.9999786164554793    0.9999786757015077
0.22720255986790747     0.9999783893685845    0.9999785935961849
0.22737439179525876     0.9999781641323962    0.9999783484540092
0.22753470541946577     0.9999779519373188    0.999977977813882
0.22770850490788974     0.9999777196270627    0.9999775142141083
0.22787078609316944     0.9999775005655259    0.9999771419041984
0.2280298836977586      0.9999772837696866    0.9999769152136198
0.2282024671665647      0.9999770463016666    0.9999768310148076
0.22836353233222653     0.9999768225077738    0.9999768279987068
0.22853808336210532     0.9999765775832221    0.9999767728196074
0.22870945081129357     0.9999763346825421    0.9999765568992375
0.22886929995733754     0.9999761059062775    0.9999761917982484
0.22904263496759847     0.9999758554056297    0.9999756962770611
0.22920445167471512     0.9999756192554768    0.9999752647356147
0.22936308480114126     0.9999753855811302    0.9999749730710227
0.22953520379178433     0.9999751295903277    0.9999748388668626
0.22969580447928312     0.9999748884104934    0.999974825432674
0.22986989103099886     0.9999746244268637    0.9999747958172215
0.2300407940020241      0.9999743626667449    0.9999746181337172
0.23020017866990505     0.9999741162037441    0.999974271483739
0.23037304920200294     0.9999738463074609    0.9999737547596416
0.23053440143095655     0.9999735919526874    0.9999732646792798
0.2307092395241271      0.999973313655543     0.999972869777122
0.23088089403660716     0.9999730376883298    0.9999726878109078
0.23104103024594294     0.9999727777736377    0.9999726585358854
0.23121465231949567     0.9999724932598028    0.9999726427323555
0.23137675608990413     0.9999722250554938    0.9999725039434993
0.23153567627962204     0.9999719596949013    0.9999721803845995
0.23170808233355691     0.9999716690815326    0.9999716516003488
0.2318689700843475      0.9999713953587657    0.9999711100562763
0.23204334369935506     0.9999710976105313    0.9999706335486336
0.23221453373367207     0.999970802392219     0.9999703793215796
0.2323742054648448      0.9999705244151744    0.9999703169430062
0.2325473630602345      0.9999702200705032    0.9999703121703041
0.2327090023524799      0.9999699332302568    0.9999702125469719
0.23286745806403478     0.9999696494459426    0.9999699249710245
0.2330393996398066      0.9999693385763122    0.9999693995289207
0.23319982291243416     0.9999690457526189    0.9999688137881065
0.23337373204927866     0.9999687252527538    0.9999682502300634
0.2335361228829789      0.9999684230785715    0.99996791782172
0.2336953301359886      0.9999681240824986    0.9999677946408612
0.23386802325321523     0.9999677966572245    0.9999677856940061
0.2340291980672976      0.999967488131333     0.9999677325181495
0.23420385874559693     0.9999671505564146    0.999967467831338
0.23437533584320572     0.9999668158318594    0.9999669540668945
0.23453529463767023     0.9999665006141453    0.9999663338457557
0.2347087392963517      0.9999661555444658    0.9999656889553793
0.23487066565188888     0.999965830286564     0.9999652677443428
0.23502940842673556     0.9999655084885937    0.9999650792608977
0.23520163706579916     0.9999651560377989    0.9999650497702165
0.2353623474017185      0.9999648240203041    0.9999650243136904
0.23553654360185478     0.999964460690797     0.9999648114811582
0.23570755622130055     0.9999641004791969    0.9999643273677721
0.23586705053760204     0.9999637613572622    0.9999636864164744
0.2360400307181205      0.9999633900716045    0.9999629625064683
0.23620149259549467     0.9999630402055233    0.9999624403540164
0.23637644033708577     0.9999626574869069    0.9999621505739006
0.23654820449798636     0.9999622780287328    0.9999620945157858
0.23670845035574267     0.9999619206803632    0.9999620828611304
0.23688218207771594     0.9999615295905823    0.9999619052494914
0.23704439549654494     0.9999611609576964    0.9999614710023748
0.2372034253346834      0.9999607962791274    0.9999608198744968
0.2373759410370388      0.9999603969743757    0.9999600272523709
0.23753693843624993     0.9999600208287701    0.9999594062292171
0.23771142169967802     0.9999596093288547    0.9999590146526713
0.23788272138241556     0.9999592014144323    0.9999589063582685
0.23804250276200883     0.99995881739824      0.9999589034303401
0.23821577000581906     0.9999583982794382    0.9999587788885821
0.238377518946485       0.9999580044715551    0.9999583942341549
0.23853608430646042     0.9999576149358148    0.9999577518276311
0.2387081355306528      0.9999571883361759    0.9999569016777357
0.23886866845170088     0.9999567865678402    0.9999561765518498
0.23904268723696592     0.9999563469428345    0.9999556617530287
0.2392051877190867      0.9999559325235202    0.9999554780889465
0.23936450462051692     0.9999555225386048    0.9999554635036049
0.2395373073861641      0.9999550736863285    0.9999554001542533
0.23969859184866701     0.9999546508080884    0.999955096924454
0.23987336217538688     0.9999541882305859    0.999954423196857
0.2400449489214162      0.999953729646647     0.9999535288371707
0.24020501736430125     0.9999532978505363    0.9999527074204064
0.24037857167140325     0.9999528252771867    0.9999520662582895
0.24054060767536098     0.9999523799001158    0.9999517922397961
0.2406994600986282      0.999951939333096     0.999951747502276
0.24087179838611236     0.9999514569134711    0.9999517187434647
0.24103261837045226     0.999951002523134     0.9999514786848372
0.24120692421900908     0.9999505053950665    0.9999508504417758
0.24137804648687539     0.9999500126169074    0.9999499316124667
0.24153765045159742     0.9999495487479578    0.9999490178893196
0.2417107402805364      0.9999490409973529    0.9999482346397012
0.24187231180633112     0.9999485625981771    0.99994784303103
0.2420306997514353      0.9999480894329372    0.9999477419964711
0.24220257356075642     0.9999475712465692    0.9999477342223883
0.24236292906693327     0.9999470833087408    0.9999475648935001
0.24253677043732708     0.9999465494118992    0.9999470076195903
0.2426990935045766      0.9999460462285866    0.9999461423317079
0.2428582329911356      0.9999455485097271    0.9999451495997468
0.24303085834191154     0.999945003643928     0.9999442008849495
0.2431919653895432      0.9999444904329488    0.999943644260411
0.24336655830139184     0.99994392909743      0.9999434319168463
0.24353796763254992     0.9999433727302682    0.9999434233701656
0.24369785866056373     0.9999428490087973    0.9999433130452828
0.2438712355527945      0.9999422759036622    0.9999428299039236
0.24403309414188099     0.9999417359429421    0.9999419854439197
0.24419176915027693     0.9999412019476146    0.9999409356857678
0.24436393002288984     0.9999406173166683    0.9999398481760836
0.24452457259235846     0.9999400691493036    0.9999391385046955
0.24469870102604405     0.9999394701317236    0.9999388045714767
0.2448696458790391      0.9999388764826134    0.9999387691092487
0.24502907242888985     0.99993831780016      0.9999387135620655
0.24520198484295758     0.9999377063144528    0.9999383241429286
0.24536337895388102     0.9999371303078494    0.9999375311271623
0.24553825892902142     0.9999365003899046    0.9999363277091627
0.2457099553234713      0.9999358760346164    0.9999351214277914
0.24587013341477693     0.9999352882406223    0.999934277509244
0.24604379737029947     0.9999346450990677    0.9999338285101419
0.24620594302267773     0.9999340390618934    0.9999337522084992
0.24636490509436548     0.9999334396729949    0.9999337315343354
0.2465373530302702      0.9999327835035833    0.9999334289174029
0.24669828266303062     0.9999321655467367    0.9999326976399201
0.246872698160008       0.9999314896299387    0.9999314757900889
0.24704393007629485     0.9999308197472275    0.9999301475725731
0.24720364368943742     0.999930189247922     0.9999291330068941
0.24737684316679695     0.9999294992640376    0.9999285122908375
0.2475385243410122      0.9999288492519154    0.9999283538841257
0.2476970219345369      0.9999282064500413    0.9999283498226621
0.24786900539227857     0.9999275026446939    0.9999281420441267
0.24802947054687596     0.9999268400052277    0.9999275017467462
0.2482034215656903      0.9999261151121202    0.9999262985991308
0.24836585428136038     0.9999254320037452    0.9999249373407072
0.2485251034163399      0.9999247564112035    0.9999237274465028
0.2486978384155364      0.999924016981231     0.9999228703265984
0.2488590551115886      0.9999233205892791    0.999922564522104
0.24903375767185776     0.9999225590565375    0.9999225407111588
0.24920527665143638     0.999921804377261     0.9999224097124207
0.24936527732787073     0.9999210940555217    0.9999218614840077
0.24953876386852203     0.9999203169139937    0.9999206965216086
0.24970073210602906     0.999919584794078     0.9999192648926921
0.24985951676284554     0.9999188608552246    0.9999178930362437
0.250031787283879       0.999918068426956     0.999916820256584
0.2501925395017682      0.9999173223594282    0.9999163583385908
0.2503667775838743      0.9999165064284802    0.9999162816970679
0.2505378320852899      0.9999156979906093    0.9999162193750089
0.2506973682835612      0.9999149381263834    0.9999157846708303
0.25087039034604947     0.999914109856847     0.9999146995948802
0.25103189410539345     0.9999133297763664    0.9999132315261066
0.2512068837289544      0.9999124769108453    0.9999115577739405
0.2513786897718248      0.9999116317508571    0.9999102976994693
0.25153897751155097     0.99991083620936      0.999909691664498
0.25171275111549407     0.9999099659829651    0.9999095488073672
0.2518750064162929      0.9999091460923293    0.9999095269838238
0.25203407813640116     0.9999083353410276    0.9999091951555026
0.2522066357207264      0.9999074480090732    0.9999082012500256
0.2523676750019073      0.9999066124770968    0.9999067220824234
0.2525422001473052      0.9999056988041262    0.9999049002751971
0.2527135417120126      0.9999047934568706    0.9999034041782087
0.2528733649735757      0.9999039414568386    0.9999025855888302
0.2530466740993558      0.9999030092984654    0.9999023157182487
0.25320846492199156     0.9999021312647459    0.9999023102037723
0.25336707216393684     0.9999012631108687    0.999902088755638
0.25353916527009907     0.9999003127886806    0.9999012248532574
0.253699740073117       0.9998994181695692    0.999899779309955
0.2538738007403519      0.9998984397278082    0.99989783834051
0.2540446778268962      0.9998974703181776    0.9998960956604648
0.2542040366102963      0.9998965582756713    0.9998950215615738
0.2543768812579133      0.9998955602766959    0.9998945640016021
0.25453820760238605     0.9998946204819177    0.9998945336497946
0.25471301981107575     0.9998935930050719    0.9998943810910991
0.25488464843907493     0.9998925749377586    0.999893608446579
0.25504475876392985     0.9998916168201638    0.9998921847185105
0.2552183549530017      0.9998905687972912    0.9998901438853915
0.2553804328389293      0.999889581602547     0.9998882776859085
0.25553932714416633     0.9998886055734212    0.9998869438898903
0.2557117073136203      0.9998875374280357    0.9998862662148392
0.25587256917993        0.9998865318825898    0.9998861679202892
0.25604691691045667     0.999885432401401     0.9998860986878431
0.2562180810602928      0.999884343173921     0.9998854829168192
0.2563777269069847      0.9998833184085826    0.9998841573555634
0.2565508586178935      0.9998821973679556    0.9998820687615962
0.25671247202565806     0.9998811417274589    0.9998799963543196
0.25687090185273204     0.9998800982159171    0.9998783732965164
0.257042817544023       0.999878959491247     0.9998774118933669
0.25720321493216963     0.9998778894412389    0.9998771857100895
0.25737709818453325     0.9998767192439861    0.9998771681180842
0.2575394631337526      0.999875616915086     0.9998767679284494
0.2576986445022814      0.9998745270750813    0.999875635709989
0.2578713117350272      0.9998733345985996    0.9998735951976374
0.2580324606646287      0.9998722119111619    0.9998713481932122
0.25820709545844717     0.9998709845384628    0.9998692202528061
0.2583785466715751      0.9998697685852539    0.9998679556621564
0.2585384795815588      0.9998686244509194    0.9998675590571316
0.25871189835575936     0.9998673729821445    0.9998675424174767
0.25887379882681566     0.9998661943520774    0.9998672816277374
0.25903251571718144     0.9998650291802034    0.9998663104388613
0.2592047184717642      0.999863754034184     0.9998643336808686
0.25936540292320265     0.9998625537980215    0.9998619649538415
0.25953957323885807     0.9998612414150939    0.9998595275453283
0.259710559973823       0.9998599413760383    0.9998579040604321
0.2598700284056436      0.9998587184305474    0.9998572653450712
0.2600429827016812      0.9998573805356681    0.9998571897774776
0.2602044186945745      0.9998561208327624    0.9998570582635229
0.26037934055168477     0.9998547439140211    0.999856153793966
0.26055107882810447     0.9998533798328795    0.9998542170902393
0.2607112988013799      0.9998520962346924    0.999851744240689
0.26088500463887226     0.9998506925003462    0.9998490460071681
0.26104719217322037     0.9998493704017902    0.999847180254732
0.26120619612687795     0.9998480634455397    0.9998462655392646
0.2613786859447525      0.9998466334486669    0.9998460738917767
0.26153965745948277     0.9998452874134499    0.9998460253694921
0.26171411483843        0.9998438159474762    0.9998453234167382
0.26188538863668664     0.9998423584249312    0.9998435336498855
0.262045144131799       0.9998409873087398    0.9998410244374072
0.26221838549112836     0.9998394876882049    0.999838059819779
0.2623801085473134      0.999838075705351     0.9998358149042655
0.26253864802280796     0.9998366801276172    0.999834551021545
0.26271067336251946     0.9998351529921187    0.9998341599748779
0.2628711803990867      0.9998337159691421    0.9998341524797377
0.26304517329987087     0.999832144875664     0.9998336707557586
0.2632076478975108      0.9998306656223578    0.9998322033737588
0.26336693891446017     0.9998292090449147    0.9998297760535894
0.26353971579562646     0.999827615709435     0.9998265970377258
0.2637009743736485      0.9998261158651846    0.9998239189395857
0.26387571881588745     0.9998244765916746    0.9998220561780463
0.2640472796774359      0.9998228528931307    0.9998214116142881
0.2642073222358401      0.9998213253310279    0.9998213853350626
0.26438085065846123     0.9998196548854347    0.9998210821289616
0.2645428607779381      0.9998180819052377    0.9998198278308373
0.26470168731672444     0.9998165271566588    0.9998174924778415
0.26487399971972775     0.9998148260825529    0.9998141665575698
0.2650347938195868      0.9998132251731883    0.9998111337044465
0.26520907378366276     0.999811475105363     0.9998087961328893
0.2653801701670482      0.999809741815408     0.9998078009025512
0.2655397482472893      0.999808111536346     0.9998076825586603
0.2657128121917474      0.9998063284442175    0.9998075388055118
0.26587435783306124     0.999804649794545     0.9998065455061401
0.26603271989368454     0.9998029907846145    0.9998043866042284
0.2662045678185248      0.9998011753248149    0.999800996423322
0.2663648974402208      0.9997994672015792    0.9997976322810163
0.26653871292613374     0.9997975996352431    0.9997947663756491
0.2667010101089024      0.9997958409067638    0.9997933517522495
0.2668601237109805      0.9997941025839266    0.999793002332101
0.26703272317727555     0.999792201032856     0.999792975339492
0.26719380434042633     0.9997904113490452    0.9997923265490799
0.26736837136779407     0.9997884553221964    0.9997902343476416
0.2675397548144713      0.9997865181376543    0.9997868408689047
0.26769961995800423     0.9997846960047063    0.9997832035602924
0.26787297096575413     0.9997827035229797    0.9997798329171241
0.26803480367035976     0.9997808276966427    0.9997779434203785
0.26819345279427487     0.999778973918946     0.9997773117648732
0.2683655877824069      0.9997769458128101    0.9997772881702142
0.2685262044673946      0.9997750375860164    0.9997768776351843
0.2687003070165993      0.9997729517558709    0.9997750908891261
0.2688712259851135      0.9997708863529822    0.9997717996098615
0.2690306266504834      0.9997689442127782    0.9997679537079972
0.26920351318007024     0.9997668202650798    0.9997640671111153
0.2693648814065128      0.999764821285473     0.9997616152859169
0.26953973549717236     0.9997626407660485    0.9997605381689637
0.2697114060071414      0.9997604859966115    0.9997604588774607
0.26987155821396613     0.9997584591275304    0.9997601937281557
0.27004519628500784     0.9997562432401063    0.9997586236119839
0.27020731605290527     0.9997541569726821    0.999755597729641
0.2703662522401121      0.9997520952544618    0.9997516122881541
0.27053867429153594     0.9997498400630329    0.9997472653007523
0.2706995780398155      0.9997477179846662    0.9997442502839903
0.270873967652312       0.9997453987666282    0.9997426772044103
0.27104517368411796     0.9997431022159896    0.9997424327781392
0.27120486141277966     0.9997409424614091    0.9997423282374197
0.2713780350056583      0.9997385808372419    0.9997411099842792
0.2715396902953927      0.9997363578615814    0.9997383292583388
0.2716981620044365      0.9997341612565952    0.9997343045417842
0.2718701195776973      0.9997317580750699    0.9997295374860445
0.2720305588478138      0.9997294972871958    0.9997259059149527
0.27220448398214725     0.9997270260488477    0.9997237031986753
0.2723668908133364      0.9997246991468407    0.9997231577773898
0.2725261140638351      0.9997223996061725    0.999723149424248
0.2726988231785507      0.9997198847161138    0.9997223830615286
0.27286001399012205     0.9997175180826342    0.9997200504495194
0.27303469066591035     0.9997149320684944    0.9997157114016433
0.2732061837610081      0.9997123713920566    0.999710603878899
0.27336615855296154     0.9997099630928785    0.999706391543972
0.27353961920913195     0.999707330228677     0.999703525153062
0.2737015615621581      0.9997048518170926    0.9997025897869853
0.2738603203344937      0.9997024028922997    0.9997025502293937
0.2740325649710463      0.9996997242517891    0.9997020850127849
0.2741932913044546      0.9996972042359403    0.9997001223822928
0.27436750350207983     0.9996944502658768    0.9996959643178712
0.2745385321190146      0.9996917236838365    0.9996906092332337
0.27469804243280505     0.9996891601047116    0.9996858122684731
0.27487103861081247     0.99968635713022      0.99968217217199
0.2750325164856756      0.9996837193655168    0.9996806929733311
0.2752074802247557      0.9996808378061349    0.9996805009156096
0.27537926038314525     0.9996779847367849    0.9996802119817345
0.2755395222383905      0.9996753014394741    0.9996785052135507
0.2757132699578527      0.9996723687046649    0.9996744734778277
0.27587549937417066     0.9996696166745629    0.9996692012791097
0.2760345452097981      0.9996668984467797    0.9996638890345011
0.27620707690964247     0.9996639258956493    0.9996594843563178
0.2763680903063426      0.9996611292333099    0.9996573941209539
0.27654258956725963     0.9996580735355838    0.9996569273993733
0.2767139052474861      0.9996550482878641    0.9996568242344592
0.27687370262456834     0.9996522036633176    0.9996555258558234
0.2770469858658675      0.9996490939250054    0.9996518509638563
0.2772087508040224      0.999646167191165     0.9996465491004544
0.2773673321614868      0.9996432756678474    0.9996407758352231
0.27753939938316813     0.9996401129820136    0.999635543620936
0.2776999483017052      0.9996371381145785    0.9996326919800408
0.2778739830844592      0.9996338871064626    0.9996317688760569
0.27803649956406895     0.999630826413819     0.9996317598664756
0.2781958324629881      0.9996278022058465    0.9996309752828444
0.27836865122612425     0.9996244955621628    0.999627927695829
0.2785299516861161      0.9996213842725159    0.9996228571054461
0.2787047380103249      0.999617985367143     0.9996161004942214
0.2788763407538432      0.999614620346804     0.9996100960283759
0.27903642519421723     0.9996114559763426    0.999606454874115
0.2792099954988082      0.999607997328814     0.9996049629767351
0.2793720475002549      0.9996047419984094    0.9996048916710847
0.2795309159210111      0.9996015258854479    0.9996044392661824
0.27970327020598423     0.9995980088774306    0.9996018960834874
0.2798641061878131      0.9995947005480978    0.9995970728093917
0.2800384280338589      0.9995910858778836    0.9995900596556554
0.2802095662992142      0.9995875077291039    0.9995832853772083
0.28036918626142515     0.9995841438852375    0.9995787343448521
0.2805422920878531      0.9995804667097664    0.9995764694649776
0.28070387961113674     0.9995770066751511    0.9995761727597698
0.2808622835537299      0.9995735888317773    0.9995759985206297
0.28103417336054        0.9995698507187173    0.9995740496469988
0.2811945448642058      0.9995663354282315    0.999569651415882
0.2813684022320886      0.9995624941587636    0.999562567061001
0.2815307412968271      0.9995588786654285    0.9995554293198907
0.281689896780875       0.9995553069814089    0.9995497682539668
0.2818625381291399      0.9995514021304167    0.9995463639708708
0.28202366117426053     0.9995477309937252    0.9995455374006254
0.2821982700835981      0.9995437313900284    0.9995455005267155
0.28236969541224516     0.9995397724208848    0.9995440764093619
0.28252960243774794     0.9995360503877038    0.9995401428747954
0.2827029953274677      0.9995319824891536    0.9995331189664057
0.28286486991404314     0.9995281545651005    0.9995254554215028
0.28302356091992803     0.9995243733500261    0.9995188640842113
0.2831957377900299      0.9995202385567151    0.999514391104227
0.28335639635698745     0.999516349879404     0.9995129344640399
0.283530540788162       0.999512101276252     0.9995128753943103
0.283701501638646       0.999507896178083     0.9995119874076354
0.28386094418598573     0.9995039436571616    0.9995086697728767
0.2840338725975424      0.9994996230402402    0.9995019328395913
0.28419528270595484     0.9994955582537566    0.9994938936545752
0.2843701786785842      0.9994911187647529    0.9994856841807296
0.2845418910705231      0.99948672431865      0.9994802864759806
0.28470208515931766     0.999482592460347     0.9994782239594581
0.2848757651123292      0.9994780774046148    0.9994780203264444
0.28503792676219647     0.9994738283401313    0.9994775329051521
0.28519690483137317     0.9994696311227983    0.999474807904048
0.2853693687647668      0.9994650422687603    0.9994684546358406
0.2855303143950162      0.9994607262463396    0.9994601921973814
0.2857047458894825      0.999456011642829     0.9994510673920983
0.28587599380325834     0.9994513454504875    0.9994444424544543
0.2860357234138899      0.9994469592670714    0.9994414299535881
0.2862089388887384      0.9994421655889903    0.999440838073847
0.2863706360604426      0.9994376555377096    0.9994406700191215
0.28652914965145626     0.999433201130707     0.9994386275989051
0.28670114910668687     0.9994283303828994    0.9994328878845508
0.2868616302587732      0.9994237505092096    0.9994246224708547
0.2870355972750765      0.9994187470154388    0.9994146860128963
0.2871980459882355      0.9994140381632273    0.9994070405366211
0.287357311120704       0.9994093870212665    0.9994026571310904
0.28753006211738946     0.999404303095291     0.9994012519974221
0.28769129481093064     0.9993995213483851    0.9994012474867804
0.2878660133686888      0.9993942992777611    0.9993997633302478
0.28803754834575634     0.9993891312499658    0.9993946709257844
0.28819756501967964     0.9993842732927859    0.9993865420968436
0.2883710675578199      0.9993789749391153    0.9993759730099214
0.28853305179281585     0.999373995282435     0.9993671543038136
0.2886918524471213      0.999369077257148     0.9993615116143388
0.2888641389656437      0.9993637006926824    0.9993591916754018
0.28902490718102186     0.9993586449151914    0.9993590838896691
0.28919916126061695     0.9993531225584592    0.9993582205189532
0.28937023175952153     0.9993476577796165    0.9993539651525557
0.28952978395528184     0.999342521979122     0.9993462324045129
0.2897028220152591      0.999336909252689     0.9993352417152951
0.2898643417720921      0.9993316296291668    0.9993252622994663
0.29003934739314197     0.9993258647133184    0.999317642784573
0.29021116943350134     0.9993201593321439    0.9993144701570867
0.29037147317071643     0.9993147956184671    0.9993141399220026
0.2905452627721485      0.999308935856477     0.9993136343131783
0.29070753407043626     0.9993034220759308    0.9993102460494694
0.2908666217880335      0.9992979764546183    0.9993030032402134
0.29103919536984774     0.9992920241002731    0.9992917848051298
0.2912002506485177      0.9992864263952901    0.9992808048140069
0.2913747917914046      0.9992803131653951    0.9992716307808015
0.2915461493536009      0.9992742637489787    0.9992671393800965
0.29170598861265296     0.999268578075374     0.999266286611164
0.29187931373592196     0.9992623655936227    0.9992661354871587
0.2920411205560467      0.9992565214386709    0.9992636084149117
0.2921997437954809      0.9992507502618495    0.9992571125031877
0.2923718528991321      0.9992444410797715    0.9992459620977744
0.292532443699639       0.9992385094049127    0.9992341159165647
0.29270652036436284     0.9992320305111689    0.999223280297281
0.29287741344839613     0.9992256200969228    0.9992171521285429
0.29303678822928514     0.9992195968076358    0.9992154354613609
0.2932096488743911      0.9992130144935942    0.9992154126042372
0.2933709912163528      0.9992068241561849    0.9992137832278569
0.29354581942253144     0.9992000652493415    0.9992074480805218
0.29371746404801957     0.999193377335177     0.9991963384880085
0.29387759037036343     0.9991870913942964    0.9991838004107638
0.29405120255692424     0.9991802246630173    0.9991715908448701
0.2942132964403408      0.9991737649490273    0.9991642744096506
0.2943722067430668      0.9991673863123977    0.9991614873311278
0.2945446029100098      0.9991604164423534    0.9991613007591021
0.2947054807738085      0.9991538789901817    0.9991603728263333
0.2948798445018241      0.9991467400090672    0.9991551070452076
0.2950510246491492      0.9991396768118956    0.9991445324353753
0.29521068649333        0.9991330397916319    0.9991315294112159
0.29538383420172776     0.999125788226819     0.9991177852263502
0.29554546360698125     0.9991189680365834    0.9991086285141436
0.29570390943154423     0.999112234021492     0.9991043974928573
0.29587584112032417     0.9991048725285301    0.9991036873564565
0.29603625450595983     0.9990979528655157    0.99910334231178
0.29621015375581244     0.9990903950811452    0.999099325718567
0.2963725347025208      0.9990832845566037    0.9990902701850453
0.29653173206853856     0.9990762630904679    0.9990772935821569
0.2967044152987733      0.9990685900759788    0.9990621039896175
0.29686558022586373     0.9990613752272375    0.9990507090367151
0.29704023101717114     0.999053497782066     0.9990439895138737
0.29721169822778803     0.9990457039420099    0.9990424600523993
0.29737164713526065     0.9990383797066426    0.9990424160250405
0.2975450819069502      0.9990303786990178    0.9990395002434158
0.2977069983754955      0.9990228530628638    0.9990314277255053
0.29786573126335025     0.999015422561244     0.9990186445673264
0.29803795001542194     0.9990073012211215    0.999002425295972
0.29819865046434935     0.9989996667977817    0.9989891740889689
0.2983728367774937      0.9989913299616987    0.9989803129651432
0.29854383950994756     0.9989830826161367    0.9989775119869571
0.29870332393925714     0.9989753342803945    0.9989774893252934
0.29887629423278367     0.9989668687040569    0.9989756890139091
0.2990377462231659      0.9989589082382833    0.9989689051144371
0.29921268407776513     0.998950218544209     0.9989552012005809
0.29938443835167383     0.9989416215110448    0.9989381516344148
0.29954467432243825     0.9989335421547333    0.9989233796991237
0.2997183961574196      0.9989247182238504    0.998912676770588
0.2998805996892566      0.9989164183107873    0.9989087145531691
0.30003961964040315     0.9989082237364154    0.9989083874128059
0.30021212545576664     0.9988992693782357    0.998907425070491
0.30037311296798586     0.9988908516828777    0.9989018798006343
0.300547586344422       0.9988816616904789    0.9988889907806747
0.3007188761401677      0.9988725710277645    0.9988714473603505
0.30087864763276906     0.9988640407024464    0.9988550083795085
0.3010519049895874      0.9988547333366845    0.9988418659971678
0.30121364404326145     0.9988459809534148    0.9988360215826892
0.30137219951624494     0.9988373406452221    0.9988349708626235
0.3015442408534454      0.9988278975447822    0.9988346631508424
0.30170476388750156     0.9988190224999668    0.9988305314722851
0.3018787727857747      0.9988093313786787    0.9988189215263811
0.30204126338090354     0.998800215103134     0.9988022902294907
0.30220057039534187     0.9987912145240435    0.9987843333075637
0.30237336327399716     0.9987813811148534    0.9987682683231638
0.3025346378495082      0.9987721361821752    0.998759705606943
0.30270939828923615     0.9987620446344608    0.9987571090369878
0.30288097514827356     0.9987520619921493    0.9987570827541941
0.3030410337041667      0.998742682119423     0.9987541678267298
0.30321457812427677     0.9987324379490728    0.998743907108348
0.3033766042412426      0.99872280375472      0.9987275846119834
0.30353544677751787     0.9987132928734728    0.9987085670418734
0.3037077751780101      0.9987029001510953    0.9986901096636058
0.3038685852753581      0.9986931318255544    0.9986790588151774
0.304042881236923       0.9986824672237984    0.9986746821023998
0.30421399361779744     0.9986719189084385    0.9986745535755167
0.3043735876955276      0.9986620100912241    0.9986728442395242
0.3045466676374746      0.9986511865095423    0.9986643080473818
0.3047082292762774      0.9986410100450809    0.9986488171090964
0.30486660733438964     0.998630965100059     0.9986291408804091
0.30503847125671885     0.9986199870642263    0.9986083507284813
0.3051988168759038      0.9986096713495491    0.9985944580919743
0.30537264835930567     0.9985984074645794    0.9985876514595337
0.3055349615395633      0.9985878138117689    0.9985868438920431
0.3056940911391304      0.9985773561442722    0.9985863004560216
0.30586670660291443     0.9985659313602487    0.9985801736773855
0.3060278037635542      0.9985551925787105    0.9985664961564357
0.30620238678841094     0.9985434710932044    0.9985446924615737
0.3063737862325771      0.9985318781183159    0.9985219322142693
0.306533667373599       0.9985209876297771    0.9985052980753044
0.30670703437883784     0.9985090945135734    0.9984958130428336
0.3068688830809324      0.9984979122069559    0.9984938585468626
0.30702754820233646     0.998486875052313     0.9984938118794874
0.3071996991879575      0.9984748358076525    0.9984894523877623
0.3073603318704342      0.9984635272158492    0.998477392061405
0.3075344504171279      0.9984511818163699    0.9984559499328151
0.307705385383131       0.9984389730850454    0.9984315347145067
0.30786480204598987     0.9984275069663805    0.9984120127648473
0.3080377045730657      0.9984149829234633    0.9983992610356489
0.3081990887969972      0.9984032100763109    0.9983954793275369
0.3083739588851457      0.9983903622317329    0.9983954447898348
0.30854564539260365     0.9983776553675943    0.9983922290545041
0.30870581359691734     0.9983657174336081    0.9983813386464053
0.308879467665448       0.9983526826091714    0.99836018635619
0.30904160343083437     0.998340425657563     0.9983357502699786
0.3092005556155302      0.9983283275654791    0.998313622209055
0.309372993664443       0.9983151108658622    0.998297555911584
0.30953391341021147     0.9983026899240331    0.9982915613230863
0.3097083190201969      0.9982891325060046    0.9982910814719337
0.30987954104949184     0.9982757252686283    0.9982893080977611
0.3100392447756425      0.9982631325123663    0.9982804060357995
0.3102124343660101      0.9982493804011049    0.9982605325521947
0.3103741056532334      0.9982364522185602    0.9982354202908148
0.31053259335976624     0.998223693105624     0.9982108001846492
0.310704566930516       0.9982097519627621    0.9981910070228488
0.3108650221981215      0.9981966535917554    0.9981820867583588
0.31103896332994396     0.9981823545336997    0.9981804301523726
0.31120138615862214     0.9981689080530356    0.9981799549433882
0.31136062540660975     0.9981556361981737    0.9981737593673663
0.3115333505188143      0.9981411402213521    0.998156444380411
0.3116945573278746      0.9981275163595139    0.9981316691371931
0.31186925000115184     0.9981126490958485    0.9981021502447834
0.3120407590937386      0.9980979472305007    0.998078811543501
0.31220074988318103     0.9980841378965571    0.9980667469209031
0.31237422653684044     0.9980690605241268    0.9980633430226018
0.3125361848873556      0.9980548859930372    0.9980633567861071
0.31269495965718014     0.9980408974738783    0.9980590907424891
0.31286722029122166     0.9980256165246649    0.9980439750778465
0.3130279626221189      0.9980112589115027    0.9980198392258179
0.3132021908172331      0.9979955888009064    0.9979885772318035
0.3133732354316568      0.9979801027982421    0.9979615511582558
0.3135327617429362      0.9979655810375277    0.9979457180291395
0.3137057739184326      0.9979497233031615    0.9979396854093078
0.31386726779078467     0.9979348189004958    0.9979396013368216
0.3140422475273537      0.9979185575268212    0.9979365372678871
0.3142140436832322      0.997902477575384     0.9979229685232909
0.3143743215359664      0.9978873728153016    0.9978993028497579
0.31454808525291755     0.9978708841706515    0.9978666859020621
0.31471033066672444     0.9978553817420648    0.9978379654049387
0.3148693924998408      0.9978400827775409    0.9978183698211456
0.31504194019717413     0.9978233732369732    0.9978092664032798
0.3152029695913632      0.997807671945545     0.9978084420529996
0.3153774848497692      0.9977905381171844    0.9978069366384802
0.3155488165274847      0.9977735971160324    0.99779596112298
0.3157086299020559      0.9977576874215419    0.9977739077039938
0.31588192914084406     0.9977403170825535    0.9977406609362817
0.31604371007648796     0.9977239896865361    0.9977089351364011
0.3162023074314413      0.9977078783827767    0.9976851586242061
0.31637439065061157     0.9976902785828731    0.9976720843404833
0.3165349555666376      0.9976737449196208    0.9976696839889218
0.31670900634688054     0.997655699844329     0.9976693223368044
0.316879873546433       0.9976378598783712    0.9976612598319918
0.31703922244284116     0.9976211102940409    0.997641617396432
0.3172120572034663      0.9976028200066024    0.9976086771029425
0.31737337366094714     0.9975856323433366    0.997574390683696
0.31754817598264495     0.9975668802984332    0.9975438205724878
0.3177197947236522      0.9975483399941398    0.9975272526373533
0.31787989516151516     0.9975309274323435    0.9975231224859438
0.3180534814635951      0.9975119202418012    0.9975231352685374
0.3182155494625307      0.9974940534796971    0.9975176133708349
0.31837443388077585     0.9974764238225025    0.9975004777943716
0.31854680416323794     0.9974571695955048    0.9974684376144205
0.31870765614255575     0.9974390810009208    0.9974322919149355
0.3188819939860905      0.9974193432041595    0.9973972502330998
0.3190531482489347      0.9973998308621957    0.9973757526784297
0.31921278420863464     0.9973815104571645    0.9973685949667161
0.3193859060325515      0.9973615094207645    0.9973683755401686
0.3195475095533241      0.9973427136128157    0.9973652029397195
0.3197059294934062      0.9973241913299234    0.9973511865668493
0.31987783529770525     0.9973039690789187    0.9973210975433437
0.32003822279886        0.997284975961314     0.9972838985197722
0.32021209616423174     0.9972642477494201    0.9972445480627822
0.3203744512264592      0.9972447621694334    0.9972183864301228
0.3205336227079961      0.9972255355589057    0.9972064133271775
0.32070628005375        0.997204541292304     0.9972045457452343
0.32086741909635963     0.9971848166115829    0.9972036075197649
0.3210420440031862      0.997163297484884     0.9971920086066263
0.3212134853293222      0.9971420243950474    0.9971642362247541
0.3213734083523139      0.9971220490863124    0.9971267058003245
0.3215468172395226      0.9971002450591335    0.9970837862592156
0.321708707823587       0.9970797530446905    0.9970524953107877
0.3218674148269609      0.9970595355213013    0.9970358800890311
0.32203960769455175     0.9970374553270404    0.9970315708100022
0.3222002822589983      0.9970167154906932    0.9970315664352056
0.32237444268766186     0.9969940850488408    0.9970234021548113
0.3225454195356348      0.9969717157506476    0.9969989281032258
0.3227048780804635      0.9969507164289888    0.9969621245904022
0.32287782248950914     0.9969277908107379    0.9969162866064172
0.3230392485954105      0.9969062501273488    0.996879635008166
0.3232141605655288      0.996882754295765     0.9968559711526445
0.3233858889549566      0.9968595277668904    0.9968491632125842
0.32354609904124015     0.9968377168416752    0.9968493079204463
0.32371979499174064     0.9968139139277186    0.9968433742847356
0.32388197263909685     0.9967915421089425    0.9968228387236802
0.3240409667057625      0.9967694705668247    0.9967873508616769
0.3242134466366451      0.9967453705755703    0.9967394854765406
0.3243744082643834      0.9967227324376878    0.996698056383359
0.3245488557563387      0.9966980358557295    0.9966682207601861
0.32472011966760345     0.9966736254022083    0.9966571819620929
0.32487986527572393     0.9966507088927458    0.9966566910627587
0.3250530967480614      0.9966256956817426    0.996653512106565
0.32521480991725454     0.996602192632223     0.9966368408004262
0.3253733395057572      0.9965790081369229    0.996603815660614
0.3255453549584768      0.996553689103299     0.9965549986227525
0.32570585210805214     0.9965299123666039    0.996509064339785
0.32587983512184443     0.9965039739161213    0.9964723273230298
0.32604229983249244     0.9964796258892361    0.9964560568242019
0.3262015809624499      0.9964556055198532    0.9964532555314166
0.3263743479566243      0.996429383112995     0.9964525313689468
0.3265355966476544      0.9964047499889689    0.9964410374328153
0.3267103312029015      0.9963778823797287    0.9964086374469912
0.32688188217745806     0.9963513268267237    0.9963597390815528
0.32704191484887035     0.9963263948043583    0.9963101485336987
0.3272154333844996      0.9962991868088322    0.9962669030730366
0.32737743361698457     0.9962736196342893    0.9962448124638251
0.327536250268779       0.9962483990858984    0.9962389584768487
0.32770855278479033     0.9962208614754328    0.9962390600222706
0.3278693369976574      0.9961949990920484    0.9962312545769179
0.32804360707474145     0.9961667858415191    0.9962031200851578
0.328214693571135       0.9961389031240732    0.996155498861576
0.3283742617643842      0.9961127315693415    0.9961030354704467
0.3285473158218504      0.9960841659774227    0.9960530815021106
0.32870885157617236     0.9960573297002493    0.9960240399156269
0.32888387319471124     0.9960280644893796    0.9960133437065285
0.32905571123255956     0.9959991397947363    0.9960133596556977
0.3292160309672636      0.9959719816005956    0.9960078827240983
0.3293898365661846      0.9959423499346286    0.9959827950681596
0.3295521238619613      0.9959145035559382    0.9959388688383467
0.3297112275770475      0.995887035256234     0.9958844986035491
0.3298838171563507      0.995857049416927     0.9958286398363525
0.33004488843250956     0.9958288861370898    0.9957926997078278
0.3302194455728854      0.9957981690654619    0.9957763867894377
0.33039081913257073     0.9957678130285808    0.9957752145826783
0.3305506743891118      0.9957393184229597    0.9957725558631204
0.3307240155098698      0.9957082238073428    0.9957524248264055
0.33088583832748353     0.9956790102705116    0.9957116591074818
0.3310444775644067      0.9956501970527203    0.9956565973325998
0.3312166026655468      0.9956187381251683    0.9955952552961965
0.33137720946354265     0.995589199181671     0.9955516975607991
0.33155130212575545     0.995556976932662     0.9955281279145286
0.33172221120727774     0.9955251374718544    0.9955241584949404
0.33188160198565575     0.9954952585381256    0.9955236218156819
0.3320544786282507      0.9954626484155321    0.9955089943213357
0.3322158369677014      0.9954320403362334    0.9954728599026597
0.33239068117136905     0.9953986832024297    0.995412357263539
0.33256234179434613     0.9953657195901551    0.9953467865293422
0.33272248411417893     0.9953347756600682    0.995297048879881
0.3328961122982287      0.9953010151027075    0.9952671194501357
0.33305822217913417     0.9952692951016002    0.9952600090513873
0.33321714847934913     0.995238010357516     0.9952603115117417
0.33338956064378106     0.9952038596100528    0.9952503474462774
0.3335504545050687      0.995171790911847     0.9952190124206792
0.3337248342305733      0.9951368155885967    0.9951603557572577
0.33389603037538734     0.9951022562691062    0.9950911655965697
0.3340557082170571      0.9950698223135869    0.9950340519867626
0.3342288719229438      0.9950344298885927    0.9949951786017703
0.33439051732568625     0.9950011847118655    0.994982605501003
0.3345489791477382      0.9949683992948877    0.994982429834078
0.33472092683400706     0.9949326041166733    0.9949769371405601
0.33488135621713166     0.9948989995964916    0.9949514948753316
0.3350552714644732      0.9948623431163558    0.9948966195934829
0.3352176684086705      0.9948278999131057    0.9948289520260417
0.3353768817721773      0.9947939298064351    0.9947641909008144
0.335549580999901       0.9947568548900961    0.9947138530708906
0.33571076192448046     0.9947220380855265    0.9946925804865564
0.3358854287132768      0.9946840729961239    0.9946896330497244
0.33605691192138265     0.9946465606407766    0.994687284067005
0.3362168768263442      0.9946113531353933    0.9946672958605018
0.33639032759552273     0.9945729419464052    0.9946168836659288
0.336552260061557       0.9945368592354026    0.9945487837264302
0.3367110089469007      0.994501276501874     0.9944784648177221
0.3368832436964614      0.9944624353217648    0.9944185323299068
0.3370439601428778      0.9944259693691475    0.9943888900721
0.33721816245351116     0.9943861999196372    0.9943816402733883
0.33738918118345396     0.9943469095146532    0.9943814418964474
0.3375486816102525      0.994310043034754     0.9943673710304989
0.33772166790126795     0.9942698157070182    0.9943231920246608
0.33788313588913915     0.9942320369326089    0.9942566081415062
0.3380580897412273      0.9941908509565811    0.9941742727727096
0.33822986001262495     0.9941501582778902    0.9941066491032216
0.3383901119808783      0.9941119643443735    0.9940698473089887
0.33856384981334864     0.9940703480399505    0.9940581531482398
0.3387260693426747      0.9940312539244144    0.9940585845472801
0.33888510529131016     0.9939927027119212    0.9940494000830834
0.3390576271041626      0.9939506299404354    0.9940114855291174
0.33921863061387075     0.9939111278076194    0.9939475234952316
0.33939311998779587     0.993868055686201     0.9938616927746932
0.33956442578103047     0.9938255036289104    0.9937850000969379
0.3397242132711208      0.9937855740307899    0.9937382383271466
0.33989748662542807     0.9937420126758599    0.9937190083784867
0.3400592416765911      0.9937010999684908    0.993718472369027
0.34021781314706356     0.9936607596874136    0.9937138464149012
0.340389870481753       0.9936167265612479    0.9936834555226187
0.3405504095132982      0.9935753939736551    0.9936241915437809
0.34072443440906025     0.9935303182844089    0.9935368421091475
0.34088694100167805     0.9934879701781865    0.9934553296318265
0.3410462640136053      0.9934462105678431    0.9933959784815072
0.34121907288974956     0.993400644952358     0.9933650971584275
0.3413803634627495      0.99335786047175      0.993360603046248
0.34155513989996644     0.9933112182422735    0.9933591823521105
0.34172673275649285     0.9932651403814555    0.9933356118678002
0.341886807309875       0.993221899452084     0.9932816206450354
0.34206036772747406     0.9931747348776094    0.9931943841773656
0.34222240984192887     0.993130435448189     0.9931064505796011
0.3423812683756931      0.9930867567999999    0.9930367405287868
0.3425536127736743      0.9930390893868212    0.9929949386480268
0.3427144388685112      0.992994342888647     0.9929851003560408
0.3428887508275651      0.9929455540570105    0.9929854114290015
0.34305987920592845     0.9928973612186401    0.9929691276232099
0.34321948928114754     0.9928521473618759    0.992922254303365
0.3433925852205836      0.9928028230138847    0.992837588232082
0.34355416285687534     0.992756507019885     0.9927446770977447
0.34371255691247654     0.9927108457493211    0.9926644065866014
0.3438844368322947      0.9926610066713802    0.9926096703496899
0.34404479844896857     0.9926142339155364    0.9925918474407236
0.3442186459298594      0.9925632279755403    0.9925917210270894
0.34438097510760596     0.9925153185990232    0.9925835844065098
0.344540120704662       0.9924680824241529    0.9925472785478958
0.344712752165935       0.9924165579843582    0.9924697851445302
0.3448738653240637      0.9923682197754455    0.992374657385828
0.3450484643464094      0.9923155257650706    0.9922750975919361
0.34521987978806457     0.9922634775336178    0.9922076479424098
0.34537977692657545     0.9922146438473267    0.9921805883524328
0.34555315992930324     0.9921613813528206    0.9921777206214868
0.34571502462888676     0.9921113645112016    0.9921743032895262
0.34587370574777976     0.9920620556472072    0.992146183557408
0.3460458727308897      0.9920082458007204    0.9920755810930708
0.3462065214108554      0.9919577431702994    0.9919802948879096
0.34638065595503803     0.9919026801829516    0.9918719141472148
0.34655160691853015     0.9918482975557792    0.9917905773468558
0.346711039578878       0.9917972862814838    0.9917517314220816
0.3468839581034428      0.9917416390533335    0.9917431876527414
0.34704535832486333     0.9916893956128755    0.9917430506594832
0.3472202444105008      0.9916324551170279    0.9917197296691322
0.3473919469154477      0.9915762137129857    0.9916542659090484
0.34755213111725036     0.9915234422132698    0.9915591272795679
0.34772580118326996     0.9914658958970615    0.9914441628294703
0.3478879529461453      0.9914118529255552    0.9913553569513365
0.3480469211283301      0.9913585759098286    0.9913042659418793
0.34821937517473184     0.9913004472598977    0.991288035994497
0.3483803109179893      0.9912458879856526    0.9912888293037461
0.34855473252546376     0.991186413888307     0.9912736363286718
0.34872597055224763     0.9911276766467274    0.9912174462315121
0.34888569027588723     0.9910725774944085    0.9911260359772661
0.3490588958637438      0.9910124831708528    0.9910058949559151
0.34922058314845605     0.9909560617025672    0.9909047588594848
0.3493790868524778      0.9909004466053362    0.990839415054174
0.34955107642071653     0.9908397570069063    0.9908122403826122
0.349711547685811       0.9907828088093137    0.9908114838664327
0.3498855048151224      0.9907207208531058    0.990803825183343
0.3500479436412895      0.990662410064922     0.990762121498851
0.3502071988867661      0.9906049276500546    0.9906795067100157
0.3503799399964596      0.9905422240265179    0.9905579707682559
0.35054116280300884     0.9904833680588674    0.9904444366982732
0.35071587147377503     0.9904192238884568    0.9903548466883562
0.3508873965638507      0.9903558775122367    0.9903154213540644
0.3510474033507821      0.990296485715702     0.9903107359207007
0.35122089600193046     0.9902317357026758    0.9903081351515138
0.35138287034993454     0.9901709401430961    0.9902761579275201
0.3515416611172481      0.9901110151588042    0.9902015991857733
0.3517139377487786      0.9900456358825461    0.9900808846853427
0.3518746960771649      0.9899842836069763    0.9899587064108166
0.3520489402697681      0.9899174074080551    0.989852888695747
0.3522200008816807      0.9898513695949286    0.9897981526636276
0.35237954319044906     0.9897894343107088    0.9897863433939835
0.3525525713634344      0.9897218865179294    0.9897868168699904
0.3527140812332754      0.9896584794610016    0.989765130348102
0.3528890769673334      0.9895893882987296    0.9896918509342053
0.3530608891207009      0.9895211578571667    0.9895719547544025
0.3532211829709241      0.9894571459331908    0.9894433768414322
0.35339496268536424     0.9893873588561484    0.9893247690967524
0.3535572240966601      0.9893218296483244    0.9892590957273567
0.35371630192726544     0.9892572398454977    0.98923820294298
0.3538888656220877      0.9891867849831968    0.989238858546151
0.3540499110137657      0.9891206655986831    0.9892254864901333
0.35422444226966066     0.9890486072022426    0.9891635952717666
0.3543957899448651      0.9889774543418677    0.9890491577045738
0.35455561931692525     0.9889107176720566    0.9889159744096272
0.35472893455320237     0.9888379480229891    0.9887825863998557
0.3548907314863352      0.9887696354122633    0.9886998227636842
0.35504934483877754     0.9887023100142367    0.9886664010718369
0.35522144405543676     0.9886288588955173    0.9886640638241333
0.3553820249689517      0.9885599452616046    0.9886581567125133
0.3555560917466836      0.9884848296201885    0.9886095987943715
0.355726974943725       0.9884106664776672    0.988504366662464
0.35588633983762213     0.9883411244377399    0.9883698203481711
0.3560591905957362      0.9882652835296277    0.9882228873423906
0.356220523050706       0.9881941060546215    0.9881212933046233
0.35639534136989276     0.9881165514323114    0.9880691104668732
0.356566976108389       0.988039974799401     0.9880623201807466
0.35672709254374096     0.9879681475033428    0.9880603858029346
0.3569006948433099      0.9878898437037388    0.988021133181344
0.35706277883973453     0.9878163327329712    0.987929309628403
0.3572216792554686      0.987743894320746     0.9877960019121246
0.35739406553541964     0.9876649220227649    0.987638606385773
0.3575549335122264      0.9875908246179496    0.9875196346001076
0.3577292873532501      0.9875100753916324    0.9874488812994363
0.3579004576135833      0.9874303531722554    0.9874329237772673
0.3580601095707722      0.9873555941743585    0.9874340321471041
0.3582332473921781      0.987274080337035     0.9874074571448859
0.3583948669104397      0.9871975744182816    0.9873283255234562
0.35855330284801074     0.9871221848795811    0.9872001378770614
0.3587252246497987      0.9870399388127798    0.9870351461568353
0.35888562814844244     0.9869627887320298    0.9868986703636551
0.3590595175113031      0.986878698473789     0.9868060487843702
0.3592218885710195      0.9867997501631689    0.9867765960319813
0.3593810760500454      0.9867219466347896    0.9867768693473172
0.3595537493932882      0.9866370985631405    0.9867642316411731
0.3597149044333868      0.9865574829596748    0.9867035469510664
0.3598895453377023      0.9864707369706409    0.9865726490212594
0.36006100266132723     0.986385096697127     0.986403413149636
0.3602209416818079      0.9863047829694934    0.9862520596068962
0.36039436656650553     0.9862172298694932    0.9861381161349326
0.3605562731480589      0.9861350509410574    0.9860930543735109
0.3607149961489217      0.9860540729584057    0.9860886884638567
0.3608872050140015      0.9859657480848055    0.9860840921404579
0.36104789557593703     0.9858828910596427    0.9860376819442713
0.3612220720020895      0.9857925987051196    0.9859191233282115
0.36139306484755146     0.9857034672902317    0.9857497597110334
0.36155253938986914     0.9856199010326404    0.9855851653429261
0.3617254997964038      0.985528787241041     0.9854481332903474
0.36188694189979415     0.9854432878815776    0.985383227557367
0.36206186986740146     0.9853501503116119    0.9853698245937382
0.3622336142543182      0.9852582036702237    0.9853691482962407
0.3623938403380907      0.9851719713384907    0.9853324334385559
0.3625675522860801      0.9850779857885404    0.9852232473909374
0.3627297459309253      0.9849897651283471    0.9850636362758523
0.3628887559950799      0.984902836535795     0.9848891439929587
0.3630612519234515      0.9848080413252838    0.9847311602665503
0.36322222954867883     0.9847191103404223    0.9846457529739456
0.3633966930381231      0.9846222194682327    0.9846195336616441
0.3635679729468768      0.9845266072396573    0.9846212120968751
0.36372773455248625     0.9844369763180839    0.9845974863694418
0.36390098202231264     0.9843392705513749    0.9845045427099491
0.36406271118899475     0.9842475808238762    0.9843518854861008
0.36422125677498635     0.9841572442632877    0.9841707435447253
0.3643932882251949      0.9840587156539224    0.9839919993897948
0.3645538013722592      0.9839663051544387    0.9838828858006639
0.3647278003835404      0.9838656061989353    0.9838384168959161
0.36489028109167737     0.983771078642269     0.9838379446960897
0.36504957821912376     0.9836779375412302    0.983828068437187
0.3652223612107871      0.9835763878133962    0.9837581544171169
0.3653836258993062      0.9834811142752902    0.9836209286385724
0.3655583764520422      0.9833773332728402    0.9834189841103409
0.3657299434240877      0.9832748940683826    0.9832228804189668
0.36588999209298895     0.9831788398803493    0.9830910621461344
0.36606352662610714     0.9830741528571654    0.9830262208145333
0.36622554285608105     0.9829759059609308    0.983019382601298
0.36638437550536446     0.9828791110605575    0.9830170487804142
0.3665566940188648      0.9827735597038394    0.982964496569311
0.3667174942292209      0.9826745567196521    0.9828414585886163
0.3668917803037939      0.9825666956002508    0.9826416210753466
0.36706288279767635     0.9824602407911639    0.9824304161287354
0.36722246698841454     0.9823604467182163    0.9822743419040552
0.3673955370433697      0.9822516656173783    0.9821841581943548
0.36755708879518056     0.9821496021683426    0.9821656149215673
0.3677154569663009      0.9820490588378309    0.9821675918047597
0.36788731100163824     0.9819394014396929    0.9821330407911472
0.3680476467338313      0.9818365733620573    0.9820284428719895
0.3682214683302413      0.981724526722639     0.9818368379791879
0.368383771623507       0.9816193677420636    0.981625255150917
0.36854289133608215     0.9815157659730063    0.9814421332489462
0.36871549691287425     0.981402815594925     0.981317647414341
0.3688765841865221      0.9812968673042314    0.9812778452606647
0.36905115732438687     0.9811814634472453    0.9812786910130075
0.36922254688156114     0.9810675690646389    0.9812584265415277
0.36938241813559114     0.9809607954656798    0.9811717956706325
0.3695557752538381      0.9808444308336106    0.9809915664285329
0.36971761406894077     0.9807352471267194    0.9807747325686333
0.3698762693033529      0.9806277284989566    0.980571576095267
0.37004841040198194     0.9805104908194624    0.9804177686311027
0.37020903319746673     0.980400548875754     0.9803561186473257
0.3703831418571685      0.9802807753232003    0.9803503530972936
0.3705540669361797      0.9801625813105533    0.9803425065652575
0.37071347371204666     0.9800518048551763    0.9802763649021204
0.37088636635213057     0.9799310571571563    0.9801135212941005
0.3710477406890702      0.9798177887254175    0.9798971026485973
0.3712226008902268      0.9796944362124864    0.9796552216972201
0.37139427751069287     0.9795727013089907    0.9794781164870109
0.3715544358280147      0.9794585705644933    0.97939737232066
0.3717280800095534      0.9793342127553761    0.9793829429828195
0.37189020588794786     0.9792175223751542    0.9793823587162862
0.37204914818565177     0.9791025761191192    0.9793341970138705
0.37222157634757264     0.978977261888337     0.9791899121988699
0.37238248620634923     0.9788597391625914    0.978978135786171
0.3725568819293428      0.9787317325686399    0.9787215491964548
0.3727280940716458      0.9786054184245174    0.9785154476033658
0.37288778791080457     0.9784870244817021    0.9784068935006632
0.3730609676141803      0.9783579999005102    0.9783756890984486
0.3732226290144117      0.9782369607316475    0.9783786629453325
0.3733811068339526      0.9781177444096852    0.9783486838848645
0.3735530705177104      0.977987752864097     0.9782277376578612
0.37371351589832397     0.9778658744876058    0.978027024886533
0.37388744714315447     0.9777331019672435    0.9777609635316045
0.3740498600848407      0.9776085093813436    0.9775359480355963
0.3742090894458364      0.9774857825080645    0.9773904855872888
0.3743818046710491      0.9773520136218333    0.9773306100936702
0.3745430015931175      0.9772265554420514    0.9773300887367087
0.37471768437940284     0.9770899336398775    0.9773145419387848
0.3748891835849976      0.9769551236543651    0.9772157469898967
0.37504916448744813     0.9768287599291203    0.9770288515582934
0.3752226312541156      0.9766910786923788    0.9767589171642819
0.3753845797176388      0.9765619124202911    0.9765116514852185
0.37554334460047145     0.9764346953486222    0.9763353568692796
0.3757155953475211      0.9762960092259984    0.9762472226071928
0.37587632779142643     0.9761659725164039    0.9762373313414116
0.37605054609954874     0.9760243529602866    0.9762336298906131
0.37622158082698054     0.9758846496439579    0.9761596030683997
0.37638109725126806     0.9757537326730112    0.9759926006082685
0.3765540995397725      0.9756110666277338    0.9757258046034408
0.3767155835251327      0.9754772573197329    0.9754596013795388
0.3768905533747098      0.9753315712022389    0.9752338085179106
0.37706233964359637     0.975187822760913     0.9751216602827915
0.3772226076093387      0.9750530732622477    0.9751010837365391
0.37739636143929794     0.9749062852804574    0.975102655521747
0.37755859696611294     0.9747685683113424    0.9750505785341709
0.3777176489122374      0.9746329331318729    0.9749040927420304
0.37789018672257885     0.9744851002389247    0.9746450960715314
0.378051206229776       0.9743464794455875    0.9743656767040545
0.3782257116011901      0.9741955299844134    0.9741076580624782
0.37839703339191366     0.9740466044288465    0.9739608959616414
0.37855683687949293     0.9739070371874097    0.97392070463387
0.37873012623128915     0.9737549756139411    0.9739241958220134
0.3788918972799411      0.9736123464705208    0.9738926125122621
0.37905048474790254     0.9734718894380667    0.9737713049519426
0.37922255808008093     0.9733187749710158    0.973527785008037
0.37938311310911504     0.9731752378997452    0.973240920582806
0.3795571540023661      0.9730189092501423    0.9729517814804024
0.3797196765924729      0.9728722337834212    0.9727716953278206
0.37987901560188914     0.972727779610467     0.9726992370363041
0.3800518404755223      0.9725703672286079    0.9726954297464415
0.3802131470460112      0.9724227562371723    0.9726847011013026
0.3803879394807171      0.9722620500042336    0.9725821081881479
0.38055954833473243     0.9721035048187161    0.972356801158652
0.3807196388856035      0.9719549145301989    0.9720681067856394
0.38089321530069153     0.9717930557661384    0.9717538150004654
0.3810552734126353      0.9716412297298526    0.9715381835999707
0.3812141479438885      0.9714917203760132    0.9714349389775082
0.38138650833935867     0.9713287720488778    0.9714175557403218
0.3815473504316846      0.9711760085313612    0.9714174505070925
0.3817216783882274      0.9710096658225743    0.9713428211125084
0.3818928227640797      0.9708455783988478    0.9711428215107403
0.38205244883678774     0.970691833259696     0.9708596843322573
0.3822255607737127      0.9705243317463004    0.9705246257161182
0.38238715440749343     0.970367268523572     0.9702717201088733
0.38254556446058363     0.9702126243096798    0.9701310806802714
0.3827174603778908      0.9700440509514138    0.9700910542786846
0.38287783799205366     0.9698860539603961    0.9700958019941675
0.3830517014704335      0.9697139842266187    0.9700493657256819
0.38321404664566905     0.9695525722667108    0.9698928189912179
0.38337320824021404     0.9693936279421237    0.9696288488617004
0.383545855698976       0.9692204325973313    0.9692809242606245
0.38370698485459365     0.9690580540860254    0.9689875010486705
0.38388159987442827     0.968881277943592     0.9687844008515649
0.3840530313135724      0.9687069052622186    0.9687168186122314
0.3842129444495722      0.9685435140992921    0.9687191733118062
0.384386343449789       0.9683655400579013    0.9686944847437904
0.3845482241468615      0.9681986310486751    0.96856746954765
0.38470692126324346     0.9680342925004592    0.9683232472748154
0.38487910424384236     0.9678551888100342    0.9679713928537864
0.385039768921297       0.9676873132055156    0.9676489972618705
0.38521391946296857     0.9675045225641515    0.9674004266429979
0.38538488642394964     0.9673242370232439    0.9672962760922599
0.3855443350817864      0.9671553483448151    0.9672880549211638
0.3857172696038402      0.9669713553111877    0.967281332880921
0.38587868582274965     0.9667988447413715    0.967187149176463
0.38605358790587607     0.9666110768503476    0.9669413454657123
0.38622530640831193     0.9664258694077906    0.9665882976782898
0.38638550660760357     0.9662523167383464    0.9662451859339914
0.3865591926711121      0.9660633135860908    0.9659611047454152
0.38672136043147637     0.965886052604988     0.9658286886177637
0.3868803446111501      0.9657115268408993    0.9658034759695172
0.3870528146550408      0.9655213605980079    0.9658065675233316
0.38721376639578725     0.9653431070209694    0.965740662622793
0.38738820400075064     0.9651490567196234    0.9655260436978856
0.3875594580250235      0.9649576738069561    0.96518283794557
0.3877191937461521      0.9647783799564822    0.9648212582578665
0.38789241533149765     0.9645830921542362    0.9644937674021957
0.38805411861369893     0.9643999828974484    0.9643176137220452
0.38821263831520963     0.9642197199684128    0.9642660797444977
0.3883846438809373      0.9640232691484778    0.9642716178222628
0.3885451311435207      0.9638391720828063    0.9642334687849049
0.388719104270321       0.9636387343339392    0.9640569305733137
0.3888815590939771      0.9634507403220314    0.9637525239396368
0.38904083033694264     0.9632656529523336    0.9633816614345203
0.38921358744412515     0.9630640209609104    0.9630080524664767
0.3893748262481634      0.962875010550479     0.9627748257513544
0.3895495509164186      0.9626692928528604    0.9626766315782431
0.3897210920039832      0.9624664113757213    0.9626762432496392
0.38988111478840354     0.9622763354229545    0.9626587427623257
0.39005462343704084     0.962069346926026     0.9625177749751549
0.39021661378253386     0.9618752572766817    0.9622375505834414
0.39037542054733637     0.9616841904664047    0.9618657896099189
0.39054771317635584     0.9614760093085492    0.9614600397399452
0.390708487502231       0.9612809088435535    0.9611800252271608
0.3908827476923232      0.9610685280766053    0.9610371731140842
0.39105382430172475     0.960859098539628     0.9610209481051849
0.3912133826079821      0.9606629377388664    0.9610196297970313
0.39138642677845636     0.9604492873013777    0.9609174311805971
0.39154795264578635     0.9602490010149314    0.9606704972960461
0.3917229643773333      0.9600310559618384    0.960266022527059
0.3918947925281897      0.959816124810033     0.9598375864960067
0.39205510237590185     0.9596147495080756    0.9595217289957239
0.39222889808783096     0.9593955021324445    0.9593414035172063
0.3923911754966158      0.9591899078812095    0.9593070293087936
0.3925502693247101      0.9589875220524745    0.959313460120264
0.39272284901702137     0.9587670544809408    0.959243921797834
0.39288391040618836     0.9585604294487804    0.9590307055134225
0.3930584576595723      0.9583355502592616    0.9586417200322616
0.3932298213322657      0.9581138053730461    0.9581940557621831
0.3933896667018148      0.9579060988203523    0.9578347702291123
0.39356299793558086     0.9576799207306954    0.9576011856141428
0.39372481086620265     0.957467880338303     0.9575352478091944
0.3938834402161339      0.9572591729488925    0.9575415569423971
0.39405555543028215     0.9570317804051071    0.9575034461216683
0.3942161523412861      0.9568187189320265    0.9573304666211336
0.394390235116507       0.9565867966366866    0.9569687249755016
0.39456113431103734     0.9563581320512128    0.9565115418007737
0.3947205152024234      0.9561439983876828    0.9561111049281867
0.3948933819580264      0.9559107762813766    0.955817602601667
0.39505473041048517     0.9556921826888801    0.9557083823885354
0.39522956472716086     0.9554543272120696    0.9557049578085388
0.39540121546314605     0.9552197970427021    0.9556854470759545
0.39556134789598696     0.9550001036399501    0.9555411915300626
0.3957349661930448      0.9547609230857552    0.9552018607806751
0.3958970661869584      0.9545366826809919    0.9547645801821973
0.3960559826001815      0.954315974321933     0.9543318742242205
0.39622838487762146     0.9540755575146671    0.9539827435646757
0.39638926885191716     0.9538502818678127    0.9538266853913014
0.3965636386904298      0.953605115799411     0.9538027874958906
0.39673482494825196     0.9533634041758275    0.9538020075914553
0.39689449290292983     0.9531370413524399    0.9536978783234057
0.39706764672182465     0.9528905588919232    0.9533990840573141
0.3972292822375752      0.9526595306403831    0.9529722175885417
0.39738773417263523     0.9524321676239287    0.9525140198631901
0.3975596719719122      0.9521844598471292    0.9521074831102698
0.39772009146804493     0.9519524111279363    0.9518949213871787
0.3978939968283946      0.9516998325215839    0.9518367807611857
0.3980563838856         0.9514630202344886    0.951845869321114
0.3982155873621148      0.9512299457229109    0.9517884613489287
0.3983882767028466      0.9509761123052536    0.9515500908749034
0.3985494477404341      0.9507382536492559    0.9511543187525244
0.39872410464223856     0.950479447771971     0.9506326390800969
0.3988955779633525      0.9502242996050945    0.9501769517910471
0.3990555329813222      0.9499853414478233    0.9499088661923876
0.3992289738635088      0.9497251989566703    0.9498091390067311
0.39939089644255116     0.9494813557604381    0.9498150882641839
0.39954963544090294     0.9492413888179044    0.9497874624495053
0.3997218603034717      0.948980004701303     0.9495973210515416
0.39988256686289614     0.9487351321550431    0.9492335196716227
0.40005675928653756     0.9484686509701582    0.9487079303700064
0.40022776812948846     0.9482059648385526    0.9482066824990734
0.4003872586692951      0.9479600094552295    0.9478772267835046
0.40056023507331867     0.9476922044151473    0.9477225765508692
0.400721693174198       0.947441241412948     0.947712929892758
0.40089663713929424     0.9471682340425116    0.9477038458697736
0.40106839752369994     0.9468990951754496    0.9475472355536615
0.40122863960496136     0.9466469928017729    0.9472086234838116
0.40140236755043973     0.9463725986063154    0.9466836321102204
0.40156457719277383     0.9461153849710606    0.946174552238472
0.4017236032544174      0.9458622686284233    0.9457891908572155
0.40189611518027796     0.9455866203068652    0.945576820282476
0.4020571088029942      0.9453283726051407    0.9455418444591684
0.40223158828992744     0.9450473954555273    0.9455483866291057
0.40240288419617015     0.9447704324923208    0.9454381416905507
0.4025626617992686      0.9445110972018658    0.9451433978674016
0.40273592526658397     0.9442287841188837    0.9446356092584086
0.4028976704307551      0.9439642140327941    0.9441005504328818
0.4030562320142356      0.9437038887555277    0.9436583621135503
0.4032282794619331      0.9434203419849366    0.9433776628276586
0.40338880860648635     0.9431547618274703    0.9433034638002222
0.4035628236152565      0.9428657597863529    0.9433144085387594
0.40372532032088243     0.9425948415086163    0.9432570041914948
0.4038846334458178      0.9423282483870307    0.9430259129771066
0.40405743243497017     0.9420379858492491    0.9425602474087782
0.40421871312097823     0.9417660342327513    0.9420131493543927
0.40439347967120326     0.9414702096744818    0.9414661925036247
0.4045650626407377      0.9411786248130539    0.9411195817738732
0.4047251273071279      0.9409055851915896    0.9409998626587743
0.40489867783773503     0.940608416742604     0.9410024779103798
0.4050607100651979      0.9403299125601794    0.9409770284391481
0.40521955871197024     0.9400558861018806    0.9407954823207376
0.40539189322295954     0.939757479744772     0.9403707285964469
0.40555270943080457     0.9394779683643395    0.9398245437443992
0.40572701150286655     0.9391738706420725    0.9392315126184169
0.40589812999423797     0.9388741641343924    0.9388129306426085
0.4060577301824651      0.9385935905661823    0.9386346388453806
0.4062308162349092      0.9382881711517813    0.9386141455917884
0.406392383984209       0.9380020055614231    0.9386137843290402
0.4065507681528183      0.9377204729289593    0.938485374057951
0.4067226381856446      0.9374138398759029    0.938115227620582
0.40688298991532657     0.9371266948739274    0.9375852880038602
0.4070568275092255      0.9368142401199837    0.9369563591559612
0.40721914679998017     0.9365213961507004    0.9364823648762185
0.4073782825100443      0.9362332529234184    0.9362214594980495
0.4075509040843254      0.9359195017977632    0.9361516738318009
0.40771200735546226     0.935625603333438     0.9361651019621101
0.407886596490816       0.935305920347344     0.9360801225842379
0.4080580020454792      0.9349908678630351    0.9357645884625295
0.40821788929699815     0.9346959138264291    0.935260209876535
0.40839126241273405     0.9343749091323373    0.9346101963852287
0.40855311722532567     0.9340741277959071    0.9340762757393372
0.4087117884572268      0.9337782267578326    0.9337444248066837
0.40888394555334484     0.9334560141516333    0.9336209302898713
0.40904458434631863     0.933154266518528     0.9336298474864845
0.4092187090035094      0.9328259927911493    0.9335889612420025
0.40938965008000955     0.9325025105372619    0.9333372341582813
0.40954907285336545     0.9321997422263707    0.9328738960341854
0.4097219814909383      0.9318701785276361    0.9322180996991197
0.4098833718253669      0.931561455304333     0.9316292368742581
0.4100582480240124      0.931225719164524     0.931189390830531
0.41022994064196744     0.9308948624284376    0.9310185933694031
0.4103901149567782      0.9305850986095441    0.9310153030397929
0.4105637751358059      0.9302480488317514    0.9309997144386329
0.4107259170116893      0.9299322202730367    0.9308097166646424
0.41088487530688217     0.9296215276984277    0.9303917763142193
0.411057319466292       0.9292832815022161    0.929743640839807
0.4112182453225575      0.9289665046401342    0.9291137514465662
0.41139265704304        0.9286219540860784    0.9285955772184158
0.411563885182832       0.9282824485379185    0.9283528954874566
0.4117235950194797      0.9279646679452941    0.9283203498657331
0.41189679072034435     0.9276188374648286    0.9283282035329181
0.41205846811806474     0.9272948618622305    0.9281966384114849
0.4122169619350946      0.9269761910868703    0.9278370619876698
0.41238894161634143     0.9266291997117366    0.9272144522631977
0.412549402994444       0.9263043143943633    0.9265548319370714
0.41272335023676343     0.925950885891401     0.9259575875903839
0.41288577917593866     0.9256196951318709    0.9256387277786902
0.4130450245344233      0.9252939011706078    0.9255496760739058
0.41321775575712494     0.9249392896244494    0.9255657586298045
0.4133789686766823      0.9246071704211634    0.9254998752977359
0.4135536674604566      0.924246008031307     0.9251792557154205
0.41372518266354036     0.9238900862712124    0.9245915575462248
0.41388517956347987     0.9235569056998879    0.9239168443588994
0.41405866232763633     0.9231943930434307    0.9232538316662867
0.4142206267886485      0.922854774670938     0.9228556754557093
0.41437940766897013     0.922520729214383     0.9227084031281515
0.4145516744135087      0.9221570754766032    0.9227130844654307
0.414712422854903       0.9218165743761042    0.9226877317062927
0.41488665716051426     0.9214462379177959    0.9224383578579454
0.415057707885435       0.9210813818397486    0.9219030474850994
0.41521724030721147     0.9207399442684293    0.9212290336277784
0.4153902585932049      0.9203683868154273    0.9205072958994038
0.41555175857605403     0.9200203830700496    0.9200230118278878
0.41572674442312013     0.919642029628995     0.9197910764614443
0.41589854668949566     0.9192692523874348    0.9197775389005809
0.4160588306527269      0.9189202978966865    0.9197742719351445
0.41623260048017513     0.9185407057723954    0.9195742638776828
0.41639485200447907     0.9181850732190269    0.919111188616516
0.4165539199480925      0.917835290485693     0.9184504173269846
0.41672647375592287     0.9174545881385561    0.9176863312511502
0.41688750926060897     0.9170981092845599    0.9171252133780292
0.417062030629512       0.9167104789915593    0.9168101076335253
0.41723336841772457     0.9163286041390936    0.9167574596028409
0.41739318790279284     0.9159712243459356    0.9167730583315118
0.417566493252078       0.9155824026793038    0.9166389089612268
0.41772828029821896     0.9152182141768442    0.9162423064667259
0.41788688376366934     0.9148600573050225    0.9156125439356615
0.41805897309333667     0.9144701741705893    0.9148198842508669
0.41821954411985973     0.9141051905489144    0.9141824706059721
0.41839360101059975     0.9137082468503224    0.9137704517954086
0.41856447432064925     0.9133172396916085    0.9136580704084024
0.4187238293275545      0.9129514061078       0.9136772390678884
0.41889667019867666     0.9125533195432342    0.913605971458948
0.41905799276665456     0.9121805459292806    0.9132877805316966
0.4192328011988494      0.9117752827549436    0.9126360293367562
0.4194044260503537      0.9113760551356068    0.9118284307360744
0.41956453259871374     0.9110024177364192    0.9111369874405724
0.4197381250112907      0.910595994424739     0.9106486871111513
0.4199001991207234      0.9102152769317851    0.9104835224560854
0.4200590896494656      0.9098408014896538    0.9104897013312466
0.42023146604242473     0.9094332419565225    0.910464066412759
0.4203923241322396      0.9090516938120847    0.9102179191892976
0.4205666680862714      0.9086368240946464    0.9096276158845827
0.42073782845961266     0.9082281789931835    0.908822904838799
0.42089747052980964     0.9078458260221499    0.9080745638118207
0.42107059846422357     0.9074298545195396    0.907486923489542
0.4212322080954932      0.9070403180023172    0.9072395073689307
0.42139063414607236     0.9066572917962981    0.9072134359319507
0.42156254606086846     0.9062403566020798    0.9072218461616828
0.4217229396725203      0.9058501311823768    0.9070518139527449
0.42189681914838906     0.9054257579846408    0.9065417696347896
0.42205918032111356     0.905028238802525     0.9058044148039505
0.42221835791314755     0.9046373337838505    0.9050105432716373
0.42239102136939843     0.9042119876596215    0.9043082407305127
0.42255216652250505     0.9038137730274102    0.9039455857485539
0.4227267975398286      0.9033808761545274    0.9038539476218185
0.42289824497646167     0.9029544969627329    0.9038769514476607
0.42305817410995045     0.9025555343865318    0.9037696992236887
0.4232315891076562      0.9021215880657272    0.9033392816904263
0.42339348580221764     0.9017152034501102    0.9026413866184944
0.4235521989160886      0.9013156266800552    0.9018269864832797
0.4237243978941765      0.9008807712603167    0.9010417546035699
0.4238850785691201      0.9004737566191204    0.9005811917087486
0.4240592451082807      0.9000312208875101    0.9004149120191658
0.4242302280667507      0.8995953960568037    0.900434205516933
0.42438969272207644     0.8991876987773655    0.9003851569093951
0.42456264324161913     0.8987441772806667    0.9000463229773539
0.42472407545801755     0.8983289292628892    0.8994093475487484
0.4248989935386329      0.8978776122103379    0.8985058561316256
0.4250707280385578      0.8974331117834007    0.8976632180043417
0.42523094423533836     0.8970171743759714    0.8971276104313534
0.4254046462963359      0.8965648618481918    0.8968959501610685
0.42556683005418916     0.8961412596821563    0.8968975267050391
0.42572583023135185     0.8957247688114675    0.8968888905622668
0.4258983162727315      0.895271603431154     0.8966322020284093
0.4260592840109669      0.8948474318047708    0.8960619007847755
0.4262337376134192      0.8943862552655891    0.8951728884373577
0.426405007635181       0.8939320413047438    0.8942713684387247
0.42656475935379856     0.8935071222129308    0.8936388554776886
0.42673799693663306     0.8930449657954216    0.8933083499424859
0.4268997162163233      0.8926122533684578    0.8932699509212941
0.427058251915323       0.8921868549854921    0.8932892315629104
0.4272302734785396      0.8917239204870533    0.8931177158482525
0.4273907767386119      0.8912907159717337    0.8926314775546063
0.4275647658629012      0.8908197297224638    0.8917817684916801
0.4277272366840462      0.8903786236850371    0.8908821071282038
0.4278865239245007      0.8899449417687546    0.8901395224654207
0.42805929702917217     0.8894731770592843    0.8896732578249432
0.42822055183069935     0.8890315807000579    0.8895575090449945
0.4283952924964435      0.8885516540776386    0.8895858290892275
0.4285668495814971      0.8880790544851002    0.8894829664317166
0.42872688836340644     0.8876369188836502    0.8890779174281634
0.42890041300953274     0.8871561442715598    0.8882804798552071
0.42906241935251477     0.886705984194056     0.8873633065187682
0.4292212421148062      0.8862634522248142    0.886543240171173
0.42939355074131463     0.8857819796244076    0.8859645124549126
0.42955434106467877     0.8853314102795576    0.8857693202827062
0.42972861725225986     0.8848416523636329    0.885783177572841
0.42989970985915044     0.8843594253065429    0.885742568546883
0.43005928416289674     0.8839083977589828    0.8854291342043921
0.43023234433086        0.8834178723050572    0.8847076386851076
0.430393886195679       0.8829586972128269    0.8837949446494908
0.4305689139247149      0.8824597744043493    0.8828243935255932
0.4307407580730603      0.8819684940770082    0.8821613623047463
0.4309010839182614      0.8815088625583505    0.8818991415659333
0.4310748956276794      0.8810091716672671    0.8818887652463892
0.4312371890339532      0.8805412815608937    0.8818876458424776
0.43139629885953645     0.8800813378790794    0.8816548230869884
0.43156889454933667     0.8795810303526427    0.881013543460351
0.4317299719359926      0.8791128150805441    0.8801228467492461
0.4319045351868655      0.8786039856853587    0.8790976409635702
0.4320759148570478      0.8781030045588922    0.8783260574917662
0.4322357762240859      0.8776344147454469    0.8779634798628541
0.4324091234553409      0.8771248839675171    0.8779006616646817
0.4325709523834516      0.8766477614697126    0.877927031367407
0.43272959773087183     0.8761787953848011    0.8777766607242188
0.432901728942509       0.8756685868494908    0.8772342728602407
0.4330623418510019      0.87519122639222      0.8763897523204197
0.43323644062371175     0.8746723744209012    0.875329107102495
0.4334073558157311      0.874161583307048     0.8744497290723209
0.43356675270460615     0.8736839412292595    0.8739701759292656
0.43373963545769817     0.8731644976452755    0.8738272259327975
0.4339009999076459      0.8726783564529568    0.8738600695569658
0.4340758502218106      0.8721501634411415    0.8737639580763029
0.43424751695528474     0.8716301473601459    0.8732944224819309
0.4344076653856146      0.8711437361168969    0.8724913260860298
0.4345812996801614      0.8706149615211686    0.8714146946357975
0.43474341567156394     0.870119945885834     0.8705001114351609
0.43490234808227596     0.8696334154251804    0.8699101206052082
0.43507476635720493     0.8691042176872885    0.8696755356460167
0.43523566632898963     0.868609073556786     0.8696920859885917
0.4354100521649913      0.8680710123387008    0.8696601424339963
0.4355812544203024      0.8675413404524736    0.8692955882236157
0.4357409383724692      0.8670460240262665    0.8685696025966296
0.43591410818885296     0.8665074795966813    0.8675009915747379
0.43607575970209245     0.8660034444419566    0.8665124437600091
0.4362342276346414      0.8655081062473416    0.8658043222333931
0.43640618143140736     0.8649692367481636    0.8654537258672387
0.436566616925029       0.8644651705703578    0.8654280771134369
0.43674053828286763     0.8639173237822098    0.8654443825589816
0.43690294133756197     0.8634044348969464    0.8652128826606181
0.43706216081156574     0.8629003583649765    0.8646068360576685
0.43723486614978646     0.8623521963844093    0.8635894301845346
0.4373960531848629      0.8618392878380894    0.8625423485333993
0.4375707260841563      0.8612820432869237    0.861629603892255
0.4377422154027592      0.8607335149591275    0.8611621024863418
0.4379021864182178      0.8602205428118286    0.8610769387536134
0.4380756432978934      0.8596629228212386    0.8611148421926722
0.4382375818744247      0.859141013284731     0.8609702182712153
0.43839633687026547     0.8586281285173324    0.8604643940029142
0.4385685777303232      0.8580702916922331    0.8595084879269356
0.43872930028723667     0.857548388004161     0.8584374139974433
0.4389035087083671      0.8569811733232227    0.8574166943387636
0.43907453354880693     0.8564228919275456    0.8568152414138606
0.4392340400861025      0.8559009316901106    0.8566442199717427
0.43940703248761503     0.8553334469872208    0.8566791549141148
0.4395685065859833      0.8548024385777296    0.8566149910126433
0.4397434665485685      0.8542256564999733    0.8561579124428034
0.4399152429304632      0.8536579279891061    0.8552540167996505
0.4400755010092136      0.8531269806666171    0.8541752040580998
0.440249244952181       0.8525499503712022    0.8530816888756272
0.4404114705920041      0.8520098567600677    0.852401692185609
0.4405705126511366      0.8514791277142769    0.8521362207381951
0.4407430405744861      0.850902014409528     0.8521440001834667
0.4409040501946913      0.8503621340308564    0.8521360485648927
0.4410785456791135      0.8497756217775461    0.8517909971656933
0.44124985758284513     0.8491983814779691    0.8509801529888393
0.4414096511834325      0.8486586770608655    0.8499188932240686
0.44158293064823684     0.8480720334246394    0.8487507301215612
0.4417446918098969      0.8475230801717828    0.8479450923162415
0.4419032693908664      0.8469837078810827    0.8475626757438439
0.4420753328360528      0.8463970970157654    0.8475142933686153
0.442235877978095       0.8458484708696723    0.8475457771081879
0.4424099089843541      0.8452523601129431    0.8473164750755809
0.44257242168746896     0.8446943889388174    0.8466678312794854
0.4427317508098933      0.8441461166419402    0.8456674716624849
0.4429045657965346      0.8435500597077075    0.8444446584375838
0.4430658624800316      0.8429924373923627    0.843496210125844
0.4432406450277456      0.8423867838634198    0.842921393201253
0.44341224399476903     0.8417907370073188    0.8427996389423486
0.4435723246586482      0.8412334264780519    0.8428450693633498
0.44374589118674435     0.8406277786924963    0.8427092476366465
0.4439079394116962      0.8400610211027062    0.8421699353742975
0.4440668040559575      0.8395041779302636    0.8412365772583561
0.44423915456443575     0.8388986994040905    0.839995878049025
0.4443999867697697      0.8383324042051508    0.8389484903925526
0.44457430483932064     0.8377172286548977    0.8382296523943871
0.44474543932818106     0.8371118759041359    0.8380057689881341
0.4449050555138972      0.836546006260177     0.8380409455505773
0.4450781575638303      0.8359307821006303    0.8379899342053864
0.4452397413106191      0.8353551834447026    0.8375734204610367
0.44539814147671736     0.8347897182670481    0.8367346577110882
0.44557002750703256     0.8341747598315272    0.8355066192713843
0.4457303952342035      0.8335997441563343    0.8343740251793423
0.4459042488255914      0.8329749939339843    0.833501067025502
0.446066584113835       0.8323903408657453    0.8331519302612808
0.4462257358213881      0.8318159413128668    0.8331354266971448
0.44639837339315813     0.8311915138653029    0.8331580860906014
0.4465594926617839      0.830607476595728     0.8328920278186241
0.44673409779462664     0.8299731703678896    0.8321001557629687
0.4469055193467788      0.8293490301337668    0.8309087747691767
0.4470654225957867      0.8287655791671119    0.8297194950808034
0.44723881170901153     0.8281315609571336    0.8287124470487958
0.4474006825190921      0.8275383846784934    0.8282334103476495
0.44755936974848215     0.8269556797366318    0.8281481824800582
0.44773154284208916     0.8263221175315398    0.828198562719094
0.4478921976325519      0.8257296869798716    0.8280378737046684
0.4480663382872316      0.8250861608304114    0.8273791827246576
0.44823729536122076     0.8244530195005821    0.8262571057041742
0.44839673413206566     0.8238613055575131    0.8250348911536611
0.4485696587671275      0.8232182011064441    0.8238977567679499
0.4487310650990451      0.822616674986954     0.8232695246155322
0.44890595729517957     0.821963520283417     0.8230827069145916
0.44907766591062354     0.8213208705669391    0.8231358676733709
0.4492378562229232      0.8207200946856423    0.823041535544732
0.4494115323994399      0.8200673956640819    0.8224808165257906
0.44957369027281224     0.8194567213193367    0.821481959952873
0.4497326645654941      0.8188568536602684    0.8202540369473035
0.4499051247223929      0.8182047764683946    0.8190153858739753
0.45006606657614745     0.8175950101917827    0.8182487082511744
0.45024049429411894     0.8169327990329827    0.8179427867082666
0.45041173843139987     0.8162813092063437    0.8179739450313317
0.4505714642655365      0.8156724224449761    0.8179541265606287
0.4507446759638901      0.8150107996171813    0.8175301877263352
0.45090636935909945     0.8143919289785729    0.8166376788701681
0.45106487917361826     0.8137840779352623    0.8154336606771034
0.45123687485235403     0.8131231569967804    0.8141105802660573
0.4513973522279455      0.8125051362184192    0.8131982914155963
0.451571315467754       0.8118338491120746    0.8127437592962587
0.45173376040441815     0.8112057612507564    0.8127146820556
0.45189302176039176     0.8105888150862648    0.8127556325153928
0.4520657689805823      0.8099183227610584    0.8124963397267838
0.4522269978976286      0.8092913142323501    0.8117631124873183
0.45240171267889184     0.8086105296150364    0.8105008462199339
0.45257324387946457     0.8079408061084468    0.8091248991814204
0.452733256776893       0.8073148562199399    0.8080886784252277
0.45290675553853843     0.8066348467515638    0.8074866323305038
0.45306873599703956     0.8059987589806027    0.8073795474855637
0.4532275328748502      0.8053740285033415    0.8074387152933483
0.45339981561687775     0.8046949633727247    0.8072915901512201
0.45356058005576105     0.8040600997584479    0.8066901742128272
0.45373483035886125     0.8033706754698428    0.8055151919231066
0.45390589708127094     0.8026925293209851    0.8041134270869132
0.45406544550053635     0.8020588696912762    0.8029590544658806
0.4542384797840187      0.8013703707301534    0.8021903495048265
0.4543999957643568      0.8007265039014174    0.8019726845177069
0.45457499760891185     0.8000275728194396    0.802027074991286
0.4547468158727764      0.7993400391816514    0.8019486468276559
0.45490711583349663     0.7986974214901401    0.8014417407303853
0.45508090165843385     0.797999462297632     0.800340682984825
0.4552431691802268      0.797346564075171     0.7990052848976217
0.45540225312132915     0.7967053533687937    0.7977611166287191
0.45557482292664847     0.7960085321704413    0.7968406509230318
0.4557358744288235      0.7953570459155719    0.7965031430825076
0.4559104117952155      0.7946497281500859    0.796520897646314
0.456081765580917       0.7939540206945929    0.7965186415270264
0.4562416010634742      0.7933039271817425    0.7961427415112644
0.4564149224102484      0.7925977297401384    0.7951664298272082
0.4565767254538783      0.7919372888009669    0.7938623768997727
0.4567353449168176      0.7912887433156541    0.7925457875431782
0.4569074502439739      0.7905838296316757    0.791467506421116
0.4570680372679859      0.7899249417283817    0.7909835726919315
0.45724211015621485     0.7892094684836173    0.7909284198383522
0.4574129994637533      0.7885058149070266    0.79097965862785
0.45757237046814747     0.7878483323664768    0.7907375489903953
0.4577452273367586      0.7871339286145742    0.7899145955736048
0.45790656590222545     0.7864659794435246    0.7886788784487723
0.45808139033190926     0.7857409483444684    0.7871802168719092
0.45825303118090255     0.785027858802147     0.7859923166155901
0.45841315372675157     0.7843614987028737    0.7853935925977471
0.4585867621368175      0.7836377937801153    0.7852653116708752
0.4587488522437391      0.7829609588935516    0.785333978760138
0.45890775876997025     0.7822963467832148    0.7851991053342944
0.45908015116041834     0.7815741353764466    0.7845231978779694
0.45924102524772215     0.7808990583878428    0.7833751747959554
0.4594153851992429      0.7801661730418272    0.7818576726551645
0.45958656157007316     0.7794454404336787    0.7805409900279973
0.45974621963775913     0.7787721108545845    0.7797834314317369
0.45991936356966207     0.7780407162855906    0.7795314513000132
0.4600809891984207      0.7773568621411117    0.779591384111888
0.4602394312464888      0.7766854355585286    0.7795526071786303
0.46041135915877385     0.7759556956281052    0.7790393738169291
0.4605717687679146      0.7752737544158352    0.7780127011299216
0.46074566424127233     0.7745332958175719    0.776514177790983
0.4609080414114858      0.773840772283495     0.7751477650881642
0.4610672350010087      0.7731607879283398    0.7741938551705548
0.4612399144547486      0.7724220401696406    0.7737535680015303
0.4614010756053442      0.7717314837294568    0.7737544294075155
0.4615757226201568      0.770981961776509     0.7737891851538381
0.4617471860542788      0.7702449116702134    0.7734179557273106
0.4619071311852565      0.7695563131513012    0.7725177693599128
0.4620805621804512      0.768808501154631     0.7710684100487321
0.4622424748725016      0.768109273896481     0.7696345797997993
0.4624012039838615      0.7674227848204216    0.7685358600188835
0.4625734189594383      0.7666768423577686    0.7679302852786067
0.4627341156318709      0.7659797348052517    0.767850182287578
0.4629082981685204      0.7652229767747454    0.7679243775317095
0.4630792971244794      0.7644788914286632    0.7676958632933932
0.46323877777729416     0.7637838950612402    0.7669470838712418
0.4634117442943258      0.7630290064617716    0.7655866134561676
0.46357319250821316     0.7623233368207112    0.7641145594334405
0.4637481265863175      0.7615575563292322    0.7627718670702807
0.4639198770837313      0.7608043982408415    0.762040564421335
0.4640801092780008      0.7601007241509211    0.7618849199366204
0.4642538273364873      0.7593367108225725    0.7619663315361591
0.4644160270918295      0.7586223116999682    0.7618471602289447
0.4645750432664812      0.7579209597862885    0.7612402494671185
0.46474754530534984     0.7571590401263466    0.7599860899181357
0.4649085290410742      0.7564469781421709    0.7585099467415138
0.46508299864101554     0.7556741607177824    0.7570457594120917
0.4652542846602663      0.7549143311124508    0.7561409706094836
0.4654140523763728      0.754204605829028     0.7558607374931806
0.46558730595669623     0.7534338959261879    0.755920370603396
0.4657490412338754      0.7527134166752588    0.7558975372772885
0.46590759293036405     0.7520061779448317    0.755445197939452
0.46607963049106965     0.7512377334917757    0.7543324961101674
0.466240149748631       0.7505197557824757    0.7528881199658972
0.46641415487040927     0.7497403907408634    0.7513231701493535
0.4665766416890433      0.7490116172514761    0.7502732559611781
0.4667359449269867      0.7482961896586372    0.7498076190268407
0.4669087340291471      0.7475191569739789    0.7497885807500274
0.46707000482816324     0.7467929489780681    0.7498465787262831
0.4672447614913963      0.7460049570317758    0.7495195689915743
0.4674163345739389      0.7452302550305537    0.7485502794561699
0.4675763893533372      0.744506613762376     0.7471635469077704
0.4677499299969524      0.7437209703784284    0.7455370181412752
0.4679119523374234      0.742986508662214     0.7443397177946826
0.46807079109720384     0.7422655788019603    0.7437160132270161
0.4682431157212012      0.7414824364631782    0.7436000109063974
0.4684039220420543      0.7407507015575079    0.7436857044044762
0.4685782142271243      0.7399565813927517    0.7435000059197846
0.46874932283150383     0.7391759393513169    0.7426995489746725
0.4689089131327391      0.7384469332689046    0.7414070018790478
0.4690819892981913      0.7376553312867206    0.7397512992060851
0.4692435471604992      0.736915482353893     0.7384134297490442
0.4694019214421166      0.7361893457664522    0.7376119215370925
0.469573781587951       0.7354004102561442    0.7373587244405021
0.4697341234306411      0.7346634462459659    0.7374368757281006
0.46990795113754813     0.7338635165637472    0.7373784428483122
0.4700702605413109      0.7331156005332925    0.736811962849994
0.4702293863643831      0.7323814181520548    0.735679500036707
0.47040199805167227     0.7315840587391779    0.7340470456039099
0.47056309143581715     0.730839014184963     0.7325730782619134
0.470737670684179       0.7300306303088043    0.7314787237302682
0.4709090663518503      0.7292360129815415    0.7310775978608006
0.47106894371637736     0.728493929862324     0.7311152465681889
0.47124230694512137     0.7276883106167551    0.7311431801107083
0.4714041518707211      0.7269353381119704    0.7307348390170983
0.47156281321563026     0.726196357270363     0.7297477132281106
0.4717349604247564      0.7253936511987923    0.7281731764921678
0.4718955893307382      0.7246438006598456    0.7266258419449878
0.472069704100937       0.7238300696060638    0.7253514909755254
0.4722406352904453      0.7230302852885249    0.7247705629551231
0.4724000481768093      0.722283567181976     0.7247288036150812
0.47257294692739027     0.7214727804301606    0.7248140356475993
0.47273432737482696     0.7207151679892226    0.724564467455059
0.4729091936864806      0.7198933351674672    0.7236306186545869
0.4730808764174437      0.7190855485187787    0.7221161449207304
0.4732410408452625      0.7183311428653594    0.720534263746482
0.4734146911372982      0.7175123331819597    0.719138500566946
0.4735768231261897      0.7167470105803058    0.7184374495797828
0.47373577153439067     0.7159959463735581    0.7182950750116283
0.4739082058068086      0.7151803018652669    0.7183969949994223
0.47406912177608224     0.7144183408911845    0.7182727902635032
0.47424352360957284     0.7135916548779503    0.717517320137405
0.4744147418623729      0.7127791864358823    0.7161169627774692
0.47457444181202874     0.7120205995495191    0.7145230257490077
0.4747476276259015      0.7111971126396625    0.7129862920817778
0.47490929513663        0.7104276089433783    0.7121020657118271
0.4750677790666679      0.7096725261827475    0.711823018432009
0.4752397488609228      0.7088523755739723    0.7119032494559243
0.4754002003520334      0.7080863962083165    0.7118889210071785
0.47557413770736096     0.7072552111999483    0.7113276527595506
0.47573655675954424     0.7064782970919278    0.7101564196053141
0.475895792231037       0.7057158925366421    0.7086018052761116
0.47606851356674673     0.7048881188668095    0.7069327840228158
0.4762297165993122      0.7041148002422493    0.7058244233547492
0.4764044054960946      0.7032758672673216    0.7053232872127219
0.4765759108121864      0.7024513873389874    0.7053490882179387
0.476735897825134       0.7016815576086394    0.705406029421451
0.4769093707022985      0.7008460540992791    0.7050124506727311
0.47707132527631874     0.7000652969261569    0.7039955077602621
0.47723009626964846     0.6992992078437407    0.7025073533304003
0.47740235312719514     0.6984672920765099    0.7007723868976414
0.47756309168159755     0.6976902993505825    0.6995024392413309
0.4777373161002169      0.6968473544254985    0.6988112028133934
0.4779083569381457      0.6960190458150196    0.6987415448096806
0.4780678794729302      0.6952458376700044    0.6988417194034147
0.4782408878719317      0.6944065268546897    0.6986140554081453
0.4784023779677889      0.6936224077496456    0.6977785692088887
0.47857735392786305     0.6927720646667285    0.6962311702771591
0.4787491463072467      0.6919364466555702    0.6944690179313212
0.47890942038348605     0.6911561930293175    0.693093548079434
0.47908318032394237     0.6903095702582033    0.6922599796957083
0.4792454219612544      0.6895184007192311    0.6920982810973236
0.47940448001787594     0.6887421340013524    0.6922052269979883
0.47957702393871443     0.6878993597465629    0.6921059892447586
0.47973804955640864     0.6871122014713108    0.6914389249355528
0.4799125610383198      0.6862584220732943    0.6900222952692293
0.4800838889395404      0.6854195183602054    0.6882588198038432
0.4802436985376167      0.6846363937368493    0.6867611174535289
0.48041699399991        0.6837865123857594    0.6857320826269604
0.480578771159059       0.6829924940829555    0.6854275744037899
0.4807373647375175      0.6822135194368943    0.6855059067971241
0.48090944418019294     0.6813676589885289    0.6855172590770645
0.4810700053197241      0.6805778150407135    0.6850310955102132
0.4812440523234722      0.6797209793648484    0.6837843774852952
0.4814149157465298      0.6788791630002442    0.6820626298276017
0.48157426086644306     0.6780935167572071    0.6804651403189773
0.4817470918505733      0.6772407526672576    0.6792316520449573
0.48190840453155925     0.6764442378075243    0.6787484874312224
0.48208320307676217     0.675580503459703     0.6787682676036186
0.48225481804127457     0.6747318678952867    0.6788380075703863
0.4824149147026427      0.67393963033878      0.6784757489378546
0.4825884972282278      0.6730800190443252    0.6773638323858518
0.4827505614506686      0.6722768292221       0.6757851972520392
0.4829094420924188      0.6714888937337322    0.6741297386135633
0.483081808598386       0.6706334972169329    0.6727253610113136
0.4832426568012089      0.6698347221366384    0.6720659874100176
0.4834169908682488      0.6689683930296917    0.6719827786249006
0.48358814135459816     0.6681173001605735    0.6720992027986255
0.48374777353780324     0.6673229687567489    0.6718916466136767
0.4839208915852253      0.666460973746809     0.670976103123816
0.48408249132950304     0.6656558124548135    0.6695057817470556
0.4842409074930903      0.6648660319647425    0.6678262213298695
0.4844128095208945      0.6640084844119524    0.66626056223905
0.4845731932455544      0.6632079011566893    0.6654030193983116
0.4847470628344313      0.662339465983491     0.6651715439142994
0.4849094141201639      0.6615280647216828    0.6652881649636065
0.48506858182520596     0.6607321132337717    0.66524228645202
0.48524123539446495     0.6598682120705505    0.6645813777213084
0.4854023706605797      0.6590614702758283    0.6632928982532131
0.48557699179091135     0.6581866986302564    0.6614634921499083
0.4857484293405525      0.6573273636504998    0.6597792843636242
0.4859083485870494      0.6565253126825147    0.658743874153897
0.48608175369776324     0.6556551382637404    0.658351447481951
0.4862436405053328      0.6548423122770447    0.6584312718110263
0.4864023437322118      0.6540450518556745    0.6584779773051581
0.48657453282330776     0.6531795801186072    0.6580080503133792
0.48673520361125944     0.6523715723983727    0.6568862162095201
0.48690936026342807     0.6514952813210214    0.6551211142425085
0.4870803333349062      0.650634545662211     0.6533436833390718
0.48723978810324003     0.6498313885822539    0.6521248708113179
0.48741272873579083     0.6489598646192402    0.6515359846248014
0.48757415106519736     0.6481459785035775    0.6515357004246584
0.48774905925882084     0.6472636582691893    0.6516434802041214
0.48792078387175375     0.6463969593486026    0.6513012234604996
0.4880809901815424      0.6455880073923557    0.65030752087385
0.488254682355548       0.644710543842209     0.6486119473799032
0.4884168562264093      0.6438908837722962    0.6468748314445937
0.4885758465165801      0.6430869581069697    0.6455060238108498
0.48874832267096785     0.6422144487175658    0.6447279745219011
0.48890928052221133     0.6413998119428244    0.6446227643510842
0.48908372423767177     0.6405165170057218    0.6447605594625376
0.4892549843724417      0.6396489525588398    0.644577448488656
0.48941472620406734     0.6388393946424005    0.6437684607041744
0.48958795389990994     0.6379611236836634    0.6422006667163097
0.4897496632926083      0.6371409109212143    0.640447269881863
0.48990818910461603     0.6363365320415844    0.6389389220029228
0.49008020078084075     0.6354633785695308    0.6379508315227918
0.4902406941539212      0.6346483742665298    0.637700803465638
0.4904146733912186      0.633764544966911     0.6378267064418395
0.4905771343253717      0.6329389136616169    0.6378009847364846
0.4907364116788343      0.6321291702669041    0.6372286231387319
0.49090917489651387     0.6312505464427591    0.6358666687042878
0.49107041981104915     0.6304302060719988    0.6341513646015351
0.4912451505898014      0.6295409398887855    0.6323599570620924
0.49141669778786307     0.6286675624511608    0.6311865869813106
0.49157672668278046     0.6278525520060572    0.6307842782168791
0.4917502414419148      0.6269685656574152    0.6308605522766807
0.4919122378979049      0.6261429898077104    0.6309250505250449
0.49207105077320445     0.6253333897774397    0.6305266420592308
0.492243349512721       0.6244547685364559    0.629347499329098
0.4924041299490932      0.623634633145138     0.6277038346690067
0.4925783962496824      0.6227454394871059    0.6258354014505818
0.49274947896958105     0.6218722255131577    0.6244724404055623
0.4929090433863354      0.6210575706032728    0.6238872500222171
0.4930820936673067      0.6201738176492876    0.6238691848791387
0.49324362564513374     0.6193486619901735    0.6239941332358011
0.49341864348717773     0.6184543763827376    0.6237093077579187
0.4935904777485312      0.6175761211381539    0.6226671587394004
0.4937507937067404      0.6167565305549796    0.6210946794477704
0.49392459552916657     0.6158677767275839    0.6191958127005743
0.49408687904844845     0.6150377228339243    0.6177667051855382
0.4942459789870398      0.6142237699088887    0.6170087230636946
0.49441856478984814     0.6133406248235863    0.6168737214130773
0.4945796322895122      0.6125162392572311    0.6170177567123217
0.49475418565339313     0.6116226379440435    0.6168850304130483
0.49492555543658356     0.6107451467022174    0.6160401103199779
0.4950854069166297      0.6099264726597362    0.6145929177724407
0.49525874426089284     0.6090385633820822    0.6126882526865168
0.4954205633020117      0.6082095003832915    0.6111220297270189
0.49557919876244        0.6073966074118966    0.6101736146507383
0.4957513200870853      0.6065144570764825    0.6098811338892115
0.4959119231085863      0.6056912025503045    0.6100064491520097
0.49608601199430424     0.6047986752760174    0.6100080632126212
0.49625691729933163     0.6039223300598002    0.6093773069451439
0.49641630430121475     0.6031049274447527    0.6080947912820311
0.4965891771673148      0.6022182392286206    0.6062303299471251
0.4967505317302706      0.6013905183708885    0.6045497318393004
0.49692537215744337     0.6004935017501332    0.6033198966813759
0.4970970290039256      0.5996127074920523    0.6029007486202549
0.49725716754726357     0.5987909214505001    0.6029888365920159
0.4974307919548185      0.5978998333603873    0.6030644202185622
0.4975928980592291      0.5970677752408792    0.6026244173375244
0.4977518205829492      0.5962519830530654    0.6015096941382817
0.49792422897088623     0.5953668862265936    0.5997214261987006
0.498085119055679       0.5945408531726747    0.5979727699659162
0.4982594950046887      0.5936455135331007    0.5965565039709818
0.4984306873730079      0.5927664570811487    0.5959469536212852
0.4985903614381828      0.5919464950699554    0.5959513144799841
0.4987635213675747      0.5910572303631964    0.5960923382030088
0.4989251629938223      0.5902270766451397    0.5958300732593206
0.49908362103937937     0.5894132392712911    0.5949110977108664
0.4992555649491534      0.5885301065160915    0.5932451732641353
0.4994159905557832      0.5877061086604135    0.5914636383898301
0.49958990202662984     0.5868128216229718    0.5898689822985742
0.49975229519433223     0.5859786830457423    0.5890632313049949
0.4999115047813441      0.5851608883475786    0.5889130514214409
0.5000842002325729      0.5842738181893717    0.5890751738001228
0.5002453773806574      0.5834459146045428    0.5890027490341657
0.500420040392959       0.5825487470161425    0.5882264114887876
0.5005915198245698      0.5816679457754672    0.5867014880474484
0.5007514809530365      0.580846325694549     0.5849284114289958
0.5009249279457201      0.5799554622008255    0.5832013953312413
0.5010868566352594      0.579123788226532     0.5822070456906461
0.5012456017441083      0.5783084973801123    0.5819085228244585
0.501417832717174       0.5774240166044382    0.5820401221685091
0.5015785453870955      0.5765987556731369    0.5820810241728839
0.5017527439212339      0.5757042985258657    0.581517046312409
0.5019237588746818      0.5748262482876554    0.5801724958166407
0.5020832555249853      0.574007395857865     0.5784489133548912
0.5022562380395059      0.5731193770261374    0.5766144598993035
0.5024177022508822      0.5722905588693934    0.5754232291985454
0.5025926523264754      0.5713925985908201    0.5749268992221859
0.5027644188213781      0.5705110680372731    0.575013792702736
0.5029246670131365      0.5696887361855338    0.5751147504088867
0.503098401069112       0.5687972985187602    0.5746948343251762
0.5032606168219431      0.5679650594428534    0.5735684563898938
0.5034196489940836      0.5671492499284858    0.5719238666065372
0.5035921670304412      0.5662643734086018    0.570029640736004
0.5037531667636544      0.5654386878141036    0.5686761640616864
0.5039276523610846      0.5645439672858186    0.5679868563112026
0.5040989543778243      0.5636657058599871    0.5679794824547645
0.5042587380914196      0.5628466237174713    0.5681317322983004
0.504432007669232       0.5619585525999654    0.5678930203169525
0.5045937589439001      0.5611296557154781    0.5669638898958503
0.5047523266378777      0.5603172078910816    0.5654386094104917
0.5049243801960722      0.5594358192929307    0.563521028850657
0.5050849154511224      0.5586135878800703    0.562013914616625
0.5052589365703897      0.5577224554912269    0.5611071847059247
0.5054214393865126      0.5568904722751278    0.560955627547479
0.505580758621945       0.5560749488095362    0.5611182545877239
0.5057535637215943      0.5551905789085284    0.5610719922383621
0.5059148505180994      0.554365335383799     0.5603980418779078
0.5060896231788213      0.5534712904208905    0.5588945122457473
0.5062612122588528      0.5525937397758452    0.556994564600693
0.50642128303574        0.5517752883569575    0.5553760470055701
0.5065948396768442      0.5508880975627876    0.5542757386335057
0.5067568780148041      0.5500599930477793    0.5539769256138484
0.5069157327720734      0.5492483570149748    0.5541067169786067
0.5070880733935597      0.5483680455281077    0.5541729527098498
0.5072488957119017      0.5475467881515714    0.5536901477521693
0.5074232038944606      0.5466569080354532    0.5523741160561719
0.507594328496329       0.5457835350635958    0.5505333500035533
0.5077539347950532      0.5449692761700412    0.5488259278498051
0.5079270269579943      0.5440864739905535    0.5475246634099937
0.5080886008177912      0.5432626634703339    0.5470449032548063
0.5082469910968974      0.542455320073136     0.5470999509491056
0.5084188672402207      0.5415795050990868    0.5472476409818982
0.5085792250803997      0.5407626393683763    0.5469568464162144
0.5087530687847955      0.5398773611690012    0.5458626785728866
0.5089153941860471      0.5390510107801579    0.5442232364177456
0.5090745360066082      0.5382411267380934    0.5424561079314754
0.5092471636913863      0.5373629078720852    0.5409270329069172
0.5094082730730202      0.5365435690097117    0.5401992367706125
0.5095828683188709      0.5356559592990842    0.5401211314616737
0.5097542799840311      0.534784856282442     0.5403036486735915
0.509914173346047       0.5339725808232234    0.5401656096415846
0.51008755257228        0.5330921196769416    0.5392804034253813
0.5102494134953687      0.5322704601723856    0.5377716753287697
0.5104080908377667      0.5314652555839333    0.5360041379937038
0.5105802540443818      0.5305919514717147    0.5343313810286683
0.5107408989478526      0.5297773926812265    0.5334091132901759
0.5109150297155403      0.5288948053515372    0.533174949017762
0.5110859769025374      0.5280287158299812    0.5333523366749925
0.5112454057863903      0.5272213104810388    0.533351480633615
0.5114183205344602      0.5263459704376051    0.5326924603773668
0.5115797169793859      0.5255292842061281    0.5313548443436245
0.5117545992885284      0.5246447392467817    0.5294427252007503
0.5119262980169804      0.5237766879459439    0.5276909647805004
0.5120864784422882      0.5229672236238955    0.5266361728460517
0.5122601447318128      0.5220900006737408    0.5262763442907515
0.5124222927181932      0.5212713314602415    0.5264134848530088
0.512581257123883       0.5204690894745895    0.5265061042132043
0.5127537073937899      0.5195991896083062    0.5260424934722927
0.5129146393605525      0.5187877731287722    0.5248796902164239
0.513089057191532       0.5179087817358916    0.5230435483936237
0.513260291441821       0.5170462651215761    0.5212100998892676
0.5134200073889656      0.5162421563668422    0.5199805198513038
0.5135932092003272      0.5153705814720051    0.5194304527863405
0.5137548927085446      0.5145573767543091    0.5194911165633566
0.5139133926360714      0.5137606193263152    0.5196493889743948
0.5140853784278152      0.5128966075232022    0.5193806254830485
0.5142458459164148      0.5120908762482215    0.5184211907497485
0.5144197992692312      0.5112178881254372    0.5167077026802697
0.5145822343189034      0.5104031388967251    0.514915164114297
0.514741485787885       0.5096047688917874    0.5134879182599797
0.5149142231210837      0.5087392549577181    0.5126791226342953
0.5150754421511381      0.5079318943718596    0.5125925241162236
0.5152501470454094      0.5070574828032808    0.5127912521123877
0.5154216683589902      0.5061994998826261    0.5126756701595075
0.5155816713694267      0.5053995796544065    0.5119045917347819
0.5157551602440801      0.5045327289445837    0.5103340129917937
0.5159171308155892      0.5037238954367415    0.5085453444118799
0.5160759178064078      0.5029313996426771    0.5069969437787083
0.5162481906614435      0.5020720936080802    0.5059897916368138
0.5164089452133348      0.5012707119085307    0.505759369175646
0.516583185629443       0.5004026189407751    0.5059399386800183
0.5167542424648608      0.49955091529692985   0.5059613759689047
0.5169137809971343      0.49875703784542025   0.5053925985660337
0.5170868053936246      0.4978965770933571    0.5040046996006602
0.5172483114869707      0.49709389321992753   0.5022605931763453
0.5174233034445338      0.4962247295410511    0.5004634126101
0.5175951118214064      0.4953719342614224    0.4993247830991932
0.5177554018951347      0.49457681254841485   0.49898258806887474
0.5179291778330799      0.49371534467179673   0.49912651538696734
0.5180914354678808      0.4929114984256642    0.4992293581547434
0.5182505095219913      0.49212391675335154   0.4988334907333236
0.5184230694403186      0.49127012198953246   0.49762864244057137
0.5185841110555016      0.4904738436450325    0.4959604564945136
0.5187586385349017      0.48961146172287084   0.4940962852826702
0.5189299824336112      0.48876540008260133   0.49278176360121895
0.5190898080291765      0.48797674407048475   0.49226972932114854
0.5192631194889586      0.4871221255803056    0.4923301158104041
0.5194249126455965      0.4863248569629619    0.49249822812955973
0.5195835222215439      0.48554379716358864   0.4922744704125275
0.5197556176617083      0.4846969152072187    0.49128074243496866
0.5199161947987284      0.4839072706633126    0.48972854436289787
0.5200902577999654      0.4830519194423863    0.4878322769996716
0.5202611372205119      0.4822129830126501    0.48634758988444443
0.5204204983379142      0.48143119096082043   0.48564153476368105
0.5205933453195333      0.48058385395327174   0.4855733736886843
0.5207546739980081      0.47979356209627694   0.48576977907827396
0.5209294885407         0.47893784339573486   0.4856675282495408
0.5211011195027013      0.4780983553152014    0.4848278199788416
0.5212612321615584      0.47731578733262664   0.48337966729212184
0.5214348306846324      0.47646794343720983   0.481491436799956
0.5215969109045621      0.4756769564912649    0.47996982702122554
0.5217558075438012      0.474902073906689     0.4790894727346216
0.5219281900472573      0.47406206484206037   0.4788779483193842
0.5220890542475691      0.4732787871901801    0.4790618829886743
0.522263404312098       0.4724305059971942    0.47909278504765446
0.5224345707959362      0.4715983874565581    0.4784616835771672
0.5225942189766303      0.47082286927374634   0.47717447472041913
0.5227673530215412      0.4699825044622282    0.4753291715536683
0.5229289687633079      0.469198673795696     0.4737039111574674
0.523087400924384       0.46843087339406403   0.4726402116375549
0.523259318949677       0.46759838160733813   0.47224910651171914
0.5234197186718258      0.4668222926147232    0.47237970405306146
0.5235936042581916      0.4659816399596858    0.47251642476280664
0.5237559715414131      0.4651973216587478    0.4721386025132123
0.523915155243944       0.46442899329176407   0.4710806081390705
0.5240878248106919      0.46359626122975334   0.4693443933494732
0.5242489760742955      0.4628197278500227    0.4676364174931346
0.5244236132021162      0.4619789223681987    0.4662619097066815
0.5245950667492463      0.46115416847543844   0.46569645050559244
0.5247550019932321      0.46038547182740375   0.4657467613306689
0.5249284231014348      0.45955267076917955   0.46594135728005065
0.5250903259064933      0.4587758554545388    0.46573017340391193
0.5252490451308612      0.45801494876497095   0.46486128659289244
0.5254212502194461      0.45719010343616845   0.4632514397130008
0.5255819370048866      0.45642110237146577   0.4615256598087269
0.5257561096545442      0.45558829903840975   0.4599959141237796
0.5259270987235113      0.45477146771583365   0.45923277709284666
0.5260865694893341      0.45401033327631557   0.4591645515565963
0.5262595261193738      0.4531855701819388    0.45937996216864974
0.5264209644462692      0.45241646556970716   0.4593242056094444
0.5265958886373816      0.4515840539907111    0.4585525530925313
0.5267676292478034      0.4507675662069374    0.45705192366331965
0.5269278515550809      0.4500065327336852    0.4553406093622587
0.5271015597265755      0.449182203032927     0.4537236129606756
0.5272637495949257      0.44841324936344157   0.45284905823028465
0.5274227558825855      0.44766006295938354   0.4526523384948736
0.5275952480344621      0.4468437531759337    0.4528493659743953
0.5277562218831945      0.44608266620097936   0.45290592636016475
0.5279306815961439      0.4452585980008938    0.452335182100556
0.5281019577284026      0.4444503598607718    0.4510021580924149
0.5282617155575171      0.4436971857458171    0.4493360526367484
0.5284349592508486      0.44288121022532134   0.4476210098039624
0.5285966846410358      0.4421202182408286    0.4465700248157858
0.5287552264505325      0.4413748953665933    0.4462141725306238
0.5289272541242461      0.4405669481174317    0.4463525851874056
0.5290877634948155      0.43981382699181826   0.4464976391069656
0.5292617587296018      0.4389982270905233    0.44613209912525176
0.5294242356612439      0.4382373710156615    0.44507154672185884
0.5295835290121954      0.43749213060188086   0.44350847872564614
0.5297563082273639      0.43668459225953643   0.44171626680696335
0.529917569139388       0.43593163682406866   0.4404580591752888
0.5300923159156291      0.43511653218436425   0.43985688056327016
0.5302638791111797      0.43431710855372463   0.43991182258333306
0.530423924003586       0.4335721011645103    0.440104168044997
0.5305974547602093      0.432765132573023     0.4399050204203306
0.5307594672136883      0.4320124957183422    0.4390277205218303
0.5309182960864768      0.43127537089216117   0.43757931864121935
0.5310906108234822      0.43047646988788707   0.4357741336007936
0.5312514072573433      0.4297317355160427    0.43438403044119944
0.5314256895554215      0.4289253771310909    0.4335925137353596
0.531596788272809       0.42813459736587184   0.433526447367707
0.5317563686870522      0.42739781328111226   0.43373464910486564
0.5319294349655125      0.4265995974506483    0.4336908525981635
0.5320909829408285      0.4258552906088451    0.43301452927484424
0.5322493473354539      0.42512638903176897   0.43171667588170093
0.5324211975942963      0.4243362448708021    0.42993786344334706
0.5325815295499944      0.4235998404483499    0.4284330687941331
0.5327553473699095      0.4228024772896474    0.42743736702410373
0.5329176468866803      0.42205885155113215   0.42720808145104183
0.5330767628227605      0.4213305673928702    0.42738287811337133
0.5332493646230577      0.4205414045109964    0.42748930890499487
0.5334104481202107      0.4198057018811922    0.42705771773326184
0.5335850174815805      0.41900927677172545   0.425830785115197
0.5337564032622598      0.41822825838745825   0.42411325120303944
0.5339162707397949      0.41750051964330365   0.42253737508053246
0.5340896240815469      0.416712254701896     0.42136981339631563
0.5342514591201547      0.4159771776313613    0.4209838963177013
0.534410110578072       0.4152573237659335    0.42109512255106285
0.5345822479002061      0.41447713619916077   0.4212768924690669
0.534742866919196       0.4137499588362004    0.4210193474280234
0.5349169718024028      0.41296260604985885   0.41999407950422607
0.5350878931049191      0.41219054234503555   0.41837766735886356
0.5352472961042911      0.4114713056896193    0.41675856146491236
0.5354201849678801      0.4106920924164578    0.4154220702683018
0.5355815555283249      0.4099656131274736    0.41485440397747214
0.5357564119529865      0.40917931781505057   0.4148834978246071
0.5359280847969576      0.40840824733936576   0.4150957699465196
0.5360882393377845      0.4076897250169632    0.41494891534417794
0.5362618797428284      0.4069115883697334    0.4140726516158589
0.5364240018447279      0.4061859054468612    0.4126362577583647
0.5365829403659369      0.4054752585212968    0.41101280841767945
0.5367553647513629      0.4047051950612484    0.4095491395407292
0.5369162708336446      0.4039874025622367    0.4088162742935661
0.5370906627801432      0.4032103561542404    0.4087215680453051
0.5372618711459513      0.40244841330863934   0.408945232683345
0.5374215612086151      0.40173855305379136   0.4089304659935365
0.5375947371354959      0.4009696431083475    0.4082569234808724
0.5377563947592324      0.40025272003034057   0.4069661989172886
0.5379148688022783      0.3995507090864644    0.405371548694943
0.5380868287095413      0.398789848680014     0.40379767102929365
0.5382472703136599      0.3980807899287051    0.40288847462480964
0.5384211977819955      0.39731304635082954   0.40263416546257963
0.5385836069471868      0.3965970073458962    0.4028180285081605
0.5387428325316875      0.3958958130065158    0.402931323445868
0.5389155439804053      0.39513613691317034   0.4025047412864031
0.5390767371259788      0.3944281782689226    0.4014248212038972
0.5392514161357692      0.3936619343849714    0.3997430731784276
0.5394229115648691      0.39291059826604613   0.39810222338856643
0.5395828886908247      0.3922105675957165    0.3970463862046056
0.5397563516809972      0.3914524452780973    0.3966339974189041
0.5399182963680255      0.39074552858539124   0.39675864110525316
0.5400770574743632      0.390053320018488     0.3969351546953745
0.5402493044449179      0.38930322140176904   0.3966819942864568
0.5404100331123284      0.3886041366545717    0.3957797354695467
0.5405842476439557      0.38784732758389784   0.39420353051643503
0.5407552785948926      0.3871052923438718    0.3925224237119694
0.5409147912426852      0.38641407405643996   0.39132232591823357
0.5410877897546947      0.38566533845663575   0.3907274632632025
0.54124926996356        0.38496731957623204   0.3907565514538624
0.5414242360366422      0.3842119505674731    0.3909796802005956
0.5415960185290339      0.38347128207652037   0.39083498227383257
0.5417562827182814      0.3827811316859414    0.39006131064758115
0.5419300327717457      0.3820338399906288    0.38858083910931884
0.5420922645220657      0.38133696529168276   0.38697809614565726
0.5422513126916952      0.3806545889106843    0.38567026966563794
0.5424238467255418      0.37991527537109926   0.38491124407701266
0.542584862456244       0.3792261846171636    0.38483171014683026
0.5427593640511632      0.37848032452211927   0.38505915543072017
0.5429306820653919      0.3777490292610476    0.38504640730325096
0.5430904817764762      0.3770677573604043    0.38444914030279537
0.5432637673517775      0.3763299257308222    0.3831164740643664
0.5434255346239345      0.375642015948858     0.3815383867465085
0.543584118315401       0.3749684679833787    0.38013641407307175
0.5437561878710845      0.37423856518730136   0.3792018156611304
0.5439167391236237      0.3735583892619669    0.37898349459870495
0.5440907762403798      0.37282202694781397   0.37917882045060125
0.5442532950539917      0.3721352891229105    0.37928448398575454
0.544412630286913       0.371462838748625     0.3789034352261639
0.5445854513840512      0.3707344086749496    0.37778399515029015
0.5447467541780452      0.37005540642979257   0.37627965353686776
0.5449215428362562      0.36932059438935616   0.3746563918250347
0.5450931479137766      0.36860013712506473   0.3735764727167356
0.5452532346881528      0.3679289821981571    0.37322230329764544
0.5454268073267459      0.3672023682985936    0.37335844893811715
0.5455888616621947      0.3665248620461068    0.37352906810347974
0.545747732416953       0.3658614991624216    0.37329875650728106
0.5459200890359283      0.3651427595430639    0.3723568176246251
0.5460809273517593      0.36447292808247367   0.3709450528988649
0.5462552515318072      0.3637478879500037    0.3692886184717455
0.5464263921311646      0.3630370549456125    0.3680644257656667
0.5465860144273778      0.3623749258898811    0.3675533631329383
0.5467591225878078      0.3616577970838753    0.3675951914811537
0.5469207124450935      0.360989268131008     0.36780408858830843
0.5470791187216888      0.3603347381137081    0.3677181258932149
0.547251010862501       0.3596254120082967    0.36697330468468314
0.547411384700169       0.35896448706762385   0.3656881100101192
0.5475852444020539      0.35824893343708497   0.3640323907647968
0.5477475858007945      0.3575816764796072    0.3627278847302939
0.5479067436188446      0.35692833975935734   0.36201062373085796
0.5480793873011116      0.3562205787888904    0.36189292899141834
0.5482405126802343      0.3555609150363135    0.36210095941902665
0.548415123923574       0.35484699506251344   0.36214338903273424
0.5485865515862233      0.3541470590696751    0.3615735988664146
0.5487464609457282      0.3534950161272703    0.3604228779454637
0.54891985616945        0.3527889258373521    0.35880346913599087
0.5490817330900276      0.35213062452154126   0.35741448743052945
0.5492404264299147      0.35148609903256794   0.3565484165156026
0.5494126056340186      0.35078772982003953   0.35628772201763903
0.5495732665349783      0.35013695092610736   0.35645835867475983
0.549747413300155       0.34943249567228857   0.35659517845000077
0.5499183764846411      0.34874187973854887   0.3562070238263079
0.550077821365983       0.3480986507608286    0.35521728699066246
0.5502507521115418      0.34740195347670483   0.3536700079494319
0.5504121645539564      0.3467525394865823    0.35221836443233107
0.5505870628605879      0.34604982518132393   0.35112910027459804
0.5507587775865289      0.3453608708729631    0.3507644500248815
0.5509189740093257      0.34471899573731557   0.3508920816445918
0.5510926562963393      0.3440240291087168    0.35107462674056306
0.5512548202802087      0.34337603759750096   0.3508342228588207
0.5514138006833875      0.34274159797104736   0.3499979590662703
0.5515862669507833      0.34205440862350367   0.3485446258344546
0.5517472149150348      0.3414140307498754    0.3470678833076182
0.5519216487435032      0.3407209453850364    0.3458464219245371
0.5520928989912812      0.34004146972461574   0.3453263656240503
0.5522526309359149      0.3394085523476193    0.3453720866463111
0.5524258487447655      0.3387231325779744    0.34559099243376584
0.5525875482504718      0.33808416634744914   0.3454899769415173
0.5527460641754877      0.3374586003229131    0.34482394192566235
0.5529180659647204      0.3367807313582649    0.3434972677882114
0.5530785494508088      0.33614911706380396   0.3420233368304399
0.5532525188011143      0.3354653637404291    0.34067990274135224
0.5534149698482754      0.3348277603656269    0.34000534180986747
0.5535742373147461      0.33420347593289423   0.33991319121256613
0.5537469906454336      0.33352725180404486   0.3401310080434567
0.5539082256729769      0.33289697880092955   0.3401753336625265
0.5540829465647372      0.33221492990765994   0.33963895415028045
0.5542544838758069      0.33154625959564615   0.33844527877726394
0.5544145028837323      0.3309233374381913    0.3370034748840872
0.5545880077558747      0.33024884213667827   0.3355775285701415
0.5547499943248728      0.3296199913946116    0.33476198557507947
0.5549087973131804      0.32900431203373853   0.3345481333462915
0.555081086165705       0.32833725669365516   0.3347298713340708
0.5552418567150852      0.32771564924071167   0.3348554707246727
0.5554161131286824      0.327042827836793     0.33448972046252423
0.5555871859615892      0.32638323679856557   0.33345505290253363
0.5557467404913516      0.32576889286172567   0.3320748516627678
0.555919780885331       0.3251035355217376    0.3305864455611637
0.5560813029761661      0.32448332280755404   0.3296267224971971
0.5562563109312181      0.3238122585921648    0.32926124860677736
0.5564281353055796      0.32315434335748816   0.3294024348603926
0.5565884413767967      0.3225413720628326    0.32956723504538404
0.556762233312231       0.3218777496363427    0.3293112776896688
0.5569245069445209      0.32125896871599074   0.3284549504242112
0.5570835969961203      0.3206531318987314    0.3271540009600756
0.5572561729119366      0.3199968388029064    0.3256394921757161
0.5574172305246087      0.31938519276067195   0.3245666990296192
0.5575917740014977      0.31872325056559614   0.324054978932161
0.5577631338976962      0.31807434324428774   0.3241180698673091
0.5579229754907503      0.3174699873889839    0.32431639854935435
0.5580963029480215      0.31681554422249947   0.32419949697676426
0.5582581121021484      0.316205437290489     0.323503251018168
0.5584167376755848      0.3156081256523785    0.32230878818004927
0.558588849113238       0.31496091727720155   0.320794012918108
0.558749442247747       0.31435785152398743   0.31961545417072146
0.558923521246473       0.3137050456262255    0.3189418179023172
0.5590944166645084      0.3130650900052031    0.31889579158617437
0.5592537937793995      0.31246907977300553   0.31910008425324965
0.5594266567585077      0.3118235225077944    0.31911044099759933
0.5595880014344715      0.31122180987014203   0.3185867674007893
0.5597628319746523      0.3105707060719954    0.31739221257353195
0.5599344789341425      0.30993237015774766   0.3159013903153004
0.5600946075904885      0.3093376824144108    0.3146686127284972
0.5602682221110514      0.30869379569784083   0.3138884472403969
0.5604303183284701      0.3080934567753912    0.31375024357661446
0.5605892309651983      0.30750568496683894   0.3139332813076968
0.5607616294661434      0.30686890067177197   0.3140326981449298
0.5609225096639442      0.3062754744863212    0.31365777927208577
0.5610968757259619      0.3056331890452124    0.3126100082702662
0.5612680582072891      0.3050035243917821    0.3111690942692445
0.561427722385472       0.30441702464947834   0.3098741749914138
0.5616008724278719      0.30378185457017287   0.3089482137651885
0.5617625041671276      0.3031897506926436    0.3086797034760774
0.5619209523256927      0.3026100701677174    0.3088129936689933
0.5620928863484748      0.30198190259709534   0.3089807273002926
0.5622533020681125      0.3013966147212979    0.30875551101159815
0.5624272036519673      0.30076299040743976   0.30787790785773234
0.5625895869326778      0.30017214745924153   0.30659153715421483
0.5627487866326977      0.2995936495150129    0.3052499442273181
0.5629214721969346      0.29896699754290146   0.3041529944149422
0.5630826394580272      0.2983829411692589    0.30370344538006505
0.5632572925833367      0.297750880645999     0.3037564392826792
0.5634287621279557      0.29713121749480975   0.30395805426953887
0.5635887133694304      0.29655396076694435   0.3038517005312856
0.5637621504751221      0.2959288846266753    0.3031317249767442
0.5639240692776696      0.29534611824662604   0.3019421084234006
0.5640828044995264      0.2947756071777344    0.3005995660814022
0.5642550255856003      0.2941575009860971    0.299397907896959
0.5644157283685298      0.29358151492236645   0.2988101987684149
0.5645899170156763      0.2929580423446066    0.2987584294049258
0.5647609220821322      0.2923468219428151    0.2989682092311845
0.5649204088454439      0.2917775348082868    0.29897184346603584
0.5650933814729726      0.2911609410771951    0.2984222890356475
0.565254835797357       0.29058618507137157   0.29735639762066113
0.5654297759859583      0.28996426767335776   0.2959012362447822
0.5656015325938691      0.28935452247776244   0.29464656670934875
0.5657617708986357      0.2887864294116766    0.2939698059071561
0.5659354950676191      0.2881713534109159    0.2938367138688035
0.5660977009334582      0.28759783462997096   0.29402380055274363
0.5662567232186069      0.2870362982112746    0.294105117280979
0.5664292313679725      0.2864279517022499    0.2937021897436761
0.5665902212141939      0.28586098373125346   0.29276116518513196
0.5667646969246322      0.28524734767405113   0.29135480986878637
0.5669359890543799      0.28464574206677307   0.29003629457620267
0.5670957628809834      0.2840853333541755    0.28923474784489744
0.5672690225718039      0.2834784310521276    0.2889744789514926
0.5674307639594801      0.2829126327570127    0.2891213268037144
0.5675893217664657      0.2823586791156028    0.289264271883419
0.5677613654376683      0.2817584008609664    0.28901031722993
0.5679218908057266      0.28119905178186744   0.2882141977382985
0.5680959020380019      0.28059351691675355   0.28688629955358197
0.5682583949671328      0.28002881896630155   0.2855847898578934
0.5684177043155733      0.27947589061318523   0.2846340413512875
0.5685904995282307      0.27887694404493346   0.28419515628723385
0.5687517764377439      0.27831866077834577   0.2842559726908355
0.568926539211474       0.2777144969648864    0.28445191180089474
0.5690981184045135      0.2771221491484945    0.28431684326024753
0.5692581792944088      0.27657028819340795   0.28365456112536563
0.569431726048521       0.2759727157819681    0.28241913120247436
0.5695937544994889      0.27541553999115803   0.2811094716863969
0.5697525993697663      0.2748699996953038    0.2800670429281894
0.5699249301042607      0.27427891164979024   0.2794925575835548
0.5700857425356108      0.2737280504587837    0.2794665389354195
0.5702600408311779      0.27313177650651965   0.27967060454715253
0.5704311555460544      0.2725472228685072    0.2796477528726146
0.5705907519577866      0.2720027207075411    0.2791311928677781
0.5707638342337358      0.27141297325078506   0.27801441914535524
0.5709253982065406      0.270863188744622     0.2767203550460008
0.571083778598655       0.2703249071501288    0.2755964904710947
0.5712556448549864      0.26974153951567664   0.2748763434722397
0.5714159928081735      0.26919796855754424   0.2747386241438955
0.5715898266255774      0.26860944239215123   0.27492138521812926
0.5717521421398372      0.26806062511184414   0.2750013488309589
0.5719112740734064      0.2675232385374096    0.2746641909454149
0.5720838918711926      0.2669410543771296    0.27371913021646754
0.5722449913658345      0.26639841425622585   0.2724789577414649
0.5724195767246933      0.26581110603558555   0.27117030755739807
0.5725909785028617      0.26523527127555024   0.27032895227270615
0.5727508619778857      0.26469881320405025   0.2700811039662859
0.5729242313171267      0.26411784576341457   0.27021936381332795
0.5730860823532233      0.2635761693783742    0.2703566067427261
0.5732447498086295      0.2630457951655061    0.27014489513014056
0.5734169031282527      0.26247106507929113   0.2693436571690222
0.5735775381447316      0.2619354653116453    0.2681748781013694
0.5737516590254274      0.2613556357676446    0.2668335098458152
0.5739225963254326      0.2607871499117863    0.2658708329720753
0.5740820153222936      0.2602576312123318    0.26549521045107677
0.5742549201833715      0.2596840372719835    0.26556047107593744
0.5744163067413052      0.2591493269974069    0.2657339008957704
0.5745911791634558      0.2585706661840612    0.2656075310448204
0.5747628680049159      0.25800327761332303   0.26490879134880985
0.5749230385432318      0.25747461109827074   0.26380460433629593
0.5750966949457645      0.2569021468108822    0.2624621130842495
0.575258833045153       0.2563683218673502    0.2614685800587481
0.575417787563851       0.25584560372017534   0.2609779725202345
0.5755902279467658      0.25527923548163983   0.2609595597785768
0.5757511500265364      0.2547513514402173    0.2611412539757538
0.575925557970524       0.2541799386239033    0.261118748902721
0.5760967823338211      0.2536196722594916    0.26056313115885166
0.5762564883939739      0.25309773266702973   0.25955855009774803
0.5764296803183436      0.25253241287520006   0.25822897911835074
0.5765913539395691      0.2520053318384503    0.25715387608856694
0.576749843980104       0.2514892259112816    0.2565387233801349
0.5769218198848559      0.2509298819871921    0.2564123127764129
0.5770822774864636      0.2504086346427052    0.2565783413370316
0.5772562209522881      0.24984426744836308   0.2566468682939131
0.5774186461149684      0.2493179171635883    0.25627589156559694
0.5775778876969582      0.24880248665940025   0.25541659148149076
0.5777506151431648      0.2482440785069616    0.25413590907209804
0.5779118242862272      0.24772353782785778   0.2529816342742431
0.5780865192935066      0.24716013631488473   0.25216037090274307
0.5782580307200955      0.24660769274068597   0.25192765999974487
0.5784180238435401      0.2460929650180513    0.2520587755071887
0.5785915028312016      0.2455355195136867    0.2521862508664171
0.5787534635157189      0.2450157122556608    0.25193559121416836
0.5789122406195456      0.24450670770425362   0.25119741378743987
0.5790845035875892      0.24395512362684463   0.24998295287018701
0.5792452482524886      0.2434410322898498    0.24879769613384176
0.5794194787816049      0.24288447499955956   0.24786239375864658
0.5795905257300308      0.24233875738956967   0.2475052673278752
0.5797500543753124      0.2418303849055158    0.247577904171044
0.5799230688848108      0.24127968550752935   0.2477453213939587
0.580084565091165       0.2407662556796784    0.2476132813694622
0.5802595471617362      0.24021061113145598   0.24692023157291157
0.5804313456516168      0.2396657411405693    0.24576544285741603
0.5805916258383531      0.23915799464291637   0.2445769167340446
0.5807653918893064      0.2386081707165473    0.24357722432807719
0.5809276396371155      0.23809539549977182   0.24314405457620214
0.581086703804234       0.23759324525130934   0.2431491679493349
0.5812592538355694      0.23704915021311415   0.24332947485028836
0.5814202855637606      0.2365419637786427    0.243288593094027
0.5815948031561687      0.23599294150965586   0.24272927463065036
0.5817661371678864      0.23545457938578435   0.2416679227185686
0.5819259528764598      0.23495298379241544   0.24048549994550594
0.58209925444925        0.23440968564748202   0.23940172200580465
0.582261037718896       0.23390308129694887   0.2388509529350955
0.5824196374078515      0.23340699159764486   0.23876781301328961
0.5825917229610239      0.23286932807396807   0.2389389145398445
0.582752290211052       0.23236822213577313   0.23898011340249944
0.5829263433252971      0.2318255866669833    0.23856323825060857
0.5830972128588517      0.23129348394715865   0.237618089590775
0.5832565640892621      0.2307978064800695    0.236464424761823
0.5834294011838893      0.23026078511278877   0.23530898741408365
0.5835907199753723      0.22976011882714978   0.23463396774224393
0.5837655246310722      0.2292182138485596    0.23444383177708955
0.5839371457060816      0.22868679948146964   0.23459425768958111
0.5840972484779466      0.2281916042623956    0.23467883110319365
0.5842708371140287      0.22765529932647274   0.2343559209183164
0.5844329074469665      0.22715514398103048   0.2335533264289745
0.5845917941992138      0.22666534011382108   0.23244623140455575
0.584764166815678       0.2261345512625444    0.23124935202964825
0.5849250211289979      0.2256397814260344    0.23047261838675015
0.5850993613065348      0.2251041290733553    0.23016852987923736
0.5852705179033811      0.22457886195953616   0.23027272664427964
0.5854301561970832      0.22408948131237186   0.23040239794191808
0.5856032803550022      0.2235593437343559    0.23020221214325662
0.585764886209777       0.22306502479702628   0.22952385996766111
0.5859233084838612      0.22258095542847356   0.22848556699849887
0.5860952166221624      0.2220562505521743    0.22726549848710853
0.5862556064573192      0.221567237037038     0.22638902900069818
0.5864294821566931      0.22103768776314647   0.22595475877640134
0.5865918395529227      0.22054376263325406   0.22597899083073925
0.5867510133684618      0.2200600315253995    0.22613611803473255
0.5869236730482178      0.21953588473223504   0.22607369332868613
0.5870848144248295      0.21904723598637624   0.22555835366751284
0.5872594416656581      0.218518270192581     0.22451639506210513
0.5874308853257962      0.21799953069050557   0.22329842775998776
0.5875908106827901      0.21751616128414325   0.22234829749130047
0.5877642219040009      0.21699259563957765   0.22179826777711273
0.5879261148220675      0.21650433460359916   0.22174404511154552
0.5880848241594435      0.21602616886691156   0.22189722740528525
0.5882570193610365      0.2155079236592372    0.22191873167175524
0.5884176962594851      0.21502486025656772   0.22152567043512897
0.5885918590221508      0.21450181330411988   0.2205924623065647
0.5887628382041259      0.21398889292811937   0.2193977202918432
0.5889222990829567      0.2135110297631143    0.21838295276234762
0.5890952458260045      0.21299327148142747   0.21770929718832274
0.589256674265908       0.2125104112376867    0.21755614721599412
0.5894315885700284      0.2119877699742562    0.2177000771782905
0.5896033192934583      0.21147520412727985   0.21776632349807537
0.589763531713744       0.21099751885166404   0.21745310981429172
0.5899372299982466      0.2104801705689317    0.2166027519371315
0.590099409979605       0.20999764056185158   0.215499868112009
0.5902584063802727      0.20952506136758645   0.2144459478652831
0.5904308886451575      0.20901293355497788   0.21367045309818644
0.590591852606898       0.20853550655584827   0.21342142946573847
0.5907663024328553      0.20801862492165452   0.21352328838647539
0.5909375686781222      0.20751172638095303   0.21363914429570569
0.5910973166202448      0.2070394091506785    0.21343291551125326
0.5912705504265844      0.20652775293283415   0.2127034638466315
0.5914322659297797      0.20605061693677282   0.21165795067588736
0.5915907978522845      0.2055833418447268    0.2105788722717493
0.5917628156390062      0.20507683984328853   0.20970161091942988
0.5919233151225837      0.2046047429006465    0.20934203656696954
0.592097300470378       0.20409351113448979   0.20937806198688869
0.5922597675150282      0.2036166236778881    0.20952184411241878
0.5924190509789877      0.20314954812325364   0.20943882628426305
0.5925918203071643      0.2026434490550577    0.20887066822017625
0.5927530713321966      0.2021715797933103    0.2079213281803198
0.5929278082214458      0.20166077846965297   0.20672608438425355
0.5930993615300044      0.20115982034294835   0.2057722674104442
0.5932593965354188      0.20069297555763468   0.20531406046152553
0.5934329174050502      0.20018731120099822   0.20527652375169456
0.5935949199715373      0.19971570064254912   0.20542519067594533
0.5937537389573339      0.19925381418243576   0.2054185114817302
0.5939260438073474      0.19875321715267322   0.20497160101255407
0.5940868303542166      0.19828656171510922   0.20411229015640622
0.5942611027653028      0.1977812852639881    0.202935795895564
0.5944321915956984      0.19728576351236773   0.20191221148297697
0.5945917621229497      0.19682406922831425   0.20134716970716457
0.5947648185144181      0.1963238640753761    0.20121554790585647
0.5949263566027422      0.1958574280441837    0.20134918811312877
0.5951013805552832      0.19535257003983517   0.20139891808680857
0.5952732209271336      0.19485741805529655   0.20103067798941773
0.5954335429958398      0.19439580821037783   0.20023953249834014
0.5956073509287629      0.19389581921470278   0.1990913106412177
0.5957696405585418      0.19342944110255714   0.19807949414476844
0.5959287466076301      0.19297265811087275   0.19742566215223709
0.5961013385209354      0.19247765739526485   0.1972022825250797
0.5962624121310964      0.19201616003181615   0.1973052396303163
0.5964369716054744      0.1915165340558549    0.19740506753489012
0.5966083474991618      0.19102653600347658   0.1971454009099055
0.596768205089705       0.190569931449433     0.19645567654772206
0.596941548544465       0.19007530870468203   0.19535791674550707
0.5971033736960809      0.1896140234084015    0.1943127988218986
0.5972620152670062      0.18916225393182268   0.19356829318919488
0.5974341427021485      0.18867257371422108   0.19323750838610262
0.5975947518341465      0.1882161243877743    0.19328991277821733
0.5977688468303614      0.18772185262591223   0.19342439750446475
0.5979397582458857      0.18723712918935245   0.19327323600130838
0.5980991513582659      0.18678552777890967   0.1926990114569787
0.598272030334863       0.186296213358773     0.19167452550333444
0.5984333910083158      0.18583996542169007   0.19061292719021042
0.5986082375459856      0.18534609281769554   0.1897149870537297
0.5987799005029648      0.18486172484709987   0.18931183060978804
0.5989400451567998      0.1844103144284534    0.1893209353121921
0.5991136756748516      0.18392138880461462   0.18946554349188383
0.5992757878897592      0.18346536515135273   0.18939433500766098
0.5994347165239763      0.18301873496458432   0.1889244050127102
0.5996071310224104      0.18253469608972925   0.18798076208690137
0.5997680272177002      0.1820834535231423    0.18692463366047518
0.5999424092772069      0.18159488980608493   0.18595394163650927
0.6001136077560231      0.1811157517614357    0.1854440780591566
0.600273287931695       0.18066930208204557   0.18538021274735378
0.6004464539715838      0.18018563971860002   0.18552284245924833
0.6006081017083283      0.17973461064548082   0.1855273248723047
0.6007665658643824      0.17929289709075522   0.18516949756163217
0.6009385158846534      0.17881407639232172   0.1843264261425239
0.6010989476017802      0.17836778428957673   0.18329502828448224
0.6012728651831238      0.17788447178024677   0.18226165846233852
0.6014352644613232      0.1774336327522766    0.18165704643708505
0.6015944801588321      0.17699206220922664   0.1814818679870851
0.6017671817205579      0.1765133684540496    0.18159234142806943
0.6019283649791395      0.1760670584883831    0.18167041095391115
0.602103034101938       0.17558390581159297   0.18140161795707666
0.602274519644046       0.17511006481886268   0.18065701627130817
0.6024344868830097      0.1746685027353802    0.17966539027756778
0.6026079399861903      0.17419020884100847   0.1785951539652083
0.6027698747862267      0.17374414048565284   0.1779018246964437
0.6029286260055725      0.17330727707929475   0.17763623395903044
0.6031008630891354      0.17283379033509702   0.17770145643820995
0.6032615818695539      0.17239242647557618   0.17781395431972274
0.6034357865141894      0.17191452845333005   0.17764928372269093
0.6036068075781343      0.17144587103585815   0.1770181353977568
0.603766310338935       0.1710092310178382    0.1760855247530542
0.6039392989639526      0.17053616833264373   0.174993669036868
0.6041007692858259      0.1700950693692448    0.1742125387778778
0.6042757254719162      0.1696176377870402    0.17382830213581865
0.604447498077316       0.16914940791688765   0.17385464982288423
0.6046077523795715      0.1687130352663447    0.17397934211384986
0.604781492546044       0.1682404425796044    0.17387780532822977
0.6049437144093721      0.16779965291506962   0.1733644485831511
0.6051027526920097      0.16736795657318898   0.1724978293471933
0.6052752768388643      0.16690015031476582   0.1714061607940936
0.6054362826825745      0.16646404283832245   0.1705596245480858
0.6056107743905018      0.16599191597179827   0.17007488621284989
0.6057820825177386      0.16552891860812324   0.17003359296123838
0.6059418723418311      0.16509751292017902   0.17016246804130597
0.6061151480301404      0.16463020101906548   0.1701407107712325
0.6062769054153055      0.16419442628932346   0.16973480384265335
0.6064354792197801      0.16376767205910747   0.16895140674010198
0.6066075388884716      0.16330512230556363   0.16787799618001892
0.606768080254019       0.16287400490566137   0.16697312206193973
0.6069421074837832      0.16240718298091      0.16638011466952018
0.6071046164104031      0.16197173838710702   0.16624988030019983
0.6072639417563326      0.1615452746200542    0.16635908852859443
0.6074367529664789      0.161083218090835     0.16641863516087282
0.607598045873481       0.16065243306997568   0.16614680347159896
0.6077728246447001      0.16018614716177093   0.16539376986345697
0.6079444198352286      0.15972876255806812   0.1643540540707842
0.6081044967226129      0.15930244371598212   0.16341296804283362
0.6082780594742141      0.1588407223620023    0.16273025628039187
0.608440103922671       0.1584101260217585    0.1625180743512355
0.6085989647904374      0.1579884437519653    0.16259461763945468
0.6087713115224207      0.15753147472479068   0.16269525317829792
0.6089321399512597      0.15710552630017666   0.1625164721303618
0.6091064542443158      0.1566443863084876    0.16187026027476945
0.6092775849566813      0.1561922014015191    0.16088291083089257
0.6094371973659025      0.15577092914258753   0.15991724250424544
0.6096102956393407      0.15531458536776052   0.15914387279308911
0.6097718756096345      0.15488909946053114   0.15883620258911801
0.6099302719992379      0.1544724590928524    0.158861331524298
0.6101021542530581      0.1540208651977696    0.1589873954142213
0.6102625182037341      0.15360002251957589   0.15889957418051573
0.6104363680186271      0.15314432334500075   0.15837343983564228
0.6105986995303758      0.15271931956725704   0.15751040596263796
0.610757847461434       0.152303124005859     0.15653599762146442
0.6109304812567091      0.15185219212534146   0.15566016607176247
0.61109159674884        0.15143184697445233   0.15522445621096997
0.6112661981051878      0.1509768643271434    0.15517040617097286
0.611437615880845       0.15053073232197622   0.15530130762738814
0.611597515353358       0.15011507473416766   0.15528257807498572
0.611770900690088       0.14966490422450326   0.15486379328396277
0.6119327677236737      0.1492451511315297    0.1540786700253425
0.6120914511765688      0.1488341346950971    0.1531204495147124
0.6122636204936809      0.14838872774506467   0.15218699434558816
0.6124242715076487      0.1479736273076526    0.15165842982591313
0.6125984083858335      0.1475242370176828    0.15152075137735654
0.6127693616833277      0.1470836264733213    0.15163840766783035
0.6129287966776777      0.14667320783663004   0.1516807646542798
0.6131017175362447      0.1462286262014413    0.1513755853980848
0.6132631200916673      0.14581417830379817   0.15068551273485345
0.6134380085113069      0.14536566989406122   0.14966125117458925
0.6136097133502559      0.14492590257656368   0.14870227585945284
0.6137698998860607      0.14451615228898615   0.14811684629826022
0.6139435722860824      0.1440724706260685    0.1479186870680207
0.6141057263829599      0.1436587467529185    0.14800950574658214
0.6142646968991469      0.14325345883975127   0.14809172619609773
0.6144371532795507      0.1428143252259992    0.14788149852008442
0.6145980913568103      0.1424050511087401    0.14728326372474693
0.6147725152982869      0.14196206196525607   0.14630798285263177
0.6149437556590729      0.14152774796703688   0.14532038342221024
0.6151034777167146      0.14112317658788948   0.14465545617138695
0.6152766856385733      0.14068502540393027   0.1443640793165672
0.6154383752572877      0.14027655766674985   0.14441305587712328
0.6155968812953116      0.139876646756227     0.1445229122409052
0.6157688731975525      0.13944328942251305   0.14440714399588478
0.615929346796649       0.13903949918687933   0.14391297015354318
0.6161033062599626      0.13860237224211025   0.1430069316634297
0.6162657474201318      0.13819475164507297   0.1420526111402408
0.6164250049996106      0.13779564980595924   0.14129530175799956
0.6165977484433063      0.13736334797804858   0.14087659470306535
0.6167589735838577      0.13696043308781206   0.14085007947427083
0.616933684588626       0.1365244306574494    0.1409800630479179
0.6171052120127037      0.1360969976846912    0.14093737796209338
0.6172652211336372      0.13569882766595018   0.14053673293557348
0.6174387161187878      0.13526771270588905   0.13970588971614792
0.617600692800794       0.13486579790769768   0.13876043938099955
0.6177594859021097      0.13447232700883952   0.13795035860844826
0.6179317648676423      0.13404605175630757   0.13743926783047242
0.6180925255300307      0.1336488533833424    0.13734399652183651
0.618266772056636       0.1332189662900551    0.13746492472002034
0.6184378350025508      0.13279757494782546   0.13748906702237118
0.6185973796453212      0.1324051325369563    0.1371885887038746
0.6187704101523087      0.13198014795209187   0.13645038959265082
0.6189319223561519      0.13158404748308775   0.1355311210912551
0.619106920424212       0.13115552326376295   0.13460066066879461
0.6192787349115816      0.13073545474175488   0.1340337263494664
0.6194390310958069      0.13034413918351498   0.13388940903525348
0.6196128131442492      0.12992054991533009   0.13399465743404412
0.6197750768895472      0.1295256471670665    0.1340578247094292
0.6199341570541547      0.1291390672544948    0.13384190846377395
0.6201067230829791      0.12872036148885205   0.13319385390446117
0.6202677708086592      0.12833021196602398   0.13231371254569863
0.6204423043985563      0.1279079960203572    0.1313544347024184
0.6206136544077628      0.12749401393253393   0.13070675783492094
0.6207734861138251      0.12710846784174318   0.13048472922968632
0.6209468036841044      0.12669105746676024   0.13055429418979161
0.6211086029512394      0.12630201609303135   0.1306527310196061
0.6212672186376839      0.12592122186660573   0.13052251541763857
0.6214393201883452      0.1255087182593617    0.1299777466055644
0.6215999034358624      0.1251244511701934    0.12915393192760907
0.6217739725475964      0.124708602131095     0.1281796954724735
0.6219448580786399      0.12430105930926387   0.1274521067912537
0.6221042253065391      0.12392161503265227   0.12714081924486242
0.6222770783986553      0.12351075098421996   0.1271552241627451
0.6224384131876273      0.1231279156947489    0.12727410733948827
0.6226132338408161      0.12271379171247704   0.12720317472750572
0.6227848709133145      0.12230793336946891   0.12673134510002299
0.6229449896826685      0.1219299616300343    0.12595671925969273
0.6231185943162396      0.1215208679470387    0.12498759173929697
0.6232806806466664      0.12113958895692613   0.12424587909942447
0.6234395833964026      0.12076643182004139   0.12385566985694979
0.6236119720103558      0.12036231742667511   0.12380930690592595
0.6237728423211647      0.11998587582930564   0.12393033002739777
0.6239471984961905      0.11957861244339764   0.12392976001306254
0.6241183710905258      0.1191795331044095    0.12355991641665987
0.6242780253817168      0.11880797894530887   0.12286048258915977
0.6244511655371249      0.11840577518653413   0.12190819564256361
0.6246127873893886      0.11803102190954429   0.12111532602836937
0.6247712256609618      0.11766430331851588   0.12064028007985374
0.6249431497967519      0.11726710509697089   0.12051664132341267
0.6251035556293978      0.11689721017310646   0.12062309629952016
0.6252774473262606      0.11649697543589758   0.12068358237837006
0.6254398207199791      0.1161239671830682    0.12044356496873086
0.625599010533007       0.11575894616023802   0.11985215808599162
0.625771686210252       0.11536375977148143   0.11894370592134204
0.6259328435843527      0.11499564857952874   0.11810311866742326
0.6261074868226704      0.11459751550712258   0.1174829462454506
0.6262789464802975      0.1142074349748642    0.11728483000258322
0.6264388878347803      0.11384427240437885   0.11736455430743639
0.62661231505348        0.11345127012844558   0.11746419509772113
0.6267742239690355      0.11308499401013138   0.11730924533777096
0.6269329493039004      0.11272657187625726   0.11680699142410512
0.6271051605029824      0.11233847804864722   0.11595240027075214
0.62726585339892        0.11197707867887756   0.11509717513702393
0.6274400321590746      0.11158615713097968   0.11440194980461885
0.6276110273385387      0.11120320215975621   0.11411729521130372
0.6277705042148585      0.11084678040511746   0.11415393569873875
0.6279434669553953      0.11046102689733      0.11427916170187367
0.6281049113927878      0.11010172507399903   0.1142077277639819
0.6282798416943972      0.10971324541106708   0.11374256217315354
0.6284515884153161      0.10933268546772891   0.11293542233905643
0.6286118168330908      0.10897841092659104   0.11208388538135544
0.6287855311150823      0.10859515448194172   0.11134876987482381
0.6289477270939295      0.10823809934710775   0.111014628094277
0.6291067394920863      0.10788879578989755   0.11100317256149553
0.62927923775446        0.10751070392072186   0.11113585131377736
0.6294402177136895      0.10715864723270828   0.11112860900056983
0.6296146835371359      0.10677796147837827   0.11076116468197958
0.6297859657798918      0.10640510079834162   0.11002674122675653
0.6299457297195034      0.10605810238369433   0.10918713898594887
0.6301189795233318      0.1056826774472486    0.10839954026047578
0.6302807110240161      0.10533302729760938   0.107985593351521
0.6304392589440098      0.1049910273077786    0.10791183844610011
0.6306112927282205      0.10462080080534675   0.10803633652360609
0.6307718082092869      0.10427617653765127   0.10808652732018215
0.6309458095545702      0.10390349028703867   0.10782260866789428
0.6311082925967093      0.10355631624350424   0.10721582906188765
0.6312675920581579      0.10321673714885865   0.10640969473346705
0.6314403773838233      0.10284930147914512   0.10557030938497763
0.6316016444063445      0.10250720051917518   0.10505545835078609
0.6317763972930827      0.10213741193618454   0.10488824329157581
0.6319479665991303      0.10177529557173874   0.10499226680008408
0.6321080176020337      0.10143832935620986   0.10508114801902817
0.632281554469154       0.10107389015191137   0.10490437764740469
0.63244357303313        0.10073450768626999   0.10438084470759347
0.6326024080164155      0.10040261066140836   0.10361816577583652
0.6327747288639181      0.10004345246516817   0.1027610271662654
0.6329355314082763      0.09970915977583471   0.10218123216617706
0.6331098198168514      0.0993476802986646    0.10193476867438285
0.6332809246447361      0.09899376791381882   0.10200266565050697
0.6334405111694764      0.09866454515006982   0.10211984314124124
0.6336135835584337      0.09830844985408616   0.10203011559870441
0.6337751376442466      0.09797694812632973   0.10160071983145277
0.6339335081493691      0.09765282143254955   0.10089750037915242
0.6341053645187086      0.09730204209802115   0.10003687862836326
0.6342657025849038      0.09697566681303407   0.09939531517959035
0.6344395265153159      0.0966228197778209    0.09905973632164912
0.6346018321425837      0.09629427791586513   0.09906787707869613
0.634760954189161       0.09597305056811684   0.09919872556110523
0.6349335620999552      0.09562557734175085   0.099204275779449
0.6350946517076052      0.09530221429249629   0.09889547909899145
0.6352692271794721      0.09495279115394516   0.09820454214487054
0.6354406190706485      0.09461076760607073   0.0973578955593311
0.6356004926586807      0.09429265144467203   0.09667395970499616
0.6357738521109297      0.09394871126078663   0.09626163629238092
0.6359356932600345      0.09362857583275036   0.09621373979000583
0.6360943508284488      0.09331563478161042   0.09633876298139528
0.63626649426108        0.09297710269365013   0.09640204197291016
0.6364271193905668      0.09266217324412224   0.09618316050934954
0.6366012303842707      0.09232184440478836   0.09557912022403822
0.6367721577972841      0.09198879713888253   0.09476225393035138
0.6369315669071532      0.09167914243112121   0.0940439471009922
0.6371044618812393      0.09134433166074604   0.09355121536105991
0.637265838552181       0.09103280709719602   0.09343390489899298
0.6374407010873397      0.09069632311846795   0.0935524207073866
0.637612380041808       0.09036705665562467   0.09364737856451942
0.6377725406931319      0.09006086058186061   0.09348791628234045
0.6379461872086728      0.08972995468358853   0.09295022817957352
0.6381083154210694      0.08942200990792051   0.09220811482642739
0.6382672600527755      0.08912106416274156   0.09147378492121112
0.6384396905486984      0.08879565478819348   0.09091777546303999
0.6386006027414771      0.08849299170846298   0.0907354119435197
0.6387750007984728      0.08816606749720637   0.09082264340409285
0.638946215274778       0.08784623373008422   0.09095071018397399
0.6391059114479389      0.08754892315828151   0.09086861027071184
0.6392790934853168      0.08722759049765058   0.09042440515018414
0.6394407572195503      0.08692865354432625   0.0897349871647445
0.6395992373730933      0.08663658194632462   0.08899576930862924
0.6397712033908532      0.08632075697482913   0.08837847526608383
0.6399316511054689      0.08602712307401882   0.0881222467202585
0.6401055846843016      0.08570994446662335   0.08816204337478271
0.64026799995999        0.08541484125729608   0.08830442134320078
0.6404272316549878      0.0851265307717187    0.08830894403516636
0.6405999492142026      0.08481493556427076   0.08798556972853179
0.6407611484702731      0.08452518845600096   0.08737702557081513
0.6409358335905606      0.08421237045544294   0.08657350636976432
0.6411073351301575      0.08390644145509586   0.08591384995721452
0.6412673183666101      0.08362212458013253   0.08559420152958316
0.6414407874672797      0.08331500792956412   0.08558334284222358
0.6416027382648051      0.08302938388683967   0.08572777566910864
0.6417615054816399      0.08275040985830186   0.08578616380336342
0.6419337585626916      0.08244890332108479   0.08555337633232833
0.6420944933405991      0.0821686548011262    0.08501853861969094
0.6422687139827236      0.08186609359896352   0.08424283539081587
0.6424397510441575      0.0815702808566875    0.08354730219260349
0.6425992698024472      0.08129548269218517   0.08316024415708756
0.6427722744249538      0.08099865039557583   0.08308554289600381
0.6429337607443161      0.08072270934904933   0.08321803755963864
0.6431087329278953      0.08042495926457464   0.08332301116814321
0.643280521530784       0.08013388125907282   0.083150196601031
0.6434407918305284      0.079863445015311     0.0826716721587834
0.6436145479944898      0.07957148487972789   0.08192825560129988
0.643776785855307       0.07930004010383346   0.08125219078778423
0.6439358401354336      0.07903501465194923   0.08081205481273739
0.6441083802797771      0.07874874627974543   0.0806774964601625
0.6442694021209764      0.07848274548939572   0.08078872087561216
0.6444439098263927      0.07819573286224904   0.08092862087709468
0.6446152339511183      0.07791524237955334   0.08083545643038138
0.6447750397726997      0.07765476253284191   0.08043659746422607
0.6449483314584981      0.0773735634196719    0.07974253403719694
0.6451101048411522      0.07711224473454878   0.07905712925605798
0.6452686946431158      0.07685718782700493   0.07856433365043466
0.6454407703092963      0.07658169982967845   0.07836065340594432
0.6456013276723325      0.07632587320664472   0.07843736777964781
0.6457753708995857      0.07604985968318165   0.07860083769941328
0.6459378958236947      0.07579333514376965   0.07859370144209289
0.646097237167113       0.07554298534611291   0.07829891797668645
0.6462700643747483      0.07527273869011127   0.07768089971216637
0.6464313732792394      0.07502171968704295   0.07699783935779811
0.6466061680479473      0.07475104562531171   0.07640102107301025
0.6467777792359648      0.0744866527195273    0.07613794666690517
0.646937872120838       0.0742412167664058    0.07617711243376778
0.6471114508699282      0.07397643147064303   0.07634941529773533
0.6472735113158741      0.07373046588324132   0.07639931040542355
0.6474323881811295      0.07349050701034225   0.07618324530004165
0.6476047509106018      0.07323149970255624   0.07563551830448337
0.6477655953369298      0.07299104387711047   0.07497317043336346
0.6479399256274747      0.07273178708539618   0.07434242245573891
0.6481110723373291      0.07247864547096136   0.074014857307076
0.6482707007440393      0.0722437775764229    0.0740052758370846
0.6484438150149664      0.0719904214594089    0.0741738254781467
0.6486054109827493      0.07175519821446398   0.07427516147851483
0.6487638233698416      0.07152580849817221   0.07414139453901547
0.6489357216211509      0.07127823828472109   0.07367634588170918
0.6490961015693159      0.07104852588664501   0.07304830104953207
0.6492699673816978      0.07080088607600202   0.07239249787900505
0.6494323148909356      0.07057096022494154   0.07200962819674839
0.6495914788194826      0.07034677359044372   0.07192592594602137
0.6497641286122467      0.0701049736135544    0.07206863523985155
0.6499252601018666      0.0698806067048072    0.07221717883380414
0.6500998774557033      0.0696388847332176    0.07215972588667889
0.6502713112288495      0.06940301464245782   0.07177342704977495
0.6504312266988515      0.06918428688745383   0.07118830074471473
0.6506046280330704      0.06894853027710211   0.0705252189469067
0.6507665110641451      0.06872976854715084   0.07009441263742108
0.6509252105145291      0.06851656383442828   0.06995507271348231
0.6510973958291302      0.06828665114515999   0.07006750219724657
0.651258062840587       0.06807344562215911   0.07023848767525895
0.6514322157162606      0.06784379601102916   0.07025475698545854
0.6516031850112438      0.0676198181072839    0.06995578525583576
0.6517626360030827      0.06741227666040889   0.0694261601278179
0.6519355728591386      0.0671887160301506    0.06876717917198732
0.6520969914120502      0.06698140532199015   0.06829109332191993
0.6522718958291788      0.0667582609141276    0.06808475636553579
0.6524436166656168      0.06654068621279394   0.06817521623305152
0.6526038191989105      0.06633905644208307   0.06835624026431374
0.6527775075964212      0.06612193083844842   0.06841872314284551
0.6529396776907875      0.06592059536362961   0.06820116494780276
0.6530986642044634      0.06572452003358309   0.06772963879630368
0.6532711365823562      0.06551328083380172   0.0670885650153739
0.6534320906571048      0.06531753065928833   0.06658337373597856
0.6536065305960703      0.06510688959033849   0.0663198353542658
0.6537777869543453      0.0649016253207796    0.06636855460138123
0.653937525009476       0.06471153898964735   0.06655277970965838
0.6541107489288237      0.06450690569070913   0.06667084865441114
0.6542724545450271      0.06431729241924408   0.06653249721309146
0.65443097658054        0.06413274016700407   0.06612956156820539
0.6546029844802698      0.06393397878342442   0.06551859168297526
0.6547634740768553      0.06374993043387583   0.06499033574375711
0.6549374495376578      0.06355195076836213   0.06466650098388274
0.6550999066953159      0.06336852340661585   0.06465641603023667
0.6552591802722836      0.06319004861405522   0.0648270068528441
0.6554319397134682      0.06299798627141089   0.06499956338195062
0.6555931808515086      0.06282016377077483   0.06495623403709869
0.6557679078537659      0.06262903638843967   0.0645988102945102
0.6559394512753327      0.06244298199635843   0.0640265481399163
0.6560994763937552      0.06227084473094769   0.06348999073380758
0.6562729873763946      0.0620857585805682    0.06312013247106678
0.6564349800558897      0.06191442572096618   0.06306295188431749
0.6565937891546944      0.06174783822164478   0.06321485627055173
0.656766084117716       0.061568651468582605  0.06341677405039835
0.6569268607775933      0.06140289966172784   0.06343985303827669
0.6571011233016876      0.061224836103285406  0.063164016862569
0.6572722022450913      0.06105164006905047   0.06264214781345584
0.6574317628853508      0.06089155023456836   0.062105428427563406
0.6576048093898271      0.06071951063945404   0.06169041827107225
0.6577663375911592      0.06056041033220808   0.06157886659407741
0.6579413516567083      0.06038965270899881   0.06172173185327646
0.6581131821415669      0.060223782812692625  0.06193943069259202
0.6582734943232812      0.06007055701792873   0.062004788274591265
0.6584472923692124      0.05990605513485968   0.061787176653290324
0.6586095721119993      0.05975397402190061   0.06133642280503539
0.6587686682740957      0.059606302845361295  0.060811216464377316
0.6589412503004092      0.05944771430804059   0.06036624604051997
0.6591023140235783      0.059301215176772136  0.060208903535000705
0.6592768636109644      0.05914409363076432   0.06031881507934049
0.6594482296176599      0.05899150462340547   0.06054842998218271
0.6596080773212112      0.05885066348747162   0.06066669913098172
0.6597814108889793      0.058699570334283656  0.06052810560777801
0.6599432261536032      0.058560051498356865  0.06013747260333129
0.6601018578375366      0.05842471812621138   0.059632863227707694
0.660273975385687       0.058279495915778606  0.05916186643900832
0.6604345746306931      0.05814551218667651   0.0589547150084859
0.6606086597399161      0.05800193823659949   0.05902111520185062
0.6607795612684486      0.05786267367211615   0.059251928428056704
0.6609389444938368      0.05773430146378432   0.059418410224122684
0.661111813583442       0.057596713986538636  0.05936267958132281
0.6612731643699028      0.057469842971531744  0.05904249218936322
0.6614480010205807      0.05733405956779116   0.05851781825202028
0.661619654090568       0.05720246366715056   0.05804210991709349
0.6617797888574111      0.05708123317591506   0.05780965194803362
0.661953409488471       0.05695147057118971   0.05784897843016328
0.6621155118163867      0.056831894970876234  0.0580650921868461
0.6622744305636119      0.056716150097193836  0.05826647147640311
0.662446835175054       0.05659224594515927   0.05828123082092754
0.6626077214833518      0.056478183666645726  0.05802862442571649
0.6627820936558666      0.05635626865609519   0.05754502676821444
0.6629532822476909      0.05623831137366548   0.05706114391974528
0.6631129525363709      0.0561298403269393    0.05678799497484075
0.6632861086892679      0.05601390137171895   0.05677942514875801
0.6634477465390205      0.0559072679103691    0.056979865627998186
0.6636062008080826      0.0558042307268983    0.05720933975367046
0.6637781409413618      0.05569410286896831   0.05729482393047549
0.6639385627714967      0.05559293095356774   0.05711731975185976
0.6641124704658484      0.055484978538516506  0.056686715289425545
0.6642748598570559      0.05538582402034667   0.056225031819541624
0.6644340656675729      0.055290331115905984  0.05590147911069038
0.6646067573423068      0.055188457573726306  0.05582285276261333
0.6647679307138964      0.055094984655563374  0.055987159923605055
0.664942589949703       0.054995443229343285  0.05625916795146106
0.6651140656048191      0.05489949193952581   0.05640195682341582
0.665274022956791       0.054811573991273135  0.056292505964479586
0.6654474661729797      0.05471797820401644   0.05591798297374962
0.6656093910860241      0.054632228920056314  0.05546975907462724
0.6657681324183781      0.05454969611881887   0.05512155006645476
0.6659403596149489      0.05446186781925506   0.05499590978095464
0.6661010685083755      0.05438152614911593   0.05512895760119348
0.666275263266019       0.05429620331269262   0.05540904259575481
0.6664462744429721      0.05421422376750064   0.05560645686991129
0.6666057673167809      0.05413936079899347   0.055569725805240215
0.6667787460548066      0.05405991006863395   0.05526172637204514
0.666940206489688       0.05398738766717103   0.054836556350870244
0.6671151527887864      0.05391059524673227   0.05443873646018974
0.6672869155071942      0.05383701110559805   0.05429016923175591
0.6674471599224577      0.053769981277631974  0.05440502424419503
0.6676208902019383      0.05369907908517046   0.054689030997253786
0.6677831021782745      0.05363454093451196   0.0549140321324003
0.6679421305739203      0.0535728305433236    0.054940360838816114
0.6681146448337829      0.05350763696237819   0.05469705820450974
0.6682756407905013      0.053448440968586856  0.05430255606054042
0.6684501226114367      0.0533860817267674    0.05389402552624624
0.6686214208516815      0.053326679230543746  0.053704817522874246
0.6687812007887821      0.05327289753926143   0.05378330080219
0.6689544665900996      0.053216352985522374  0.05406050775829493
0.6691162140882728      0.05316523755782673   0.054321775641120604
0.6692747780057555      0.05311669554076327   0.05441318209311333
0.6694468277874551      0.053065782465015955  0.05424307817796307
0.6696073592660104      0.053019929406571874  0.053888969852379294
0.6697813766087828      0.052972027404312116  0.0534763883910121
0.6699438756484108      0.05292899172286227   0.05325031242186756
0.6701031911073484      0.05288839089170178   0.05327180458768182
0.6702759924305028      0.052846137110387534  0.05352305571397056
0.670437275450513       0.05280837662936855   0.05381545162318479
0.6706120443347402      0.05276942776563585   0.053986989659177274
0.6707836296382768      0.052733173358585086  0.05388438965486241
0.670943696638669       0.0527010091734717    0.053574146687119156
0.6711172495032783      0.052667942874348415  0.0531689357510082
0.6712792840647434      0.05263876980179488   0.0529152099273137
0.6714381350455179      0.05261176316048761   0.05289807441113225
0.6716104718905093      0.0525842487587146    0.05312584581003587
0.6717712904323564      0.0525602498386519    0.053434001782955674
0.6719455948384205      0.052536067324783935  0.05366354628627962
0.672116715663794       0.05251417824224729   0.053634818085439476
0.6722763181860233      0.05249541707546695   0.05337768268729324
0.6724494065724695      0.052476876699662875  0.05298805208094952
0.6726109766557715      0.052461266883475584  0.05270831702131746
0.672769363158383       0.05244755579707152   0.05264735371670835
0.6729412355252113      0.05243446049215288   0.05284115434587442
0.6731015895888954      0.05242391733695037   0.0531559048139001
0.6732754295167965      0.05241431466466203   0.05343983564966352
0.6734377511415534      0.05240706535320744   0.053494117027703396
0.6735968891856197      0.05240156892170276   0.05331107119222819
0.6737695130939029      0.05239741089721562   0.05295150511157857
0.6739306186990418      0.052395225047035844  0.052643979056655306
0.6741052101683976      0.05239470469835387   0.052524777215944105
0.674276618057063       0.05239606536447978   0.052685833981625074
0.674436507642584       0.052399007127805826  0.05299978707495776
0.6746098830923221      0.05240402241671683   0.05332531718725312
0.6747717402389158      0.0524104196540779    0.05344514858402982
0.6749304138048191      0.05241829961949619   0.05332503445165987
0.6751025732349393      0.05242865166952892   0.05300262481809899
0.6752632143619152      0.05244000381826514   0.052688374636943154
0.6754373413531081      0.05245415567924274   0.0525284771659683
0.6756082847636103      0.052469918580198664  0.052647744285511736
0.6757677098709683      0.052486289801403345  0.05295108774372634
0.6759406208425434      0.05250586942699551   0.05331191543554725
0.6761020135109741      0.05252585787183817   0.05349885278807295
0.6762768920436217      0.05254938476475744   0.05343135709601902
0.6764485869955789      0.05257437484188978   0.053142801556333996
0.6766087636443917      0.05259937912431969   0.052834184106968295
0.6767824261574216      0.0526283335447281    0.05265616015520364
0.6769445703673072      0.05265735547698691   0.05274099203254829
0.6771035309965021      0.05268744432332373   0.05302992002917332
0.6772759774899141      0.05272190522709062   0.05341505857740098
0.6774369056801818      0.05275577214034203   0.0536581924838266
0.6776113197346664      0.05279433873610196   0.053662695155983285
0.6777825502084605      0.052834084968028446  0.05342305385378196
0.6779422623791103      0.052872839543209774  0.05312167304804035
0.6781154604139771      0.05291670116593991   0.05291256065324181
0.6782771401456996      0.052959368434806114  0.052955307295483194
0.6784356362967316      0.05300280994364129   0.053219993965927415
0.6786076183119806      0.053051755667722675  0.05362105555976963
0.6787680820240852      0.053099120030595706  0.05391899013708856
0.6789420316004068      0.05315231505791029   0.054002116240223254
0.6791044628735841      0.05320372521062225   0.05383578667417871
0.6792637105660708      0.05325575658294201   0.05355228712564107
0.6794364441227746      0.05331401739748831   0.05330652564610505
0.6795976593763341      0.053370104749932055  0.053288450432719146
0.6797723604941105      0.053432749300883896  0.05354110719864529
0.6799438780311964      0.05349613967870259   0.053949719043658244
0.680103877265138       0.05355695877251682   0.05429142538679527
0.6802773623632966      0.053624742584520414  0.0544447423797011
0.680439329158311       0.053689752331227365  0.05433660998442443
0.6805981123726347      0.053755102340253345  0.0540794772158294
0.6807703814511754      0.05382781440273621   0.053819639573225504
0.6809311322265719      0.053897365223881326  0.05376251082270685
0.6811053688661852      0.05397460467917791   0.053976934466568205
0.6812764219251081      0.054052308559876604  0.05438321722384273
0.6814359566808866      0.054126454795758526  0.054764550069427584
0.6816089773008822      0.05420869573690397   0.054991780995174204
0.6817704796177335      0.05428717698265084   0.05495132173496812
0.6819454677988017      0.054374080817239094  0.05470283134631134
0.6821172723991794      0.05446129478625581   0.05444343012228808
0.6822775586964128      0.0545443510869926    0.054367746186479646
0.6824513308578631      0.05463623764605105   0.05456087547058306
0.6826135847161692      0.054723763673766565  0.05494403722450895
0.6827726549937847      0.054811194107817755  0.05535485664702761
0.6829452111356172      0.054907852298259854  0.0556456627124692
0.6831062489743055      0.05499984376689139   0.05566944061480256
0.6832807726772107      0.055101591113767195  0.05546445608201418
0.6834521127994253      0.055203360280501206  0.05520419528388435
0.6836119346184957      0.055299963911737235  0.055095423066330716
0.683785242301783       0.05540654571013626   0.055245404903355805
0.6839470316819259      0.05550775799201484   0.05561185378609638
0.6841056374813784      0.05560858411719669   0.05604555892273733
0.6842777291450479      0.055719779850349836  0.05640051396977018
0.6844383025055731      0.0558252181987481    0.05649620318788531
0.6846123617303153      0.05594134785632551   0.05634641634860552
0.6847832373743669      0.05605720996157027   0.05609352790758224
0.6849425947152742      0.05616691837203709   0.05595192996277547
0.6851154379203985      0.05628771680011971   0.05605034978825233
0.6852767628223786      0.05640215934551066   0.05638799685176257
0.6854515735885756      0.05652801379986976   0.056881516132815266
0.685623200774082       0.056653441872054566  0.05728057053027471
0.6857833096564442      0.05677211744861206   0.05742880013487116
0.6859569044030233      0.05690260384249206   0.05732471805561526
0.6861189808464581      0.057026135381760726  0.05709858034147072
0.6862778737092023      0.05714883580635202   0.056935179536454175
0.6864502524361636      0.05728373526025825   0.05698922575192231
0.6866111128599807      0.057411295089683685  0.05729620501214268
0.6867854591480146      0.0575513729903468    0.057796298962465915
0.686956621855358       0.05769073734279783   0.05824752996452015
0.6871162662595571      0.05782236895350699   0.05846690519258383
0.6872893965279732      0.05796691402170401   0.05842966102170939
0.6874510084932449      0.05810352578102231   0.05822847950158999
0.6876094368778262      0.05823902284336027   0.05804693689031722
0.6877813511266244      0.05838781811786295   0.05805141748318788
0.6879417470722784      0.05852829873794141   0.05831649112522854
0.6881156288821493      0.058682393779001245  0.05881043233466449
0.6882779923888759      0.05882797342363248   0.059286689338893736
0.688437172314912       0.05897228384110475   0.05959253059027671
0.688609838105165       0.05913059367234416   0.0596478543268818
0.6887709855922737      0.05928000642618455   0.05949107463412827
0.6889456189435994      0.05944373498445908   0.059280384595554805
0.6891170687142346      0.05960631084713161   0.059245152989976554
0.6892770001817256      0.059759599697900005  0.05947033265521904
0.6894504175134334      0.05992779094224152   0.05995077216619848
0.689612316541997       0.06008654767107042   0.06045772443008617
0.68977103198987        0.06024374724070794   0.06082689939543704
0.6899432333019601      0.06041605356529366   0.06095888506874333
0.6901039163109058      0.06057847465219189   0.06084880331670679
0.6902780851840685      0.060756313164486     0.06064002535617506
0.6904490704765407      0.06093270541794137   0.060563663795267314
0.6906085374658686      0.06109882399508057   0.06073965808555218
0.6907814903194134      0.061280743756152375  0.06119290552647286
0.6909429248698139      0.06145219148482654   0.061721165886599344
0.6911178452844314      0.061639750177083466  0.06218661316139325
0.6912895821183584      0.061825702506833044  0.062374975506158135
0.6914498006491412      0.0620007952947557    0.062304158899441624
0.6916235050441408      0.06219238164708953   0.06210651370889261
0.6917856911359962      0.06237291055482948   0.062009494991328955
0.691944693647161       0.06255143726815848   0.06214187553274868
0.6921171820225428      0.0627468291699811    0.06256448877594219
0.6922781520947803      0.06293078888010013   0.06310338812605777
0.6924526080312348      0.06313191914936897   0.06362603956707945
0.6926238803869987      0.06333115606004189   0.06389255563696154
0.6927836344396184      0.06351857870146974   0.06388092697883277
0.692956874356455       0.06372354904867315   0.06370353446295315
0.6931185959701474      0.0639165100046985    0.06357972516353251
0.6932771340031492      0.0641071873319656    0.06366211853950463
0.6934491579003679      0.06431577865242866   0.06404081797965358
0.6936096634944423      0.06451199113704624   0.0645783727609531
0.6937836549527338      0.06472641860613637   0.06515284644906971
0.6939461281078809      0.06492827256387354   0.06549283699502703
0.6941054176823376      0.06512768951892998   0.06556562546464975
0.6942781931210111      0.0653456866574581    0.06542881039251146
0.6944394502565404      0.0655507417101809    0.06527993962925673
0.6946141932562867      0.06577467703180649   0.06531241395893912
0.6947857526753425      0.06599628212745413   0.06564799168929233
0.694945793791254       0.06620456940268948   0.06617619088673415
0.6951193207713824      0.0664321074999061    0.06678891353393795
0.6952813294483665      0.06664613587982111   0.06719820832172643
0.6954401545446601      0.06685745021363791   0.06734188179095917
0.6956124655051706      0.06708840710093993   0.06725032775452923
0.6957732581625369      0.06730568053952422   0.06709663759717474
0.6959475366841201      0.06754287778879975   0.06708185873968565
0.6961186316250129      0.06777745957791224   0.06736284579978043
0.6962782082627613      0.06799777947615675   0.06786832554201024
0.6964512707647267      0.0682383847693657    0.06850964243272341
0.6966128149635478      0.06846453756076382   0.06898872828189068
0.6967711755816783      0.06868769363319507   0.06921275439687637
0.6969430220640258      0.06893148558731442   0.06918146758865895
0.697103350243229       0.06916046555962986   0.06903378366002108
0.6972771642866492      0.06941036938682688   0.06897312941530455
0.6974394600269251      0.06964527162123908   0.06917118554734063
0.6975985721865106      0.06987702453786229   0.06962698903212552
0.6977711702103129      0.07013004952540527   0.07028012960231393
0.6979322499309709      0.0703677153845138    0.07083354550723814
0.698106815515846       0.07062693941237697   0.07117807432591654
0.6982781975200304      0.07088311241055564   0.07121078038397484
0.6984380612210705      0.07112356253247601   0.07108075913885224
0.6986114107863277      0.07138592287597971   0.07098857374507775
0.6987732420484406      0.07163237439800654   0.07113162733753065
0.698931889729863       0.07187540296293003   0.07154386738327156
0.6991040232755023      0.07214068294824136   0.07219466181331327
0.6992646385179973      0.07238970355897349   0.07279676659752751
0.6994387396247093      0.07266125592560661   0.07322841351877021
0.6996096571507308      0.07292948151366288   0.073339097614027
0.699769056373608       0.07318109109303458   0.07324029057932303
0.6999419414607021      0.07345557734991372   0.07312353371358972
0.700103308244652       0.07371326529243838   0.07320680630405359
0.7002781608928188      0.07399410829346026   0.07361436595982508
0.700449829960295       0.07427147285153977   0.07426132936302177
0.700609980724627       0.07453168460804448   0.07489511224235083
0.700783617353176       0.0748153938532298    0.07538888591313779
0.7009457356785807      0.07508176906260002   0.07555947184607814
0.7011046704232948      0.07534430332350187   0.07550034913026189
0.7012770910322259      0.07563066689415539   0.07537386045802744
0.7014379933380127      0.07589935488975026   0.07540858770465353
0.7016123815080165      0.07619214481944075   0.0757566253927638
0.7017835860973297      0.07648118697733616   0.07637955460807513
0.7019432723834986      0.07675231176610192   0.07704201007908358
0.7021164445338846      0.07704797704024251   0.07761368127274244
0.7022780983811262      0.07732542658851828   0.07786757659081266
0.7024365686476773      0.07759876734533697   0.0778626997820879
0.7026085247784454      0.07789688312132655   0.07773815916189881
0.7027689626060691      0.07817644659373378   0.07772512857390079
0.7029428862979099      0.07848104958645828   0.07800233356366844
0.7031052916866064      0.07876692255440701   0.07855006141866785
0.7032645134946123      0.07904854034058789   0.07922581317870434
0.7034372211668352      0.07935551622538818   0.07988157104476243
0.7035984105359138      0.0796434284724431    0.08024298966659792
0.7037730857692093      0.07995696052744705   0.0803168712258753
0.7039445774218143      0.0802663235650039    0.08020755888553024
0.704104550771275       0.08055628433538362   0.080160572088861
0.7042780099849527      0.0808721856424293    0.08037450587612283
0.7044399508954862      0.08116851139279167   0.08087689882829176
0.7045987082253291      0.08146032150032514   0.08155130759209323
0.704770951419389       0.08177838225215747   0.08226211833594425
0.7049316763103045      0.08207654227060707   0.08270666005038263
0.7051058870654371      0.0824012076840944    0.08286152979491346
0.705276914239879       0.08272144107584571   0.08278233296628063
0.7054364231111767      0.08302144370552073   0.08270763491306667
0.7056094178466914      0.0833482638453576    0.08285243141339715
0.7057708942790618      0.08365468433243141   0.08329454305080299
0.7059458565756491      0.08398817418932897   0.08402642609913119
0.7061176352915459      0.08431708721847041   0.08477255570933043
0.7062778957042984      0.08462527378485853   0.08527575048563295
0.706451641981268       0.08496083817047116   0.0854937271025677
0.7066138699550932      0.08527550881811526   0.08545149697844966
0.7067729143482279      0.08558526804112317   0.08536351266358529
0.7069454446055795      0.08592270321875291   0.08545147961098956
0.7071064565597869      0.08623893099177764   0.08583477583745515
0.7072809543782111      0.08658307962761157   0.08653948116694547
0.7074522686159449      0.08692239662955857   0.08731861329860431
0.7076120645505344      0.0872401880099321    0.08789637687031468
0.7077853463493409      0.08758620009762155   0.08820495962142427
0.7079471098450031      0.0879105236833102    0.0882149673310727
0.7081056897599747      0.08822968982793693   0.08812453409844882
0.7082777555391633      0.08857752704926213   0.0881559831366824
0.7084383030152076      0.08890337530054575   0.08846927737461273
0.7086123363554688      0.08925798442956338   0.08912821703045239
0.7087748513925858      0.08959042460218496   0.08988724785799262
0.7089341828490122      0.08991756738031796   0.09054655045883409
0.7091070001696557      0.09027375448770088   0.09097326746323489
0.7092682991871548      0.09060746896129998   0.09106604063424804
0.7094430840688709      0.09097046057815664   0.09098284059776784
0.7096146853698965      0.09132822750988256   0.09097306404740328
0.7097747683677778      0.09166321436747109   0.09122350031384162
0.709948337229876       0.09202776230434048   0.09183154718400567
0.7101103877888301      0.0923693726471366    0.09259161555465173
0.7102692547670935      0.09270544321050717   0.09330395344472378
0.7104416076095739      0.0930713486119462    0.09382171945377732
0.71060244214891        0.09341402240124479   0.09398980403792893
0.710776762552463       0.09378675587793664   0.09393676009903969
0.7109478993753255      0.09415401937705305   0.09389210148898988
0.7111075178950438      0.09449775367602847   0.09407366887929293
0.711280622278979       0.09487182173794623   0.09461413567638394
0.71144220835977        0.0952222081175444    0.09535813055911471
0.7116006108598704      0.09556681950093841   0.09611381227702338
0.7117724992241877      0.09594202878817282   0.09672574868673686
0.7119328692853608      0.09629327196021892   0.09698348667982606
0.7121067252107508      0.09667532998186674   0.09698018882004333
0.7122690628329965      0.09703327145414162   0.09691333629917434
0.7124282168745517      0.09738530970995563   0.09701089722060063
0.7126008567803239      0.09776842246750178   0.0974501190111657
0.7127619783829519      0.09812713822030661   0.09814694697299786
0.7129365858497967      0.09851714172842406   0.0990110239810819
0.713108009735951       0.09890130730957103   0.09970172351455264
0.713267915318961       0.09926079221209864   0.10004532457965819
0.713441306766188       0.09965182449114837   0.10010190789782134
0.7136031799102707      0.10001803069057849   0.10003258319914456
0.7137618694736628      0.10037810810667266   0.10007666008122262
0.713934044901272       0.10076998294384132   0.10043649342913329
0.7140947020257369      0.10113676082201069   0.10108413775322064
0.7142688450144187      0.10153554147947272   0.10195851367566434
0.71443980442241        0.10192829970735469   0.10272390073900523
0.714599245527257       0.10229578101028286   0.10316252368503774
0.714772172496321       0.10269552578477144   0.10329791103172284
0.7149335811622407      0.10306974802801966   0.10324257319318007
0.7151084756923773      0.10347643379554204   0.10325069544500745
0.7152801866418235      0.10387692171762955   0.10355754814318775
0.7154403792881253      0.10425161613865283   0.10416685000470043
0.7156140577986441      0.10465901655293981   0.10504106124098413
0.7157762180060185      0.10504048442713194   0.10581961622544987
0.7159351946327025      0.10541547631665285   0.10634305195404754
0.7161076571236035      0.10582340712430406   0.10656017457901505
0.7162686013113602      0.1062051475494484    0.10653339203191385
0.7164430313633338      0.106620018158918     0.10650641127879606
0.7166142778346168      0.10702846773874702   0.10673599421533018
0.7167740060027556      0.10741046684208039   0.10727851092638521
0.7169472200351114      0.10782582771953705   0.1081318864220124
0.7171089157643228      0.10821460464155379   0.10895550706929581
0.7172674279128437      0.10859669351503648   0.1095664982333399
0.7174394259255816      0.10901236648100028   0.10988077959007227
0.7175999056351753      0.10940120822464625   0.1099009735738191
0.7177738712089858      0.10982381660915198   0.10985200401150276
0.7179363184796521      0.11021946276823678   0.10998730797396697
0.7180955821696279      0.11060830526398155   0.11043118161300168
0.7182683317238205      0.11103113132701561   0.11122966216568143
0.7184295629748689      0.11142675270522671   0.11208399334096358
0.7186042800901343      0.111856535767438     0.11285196340812312
0.7187758136247092      0.11227956500512283   0.1132586669596208
0.7189358288561398      0.11267514524203225   0.1133349053591225
0.7191093299517873      0.11310510236348256   0.1132831115138806
0.7192713127442906      0.11350748503600037   0.11336010148920948
0.7194301119561033      0.1139028629265675    0.11372690727337387
0.719602397032133       0.11433282403318733   0.11446909593401924
0.7197631638050185      0.11473497879665438   0.11532973949917538
0.7199374164421208      0.11517188618428692   0.11617093192110643
0.7201084854985327      0.11560183545361803   0.11667983290304769
0.7202680362518002      0.11600374484577723   0.11682967134583286
0.7204410728692847      0.11644061127936671   0.11679288671148784
0.7206025911836249      0.11684931789925684   0.11681766034991045
0.7207775953621821      0.11729323183668636   0.11714831280804568
0.7209494159600488      0.11773010938145106   0.1178467994368789
0.7211097182547712      0.118138589851428     0.11870426882354311
0.7212835064137105      0.11858239879101276   0.11959076498894533
0.7214457762695056      0.11899769247174957   0.12015296149611662
0.7216048625446101      0.11940567616244988   0.12037854144663837
0.7217774346839315      0.11984917657719311   0.12037223097471325
0.7219384885201087      0.12026394444197526   0.12036352290564589
0.7221130282205028      0.12071438356513067   0.12061340177625657
0.7222843843402065      0.12115755197606998   0.12123724398036023
0.7224442221567658      0.12157176963487154   0.12207214964482052
0.7226175458375421      0.12202184387066271   0.1230045562521871
0.7227793512151741      0.12244285513905569   0.12365648973458374
0.7229379730121156      0.12285637119824389   0.12397198574644896
0.723110080673274       0.12330592071032329   0.12401584324545649
0.7232706700312881      0.12372620131372927   0.12398600013810548
0.7234447452535192      0.12418266055035054   0.12415389442248606
0.7236156368950598      0.12463166044698215   0.12468715374578915
0.7237750102334561      0.12505118481835625   0.12547880046748866
0.7239478694360694      0.1255070615581565    0.12644021979199985
0.7241092103355383      0.12593335647912454   0.12717990050816066
0.7242840370992243      0.12639614381679745   0.12762262223606125
0.7244556802822197      0.1268513675135419    0.12771316750615258
0.7246158051620709      0.12727680857124546   0.12768031570376978
0.724789415906139       0.12773890931091367   0.1277978597611923
0.7249515083470628      0.12817112404439598   0.12823328423444638
0.7251104172072961      0.12859556833607758   0.12897229593167467
0.7252828119317463      0.12905683157245013   0.12994150570850393
0.7254436883530522      0.12948801979985403   0.1307494160873795
0.7256180506385751      0.12995615767642266   0.13129556031648423
0.7257892293434075      0.13041655534968347   0.13146168714624953
0.7259488897450956      0.13084668876829858   0.1314384960514291
0.7261220360110007      0.13131392753040153   0.1314959222014992
0.7262836639737614      0.13175080458662058   0.1318414998218529
0.7264421083558317      0.13217974647065853   0.13250918298968561
0.7266140386021188      0.13264594168675853   0.13346514196746698
0.7267744505452617      0.13308159744187953   0.13433131560903278
0.7269483483526216      0.1335546418489043    0.1349868235212152
0.7271107278568372      0.13399712450365622   0.13524310862360492
0.7272699237803623      0.13443158007941047   0.13525726385727044
0.7274426055681043      0.13490355841808024   0.13526047221912393
0.727603769052702       0.13534472349127447   0.13549547468535553
0.7277784184015167      0.13582352662105943   0.13613376429873017
0.7279498841696408      0.1362943256987592    0.1370601584895457
0.7281098316346206      0.13673413884272806   0.1379633972904278
0.7282832649638175      0.1372117261056191    0.13871165366176624
0.72844517998987        0.1376582383592629    0.1390589501352668
0.7286039114352321      0.1380965683391657    0.13912133600822488
0.728776128744811       0.13857280103413627   0.13910429452166057
0.7289368277512457      0.1390177975243869    0.13926351327952183
0.7291110126218974      0.1395008022146526    0.13980557028821508
0.7292820139118585      0.13997564408618454   0.14068059706771072
0.7294414968986754      0.14041908923917512   0.14160491002721873
0.7296144657497092      0.14090066650698516   0.14244283703534275
0.7297759162975987      0.14135076419236192   0.14289341833682975
0.7299508527097052      0.1418390937999099    0.14302581445004978
0.7301226055411211      0.14231917259870014   0.14300698004645518
0.7302828400693927      0.14276761772858626   0.14312024466963622
0.7304565604618813      0.14325441154451712   0.1435933550972345
0.7306187625512257      0.14370949207572398   0.1443760100176683
0.7307777810598796      0.14415616249578841   0.1453021528401277
0.7309502854327503      0.1446412913672873    0.14620827225888103
0.7311112715024768      0.14509456386917144   0.14675263040044564
0.7312857434364203      0.14558638483195321   0.14696478940550464
0.7314570317896731      0.14606980911963785   0.14695813596085344
0.7316168018397817      0.14652123509391823   0.14701702130765867
0.7317900577541073      0.1470113141586912    0.1473930891348593
0.7319517953652886      0.14746932149972436   0.14810075801458003
0.7321103493957795      0.14791878527444108   0.14900946095481168
0.7322823892904872      0.1484070000861022    0.14997266026765982
0.7324429108820506      0.14886301188453485   0.15061528901645016
0.732616918337831       0.14935785501093643   0.15092593170384377
0.7327794074904672      0.14982042494336403   0.15095457102031581
0.7329387130624128      0.1502743785141719    0.15096524426215716
0.7331115044985753      0.15076725439435446   0.15122311863965793
0.7332727776315936      0.1512277611086288    0.15181996180757432
0.7334475366288288      0.15172729595029688   0.15277737110557407
0.7336191120453734      0.15221822035902347   0.15377510462604999
0.7337791691587738      0.15267661858603823   0.154499655694794
0.7339527121363912      0.15317410274971843   0.15490642833963614
0.7341147368108644      0.15363899636015865   0.15498308093309876
0.734273577904647       0.1540951507212511    0.1549764705193383
0.7344459048626465      0.15459046919501968   0.1551542452824904
0.7346067135175017      0.15505308347429658   0.15565994523314586
0.7347810080365739      0.15555492590164882   0.15655808697121504
0.7349521189749555      0.1560480318694529    0.15757337468118046
0.7351117116101928      0.156508322014154     0.15837637258627693
0.7352847901096472      0.15700791236711292   0.15889134883158804
0.7354463503059573      0.15747462890242012   0.15903619392440674
0.7356047269215769      0.15793249291593597   0.1590302541787299
0.7357765894014133      0.15842972317264428   0.1591345629970078
0.7359369335781055      0.15889397808985734   0.1595392776544813
0.7361107636190147      0.15939765330129108   0.1603550227527623
0.7362730753567797      0.1598682986047222    0.16131522217959404
0.736432203513854       0.16033002958970033   0.16219649483844667
0.7366048175351453      0.16083123952865255   0.16284656465395456
0.7367659132532923      0.16129932468774877   0.1630966852720704
0.7369404948356563      0.16180693679839234   0.16311880336519435
0.7371118928373297      0.16230563114238858   0.1631724651995856
0.7372717725358588      0.16277110832095718   0.16348866387932898
0.737445138098605       0.1632761645263998    0.1642174581659572
0.7376069853582069      0.16374795536872352   0.16515093477818513
0.7377656490371182      0.16421073197574523   0.166073686712513
0.7379377985802464      0.1647131341422159    0.16682204355394173
0.7380984298202304      0.1651821881807097    0.16716505747604327
0.7382725469244313      0.16569090580290782   0.16723311771868138
0.7384434804479417      0.16619060211331377   0.16725035130500027
0.7386028956683078      0.16665687027705514   0.1674769250632826
0.738775796752891       0.16716284174680457   0.16810167596699835
0.7389371795343298      0.16763534319107734   0.1689855971643035
0.7391120481799855      0.1681475797553106    0.17002648344440519
0.7392837332449508      0.16865073725647653   0.17083842141268346
0.7394439000067717      0.1691203517647624    0.17125112476594662
0.7396175526328096      0.16962975534421046   0.17136433856786112
0.7397796869557032      0.17010557752068406   0.17136681058536987
0.7399386376979062      0.17057224333610888   0.17152230982729805
0.7401110743043263      0.17107870455456642   0.1720489288284517
0.7402719926076021      0.17155151874308827   0.17287172890471222
0.7404463967750948      0.17206414992414806   0.17391856435104552
0.740617617361897       0.17256761096641277   0.17480700935839033
0.7407773196455549      0.17303736419320973   0.17531651725582936
0.7409505077934299      0.17354695354728128   0.17550466527610492
0.7411121776381605      0.17402280298323117   0.17550963599034988
0.7412706639022005      0.17448941800083237   0.1756010444855162
0.7414426360304576      0.17499588349989784   0.1760206682841069
0.7416030898555703      0.17546855717447793   0.17676241066358342
0.7417770295449         0.17598109290007377   0.17779220572380888
0.7419394509310854      0.17645980829957306   0.17870505129413133
0.7420986887365802      0.17692924666750204   0.1793343664918203
0.7422714124062921      0.1774385537083144    0.17963576512359633
0.7424326177728597      0.17791399550581188   0.17967217203087887
0.7426073090036442      0.17842931120114852   0.179719763454019
0.7427788166537382      0.17893532972261725   0.18004698347795625
0.742938806000688       0.17940744184990723   0.1807055989897953
0.7431122812118546      0.1799194266191223    0.18169943935381497
0.7432742381198769      0.18039748288848423   0.18264883043153846
0.7434330114472087      0.180866197698256     0.18336280244845302
0.7436052706387576      0.18137477960324214   0.18376126071610668
0.7437660115271622      0.1818494002433182    0.18384307920365242
0.7439402382797836      0.18236388319483024   0.18386154861235574
0.7441112814517146      0.1828690009054751    0.18409629637696212
0.7442708063205012      0.1833401286163027    0.18465760925063057
0.7444438170535048      0.18385110498483448   0.18559237740621948
0.7446053094833641      0.18432807559109357   0.1865608379120158
0.7447802877774404      0.184844883567477     0.18742206132775405
0.7449520824908262      0.18535228798266964   0.18789115831942965
0.7451123589010678      0.18582566529715422   0.18801513368029246
0.7452861211755262      0.18633885815178802   0.18802627153021398
0.7454483651468403      0.18681801188802813   0.18818650363010664
0.745607425537464       0.18728773795320464   0.188657581475461
0.7457799717923046      0.18779724801442976   0.18952310980623915
0.745940999744001       0.18827269504180438   0.19048906676404403
0.7461155135599142      0.1887879094261037    0.19141777481104255
0.746286843795137       0.18929366644468512   0.19198528514265747
0.7464466557272155      0.1897653633452595    0.19217835561459745
0.7466199535235108      0.19027679365634906   0.19219641177983562
0.746781733016662       0.19075415827625716   0.19229360197310782
0.7469403289291225      0.19122205431059267   0.19266759089169921
0.7471124107058001      0.1917296463820094    0.19344423761675883
0.7472729741793335      0.19220317145691862   0.19438726174088938
0.7474470235170837      0.19271636162476852   0.1953710769928578
0.7476178892741434      0.19322004973567053   0.1960417220144391
0.7477772367280588      0.19368967427373837   0.19632139543696514
0.7479500700461912      0.19419891780815812   0.19636944339183302
0.7481113850611792      0.19467409841383565   0.1964170394583538
0.7482861859403843      0.19518886053206871   0.19673987912528748
0.7484578032388989      0.1956940951048493    0.19744524284498907
0.7486179022342692      0.19616527761697503   0.19835758289754069
0.7487914870938563      0.19667598732679772   0.19936604847009817
0.7489535536502993      0.1971526493905053    0.20007739900097218
0.7491124366260516      0.19761979266420748   0.20044198529341156
0.7492848054660209      0.19812640588051453   0.20053601287778006
0.749445656002846       0.1985989892699269    0.20055747266903629
0.749619992403888       0.19911099532120505   0.20078995087674972
0.7497911452242395      0.19961344275891385   0.20139331456223689
0.7499507797414467      0.20008188339614957   0.20225017282933622
0.7501239001228709      0.2005896802440024    0.20327494342842295
0.7502855022011508      0.2010634809167975    0.20406439368094018
0.7504439206987401      0.20152774415342647   0.2045244995965258
0.7506158250605465      0.20203129453740062   0.20468601125782976
0.7507762111192086      0.2025008783624195    0.2046999644697316
0.7509500830420875      0.20300969233979377   0.20484909326059325
0.7511124366618223      0.20348455413483957   0.20530648523347014
0.7512716067008665      0.20394986825429404   0.2060727144926541
0.7514442626041276      0.20445433543803343   0.2070879596581339
0.7516054002042444      0.20492488733525804   0.20795587309845423
0.7517800236685782      0.2054345287201039    0.2085768225397958
0.7519514635522215      0.20593457626875522   0.20881199040437262
0.7521113851327206      0.20640070654992268   0.20883794951567378
0.7522847925774365      0.20690582145144126   0.20892787778221653
0.7524466817190082      0.2073770888290159    0.20928979326509892
0.7526053872798893      0.2078388052458917    0.20997158222313264
0.7527775787049874      0.20833943171190822   0.21095360361707077
0.7529382518269412      0.20880625991431487   0.2118610938420576
0.753112410813112       0.20931192562139858   0.21257718538026427
0.7532833862185923      0.209807991791989     0.2129025103083454
0.7534428433209283      0.21027031473488844   0.21296051570974256
0.7536157862874812      0.21077137761236256   0.21300597086454778
0.7537772109508898      0.21123872424820173   0.21327162310630246
0.7539521214785154      0.21174473178300798   0.2139332172569052
0.7541238484254504      0.21224113526173766   0.21487705059788992
0.7542840570692412      0.21270388493513157   0.21579835689184465
0.7544577515772489      0.213205189785248     0.2165742196591352
0.7546199277821124      0.2136728715665494    0.2169567259074486
0.7547789204062854      0.21413101069394838   0.21705903938774415
0.7549513988946752      0.2146275977615085    0.2170834580258643
0.7551123590799208      0.21509062952748786   0.21727193017383556
0.7552868051293834      0.21559202088194182   0.21783102898004128
0.7554580675981555      0.21608381486085435   0.21871095279075298
0.7556178117637833      0.21654212754125127   0.21963754770138422
0.755791041793628       0.21703868254624112   0.22048572009455383
0.7559527535203284      0.2175017928971783    0.22095918593609565
0.7561112816663383      0.21795537909082924   0.221123774350961
0.7562832956765652      0.218447089146762     0.22114618626778515
0.7564437913836477      0.21890543353476138   0.2212649507520028
0.7566177729549473      0.21940180419686506   0.22171324232059583
0.7567802362231025      0.219864849567449     0.22246349943207896
0.7569395159105673      0.22031838077945837   0.2233718012362361
0.757112281462249       0.22080981183732523   0.22429075883493724
0.7572735287107863      0.22126800383165313   0.22487720161475866
0.7574482618235407      0.22176399155207657   0.22514906142875665
0.7576198113556045      0.22225040426102208   0.2251879404042279
0.757779842584524       0.22270367082785517   0.22525859851036414
0.7579533596776605      0.22319459621247473   0.22561060506748892
0.7581153584676528      0.22365242173338284   0.22627590816995136
0.7582741736769545      0.22410075262684123   0.227147499478955
0.7584464747504731      0.22458650784955847   0.2280982172547227
0.7586072575208475      0.22503926841249758   0.22876325675695192
0.7587815261554388      0.2255294293275243    0.22912343309922356
0.7589526112093395      0.22601004700162264   0.22919846160011245
0.759112177960096       0.22645777547704282   0.2292344334907061
0.7592852305750695      0.2269427577026759    0.22949083065346187
0.7594467648868987      0.2273949034849064    0.23005784544988048
0.7596217850629448      0.22788418440846536   0.23096666479097228
0.7597936216583003      0.22836393885049472   0.23192433174318422
0.7599539399505116      0.22881096944832913   0.23263613171956019
0.76012774410694        0.2292949807401359    0.2330601600121597
0.760290029960224       0.22974632456291738   0.23317004775267908
0.7604491322328175      0.2301882545885067    0.2331912827093223
0.7606217203696279      0.2306670108270771    0.23337231022610438
0.7607827902032941      0.23111321573802238   0.23384772400088152
0.7609573459011771      0.23159611969562516   0.2346867022059843
0.7611287180183697      0.23206954172302058   0.2356420808595333
0.7612885718324179      0.23251053565236085   0.23641069102809584
0.7614619115106832      0.23298806357823842   0.23692387944442506
0.7616237328858042      0.23343322521009563   0.2370935777594845
0.7617823706802347      0.2338690284539968    0.23711647867887406
0.7619544943388821      0.23434120090487706   0.23723045843712165
0.7621150996943852      0.2347811335095049    0.23760847893564482
0.7622891909141053      0.23525729967147233   0.23835929949475917
0.7624600985531348      0.2357240362348469    0.2392921010419507
0.76261948788902        0.2361586667213576    0.24010672387777024
0.7627923630891222      0.23662935520365055   0.24071283108338629
0.7629537199860802      0.23706800479532592   0.24095868057157518
0.763128562747255       0.23754257038841511   0.24100452452991122
0.7633002219277394      0.2380077341847978    0.2410810073302846
0.7634603628050795      0.23844100011090438   0.2413909455847627
0.7636339895466365      0.23890999851643818   0.24206880312214157
0.7637960979850493      0.23934716991622335   0.24292230895517047
0.7639550228427715      0.23977508118242494   0.2437607621879826
0.7641274335647107      0.2402385421025093    0.2444421572828905
0.7642883259835056      0.24067031981624254   0.2447629148569252
0.7644627042665174      0.2411374967162755    0.24484506303741221
0.7646338989688387      0.24159523626214488   0.24488640593609307
0.7647935753680157      0.24202140220456292   0.24511292209565386
0.7649667376314098      0.24248276238988256   0.2456906972307026
0.7651283815916595      0.24291267897363625   0.2464892740128418
0.7652868419712188      0.24333341574840822   0.24733602162959284
0.7654587882149949      0.2437891560310186    0.24808778244614377
0.7656192161556268      0.24421360801457115   0.24849270058724718
0.7657931299604755      0.2446729065475104    0.24863101870698126
0.76595552546218        0.24510099747837047   0.24865285795745445
0.766114737383194       0.2455199522077899    0.24879106005991555
0.766287435168425       0.24597355570000073   0.24924256550416266
0.7664486146505117      0.24639611333002048   0.24995493834446791
0.7666232799968153      0.2468531569855309    0.2508758139003194
0.7667947617624284      0.24730098473952455   0.2516760749485391
0.7669547252248973      0.24771793611109622   0.25215482553386964
0.7671281745515831      0.24816916491931681   0.25235565589520226
0.7672901055751247      0.2485896028234142    0.2523823986768009
0.7674488530179756      0.24900099572364784   0.2524672125614583
0.7676210863250436      0.24944645920256217   0.2528221941428121
0.7677818013289672      0.24986130259120554   0.2534527407859711
0.7679560021971078      0.2503100463544052    0.25433971224933616
0.7681270194845579      0.25074966270741117   0.2551768872266947
0.7682865184688636      0.2511588377329489    0.2557310092395662
0.7684595033173864      0.2516016952911461    0.25600888489261475
0.768620969862765       0.2520142017397128    0.25605892269091507
0.7687959222723604      0.2524602146117878    0.25611620235500093
0.7689676911012653      0.2528971477880812    0.2564050680952274
0.7691279416270259      0.2533039153777508    0.2569697631521731
0.7693016780170034      0.2537439641456724    0.2578163569996466
0.7694638961038367      0.2541539409765598    0.2586256306784396
0.7696229306099794      0.2545550253517992    0.2592385976068153
0.7697954509803392      0.25498916778126424   0.2595898015844424
0.7699564530475547      0.2553934247375077    0.259675064605759
0.770130940978987       0.25583055618104855   0.25970710473856473
0.7703022453297289      0.2562587075684491    0.25991586482301055
0.7704620313773265      0.2566571682788593    0.2603921938694447
0.770635303289141       0.25708826898931314   0.2611766219529039
0.7707970568978112      0.25748976746868824   0.26198717960597473
0.7709556269257909      0.2578823148300584    0.2626536775259054
0.7711276828179876      0.25830725581302705   0.26308568433168456
0.77128822040704        0.25870281433814285   0.2632220092955215
0.7714622438603094      0.25913057739990875   0.2632462041461117
0.7716247490104344      0.25952906113223645   0.26336977118594335
0.771784070579869       0.25991882798491084   0.26373698952574187
0.7719568780135205      0.26034056319211685   0.26442914467224404
0.7721181671440278      0.2607332225670394    0.2652206637649922
0.772292942138752       0.261157656029236     0.2660033779721339
0.7724645335527857      0.26157328342351965   0.26650422578353994
0.772624606663675       0.26196004666370104   0.26669486512938334
0.7727981656387813      0.26237833707818403   0.26672866001146345
0.7729602063107434      0.26276787045417294   0.26680421523916886
0.7731190634020149      0.26314881194317385   0.2670897252064676
0.7732914063575034      0.26356103686844157   0.26769726643256486
0.7734522310098476      0.2639447158156675    0.2684518459733944
0.7736265415264088      0.26435947771057117   0.2692584026102895
0.7737976684622794      0.26476555627838494   0.26982833483396074
0.7739572770950057      0.26514330828801425   0.2700842597550954
0.774130371591949       0.26555188862142054   0.2701449659486987
0.774291947785748       0.2659322534637963    0.27018490150723035
0.7744503403988564      0.2663041570632386    0.2703906532242761
0.7746222188761819      0.2667066377616643    0.2709026716045422
0.7747825790503631      0.26708112152431374   0.2716029628377398
0.7749564250887612      0.2674859757325056    0.2724168424081753
0.775118752824015       0.26786294710277475   0.2730265649872511
0.7752778969785783      0.2682315283595662    0.2733702708415224
0.7754505269973585      0.26863022242334134   0.27348617982572654
0.7756116387129944      0.2690012580838939    0.27350633131419916
0.7757862362928474      0.26940219464450715   0.2736547548227438
0.7759576502920098      0.2697946452132648    0.2740784607699859
0.776117545988028       0.27015967049765094   0.27471633735663475
0.776290927548263       0.270554327836385     0.27551699384629524
0.7764527908053538      0.2709216779359211    0.27616746771854395
0.776611470481754       0.27128077808371065   0.2765761550520865
0.7767836360223713      0.2716692451700593    0.2767470240119591
0.7769442832598443      0.2720306369382018    0.2767702920721765
0.7771184163615342      0.2724210949518166    0.27686112202099505
0.7772893658825336      0.2728030660090334    0.2771930696731623
0.7774487971003887      0.2731582186631039    0.2777553766857979
0.7776217141824607      0.27354222861144284   0.2785250766181073
0.7777831129613885      0.2738995436831753    0.2792053743282534
0.7779579976045332      0.27428549505606475   0.27971482688610455
0.7781296986669874      0.27466318255666106   0.2799280077249896
0.7782898814262974      0.27501442389878855   0.2799626798146945
0.7784635500498243      0.27539402242781924   0.28002159819042627
0.7786257003702068      0.27574730094439703   0.28026929559160835
0.7787846671098989      0.2760925678404573    0.28075662223942743
0.7789571197138079      0.27646591735689074   0.28148149417092855
0.7791180540145726      0.2768131930092496    0.2821718999469334
0.7792924741795543      0.2771883254870564    0.2827375476533623
0.7794637107638455      0.27755534818762106   0.28301343600661055
0.7796234290449924      0.2778965517261084    0.28307510017930754
0.7797966331903562      0.2782653270226486    0.2831060913222686
0.7799583190325757      0.27860841235616746   0.28328207030001934
0.7801168212941048      0.27894364627041185   0.2836870392373168
0.7802888094198507      0.27930617160247706   0.2843514984955442
0.7804492792424523      0.2796432589061454    0.28503795469315907
0.780623234929271       0.2800074070761054    0.28565353634081286
0.7807856723129454      0.2803462490256837    0.28598819880650717
0.7809449261159292      0.28067732679951357   0.2861000056721106
0.78111766578313        0.2810351796009006    0.2861195101906176
0.7812788871471865      0.28136798317718026   0.2862226128895933
0.78145359437546        0.281727326728004     0.28657462925229515
0.781625118023043       0.2820788043485155    0.2871731476953904
0.7817851233674817      0.28240549855276076   0.28783969597743947
0.7819586145761374      0.28275843613477086   0.28848564604117266
0.7821205874816487      0.28308672515818734   0.2888762323849349
0.7822793768064695      0.28340741792642327   0.289034781937395
0.7824516519955073      0.28375406246761486   0.28906225122554924
0.7826124088814007      0.28407632122492116   0.28912227068773066
0.7827866516315112      0.2844242919516325    0.2893917663582595
0.7829577108009311      0.2847645629535628    0.2899132458316234
0.7831172516672068      0.2850807199025163    0.2905457442848713
0.7832902783976994      0.2854222863340716    0.29121084934950187
0.7834517868250477      0.2857396928155923    0.2916565409506097
0.783626781116613       0.2860822024708247    0.29188196556482016
0.7837985918274877      0.28641710939356185   0.29192286938696255
0.7839588842352181      0.28672833590829955   0.29196009527816297
0.7841326625071655      0.28706440474969386   0.2921729525143685
0.7842949224759687      0.28737693531470726   0.2926039948673476
0.7844539988640813      0.2876821473811166    0.2931926610993433
0.7846265611164108      0.28801190296003637   0.293858574657798
0.7847876050655961      0.28831839556758143   0.2943443457290871
0.7849621348789984      0.2886491859978437    0.2946254432547486
0.78513348111171        0.2889725553902434    0.2946947599153152
0.7852933090412774      0.28927294566223694   0.2947134266942618
0.7854666228350617      0.28959732507127844   0.2948610836059237
0.7856284183257018      0.28989886957942484   0.29521594403990375
0.7857870302356514      0.29019328145946655   0.2957490327910001
0.7859591280098179      0.29051137972774604   0.29640247335841796
0.7861197074808401      0.290806922351523     0.2969220810541468
0.7862937728160794      0.29112590242917646   0.2972627539365286
0.7864646545706281      0.29143764617248924   0.2973730665959402
0.7866240180220325      0.2917271221774528    0.2973865238268373
0.7867968673376539      0.29203972292575814   0.2974771940518492
0.7869581983501309      0.2923302022223744    0.2977532574978178
0.7871330152268249      0.2926435536550726    0.2982770888235228
0.7873046485228283      0.29294976961458397   0.2989056011490398
0.7874647635156875      0.29323415627200133   0.2994363115678505
0.7876383643727637      0.29354109776747955   0.29981365449594094
0.7878004469266956      0.29382635840948446   0.2999543272509684
0.787959345899937       0.29410478044622884   0.2999751978751432
0.7881317307373953      0.29440544615367537   0.30002690578056335
0.7882925972717093      0.2946847183674138    0.300235670130918
0.7884669496702403      0.29498597840451785   0.30068681631160615
0.7886381184880807      0.29528029344494466   0.30127852834290636
0.7887977690027769      0.2955535111915733    0.30181938686100035
0.78897090538169        0.29584839544417085   0.302243984440111
0.7891325234574589      0.2961223329161956    0.3024323424955146
0.7892909579525372      0.29638962638876076   0.3024727798736669
0.7894628783118325      0.2966782713112292    0.3024971984674203
0.7896232803679835      0.296946209489111     0.30264216838999336
0.7897971682883514      0.2972350397779394    0.3030139173109616
0.7899595379055752      0.29750338512114105   0.3035265689721128
0.7901187239421082      0.29776519914630595   0.3040671126991536
0.7902913958428583      0.2980477701690619    0.3045422332534822
0.7904525494404642      0.2983101541475987    0.30479355647098855
0.7906271889022869      0.2985930352386274    0.3048747717787925
0.7907986447834191      0.2988692803693364    0.30488733115151867
0.790958582361407       0.29912564471497827   0.3049832205798881
0.791132005803612       0.29940218099484966   0.30528266591779735
0.7912939109426727      0.29965899230004306   0.30574052063242063
0.7914526325010428      0.29990947924762645   0.30626200555526106
0.7916248399236299      0.30017981994002846   0.306759923659159
0.7917855290430726      0.30043073564084943   0.3070555945347715
0.7919597040267323      0.30070124349729993   0.30717623070268407
0.7921306954297015      0.3009653217476449    0.30718916803539226
0.7922901685295264      0.3012102834113381    0.307243735718501
0.7924631274935683      0.30147450993544916   0.3074704706271298
0.7926245681544659      0.30171977677472867   0.3078648485018951
0.7927994946795804      0.301984044129781     0.308407975328533
0.7929712376240045      0.3022419947583711    0.3089083152868623
0.7931314622652842      0.30248129731064816   0.309228647207436
0.7933051727707809      0.30273926965099845   0.3093786814123523
0.7934673649731334      0.30297875245308203   0.3093990249272095
0.7936263735947953      0.3032122355866961    0.30942677233779964
0.7937988680806741      0.3034640648891948    0.30959305696902634
0.7939598442634087      0.30369770986533545   0.30992567219496153
0.7941343063103602      0.3039494349643987    0.3104266915626086
0.7943055847766212      0.30419505341656944   0.3109274307299809
0.7944653449397379      0.304422801293671     0.31127951037970203
0.7946385909670716      0.30466829645243587   0.3114723694902814
0.794800318691261       0.304896080664038     0.31151195960345596
0.7949588628347598      0.305118077466511     0.31152256722713595
0.7951308928424756      0.30535749589857725   0.3116331338984044
0.7952914045470472      0.3055795106493101    0.3119004560781601
0.7954654021158356      0.30581867930268414   0.3123487130810197
0.7956278813814798      0.30604060553412876   0.3128155311331042
0.7957871770664334      0.3062568599891394    0.3132000934712535
0.7959599586156041      0.3064897899321624    0.3134478429389014
0.7961212218616305      0.3067057094612721    0.31352264521226103
0.7962959709718738      0.306938166036855     0.31352955473087546
0.7964675365014265      0.30716484942897093   0.3135985564836203
0.796627583727835       0.30737493937923477   0.3138080516835023
0.7968011168184604      0.3076012317994116    0.314200580909438
0.7969631316059417      0.30781109459787626   0.31464399643266017
0.7971219628127323      0.3080155118460489    0.31503904767262414
0.7972942798837399      0.3082358048400469    0.3153226415471512
0.7974550786516033      0.3084399821069603    0.3154291792528053
0.7976293632836835      0.3086597665508827    0.3154420298927828
0.7978004643350732      0.30887400076179666   0.3154773877273164
0.7979600470833187      0.3090724411829479    0.3156303361136591
0.7981331156957812      0.30928615393976394   0.3159608652440204
0.7982946660050994      0.30948423687099547   0.3163708764608001
0.7984530327337269      0.30967709749049044   0.31676769967100993
0.7986248853265715      0.3098849034884451    0.31708403047169065
0.7987852196162718      0.31007739381221094   0.31722710307460955
0.798959039770189       0.31028456074488925   0.3172568626560714
0.7991213416209619      0.3104765770656154    0.31726698017068006
0.7992804598910443      0.31066349270205845   0.3173612041100452
0.7994530640253438      0.3108647557505477    0.31761764372786533
0.7996141498564989      0.3110511844662443    0.31797983104041566
0.799788721551871       0.3112516899983505    0.3184074205786841
0.7999601096665525      0.3114469893784248    0.3187436859081568
0.8001199794780898      0.31162777879343456   0.31891829949361816
0.8002933351538439      0.3118223076653449    0.3189703306916875
0.8004551725264538      0.31200249177786904   0.31897182958699927
0.8006138263183732      0.3121778001426209    0.3190276613434748
0.8007859659745096      0.31236651853563585   0.319224525314455
0.8009465873275017      0.3125412087039898    0.31953783486171045
0.8011206945447107      0.31272903810530994   0.3199423439703061
0.8012916181812292      0.3129118862323919    0.3202917524305188
0.8014510235146034      0.3130810307832784    0.32049783712974456
0.8016239147121945      0.3132629769539766    0.32057919142509284
0.8017852876066415      0.31343138489256894   0.3205819464850343
0.8019601463653053      0.313612321814395     0.32061467086962553
0.8021318215432787      0.31378838640407997   0.320767823288893
0.8022919784181077      0.31395104226020276   0.3210382627204548
0.8024656211571537      0.31412587041727685   0.3214128443047509
0.8026277455930554      0.3142876706332729    0.321744280050945
0.8027866864482666      0.3144449524180849    0.3219727755014136
0.8029591131676947      0.3146140778832439    0.3220822618534616
0.8031200215839787      0.3147704964074902    0.3220939946616686
0.8032944158644795      0.31493848851719225   0.3221047747963073
0.8034656265642898      0.3151018599577168    0.322209506977781
0.8036253189609558      0.31525285300205363   0.3224287830383049
0.8037984972218387      0.31541508395640344   0.3227649866854629
0.8039601571795774      0.31556510400192284   0.3230896802438959
0.8041186335566255      0.3157108383561118    0.32333667404185346
0.8042905957978906      0.31586748343662036   0.32347669687080244
0.8044510397360115      0.3160122371975183    0.3235048843363592
0.8046249695383493      0.3161676327627307    0.3235032227491622
0.8047873810375428      0.3163113050939505    0.32355938211581264
0.8049466089560458      0.3164508185494424    0.3237203921705801
0.8051193227387656      0.3166006454583704    0.32400630600946523
0.8052805182183412      0.3167390698802733    0.324315149595762
0.8054551995621339      0.31688753788745366   0.32460103080409775
0.805626697325236       0.3170317448723197    0.32476531061624175
0.8057866767851938      0.3171648776414047    0.32481218117861904
0.8059601421093685      0.3173077186101175    0.3248080967122817
0.806122089130399       0.31743965271520846   0.3248352429229917
0.8062808525707389      0.31756766050544155   0.32495104275606007
0.8064531018752957      0.3177050495807399    0.3251887539307945
0.8066138328767084      0.3178318511289701    0.3254712692611103
0.806788049742338       0.3179677652665196    0.3257578705788321
0.806959083027277       0.31809965138268353   0.32594459359345135
0.8071185980090718      0.31822127682148005   0.3260140585731816
0.8072915988550835      0.31835168090993343   0.32601581552599035
0.8074530813979509      0.3184719909419441    0.3260213835526408
0.8076280498050353      0.31860080998842893   0.32610847697498113
0.8077998346314292      0.31872572878722766   0.3263052327929927
0.8079601011546789      0.31884088151984735   0.32655830109868367
0.8081338535421454      0.3189642081788225    0.32683335190586627
0.8082960876264677      0.31907793598474027   0.3270214730759872
0.8084551381300994      0.3191880185196819    0.32711085047874766
0.808627674497948       0.31930582514369515   0.3271236208424722
0.8087886925626524      0.31941436666059303   0.3271174762149418
0.8089631964915738      0.31953047339348817   0.32716778688952614
0.8091345168398046      0.31964291943323137   0.3273177226895469
0.8092943188848912      0.3197464289012233    0.32753508079767857
0.8094676067941947      0.3198571732404363    0.3277945677633175
0.8096293764003539      0.31995914863022024   0.32799134348347975
0.8097879624258226      0.32005779858373695   0.3281008224161483
0.8099600343155082      0.32016336194693507   0.32813000772846496
0.8101205879020495      0.32026047505273336   0.3281195449881419
0.8102946273528079      0.32036423737936714   0.3281393271305284
0.8104571485004219      0.3204597172093876    0.3282360263792037
0.8106164860673455      0.3205520007887503    0.3284098458978467
0.8107893094984859      0.3206506119447587    0.3286451141987137
0.8109506146264821      0.3207412595616007    0.32884670220449014
0.8111254056186953      0.32083797042764556   0.3289877661550645
0.8112970130302178      0.3209313877329764    0.3290335366090146
0.8114571021385962      0.3210171673373677    0.3290252782357513
0.8116306771111915      0.3211086826528166    0.32902468760043485
0.8117927337806425      0.321192726520339     0.3290849369438342
0.811951606869403       0.3212738093454347    0.32921913346929876
0.8121239658223804      0.32136030914593877   0.3294234667244537
0.8122848064722136      0.3214396535747442    0.32961676007668583
0.8124591329862637      0.32152415303504406   0.3297693748256271
0.8126302759196233      0.32160559502293296   0.3298335829338595
0.8127899005498387      0.3216802045526414    0.3298331884295617
0.8129630110442709      0.32175964452203215   0.3298194816178697
0.8131246032355589      0.3218324167871731    0.3298475281735382
0.8132830118461564      0.3219024610999913    0.32994169574083965
0.8134549063209707      0.32197702000852235   0.3301097453610825
0.8136152824926408      0.3220452244245118    0.3302878250084017
0.8137891445285279      0.32211768386205913   0.33044659494943385
0.8139514882612707      0.32218395368862707   0.3305270421753498
0.814110648413323       0.3222476225789746    0.33054217359372395
0.8142832944295922      0.32231523048561384   0.3305238764660823
0.8144444221427172      0.3223769622374408    0.3305238136909368
0.8146190357200591      0.32244237331015985   0.33058552235112715
0.8147904657167104      0.3225049522787422    0.3307160221379626
0.8149503774102175      0.3225619523038526    0.33087218481952846
0.8151237749679415      0.3226222978117303    0.33102786695734526
0.8152856542225213      0.3226772640226014    0.3311201625687794
0.8154443498964106      0.3227298660592279    0.33114907637290725
0.8156165314345167      0.32278550305463705   0.3311337831779004
0.8157771946694786      0.32283607277851356   0.3311177671547566
0.8159513437686575      0.32288942228753015   0.33114454826697975
0.8161223092871458      0.3229403160400579    0.3312356995635857
0.8162817565024899      0.3229864607962881    0.3313645194783443
0.8164546895820509      0.3230350699589006    0.33151035686841296
0.8166161043584677      0.3230790926277744    0.3316108955006207
0.8167910049991013      0.3231253251156183    0.33165607972869177
0.8169627220590444      0.32316923244025314   0.3316457657080206
0.8171229208158433      0.3232088709485508    0.3316213976018679
0.8172966054368591      0.32325040466511634   0.3316231734395851
0.8174587717547307      0.3232878317792931    0.331676670717903
0.8176177544919118      0.3233232581781484    0.33177528670809925
0.8177902230933097      0.3233602741029578    0.33190370999910923
0.8179511733915634      0.32339349079957813   0.3320051282046423
0.8181256095540341      0.3234280458318193    0.3320628283981837
0.8182968621358142      0.3234605101184536    0.33206275942698427
0.81845659641445        0.32348948867103006   0.3320348817689176
0.8186298165573028      0.3235194951505582    0.33201420096806966
0.8187915183970114      0.32354617596807506   0.3320345458203111
0.8189500366560294      0.32357108620401354   0.3321002485772985
0.8191220407792643      0.3235967227975155    0.33220514385432887
0.819282526599355       0.3236193370180684    0.33230187033759284
0.8194564982836626      0.32364242978047947   0.33236967212286067
0.819618951664826       0.3236626599586001    0.3323832076868037
0.8197782214652988      0.32368124496971756   0.33235925811108236
0.8199509771299885      0.32370000766883383   0.3323227340968359
0.820112214491534       0.3237162105196045    0.3323115764773191
0.8202869377172964      0.32373234384030697   0.33234640097936224
0.8204584773623683      0.32374674351156674   0.3324234361395371
0.8206184987042959      0.3237588920612698    0.3325085901829343
0.8207920059104405      0.32377066562840284   0.3325801291100703
0.8209539948134409      0.32378031938159807   0.33260472288289517
0.8211128001357507      0.3237884554126179    0.33258774598941704
0.8212850913222774      0.3237959097180936    0.33254442353477526
0.8214458642056599      0.32380157944745985   0.33251108796061063
0.8216201229532593      0.32380632494935807   0.33251138032620137
0.8217911981201681      0.323809569823293     0.33255663196449564
0.8219507549839327      0.3238113359190331    0.33262440949807565
0.8221237977119142      0.3238118788142597    0.3326945626690475
0.8222853221367515      0.32381109916969447   0.33272928735547197
0.8224603324258057      0.323808855004451     0.33271984939992066
0.8226321591341694      0.3238052382308853    0.33267420730784214
0.8227924675393888      0.32380060343702405   0.3326255504398528
0.8229662618088253      0.3237942066646324    0.3325975759259542
0.8231285377751174      0.3237869474764755    0.3326108608786795
0.8232876301607189      0.3237786269152124    0.3326565352375951
0.8234602084105375      0.3237682557775052    0.3327184375478361
0.8236212683572117      0.32375731618292364   0.3327588095773499
0.8237958141681029      0.32374408888322287   0.3327605586259523
0.8239671763983035      0.32372971738558476   0.33271855715943877
0.8241270203253599      0.32371507678018      0.33265913683410975
0.8243003501166333      0.32369785619815766   0.3326041926443014
0.8244621616047624      0.32368051946691      0.3325857686218248
0.8246207895122009      0.32366234448089026   0.3326048452424943
0.8247929032838563      0.32364130599010826   0.33265227718950807
0.8249534987523676      0.32362044026699616   0.3326944547218401
0.8251275800850957      0.32359647807256275   0.33270788331602186
0.8252984778371333      0.32357159632383514   0.3326746770403627
0.8254578572860266      0.32354718155737716   0.33261113716454743
0.825630722599137       0.3235193831982622    0.332534668754362
0.825792069609103       0.3234922021524571    0.33248529996466075
0.825966902483286       0.3234614057662868    0.3324750588362106
0.8261385517767784      0.32342981332717835   0.33250392580461324
0.8262986827671266      0.32339913109284285   0.33254200751801
0.8264722996216917      0.3233645479575514    0.3325620308608463
0.8266343981731125      0.32333102470106273   0.33253922794482954
0.8267933131438427      0.32329700476582307   0.3324768829565806
0.82696571397879        0.3232588069396592    0.33238582351755885
0.8271265965105931      0.32322195162557255   0.33230969573287517
0.827300964906613       0.32318062481938914   0.33226590291257735
0.8274721497219424      0.3231386889829464    0.332269802527382
0.8276318162341275      0.3230983907506757    0.33229920531561535
0.8278049686105297      0.32305340001529836   0.3323253016231636
0.8279666026837875      0.3230101946245193    0.33231453784781084
0.8281250531763548      0.32296671100566504   0.3322590965066657
0.828296989533139       0.3229182645116034    0.33216056968409646
0.828457407586779       0.32287188174470655   0.3320617913011202
0.8286313115046359      0.322820313961813     0.33198370634076013
0.8287936971193485      0.3227709567728359    0.3319572527893827
0.8289528991533706      0.3227214401224251    0.3319698775740014
0.8291255870516097      0.32266647005501387   0.3319975126998332
0.8292867566467045      0.32261398726565393   0.3320011043653132
0.8294614121060162      0.32255583054521486   0.3319530527857082
0.8296328839846374      0.32249743923895124   0.3318531069284634
0.8297928375601143      0.32244181656851467   0.3317376015701031
0.8299662769998082      0.32238024883956845   0.33162849732792765
0.8301281981363577      0.32232159360672996   0.33157123443776193
0.8302869356922168      0.32226299153922816   0.3315625997634021
0.8304591591122928      0.32219818176526144   0.33158407870331325
0.8306198642292246      0.3221365554456614    0.331596078151764
0.8307940552103733      0.3220685056301832    0.33156221179699075
0.8309650626108315      0.32200043558461755   0.3314678545820336
0.8311245517081454      0.32193582451678376   0.33134196766748425
0.8312975266696762      0.32186452467300697   0.3312048672950172
0.8314589833280627      0.32179682468246174   0.33111450233393486
0.8316339258506662      0.32172222181877386   0.33107774081159863
0.8318056847925792      0.32164771664845243   0.33108888321522123
0.831965925431348       0.32157708498870724   0.33110417450534635
0.8321396519343336      0.3214992873294967    0.3310807761565892
0.832301860134175       0.32142550312355306   0.3309982279142007
0.8324608847533258      0.32135209652701      0.33086780867523236
0.8326333952366937      0.32127126908150727   0.3307084360372087
0.8327943874169172      0.3211947185512625    0.3305869188183519
0.8329688654613577      0.3211105377948278    0.33051799995234865
0.8331401599251077      0.32102666374833766   0.3305116537099069
0.8332999360857134      0.32094733444513185   0.3305276456235433
0.833473198110536       0.3208601104432958    0.3305182166217277
0.8336349418322143      0.32077751513726493   0.33044940263842565
0.8337935019732021      0.3206955019378366    0.3303215303832329
0.8339655479784069      0.3206053476431202    0.3301459359205341
0.8341260756804675      0.32052013783746486   0.3299944411087675
0.8343000892467449      0.3204265829173519    0.3298890321313119
0.8344625845098781      0.3203381089317913    0.3298578888239593
0.8346218961923207      0.32025032885698596   0.32986762753326765
0.8347946937389803      0.3201539577103519    0.3298722138377564
0.8349559729824956      0.32006292389628554   0.329824631961285
0.8351307380902279      0.3199630969523628    0.3296946443201067
0.8353023196172696      0.31986389672233356   0.329509256482258
0.8354623828411671      0.3197702942126455    0.3293324421602922
0.8356359319292815      0.3196676507130255    0.3291913555646865
0.8357979627142517      0.31957073815327125   0.32913240071762534
0.8359568099185313      0.31947471877996664   0.3291299676387101
0.8361291429870278      0.3193694186272068    0.32914061886406115
0.83628995775238        0.319270099670537     0.3291095249957332
0.8364642583819493      0.3191613029428228    0.32899438176616463
0.836635375430828       0.3190533340569639    0.32880694973118807
0.8367949741765625      0.3189516004943323    0.3286092178260328
0.8369680587865138      0.31884014757170553   0.3284315515259801
0.8371296250933209      0.31873505998961443   0.3283390664069522
0.8372880078194375      0.3186310603174088    0.32831712449963973
0.837459876409771       0.31851710761745655   0.32832828666532443
0.8376202266969603      0.3184097644181079    0.32831375892236914
0.8377940628483665      0.3182922762282327    0.3282195848404952
0.8379563806966285      0.31818152645889997   0.3280494928342853
0.8381155149641999      0.3180719707334672    0.3278384116727185
0.8382881350959882      0.31795203878780426   0.32762331503976094
0.8384492369246322      0.31783908729055643   0.32748840599124035
0.8386238246174933      0.31771556961885966   0.3274332863817047
0.8387952287296638      0.3175931833857101    0.3274383365681562
0.8389551145386901      0.3174780232833992    0.3274359215547964
0.8391284862119333      0.31735206419156664   0.3273626918525912
0.8392903395820321      0.31723345695111077   0.32720452274514716
0.8394490093714405      0.317116232623749     0.326987368258729
0.8396211650250658      0.3169879841925854    0.32674427712621973
0.8397818023755468      0.31686731273982743   0.32657247347743246
0.8399559255902448      0.3167354124253416    0.3264826870117518
0.8401268652242523      0.3166048350010967    0.3264738198020192
0.8402862865551155      0.31648208713850867   0.32647990368239554
0.8404591937501956      0.3163479017524509    0.3264301479411972
0.8406205826421315      0.3162216686713131    0.3262913595159372
0.8407954573982843      0.3160838155705419    0.326051332114116
0.8409671485737465      0.3159473909897838    0.3257871264630057
0.8411273214460645      0.31581915608437494   0.32558517650569757
0.8413009801825995      0.3156790777839295    0.3254642837355117
0.8414631206159902      0.3155473106709673    0.3254389882589116
0.8416220774686903      0.3154172151371819    0.3254473772668603
0.8417945201856074      0.31527506037721326   0.32541739601632413
0.8419554445993802      0.31514144451345205   0.3252995074651027
0.84212985487737        0.3149955921288765    0.3250674632081178
0.8423010815746693      0.3148513539263894    0.32478681215335564
0.8424607899688243      0.3147158855599505    0.3245512739633723
0.8426339842271963      0.31456796373618423   0.3243890565685751
0.8427956601824239      0.31442892997189176   0.32433747562097626
0.842954152556961       0.31429174687133743   0.32434168992305973
0.8431261307957151      0.3141419007656547    0.32432992226204527
0.8432865907313248      0.3140011641688798    0.32423769856001944
0.8434605365311516      0.31384759243468835   0.32402310686124475
0.8436229640278341      0.3137032472108177    0.32374930009314723
0.8437822079438261      0.3135608502665753    0.3234812050115124
0.843954937724035       0.31340541133399835   0.32326840906731386
0.8441161492010996      0.3132594179873016    0.3231768134720854
0.8442908465423812      0.31310021274002725   0.3231665917021579
0.8444623603029722      0.31294290124847723   0.3231667292698695
0.8446223557604189      0.3127952574241514    0.3230981017922817
0.8447958370820826      0.31263419396589615   0.3229044856930544
0.8449578001006021      0.31248291189050176   0.322631087388974
0.845116579538431       0.31233375103696337   0.32234069792128084
0.8452888448404768      0.3121709699753117    0.3220866487916433
0.8454495918393784      0.3120181831186953    0.32195723561113915
0.845623824702497       0.31185161129880706   0.3219253950695859
0.8457948739849249      0.3116871080604506    0.32193216424152654
0.8459544049642086      0.3115328147314485    0.32188829222099824
0.8461274218077093      0.31136453508422013   0.32172324034046396
0.8462889203480657      0.31120657580006805   0.3214597363479219
0.846463904752639       0.31103446774247967   0.3211222577756154
0.8466357055765218      0.31086452457781555   0.32083629887696674
0.8467959880972603      0.31070511491717745   0.3206752326129439
0.8469697564822158      0.3105313580389887    0.32062182609470746
0.8471320065640271      0.31036824383006006   0.3206283508438632
0.8472910730651478      0.3102075128794431    0.320604153547444
0.8474636254304855      0.31003224319549494   0.32046738365293936
0.8476246594926788      0.30986782029378673   0.320218884897896
0.8477991794190891      0.3096887013500942    0.31987152698628063
0.8479705157648089      0.30951191592204513   0.31955044321078424
0.8481303338073843      0.3093461840445638    0.3193475379919554
0.8483036377141768      0.3091655640586755    0.31925974845064486
0.848465423317825       0.30899610353052503   0.3192587849104993
0.8486240253407827      0.30882918836718004   0.31925167559386974
0.8487961132279573      0.30864719980169114   0.3191474998274581
0.8489566828119877      0.30847656860705525   0.3189228844538769
0.849130738260235       0.308290711814184     0.31857553682345113
0.8493016101277917      0.3081073525102062    0.31822389343568647
0.8494609636922041      0.3079355510220354    0.3179763904027995
0.8496338031208336      0.30774833819858044   0.3178450807811691
0.8497951242463188      0.307572785864721     0.31782695916860015
0.8499699312360208      0.30738167234369856   0.31782949581713316
0.8501415546450324      0.30719314667059977   0.31774789446744073
0.8503016597508997      0.30701647930752907   0.3175423009561818
0.850475250720984       0.30682406798604356   0.31719785660195554
0.850637323387924       0.30664361632809123   0.31684257942693905
0.8507962124741734      0.30646595525995446   0.3165554781193415
0.8509685874246398      0.3062723740042298    0.3163795988955538
0.8511294440719619      0.30609094160959105   0.316337700008611
0.8513037865835009      0.305893444282703     0.3163451060228282
0.8514749455143494      0.305698692569048     0.316293276474069
0.8516345861420537      0.3055162812193323    0.31611980474294515
0.8518077126339749      0.305317628488957     0.3157919750136899
0.8519693208227519      0.3051314142456719    0.3154228635950607
0.8521277454308384      0.30494814189670677   0.3150975810656253
0.8522996559031417      0.3047484650201669    0.31487100897370685
0.8524600480723008      0.3045614268243292    0.3147954376879436
0.8526339261056768      0.3043578393914815    0.3147983881048358
0.8527962858359085      0.3041669683547691    0.31477758417341967
0.8529554619854497      0.30397912090255397   0.3146488443780785
0.8531281239992079      0.30377455664451397   0.3143562611004868
0.8532892677098218      0.30358288905647596   0.31398618311231585
0.8534638972846527      0.30337436703342235   0.31359040491214357
0.853635343278793       0.30316882604355677   0.3133164207717449
0.853795270969789       0.30297636414784407   0.31320377246431613
0.8539686845250021      0.3027668801178829    0.3131934463519099
0.8541305797770709      0.30257056866708215   0.3131898497632845
0.854289291448449       0.30237742525316635   0.3130949271357385
0.8544614889840442      0.3021670981080938    0.3128351689412105
0.8546221682164951      0.3019701178642264    0.31247228668339894
0.8547963333131628      0.3017558211674598    0.3120501568969798
0.8549673148291401      0.3015446517474415    0.31172676873723315
0.8551267780419731      0.30134700558149136   0.3115688202412747
0.8552997271190231      0.3011318813391321    0.3115341575711194
0.8554611578929289      0.3009303707315905    0.3115411610135045
0.8556360745310515      0.30071125166114904   0.3114671813217402
0.8558078075884836      0.30049534049305765   0.3112322711127483
0.8559680223427715      0.300293216693076     0.31087745642643544
0.8561417229612762      0.3000733257587986    0.31043871701577497
0.8563039052766367      0.2998673112545159    0.310093244488744
0.8564629040113066      0.2996646828100928    0.3098906067506409
0.8566353886101936      0.2994441344392458    0.3098253251453359
0.8567963549059363      0.2992376283543055    0.3098337156720729
0.8569708070658959      0.29901307685872924   0.30979063021865333
0.857142075645165       0.2987918734787978    0.3095954260873068
0.8573018259212898      0.2985848800523824    0.3092627400151074
0.8574750620616315      0.29835968861333967   0.3088146799398553
0.8576367798988289      0.29814879308836817   0.3084305139179413
0.8577953141553358      0.29794141763379783   0.308178069598012
0.8579673342760598      0.29771569729348446   0.3080714279096512
0.8581278360936394      0.29750443282309585   0.3080717220697912
0.858301823775436       0.29727470284713986   0.3080557465353338
0.8584642931540882      0.2970595133916137    0.3079170423438921
0.85862357895205        0.29684794495052486   0.3076248716565813
0.8587963506142288      0.2966177855831362    0.3071830131832512
0.8589576039732633      0.2964023207470704    0.3067627072536752
0.8591323431965147      0.2961681311982311    0.30642587494833445
0.8593038988390755      0.2959374980513856    0.3062754965723661
0.8594639361784921      0.29572171799784397   0.3062597443163905
0.8596374593821257      0.29548706885909626   0.3062608381226407
0.8597994642826149      0.2952673541076997    0.30615969803901016
0.8599582856024136      0.295051358378005     0.3059033245382579
0.8601305927864293      0.2948163545950815    0.3054751698297181
0.8602913816673008      0.29459643649033385   0.3050339906102607
0.8604656564123891      0.2943573962041475    0.3046466365264108
0.860636747576787       0.2941220406838258    0.30444391851097435
0.8607963204380406      0.2939019237368834    0.30440150026932333
0.860969379163511       0.2936625459060234    0.30441170802395506
0.8611309195858372      0.2934384850453143    0.3043483552042971
0.8613059458723804      0.2931950514136226    0.3041035306300109
0.8614777885782331      0.29295537334511007   0.3036882103251206
0.8616381129809415      0.29273116262438803   0.30323501000610836
0.8618119232478668      0.29248744326170023   0.3028119698429244
0.8619742152116479      0.2922592683753152    0.30257570030840286
0.8621333235947384      0.2920350033408014    0.30250156916832044
0.8623059178420458      0.29179109898838923   0.3025107911361716
0.862466993786209       0.29156288242968925   0.30247673722636326
0.8626415555945891      0.29131491924071756   0.3022766655055745
0.8628129338222787      0.2910708342995026    0.3018903325623951
0.8629727937468241      0.29084258186812284   0.3014326885476668
0.8631461395355864      0.29059445259951133   0.3009695300668241
0.8633079670212044      0.29036223007472      0.30068033937584904
0.8634666109261319      0.2901340342972352    0.30056450884901875
0.8636387406952764      0.28988583652945205   0.3005621038264917
0.8637993521612766      0.28965368331683583   0.300553114600816
0.8639734494914937      0.28940142534067526   0.30040235290745465
0.8641443632410203      0.28915316416342357   0.3000565293950726
0.8643037586874026      0.2889210866393884    0.2996059594921807
0.8644766399980018      0.2886687797650528    0.29910927073844834
0.8646380030054568      0.288432727905776     0.2987643485511216
0.8648128518771288      0.28817635548296405   0.2985885351606193
0.8649845171681102      0.28792409693875826   0.2985711630162669
0.8651446641559474      0.28768822556738066   0.29857498990807113
0.8653182970080014      0.28743190685186554   0.2984579241863038
0.8654804115569112      0.28719204493265765   0.2981639223243825
0.8656393425251305      0.2869563833322984    0.29772866063706704
0.8658117593575667      0.286700156582719     0.2972104306853459
0.8659726578868586      0.2864605159684194    0.29681773761890207
0.8661470422803675      0.2862002134631437    0.2965860146786937
0.866318243093186       0.28594408286545836   0.29653908424519626
0.8664779256028601      0.2857046689901791    0.2965521395477888
0.8666510939767511      0.28544447587100236   0.2964782469023673
0.8668127440474979      0.2852010664624857    0.29623198600983325
0.8669712105375541      0.2849619629668659    0.29582393792175116
0.8671431628918272      0.284701967450176     0.29529466264111487
0.8673035969429561      0.2844588799704376    0.2948564581480197
0.867477516858302       0.2841948078602734    0.2945614975075311
0.8676399184705036      0.28394770962498284   0.29447279952091604
0.8677991365020147      0.28370497486900276   0.29448259761626516
0.8679718403977427      0.2834411452140177    0.2944540833053245
0.8681330259903264      0.28319441152644437   0.2942716744122018
0.8683076974471271      0.28292649239268236   0.2938654867595353
0.8684791853232373      0.28266291073724387   0.29333622148442917
0.8686391548962032      0.2824165482034294    0.2928632121573763
0.8688126103333861      0.282148890591502     0.29251043528620485
0.8689745474674246      0.2818985153063224    0.2923765779340652
0.8691333010207727      0.28165260393856206   0.29237238200526067
0.8693055404383376      0.28138529224654774   0.2923693523887897
0.8694662615527583      0.28113537996602517   0.29223485418819345
0.869640468531396       0.2808639809421133    0.2918759513498407
0.8698114919293432      0.2805970213400394    0.29135927509248755
0.869970997024146       0.28034757919355346   0.29085821466415535
0.8701439879831658      0.2800765457831106    0.29044536663890164
0.8703054606390414      0.2798230904210293    0.29025623720135196
0.8704804191591339      0.27954795949978145   0.29022725317072706
0.8706521940985358      0.2792773242758401    0.29023656035963136
0.8708124507347934      0.2790243827526081    0.290133943847256
0.870986193235268       0.2787496638304244    0.2898109455669849
0.8711484174325984      0.2784927364748139    0.2893355441877482
0.8713074580492383      0.27824044935128117   0.2888177906393924
0.871479984530095       0.27796629217112545   0.28835452920723553
0.8716409927078075      0.2777099925317354    0.28811155573317015
0.871815486749737       0.27743174195509757   0.2880477076752867
0.8719867972109758      0.2771580811752721    0.28806496350815153
0.8721465893690704      0.2769023878792326    0.2880029797001019
0.872319867391382       0.2766246457040822    0.287734523826057
0.8724816271105493      0.27636492743424235   0.28729019544249546
0.8726402032490261      0.2761099118299501    0.286765748397176
0.8728122652517198      0.2758327533187514    0.28625501373971385
0.8729728089512693      0.27557372324381574   0.2859519002203609
0.8731468385150357      0.2752924730541955    0.2858400594108248
0.8733093497756579      0.27502940668030607   0.2858523720035857
0.8734686774555895      0.27477109182292736   0.28583259029441366
0.8736414909997381      0.27449046509263664   0.28563599491821823
0.8738027862407424      0.27422812468559765   0.285245152616694
0.8739775673459635      0.2739433970632456    0.2846739525935146
0.8741491648704942      0.2736633998463279    0.28412774470634433
0.8743092440918806      0.27340179225041067   0.28377067069091655
0.874482809177484       0.27311770640002564   0.2836080312945971
0.8746448559599431      0.2728520632098084    0.2836052203248902
0.8748037191617117      0.2725912565290907    0.28360849896513174
0.8749760682276972      0.27230788429938313   0.2834647490812012
0.8751368989905385      0.27204305278620894   0.28312137609233357
0.8753112156175967      0.27175558405060896   0.2825685545191034
0.8754823486639643      0.2714729320511549    0.281994301762186
0.8756419634071877      0.27120891952356435   0.28158154476163694
0.875815064014628       0.270922183277407     0.28135757091865465
0.8759766463189241      0.2706541372202185    0.2813269089419221
0.8761350450425297      0.27039100935904525   0.28134455851812834
0.8763069296303522      0.2701050749131059    0.2812537707649134
0.8764672959150304      0.26983792429638953   0.28096724959588204
0.8766411480639256      0.2695478990615398    0.2804478887271914
0.8768034819096765      0.269276707308229     0.2798850372433137
0.8769626321747368      0.2690104782868255    0.27941074320686365
0.8771352683040141      0.26872129409546774   0.2791044902253989
0.8772963861301472      0.268451034958489     0.2790231971740014
0.8774709898204971      0.26815781746605016   0.2790432231522325
0.8776424099301565      0.26786955327912043   0.2789938009542928
0.8778023117366717      0.26760030093465753   0.27876046592072445
0.8779756994074038      0.2673079523546595    0.2782811062974534
0.8781375687749917      0.26703466245874086   0.277716621570974
0.8782962545618891      0.26676640977340765   0.2772023185275318
0.8784684262130034      0.2664749835649389    0.27683109189959537
0.8786290795609734      0.26620270267445156   0.2767005496536051
0.8788032187731603      0.26590718480280806   0.2767085864028017
0.8789741744046566      0.2656166865305662    0.27669527562270485
0.8791336117330087      0.2653454208482273    0.27651981820736476
0.8793065349255779      0.26505084156682884   0.2760936980521182
0.8794679398150027      0.2647755396890545    0.27554113700641925
0.8796428305686445      0.26447686242554286   0.2749426191322879
0.8798145377415957      0.26418324677973826   0.27452557080769563
0.8799747266114026      0.263908993926152     0.27435561361516786
0.8801484013454266      0.2636112911779316    0.27434735685293404
0.8803105577763063      0.2633329950857026    0.27435557042515485
0.8804695306264954      0.26305984836124174   0.274229661836464
0.8806419893409014      0.2627631803822999    0.27385742128734486
0.8808029297521632      0.26248599999289657   0.27332650134953806
0.8809773560276419      0.2621852397631254    0.272704636885351
0.8811485987224301      0.2618896131852424    0.27222877803369194
0.881308323114074       0.261613555622245     0.27200038707072355
0.881481533369935       0.26131384767335986   0.2719589369306079
0.8816432253226516      0.26103375057719597   0.2719812075134724
0.8818017336946776      0.2607588708310438    0.2719044760169514
0.8819737279309207      0.2604602732021774    0.2715956194415293
0.8821342038640194      0.26018136350685356   0.27110007105176925
0.8823081656613351      0.2598786805184342    0.27046700039784843
0.8824706091555065      0.25959572635732703   0.2699560463823925
0.8826298690689873      0.259318026612586     0.2696497497651101
0.8828026148466852      0.2590164881277714    0.26955021037573507
0.8829638423212388      0.25873475371812416   0.2695723895639621
0.8831385556600093      0.25842912684221275   0.26953578028747494
0.8833100854180893      0.25812874055745966   0.26928523661109366
0.883470096873025       0.2578482339421453    0.26882999339266916
0.8836435941921776      0.2575437927553557    0.2681989878570952
0.883805573208186       0.2572593110944905    0.2676480242402523
0.8839643686435038      0.2569801462087352    0.267281595218283
0.8841366499430386      0.25667696737451595   0.26712748042816326
0.8842974129394292      0.2563937731642287    0.2671360337734752
0.8844716618000367      0.25608651372951874   0.26713350109560785
0.8846427270799536      0.25578455722863347   0.26694594234595953
0.8848022740567263      0.25550265648175885   0.2665428789903766
0.884975306897716       0.2551966287372308    0.2659282855910634
0.8851368214355614      0.2549106932897596    0.26534484988908136
0.8853118218376237      0.25460058103313654   0.2648832357756214
0.8854836386589955      0.2542958063811269    0.26468679993364774
0.885643937177223       0.2540111935048925    0.2646790495715685
0.8858177215596674      0.253702343893673     0.26469387984145515
0.8859799876389676      0.25341369176808304   0.26456143615064615
0.8861390701375772      0.2531304486863037    0.26421011131371264
0.8863116385004038      0.2528229115565298    0.2636214533115166
0.8864726885600862      0.25253563765110104   0.2630191482058833
0.8866472244839855      0.2522240226481979    0.2624994669622999
0.8868185768271942      0.25191780449191326   0.2622398655135888
0.8869784108672587      0.2516319156183283    0.2621998225148657
0.8871517307715401      0.25132162914156614   0.2622280487190213
0.8873135323726772      0.25103170590898144   0.26214678033555405
0.8874721503931238      0.25074724746540544   0.2618557036853892
0.8876442542777874      0.25043833743890465   0.2613073368208528
0.8878048398593067      0.2501498530705999    0.2606973149385697
0.8879789113050429      0.2498368728100224    0.26012241212443465
0.8881497991700886      0.24952934613985855   0.25979131396129074
0.8883091687319901      0.24924230779261664   0.2597051764760071
0.8884820241581084      0.24893072044056305   0.25973335867063807
0.8886433612810825      0.24863965363830834   0.2596991624051033
0.8888181842682735      0.2483239950139912    0.25943772542778565
0.8889898236747741      0.24801382134875483   0.25892328211816557
0.8891499447781304      0.24772422921336273   0.2583148132231812
0.8893235517457035      0.24740999402470787   0.25770577051840976
0.8894856404101324      0.2471163717423423    0.2573353909551183
0.8896445454938708      0.24682829719141533   0.25719961627468474
0.8898169364418262      0.24651553075327687   0.257214735700462
0.8899778090866374      0.24622347600582414   0.25721352016435256
0.8901521675956654      0.24590670552632857   0.2570170590977478
0.8903233425240029      0.2455954717277026    0.2565583939566885
0.8904829991491962      0.2453049614947507    0.2559646212295545
0.8906561416386063      0.24498967489129864   0.2553200030526021
0.8908177658248722      0.2446951411326744    0.2548849698437533
0.8909762064304475      0.24440620263007987   0.25468828998939874
0.8911481329002399      0.24409244126133395   0.25467530143601497
0.891308541066888       0.243799486536811     0.25469799002024673
0.8914824350977529      0.24348167078066077   0.2545678749259556
0.8916448108254736      0.24318469023007294   0.254202352835761
0.8918040029725038      0.2428933311550674    0.2536459079507586
0.8919766809837509      0.24257706610965335   0.2529759004668755
0.8921378406918538      0.24228168867556885   0.25246695131476293
0.8923124862641736      0.24196136838459298   0.25217087034542757
0.8924839482558029      0.24164666050210065   0.2521212933474083
0.892643891944288       0.2413528927994922    0.2521539720118809
0.8928173214969899      0.24103413811325977   0.25207717255999107
0.8929792327465476      0.240736350655462     0.25177390843059577
0.8931379604154147      0.24044422972013102   0.25125748815468685
0.8933101739484988      0.24012707974595107   0.2505826229937471
0.8934708691784385      0.23983094652948145   0.25002498254181277
0.8936450502725953      0.23950974975824213   0.2496560571150345
0.8938160477860616      0.23919421130608726   0.24955555046605812
0.8939755269963836      0.23889973924382782   0.2495859096437366
0.8941484920709225      0.23858016243565638   0.24955775199449895
0.8943099388423171      0.23828167756510868   0.24932324703343364
0.8944848714779288      0.23795805474785933   0.24880096792914141
0.8946566205328499      0.2376401156459469    0.24812994339779954
0.8948168512846267      0.23734331662775796   0.2475431578622472
0.8949905679006205      0.23702134012935852   0.24712296882002954
0.89515276621347        0.23672052851197256   0.24698237895523403
0.8953117809456289      0.23642545022141437   0.2469990842293931
0.8954842815420048      0.23610515687684588   0.24700451352726271
0.8956452638352363      0.23580607363274733   0.24683007550289274
0.8958197319926849      0.23548174450182      0.2463672482894844
0.895991016569443       0.23516314192693816   0.24571420219915516
0.8961507828430568      0.23486580208790556   0.24509691000210232
0.8963240349808875      0.2345432135417681    0.2446088024694051
0.896485768815574       0.2342419006695456    0.2444063931618815
0.8966443190695699      0.23394635987559292   0.2443958968962554
0.8968163551877827      0.23362550449156483   0.24442536171449936
0.8969768730028512      0.23332596675043601   0.24431196496710073
0.8971508766821368      0.23300108547095746   0.24392032073666087
0.897313362058278       0.23269754414595717   0.24333461488033895
0.8974726638537288      0.23239979591892812   0.2426961905182955
0.8976454515133965      0.23207667029574952   0.24213023897182706
0.8978067208699199      0.23177492532530392   0.24184268760413258
0.8979814760906603      0.2314477751743527    0.24178243505612954
0.89815304773071        0.23112641199122005   0.24182152768667287
0.8983131010676155      0.2308264701579404    0.24175678976527987
0.898486640268738       0.23050109015096887   0.2414320048346938
0.8986486611667163      0.23019715246493702   0.2408871350880033
0.898807498484004       0.22989904370394476   0.2402459514254988
0.8989798216655086      0.22957546546587448   0.2396295045166462
0.899140626543869       0.22927336769840406   0.23927508763452804
0.8993149172864463      0.2289457747696256    0.2391610641211024
0.8994860244483331      0.22862400540066494   0.23919706598922535
0.8996456133070757      0.2283237545974569    0.23917677501570409
0.8998186880300352      0.22799797824054951   0.23892555254693443
0.8999802444498504      0.22769374009065546   0.238435309475222
0.900138617288975       0.22739536479810638   0.23780525531805116
0.9003104759923166      0.22707143518219267   0.23714574449125034
0.9004708163925139      0.22676907939485236   0.23672029220871735
0.9006446426569282      0.2264411456831442    0.23653799020406854
0.9008069506181983      0.22613480477342895   0.2365520771104289
0.9009660749987778      0.22583434522566959   0.236572195740054
0.9011386852435742      0.22550828042719495   0.23641057561633513
0.9012997771852264      0.2252038427558218    0.2360013408045274
0.9014743549910955      0.22487377741794418   0.23533738688018224
0.901645749216274       0.22454958934822872   0.23465095782179182
0.9018056251383083      0.22424706257904892   0.2341655256218129
0.9019789869245596      0.22391888182414427   0.2339154747810898
0.9021408304076666      0.22361238001463313   0.23389885284561598
0.902299490310083       0.22331179077803143   0.233936369284733
0.9024716360767164      0.22298553275744193   0.23383706757937966
0.9026322635402055      0.2226809942368934    0.23349550878162775
0.9028063768679117      0.2223507580441612    0.23287355675594276
0.9029773066149271      0.2220264304979907    0.23217209438729933
0.9031367180587984      0.22172384369371798   0.23162833570073899
0.9033096153668866      0.22139553516708566   0.23130069395983852
0.9034709943718305      0.22108898358812273   0.23123801367006164
0.9036458592409914      0.22075669091731437   0.23128294162403754
0.9038175405294617      0.22043032420848327   0.23122254400025802
0.9039777035147878      0.22012574454261163   0.2309297121664283
0.9041513523643309      0.21979540116761964   0.23034433340274182
0.9043134829107297      0.21948686039036022   0.2296750764451485
0.9044724298764379      0.21918427669877466   0.22908705564510717
0.9046448627063631      0.21885590806229308   0.2286889893287725
0.904805777233144       0.21854936993760976   0.22857537423181962
0.904980177624142       0.21821702945585236   0.2286128354301496
0.9051513944344493      0.21789064359459231   0.22859825825110983
0.9053110929416123      0.21758611591461807   0.22837323087838157
0.9054842773129923      0.2172557656843622    0.22784782141477808
0.9056459433812281      0.2169472879504835    0.22719046478422225
0.9058044258687733      0.21664479348139556   0.2265649385798917
0.9059763942205354      0.2163164575872439    0.2260922959817292
0.9061368442691533      0.21601001974719453   0.22591491137666042
0.9063107801819882      0.21567772501132002   0.22592869737355853
0.9064731977916788      0.21536734199733742   0.22595302825474303
0.9066324318206788      0.215062956527914     0.22581055768876718
0.9068051517138958      0.21473269668360406   0.22537314660848792
0.9069663533039685      0.21442437286241695   0.2247511817636793
0.9071410407582581      0.21409016034138378   0.22403257523652562
0.9073125446318572      0.2137619441005828    0.2234975162116936
0.907472530202312       0.2134556878481847    0.22325785388788946
0.9076460016369838      0.21312352654984706   0.2232375492065628
0.9078079547685114      0.21281333766920915   0.2232799458568789
0.9079667243193483      0.21250916994153257   0.2231942376917973
0.9081389797344023      0.21217908204716765   0.2228290856477181
0.9082997168463119      0.21187098851778732   0.2222477727816498
0.9084739398224385      0.21153696243604345   0.22151815816504145
0.9086449792178746      0.21120895694870165   0.22092229828331392
0.9088045003101664      0.2109029635177525    0.2206116094089501
0.9089775072666751      0.210571023065524     0.2205414072470978
0.9091389959200397      0.21026111006666495   0.2205891608737657
0.9093139704376211      0.2099252390782039    0.2205417526030252
0.909485761374512       0.20959540236018775   0.2202276491309398
0.9096460340082586      0.20928761336380106   0.21968060761094788
0.9098197925062222      0.20895385398785796   0.21895187152244802
0.9099820327010416      0.20864215288366675   0.21834494072865904
0.9101410893151703      0.20833650716452312   0.2179703976935468
0.9103136317935161      0.20800487981729768   0.21784606084005384
0.9104746559687176      0.20769532906584723   0.217885900000876
0.910649166008136       0.20735978748332776   0.21788239786819646
0.9108204924668638      0.20703030202320605   0.21764010020544328
0.9109803006224474      0.2067229109924722    0.21714896134040199
0.911153594642248       0.206389519172808     0.21643431413882372
0.9113153703589043      0.20607823108859322   0.21578889711819904
0.91147396249487        0.2057730177532172    0.2153466280906402
0.9116460404950527      0.2054417947484808    0.2151548557435086
0.9118066001920911      0.20513269143904933   0.21517235459327544
0.9119806457533465      0.20479757120479092   0.21520367005391747
0.9121515077339113      0.20446852716522146   0.21503626760079678
0.9123108514113318      0.20416161823961904   0.21461360662368623
0.9124836809529694      0.20382868484355454   0.21393017078168855
0.9126449921914627      0.2035178946336065    0.2132568855813439
0.9128197892941728      0.20318107391149176   0.21270666039476657
0.9129914028161925      0.20285034058030169   0.21246414864161955
0.9131514980350679      0.20254176447570812   0.21245821896165934
0.9133250791181602      0.20220715184020813   0.21250555391681497
0.9134871418981082      0.2018947038061477    0.21239981249783646
0.9136460210973657      0.20158835792704036   0.21204298763186138
0.9138183861608402      0.20125597048089947   0.2114005253407283
0.9139792329211704      0.2009457598984488    0.21071556731493152
0.9141535655457176      0.20060950365258307   0.21010427360066863
0.9143247145895742      0.2002793522360308    0.209787594423767
0.9144843453302866      0.19997138931108063   0.2097385330236492
0.9146574619352159      0.19963737711351592   0.20979458285843894
0.914819060237001       0.19932555954972897   0.2097458695032594
0.9149774749580954      0.19901984859510732   0.2094628734289194
0.9151493755434069      0.19868806515508972   0.20887745051367967
0.9153097578255741      0.19837848851546297   0.20819486809944066
0.9154836259719582      0.19804285564150953   0.2075281742678464
0.915645975815198       0.19772943523976416   0.20714208624126518
0.9158051420777473      0.197422141210406     0.20702282522416024
0.9159777942045136      0.19708879023630255   0.20706725197711612
0.9161389280281357      0.19677766055533794   0.20707551146904077
0.9163135477159746      0.19644047339200077   0.20684928982349796
0.916484983823123       0.19610941707319868   0.2063240252179582
0.9166449016271272      0.19580059005619813   0.2056568787215522
0.9168183052953482      0.1954657065270765    0.20495220918946297
0.916980190660425       0.19515305662058646   0.20449849692833139
0.9171388924448113      0.19484654621601907   0.20431736726131772
0.9173110800934146      0.19451398115912943   0.20433666955176763
0.9174717494388736      0.19420365597900346   0.2043749927577813
0.9176459046485494      0.1938672777229758    0.20422277389055454
0.9178168762775348      0.19353704420961595   0.20377059593433455
0.9179763296033759      0.1932290559590994    0.20313410184603653
0.918149268793434       0.19289501822162697   0.20240244039120006
0.9183106896803477      0.19258322872798878   0.20188042128044864
0.9184855964314784      0.19224539270113455   0.20161452586478018
0.9186573196019187      0.191913709248172     0.20160893767092622
0.9188175244692146      0.19160427785859943   0.20166075111839488
0.9189912152007275      0.19126880525619458   0.2015565398910056
0.919153387629096       0.19095558690502504   0.20118538470511782
0.9193123764767741      0.19064852551919265   0.2005875870155016
0.9194848511886692      0.19031542902630294   0.19984546206877318
0.91964580759742        0.1900045889546091    0.19926938590804888
0.9198202498703877      0.1896677188412322    0.19892921508648984
0.9199915085626649      0.18933701229061348   0.19887790527244972
0.9201512489517978      0.18902856336078053   0.19893676519251932
0.9203244752051476      0.18869409232539203   0.19889174861295697
0.9204861831553531      0.18838187976016302   0.19859495800713745
0.9206447075248682      0.18807583247215048   0.19805012664140476
0.9208167177586002      0.18774377172987602   0.19731186786735969
0.920977209689188       0.18743396908509374   0.19668676795928683
0.9211511874839926      0.18709816014536454   0.196265318922233
0.921313646975653       0.18678456613150854   0.19615247911236391
0.9214729228866229      0.1864771314380298    0.1962001452207269
0.9216456846618096      0.18614369667367597   0.1962147047209464
0.9218069281338521      0.18583252265540817   0.19601169086277506
0.9219816574701116      0.18549535771597675   0.1954851692663834
0.9221532032256806      0.18516437280826545   0.1947638514891705
0.9223132306781053      0.1848556461677204    0.19410546522154004
0.922486743994747       0.18452094190342733   0.19361387687369846
0.9226487390082443      0.1842084949228541    0.19344020490391522
0.9228075504410511      0.18390222602546893   0.19346519358819045
0.9229798477380748      0.18356999354040457   0.19351117513083874
0.9231406267319543      0.18326001412156306   0.19337557976344272
0.9233148915900508      0.18292408270717547   0.19292351049230133
0.9234859728674567      0.18259433922612914   0.19223465800483155
0.9236455358417184      0.18228684330126796   0.19155253806477585
0.923818584680197       0.18195341170394042   0.1909898883275865
0.9239801152155313      0.18164222513096367   0.19074412108732283
0.9241551316150826      0.18130511613061526   0.19073726670774327
0.9243269644339434      0.18097419964392086   0.1907975120015703
0.9244872789496599      0.1806655207889157    0.1907051298759081
0.9246610793295933      0.18033093790636917   0.19030751302467333
0.9248233614063824      0.18001858916994903   0.18968537139680874
0.9249824599024811      0.17971242599444953   0.18899417948778505
0.9251550442627966      0.1793803778082983    0.18837546050536896
0.9253161103199679      0.17907055475953856   0.18806280093629096
0.9254906622413561      0.17873486238482553   0.1880092452852999
0.9256620305820539      0.17840536701246035   0.18807779762301813
0.9258218806196074      0.17809808635676272   0.1880396441285404
0.9259952165213777      0.17776495779716178   0.18772020339182663
0.9261570341200038      0.17745403892187392   0.18714966469716598
0.9263156681379394      0.17714930700256615   0.1864618458202931
0.9264877880200919      0.17681874913648166   0.18579176214686915
0.9266483895991001      0.1765103890081992    0.18540574593165535
0.9268224770423253      0.17617622103678668   0.1852901737098361
0.92699338090486        0.17584825234139465   0.1853521348152711
0.9271527664642505      0.17554246796024972   0.18536221027040337
0.9273256378878578      0.17521090015479465   0.18512706908733406
0.9274869910083209      0.17490149331637656   0.18462299374524171
0.927661829993001       0.17456625475329934   0.18388105271839064
0.9278334853969904      0.17423721896494648   0.1831829845199252
0.9279936224978356      0.1739303518911139    0.18274771928993672
0.9281672454628977      0.1735977424302881    0.18258375782799835
0.9283293501248157      0.17328729470894394   0.18262801334941844
0.928488271206043       0.17298303583438746   0.18267032428850558
0.9286606781514873      0.17265306289797333   0.18250740350982803
0.9288215667937874      0.17234523555160539   0.18207007014999144
0.9289959413003044      0.17201171749095798   0.1813592152358223
0.9291671322261309      0.17168440374237684   0.1806325273952914
0.9293268048488131      0.17137921760781652   0.18013175768239642
0.9294999633357123      0.17104837209972215   0.17989419360528283
0.9296616035194671      0.17073964537831907   0.17990649563584624
0.9298200601225315      0.17043710533640238   0.17997067433722747
0.9299920025898127      0.17010893776628588   0.17987979777972532
0.9301524267539497      0.16980286931576868   0.17951991343222692
0.9303263367823037      0.16947119958089893   0.17885762942023262
0.9304887285075134      0.16916161898463875   0.17815034660058712
0.9306479366520326      0.1688582239112548    0.1775753975348385
0.9308206306607687      0.16852926186305375   0.1772377757227718
0.9309818063663605      0.16822236704764829   0.17719064917511734
0.9311564679361694      0.16788993352194917   0.17726740303195454
0.9313279459252876      0.16756370289636194   0.177232956130883
0.9314879056112615      0.16725951562819588   0.176944286290171
0.9316613511614524      0.1669298271163295    0.1763372504447139
0.9318232784084991      0.1666221701304641    0.17563275839232434
0.9319820220748553      0.1663206924039873    0.17501249321585471
0.9321542516054283      0.16599375125934962   0.17459892722862339
0.9323149628328571      0.16568881622098916   0.17449545629789395
0.9324891599245029      0.16535844892192372   0.17456308915749894
0.932660173435458       0.16503427930977072   0.1745792636699335
0.9328196686432689      0.16473208833616254   0.17436735741863632
0.9329926497152968      0.16440450615261856   0.17383076370629563
0.9331541124841805      0.1640988889414461    0.1731443953262694
0.933329061117281       0.16376791374581565   0.17242666934602788
0.933500826169691       0.1634431336780899    0.17196321484581267
0.9336610729189567      0.16314028875514935   0.1718169959963323
0.9338348055324395      0.16281205893620962   0.17187003461848696
0.9339970198427779      0.1625057213700254    0.1719159917522006
0.9341560505724259      0.16220555182254087   0.17176964060474226
0.9343285671662908      0.1618801048328851    0.17130309041536476
0.9344895654570113      0.16157655528999396   0.17064610795815383
0.9346640496119488      0.16124776557558917   0.16990292550692845
0.9348353501861958      0.16092516607666896   0.16937175713905026
0.9349951324572985      0.16062443141154234   0.1691596633793157
0.9351684005926182      0.16029850573048277   0.16917963061114277
0.9353301504247936      0.15999442866872451   0.1692498913275021
0.9354887166762785      0.1596965094202765    0.16916932746654462
0.9356607687919803      0.15937344865120698   0.16878399251201387
0.9358213026045379      0.15907220211666898   0.1681717954108835
0.9359953222813124      0.15874585482661355   0.16741619487058118
0.9361578236549427      0.15844130413007124   0.16684182355717542
0.9363171414478824      0.1581429057167687    0.16653948010078073
0.936489945105039       0.1578194590867222    0.16649788613380773
0.9366512304590514      0.15751777189727906   0.1665783949103672
0.9368260016772807      0.1571910797514947    0.16655603463975235
0.9369975893148195      0.15687056441474498   0.16624510905103343
0.937157658649214       0.1565717687192378    0.1656831398296697
0.9373312138478255      0.15624802468508736   0.16493096561395618
0.9374932507432927      0.15594598041804264   0.16431052451003073
0.9376521040580693      0.15565007323861552   0.1639404839781355
0.9378244432370629      0.15532927450112252   0.16384139974758913
0.9379852641129123      0.15503013413773103   0.16391453865939506
0.9381595708529785      0.1547061489670716    0.16394173425266856
0.9383306940123542      0.15438832667655578   0.16371142504034675
0.9384902988685857      0.15409211862542865   0.1632132050745531
0.9386633895890342      0.15377112670931475   0.16247983956068884
0.9388249620063384      0.15347172694437083   0.16182055532237766
0.938983350842952       0.15317844590991483   0.16137922636822263
0.9391552255437826      0.15286044199592236   0.16120860484008714
0.9393155819414689      0.1525639846365789    0.161258643763519
0.9394894242033721      0.15224285460873072   0.16132503135807744
0.939651748162131       0.15194324759705927   0.16119242853066043
0.9398108885401995      0.15164974934251949   0.16078540573217612
0.9399835147824849      0.15133164249515893   0.16009750500018752
0.940144622721626       0.1510348997460566    0.1594078859741927
0.9403192165249841      0.1507135908324657    0.15884275821980795
0.9404906267476516      0.1503984220497833    0.15860331286418608
0.9406505186671749      0.1501046858579323    0.1586206377290359
0.9408238964509151      0.14978645515017505   0.15870827635608747
0.9409857559315111      0.14948963203982088   0.15864032893097968
0.9411444318314165      0.14919889910572984   0.15830773912951515
0.9413165935955389      0.14888374198603674   0.15767047780488921
0.941477237056517       0.148589940615519     0.15697492611955094
0.941651366381712       0.1482717729652721    0.15634929985144633
0.9418223121262165      0.1479597271065482    0.1560316578165204
0.9419817395675767      0.14766898180809107   0.15600128745443878
0.9421546528731539      0.1473539455998847    0.156095961652502
0.9423160478755869      0.14706018230920173   0.15608875464202546
0.9424909287422367      0.14674218908198844   0.1557952992028452
0.9426626260281961      0.14643030802695098   0.15520064373314632
0.9428228050110111      0.1461396411971524    0.15451074683457292
0.9429964698580431      0.14582482365018004   0.15385101124706768
0.9431586164019308      0.14553119090892008   0.15349039091465036
0.943317579365128       0.1452436115330496    0.153409006982029
0.9434900281925421      0.14493196058100677   0.15349608821416696
0.943650958716812       0.144641434110255     0.15353319065003998
0.9438253751052988      0.14432690120800518   0.15331945663263558
0.9439966079130951      0.14401845610055566   0.15279159820194432
0.9441563224177472      0.14373107159974147   0.1521199818964721
0.9443295227866161      0.14341976505538162   0.15142152096685824
0.9444912048523407      0.14312948705161607   0.1509907919832965
0.9446497033373749      0.14284523274274571   0.15084603853039172
0.944821687686626       0.14253714055610164   0.15090978432577673
0.9449821537327329      0.1422500115616519    0.15098232457285413
0.9451561056430566      0.14193911396932102   0.15085079506435436
0.9453185392502361      0.14164914573479      0.15043213908713343
0.9454777892767251      0.14136518506454696   0.14980238889170325
0.9456505251674311      0.14105754372678952   0.14907328845860057
0.9458117427549928      0.14077076296723093   0.14855934797618717
0.9459864462067714      0.14046037390580943   0.14831619404948546
0.9461579660778595      0.14015602964559903   0.14834732142961662
0.9463179676458033      0.13987242540274392   0.14843957371650426
0.9464914550779641      0.1395652278688697    0.14837532963826605
0.9466534242069805      0.13927879100192578   0.14803032252359324
0.9468122097553064      0.13899832945407647   0.1474463850063313
0.9469844811678494      0.1386944380630644    0.14671169714909987
0.9471452342772481      0.13841123474906483   0.14614460297960033
0.9473194732508636      0.13810467953702604   0.14582528280050178
0.9474905286437887      0.13780414060187426   0.14580869656969186
0.9476500657335695      0.13752421294345213   0.145908558419197
0.9478230886875672      0.1372210342768944    0.14590811262001518
0.9479845933384206      0.13693842831671657   0.14564472168751533
0.948159583853491       0.13663265294771518   0.14505496366630555
0.9483313907878709      0.1363328779539224    0.1443264793593035
0.9484916794191065      0.13605359469022443   0.14373033737932447
0.948665453914559       0.13575124749231865   0.14336039681475995
0.9488277101068673      0.13546935133574445   0.14330309303194846
0.948986782718485       0.13519337421480482   0.1433979229223452
0.9491593411943197      0.13489443824174133   0.14344452335843585
0.94932038136701        0.13461587072357314   0.14325320501887587
0.9494949074039174      0.13431443083514777   0.1427305499601374
0.9496662498601343      0.13401895425877328   0.1420204470320786
0.9498260740132068      0.13374375909401393   0.14139001088773326
0.9499993840304963      0.13344580308348744   0.1409487194354504
0.9501611757446415      0.13316808469769797   0.140831663836758
0.9503197838780962      0.1328962420880499    0.14090800797869882
0.9504918778757679      0.13260174959542353   0.1409933521665593
0.9506524535702954      0.13232740626040618   0.1408767699646431
0.9508265151290397      0.13203050436421      0.14043440645903893
0.9509973931070935      0.13173952427571214   0.1397589447559908
0.9511567527820031      0.1314686002866125    0.1391041669512508
0.9513295983211295      0.13117523525818325   0.13859053082332523
0.9514909255571117      0.13090187947098586   0.13840199701196787
0.9516657386573109      0.13060617767426846   0.13845540380695542
0.9518373681768195      0.13031637544168462   0.1385602308747944
0.9519974793931839      0.13004648515123143   0.13849276472252656
0.9521710764737652      0.12975437102391682   0.13811070359111155
0.9523331552512022      0.12948211982857938   0.13750469889015154
0.9524920504479487      0.1292156702149804    0.1368416063844832
0.9526644315089121      0.12892701579548652   0.13627159622147672
0.9528252942667312      0.12865811557613233   0.13601678101833511
0.9529996428887674      0.12836720428114218   0.13602676979063208
0.953170807930113       0.12808214765396864   0.13614588081474188
0.9533304546683143      0.12781676191347052   0.13613936583525202
0.9535035872707326      0.127529495159795     0.1358419046531655
0.9536652015700066      0.12726184792542194   0.13529116531276705
0.95382363228859        0.12699995266735836   0.13463265692229642
0.9539955488713904      0.1267163057452256    0.13401120073706044
0.9541559471510465      0.12645217471805942   0.13368353074935377
0.9543298312949196      0.1261663984807446    0.13363443928571386
0.9544921971356484      0.1259000843581565    0.13374657387694663
0.9546513793956867      0.1256394942969559    0.13380306595807182
0.9548240475199419      0.1253573935061905    0.13361215915145805
0.9549851973410528      0.12509464652420701   0.13314491680282767
0.9551598330263807      0.12481049947225019   0.13244220521405714
0.9553312851310182      0.12453213163404592   0.13178671681110118
0.9554912189325113      0.12427300386537166   0.1313970182805594
0.9556646385982214      0.12399261820091824   0.13128776503153464
0.9558265399607871      0.12373141529769012   0.13138480774184946
0.9559852577426624      0.12347587703491857   0.1314769535054325
0.9561574613887546      0.12319922205199252   0.13136368700330153
0.9563181467317025      0.12294163482823417   0.13096898860570927
0.9564923179388674      0.12266304705442749   0.13030314317302719
0.9566633055653417      0.12239018107113991   0.12962314522547408
0.9568227748886718      0.12213626225802682   0.1291702383114312
0.9569957300762189      0.12186149189760452   0.12898896688908493
0.9571571669606216      0.12160560790168877   0.12905581846703615
0.9573320897092413      0.12132899282695923   0.1291798399770639
0.9575038288771704      0.12105806864807603   0.1291172813720367
0.9576640497419553      0.12080590532587925   0.12877634633997528
0.9578377564709571      0.12053316531761267   0.12814591221766083
0.9579999448968147      0.12027912286379429   0.12749135126503172
0.9581589497419818      0.12003064474473364   0.12698950154922786
0.9583314404513658      0.11976174304693613   0.12674202447709315
0.9584924128576054      0.11951141242194961   0.12677185158347518
0.9586668711280621      0.11924078419058867   0.12690961974677345
0.9588381458178282      0.11897574918419034   0.12691122380563058
0.95899790220445        0.11872908812894577   0.12664624596930085
0.9591711444552888      0.11846228308641797   0.12607286915089844
0.9593328684029834      0.11821385787380363   0.1254208209780007
0.9594914087699874      0.1179709275950558    0.12487371437193648
0.9596634350012083      0.1177080145913671    0.12455301006224867
0.959823942929285       0.11746334922357796   0.12453227522851397
0.9599979367215786      0.11719883382871712   0.12466957976986813
0.960160412210728       0.11695249734754157   0.1247322768355921
0.9603197041191869      0.11671161806257807   0.12456370195527965
0.9604924818918626      0.11645105578923833   0.12407632244920597
0.9606537413613941      0.11620853496076725   0.12344567091332666
0.9608284866951425      0.1159464685754024    0.12280148272928974
0.9610000484482004      0.11568992619103377   0.12241872883643791
0.961160091898114       0.11545128133824731   0.12234692354561494
0.9613336212122446      0.11519326690362972   0.12247139191845975
0.961495632223231       0.11495307738722552   0.12257297559479735
0.9616544596535268      0.11471826669312586   0.12247465721856357
0.9618267729480395      0.11446426083769741   0.12206159990830513
0.961987567939408       0.11422793504730516   0.1214635603314838
0.9621618487949934      0.11397255760331083   0.12079717144816779
0.9623329460698883      0.11372262798840908   0.12035039075659196
0.9624925250416388      0.11349022697048235   0.12021699472040195
0.9626655898776064      0.11323895780391394   0.12031384983080462
0.9628271364104297      0.11300514076385343   0.12044498762535724
0.9629854993625625      0.11277661865986317   0.12041710140041752
0.9631573481789122      0.11252941014347116   0.1200893618617663
0.9633176786921176      0.1122995015814861    0.11953918678415243
0.96349149506954        0.11205105613711268   0.11886364154491384
0.9636537931438182      0.11181983141556853   0.11837318715599562
0.9638129076374057      0.11159385597448739   0.11815433228802093
0.9639855079952102      0.1113495312472846    0.11819601558319746
0.9641465900498705      0.11112226971387965   0.11834343835904529
0.9643211579687476      0.11087681322597541   0.1183845602819337
0.9644925423069343      0.11063667923309749   0.11813534269836673
0.9646524083419766      0.11041344397985624   0.11763874895545011
0.964825760241236       0.11017221075978287   0.11697036560741217
0.9649875938373511      0.10994779319470721   0.11643795164787883
0.9651462438527757      0.10972849032666579   0.11615623841360641
0.9653183797324172      0.10949135022617815   0.1161463086332895
0.9654789973089144      0.1092708664897553    0.11629176693488107
0.9656531007496285      0.10903273498271389   0.11638751757940952
0.9658240206096521      0.10879983781516613   0.11622280236570168
0.9659834221665314      0.10858342579741023   0.11579293555727649
0.9661563095876278      0.1083495716927725    0.11514678013107126
0.9663176787055798      0.10813211625291294   0.11457962946565789
0.9664925336877488      0.10789738499182684   0.11421096835115076
0.9666642050892272      0.10766784050719132   0.11416400624785703
0.9668243581875614      0.1074545175107776    0.11430241918540514
0.9669979971501126      0.10722413090177      0.11443017548632553
0.9671601178095195      0.1070098762686809    0.11433625309817567
0.9673190548882358      0.10680062943893871   0.11397250803341313
0.9674914778311691      0.1065745289618791    0.11336003581925684
0.9676523824709581      0.10636438287205183   0.11277455797466787
0.967826772974964       0.1061375558661758    0.11234625168494433
0.9679979798982794      0.10591581787396519   0.11223901518633532
0.9681576685184505      0.10570984905117316   0.11235552023210972
0.9683308430028387      0.10548741953379469   0.1125155625481652
0.9684924991840825      0.10528066556838338   0.11249267013851093
0.9686509717846358      0.10507881343309422   0.11220483121266374
0.968822930249406       0.10486071835175531   0.11164101946210873
0.968983370411032       0.10465811344526108   0.11104792213166718
0.9691572964368749      0.10443944473175899   0.11056149520495565
0.9693197041595735      0.1042361695618155    0.11038573448239804
0.9694789283015816      0.1040377385375071    0.11045652927881909
0.9696516383078067      0.10382346784429813   0.11063677646322073
0.9698128300108875      0.10362439947044468   0.1106929012149081
0.9699875075781852      0.10340967580151131   0.11046762135895986
0.9701590015647924      0.10319988201434034   0.10995791565643132
0.9703189772482553      0.1030050913606249    0.10937116036510793
0.9704924387959352      0.10279488013515119   0.10884304507549299
0.9706543820404707      0.10259957131102639   0.1086082075068586
0.9708131417043158      0.10240899095589888   0.10863675266217218
0.9709853872323778      0.10220322194062237   0.10881873976189049
0.9711461144572956      0.10201215632277096   0.1089254767177951
0.9713203275464303      0.10180609242213406   0.10878311280112571
0.9714913570548744      0.10160480919641707   0.10834072244357028
0.9716508682601743      0.1014180248053579    0.10777275458174261
0.9718238653296912      0.10121648265170129   0.10720923540242132
0.9719853440960639      0.10102933479939781   0.10691109493260045
0.9721603087266534      0.1008276256906985    0.10689588357824913
0.9723320897765524      0.10063067311341524   0.10707439791415091
0.9724923525233071      0.10044790186436499   0.1072113945179697
0.9726661011342788      0.10025081936495585   0.10712587808333213
0.9728283314421061      0.10006781046413382   0.10676159905553788
0.972987378169243       0.09988934268861821   0.10622220509015749
0.9731599107605968      0.09969681059599295   0.10563956776299291
0.9733209250488064      0.09951813948423271   0.10529025382042827
0.9734954252012329      0.09932560717407579   0.10521936419495194
0.9736667417729689      0.09913770932386087   0.105380529502398
0.9738265400415606      0.09896345125429071   0.10554982291356876
0.9739998241743691      0.09877559018780001   0.10553870430982129
0.9741615900040335      0.09860125697272586   0.10524747406054855
0.9743201722530073      0.09843133441217147   0.1047492374025929
0.9744922403661981      0.09824806374633027   0.10415713226957084
0.9746527901762446      0.09807810034814789   0.10375698906435557
0.9748268258505081      0.09789499851617696   0.10362091072922139
0.9749893432216272      0.09772508884018914   0.10374075866829915
0.9751486770120559      0.09755951879011807   0.10393374297131251
0.9753214966667014      0.09738107193613663   0.10400740423374825
0.9754827980182027      0.09721559048040333   0.10381245548570842
0.975657585233921       0.09703744741176627   0.10332437691431749
0.9758291888689488      0.09686374199625807   0.10273668394601067
0.9759892742008323      0.09670276629535118   0.10229984614103182
0.9761628453969327      0.09652940221940275   0.10210640928451618
0.9763248982898888      0.09636864855126637   0.10219116607313113
0.9764837676021544      0.09621209397580588   0.1023910524560079
0.976656122778637       0.09604342060349004   0.10252035043383818
0.9768169596519753      0.09588712292421533   0.10239965660389366
0.9769912823895306      0.09571892818386314   0.101976407443808
0.9771624215463953      0.09555503293175842   0.10140464893314781
0.9773220424001158      0.09540326954358072   0.10093530558696287
0.9774951491180531      0.09523989044424354   0.10067922669437258
0.9776567375328462      0.09508852479322008   0.10071728726395891
0.9778151423669488      0.09494121783901814   0.10091223717889544
0.9779870330652684      0.0947825735032349    0.10109088138665256
0.9781474054604437      0.09463569404884521   0.10104824027128277
0.9783212637198359      0.09447770531299947   0.10070273251740046
0.9784836036760838      0.09433135468395887   0.10018948084086914
0.9786427600516412      0.09418897645329388   0.09969006313840918
0.9788154022914155      0.09403577302272159   0.09935489213149541
0.9789765262280455      0.09389395874989546   0.09932119914045136
0.9791511360288926      0.09374155292456882   0.09951351625748915
0.9793225622490491      0.09359322436211467   0.09972640832140255
0.9794824701660614      0.09345602658443493   0.09975000137987997
0.9796558639472905      0.09330853338357295   0.09947968088795946
0.9798177394253754      0.09317204008640712   0.09900595218311617
0.9799764313227698      0.09303936234248122   0.09849973773392112
0.9801486090843811      0.09289668103366705   0.09811519107280195
0.9803092685428481      0.0927647428921412    0.09802627526122862
0.9804834138655321      0.092623041237485     0.09818883151765952
0.9806543756075256      0.09248526231691698   0.09842663947010925
0.9808138190463748      0.09235796021701922   0.09851591116329492
0.9809867483494409      0.09222119867194802   0.09832990503488534
0.9811481593493627      0.09209477899720929   0.09790906150142799
0.9813230562135016      0.09195914558877141   0.09735680851457701
0.9814947694969499      0.09182735091008912   0.09694814053258093
0.981654964477254       0.09170562525498697   0.09682544911245634
0.9818286453217749      0.09157499702062119   0.09696595790854073
0.9819908078631516      0.09145429955556796   0.09720504751285733
0.9821497868238378      0.09133716436434956   0.09734662834316618
0.9823222516487409      0.09121143312944004   0.09723705930512988
0.9824831981704997      0.09109536180409654   0.0968724735306744
0.9826576305564755      0.09097094654282611   0.0963356330448135
0.9828288793617608      0.09085020542141634   0.09589301852613018
0.9829886098639018      0.09073884339791455   0.09571706663122945
0.9831618262302597      0.09061945652170683   0.09581474769755711
0.9833235242934734      0.09050930642444659   0.0960564791186558
0.9834820387759965      0.09040254588159398   0.0962462153068616
0.9836540391227365      0.09028807471586923   0.09621809201274369
0.9838145211663323      0.09018256176271745   0.09592121838551947
0.983988489074145       0.09006964978944956   0.09541304951542576
0.9841509386788135      0.08996556013537364   0.0949617796118697
0.9843102047027915      0.08986476429963229   0.09471601195970242
0.9844829565909864      0.08975684185677188   0.09474485436177678
0.984644190176037       0.08965744169297407   0.0949705834269289
0.9848189096253046      0.08955117831105125   0.0952230616583999
0.9849904454938816      0.08944832410679492   0.09526498384247602
0.9851504630593143      0.08935369643438454   0.09503384476132312
0.985323966488964       0.08925253836409515   0.09456337870117919
0.9854859516154695      0.08915945685209295   0.09410152112745465
0.9856447531612844      0.08906948462018832   0.09381217567147887
0.9858170405713162      0.08897330946691204   0.09378850137872766
0.9859778096782038      0.08888491802761105   0.0939935601527198
0.9861520646493084      0.08879059298275427   0.09427494288850002
0.9863231360397224      0.0886994947185557    0.09438798708561186
0.9864826891269922      0.08861587703633239   0.09423134536261062
0.9866557280784789      0.08852666591655435   0.09381174261330194
0.9868172487268213      0.08844478163727836   0.09334844626578369
0.9869922552393806      0.08835757877095791   0.0929948969897102
0.9871640781712494      0.08827350336599872   0.09294053438737065
0.9873243827999739      0.0881964453037785    0.09313121257028548
0.9874981732929154      0.08811441574821936   0.09342918749831486
0.9876604454827127      0.08803924655640705   0.0935874483515627
0.9878195340918194      0.08796689091760565   0.09349896678558146
0.9879921085651431      0.08788990520729788   0.09313401245130974
0.9881531647353224      0.08781947359646872   0.0926811672700615
0.9883277067697188      0.08774469268386063   0.09229414553365706
0.9884990652234246      0.0876728472755292    0.09218767358278299
0.988658905373986       0.0876072390923754    0.09234609972366271
0.9888322313887645      0.0875376360396083    0.09265432615003708
0.9889940391003987      0.08747410946942012   0.09286665306025683
0.9891526632313424      0.08741319696780708   0.09285207187369161
0.989324773226503       0.08734863813563382   0.09255370267699245
0.9894853649185194      0.08728984238888077   0.09212199578173805
0.9896594424747527      0.08722768694477084   0.09170617062018466
0.9898303364502954      0.08716826963779517   0.09154242851409776
0.9899897121226938      0.08711429123349074   0.09165743758817364
0.9901625736593093      0.08705731483377596   0.09196345180950044
0.9903239168927804      0.08700561286561113   0.0922243913995752
0.9904987459904685      0.08695120514032167   0.09227905233922504
0.990670391507466       0.08689942802132138   0.09203270943044599
0.9908305187213193      0.08685259470049822   0.0916255513850698
0.9910041317993896      0.08680342433664355   0.09120130268758113
0.9911662265743156      0.08675903001841803   0.09100804589860763
0.9913251377685511      0.08671693085096704   0.09108156847169763
0.9914975348270035      0.08667285713367888   0.09137679738097891
0.9916584135823117      0.08663323263602873   0.09167281548219451
0.9918327782018367      0.08659193168270639   0.0918016158301311
0.9920039592406712      0.08655305440195452   0.09162971127314355
0.9921636219763614      0.08651828840725653   0.09126088055135477
0.9923367705762687      0.08648222193147535   0.09082719720092562
0.9924984008730316      0.08645009526498088   0.09058813121436685
0.9926568475891041      0.08642004936696576   0.09061134925678875
0.9928287801693935      0.08638907252055356   0.09088306276078477
0.9929891944465385      0.08636170160061377   0.09120578531974344
0.9931630945879006      0.08633370362663911   0.09140885675480133
0.9933254764261185      0.08630913683389597   0.09133232462344418
0.9934846746836457      0.08628653405415822   0.09102399671661122
0.9936573588053899      0.08626368063494211   0.0905910895990174
0.9938185246239898      0.086243918174424     0.09029742894205628
0.9939931763068066      0.08622421460216541   0.09025887727045802
0.9941646444089329      0.086206608007744     0.09050339733677398
0.994324594207915       0.0861917407217332    0.09084152536758448
0.994498029871114       0.08617732253668509   0.0911051759615885
0.9946599472311688      0.08616546519475357   0.09110254289448344
0.994818681010533       0.08615534794959982   0.0908511313150699
0.9949909006541141      0.0861460632500629    0.09043558098211307
0.995151601994551       0.08613899212267635   0.0901135937914754
0.9953257891992048      0.08613306886137016   0.0900200409889697
0.995496792823168       0.08612902075178178   0.09022545472016205
0.995656278143987       0.08612682733995058   0.09056875994445604
0.9958292493290231      0.08612617918966133   0.09088917627577998
0.9959907022109148      0.08612720358688924   0.09096599122416936
0.9961656409570234      0.08613009430552092   0.09075146313128239
0.9963373961224415      0.08613473923962524   0.09035789935860435
0.9964976329847154      0.08614069123587736   0.09002660819208473
0.9966713557112061      0.0861489141264919    0.0899030058828821
0.9968335601345527      0.08615825856482631   0.09006904041051055
0.9969925809772087      0.08616898623939137   0.09040956974659699
0.9971650876840816      0.08618238228522695   0.09077326014782641
0.9973260760878103      0.08619653910319043   0.09091962368785579
0.9975005503557559      0.08621369113741041   0.09077778734522943
0.997671841043011       0.08623236594041551   0.09041877316917928
0.9978316134271218      0.08625142881431556   0.09007453349003831
0.9980048716754496      0.08627389865089793   0.08990225898426604
0.9981666116206331      0.08629656736330743   0.09002262348295967
0.998325167985126       0.08632038050384223   0.09034927308018777
0.998497210213836       0.08634800503628516   0.09074918436355954
0.9986577341394016      0.08637546069664795   0.09096701837663426
0.9988317439291842      0.08640706032331029   0.0909085380079038
0.9989942354158224      0.08643829853159825   0.09061594651048577
0.9991535433217702      0.08647055110548317   0.09026501755889106
0.999326337091935       0.0865073591220286    0.09003130922993414
0.9994876125589555      0.0865434314722938    0.0900820031849989
0.9996623738901929      0.08658439760732119   0.09041155286180085
0.9998339516407397      0.08662652210790671   0.09083593746344165
0.9999999998333056      0.08666909059126325   0.09111962451147294
0.00008016381054133934  1.0000000000018532    1.0000000000026863
0.005912870906072291    1.0000000000019638    1.0000000000753562
0.006081034554831426    1.0000000000019653    1.0000000000774556
0.0072482197001210134   1.0000000000019733    1.0000000000920186
0.007415918870717115    1.000000000001974     1.0000000000941114
0.007576267099395813    1.0000000000019746    1.0000000000961138
0.00774176646983772     1.0000000000019753    1.0000000000981806
0.00807571655233055     1.0000000000019762    1.0000000001023475
0.008248334625608342    1.0000000000019764    1.0000000001044984
0.009589682809055193    1.0000000000019755    1.0000000001212246
0.009750318101025678    1.000000000001975     1.0000000001232239
0.010082874749029405    1.0000000000019735    1.0000000001273646
0.010250628743835775    1.0000000000019729    1.0000000001294571
0.010423533880405355    1.0000000000019718    1.0000000001316158
0.010589088075057533    1.0000000000019709    1.0000000001336822
0.010755626050246048    1.0000000000019698    1.000000000135758
0.0109231478059709      1.0000000000019684    1.0000000001378437
0.011083318619778352    1.0000000000019673    1.0000000001398348
0.011414946311456009    1.0000000000019644    1.0000000001439595
0.011582235828099346    1.0000000000019629    1.0000000001460445
0.01241260331322675     1.0000000000019535    1.000000000156391
0.012585795513088117    1.000000000001952     1.0000000001585412
0.01275163677103208     1.0000000000019516    1.000000000160602
0.012918461809512383    1.000000000001951     1.0000000001626788
0.013086270628529025    1.0000000000019507    1.0000000001647722
0.01324672850562826     1.00000000000195      1.000000000166775
0.013412337524490708    1.000000000001949     1.00000000016884
0.013578930323889495    1.0000000000019482    1.0000000001709128
0.013746506903824618    1.000000000001947     1.0000000001729932
0.01391923462552295     1.0000000000019458    1.0000000001751348
0.014084611405303881    1.0000000000019444    1.0000000001771863
0.673188509552846       0.052418878322188237  0.05331831807276348
0.6733565903291749      0.05241048266569162   0.05349824424246185
0.6735173201635865      0.052404117739664344  0.05342896070550152
0.6736832011397613      0.05239925515341129   0.0531408680834838
0.6738500658964723      0.05239611338415629   0.05278260215440488
0.6740179144337197      0.05239472446408234   0.05254882606112784
0.6741909141127302      0.0523951531755103    0.05257346609059526
0.6743565628498235      0.05239733439156276   0.05283265386467537
0.6745231953674531      0.052401277298048325  0.05317784312144855
0.674690811665619       0.05240701395007692   0.05341482459305863
0.6748510770218674      0.05241416051069307   0.053413411127145816
0.6750164935198792      0.052423241102054464  0.053178855665380526
0.8211989457290141      0.3237923609783807    0.33256730888670905
0.8213654777639687      0.32379889966099273   0.3325250134281344
0.8215329935794595      0.32380413405672087   0.3325053706008039
0.8217056605367137      0.3238081223352235    0.3325289520575909
0.8218709765520504      0.323810604784        0.3325889973105492
0.8220372763479235      0.3238117857237556    0.3326620675445161
0.8222045599243328      0.3238116441143088    0.33271687812789885
0.8223728272812787      0.32381015886168324   0.3327305667455745
0.8225462457799876      0.3238072215096044    0.3326999249874021
0.8227123133367791      0.3238030727895269    0.3326487457715606
0.822879364674107       0.3237975833233742    0.33260673849168626
0.8230473997919714      0.3237907322111231    0.3325992505493244
0.9949047908323235      0.08615048517325122   0.09064870717535183
0.9950712513243325      0.0861423352335952    0.09025434647959722
0.9952386955968779      0.08613580376253117   0.09002686508707773
0.9954112910111864      0.08613082567261153   0.09009058425404419
0.9955765354835775      0.08612773292472992   0.09039004684300138
0.995742763736505       0.08612627784772621   0.09074987383703767
0.995909975769969       0.08612649446063993   0.09096116358610373
0.996078171583969       0.08612841711784544   0.09089381706471517
0.9962515185397325      0.08613219268053937   0.09056424106915796
0.9964175145535785      0.08613751967355013   0.09017528985128623
0.9965844943479607      0.08614457219801709   0.0899260061614895
0.9967524579228795      0.08615338487282732   0.08995449403647228
0.9999169757370227      0.0866475845570756    0.09100549469235805
0.6731450495708707      0.052421338418622704  0.053240648805004495
0.6733160099229858      0.05241234683645518   0.05347648761853022
0.67347753565257        0.05240554170366835   0.05346879659398318
0.6736400451626905      0.05240035335707147   0.05323007484974091
0.6738097894951875      0.0523967110006246    0.05286459279333707
0.6739742665663808      0.052394914662227274  0.05258816917814853
0.674148062140564       0.05239487098130296   0.05254041592804964
0.6743165904534432      0.0523966494211515    0.052755368867018246
0.6744798515050185      0.052400082853012106  0.053090950482680595
0.6746503473789704      0.052405466419095215  0.0533768793216434
0.6748114086303916      0.05241224030748382   0.05343681075125262
0.6749734536623492      0.05242071173361859   0.05325727874880283
0.8211558729323825      0.32379045281933033   0.33257812519938873
0.8213252845431231      0.3237974434758151    0.3325342397678497
0.8214894288925597      0.3238029022379561    0.3325068521097383
0.8216628917449864      0.3238072673996575    0.3325187736001116
0.8218310873361092      0.32381012529819725   0.33257221882736676
0.821994015665928       0.3238116054315168    0.3326436117890892
0.8221641788181235      0.3238118002618207    0.33270685388650406
0.8223290747090151      0.32381067448090584   0.3327315092484062
0.8225032891028967      0.32380808200042055   0.33271085494515634
0.8226722362354743      0.3238041935156727    0.3326614154478341
0.822835916106748       0.32379913796945364   0.33261510586076315
0.8230068308003984      0.3237925082402113    0.3325971418330668
0.9948617359214282      0.08615286149377686   0.09075317732357695
0.9950310759892234      0.08614415116031905   0.09034123951606969
0.9951951487957145      0.08613734130057901   0.09006070775134993
0.9953685401051956      0.08613189252167588   0.09004625330459617
0.9955366641533727      0.08612832908894158   0.09030447627565973
0.995699520940246       0.08612649627907759   0.09066280859000092
0.995869612549496       0.086126287625372     0.09093309820417723
0.9960344368974419      0.08612775243615022   0.09093961534727978
0.996208579748378       0.08613108750870342   0.0906621014699547
0.99637745533801        0.08613608059694706   0.09026376999932764
0.9965410636663381      0.0861425741361817    0.08996729010235925
0.9967119068170428      0.08615109916403266   0.0899201863739505
0.9998754636888811      0.08663699791931632   0.09092639792746081
0.6733971707353641      0.05240872216654961   0.053504230130767834
0.6735571046746031      0.05240279347455571   0.05337582989666363
0.6737263571168322      0.05239827432777505   0.05304682287545549
0.6738903422977571      0.0523956180616727    0.05270824598887372
0.6740615623010586      0.052394654470243655  0.052527437661711175
0.6742337660848966      0.0523955512976709    0.05262258924743393
0.6743965352462038      0.05239812028926404   0.05291505900351855
0.6745665392298876      0.05240259048019105   0.05325701303124952
0.8214056709848143      0.32380027830706226   0.33251717021340904
0.8215765582663594      0.3238052749463297    0.33250684944065756
0.8217484293284409      0.32380888979634054   0.3325416809763755
0.8219108657679915      0.3238110083133842    0.3326065437788371
0.8220805370299189      0.3238118768374397    0.3326792210792378
0.8222449410305421      0.32381141040563105   0.3327244063839571
0.8224165798535422      0.3238095523551554    0.332726597749809
0.8225892024570784      0.3238062735728201    0.33268753467459633
0.822752390438084       0.323801876085595     0.3326366479179229
0.9951114266594417      0.08614061553325855   0.0901776596181121
0.9952822423980414      0.08613437958922937   0.09001319836338899
0.9954540419171772      0.08612986839052938   0.09015115120766898
0.9956164068137823      0.08612723232108638   0.09047917659061976
0.9957860065327641      0.08612617212491469   0.09082638196622775
0.9959503389904418      0.08612679975972554   0.09097229374038526
0.9961219062704962      0.08612919771372048   0.09083029528088676
0.996294457331087       0.08613340989839217   0.09046155291914305
0.996457573769147       0.08613905653225377   0.09009536247037517
\end{filecontents}


\begin{tikzpicture}[
    spy using outlines={circle, magnification=6, connect spies},
  ]
  \begin{axis}[
      xlabel = {Separation (\si{\nano\meter})},
      ylabel = {$\rho_{00}(t = \SI{10}{\pico\second})$},
    ]
    \addplot[no marks, smooth, red] table [x index = {0}, y index = {1}]{frames.dat};
    \addplot+[no marks, smooth, red, thick]  table [x index = {0}, y index = {2}]{frames.dat};

    \coordinate (spypoint) at (0.515,0.510983);
    \coordinate (magnifyglass) at (0.2,0.3);
  \end{axis}
  \spy [gray, size=2.5cm] on (spypoint) in node[] at (magnifyglass);
\end{tikzpicture}

%\begin{tikzpicture}[spy using outlines={circle, magnification=6, connect spies}]
  %\begin{axis}[no markers,grid=major,every axis plot post/.append style={thick}]

    %\addplot coordinates {(0, 0.0) (0, 0.9) (1, 0.9) (2, 1) (3, 0.9) (80,0)};
    %\addplot +[line join=round] coordinates {(0, 0.0) (0, 0.9) (2, 0.9) (3, 1) (4, 0.9) (80,0)};
    %\addplot +[line join=bevel] coordinates {(0, 0.0) (0, 0.9) (3, 0.9) (4, 1) (5, 0.9) (80,0)};
    %\addplot +[miter limit=5] coordinates {(0, 0.0) (0, 0.9) (4, 0.9) (5, 1) (6, 0.9) (80,0)};

    %\coordinate (spypoint) at (3,1);
    %\coordinate (magnifyglass) at (60,0.7);
  %\end{axis}
  %\spy [blue, size=2.5cm] on (spypoint) in node[fill=white] at (magnifyglass);

%\end{tikzpicture}

  \caption{\label{fig:frame comparison}
    Comparison of $\rho_{00}(t)$ using both fixed and rotating reference frames for interacting quantum dots.
    Both frames produce largely the same trajectory, however the inset spy reveals the fixed frame contains a minute oscilliatory term that the rotating frame does not.
  }
\end{figure}

%In single-dot system where one seeks to describe $\hat{\rho}(t)$ in response to an incident pulse, this computational bottleneck is overcome by using a rotating wave approximation.
%When one is aggregating response due to multiple interactions, the computational infrastructure is more complex.
\Cref{eq:liouville,eq:efield} sufficiently detail the physics of semiclassical radiation processes in quantum dot systems, however the nature of \cref{eq:liouville} imposes a significant computational bottleneck on na\"ive implementations.
For the systems under consideration here, $\omega_0 \sim \SI{1500}{\milli\eV}/\hbar$ and we must resolve our system on such a timescale.
By expanding the commutator in \cref{eq:liouville} for $\chi(t) \in \real$,
\begin{equation}
  \commutator{\hat{\mathcal{H}}}{\hat{\rho}} =
  \mqty(
    -2 i \chi \Im[\rho_{01}] & \chi\qty(1 - 2 \rho_{00}) - \omega_0 \rho_{01} \\
    \omega_0 \rho_{01}^* - \chi\qty(1 - 2\rho_{00}) & 2 i \chi \Im[\rho_{01}]
  ),
\end{equation}
it becomes apparent that the terms contributing to radiation physics, $\rho_{00}$ and $\rho_{11}$, contain only terms proportional to $\chi(t)$.
To have $\chi(t)$ approximately equal to $\omega_0$ would require an incident electric field roughly as strong as that used to bind a quantum dot together.
In such a regime the stability of the nanostructures becomes questionable, let alone the nature of energy transitions, so we may safely choose $\chi(t) \ll \omega_0$.


Introducing a time-dependent unitary transformation, $\hat{\rho}_\text{rot} = \hat{U}\hat{\rho}\hat{U}^\dagger$, where
\begin{equation}
  \hat{U} = \mqty(\dmat{1, e^{i \omega_L t}}),
  \label{eq:unitary transformation}
\end{equation}
we may write a rotating version of \cref{eq:liouville} as
\begin{equation}
  i \hbar \pdv{\hat{\rho}_\text{rot}}{t} = \commutator{\hat{U} \hat{\mathcal{H}} \hat{U}^\dagger - \hat{U} \, \partial_t \hat{U}^\dagger}{\hat{\rho}_\text{rot}} - \hat{\mathcal{D}}[\hat{\rho}_\text{rot}].
  \label{eq:rotating liouville}
\end{equation}
with $\hat{U} \hat{\mathcal{H}} \hat{U}^\dagger - \hat{U} \, \partial_t \hat{U}^\dagger \equiv \hat{\mathcal{H}}_\text{rot}$.
As $\chi(t) \sim \cos(\omega_L t)$, the matrix elements in \cref{eq:rotating liouville} will contain frequencies proportional to $\omega_0 \pm \omega_L$.
As $\omega_0 \approx \omega_L$, the high-frequency terms will approximately integrate to zero over any appreciable timescale and we may safely ignore them, instead considering only the envelope function of $\chi(t)$. This forms the basis of the so-called Rotating Wave Approximation (RWA)\cite{Allen1987}.

\section{Results}

\begin{figure}
  \centering
  \begin{filecontents}{echo.dat}
0.  0.00006118460941923795  -0.000022937795408427953
0.006133950247927229  0.0003282044434537532 -0.0002681399065804046
0.012267900495854458  0.0006319539847176276 -0.0005699639846424397
0.01840185074378169 0.0009675761329659023 -0.0009025630029862798
0.024535800991708916  0.0013461633366004856 -0.0012718823069151398
0.030669751239636146  0.0017745192612752141 -0.00168752584006078
0.03680370148756338 0.0022368629837918852 -0.002131936732534368
0.042937651735490606  0.0027618298306456806 -0.002639433077335105
0.04907160198341783 0.0033293981617384207 -0.0031898342044043176
0.055205552231345066  0.0039626734127889505 -0.0037836791700667626
0.06133950247927229 0.004665849175981548  -0.0044816249340415846
0.06747345272719953 0.005414477094039715  -0.005196620239359659
0.07360740297512676 0.0062743113624993815 -0.0059837741806931235
0.07974135322305398 0.0071669419941363935 -0.006941123283328351
0.08587530347098121 0.008174707546918224  -0.00785393330692725
0.09200925371890843 0.009295733790956643  -0.008929174644464645
0.09814320396683567 0.010431796827600485  -0.010191265100724597
0.10427715421476291 0.011786115238418576  -0.011346054808552731
0.11041110446269013 0.01320347144150243 -0.012836819257089266
0.11654505471061734 0.014653751842549082  -0.014439647165262876
0.12267900495854459 0.01647950093795158 -0.01599019522726918
0.12881295520647182 0.018227411930989732  -0.01793756058159242
0.13494690545439905 0.020088948551698408  -0.019933142267598947
0.14108085570232629 0.022507758356101602  -0.02201955562554118
0.14721480595025352 0.024633154060799214  -0.024501578293959204
0.15334875619818072 0.027177402133113992  -0.026972840875613305
0.15948270644610796 0.030162573719376296  -0.029712989600549404
0.1656166566940352  0.032806550268243966  -0.03276500724262505
0.17175060694196242 0.0361790109713085  -0.03583518897924085
0.17788455718988966 0.03977018394685434 -0.03936299077550867
0.18401850743781686 0.04322997857237249 -0.042917992660630726
0.19628640793367133 0.05170179935251594 -0.05104862575292261
0.2024203581815986  0.05616254893578861 -0.055337612864935666
0.20855430842952583 0.0613340640956221  -0.06027091433752427
0.214688258677453 0.06630477104183495 -0.06497308005366183
0.22082220892538026 0.07193595298773277 -0.07035357806786266
0.2269561591733075  0.0778373446109328  -0.07627164615790152
0.23309010942123468 0.083892887216346 -0.08157269202548549
0.23922405966916194 0.09080999700624469 -0.08832183285317315
0.24535800991708917 0.09731465484723986 -0.09523315620391047
0.2514919601650164  0.10476018619114141 -0.10121298144279103
0.25762591041294364 0.11222707857428359 -0.10988870851715099
0.2698938109087981  0.12905813035560976 -0.12454497287584322
0.27602776115672534 0.13688650318491324 -0.13525026436625462
0.28216171140465257 0.14648485131255035 -0.1435945974777046
0.2882956616525798  0.1567324352725454  -0.15273177416601424
0.29442961190050704 0.1650386579220788  -0.16468645141636198
0.30056356214843427 0.17695326975917133 -0.1741167833941215
0.30669751239636145 0.18822545137128366 -0.1854870926069769
0.3189654128922159  0.2119232148494281  -0.20957932410881328
0.32509936314014315 0.22370657903808647 -0.22286246818496955
0.3312333133880704  0.23430152759279366 -0.2362177437134504
0.34350121388392485 0.26328556944032344 -0.2646158455779388
0.3496351641318521  0.2769353853252722  -0.27800318337947383
0.3557691143797793  0.2946719467764216  -0.29337581126186
0.36803701487563373 0.32409656957029714 -0.3233422600841514
0.39257281586734266 0.3918140562537514  -0.3870190793261864
0.39922272083107024 0.41175791434877096 -0.40325339037696045
0.4058726257947978  0.42828874028524083 -0.4225786991706422
0.4125225307585254  0.44722058908921913 -0.43382848694863024
0.419172435722253 0.4582672600471715  -0.4615851792617994
0.4324722456497082  0.5004737784830585  -0.5035225317098451
0.4457720555771633  0.5446121356433427  -0.5405794723088617
0.45242196054089096 0.565212642888546 -0.5557316019983962
0.4590718655046185  0.5826574687296112  -0.5761164351615683
0.4723716754320737  0.6112820264208068  -0.6138784473150852
0.4790215803958012  0.6326345352299176  -0.6317615779120127
0.4856714853595288  0.6513753430360566  -0.6542859519318298
0.49232139032325645 0.6672117672786018  -0.6721898130793266
0.498971295286984 0.6928744400554564  -0.6908505746477571
0.5056212002507116  0.7105022231189145  -0.7014811569699453
0.5122711052144391  0.7265938287807325  -0.7195529144930918
0.5189210101781667  0.7409443501819963  -0.7227456933017956
0.5255709151418942  0.7487274295806179  -0.74932688753024
0.5322208201056218  0.7628722491523539  -0.7641226469168465
0.5388707250693494  0.7802238450646682  -0.782398112124769
0.545520630033077 0.7890731156893056  -0.7981859460131044
0.5521705349968046  0.813423963355451 -0.8136636166129103
0.5588204399605321  0.8256390968582774  -0.8184719347695228
0.5654703449242597  0.8389869059138242  -0.8315228532905165
0.5721202498879873  0.8470790455465875  -0.8303684455532524
0.5787701548517148  0.8517945232560005  -0.8492473152942416
0.5854200598154424  0.8562774397401409  -0.8602696309189427
0.5920699647791701  0.8714611126955396  -0.8723781681560615
0.5987198697428976  0.8724901910128348  -0.8856174069105514
0.6053697747066252  0.8934768487180137  -0.8953037816698246
0.6120196796703528  0.9006825506024697  -0.8962285108686747
0.6186695846340803  0.9113868446826968  -0.9033075509765903
0.6253194895978079  0.9132849345024845  -0.8991183758538261
0.6319693945615356  0.9154658666230301  -0.9089866740258288
0.6386192995252631  0.9117307295652717  -0.9178020591178502
0.6452692044889907  0.9239875366776553  -0.9233374807052742
0.6519191094527184  0.9185660990002156  -0.9349224887101475
0.6585690144164459  0.9359077630222744  -0.9393676705876591
0.6652189193801734  0.9397771348340267  -0.9382664345206271
0.6718688243439009  0.9489562237368399  -0.9399576863133088
0.6785187293076285  0.9462283625931865  -0.9348235576838306
0.6851686342713561  0.947040181588845 -0.9363188837840543
0.6918185392350839  0.9381625149568988  -0.9449246262064147
0.6984684441988114  0.9472804301188287  -0.9450268146097098
0.7051183491625389  0.9377259096131995  -0.9562791215638429
0.7117682541262664  0.9517810849012822  -0.956910592092213
0.718418159089994 0.9544653453032825  -0.9557319774671849
0.7250680640537215  0.9609124879886818  -0.9535742889072237
0.731717969017449 0.9578370865690741  -0.9488952915022052
0.7383678739811766  0.9582512467705939  -0.9433942292397197
0.7450177789449044  0.9465863285665396  -0.9530538329207947
0.7516676839086319  0.9531299036532793  -0.949488137414424
0.7583175888723594  0.9420582731099361  -0.9610173578611318
0.764967493836087 0.9526096756171683  -0.9595475786937873
0.7716173987998145  0.9558619767851899  -0.9596040773774389
0.7782673037635423  0.9600549076991689  -0.9548856804729006
0.7849172087272699  0.9583016935229868  -0.9513366581635574
0.7915671136909974  0.9587285058391746  -0.9401373208628478
0.7982170186547249  0.9460750167812678  -0.9514328371579904
0.8048669236184525  0.95023489389685  -0.9455930277991079
0.81151682858218  0.9409470844168882  -0.9562089224393077
0.8181667335459077  0.9461494021154363  -0.9551409824633628
0.8243759605723745  0.9499078719907136  -0.9583089448110336
0.8305851875988411  0.9455755185182226  -0.9544386840300854
0.8367944146253079  0.9517198785532798  -0.9545139522804147
0.8430036416517745  0.9502609471674877  -0.951294651700552
0.8492128686782412  0.9507202280050051  -0.9484470891353773
0.8554220957047078  0.9527862714080751  -0.9496883128295194
0.8616313227311745  0.9499725074434925  -0.9399100177799138
0.8678405497576412  0.9528123184132371  -0.9420579066216446
0.8740497767841079  0.9473993786713416  -0.9409982257912147
0.8802590038105745  0.9480544192944007  -0.9290066420082522
0.8864682308370412  0.9423621941626263  -0.9393598941407628
0.892677457863508 0.9404336136394346  -0.9329947454156028
0.8988866848899748  0.9413874686476577  -0.9277063152151721
0.9050959119164416  0.9302287733192035  -0.9377725272541825
0.9113051389429082  0.9345155230367496  -0.9295368108868883
0.9175143659693749  0.9327945558542592  -0.9324695848175886
0.9237235929958415  0.9195488523949769  -0.9368766469110322
0.9299328200223081  0.9321974543175057  -0.9331691272646376
0.9361420470487749  0.9253831672384322  -0.9378168396501015
0.9423512740752417  0.921512677549337 -0.9359694458174852
0.9485605011017085  0.9309699315042537  -0.9365461721694827
0.9547697281281752  0.9242110476638069  -0.9370261643734961
0.9609789551546418  0.9267897551377344  -0.9340988161901758
0.9671881821811085  0.9301693181021481  -0.935552412282463
0.9733974092075751  0.9275657711539115  -0.9301481514127511
0.9796066362340419  0.9319477614153634  -0.9307656372310968
0.9858158632605085  0.928647001370221 -0.9235470128691987
0.9920250902869753  0.9299543058966385  -0.9219949901718296
0.9982343173134421  0.9282825210036161  -0.9235157565566032
1.004443544339909 0.9259931137316605  -0.9105127447503316
1.0106527713663755  0.9252606060454134  -0.9162118482546928
1.0168619983928422  0.9205219685676753  -0.9155439673329597
1.0230712254193088  0.9217575385901063  -0.9001499738030764
1.0292804524457755  0.9123265808765535  -0.9144907638064022
1.0354896794722421  0.9119821854360607  -0.9089024470306322
1.041698906498709 0.9139425990476587  -0.9035204048120927
1.0479081335251759  0.8992562055883784  -0.9138451094628307
1.0541173605516425  0.9076607195254032  -0.9084603008802175
1.0603265875781092  0.9064613383871871  -0.9093336774189483
1.0665358146045758  0.892282228241529 -0.9129918643357187
1.0727450416310425  0.9065100200747686  -0.9108431534009923
1.078954268657509 0.9005654744932041  -0.9145618020171961
1.0851634956839757  0.8966938052300377  -0.9104143156715083
1.0913727227104426  0.9062084428484565  -0.9118025649323104
1.0975819497369095  0.9017636728633612  -0.9088952510137789
1.1037911767633761  0.9028261287247398  -0.9066270352058019
1.1100004037898428  0.9048027756804699  -0.9045371011843254
1.1162096308163094  0.9035877462901168  -0.9000830455020781
1.122418857842776 0.9060431984663077  -0.9013357870443632
1.1286280848692427  0.9015584920990833  -0.8909686367990762
1.1348373118957094  0.903382732680587 -0.8908504931204615
1.1410465389221762  0.8985608837817857  -0.8940834758463687
1.147255765948643 0.8970970729922334  -0.878416780919638
1.1534649929751097  0.8931840046000025  -0.8875354015761275
1.1596742200015764  0.8890992930827235  -0.8874285044557308
1.165883447028043 0.8910641684386646  -0.8738979981700438
1.1720926740545097  0.8799484361359547  -0.887099555371249
1.1783019010809763  0.8801880957086448  -0.8824895013494293
1.184511128107443 0.8841351937884427  -0.8788986230679849
1.1907203551339098  0.8684187257116693  -0.8873106197192736
1.1969295821603767  0.8786862730212576  -0.8834068794450765
1.2031388091868433  0.8780602375122804  -0.8845523904644239
1.20934803621331  0.8674112925503352  -0.885220923896001
1.2155572632397766  0.8787566364242821  -0.8840672631615287
1.221644689880003 0.8741844059256091  -0.8858006210055902
1.2277321165202293  0.8697966956642078  -0.881232623936706
1.2338195431604555  0.8773282952631672  -0.8819258862192672
1.2399069698006817  0.8728401789040096  -0.8798471598929667
1.245994396440908 0.8711341974043397  -0.876778899894911
1.2520818230811341  0.8757146747402953  -0.8786868331902957
1.2581692497213603  0.8714977252495768  -0.8729422774854996
1.2642566763615866  0.8720635904043139  -0.8719272834828561
1.2703441030018128  0.8721077001904208  -0.8707138774392035
1.276431529642039 0.8698525143457017  -0.8654926139772691
1.2886063829224914  0.8679538667114408  -0.8622951653867702
1.2946938095627176  0.8676378598566469  -0.8579139984697367
1.3007812362029438  0.8671553082986795  -0.8614468196159971
1.3068686628431703  0.8633048063880338  -0.8539000878903297
1.3129560894833963  0.8646241450262685  -0.8506308386176613
1.319043516123623 0.8602454848337271  -0.8562430381816953
1.325130942763849 0.8582435378679285  -0.8459576300176772
1.331218369404075 0.8586329197392251  -0.846081785523862
1.3373057960443016  0.852654737196712 -0.8512960494130416
1.3433932226845275  0.8528835335399154  -0.8388571034086252
1.3494806493247538  0.850574642470301 -0.8438442724527951
1.3555680759649802  0.8448029036224604  -0.8466862279085017
1.3616555026052064  0.8474709535894587  -0.8345359699183393
1.3677429292454328  0.8423758760606987  -0.8422475130555667
1.373830355885659 0.837117691720525 -0.8424608525975555
1.379917782525885 0.842345397170226 -0.8341636504504565
1.3860052091661117  0.8344439447857177  -0.8410261229822956
1.3920926358063377  0.830509036119156 -0.838810177497893
1.3981800624465637  0.8374698238836427  -0.8343284387707
1.4042674890867903  0.8271473701276065  -0.839888518258701
1.4103549157270163  0.8280027240198082  -0.8365563766327847
1.6051525682142558  0.7812555150568724  -0.7891249164417888
2.0277689749999572  0.6780629676628562  -0.6840640407995631
2.422181993800963 0.5794178935279127  -0.5700643754083785
2.428860651935529 0.5737638895783916  -0.5733126272057647
2.442217968204661 0.5663898041221831  -0.5706389655655224
2.455575284473793 0.5669205086456733  -0.572021525181843
2.4689326007429253  0.5673817266681503  -0.5656078682917968
2.52904052395402  0.5508774821114724  -0.5415324642505609
2.8496161144131906  0.4678742363014472  -0.4590389145134331
2.8561729721615166  0.463522978842951 -0.46251121849933213
2.869286687658167 0.4595855652389949  -0.45757380153456895
2.8824004031548185  0.4571051614499329  -0.45983220783117756
2.89551411865147  0.45546331961908765 -0.4567116664987003
2.9086278341481204  0.45487691987216056 -0.44897868461200013
2.921741549644772 0.44904600437247194 -0.43949844176990405
2.9348552651414233  0.4424975681998261  -0.44373691304655183
2.9479689806380738  0.4367560037663994  -0.4438807732344057
2.9610826961347256  0.4397057673689818  -0.4391887760319022
2.974196411631376 0.4371631924836927  -0.42913018233622485
2.987310127128028 0.4300636913086958  -0.4296820495209074
3.000423842624679 0.4195615211845428  -0.43072515782619575
3.013537558121331 0.42409072687950417 -0.42916257317666057
3.0266512736179814  0.4246029564418144  -0.42236600697558
3.0397649891146328  0.42059164464205323 -0.41888477491191367
3.052878704611284 0.41151824006293247 -0.4163157291748976
3.2692550103060283  0.3667879805656951  -0.36921603268994196
3.275371190117093 0.3675833061477652  -0.36880626574136616
3.2814873699281577  0.3653713133162675  -0.3657913782575847
3.2876035497392224  0.36522676373057655 -0.36489609305743315
3.293719729550287 0.36375895923514534 -0.3625319739719764
3.3059520891724166  0.36233970340874644 -0.3603805059251599
3.3181844487945455  0.359484589279748 -0.3548883261490146
3.3243006286056103  0.35739746813723683 -0.3556485551273174
3.3304168084166745  0.3556462806435859  -0.3498127366959682
3.3365329882277392  0.354956082657388 -0.3504928173292734
3.3426491680388044  0.3521185478303614  -0.3509869523678131
3.348765347849869 0.35141497538711153 -0.344029454192308
3.354881527660934 0.3488138624786094  -0.3473307819323029
3.3609977074719986  0.3467596330820546  -0.3464879177633248
3.3671138872830633  0.34696416962062876 -0.34030808329923234
3.373230067094128 0.34255925528845277 -0.344431408653206
3.3854624267162574  0.3423660090082017  -0.3385146940984223
3.3976947863383873  0.3375854181580509  -0.33898312704080885
3.4099271459605163  0.33099244049118265 -0.3388503944439796
3.4221595055826457  0.3334962642757079  -0.3356815148147999
3.4343918652047747  0.33174004451423095 -0.33385669636179544
3.446624224826904 0.3261784975790166  -0.33186356253162774
3.4588565844490335  0.32644092840066774 -0.3301947273939954
3.4649727642600987  0.3247904435607693  -0.32807387540967614
3.477205123882228 0.32394783712402325 -0.32544506227263026
3.4894374835043576  0.3229127309285816  -0.3228254169195234
3.501669843126487 0.3216649465957813  -0.32021637471452913
3.513902202748616 0.318747519773694 -0.3155727546981969
3.526134562370745 0.31565550021566463 -0.311217156832802
3.5444831018039396  0.31188386359201403 -0.30572063590277915
3.556715461426069 0.30817466619222844 -0.3074440262950825
3.5689478210481984  0.304590663786169 -0.3050429472612077
3.581180180670328 0.3038239882369119  -0.2990429242631515
3.5934125402924577  0.2991617629231734  -0.2998371860750635
3.605644899914587 0.29369344740992903 -0.2998027523141184
3.6178772595367166  0.2955670725859769  -0.29633040828432944
3.6301096191588464  0.2935286140463573  -0.2951642423462888
3.642341978780976 0.2877543148153931  -0.2938160040602956
3.654574338403105 0.2883949205522345  -0.2926456499380957
3.6606905182141696  0.2864061817094201  -0.29051426215167725
3.6673226527410345  0.28882499392000227 -0.2898276253667259
3.6739547872679 0.288024024117754 -0.286373637193768
3.680586921794765 0.28696497232770374 -0.2855454095084677
3.68721905632163  0.2852618511883515  -0.2787243790849417
3.6938511908484952  0.2808081050353604  -0.28243043390568334
3.70048332537536  0.28002876332113574 -0.28051896975119767
3.707115459902225 0.27920166225010534 -0.28075177106730653
3.7137475944290905  0.27560066152992063 -0.2809211608052252
3.7668046706440106  0.2654990501715301  -0.2715096348041698
3.7800689396977405  0.2678037055045858  -0.2694248869889397
3.793333208751471 0.267047486462723 -0.2633692992612485
3.806597477805201 0.2628583751340031  -0.26205628965685074
3.819861746858931 0.2560906573369722  -0.26229778209156923
3.8331260159126606  0.2580904085967617  -0.2611028379727979
3.846390284966391 0.2581993148529935  -0.25614659327231687
3.8596545540201213  0.2549744100819103  -0.2537798988940226
3.872918823073851 0.2485450162961166  -0.25322975154814237
3.879550957600716 0.2510693264301244  -0.25201315968736543
3.886183092127581 0.24875013264217874 -0.25282876120700437
3.8928152266544465  0.24941885594856533 -0.25108601183357543
3.8994473611813114  0.24953073341297355 -0.24907035378156958
3.9060794957081764  0.24950134873706373 -0.24697163868971533
3.9193437647619067  0.2463034133944274  -0.2415395600758105
3.9259758992887717  0.24158747958051802 -0.2445292975502362
3.9790329755036926  0.2349549282684248  -0.2362560961539158
3.992297244557423 0.2326636302601544  -0.2352055578516913
4.005561513611153 0.23282594991796216 -0.23488991403917375
4.018825782664883 0.23224882677066705 -0.2313873165573005
4.032090051718614 0.2285333311241856  -0.2284387925283663
4.085147127933533 0.2222261683401208  -0.2210881314979242
4.091338584523138 0.22096483190811392 -0.21902368444696357
4.097530041112742 0.22062605200395474 -0.21700181421277076
4.103721497702346 0.21795634605762224 -0.2186870925725673
4.10991295429195  0.21775488737633486 -0.2155085335837295
4.1161044108815545  0.21673069067564227 -0.21598427907077838
4.122295867471158 0.21391769464996724 -0.21638923841065527
4.128487324060763 0.2152122616825677  -0.21427684208330816
4.134678780650367 0.212875249149263 -0.21515858863774415
4.140870237239971 0.21147400559809654 -0.21436145246796606
4.147061693829576 0.21281162285159633 -0.213463325804531
4.15325315041918  0.20957333978161696 -0.21401863938098092
4.159444607008783 0.21045570058714735 -0.21257770845150942
4.165636063598388 0.21051469228340164 -0.212537624322539
4.171827520187992 0.20853893901404102 -0.21139441043961668
4.178018976777596 0.2095736609783041  -0.2107162071156328
4.1842104333672 0.20861054433824563 -0.20990810927932704
4.190401889956805 0.20784528683490908 -0.20856372146040394
4.196593346546409 0.2082730669554529  -0.20835653879943355
4.202784803136013 0.2069687653882569  -0.20615812899321304
4.208976259725617 0.20701538469088435 -0.20572048520149513
4.2151677163152215  0.20577308945809808 -0.20459678782346535
4.221359172904825 0.20523597515391687 -0.2024629256238603
4.227550629494429 0.20432314115609757 -0.20313726824466077
4.233742086084034 0.20310996921113997 -0.200752013119951
4.239933542673638 0.20282256078549124 -0.19965155636333895
4.246124999263242 0.20076273947946388 -0.20068638380197648
4.252316455852847 0.20043950892826823 -0.1972511989733322
4.258507912442451 0.19906774393460244 -0.19845614416907031
4.264699369032055 0.19727853385522176 -0.19833485558138092
4.27089082562166  0.19792316041064517 -0.19594558814176716
4.277082282211263 0.19525936324731313 -0.19745827505245908
4.283273738800868 0.19477034341077748 -0.19640558803275796
4.289465195390472 0.19547017522880397 -0.19546980324475524
4.295656651980075 0.1918478122322802  -0.19635146532744738
4.30184810856968  0.1935785565369208  -0.19508658538279877
4.308039565159285 0.1931249059356089  -0.1950349573597584
4.314231021748888 0.19088550499013565 -0.19432616743057052
4.320422478338493 0.19253922766286338 -0.19376011225605314
4.326613934928098 0.1913527712602415  -0.19335808698108217
4.332805391517701 0.19055436136418905 -0.1921106738868524
4.338996848107306 0.19137287098261552 -0.19209922995481224
4.34518830469691  0.19019472313455238 -0.19034436347698627
4.357571217876118 0.1894377390153388  -0.18852409989906593
4.363762674465723 0.18901611691607917 -0.18726467894141485
4.3699541310553265  0.18852907808996328 -0.1874862252022394
4.37614558764493  0.18740644493896397 -0.18476273806379767
4.382337044234535 0.18730662124290404 -0.1846159802675733
4.3885285008241395  0.18571566664932437 -0.18503741191644607
4.394719957413743 0.18533806254145166 -0.18130741755802265
4.400911414003347 0.1837575822971424  -0.18333022500451002
4.407102870592952 0.18286472804298554 -0.1826844235157156
4.413294327182556 0.18305563822182244 -0.1799430252828939
4.41948578377216  0.18007584361927995 -0.18224748822364673
4.425677240361764 0.18053591897575128 -0.18082691080427973
4.431868696951368 0.18067070480151384 -0.1798951769264286
4.438060153540973 0.17689995177057796 -0.18118875063036965
4.444251610130577 0.17934286407438002 -0.18006789227942263
4.45044306672018  0.17842852057800804 -0.18012484568615686
4.456634523309785 0.17617691731623472 -0.17984563239189108
4.46282597989939  0.1783574750531105  -0.17944264807431715
4.469017436488993 0.17688844314198157 -0.17948398650799202
4.475208893078597 0.17644449762694048 -0.17845689724445574
4.481400349668202 0.1774560093823604  -0.17865764538169934
4.487470005871565 0.17630349709987633 -0.17758368536089347
4.493539662074929 0.17632699530170765 -0.17714349766833296
4.499609318278293 0.17634270732620624 -0.1768926598292405
4.505678974481657 0.17563174163588044 -0.17597614287386876
4.51174863068502  0.17600551004147685 -0.17598591658051704
4.517818286888383 0.1754018652827595  -0.17504554957972063
4.523887943091748 0.17516627310182767 -0.17456968435099252
4.529957599295111 0.17554923014101376 -0.17493232895698083
4.536027255498475 0.17470773921209123 -0.17340306221075152
4.542096911701838 0.17491873021858134 -0.17345194838485112
4.548166567905202 0.17476018615035285 -0.1739147269202328
4.554236224108566 0.1743188213950777  -0.1720814310824918
4.56030588031193  0.1749099527403137  -0.1727181254637372
4.566375536515293 0.1742490059436478  -0.1732211324406214
4.572445192718657 0.17430233726538477 -0.17120726397844563
4.578514848922021 0.1745202821148817  -0.17275789643665088
4.584584505125385 0.17413920729610452 -0.17298687714264602
4.590654161328748 0.17473808216941175 -0.1709223889455345
4.596723817532112 0.17438573874696747 -0.1734452945331593
4.602793473735477 0.1745632543971352  -0.17337334901292853
4.60886312993884  0.1756254515313565  -0.1716886381803521
4.614932786142203 0.17481037522965026 -0.17484720621235658
4.621002442345568 0.17566706045183444 -0.17456618973319402
4.627072098548931 0.17702972292377067 -0.17396034958073886
4.633141754752295 0.1759689228603415  -0.17712185035213185
4.639211410955658 0.17761605336226385 -0.1767703376489631
4.645281067159022 0.17935808879174103 -0.17730145153804594
4.651350723362386 0.17805448142748972 -0.18044932215791135
4.65742037956575  0.18098871932595828 -0.18066467979028164
4.663490035769113 0.18284785203478868 -0.18186806612818
4.669559691972477 0.18128396459318338 -0.18501964818585756
4.675629348175841 0.1857515938030552  -0.18595180598040967
4.681699004379204 0.187748173376513 -0.18789554960829277
4.687768660582568 0.1868345737322986  -0.19092915381972025
4.693838316785932 0.1920705886619888  -0.19262023170317402
4.699907972989295 0.1943026992672301  -0.19551015402994837
4.705977629192659 0.19495863179287795 -0.1981616685680664
4.7120472853960225  0.20016076750016196 -0.20076296825835915
4.718116941599386 0.20272461180525214 -0.2045525742209196
4.72418659780275  0.20496145770027382 -0.20689892761037487
4.730256254006114 0.21019875822427395 -0.21044708596261502
4.7363259102094775  0.21338569168914195 -0.2141350870224814
4.748465222616205 0.2222828377256796  -0.2217133855011154
4.754534878819569 0.22589461026546045 -0.22499766947873479
4.7606045350229325  0.23052028366230926 -0.22905510335295795
4.77274384742966  0.24010096237247075 -0.23723705818220253
4.778813503633024 0.24580588651487337 -0.24244646034696954
4.7848831598363875  0.2508834446410331  -0.24848966588600738
4.790952816039751 0.25579868063487093 -0.25078863702037185
4.797022472243115 0.2623752292295987  -0.2572738536731174
4.8091617846498425  0.2726773887629891  -0.2664405549142444
4.82130109705657  0.28355298075470703 -0.2817799909388836
4.827370753259934 0.2903136019067998  -0.2839220874356467
4.8334404094632974  0.2960715953954747  -0.2925980364014355
4.839510065666661 0.30050510669808056 -0.3002654193521897
4.845579721870025 0.3082643478018373  -0.3030017680605784
4.8577190342767524  0.3171669619159641  -0.3191238083131493
4.869858346683479 0.33068521682935603 -0.3308092763898861
4.8764439576026435  0.3345082947942341  -0.3384374805073201
4.883029568521808 0.3446664929219909  -0.34571652931069213
4.8896151794409715  0.3494351608773626  -0.351005873430806
4.896200790360136 0.35809351116333066 -0.35641862458251417
4.9027864012793 0.36205475887693805 -0.3568792237572411
4.909372012198464 0.3666031247891982  -0.36220449522346476
4.915957623117628 0.3679653707522461  -0.3655055299093021
4.922543234036792 0.37146806251017683 -0.3662125030232219
4.929128844955956 0.3669725192709191  -0.37430038237926944
4.935714455875121 0.37488342455995666 -0.37613713840511664
4.942300066794284 0.37466012437421364 -0.37982125400448086
4.948885677713449 0.37657992547203706 -0.3790053903155881
4.955471288632612 0.37841676127313334 -0.3773291957830561
4.962056899551777 0.3782309217176407  -0.37423318376925807
4.968642510470941 0.37462543620177335 -0.37314308611634084
4.975228121390105 0.3715440104584407  -0.3628977746603125
4.981813732309269 0.3623881018007179  -0.36675171978440346
4.9883993432284335  0.3597322633493045  -0.36131880418322126
4.994984954147597 0.35410712178499076 -0.35803232685804454
5.0015705650667615  0.3456643683734175  -0.352693516233001
5.008156175985925 0.3434144042552986  -0.34696914358680025
5.014741786905089 0.33538254951630103 -0.336966917940676
5.021327397824254 0.3284385491086354  -0.32858164700475995
5.027913008743417 0.3182915527813109  -0.31464072781095676
5.034498619662582 0.3070549809690982  -0.30759435279760555
5.041084230581745 0.2955886658297949  -0.2952587906397715
5.04766984150091  0.2839882002812356  -0.28546785132737296
5.054255452420074 0.2697824405993605  -0.275857221664844
5.067426674258402 0.24671289469271002 -0.2533006302599659
5.080597896096729 0.22467822252623312 -0.22750387787121557
5.087183507015894 0.21337028028683075 -0.21482365568922393
5.093769117935057 0.19987846675021959 -0.19969475153528557
5.106940339773385 0.17388870924600078 -0.1736631499861849
5.11352595069255  0.16040473129185723 -0.1623242690358352
5.120111561611714 0.148097102518639 -0.14948878657938572
5.133282783450042 0.12396890580897837 -0.12641748693437932
5.139868394369206 0.11216559239242169 -0.11544786039157794
5.14645400528837  0.10198358583922758 -0.10439721091793625
5.1530396162075345  0.09042669316695161 -0.09434870740686402
5.159625227126698 0.08203274372692583 -0.08397760224947456
5.1662108380458625  0.07270642989917601 -0.07494770456024197
5.172796448965026 0.06462318903924535 -0.06639508978296346
5.17938205988419  0.058046420940750076  -0.05843508615732328
5.185967670803355 0.052389321778423044  -0.05241857253595831
5.192553281722518 0.04748908015076101 -0.04830095190108717
5.199138892641683 0.0436588595812999  -0.04515667255537642
5.205724503560847 0.04291583964242059 -0.042795439025686555
5.212310114480012 0.04369973760572152 -0.04292820357830898
5.218895725399176 0.044962993507088475  -0.04487416180564207
5.22548133631834  0.04701525982122592 -0.04741089748585024
5.232066947237503 0.05004692035661448 -0.050378210789911046
5.238652558156668 0.05418093457323227 -0.05351163061737026
5.245238169075833 0.05861230680494307 -0.05721964412709193
5.251823779994996 0.062420765005062336  -0.06169654300679271
5.2649950018333245  0.07048083194630282 -0.06951290170691686
5.271580612752489 0.0741531128502964  -0.0733895675475717
5.2781662236716524  0.07785396052600696 -0.07713167473398037
5.29133744550998  0.08458879394789912 -0.08457701093953744
5.2974823784918845  0.08773099803338949 -0.08738976604094514
5.303627311473788 0.08965899864291811 -0.09063183102887568
5.30977224445569  0.09352785027498556 -0.09297781294385224
5.315917177437593 0.09476164691776788 -0.09613371742716506
5.3220621104194965  0.09817188779509486 -0.09822358793818077
5.3282070434014 0.1004083251820644  -0.10052772408384854
5.340496909365205 0.10494199748013484 -0.10486549420852512
5.3466418423471085  0.10571104772671876 -0.10735729354508386
5.352786775329011 0.10810071572971633 -0.10883602614856
5.358931708310914 0.11024320685345426 -0.11074570704078342
5.365076641292818 0.11069429821701866 -0.11246564712475146
5.371221574274722 0.11350148378342495 -0.11392145980232024
5.3773665072566255  0.11450331600835442 -0.11577133315097035
5.383511440238528 0.11538179464732123 -0.1167101110006659
5.389656373220431 0.11786235223309235 -0.11831186914842869
5.395801306202334 0.11852214018343034 -0.11933009932879615
5.4019462391842366  0.11991785818499874 -0.12036364673779174
5.414236105148044 0.12217223678554004 -0.12222327857387462
5.420381038129948 0.12379462504633845 -0.12351137118257699
5.426525971111851 0.1243860560443405  -0.12380397984027257
5.4326709040937535  0.12548428222623445 -0.12468913614688061
5.438815837075657 0.12648016546084204 -0.12581090761454164
5.44496077005756  0.12709246976330235 -0.12560552040298492
5.451105703039462 0.1283691395880973  -0.12689282310785416
5.4572506360213655  0.1285758552102317  -0.12742961584700524
5.46339556900327  0.12951748406325403 -0.12721989125332422
5.469540501985174 0.12994838695638805 -0.12902993667369397
5.475685434967077 0.13042068793499711 -0.12865997879378105
5.481830367948979 0.1313813500388333  -0.12910621842037495
5.4879753009308825  0.13123009882816764 -0.1308873279659413
5.494120233912786 0.1321197876853288  -0.12973823978827703
5.500265166894688 0.13238728387489299 -0.1315155990196195
5.506410099876591 0.13239287589606025 -0.1325358160580421
5.512555032858495 0.13371485191560312 -0.1315838884761912
5.5186999658403995  0.13309282851941764 -0.13384583642231415
5.524844898822303 0.1335955369921897  -0.13407506139712524
5.530989831804205 0.1351455568557914  -0.134037294343108
5.537134764786108 0.13371231457234506 -0.13603453971511262
5.543279697768011 0.13556987285975733 -0.13608711888699496
5.549424630749914 0.13645230157613145 -0.13664175139986043
5.555569563731817 0.13478572704978697 -0.13797916486799702
5.561714496713721 0.13761001299709982 -0.13821076520829922
5.567859429695625 0.1377212919219477  -0.13924087246068187
5.574004362677528 0.1371032250422352  -0.13962026655038617
5.580149295659431 0.13960907181170007 -0.14031555544854932
5.586294228641334 0.13941411867563525 -0.14109234888391087
5.5985840946051395  0.14152108397354762 -0.14227551718682085
5.610873960568947 0.14258370983680047 -0.14281082305542359
5.623163826532754 0.1437645008830973  -0.14330446048197984
5.63545369249656  0.14496770072319184 -0.14365787975704786
5.647743558460366 0.14624736878898476 -0.1453299276460476
5.660033424424172 0.14741096425010705 -0.14580900404266012
5.672323290387977 0.148341714351118 -0.1450613777103675
5.6846131563517845  0.14860599937336877 -0.1471825149475658
5.6912740440494884  0.14892550415288902 -0.14780980461005555
5.6979349317471915  0.14740601143986126 -0.14967463545172205
5.7045958194448945  0.15008449598387266 -0.150515428629943
5.7112567071425975  0.14976471106283665 -0.15176258215708702
5.7179175948403005  0.15212517209595522 -0.15235920083637
5.724578482538005 0.1527443310105578  -0.15172441648841714
5.731239370235709 0.15389732892455027 -0.15235852712015047
5.737900257933412 0.15363210065338576 -0.15183692606542556
5.791187359515042 0.1587581042947767  -0.15671715177677775
5.897761562678297 0.16929586891000103 -0.16714705711751013
6.110909969004811 0.19134874125451465 -0.19072056912924026
6.117449056316274 0.19273038682402122 -0.19069367830757972
6.130527230939199 0.193674759157891 -0.18899308061992698
6.143605405562127 0.19336301391494706 -0.1926331121960422
6.156683580185052 0.19309085009441818 -0.19613988856447667
6.169761754807977 0.19670417646510716 -0.19776864348812545
6.182839929430905 0.19942082061591393 -0.19706145055200136
6.195918104053831 0.20003451131063207 -0.19846709240608637
6.208996278676756 0.1985532861642897  -0.20121646885389535
6.222074453299684 0.20074710254320127 -0.2040719406643525
6.235152627922609 0.20465479664882244 -0.20553031360920773
6.248230802545534 0.2070358791278721  -0.20585493424399395
6.261308977168462 0.2070709669288817  -0.20652003511159703
6.274387151791388 0.20782117073834167 -0.20748317663703572
6.287465326414313 0.20986992183805814 -0.21068939867805087
6.300543501037241 0.21219840993698222 -0.21269175679898641
6.313621675660166 0.21489559258987248 -0.21253058971885547
6.326699850283092 0.21556465702430117 -0.21120465240601274
6.339778024906018 0.21550595566026148 -0.2143370469008006
6.352856199528945 0.21560778015388868 -0.21817099013091593
6.365934374151871 0.21886831032925563 -0.2197606943084973
6.379012548774796 0.22172820770127805 -0.21882844534660065
6.3920907233977236  0.22216976301655925 -0.2199299052741786
6.405168898020649 0.2204510641447499  -0.22337181359132804
6.418247072643575 0.2225672613365592  -0.22656169020826836
6.431325247266502 0.2270134628780085  -0.22764227658078753
6.444403421889428 0.22938475005971987 -0.22789180051512797
6.457481596512353 0.2291213648020875  -0.22901107129082385
6.470559771135281 0.22971246265355835 -0.23048246150479196
6.483637945758206 0.23239015571404903 -0.23362316624141902
6.496716120381132 0.23529553065604727 -0.2354934257559815
6.509794295004059 0.23780055868617503 -0.23509054428575724
6.522872469626985 0.2379905203185351  -0.23408574814365504
6.529411556938447 0.2357695032403468  -0.23754970601420555
6.5416083756868515  0.23832602446580034 -0.23824635772222472
6.553805194435256 0.240320128920272 -0.23930974411229636
6.5660020131836605  0.23899674091859127 -0.24235162076487426
6.578198831932064 0.24093504405134825 -0.24480378046836257
6.590395650680469 0.24478536845761495 -0.24586046536734463
6.602592469428873 0.2452732333644372  -0.2474657610844016
6.614789288177277 0.24633616862366595 -0.24931582968250646
6.626986106925682 0.24956693830459953 -0.25124054102585053
6.639182925674088 0.25192998424147434 -0.2522553994268497
6.651379744422492 0.2529730985748421  -0.25264934032774294
6.6635765631708965  0.2547246168989947  -0.2531359475729192
6.681871791293503 0.25721346668650885 -0.25391061163202167
6.694068610041907 0.25845978155652816 -0.25646271751626043
6.706265428790311 0.25984650282256977 -0.2579727140834133
6.7184622475387155  0.2616596772516895  -0.25606702535009007
6.73065906628712  0.2615958405734213  -0.258843532052052
6.7428558850355245  0.2613745479622902  -0.26241208444893044
6.755052703783928 0.2639489587130843  -0.26224880789814603
6.767249522532333 0.2652141724233648  -0.2634641977977813
6.779446341280737 0.2631780957657011  -0.2671514542677359
6.791643160029142 0.26625482068667045 -0.26912759749019477
6.803839978777545 0.26964638639178495 -0.2701919267103012
6.81603679752595  0.2691701921177487  -0.2723178535803068
6.828233616274356 0.2705275166087 -0.2746846524710887
6.8404304350227605  0.27440313613712813 -0.27657537317741576
6.852627253771165 0.276130810346286 -0.277214288828215
6.8648240725195695  0.2772457540030484  -0.27797932514221324
6.877020891267973 0.2795231489349254  -0.27886499807879583
6.889217710016378 0.2820258423187349  -0.28095056897105336
6.901414528764782 0.28370157630040654 -0.2808236784111412
6.913611347513186 0.28440233334734266 -0.2799552785717979
6.919709756887388 0.28557213531495473 -0.2825837105618217
6.9329384850673925  0.2851597460695765  -0.2818532893242266
6.946167213247398 0.28600529891766846 -0.28632653523269935
6.959395941427404 0.2880882274019701  -0.28952028542761216
6.972624669607408 0.29187525113313806 -0.28894732440891535
6.985853397787413 0.2923083501251324  -0.2864394068876558
6.999082125967418 0.29177432757009825 -0.2910409821940546
7.012310854147424 0.29234177344090534 -0.2953168983047387
7.025539582327429 0.29685688896818946 -0.295720726440276
7.038768310507433 0.29876463703928663 -0.2924436276588678
7.051997038687439 0.297526969971065 -0.2958528454156923
7.065225766867445 0.2964780218998632  -0.3009840273923147
7.078454495047449 0.30154617706412773 -0.3025168906614469
7.091683223227454 0.30463778191341895 -0.29977076142297415
7.104911951407459 0.30379742950973176 -0.30177350252677887
7.118140679587465 0.30073104555504293 -0.3065589255002575
7.13136940776747  0.306134384481727 -0.3092009319906387
7.343029058647552 0.3294785581599586  -0.32690338641668093
7.3492027448002935  0.32453206379468386 -0.3312301951942728
7.355376430953035 0.329858239119757 -0.33063831750610445
7.361550117105777 0.33035250345899625 -0.3319475749621378
7.367723803258518 0.3264432775638332  -0.3338815790281568
7.37389748941126  0.33272316991506534 -0.3343242053053837
7.380071175564002 0.3316711629614458  -0.3366629767430269
7.386244861716744 0.33113861826325375 -0.33600175818083877
7.392418547869485 0.33578173730475946 -0.3375570809264807
7.398592234022227 0.33517612926227003 -0.3376349315272361
7.4047659201749685  0.33631083423703245 -0.3377918866773939
7.41093960632771  0.33847143284313136 -0.33859118446434355
7.417113292480452 0.33878612970485356 -0.3378328867286317
7.423286978633193 0.341067827100439 -0.3392590482151451
7.429460664785935 0.34051644713999324 -0.3373711804504811
7.435634350938677 0.3419794033588109  -0.33780277614538406
7.441808037091419 0.34216150322979616 -0.3400664405268877
7.44798172324416  0.3422171824576628  -0.3361187097922124
7.454155409396902 0.3432141669019734  -0.3388250954346003
7.4603290955496435  0.3422572867669588  -0.34087958104565474
7.466502781702385 0.34360111070336874 -0.3355917845558423
7.472676467855127 0.34202996854350953 -0.34155838375591135
7.478850154007868 0.34210472722659757 -0.34197244718972736
7.48502384016061  0.3444519037764743  -0.33918491275843793
7.491197526313352 0.3407741652003746  -0.3446863888870137
7.497371212466094 0.3429181529691624  -0.34403712350852017
7.503544898618835 0.345296999542567 -0.34407337871511917
7.509718584771577 0.34019281058019435 -0.34789505636852047
7.5158922709243186  0.34567246508250604 -0.3474522340215346
7.52206595707706  0.3464115128050234  -0.3490196457548661
7.528239643229802 0.34345898570485944 -0.3501395781173461
7.534413329382543 0.348859140396345 -0.35075403413404327
7.540587015535285 0.34835063243882797 -0.3524646085405732
7.546760701688027 0.3482563856061364  -0.3517941575412396
7.552934387840769 0.35213876981010733 -0.3534752621404156
7.55910807399351  0.3516693296351541  -0.35268115407995454
7.565281760146252 0.3531395503077957  -0.35309211044439354
7.571455446298994 0.3544098830316415  -0.3534268623208734
7.577629132451735 0.35488906754229177 -0.35228891715304944
7.583802818604477 0.35673363716782835 -0.354109475987861
7.589976504757218 0.355967800609031 -0.3521786475466029
7.5961501909099605  0.3575598874839734  -0.35190151486883375
7.602323877062702 0.3568543551655736  -0.35497617510974966
7.608497563215444 0.357155134036956 -0.3511288661752588
7.614671249368185 0.357709177618881 -0.3535713594847103
7.620844935520927 0.3563319750853028  -0.35601343163946325
7.627018621673669 0.35809511021289836 -0.3513935112681103
7.63319230782641  0.3564971275793894  -0.35667584099732447
7.639365993979152 0.35579935206994023 -0.35737907501833144
7.645539680131893 0.3589924777361641  -0.35557237415680526
7.6517133662846355  0.3554449252827489  -0.36012597001497487
7.657887052437377 0.3572611088559282  -0.35985738511177706
7.664060738590119 0.3600704821352607  -0.36029813710616404
7.67023442474286  0.35544177606285243 -0.36340688403903765
7.676408110895602 0.3604024059250461  -0.36294272865329147
7.682581797048344 0.3614646988869436  -0.3647838276871908
7.688755483201085 0.3595000010484144  -0.36499287094665195
7.694929169353827 0.3639270879439881  -0.3657474250621673
7.701102855506568 0.36381181208058333 -0.3666665183030573
7.7072765416593105  0.3641267673696533  -0.3660833327626106
7.713450227812052 0.36737635057393714 -0.36788286012209576
7.719623913964794 0.36678783279040067 -0.36610289721443356
7.725797600117535 0.3685384357816625  -0.366843020167253
7.731971286270277 0.36888339794266306 -0.36699012939285136
7.738144972423019 0.36949685196657217 -0.36513839504219187
7.7448346132915615  0.3690665607640284  -0.36730289282210093
7.7515242541601035  0.3686762804226488  -0.3627640325078691
7.7582138950286454  0.36628706921484777 -0.3681739167615495
7.764903535897187 0.36613555528500313 -0.3696319796930949
7.7715931767657285  0.3692879337627792  -0.37230590949776937
7.77828281763427  0.36930315859176865 -0.3722454444260249
7.784972458502812 0.37353942786233696 -0.3729472505646421
7.791662099371354 0.3740061763960847  -0.3689955431279779
7.845179226319691 0.3783106331112531  -0.3726496749375017
7.851868867188233 0.3768453565519276  -0.3757281903977486
7.858558508056775 0.3766480058187819  -0.37185984701499586
7.865248148925316 0.37460619591655026 -0.37710739309116725
7.871937789793859 0.37447081681329947 -0.37871620409460865
7.8786274306624025  0.37814726519853264 -0.3814049401457853
7.8853170715309435  0.3787437351049419  -0.3803837018774337
7.8920067123994855  0.3820573020859319  -0.3807352286114542
7.898696353268027 0.38240304434694755 -0.376097290239284
7.952213480216364 0.38627689413412913 -0.3793355338066398
7.958903121084906 0.3837217372170373  -0.38354834733440396
7.965592761953448 0.38398227324959927 -0.3807487193361717
7.97228240282199  0.38235128365520055 -0.38544069993501934
7.978972043690533 0.38220538912992597 -0.3869976604516653
7.9856616845590755  0.3864039736912209  -0.38961329064644445
7.9923513254276175  0.3873932471376187  -0.3876390347211551
7.999040966296159 0.38971740029695323 -0.3878301647621035
8.0057306071647 0.38992635377469587 -0.3823621967761651
8.059247734113036 0.3933463040326599  -0.3851757152559215
8.07262701585012  0.39062091688319944 -0.3889407350904558
8.086006297587204 0.3893116484423027  -0.3944078069289389
8.099385579324288 0.39517460736793836 -0.39396419218597994
8.112764861061372 0.3965323539302319  -0.3877751662048146
8.126144142798456 0.39367554381589104 -0.39275224589922775
8.13952342453554  0.3926206229016645  -0.39776707863725463
8.152902706272625 0.398717463727034 -0.39676619343711805
8.166281988009708 0.39948085605552874 -0.39016026061729275
8.17284982849201  0.39525112541988167 -0.39708369987499414
8.179417668974311 0.396361018856909 -0.39564157049412
8.185985509456613 0.3965963905649595  -0.3983102373853431
8.192553349938914 0.3933954896896353  -0.4006150677316191
8.199121190421215 0.39991251085758317 -0.40230780016815887
8.205689030903518 0.39982270401295833 -0.401253012058267
8.212256871385819 0.4035023304374195  -0.4016713943512358
8.218824711868121 0.4024152062576006  -0.39650498037011545
8.225392552350424 0.4024814985868848  -0.39866185151437517
8.231960392832725 0.4003574520959073  -0.398857491957273
8.238528233315026 0.4013350992330541  -0.39624879117461675
8.245096073797328 0.3939486971458653  -0.40268293559858465
8.251663914279629 0.401130788591136 -0.40285430904367364
8.258231754761931 0.4001443952951394  -0.40573016398750866
8.264799595244233 0.40194290252025305 -0.4046488761149652
8.271367435726534 0.40492614022269663 -0.4037495205355266
8.277935276208835 0.40634193873014796 -0.401887701553657
8.284503116691136 0.40539169147911597 -0.4031995544763587
8.291070957173439 0.4054882043464255  -0.39546093414276007
8.29763879765574  0.4006246701248188  -0.40357301749646457
8.30420663813804  0.4020950080595572  -0.4029953999742474
8.310774478620342 0.4031855856589894  -0.4051296089509974
8.317342319102643 0.39982860973866  -0.40718811031165636
8.323910159584946 0.40662627192602413 -0.40869138727787035
8.330478000067247 0.4066844461745988  -0.4068323590681702
8.337045840549548 0.409670010159023 -0.4071342232399147
8.343613681031849 0.40830093911832976 -0.40197484008953865
8.35018152151415  0.4078746408153697  -0.4040790924754517
8.356749361996453 0.40525817629427635 -0.4051513144545914
8.363317202478754 0.4069095670047314  -0.4029367787853589
8.369885042961055 0.39989182875160345 -0.40906424013181286
8.376452883443358 0.4068402115938589  -0.40915647373128766
8.383020723925661 0.4067343519004554  -0.4113205709475979
8.389588564407962 0.4085258753023945  -0.41000014004800944
8.396156404890263 0.410748516958461 -0.40892644133449313
8.402724245372564 0.41191815130521603 -0.4062681028465902
8.409292085854867 0.4101185038266874  -0.40845801889475036
8.415859926337168 0.4102100197951651  -0.4017452793906733
8.42242776681947  0.4059666904874498  -0.40897928214379753
8.42899560730177  0.4067042734627693  -0.40911601066099185
8.435563447784071 0.4087073662715652  -0.41124764653948664
8.442131288266374 0.40625278281424637 -0.4124796605234051
8.448699128748675 0.41223489005274566 -0.4137285505419821
8.455266969230976 0.4122068230628363  -0.4110030286977275
8.461834809713277 0.4144955143967214  -0.41142965072109766
8.468402650195578 0.412873156536536 -0.40701738066729415
8.474970490677881 0.4122692927309378  -0.4083718384659215
8.481538331160182 0.40883401025069915 -0.4103639477816864
8.488106171642483 0.4113708785257689  -0.4088997322846498
8.494674012124785 0.4052434993697061  -0.41431683412266385
8.501241852607087 0.41144436061724304 -0.4141401568899341
8.507809693089389 0.41208873895131143 -0.4155484359547702
8.51437753357169  0.4138832363984725  -0.4140251895154667
8.52094537405399  0.41526433631226  -0.4129653699188047
8.527513214536292 0.4161747131608878  -0.4093584568922339
8.534081055018595 0.4134958098700666  -0.41260158643333883
8.540648895500897 0.41369120341200516 -0.40722450075026034
8.547216735983199 0.41022092936628646 -0.41326997667823456
8.5537845764655 0.4101711897094307  -0.41397269953283156
8.560352416947802 0.41312482747802143 -0.41604815262026257
8.566920257430104 0.41144649449349413 -0.41644362704960924
8.573488097912405 0.4166949479854323  -0.4174379298929487
8.580055938394706 0.4164079280038931  -0.41382150152971053
8.586623778877007 0.41794308060564156 -0.4145439771651615
8.592750941422048 0.4167046648589799  -0.4126861203822157
8.59887810396709  0.4180307298298059  -0.41036619863279294
8.60500526651213  0.4163752388692775  -0.41459121814772426
8.611132429057172 0.4162729697411271  -0.41104908988320366
8.617259591602211 0.4171284157445765  -0.41094756094073864
8.623386754147253 0.41434599510431513 -0.4148488594587882
8.629513916692293 0.41569295030571135 -0.41001474604999566
8.635641079237333 0.41531655547583934 -0.4129822796739834
8.641768241782373 0.41228382779135725 -0.41534163141413666
8.647895404327414 0.41559243956742137 -0.41196938189909443
8.654022566872454 0.4135399943237366  -0.4154298296399889
8.660149729417496 0.4107007960544017  -0.41607691986048834
8.666276891962537 0.41577507877226805 -0.414786222145765
8.672404054507577 0.4121368106004421  -0.4179657449624652
8.678531217052619 0.4123493327526038  -0.41744933860737904
8.684658379597659 0.4161913482061617  -0.4176416988305139
8.6907855421427 0.4118941336242284  -0.4198820031199368
8.69691270468774  0.41463783746751703 -0.41870654888259734
8.703039867232782 0.41682999940181276 -0.4200985151862226
8.709167029777822 0.4143048099080194  -0.41980715613199815
8.715294192322862 0.4172373914536206  -0.4196791676195099
8.721421354867902 0.41777893857095955 -0.42070793339137297
8.727548517412943 0.41700711587060574 -0.41922330934771634
8.733675679957983 0.41978623548605054 -0.4202292926806255
8.739802842503025 0.4189460526934789  -0.41867122389815875
8.745930005048066 0.41956559258533455 -0.418311044271844
8.752057167593106 0.4208405764066167  -0.4195116139049261
8.758184330138148 0.4199129079832231  -0.41615318552306035
8.764311492683188 0.42155966099384684 -0.4172834788997133
8.77043865522823  0.42037542280029816 -0.41786022306346515
8.77656581777327  0.4205459425122026  -0.4136135448695156
8.78269298031831  0.42063813947832845 -0.4170398716913322
8.788820142863349 0.4194410966226079  -0.4162405668068138
8.79494730540839  0.4207435767631585  -0.41152902847423556
8.80107446795343  0.41815493702701007 -0.4173520443338707
8.807201630498472 0.418255182076246 -0.41493551424248504
8.813328793043512 0.4194171694130087  -0.41283612686641247
8.819455955588554 0.4153683732839471  -0.4179318355422679
8.825583118133595 0.41713011186872556 -0.4143194193372982
8.831710280678635 0.417812645379521 -0.4152345618850646
8.837837443223677 0.4127575299431238  -0.41871072950095584
8.843964605768717 0.41729050263957906 -0.4163307767945544
8.850091768313758 0.41635121505004363 -0.41801463612952466
8.856218930858798 0.41149942131382666 -0.4195768823697376
8.862346093403838 0.41777681080626017 -0.4185894336806079
8.868473255948878 0.41527943345509705 -0.4207562796000599
8.87460041849392  0.4134895310613318  -0.4203136249771985
8.88072758103896  0.4185161146279565  -0.4206866383717471
8.886854743584001 0.41560484102789563 -0.4220607899006116
8.892981906129041 0.41614490784552755 -0.42080686785486343
8.899109068674083 0.4194029238937797  -0.42222957608707207
8.905236231219124 0.4175952771954112  -0.42120979688337773
8.911363393764164 0.4190410238551858  -0.4209802526532682
8.917490556309206 0.4201649769320175  -0.42126188283594085
8.923617718854246 0.4196503313557649  -0.41980920252415826
8.929744881399285 0.4217208038187267  -0.42079002100247426
8.935872043944325 0.4206306135663928  -0.4183001535233946
8.941999206489367 0.42137938806977826 -0.4181346805001193
8.948126369034407 0.42183600512314323 -0.4196521775104457
8.954253531579448 0.4208052745900539  -0.41501945372485644
8.960380694124488 0.4224082651721691  -0.41649446483663033
8.96650785666953  0.4205640062080567  -0.4182180401589634
8.972635019214572 0.42064968402003944 -0.4119281536786869
8.978762181759611 0.41991977000251124 -0.41658114925978945
8.985405299020453 0.4172938479892504  -0.416039796219963
8.992048416281294 0.417875930042924 -0.41447564771566053
8.998691533542134 0.40982358730044477 -0.41998872780545726
9.005334650802975 0.41816053273426473 -0.4204479230368672
9.011977768063817 0.41738469221635177 -0.4210570255291531
9.018620885324658 0.42043485079881315 -0.42012519559820455
9.025264002585498 0.4203234461127336  -0.41548611808357366
9.03190711984634  0.4208712508091798  -0.4155679891168987
9.03855023710718  0.41746114314866284 -0.4163159248672279
9.045193354368022 0.4180182992038393  -0.41242974228289264
9.051836471628862 0.40981299500338586 -0.41926284593838575
9.058479588889703 0.4170229139210474  -0.4192403454953166
9.065122706150545 0.4163017670969508  -0.421096442678025
9.071765823411386 0.41850446217351733 -0.4195736216044739
9.078408940672226 0.4197680022218955  -0.4167569768478715
9.085052057933067 0.42111255744007603 -0.41456938696079343
9.091695175193909 0.41748972646948435 -0.4162888537280171
9.09833829245475  0.4177801169208132  -0.4108348205861367
9.10498140971559  0.4107599904376849  -0.41806169696073275
9.111624526976431 0.4155296343734065  -0.41785028807785957
9.118267644237273 0.4151682773427338  -0.4208354083001111
9.124910761498112 0.41639775660311545 -0.41885766831819227
9.131553878758954 0.41904867362994824 -0.41759649510432273
9.138196996019795 0.41988832322464037 -0.41419046197925247
9.144840113280637 0.4173848210390626  -0.41592520669540967
9.158126347802318 0.41154960764525  -0.4164605159611263
9.171412582324  0.41449567685284905 -0.419090289106731
9.184698816845682 0.4181439035852697  -0.41795652772013636
9.191341934106525 0.41835948240693976 -0.41375810817751085
9.197985051367366 0.4171198327080944  -0.41520662612535064
9.204628168628208 0.4161914540929564  -0.40840791687370304
9.21127128588905  0.41209429153562616 -0.414471804378265
9.217914403149889 0.4116249210198296  -0.4147382809863539
9.22455752041073  0.41370919821311786 -0.4167776261500498
9.231200637671572 0.41183835765130977 -0.416975678473816
9.237843754932411 0.4170183623413693  -0.41780465944678596
9.244486872193253 0.4165945861178617  -0.41325633725951916
9.257773106714936 0.4150083131911188  -0.408470425289817
9.271059341236617 0.4093238911216584  -0.41306043886617044
9.2843455757583 0.40948117104893267 -0.4157961341753818
9.29763181027998  0.4146478788134722  -0.4126622613147779
9.304274927540822 0.41597042875910606 -0.4127064061078888
9.310918044801664 0.41369258550773996 -0.4083763907648845
9.317561162062503 0.41217499940425656 -0.40945543968714004
9.324204279323345 0.4074527129547848  -0.41148274723659295
9.330847396584186 0.4112123293800086  -0.4115690082824494
9.337490513845028 0.40712296884653926 -0.4144255132015701
9.344133631105867 0.41244147305037665 -0.4142397868150196
9.350776748366709 0.4125591548584309  -0.41194627427680064
9.35741986562755  0.41497791488340363 -0.41096669710032296
9.364062982888392 0.4122545803961631  -0.4080466003259765
9.370706100149231 0.41161295175495183 -0.40652554057323453
9.377349217410073 0.40622966973583163 -0.40988202518234546
9.383992334670914 0.4094633483586744  -0.4087773361815467
9.390635451931756 0.4047918322665437  -0.4128229691028122
9.397278569192595 0.40948984371702946 -0.41204349940401697
9.403921686453437 0.4103537512086525  -0.41107229640901444
9.410124125777017 0.41074930707243734 -0.4100782050905538
9.416326565100597 0.4117763670363078  -0.4106058260170739
9.422529004424177 0.41082325315092  -0.4068527250770151
9.428731443747758 0.4122346410217012  -0.4079348338398232
9.434933883071336 0.41012041367074564 -0.40709301644361495
9.441136322394916 0.41064776370591427 -0.4028154357119608
9.447338761718498 0.4085320424776264  -0.4070053803804488
9.453541201042079 0.40786088430500567 -0.40389470730127447
9.459743640365659 0.40826356709086  -0.4024993401611699
9.465946079689239 0.4040894208934028  -0.406532046406228
9.47214851901282  0.4058383733474273  -0.4026029394411858
9.478350958336398 0.4048647039580883  -0.40470738675990753
9.484553397659978 0.400115497530235 -0.4064241018253442
9.490755836983558 0.4050961462388869  -0.4047610030554221
9.496958276307138 0.40193619632687044 -0.40726290478158295
9.503160715630719 0.4008674230022235  -0.40670350039554193
9.509363154954299 0.4048145285747335  -0.40699460682993444
9.515565594277879 0.4015163753145046  -0.40786247054099883
9.521768033601461 0.40331935960686544 -0.40672248824591006
9.52797047292504  0.4048424396156638  -0.40803140681815125
9.534172912248618 0.4036179735719724  -0.4058051191330607
9.5403753515722 0.4059062031166455  -0.4061392746792102
9.54657779089578  0.40497301000822267 -0.40388777629743017
9.55278023021936  0.4055373215497355  -0.40312856997956104
9.55898266954294  0.4056558328409567  -0.4038043641157827
9.56518510886652  0.4047060482917923  -0.3990639761111506
9.571387548190101 0.4053167163026204  -0.40087031198504625
9.577589987513681 0.40325691194097657 -0.40068840610401857
9.583792426837261 0.4038161198827407  -0.39481488027525224
9.589994866160842 0.4007621181976635  -0.4003465563584347
9.596197305484424 0.4003670586871273  -0.39802328311210183
9.602399744808002 0.4011274492373636  -0.39560879470706134
9.60860218413158  0.3958044761035379  -0.4003214151193406
9.614804623455163 0.39845250091475704 -0.3975364126803394
9.621007062778743 0.39817112418927214 -0.39827919018268687
9.627209502102323 0.3918815386028058  -0.40050906156317523
9.633411941425903 0.39814327410822115 -0.399259865982923
9.639614380749483 0.39577213072100687 -0.40106456196286494
9.645816820073064 0.3937012526502991  -0.4002604102387634
9.652019259396644 0.39829070753631113 -0.4007119451555468
9.658221698720224 0.39596322436809034 -0.40069811223978763
9.664424138043804 0.3965782134213444  -0.399556660596598
9.670626577367385 0.3984491821091557  -0.40006674136721926
9.676829016690965 0.3975468793906468  -0.39790757649245684
9.683031456014543 0.3993174237341716  -0.39829352900579423
9.689233895338123 0.39795847168171017 -0.3951050552942352
9.695436334661705 0.39863112734711426 -0.39463101673368745
9.701638773985286 0.3979180821941035  -0.39575443041090247
9.707841213308866 0.3969706971587569  -0.3898071302856157
9.714043652632446 0.39674182204669955 -0.39258095518509023
9.720246091956026 0.39480570866766684 -0.3930850139584303
9.726448531279605 0.39542852802920514 -0.385604087345289
9.732650970603185 0.3914485321122233  -0.392497913115042
9.738853409926765 0.3913778497188162  -0.3909314722681103
9.745055849250345 0.39278341420671237 -0.38761483776203026
9.751258288573926 0.3863074221653367  -0.39288770437446535
9.757460727897506 0.38988156604146246 -0.3909607362292339
9.763663167221086 0.39030141760369713 -0.3908200919903365
9.769865606544668 0.3836123843287262  -0.3931127424432759
9.776068045868247 0.39002280049500565 -0.3921223792825028
9.782270485191825 0.38839397680804144 -0.39360618561463767
9.788472924515407 0.3860470558601282  -0.39224171298904154
9.794675363838987 0.39055076486734225 -0.39274527955660143
9.800877803162567 0.38886188368352326 -0.3919168611934744
9.806958442099907 0.38846527307069373 -0.39090889474049983
9.813039081037248 0.3904394887855307  -0.39173851799997444
9.819119719974587 0.38890634095339455 -0.389554654882841
9.825200358911927 0.3894589169028467  -0.38943186016329123
9.831280997849266 0.38947032018751687 -0.38872109738002364
9.837361636786605 0.3888079202066348  -0.3869657833044126
9.843442275723945 0.3900596814258105  -0.38780608914433523
9.849522914661286 0.388289870658638 -0.3855203524339383
9.855603553598625 0.38845265907198845 -0.3843074899418098
9.861684192535964 0.3883966774027882  -0.3859628558789867
9.867764831473304 0.38690622988384016 -0.38232950444000063
9.873845470410643 0.3877410680344994  -0.3817416835990196
9.879926109347982 0.38605030645565724 -0.38413676926902496
9.886006748285324 0.385340487390344 -0.3793264781219139
9.892087387222663 0.3856849750891377  -0.38022666521164994
9.898168026160002 0.38339925336579844 -0.3824038387312972
9.904248665097342 0.38362650460362996 -0.37667402817644413
9.910329304034681 0.38264815418107273 -0.37964211575860884
9.916409942972022 0.38060253154401913 -0.3808072306357704
9.922490581909361 0.3818124564125518  -0.3750297679147492
9.9285712208467 0.3795217769124756  -0.37931779916726094
9.93465185978404  0.3778236156093606  -0.37937662472085826
9.94073249872138  0.3800054177906271  -0.3753665780405908
9.946813137658719 0.37647754811316525 -0.37916250670579127
9.95289377659606  0.37523024519713827 -0.37812842770516175
9.9589744155334 0.378279256771747 -0.3759736102892623
9.96505505447074  0.37367214836504065 -0.3790733634081007
9.971135693408078 0.3743756336715854  -0.37766461757664616
9.977216332345417 0.37667153518868307 -0.3766530962892387
9.983296971282757 0.37124739177522703 -0.37893611244384284
9.989377610220098 0.3739464808314028  -0.377244651121212
9.995458249157437 0.3752065480157142  -0.37721623069075766
10.001538888094778  0.3704738419236369  -0.37805524822103287
10.007619527032116  0.3737547627507305  -0.376725602594677
10.013700165969455  0.37389555281527415 -0.3774831436435472
10.019780804906796  0.37077073278731065 -0.3765638125571384
10.025861443844136  0.3736929352043037  -0.3760214681808927
10.031942082781475  0.372895233073544 -0.37670604799021423
10.038022721718816  0.3712334125492259  -0.37483045916770147
10.044103360656155  0.37364237625191443 -0.37506007045798007
10.050183999593493  0.3723112716523717  -0.37411903432873433
10.056264638530834  0.3716769511663968  -0.37290089059482645
10.062345277468173  0.37347369273411857 -0.37378284139252055
10.068425916405513  0.3716698156314031  -0.37117706827625
10.074506555342854  0.37192532248853116 -0.37083269230530636
10.080587194280193  0.37201032374617565 -0.3710277965273637
10.08666783321753 0.3708906058474378  -0.36804718367242784
10.092748472154872  0.3718113587780835  -0.3686950492687502
10.098829111092211  0.37025374317291465 -0.3681368408918172
10.104909750029552  0.36990698579002823 -0.36490016203770875
10.110990388966892  0.36994108450330115 -0.36682828741313045
10.117071027904231  0.36826353143785606 -0.3652625655591353
10.12315166684157 0.36866566913409377 -0.3619103227937179
10.12923230577891 0.3670257728600859  -0.3651985950040688
10.135312944716249  0.366102029436205 -0.3625329891224611
10.14139358365359 0.36676128173196454 -0.3598710189708285
10.14747422259093 0.3638345290405601  -0.3637030934176728
10.153554861528269  0.36384287437718177 -0.3600619276495606
10.159635500465608  0.364002274879624 -0.35935213804012733
10.165716139402948  0.36053940276241236 -0.36234100785648404
10.171796778340287  0.36159352073816664 -0.35801811626288893
10.177877417277628  0.3611933189870253  -0.35916394756005005
10.183958056214967  0.35731575774280216 -0.36109941759328695
10.190038695152307  0.35993002577174277 -0.3578034983623352
10.196635288805446  0.35708865470044737 -0.36107303582395645
10.203231882458587  0.35651688104610935 -0.361050277614016
10.209828476111726  0.3592383469821317  -0.36236045673828177
10.216425069764867  0.35877311997475825 -0.35982567976955626
10.223021663418006  0.360564688393038 -0.3593837799709231
10.229618257071147  0.35946304312916105 -0.3538449937180222
10.236214850724286  0.3573769565581891  -0.356077843513862
10.242811444377427  0.3556308864192453  -0.3528729945508471
10.249408038030568  0.3544996046304874  -0.3539493608480133
10.256004631683707  0.34967750711832374 -0.3559780180138356
10.262601225336848  0.35431403073586437 -0.3562301866307785
10.269197818989987  0.35232922174871156 -0.35631493708856804
10.275794412643126  0.35519438267070647 -0.3555648697763096
10.282391006296265  0.3544937657536734  -0.3517556846877346
10.288987599949406  0.35512005558998716 -0.35113422330254623
10.295584193602545  0.3521435206060253  -0.3498552968089408
10.302180787255686  0.3516225903079279  -0.34663820950719015
10.308777380908827  0.3454019271614415  -0.35085687102346436
10.315373974561966  0.34909963709627556 -0.3497690828889926
10.321970568215107  0.3462649186693627  -0.35233987232832464
10.328567161868246  0.3478424958392741  -0.35070362326215687
10.335163755521387  0.34941380013979423 -0.3496994099680351
10.341760349174528  0.3499318416016021  -0.3472018271558237
10.348356942827667  0.3486814780893022  -0.34675798050090073
10.354953536480805  0.3481034666341017  -0.34000466553148173
10.361550130133946  0.34302519910028056 -0.3454321683142781
10.368146723787087  0.3437063491289132  -0.34341822892853646
10.374743317440226  0.3425149747490129  -0.3453104755670904
10.381339911093367  0.34009039552943904 -0.34567556224692464
10.387936504746506  0.34419169791925386 -0.3464877071667726
10.394533098399647  0.34359575988515983 -0.34373554959073915
10.401129692052788  0.3448893459384466  -0.3432661048873128
10.407726285705927  0.343310884531049 -0.3371146540013192
10.414322879359066  0.34095873896538975 -0.33975926987278965
10.420919473012205  0.3389222042262004  -0.3381742423708777
10.427516066665344  0.338946110583628 -0.33740241559235423
10.434112660318485  0.33256850686483785 -0.34054833289564196
10.440709253971624  0.3384456316758585  -0.3404400665338874
10.447305847624765  0.33715549727182326 -0.34015620273302594
10.453902441277904  0.3389501240590572  -0.33891372463435304
10.460499034931045  0.3383447796208471  -0.3348850345620216
10.473692222237325  0.3353533559009722  -0.334264052198272
10.486885409543605  0.32837100696726884 -0.3348807733839644
10.500078596849885  0.3309613302920607  -0.3360232324635085
10.506675190503026  0.3311504025625405  -0.33402595141608293
10.513271784156167  0.33326252543780466 -0.3328958128444322
10.519868377809306  0.33333106292959025 -0.32994113532208447
10.526464971462447  0.33179629980615793 -0.33023618855766307
10.533061565115586  0.3308275749857922  -0.3233706343696267
10.539658158768727  0.3262871080655298  -0.3286650276978939
10.546254752421868  0.326244214784047 -0.3276287383907321
10.552851346075007  0.32653758383139764 -0.32860056855946107
10.559447939728148  0.3235125857710854  -0.32902647977072574
10.566044533381287  0.32778071796306574 -0.3293565719844487
10.572641127034428  0.3270304198265424  -0.326368704420773
10.57923772068757 0.3279132988972094  -0.3259018689590587
10.585834314340708  0.3259811677293056  -0.32072318053865756
10.59243090799385 0.32405198446178657 -0.32223670950211947
10.599027501646988  0.3210909212511998  -0.32211925323052665
10.60562409530013 0.3221096200163703  -0.32090208569132633
10.612220688953267  0.31652176466250254 -0.3238731970153589
10.618376604669146  0.3204384052210999  -0.32259124843329257
10.624532520385024  0.3195870094518939  -0.32326409765162517
10.630688436100904  0.316964572904686 -0.3220678581400251
10.636844351816784  0.32001403451935456 -0.32182058323811297
10.643000267532663  0.3185326727639018  -0.3214195042200219
10.649156183248543  0.31776269145761227 -0.31991923508588915
10.655312098964421  0.31951434013186114 -0.32042854355432654
10.6614680146803  0.31794612183252324 -0.31795037691182243
10.66762393039618 0.3183467709942087  -0.3175164957981901
10.67377984611206 0.31771030732688244 -0.3164743944031132
10.679935761827938  0.31714373796968437 -0.31417596210041915
10.686091677543818  0.31724639561258233 -0.31501014658945753
10.692247593259697  0.31552993147922975 -0.31233452079948926
10.698403508975575  0.31585537089614035 -0.3105644677597422
10.704559424691453  0.3138271327559396  -0.3126136330910319
10.710715340407333  0.31307715338208464 -0.3084862102256769
10.716871256123213  0.3125658711381627  -0.3089825248449213
10.723027171839092  0.31000757578963106 -0.31041319757447466
10.72918308755497 0.31051156580213396 -0.3056607952854435
10.73533900327085 0.30855144675274093 -0.3084392468606682
10.74149491898673 0.3062583860554826  -0.30847358020352705
10.747650834702606  0.308163469137206 -0.3057478786286695
10.753806750418486  0.3047094490025869  -0.30817074624931423
10.759962666134365  0.30401371728467813 -0.3071621379706717
10.766118581850245  0.306004244388029 -0.30618664163668824
10.772274497566125  0.3014522753389074  -0.30784211336874984
10.778430413282004  0.30341318564145786 -0.3062820193343738
10.784586328997884  0.30408297177293087 -0.306454685666655
10.790742244713764  0.3009681295406021  -0.306059675670431
10.796898160429643  0.30317367197258666 -0.3052163251229233
10.803054076145521  0.3025483960926167  -0.30551516456287336
10.809209991861401  0.30124000326643297 -0.30368112344068665
10.821521823293157  0.3015676482000731  -0.30196109813323163
10.833833654724916  0.30157890128248127 -0.30089187524117283
10.846145486156676  0.3011285399121534  -0.29827106487865856
10.858457317588435  0.29900278924501084 -0.29409285888281733
10.870769149020195  0.29651499884766097 -0.29357870873433833
10.88308098045195 0.2937565969845928  -0.294111128162059
10.895392811883708  0.2930558617760271  -0.29090670912390865
10.907704643315467  0.2913669989210539  -0.2894133869459216
10.920016474747227  0.2863561407804618  -0.2908306371926589
10.932328306178983  0.28633426672076545 -0.2907169855062541
10.94464013761074 0.2877234089953953  -0.2893003091441861
10.9569519690425  0.28556884496849205 -0.2882631642809839
10.969263800474259  0.28441441147933844 -0.2870958774063245
10.981575631906018  0.28486846171427843 -0.2856084289494289
10.993887463337778  0.285005338737003 -0.2844322933650049
11.006199294769534  0.28375480512085405 -0.28126506057513967
11.012871165201213  0.281845171682149 -0.2799847313145348
11.02621490606457 0.2762090070033339  -0.2794477658999403
11.03955864692793 0.2751969327554266  -0.2803215145556965
11.05290238779129 0.2774993315356159  -0.2776660155529882
11.112949221676407  0.27215661898960064 -0.26978548182381146
11.433199002397025  0.23731512991790055 -0.23543938385875204
11.852403485305125  0.19113397435850465 -0.19336777283506695
11.8585128774133  0.19175862948272238 -0.1932081527042111
11.870731661629659  0.19028800393790254 -0.19182269777300076
11.882950445846015  0.18887070847104914 -0.19055717445259893
11.895169230062368  0.1886853046138286  -0.1893300655915967
11.907388014278727  0.1884656423818683  -0.18831830495828122
11.919606798495083  0.18704181005994255 -0.18598247590166292
11.931825582711436  0.18562878040944344 -0.18355067014494503
11.944044366927795  0.18434864836387224 -0.18325078817317866
11.956263151144153  0.18309046780758953 -0.18191274141199146
11.96848193536051 0.18209798639095692 -0.1782652276891998
11.980700719576863  0.17979791370668902 -0.17842139442749982
11.992919503793221  0.17737675114855037 -0.1785979667289508
12.005138288009578  0.17719828089548903 -0.17627897902044207
12.017357072225936  0.1758353071548321  -0.1753502812788466
12.029575856442289  0.1722389210260632  -0.17555966320230543
12.041794640658646  0.17254475435463063 -0.1747620196818636
12.054013424875004  0.17264786442728305 -0.1734951884428762
12.06623220909136 0.17034295018444576 -0.17256696705804556
12.078450993307714  0.16956541141036635 -0.17180223556207397
12.090669777524072  0.16970518027437675 -0.17097754507504617
12.10288856174043 0.16888416061165434 -0.16929716768881417
12.115107345956787  0.16767084991909958 -0.16746537536122186
12.12732613017314 0.1667129006606392  -0.16562027623956038
12.145654306497677  0.16512597457166783 -0.16286779065783502
12.157873090714034  0.16382783365411502 -0.1629357169292716
12.194529443363102  0.1594513710021715  -0.15867926816555494
12.20674822757946 0.157111672688071 -0.15865046598119736
12.218967011795815  0.15725952398098275 -0.15642001887080156
12.23118579601217 0.1558857197121585  -0.15591447306373032
12.243404580228528  0.15252536734441757 -0.15591388976445475
12.250029927052507  0.15466768779591628 -0.1555533491234515
12.256655273876486  0.1535844266743343  -0.1553181694561173
12.263280620700465  0.15432346821306986 -0.1545158020266448
12.26990596752444 0.15375692175903952 -0.15246490234790439
12.276531314348418  0.15360293364770425 -0.15183929717882513
12.283156661172397  0.15190759035317072 -0.1511826555866588
12.289782007996376  0.15138831509948003 -0.1493476088472744
12.296407354820355  0.14814471931798814 -0.15101502287901414
12.349410129412185  0.14426777395153112 -0.14617456698619619
12.45541567859584 0.13698515268787811 -0.13681850830568112
12.667426776963154  0.12122880329560894 -0.12071573689412787
12.673611445849872  0.12115174705324538 -0.12063738375198094
12.679796114736588  0.12032059298841463 -0.1189274501956755
12.685980783623306  0.12030722585026402 -0.11876782898894318
12.692165452510025  0.11936832171041596 -0.11878693964701778
12.704534790283459  0.11814909943586377 -0.11748842184102257
12.716904128056893  0.11736086959663367 -0.11496327499129036
12.723088796943612  0.11567392518439616 -0.11645179049382663
12.735458134717046  0.11555036863375674 -0.11459942735091731
12.747827472490483  0.1143219514257457  -0.11454080406333877
12.760196810263917  0.11189362677903378 -0.11444018121477051
12.766381479150635  0.1133073067872921  -0.11389535249939534
12.778750816924072  0.11154303396894738 -0.11326175597470724
12.791120154697506  0.11144089285213202 -0.11257250480415154
12.803489492470941  0.1113273417377863  -0.11164185934762183
12.815858830244379  0.11099738041748752 -0.11066623272811332
12.828228168017814  0.11007540344944039 -0.10900398378442647
12.84059750579125 0.10896725846983848 -0.10710190218561176
12.852966843564683  0.10777605785367389 -0.10725755658931205
12.86533618133812 0.10636311443600022 -0.10655722743837756
12.871520850224838  0.1060382943803558  -0.10566047442169588
12.877705519111556  0.10598545295631553 -0.10466922945188555
12.883890187998272  0.10418103461195145 -0.10561976956692884
12.890074856884992  0.10467609656481544 -0.104615752876623
12.89625952577171 0.10430992558937936 -0.10442602714071539
12.902444194658425  0.10234377995270529 -0.10469158026649737
12.908628863545143  0.10367452478979461 -0.10403120685793919
12.914813532431863  0.10274827106693378 -0.10415922620781146
12.920998201318579  0.10188262854665356 -0.10365803320444716
12.927182870205296  0.10274952950909313 -0.10342616696742041
12.933367539092014  0.10176282990960264 -0.1031951074509836
12.93955220797873 0.10163603802804774 -0.10257401378191737
12.94573687686545 0.1018318995539753  -0.10253046779940327
12.951921545752167  0.10120580343879769 -0.1016629486193261
12.958106214638883  0.10135691088921046 -0.10141492111316401
12.964290883525601  0.10083585328473238 -0.10055171087156578
12.976660221299037  0.1003784879008382  -0.09996943697449343
12.989029559072472  0.09972688150845745 -0.09856920914077483
13.001398896845908  0.09867818716764729 -0.09661933020237613
13.013768234619342  0.0973510415661029  -0.09679583177893712
13.026137572392779  0.09583804054604132 -0.09663036236697777
13.038506910166213  0.095662878929824 -0.09538974680028753
13.050876247939646  0.09494076398151899 -0.09498671394619589
13.063245585713084  0.09319123645688658 -0.09485107306946283
13.069308454213559  0.09400367721458666 -0.09442267605917624
13.075371322714037  0.0928653201240209  -0.09454155504524829
13.081434191214516  0.09251993976497673 -0.09388432713553281
13.087497059714991  0.09304970979678703 -0.09365608320940247
13.09355992821547 0.09197813749076358 -0.09352134006155627
13.099622796715947  0.09188778171768897 -0.09292167392207405
13.105685665216424  0.09210631402639882 -0.0928569700663483
13.1117485337169  0.0912242796602791  -0.09238807473181129
13.117811402217379  0.0912671435406555  -0.09195860403661964
13.123874270717854  0.0911737023276095  -0.09193259447314497
13.129937139218333  0.09051164088688866 -0.09123377031797121
13.13600000771881 0.0906328925885371  -0.09098949258825423
13.142062876219287  0.09026731483739514 -0.09062412131964304
13.148125744719763  0.089818274762915 -0.09007548917551664
13.154188613220242  0.0899624027848155  -0.09000753303529742
13.160251481720717  0.08937358752678286 -0.0892887312058014
13.166314350221196  0.08912262904305437 -0.08892824592387036
13.172377218721675  0.08907440193507098 -0.08889641263259719
13.17844008722215 0.08848803112138943 -0.08795426936623203
13.184502955722628  0.08840347947091785 -0.0878049743746649
13.190565824223105  0.08803022032731252 -0.08768531956702759
13.202691561224059  0.08763986616466436 -0.08671649934130231
13.208754429724538  0.08696708211640124 -0.08646624829278267
13.220880166725491  0.0864477778614231  -0.08577238433908745
13.233005903726445  0.08581771231471796 -0.08420538278532116
13.2451316407274  0.08484190960761938 -0.0840689965039726
13.257257377728354  0.08398616545794646 -0.0839532425919587
13.263320246228831  0.08380358957108178 -0.0829252981204666
13.269383114729308  0.08366274375020977 -0.08258350473718745
13.275445983229785  0.08277254050915805 -0.08306870584524462
13.281508851730264  0.08279370083220905 -0.0818427169607557
13.287571720230739  0.08249401789853343 -0.08197137561546361
13.293634588731218  0.08160028205149591 -0.0821985683917323
13.299697457231696  0.08187461991494269 -0.08109327567516907
13.305760325732171  0.0813274174472849  -0.08138831461908777
13.31182319423265 0.08048945853387285 -0.08134328490152058
13.317886062733127  0.08099030907568958 -0.08048320084580232
13.323948931233604  0.08018064152557768 -0.08081403426249223
13.33001179973408 0.07951731204272444 -0.0805207725839455
13.33607466823456 0.08012099006152924 -0.07992303292780585
13.342137536735034  0.07907235912536648 -0.08023077443627899
13.348200405235513  0.07887098523815249 -0.07978989708661499
13.35426327373599 0.07926536533847707 -0.07939175596224904
13.360326142236467  0.07802226203293827 -0.07962329600009847
13.372451879237422  0.07842389853309248 -0.07886846093659135
13.384577616238376  0.07770590618404152 -0.0783918449413461
13.39670335323933 0.07668413333722265 -0.07803346348384513
13.408829090240285  0.07679421887854791 -0.0777624093044015
13.42095482724124 0.07658820017558919 -0.07701761232954447
13.451269169743625  0.07526015056150664 -0.07550990930713644
13.457847992959902  0.07532562174807103 -0.07505143519004628
13.46442681617618 0.07517600802830421 -0.07413482853025566
13.471005639392457  0.07450721997615004 -0.0742985164467615
13.477584462608736  0.07420912042387921 -0.0728496179637296
13.484163285825012  0.07316937655435538 -0.07377542537138274
13.490742109041289  0.07299294976961693 -0.07351662175030357
13.497320932257567  0.07303906010957126 -0.07360022767928533
13.503899755473846  0.07233378833479674 -0.07347461807050755
13.510478578690122  0.07308217718374452 -0.07341384848346864
13.523636225122678  0.07284425012400998 -0.07240980577360331
13.536793871555233  0.07178928460190198 -0.07138549740219421
13.549951517987788  0.07113305809950596 -0.0707640019858965
13.556530341204065  0.06963738215003415 -0.07138667163505419
13.563109164420343  0.0706986698396685  -0.07115509225056954
13.569687987636621  0.07036114866617268 -0.07105261702537917
13.576266810852898  0.07050349389974367 -0.07061767120298539
13.582845634069175  0.0703909494519751  -0.0698778900316415
13.589424457285453  0.07039516399393234 -0.06938067089579283
13.59600328050173 0.06964767142371524 -0.06940828871129545
13.602582103718008  0.06946402025467203 -0.06808634435795237
13.609160926934285  0.06807596109677785 -0.06915952922588674
13.622318573366842  0.06820552399016017 -0.06906959562397438
13.635476219799395  0.06842928512296108 -0.0687210626039047
13.64863386623195 0.06818839522578891 -0.06784630522241229
13.661791512664506  0.06708220676403631 -0.06711584111724365
13.668370335880784  0.06672478972104048 -0.06658839145612984
13.674949159097062  0.06660055887127461 -0.06659995293326555
13.681527982313337  0.06553307502754836 -0.06691943932094796
13.688106805529616  0.06642921878071607 -0.0668600959007205
13.694685628745894  0.06605101033993609 -0.06662692708861646
13.70126445196217 0.0664062415936948  -0.06638072292297402
13.707843275178448  0.06611639191068443 -0.06548628921625822
13.714422098394726  0.06606629277988883 -0.06535368638758837
13.721000921611004  0.06543534720765413 -0.06505225556796117
13.727579744827283  0.06525508044309443 -0.0643680258549692
13.734158568043558  0.0640335442211665  -0.06504880593307868
13.740737391259836  0.06461990040717193 -0.06473560280232603
13.747316214476115  0.06394118083463271 -0.06513335718023423
13.75389503769239 0.0641907564131276  -0.06475586650296355
13.760473860908668  0.06435967870304854 -0.0645586099749761
13.767052684124947  0.06436090633750084 -0.06403480795794678
13.773631507341225  0.06415530419092037 -0.06380445630249207
13.780210330557503  0.0639913981955003  -0.06267688964558053
13.786789153773778  0.06315856971122753 -0.06334372393232203
13.793367976990057  0.06307344891905092 -0.06267256392951384
13.799946800206335  0.06258947994215651 -0.0630842618403981
13.80652562342261 0.06219569223291971 -0.06308209936277123
13.813104446638889  0.0626558057146996  -0.06317344053644054
13.819683269855167  0.06235598209190561 -0.06282921153204937
13.826262093071445  0.06277763337586714 -0.06272537569124596
13.832840916287724  0.062478084149088474  -0.06181681667581049
13.839419739503999  0.06230615514872451 -0.06188740944139036
13.845998562720277  0.061863145945320525  -0.06121676869145704
13.852577385936556  0.0615486172310248  -0.061187329730760714
13.85915620915283 0.0606454887656459  -0.061435652239130834
13.865735032369109  0.06117085709429534 -0.06126761559442133
13.872313855585388  0.06041723786331005 -0.06155637820417474
13.878452000864405  0.06075134296906955 -0.06127326534396179
13.88459014614342 0.06072189316880219 -0.061359152834898456
13.890728291422436  0.0602763828079906  -0.06101622902370101
13.896866436701453  0.060622485142070484  -0.060921694947692284
13.90300458198047 0.060380311616670394  -0.06075782505368307
13.909142727259487  0.06019771590167938 -0.06045675492791395
13.915280872538505  0.06043440908959312 -0.06051314924756567
13.92141901781752 0.06009263802491923 -0.06000208320996694
13.927557163096536  0.06011501365410078 -0.059903699250310825
13.933695308375553  0.05997141983362481 -0.059750203232183365
13.93983345365457 0.05980955628276299 -0.05924771729881789
13.945971598933589  0.05983760791849895 -0.05938259007122479
13.952109744212606  0.0594801574902973  -0.058938909337567624
13.958247889491622  0.05949502683483617 -0.05855353248237824
13.964386034770637  0.05917519762548449 -0.05889538552398849
13.970524180049654  0.05898393919326815 -0.0581406643734625
13.976662325328672  0.05893147263566017 -0.05814312242403667
13.982800470607687  0.05847897359153759 -0.05841430784876933
13.988938615886703  0.05849719930985661 -0.057423429518878064
13.99507676116572 0.05814401380769593 -0.057961692163473
14.001214906444737  0.05780636464962738 -0.05794932661899451
14.007353051723754  0.05801565562650654 -0.05720041578678611
14.013491197002772  0.057335510504123846  -0.05781375573106247
14.019629342281787  0.057225483415649855  -0.05752866179870199
14.025767487560802  0.057530272957841265  -0.057191589615576015
14.03190563283982 0.056575587159959466  -0.0576595232933571
14.038043778118837  0.057009264874067836  -0.057339724689637814
14.044181923397856  0.05705333443553076 -0.057242763380375795
14.050320068676873  0.05608011090268437 -0.05743111740569201
14.056458213955889  0.056842632435851165  -0.05719057096412819
14.062596359234902  0.056604642324084214  -0.057282598148886296
14.068734504513921  0.056055889558249215  -0.05709010378575794
14.074872649792939  0.05668819697709257 -0.05703727351996367
14.081010795071954  0.05633407104567063 -0.057017243236333594
14.08714894035097 0.056132704693134146  -0.056739314947333794
14.093287085629987  0.05651732180956823 -0.05682908272472146
14.099425230909004  0.05622098880482845 -0.056497731680047375
14.105563376188021  0.056238482342724244  -0.05638409370141744
14.111701521467038  0.056248880570856294  -0.056200377262253846
14.117839666746054  0.05612527300722511 -0.05594717578004784
14.12397781202507 0.05627035546371634 -0.05601559264665258
14.130115957304087  0.055956046726662864  -0.055454156512228014
14.136254102583104  0.05599960105317258 -0.05541290836903637
14.142392247862121  0.05583951827829574 -0.0555437295986804
14.14853039314114 0.05565822839938513 -0.054737052860610876
14.154668538420154  0.05566993494914485 -0.05500970179468424
14.16080668369917 0.05535185458324997 -0.055057122529582156
14.166944828978188  0.05535101816986254 -0.054117960087184225
14.173082974257206  0.05499965775862777 -0.05480548104608896
14.179221119536221  0.05485824918142511 -0.05458350697834474
14.185359264815235  0.05495165482611006 -0.05390439216458342
14.191497410094254  0.05431785556862551 -0.05462915195923206
14.197635555373271  0.05440120759161664 -0.05415739859135221
14.203773700652288  0.054471116276675696  -0.05399347274618245
14.209911845931305  0.053692325036559664  -0.054460878431625274
14.21604999121032 0.05414735236760024 -0.054042597457965066
14.222188136489336  0.05398666548047641 -0.054128137366755036
14.228326281768354  0.05322158055539977 -0.054287956680194795
14.23446442704737 0.05395638588137345 -0.054041885990034205
14.240602572326388  0.0535344815373204  -0.05424213843003527
14.246740717605407  0.05325801017647654 -0.054112332095384665
14.25287886288442 0.053781320853959196  -0.0540603809710362
14.259017008163436  0.053256894161599594  -0.0541787084505972
14.265155153442453  0.05337017728894822 -0.05394186451469646
14.271809253437272  0.05368538553762559 -0.05398456105675655
14.27846335343209 0.053756809357559845  -0.05356857978241147
14.285117453426906  0.05372307230050677 -0.05341816529345983
14.291771553421723  0.05366612400171593 -0.052551566109285
14.298425653416539  0.05298233851617476 -0.05318524360142636
14.305079753411356  0.05305550377934988 -0.05274774954426795
14.311733853406173  0.052598947182328944  -0.053260608932666376
14.318387953400991  0.05264375088914475 -0.05324140935687915
14.331696153390626  0.052992637977432505  -0.05302138492097842
14.34500435338026 0.05306916159730393 -0.05211100706695544
14.358312553369892  0.05250921047406156 -0.05199659069050745
14.371620753359528  0.05201321753783695 -0.052598094198474755
14.378274853354345  0.05232031637631076 -0.05283773358288498
14.384928953349164  0.052284410602536656  -0.05252678296299114
14.39158305334398 0.052586680441361666  -0.05243792258970931
14.398237153338796  0.052523424958290474  -0.05173853880349345
14.404891253333611  0.052180082144810215  -0.052046656241557764
14.411545353328428  0.05204612210402063 -0.051328598421929154
14.418199453323247  0.05158612504565059 -0.05195404572342846
14.424853553318064  0.05147412858798176 -0.052012275565907186
14.431507653312881  0.05177486569825656 -0.05223952947411478
14.444815853302513  0.052094889984273654  -0.05206372982156938
14.458124053292147  0.05184851009791044 -0.051609033617816935
14.471432253281783  0.051217636533644854  -0.051427407536432275
14.478086353276598  0.051021456334373744  -0.0514847986478515
14.484740453271414  0.05131498795354316 -0.05168373450683887
14.49139455326623 0.05104494296218816 -0.05168435647853673
14.49804865326105 0.05165378249166927 -0.0517551591447124
14.504702753255867  0.05157542921060601 -0.051168513689317845
14.511356853250684  0.05155726349629098 -0.051250999627739255
14.5180109532455  0.05134382577807582 -0.05048001705637829
14.524665053240316  0.05093237507035941 -0.050981218603619205
14.531319153235133  0.05064978219788973 -0.05101686182531985
14.53797325322995 0.05093729603098354 -0.05117229349838723
14.544627353224769  0.050518233179849784  -0.051332846776340986
14.551281453219586  0.05121998177394686 -0.051411981625910416
14.557935553214401  0.05117024663335649 -0.050958485154491785
14.564589653209216  0.05130245388165792 -0.050965831564147246
14.571243753204033  0.05106793639471752 -0.05020329187712721
14.577897853198852  0.050722995378684334  -0.05061155633838288
14.58455195319367 0.050353894281161216  -0.05061014331736659
14.591206053188488  0.05063750021938515 -0.05070719040846163
14.597860153183305  0.050056449170125285  -0.05102623617424996
14.604514253178122  0.050846709069572366  -0.0510986692048963
14.611168353172937  0.05081017148124977 -0.050791376819584205
14.617822453167753  0.05108058215048523 -0.050746537297086586
14.624476553162571  0.05083571821895067 -0.050009366621627416
14.631130653157388  0.05058167577301438 -0.050313975513243736
14.637784753152205  0.05021076977402765 -0.050326792369211414
14.644438853147024  0.05041057539708968 -0.05029041023881246
14.651092953141841  0.04966163137414731 -0.050763776953231095
14.657747053136658  0.05053786302076503 -0.05082260829606739
14.664401153131474  0.05049530766741659 -0.05066265350714172
14.67105525312629 0.05087503074819006 -0.05058435260180465
14.677709353121108  0.05064433550772809 -0.049891710350669784
14.684363453115925  0.05050025606220239 -0.05008369066464479
14.691017553110743  0.05011983319749954 -0.05011501321195192
14.697230975168297  0.050347143865414674  -0.049646246071216295
14.703444397225855  0.049614483955715785  -0.050345491623180263
14.70965781928341 0.04998419623531467 -0.050138007903689115
14.715871241340967  0.05011677999983526 -0.0501197198673251
14.722084663398524  0.04924836609207145 -0.050482692968545705
14.72829808545608 0.05011267326996019 -0.05037133243587215
14.734511507513636  0.04993961931605426 -0.05058663492868685
14.740724929571192  0.04963559148655034 -0.05047019218906244
14.74693835162875 0.05027932347774033 -0.05057003992524109
14.753151773686305  0.05008298408437736 -0.050514519035050734
14.759365195743861  0.050146995109979234  -0.05041546944557879
14.765578617801419  0.05038039416663399 -0.05041656609753953
14.771792039858973  0.050340278415624884  -0.050199967727773014
14.77800546191653 0.05057689217015183 -0.0503152388427283
14.784218883974088  0.05037629448406584 -0.04982940043926032
14.790432306031642  0.050527038954593595  -0.0498727414579413
14.796645728089198  0.050366468371986536  -0.0500853469248367
14.802859150146753  0.05033054128990185 -0.04927413396402177
14.809072572204311  0.05020259694165439 -0.04983787010217487
14.815285994261865  0.050054641233211954  -0.04986240266170258
14.821499416319421  0.05020374810110147 -0.04911570596199502
14.827712838376979  0.0496270911273178  -0.04997883202606512
14.833926260434534  0.049761375141983165  -0.04971083303796976
14.84013968249209 0.049971757607466354  -0.049581394228433454
14.846353104549644  0.04912719088144385 -0.050148642402605484
14.852566526607202  0.04984015390276882 -0.049967612129122624
14.858779948664758  0.04976438714150872 -0.05012147718677221
14.864993370722313  0.049212971293903014  -0.050248776856332465
14.871206792779871  0.049996446963239405  -0.05026358579905157
14.877420214837427  0.049728175630079 -0.05044300899663736
14.883633636894983  0.049718805565788886  -0.0502796722310726
14.889847058952538  0.050168236590695935  -0.050466820361652565
14.896060481010096  0.05002893406149481 -0.05024256597821795
14.902273903067652  0.050250353624286655  -0.0502653681049768
14.908487325125208  0.05024975401143517 -0.05005411223283431
14.914700747182765  0.050327620617333806  -0.04996422413152837
14.92091416924032 0.050399371989337804  -0.05012873369460872
14.927127591297877  0.05028404923427443 -0.04951586473048067
14.933341013355435  0.0504142389358594  -0.049765156762313745
14.939554435412989  0.050179657249817176  -0.04990382525037004
14.945767857470544  0.05026864076664099 -0.04908036901056328
14.9519812795281  0.04992512220917845 -0.04987760400594207
14.958194701585658  0.049912691497194794  -0.04971893976277032
14.964408123643212  0.050088562924849186  -0.04934295048010269
14.970621545700768  0.049409311967425415  -0.05004922014388985
14.976834967758325  0.04982079509766669 -0.049790467764272726
14.983048389815881  0.049871894260083593  -0.04987531255451675
14.989261811873437  0.04909448374565208 -0.0502256471581221
14.995475233930991  0.04995926850818854 -0.05013220403629778
15.001688655988549  0.04971700665573317 -0.050402117038984354
15.007902078046106  0.04952906038886206 -0.05032254746429895
15.01411550010366 0.0501437822653835  -0.050447499416658985
15.020328922161218  0.049923798744222335  -0.050445982112016195
15.026542344218774  0.05008217145271499 -0.050377009581089496
15.03275576627633 0.05030694881282185 -0.05042775867228749
15.038969188333887  0.05028226364389749 -0.05023452689803823
15.045182610391443  0.05057734758455556 -0.05037306377742242
15.051396032448999  0.05040350394965246 -0.04994362232395669
15.057609454506554  0.05057756889252235 -0.04999464036253084
15.063822876564112  0.05046285921862792 -0.050173052306580375
15.070036298621666  0.05045283043738642 -0.04946338235876897
15.076249720679224  0.05037475637429952 -0.04998937871417345
15.082463142736781  0.050240575246197065  -0.04997625632164195
15.088676564794335  0.05039854709547143 -0.049324252053476465
15.09476818646565 0.049960185997665246  -0.05013245354526502
15.100859808136967  0.050089466627735606  -0.04989232566144547
15.106951429808284  0.05030602704184107 -0.049592149274487546
15.113043051479599  0.049706776020567646  -0.050260644329602346
15.119134673150912  0.05005363896114692 -0.0499459110301814
15.125226294822228  0.05021292518661484 -0.049912528532534056
15.131317916493545  0.049493674251951794  -0.050397275874770525
15.13740953816486 0.050141692915222004  -0.050160558950725895
15.143501159836175  0.0501327171711319  -0.05025615002603898
15.149592781507492  0.04940692268160699 -0.050528899903598226
15.155684403178808  0.050257504769679454  -0.05039699261460854
15.161776024850123  0.05007885837259203 -0.05059360745966927
15.16786764652144 0.049660385768615985  -0.05062893205780151
15.173959268192757  0.050393179554988864  -0.050632036502888104
15.18005088986407 0.050065574116909255  -0.050893917868709476
15.186142511535387  0.04997746966472772 -0.05071814667578398
15.192234133206703  0.05054064437437502 -0.05084238512935519
15.198325754878018  0.05027959215601574 -0.05088739670149833
15.204417376549335  0.0503288872150839  -0.05079359858265941
15.210508998220652  0.05069151239619889 -0.051004660733927336
15.216600619891967  0.05052233347812692 -0.050831641060859856
15.222692241563283  0.05068500521582485 -0.050852550538013866
15.2287838632346  0.05080997643788109 -0.050872760440796756
15.234875484905915  0.05077041583102256 -0.050745171092843454
15.240967106577228  0.05101570225776471 -0.05089260031134159
15.247058728248545  0.05091169368841891 -0.050644090501421525
15.253150349919862  0.05100039845885706 -0.05064688429329048
15.259241971591177  0.05112515379231096 -0.050863259899597536
15.265333593262492  0.05099921465693619 -0.05039579831573529
15.271425214933808  0.05118883751214625 -0.05055618874582945
15.277516836605125  0.05109058900300719 -0.050797534919010165
15.283608458276438  0.05107000850733389 -0.05015929349837467
15.289700079947753  0.05118021242204359 -0.0505638582833741
15.29579170161907 0.051017142274761215  -0.050734791179890125
15.301883323290387  0.05112202348495381 -0.04996649799871501
15.307974944961702  0.05096803808817515 -0.050683658053367026
15.314066566633016  0.050924015266783175  -0.05068810221042963
15.320158188304333  0.05113206123548333 -0.04994851484964532
15.32624980997565 0.05072442505556756 -0.05083274521373982
15.332341431646963  0.05083107056216946 -0.05067005550166579
15.338433053318278  0.05107942011945215 -0.050239570897766476
15.344524674989595  0.05048119205023574 -0.05100243426479994
15.350616296660911  0.05077979323599363 -0.050716457818752614
15.356707918332226  0.05102624682977058 -0.050593875434933946
15.36279954000354 0.05027074480961484 -0.05118337003171715
15.368891161674856  0.050898484491835046  -0.05095687941117297
15.374982783346173  0.05098534689466102 -0.05098094408739565
15.381074405017488  0.05012613620724282 -0.051365416708245626
15.387166026688803  0.051050888343785926  -0.05121879286188911
15.39325764836012 0.05096891315019742 -0.051369435942611764
15.399349270031436  0.05038671796560823 -0.05149265568394109
15.405440891702751  0.051227947674787944  -0.05147858837161365
15.411532513374064  0.05098842592831809 -0.05172704040264612
15.417624135045381  0.05073538584686464 -0.051601576559114855
15.423715756716698  0.051419804202163494  -0.05171280909389574
15.429807378388013  0.05118581896964937 -0.05180853182456706
15.435899000059328  0.05112877313994743 -0.05169216761031118
15.441990621730644  0.05161569597837445 -0.051898198702714185
15.448082243401961  0.05144696678966062 -0.05177142100013753
15.454173865073276  0.051534927260185166  -0.051763453411247363
15.460265486744593  0.0517698476736048  -0.05184267497371163
15.46635710841591 0.0517111745945482  -0.051693257255793205
15.472448730087223  0.051922175905289296  -0.051815346342164066
15.47854035175854 0.05188500511499064 -0.05162509239302745
15.49175550453277 0.051830441161540834  -0.051581300879261695
15.504970657307005  0.05107937536233644 -0.051747430214110934
15.518185810081237  0.05153857256197  -0.05209745229454515
15.531400962855468  0.05211224932136459 -0.052247197257201256
15.544616115629701  0.05227669698955367 -0.05196903847860161
15.557831268403932  0.05174139302728276 -0.05184350782725458
15.571046421178163  0.051896495827518416  -0.0519847084328019
15.584261573952398  0.0522585531035021  -0.052530624662101136
15.59747672672663 0.05267439222704885 -0.052346282200129
15.61069187950086 0.05240130015440304 -0.051950112659526926
15.623907032275094  0.05227488797827171 -0.05186843218497916
15.637122185049327  0.05235129472006055 -0.05264756674633921
15.650337337823558  0.05267577091460354 -0.05266925441393524
15.663552490597791  0.05298516846125376 -0.05210103026530691
15.676767643372022  0.052650983226487376  -0.051807750628806055
15.689982796146253  0.05243149468232284 -0.05273696168924048
15.703197948920485  0.05258268938991162 -0.052983411581345764
15.716413101694716  0.05316633881926751 -0.05255805881751215
15.72962825446895 0.0530065225350367  -0.05193160013158706
15.742843407243182  0.052524727115983195  -0.05283631720415503
15.756058560017413  0.052498622844005215  -0.05329139742990521
15.769273712791646  0.05329199850514191 -0.05306587931080806
15.782488865565877  0.05334028348854233 -0.052495445203973076
15.795704018340109  0.05275853217151138 -0.05302535786083633
15.808919171114342  0.052479528171926346  -0.0535823314006402
15.822134323888573  0.0534006376020282  -0.053582570756381766
15.835349476662806  0.05367275380669177 -0.053117500727858695
15.848564629437037  0.05324474859260202 -0.05333903436188823
15.861779782211268  0.052760474284935875  -0.0537564398137877
15.874994934985503  0.05352326719996086 -0.05406151221373631
15.888210087759735  0.05399416737800061 -0.053735826297540604
15.901425240533966  0.0537740505908066  -0.05365462051171525
15.907592138983821  0.053869882600949506  -0.05300846346011036
15.913759037433676  0.05381366222706683 -0.05342022473526249
15.91992593588353 0.05350759711503051 -0.05370997411929279
15.926092834333385  0.05385550231770936 -0.0531909942285959
15.932259732783242  0.053519826357722546  -0.05379028389216588
15.938426631233096  0.05330341392913846 -0.05381590187522743
15.94459352968295 0.053879809963373895  -0.053677667327180656
15.950760428132806  0.053290418490045836  -0.054181031052782616
15.956927326582663  0.05358209862120967 -0.05408903602438368
15.963094225032517  0.05394137557181431 -0.054187879247281225
15.96926112348237 0.053393349417919636  -0.054422046397483693
15.975428021932228  0.05395739434863473 -0.0543532289877948
15.981594920382083  0.05404794915540276 -0.05461886307394269
15.987761818831936  0.053861429803164426  -0.05445339837048666
15.993928717281792  0.05436207606687604 -0.054561211588957936
16.00009561573165 0.05429522962898832 -0.05450563492580909
16.006262514181504  0.05436353169042975 -0.05442231567105969
16.01242941263136 0.054648500918898155  -0.05458268875195459
16.01859631108121 0.05455105334004859 -0.05422811526672204
16.024763209531066  0.05479781363517278 -0.05436157335664795
16.030930107980925  0.054678485471188726  -0.0543234971900574
16.037097006430777  0.05475652347835399 -0.05393442615320216
16.043263904880632  0.054750520037351924  -0.054363978080569246
16.049430803330488  0.054639713152299214  -0.05405957343527824
16.055597701780343  0.0548244800151944  -0.05379130208137905
16.061764600230198  0.05446658572823103 -0.05442041746413504
16.067931498680053  0.05456767580080688 -0.05387051305944782
16.07409839712991 0.054580499746824856  -0.054114775695696594
16.080265295579764  0.05415577383292768 -0.05451675753981509
16.08643219402962 0.05455698552974695 -0.05406425297746795
16.092599092479475  0.05432262063993779 -0.05453060329824742
16.09876599092933 0.05393341513140517 -0.05465511986382612
16.104932889379185  0.05461870257567663 -0.0545231882204401
16.11109978782904 0.05412808003958159 -0.054962133194494105
16.117266686278892  0.05424807117125761 -0.05489303356175812
16.12343358472875 0.054720909863301886  -0.05499140695877261
16.129600483178606  0.054247403029443794  -0.055197224119825285
16.13576738162846 0.054668091969691725  -0.05511107056273068
16.141934280078313  0.05486008731951717 -0.05537035002099131
16.14810117852817 0.05468817668432551 -0.055181758757415
16.154268076978028  0.05511397366682613 -0.055272655183653326
16.16043497542788 0.055080077349375284  -0.05518725240389725
16.166601873877735  0.05514816082133092 -0.055098853193772654
16.17276877232759 0.05540308179234759 -0.05526850290064042
16.17893567077745 0.055289018107592694  -0.05485300192180398
16.1851025692273  0.05552982889068424 -0.05499312611797003
16.191269467677156  0.055386436033382766  -0.05503687564663786
16.19743636612701 0.05544609957027734 -0.05451470281256022
16.203603264576866  0.0554121179236705  -0.05500889545810466
16.20977016302672 0.055294610911204314  -0.054807644331544135
16.215937061476577  0.05547899892414624 -0.05436214698555846
16.222103959926432  0.055069393615152604  -0.05510207994749548
16.228270858376288  0.055174961790386653  -0.054650077220200585
16.234437756826143  0.05526005760936553 -0.054721753571836965
16.240604655275995  0.05471116586296104 -0.055236201578156445
16.246771553725853  0.055167455257482866  -0.05484851624484123
16.25293845217571 0.05503563350777441 -0.05518008312343925
16.259105350625564  0.05447008108061083 -0.055400768542372886
16.265272249075416  0.05526394113607094 -0.05527373189873663
16.27143914752527 0.05487259238123011 -0.05564817168503809
16.27760604597513 0.05481805675275339 -0.055598110749596044
16.283772944424985  0.055403328458591736  -0.05569432806887429
16.28993984287484 0.05500217373581041 -0.05587033437528699
16.296106741324696  0.055279217566669085  -0.0557653898751341
16.30278959449035 0.0556594571227649  -0.055801412773203135
16.309472447656006  0.055904880284032454  -0.05550505432406336
16.316155300821663  0.0558140556446825  -0.05544391066245013
16.32283815398732 0.055849714401550093  -0.054684987467238676
16.329521007152977  0.0550666954872421  -0.055562274951857234
16.33620386031863 0.05548027185436415 -0.055401065867602614
16.342886713484283  0.05512279690512286 -0.05601603660916846
16.34956956664994 0.0554665330876235  -0.05595681005534581
16.40303239197518 0.05563970729610239 -0.05613193470039006
16.409715245140838  0.05605608184457482 -0.05610812915346811
16.41639809830649 0.056287573105304384  -0.05579753324849156
16.423080951472148  0.05613745741100129 -0.055787777515090564
16.4297638046378  0.056153960334796985  -0.054962572288639606
16.436446657803458  0.05530660264586304 -0.05592552611750016
16.443129510969115  0.055792491450340626  -0.05581336349008319
16.44981236413477 0.05550973516258904 -0.05638834713496776
16.456495217300425  0.05579795273849911 -0.05628998062541428
16.50995804262567 0.055940555580357876  -0.056430256118104795
16.516640895791326  0.05638351121344803 -0.056346404850872334
16.52332374895698 0.05660001671333402 -0.05602402432009652
16.530006602122636  0.0563919159712029  -0.056068997063686996
16.53668945528829 0.05639504624030993 -0.05518182441253594
16.543372308453947  0.05548165013561163 -0.05622511972420809
16.550055161619603  0.05604281243321418 -0.056154658582481094
16.55673801478526 0.05582987486484723 -0.0566944238181184
16.563420867950914  0.05606687435100181 -0.056552158516929456
16.570103721116567  0.05651944813884132 -0.05643840366090784
16.58346942744788 0.05649178231356022 -0.05618434876339387
16.596835133779194  0.05554369303605121 -0.056349262274293176
16.6102008401105  0.0559629972744793  -0.05682086123982419
16.616883693276158  0.056176338560757925  -0.05665517159586292
16.623566546441815  0.056636073832564915  -0.05651169424005984
16.630249399607468  0.056836962641678354  -0.056180308251364365
16.63693225277312 0.056572773009473824  -0.056282073822299154
16.64361510593878 0.05656824060895793 -0.055338651597077335
16.650297959104435  0.05558827846850357 -0.0564555060600607
16.656980812270092  0.05622647009045832 -0.056419327582053556
16.66366366543575 0.056077312612038216  -0.056928783358540846
16.670346518601402  0.056268447196517996  -0.05673886277003345
16.677029371767055  0.05673290399229845 -0.05656591622577587
16.683712224932712  0.056925550538804326  -0.056231146545792386
16.69039507809837 0.056633959159746505  -0.056361674568740375
16.697077931264026  0.05662821526563217 -0.05541237959669486
16.703760784429683  0.05564244398766896 -0.05654334838098352
16.710443637595336  0.056292116031864105  -0.056519136388502714
16.71712649076099 0.056172293140946815  -0.05701794206528679
16.723809343926646  0.0563673226296188  -0.05680288013700145
16.730370396706064  0.056803218424052275  -0.05670737877291833
16.73693144948548 0.05693350921928903 -0.056360451580899217
16.74349250226489 0.05681884974430979 -0.0565132924101382
16.750053555044307  0.05676593294954506 -0.055438305854729635
16.75661460782372 0.056141184766526436  -0.056428934434778345
16.763175660603135  0.056185351986386196  -0.05631279786287144
16.76973671338255 0.05635752028637547 -0.05651736534693725
16.776297766161967  0.05573936196375877 -0.05685707944736515
16.78285881894138 0.05669407745422322 -0.057001904716644576
16.789419871720796  0.05664947256950123 -0.05679944158880051
16.79598092450021 0.05707710914934376 -0.05677414339320875
16.802541977279624  0.05687160151595618 -0.056097768829390715
16.809103030059042  0.05687880152243615 -0.056217343880377746
16.81566408283846 0.05645617215960148 -0.05637525753977473
16.822225135617874  0.05660317754959681 -0.05586917403103308
16.828786188397288  0.05561381042849461 -0.05676304934110567
16.835347241176702  0.05642167104130514 -0.056728010217637286
16.841908293956116  0.05633947261336636 -0.057116780336281044
16.848469346735534  0.0564685985035652  -0.05688046408649302
16.85503039951495 0.05690336486913047 -0.05681579576435447
16.861591452294366  0.05699791671130064 -0.05640020394599665
16.86815250507378 0.05688010345365051 -0.05658526184289594
16.874713557853195  0.0568010940150701  -0.05558946902898701
16.88127461063261 0.05625718954420418 -0.056462534205378334
16.887835663412027  0.056178729711404814  -0.05640186363257682
16.894396716191437  0.0564345484797203  -0.05657531995705371
16.90095776897085 0.05582169461573024 -0.0569152287886491
16.90751882175027 0.05673343727754394 -0.05702584621573948
16.914079874529683  0.05669149102852747 -0.05682012151623612
16.920640927309098  0.05709089642846183 -0.05677011260799173
16.927201980088512  0.056885326015771655  -0.05617752593005502
16.93376303286793 0.0569104543419198  -0.05617653676246225
16.940324085647344  0.05643455782737024 -0.05641579803810443
16.946885138426758  0.05659978190471683 -0.055907078609214
16.953446191206172  0.05568850430523969 -0.05676622880341687
16.960007243985586  0.05638051224172183 -0.056726956760026
16.966568296765004  0.05635941525039305 -0.057114173608400265
16.97312934954442 0.05645879653712642 -0.0568526703706999
16.979690402323833  0.05689366034289909 -0.05681264469933522
16.986251455103247  0.05695474689846764 -0.056339508980684024
16.992812507882665  0.056837049578789677  -0.056546896286754425
16.99937356066208 0.05673258209683944 -0.055624012204464716
17.00593461344149 0.056258711329906064  -0.056386663140283115
17.012495666220907  0.056068973814739034  -0.05637811869074391
17.01905671900032 0.05639756998607755 -0.056521830432140076
17.02561777177974 0.05578999366146451 -0.05686344123456812
17.032178824559153  0.0566591669719467  -0.05694193011452326
17.038739877338568  0.05662292461756134 -0.056736629598881684
17.045300930117982  0.056995270757858256  -0.056662270586212135
17.0518619828974  0.05679226257626442 -0.056142147863943996
17.058423035676814  0.05683063371312117 -0.0560319333955119
17.06498408845623 0.05631167160740601 -0.05634170556105554
17.071545141235646  0.056486593742688246  -0.05583077540758515
17.07810619401506 0.0556461205373599  -0.05665522634682896
17.084667246794474  0.05623082142811013 -0.056615426394849996
17.091228299573892  0.05626679442178608 -0.056999027321939795
17.097789352353306  0.056336556516531425  -0.0567184198875289
17.10435040513272 0.056772913452935364  -0.05669767355142342
17.110911457912135  0.056803206854674125  -0.05617790839010459
17.11747251069155 0.05668888332532016 -0.056398348869280135
17.124033563470967  0.05656004468735587 -0.05554133942021419
17.13059461625038 0.05614572427754869 -0.05620183477347556
17.137155669029795  0.05585683580756588 -0.05624080063949101
17.14371672180921 0.05624689331437159 -0.05635626558540518
17.149837096651364  0.055427198068423056  -0.05669679177236734
17.15595747149352 0.056118319411062476  -0.05651962153432805
17.162077846335674  0.05617547782463725 -0.0566727941561392
17.16819822117783 0.05567793396636452 -0.05666549356934667
17.17431859601998 0.056302191128437803  -0.056641699520815415
17.180438970862134  0.056173912042497165  -0.05681427597903813
17.186559345704293  0.05599680782233072 -0.056589576746478104
17.192679720546444  0.05648587707991653 -0.05671148824693841
17.1988000953886  0.05630431736931886 -0.05657572221285494
17.204920470230753  0.056327730363655244  -0.05647961219272398
17.211040845072908  0.05658084276862408 -0.05659999646901831
17.217161219915063  0.05643394294249996 -0.05626902293294996
17.223281594757214  0.056615375439200324  -0.05634728398247602
17.22940196959937 0.05652675237416141 -0.056238425188380195
17.23552234444152 0.05653375344386232 -0.05594102379799902
17.241642719283675  0.056633209937055944  -0.056211776285691434
17.247763094125833  0.056429912223625034  -0.05585181455976814
17.253883468967985  0.05657587527293071 -0.05563891970969195
17.26000384381014 0.05636263121102433 -0.05608785289293995
17.26612421865229 0.05630206667417213 -0.0554886597796424
17.272244593494445  0.05637392086628637 -0.055576554081003565
17.2783649683366  0.0560359250288291  -0.05597837813050673
17.284485343178755  0.05615594279268166 -0.055196286355079144
17.29060571802091 0.05600087269797232 -0.055703097138171696
17.29672609286306 0.05570025726413637 -0.05589116419855016
17.302846467705216  0.05601890365698311 -0.05527417834539969
17.308966842547374  0.055614587608619516  -0.05587387639508878
17.315087217389525  0.05540305570964398 -0.05583303467642412
17.32120759223168 0.05589577946703468 -0.05554373389465483
17.327327967073835  0.055262339697982756  -0.05605614549402895
17.33344834191599 0.05542227221180453 -0.055907564502587326
17.339568716758144  0.05578915637435006 -0.05584594148838133
17.345689091600295  0.05499032566856132 -0.05621637795643247
17.35180946644245 0.0555594669581118  -0.05601226085465299
17.35792984128461 0.05570582215414735 -0.05612671339651787
17.36405021612676 0.055168559172397574  -0.056167462499399745
17.370170590968915  0.05573325367187322 -0.056098926977155954
17.376290965811066  0.05565902051680801 -0.05631025973867421
17.38241134065322 0.055439375949481104  -0.05606056622571104
17.388531715495375  0.055910088526463765  -0.05614129766468141
17.39465209033753 0.05574862101945336 -0.05605002625011331
17.400772465179685  0.05572938435576772 -0.0559179423822319
17.406892840021836  0.05602514982483243 -0.05606795676033415
17.41301321486399 0.05584036642981514 -0.05571645386221585
17.419133589706142  0.05598525617658169 -0.05575358794529348
17.4252539645483  0.055944988421389445  -0.055709661690854284
17.431374339390455  0.05590805260897961 -0.055356975057073134
17.437494714232606  0.056049293846158074  -0.055591146374034416
17.44361508907476 0.0558185795358696  -0.055319750981336305
17.449735463916916  0.055926351215102046  -0.055019135967628315
17.45585583875907 0.05575806609508492 -0.05545690312323416
17.461976213601226  0.055659966952122476  -0.0549438680063437
17.468096588443377  0.05576766019294597 -0.054869697980324865
17.474216963285535  0.05540550899891981 -0.055336954770569206
17.48033733812769 0.05548383212794537 -0.05462673474720878
17.48645771296984 0.055395708134646594  -0.05497358136554719
17.492578087811996  0.055039134902755284  -0.055236798947018335
17.49869846265415 0.0553234122073814  -0.05459332606948261
17.504818837496305  0.055003272654241105  -0.055128025076000335
17.51093921233846 0.05470649209126557 -0.05516125169017343
17.51705958718061 0.05518896063022107 -0.05481549332142322
17.523179962022766  0.054635033391517554  -0.055298579081673904
17.529300336864917  0.054640491063389494  -0.055181354492452594
17.535420711707076  0.05507027167530244 -0.05507718969141376
17.54205704126503 0.054459788892215914  -0.05549564467093923
17.548693370822985  0.055211151540801105  -0.05547948732487206
17.555329700380938  0.055220517958451024  -0.05522995412670407
17.561966029938894  0.055552545784653865  -0.05509585894176922
17.568602359496847  0.05524679122482436 -0.05464652346598065
17.575238689054803  0.05517185850048625 -0.05447929457059085
17.58187501861276 0.054526937754121964  -0.054851849524056924
17.588511348170712  0.05485320242658288 -0.05462725056501294
17.64160198463435 0.0546233810130039  -0.05417392156604739
17.654874643750258  0.054384998054221215  -0.054830351219455356
17.668147302866167  0.05468937932538961 -0.054712092134345656
17.681419961982073  0.054870412993868675  -0.05390483921679994
17.694692621097982  0.05438103696697361 -0.05373015007286394
17.70796528021389 0.05395515162897058 -0.054467629524400124
17.7212379393298  0.05420461649381376 -0.0544934287355941
17.73451059844571 0.054547098389221414  -0.053776492736422155
17.74778325756162 0.05412838688983087 -0.05332208031355207
17.75441958711957 0.05369050040415485 -0.05399599256653815
17.761055916677527  0.05352800986212409 -0.05408792089795785
17.767692246235484  0.053899317232131434  -0.05431974411356721
17.774328575793437  0.053702466168277174  -0.054251842668202266
17.78096490535139 0.05428091214299713 -0.054301482793688204
17.787601234909346  0.0541878445549201  -0.053654523344845016
17.794237564467302  0.05412679371502759 -0.05380603415657475
17.800873894025255  0.05389138321568884 -0.05314142870270216
17.853964530488888  0.053640733935947676  -0.052978767689516065
17.860600860046844  0.053447951756777 -0.053060277302992814
17.867237189604797  0.05281696802056467 -0.053330193354745266
17.873873519162753  0.05327751145063234 -0.053298486553732864
17.880509848720706  0.052698818418737234  -0.0536839852666522
17.88714617827866 0.053380404719219954  -0.053640173443524355
17.893782507836615  0.05338840858082298 -0.05338146846426038
17.90041883739457 0.05368429990647149 -0.053229320063527416
17.907055166952524  0.053372679895672986  -0.05282319897750794
17.913691496510477  0.05329588152295455 -0.05259713904885481
17.920327826068434  0.052634547534087923  -0.052987479720961426
17.92696415562639 0.052955587515636296  -0.05275994533011428
17.933600485184343  0.05222311881603093 -0.05334985413106792
17.9402368147423  0.05287863594927695 -0.05323998998234013
17.946873144300255  0.052960982066405086  -0.053206603569691785
17.953509473858208  0.05318789915087872 -0.05295344536670284
17.960145803416165  0.053083399329968244  -0.05266422800243189
17.966341455036854  0.053149013306103415  -0.052249566200472095
17.97253710665755 0.05292585057372329 -0.05267753847474058
17.978732758278245  0.052820945337878164  -0.05215983591297257
17.984928409898934  0.05283472573945461 -0.05212464406796261
17.99112406151963 0.05239816341541673 -0.05252217969305002
17.997319713140325  0.05255442724768427 -0.05185283051288849
18.003515364761018  0.0523096063229629  -0.052308066842652365
18.00971101638171 0.051905487623558295  -0.05240610616999038
18.01590666800241 0.052368525891319885  -0.052101518144228746
18.022102319623098  0.05181565857856537 -0.05253250731904768
18.02829797124379 0.051857596110594664  -0.05241887576356981
18.03449362286449 0.05221574589947588 -0.05241270721770372
18.04068927448518 0.051594057274033114  -0.052624070384106526
18.046884926105875  0.052048286301892864  -0.052467862009476805
18.05308057772657 0.052102853667165216  -0.052650954814591096
18.05927622934726 0.051853574126528544  -0.0524268864428409
18.06547188096796 0.05226609271566673 -0.05246021901892047
18.07166753258865 0.05213159028307456 -0.05230280559002772
18.077863184209342  0.05215628592469513 -0.05216273667860228
18.08405883583004 0.05230314227746233 -0.05220734237007236
18.090254487450732  0.05217133928749474 -0.05177683923184051
18.096450139071422  0.05236783697241013 -0.051877666848779376
18.10264579069212 0.05209054281313274 -0.05168650916360558
18.108841442312812  0.05214675588840455 -0.05126666838222103
18.115037093933502  0.05192874677264182 -0.05168183360119817
18.1212327455542  0.05181772778428711 -0.05117963306106587
18.127428397174892  0.05183186303486712 -0.051122843032985595
18.133624048795586  0.0513959333676876  -0.05151804274734495
18.13981970041628 0.05154048859066319 -0.05085735438497397
18.146015352036972  0.05130628286911147 -0.05129055825062132
18.152211003657666  0.05089667028262941 -0.051392497573283645
18.15840665527836 0.05134663751950023 -0.05108474058005054
18.164602306899056  0.05080995662266297 -0.051499964090345
18.170797958519746  0.05082914463005586 -0.05138883689377573
18.17699361014044 0.05118569678350342 -0.05137462434392307
18.183189261761136  0.05057146270845464 -0.05158289230329948
18.18938491338183 0.05100433460369774 -0.051421912059210685
18.19558056500252 0.051063558237401205  -0.051594551631874534
18.201776216623216  0.050809845387655685  -0.051375911914791184
18.20797186824391 0.051207545792325664  -0.05140047841380554
18.2141675198646  0.051075466955182 -0.05124745195124297
18.220363171485296  0.05109173895833062 -0.05110248667043346
18.22655882310599 0.051236882977099225  -0.05114610427707763
18.232754474726683  0.051099795287420785  -0.05071770366019937
18.238950126347376  0.051285820880212704  -0.05080851503858346
18.24514577796807 0.05101494947069725 -0.050626330588074975
18.251341429588763  0.05106213878383412 -0.05020247680822797
18.257537081209456  0.05084972965110399 -0.05060465148084008
18.263732732830153  0.05073324916483433 -0.05011926164828651
18.269928384450843  0.05074794630034895 -0.05004115843750013
18.276124036071536  0.05031346922842306 -0.05043405787507217
18.282319687692233  0.05044659150900445 -0.04978432911647491
18.288515339312926  0.05022422565739822 -0.050194190537018755
18.29471099093362 0.04980917012271604 -0.050300902913051275
18.300906642554313  0.05024648515974441 -0.04999149640813304
18.307102294175007  0.04972817652304045 -0.05038980184642874
18.313297945795703  0.04972349656825251 -0.050282308482999284
18.319493597416397  0.05007941016777493 -0.05026065604578563
18.325689249037087  0.04947544416746011 -0.05046536921293639
18.331884900657784  0.04988460252573763 -0.05030021024570775
18.338080552278477  0.04995022017220614 -0.05046249862252601
18.344276203899167  0.04969354435156706 -0.0502495779418651
18.350471855519864  0.05007485524946134 -0.050265518562728054
18.356667507140557  0.0499465195007537  -0.05011689052542125
18.362741358375008  0.04989321250473642 -0.04999151975309022
18.374889060843916  0.049886500472416696  -0.04977252563287577
18.38703676331282 0.049940189315308525  -0.049789656101505146
18.399184465781726  0.04995731374661087 -0.049533078488395346
18.411332168250635  0.04972645597191976 -0.04906319270528738
18.42347987071954 0.049535592117798274  -0.04910086907060949
18.435627573188444  0.04941652242160616 -0.049160460243222036
18.447775275657353  0.04941274229911848 -0.04851498826923404
18.45992297812626 0.04907539947506635 -0.048457063471512775
18.472070680595166  0.04870495355621446 -0.048833328721289974
18.48421838306407 0.04874089832417856 -0.04847877018052508
18.49636608553298 0.04865170935684644 -0.04816859249289641
18.508513788001885  0.04802692741034923 -0.048546006293059496
18.52066149047079 0.04806540941956561 -0.048521157481539165
18.5328091929397  0.04830573406032802 -0.04825488836946474
18.544956895408603  0.04781561217628899 -0.04837100821073481
18.55710459787751 0.047522365651944835  -0.04851857614931538
18.569252300346417  0.04799897301861823 -0.04834033145927672
18.58140000281532 0.04783153678823264 -0.048210449661868396
18.59354770528423 0.04755294660777246 -0.04814308821246942
18.605695407753135  0.04776398270489636 -0.048128514534707276
18.617843110222044  0.04786142056837911 -0.047979725287690386
18.648212366394308  0.04766083576412038 -0.04743831549240811
18.660360068863213  0.04750808503616082 -0.047073345769060145
18.67250777133212 0.04741682025936511 -0.0471226998783444
18.684655473801026  0.04737959162471368 -0.046996470783511296
18.69680317626993 0.04725634131149131 -0.04637814729330602
18.70895087873884 0.046941730480377634  -0.046457924530577566
18.721098581207745  0.04667549434905645 -0.04664805109219087
18.73324628367665 0.04670600614646996 -0.04613222043399805
18.745393986145558  0.046444662185522324  -0.04590340794262122
18.75198379209581 0.04615260570113092 -0.04634641956571651
18.76516340399632 0.04626102735580715 -0.0465706142634356
18.778343015896827  0.04648088200664155 -0.04651696166775983
18.79152262779733 0.04640091985739586 -0.04601155707129338
18.80470223969784 0.04596857222487138 -0.045546014812614374
18.817881851598344  0.04575086372059909 -0.04576883784468742
18.831061463498855  0.0456894127232594  -0.045980202388130685
18.84424107539936 0.045901612703211477  -0.045607469721756284
18.857420687299868  0.045741321626810855  -0.04481566528858907
18.870600299200373  0.045251134402193825  -0.04497557774880124
18.883779911100884  0.0448677516489077  -0.04539371783269413
18.89695952300139 0.045121699246025576  -0.04525970790645967
18.910139134901897  0.04523947068802549 -0.04455243931746993
18.9233187468024  0.04479956075920379 -0.044367115438851565
18.936498358702913  0.04409366716093203 -0.044793037265034016
18.949677970603418  0.04432789447337905 -0.044888775789111816
18.962857582503926  0.044667241421909656  -0.044398199048304464
18.97603719440443 0.04444896785276977 -0.044087929766946074
18.989216806304942  0.0436877221814622  -0.04422158772833137
19.002396418205446  0.043612224683735185  -0.04444296843765432
19.061704671757727  0.04305610336825634 -0.04369659691503085
19.167141566961785  0.04200372268803415 -0.042520643871863974
19.173290694974774  0.04229163006991934 -0.04218768356385808
19.179439822987767  0.0422326120637449  -0.04223036246475877
19.18558895100076 0.041562145496180015  -0.042436228570200164
19.191738079013753  0.04216112917673142 -0.04223768816891477
19.19788720702675 0.041917937811395654  -0.04237138124490923
19.20403633503974 0.04155643176193645 -0.04232913946975428
19.21018546305273 0.042064873955293125  -0.0422835289517284
19.216334591065724  0.04172101572080172 -0.04239518246550616
19.222483719078713  0.04167632261730988 -0.0421956069012759
19.22863284709171 0.04198708845669213 -0.04227207208139547
19.234781975104703  0.041752012837615995  -0.04211753258806048
19.240931103117695  0.04182848943038927 -0.04202860035335105
19.24708023113069 0.041890069453276724  -0.04198093284967579
19.253229359143678  0.041793701765228446  -0.041776227843388775
19.25937848715667 0.04194883268342679 -0.04182316564909897
19.265527615169667  0.04175566752924567 -0.0414660043533662
19.27167674318266 0.041793741535674465  -0.0414114999930942
19.277825871195652  0.04172598373344326 -0.041509202198973846
19.28397499920864 0.041588997428472496  -0.040931013184866104
19.290124127221635  0.04163554552991739 -0.04110326108461326
19.296273255234627  0.041392421050570276  -0.041185248552814815
19.30242238324762 0.041386545861250734  -0.04044546927326096
19.308571511260617  0.04113319313668354 -0.04098019353303335
19.314720639273606  0.04102076741935111 -0.04088903922865476
19.3208697672866  0.041112973777138254  -0.04027336599989266
19.32701889529959 0.04058859516616481 -0.04090196678189496
19.333168023312584  0.04065562740402252 -0.04064428760877897
19.339317151325577  0.040800456465892566  -0.04038573469002353
19.34546627933857 0.04008262887064613 -0.04084812475142402
19.351615407351563  0.04049955355439355 -0.040618414139012175
19.357764535364556  0.04050842982080513 -0.04056863396762611
19.36391366337755 0.039773901770369796  -0.04078073553008435
19.370062791390538  0.04042107474210001 -0.04063213991451793
19.37621191940353 0.04025888263897645 -0.040725046710367314
19.382361047416524  0.0398782103563246  -0.040631250129936396
19.388510175429516  0.040375323653849805  -0.04061575006554172
19.394659303442513  0.040152578944721226  -0.04062778230903346
19.400808431455502  0.0400435233491464  -0.04044550386684467
19.413106687481488  0.04015224808027454 -0.0402884456419074
19.41925581549448 0.04019971012719674 -0.04022571314011353
19.425404943507473  0.04020143359651848 -0.040115256690535624
19.431554071520466  0.04013508891136927 -0.03989692299592986
19.43770319953346 0.04022585695136115 -0.039974087612999656
19.443852327546452  0.040014492968403516  -0.039623796271633135
19.450001455559445  0.040062230629975354  -0.03950394099811921
19.456150583572434  0.039900787597088226  -0.03969607971265896
19.46229971158543 0.039791481174909125  -0.03915730923371863
19.468448839598423  0.03977562514421075 -0.039267907653624055
19.474597967611416  0.039508596695387484  -0.03943761849283973
19.48074709562441 0.03954114535507699 -0.03877298820930661
19.486896223637398  0.03931023847177453 -0.039205189624916786
19.49304535165039 0.03910077705108339 -0.039213898282284755
19.499194479663384  0.03927643583044113 -0.038768967966051913
19.50534360767638 0.03885136669960508 -0.039188483393357204
19.511492735689373  0.03876723250708792 -0.039053194496376974
19.517641863702362  0.03902840747192769 -0.03888671820535047
19.523790991715355  0.038453901517122496  -0.03918576045171413
19.529940119728348  0.038695988117058634  -0.039015948625990386
19.53608924774134 0.038811676313579464  -0.03900902717311803
19.542238375754337  0.03830747650141693 -0.03909604046369346
19.548387503767326  0.038684079603598015  -0.03896844466671972
19.55453663178032 0.0386338068013503  -0.039073657519877354
19.560685759793312  0.03839590681040539 -0.03889008065780001
19.56755004479654 0.03884395211338844 -0.038849852201433104
19.57441432979977 0.03878541543895207 -0.038268982582301464
19.581278614803 0.038467661281035286  -0.0384166752013814
19.58814289980623 0.038375537259715944  -0.03795940129946244
19.595007184809457  0.03784959190540215 -0.03849310320566708
19.60187146981269 0.038161265011495 -0.038458117266602504
19.60873575481592 0.03825587340382744 -0.038391784435784074
19.61560003981915 0.03846122334252933 -0.0381457238055142
19.622464324822378  0.03817718362081223 -0.037781206385207496
19.629328609825606  0.037976028296490015  -0.037716005431611216
19.636192894828838  0.037427113096841706  -0.03793179288947887
19.643057179832066  0.03781171129726881 -0.038042046901486506
19.649921464835295  0.03767653191554683 -0.03799066107749052
19.656785749838527  0.0379899223990508  -0.0378959229169713
19.663650034841755  0.03790151047860279 -0.03724906472360539
19.670514319844983  0.03746035988844103 -0.03754469429990461
19.67737860484821 0.03744945715295618 -0.037253238171778294
19.684242889851443  0.03704934912036376 -0.037693040259161564
19.691107174854675  0.03732890660742508 -0.037606296154867436
19.697971459857904  0.03745135750052596 -0.037409125698323
19.704835744861132  0.03760099370432466 -0.037178070035681444
19.71170002986436 0.03722896060359235 -0.03693631054184712
19.718564314867592  0.03706711666248763 -0.03684802745666028
19.72542859987082 0.03649567251865448 -0.03715047405584101
19.73229288487405 0.03699654979594234 -0.03724976180398259
19.73915716987728 0.03694302501607921 -0.03706864188259617
19.746021454880513  0.03711798573309466 -0.0369545205464024
19.75288573988374 0.036994871217662206  -0.03621620948346301
19.75975002488697 0.036431260750923775  -0.0366937479845493
19.766614309890198  0.03653853440614886 -0.036542237181715474
19.773478594893426  0.03627799885534719 -0.03690214669831638
19.780342879896654  0.03651146689299305 -0.036730770298930855
19.787207164899886  0.036627826131752035  -0.03640014722186211
19.794071449903115  0.03665330904597949 -0.036228648203583416
19.800935734906343  0.03625803740292005 -0.03612149136727818
19.80780001990957 0.03618324392887891 -0.03599576210963802
19.8146643049128  0.03557605977211038 -0.036356626626233075
19.82152858991603 0.03620289774498203 -0.03642821251065319
19.828392874919263  0.03619073749940305 -0.03612227444248207
19.83525715992249 0.03622485162706764 -0.03603101728654756
19.84212144492572 0.03606645446988263 -0.03519683168277045
19.84898572992895 0.03542286500535797 -0.03586583033640793
19.85585001493218 0.03564666679130094 -0.035816662969080856
19.86271429993541 0.03556465048498802 -0.03605638854851601
19.869578584938637  0.03574268522201743 -0.03583050298985409
19.87644286994187 0.03578192225069692 -0.03544293645407135
19.883307154945097  0.03570353068356537 -0.03529238734095539
19.890171439948325  0.03526002890969666 -0.03533753496804923
19.897035724951554  0.03532787533554183 -0.035275583248968304
19.903900009954786  0.03488371126284797 -0.03554566241200328
19.910764294958017  0.03542996419129393 -0.03556807005452027
19.917628579961246  0.03539453861051451 -0.03513496651535839
19.924492864964474  0.03528749973756505 -0.035130271949411485
19.931357149967702  0.0351190919921819  -0.034513548149137055
19.938221434970934  0.03462415947721495 -0.035062029797490166
19.945085719974163  0.03477720612264798 -0.03506046066168303
19.95195000497739 0.03485364476329291 -0.03516803711027092
19.958814289980623  0.03501375967786582 -0.034905225923372134
19.965678574983855  0.03491070897928411 -0.034590572368549975
19.972542859987083  0.03478500742891713 -0.034373769956979765
19.97940714499031 0.03424253893934205 -0.03458398378262657
19.98627142999354 0.03450358748305156 -0.03459577172145213
19.99313571499677 0.03399207038657715 -0.035628743976049405
20. 0.05096923766651328 -0.04404391618484699
\end{filecontents}

\usepgfplotslibrary{fillbetween}
\begin{tikzpicture}[>=latex]
  \begin{axis}[
      xlabel={Time (\si{\pico\second})},
      ylabel={$\Re[\rho_{01}(t)]$ (system average)}
    ]
    \addplot[smooth, no marks, name path=A, black] table [x index=0, y index = 1]{echo.dat};
    \addplot[smooth, no marks, name path=B, black] table [x index=0, y index = 2]{echo.dat};

    \addplot[blue!25] fill between [of=A and B];

    \draw[<-] (axis cs:1.35,0.90) -- (axis cs:3,0.90) node[right]{Dephasing}; 

    \draw[<-] (axis cs:5.1,0.4) -- (axis cs:6.5,0.55) node[right]{Rephasing ($\pi$-pulse)};

    \draw[<-] (axis cs:9.1,-0.45) -- (axis cs:9.1,-.65) node[below]{Echo effect};

    \draw[|-|] (axis cs:0.5,0) -- (axis cs:5,0) node [midway, fill=blue!25]{$\tau$};
    \draw[-|] (axis cs:5,0) -- (axis cs:9.5,0) node [midway, fill=blue!25]{$\tau$};
  \end{axis}
\end{tikzpicture}

  \caption{\label{fig:echo} Photon echo effect.
    A $\pi$-pulse incident on a collection of 64 nearly-resonant quantum dots serves to re-phase their coherences (polarizations) as an ``echo'' flash long after both pulses have passed through the system.
  }
\end{figure}

\begin{figure}
  \centering
  \begin{tikzpicture}
  \begin{axis}[
    xtick = {7.75, 6.75, 5.75, 4.75, 3.75, 2.75, 1.75, 0.75},
    minor x tick num = {1},
    xticklabels={$2\pi$, $\frac{7\pi}{4}$, $\frac{3\pi}{2}$, $\frac{5\pi}{4}$, $\pi$, $\frac{3\pi}{4}$, $\frac{\pi}{2}$, $\frac{\pi}{4}$},
    xlabel = {Integrated pulse angle},
    ylabel = {Population ($\rho_{00}$)}
    ]
    \addplot graphics[xmin=0, xmax=8, ymin=-1, ymax=1] {figures/rainbow.png};
  \end{axis}
\end{tikzpicture}

  \caption{\label{fig:decoherence}
    Population distribution after $\SI{10}{\pico\second}$ for $128$ homogeneous dots exposed to different incident pulses. 
    Each column---shaded by the relative density of points---depicts one simulation for a given pulse intensity and each point within a column represents the excitation level, $\rho_{00}$, of a single dot.
    The horizontal variation within a column has no meaning other than to distinguish individual points, and the red circles outline the ``idealized'' population for a system without interactions.
  }
\end{figure}

\begin{figure}
  \centering
  \begin{filecontents}{screening.dat}
5.  0.9999969708869827
5.1 0.999996584341406
5.2 0.9999961578473717
5.3 0.9999956881502219
5.4 0.999995171796191
5.5 0.999994605124185
5.6 0.9999939842564228
5.7 0.9999933050890679
5.8 0.9999925632825548
5.9 0.9999917542513345
6.  0.9999908731534877
6.1 0.9999899148794416
6.2 0.99998887404059
6.3 0.9999877449571545
6.4 0.9999865216453998
6.5 0.9999851978045754
6.6 0.9999837668024254
6.7 0.9999822216610511
6.8 0.9999805550406343
6.9 0.9999787592233136
7.  0.9999768260956522
7.1 0.9999747471297896
7.2 0.9999725133637841
7.3 0.999970115380397
7.4 0.9999675432844903
7.5 0.9999647866788086
7.6 0.999961834637908
7.7 0.999958675680068
7.8 0.9999552977372679
7.9 0.999951688122596
8.  0.9999478334945633
8.1 0.9999437198185304
8.2 0.999939332325774
8.3 0.9999346554666029
8.4 0.9999296728610068
8.5 0.9999243672441174
8.6 0.9999187204046813
8.7 0.9999127131198783
8.8 0.9999063250797342
8.9 0.9998995348074716
9.  0.9998923195666556
9.1 0.9998846552569831
9.2 0.9998765162982495
9.3 0.9998678754990696
9.4 0.9998587039109255
9.5 0.9998489706403029
9.6 0.9998386426080649
9.7 0.9998276842575762
9.8 0.9998160571842911
9.9 0.9998037196272389
10. 0.9997905958321476
10.1  0.9997766948289732
10.2  0.9997619289495905
10.3  0.9997462310636236
10.4  0.9997295212427062
10.5  0.9997117013551837
10.6  0.9996926453545989
10.7  0.9996721842075371
10.8  0.9996500851540611
10.9  0.9996260207814038
11. 0.9995995167611674
11.1  0.9995698663057124
11.2  0.9995360064460355
11.3  0.9994963557109903
11.4  0.9994486017629833
11.5  0.999389403130976
11.6  0.9993139541137627
11.7  0.9992153811177747
11.8  0.9990839890177108
11.9  0.9989064160348181
12. 0.9986647350591142
12.1  0.9983354512296423
12.2  0.9978882505404906
12.3  0.9972843455975199
12.4  0.9964743931021984
12.5  0.9953961747859807
12.6  0.9939724036263209
12.7  0.992108995891713
12.8  0.9896938818995537
12.9  0.9865960096359075
13. 0.9826638640533641
13.1  0.9777228860040961
13.2  0.9715718642761402
13.3  0.9639796756397302
13.4  0.954685203315259
13.5  0.9434039173476882
13.6  0.9298433393740776
13.7  0.913726143821084
13.8  0.8948157275445259
13.9  0.8729384991550126
14. 0.8480025051116789
14.1  0.8200195308270347
14.2  0.7891372275622932
14.3  0.7556736578217932
14.4  0.7201306563908075
14.5  0.6831648073189316
14.6  0.6455180104740881
14.7  0.6079320386478734
14.8  0.5710745348966658
14.9  0.5354914299902651
15. 0.5015872296815752
15.1  0.4696273712903291
15.2  0.43975485771289563
15.3  0.4120139310011689
15.4  0.3863751774770793
15.5  0.3627584471700098
15.6  0.3410517996823668
15.7  0.3211260348212959
15.8  0.302845163119715
15.9  0.28607352498267324
16. 0.2706803368850458
16.1  0.25654236606822256
16.2  0.2435453030880707
16.3  0.23158426483635058
16.4  0.22056374186519068
16.5  0.21039720967755937
16.6  0.2010065531203306
16.7  0.19232140221142352
16.8  0.18427844218127615
16.9  0.1768207362147587
17. 0.16989708313129664
17.1  0.16346142160938754
17.2  0.15747228577800482
17.3  0.15189231280624887
17.4  0.1466878006501658
17.5  0.14182831276244062
17.6  0.13728632593100454
17.7  0.1330369172153108
17.8  0.1290574860153914
17.9  0.12532750752275723
18. 0.1218283140900071
18.1  0.11854290137192308
18.2  0.11545575640758532
18.3  0.11255270511528931
18.4  0.10982077695201137
18.5  0.10724808474365805
18.6  0.10482371792100248
18.7  0.10253764760014988
18.8  0.10038064212749753
18.9  0.09834419186966098
19. 0.09642044217091461
19.1  0.09460213352646007
19.2  0.09288254813116914
19.3  0.09125546206205387
19.4  0.08971510244006381
19.5  0.08825610899419256
19.6  0.08687349951943824
19.7  0.0855626387808388
19.8  0.08431921046953428
19.9  0.08313919186425561
20. 0.08201883089356954
20.1  0.08095462533117209
20.2  0.07994330388906246
20.3  0.07898180900204727
20.4  0.07806728112214567
20.5  0.07719704436349667
20.6  0.07636859335763602
20.7  0.0755795811958781
20.8  0.07482780835025193
20.9  0.07411121247728958
21. 0.07342785902015775
21.1  0.07277593253438552
21.2  0.07215372867094622
21.3  0.07155964675784915
21.4  0.07099218292786397
21.5  0.07044992374562786
21.6  0.06993154029229598
21.7  0.06943578267020656
21.8  0.06896147489378007
21.9  0.06850751013618156
22. 0.06807284630418192
22.1  0.06765650191620232
22.2  0.06725755226079799
22.3  0.0668751258148319
22.4  0.0665084009023878
22.5  0.06615660257704742
22.6  0.065818999711588
22.7  0.06549490228044297
22.8  0.06518365882140487
22.9  0.06488465406410665
23. 0.06459730671375852
23.1  0.06432106737947163
23.2  0.06405541663729653
23.3  0.06379986321880882
23.4  0.06355394231674188
23.5  0.06331721399976836
23.6  0.06308926172908674
23.7  0.06286969096998674
23.8  0.06265812789204106
23.9  0.06245421815200786
24. 0.06225762575393867
24.1  0.062068031981362126
24.2  0.06188513439676496
24.3  0.061708645903918724
24.4  0.06153829386889936
24.5  0.061373819295932996
24.6  0.061214976054462245
24.7  0.06106153015406873
24.8  0.0609132590641156
24.9  0.06076995107519434
25. 0.06063140469963696
25.1  0.060497428108560125
25.2  0.060367838603070545
25.3  0.0602424621174123
25.4  0.06012113275199732
25.5  0.06000369233439745
25.6  0.059889990006493976
25.7  0.05977988183611871
25.8  0.059673230451616266
25.9  0.059569904697870324
26. 0.05946977931243652
26.1  0.05937273462050424
26.2  0.05927865624750844
26.3  0.05918743484827582
26.4  0.05909896585167951
26.5  0.05901314921983042
26.6  0.05892988922090625
26.7  0.05884909421477458
26.8  0.05877067645062167
26.9  0.058694551875854595
27. 0.05862063995558381
27.1  0.05854886350204791
27.2  0.05847914851337843
27.3  0.058411424021139524
27.4  0.05834562194612036
27.5  0.058281676961885964
27.6  0.05821952636562672
27.7  0.05815910995587387
27.8  0.05810036991668277
27.9  0.05804325070789623
28. 0.05798769896114281
28.1  0.05793366338123551
28.2  0.05788109465265545
28.3  0.057829945350836076
28.4  0.05778016985796758
28.5  0.05773172428307194
28.6  0.0576845663861017
28.7  0.0576386555058383
28.8  0.057593952491380274
28.9  0.057550419637021843
29. 0.057508020620328226
29.1  0.0574667204432428
29.2  0.05742648537605105
29.3  0.05738728290405254
29.4  0.05734908167678876
29.5  0.057311851459696805
29.6  0.057275563088052595
29.7  0.057240188423084704
29.8  0.057205700310146546
29.9  0.057172072538833496
30. 0.05713927980494771
30.1  0.05710729767421219
30.2  0.057076102547647056
30.3  0.05704567162851548
30.4  0.05701598289076992
30.5  0.05698701504891018
30.6  0.05695874752919344
30.7  0.05693116044211827
30.8  0.056904234556128586
30.9  0.05687795127246692
31. 0.056852292601129895
31.1  0.05682724113786358
31.2  0.056802780042155165
31.3  0.0567788930161644
31.4  0.056755564284558246
31.5  0.05673277857519904
31.6  0.056710521100646616
31.7  0.05668877754043811
31.8  0.056667534024105515
31.9  0.05664677711489868
32. 0.05662649379418083
32.1  0.05660667144646098
32.2  0.05658729784504046
32.3  0.05656836113824237
32.4  0.05654984983619277
32.5  0.05653175279813849
32.6  0.056514059220267365
32.7  0.056496758624011434
32.8  0.05647984084481855
32.9  0.05646329602135891
33. 0.05644711458515589
33.1  0.05643128725062235
33.2  0.05641580500548021
33.3  0.056400659101549566
33.4  0.056385841045890506
33.5  0.0563713425922846
33.6  0.05635715573303668
33.7  0.056343272691087354
33.8  0.0563296859124221
33.9  0.056316388058763434
34. 0.05630337200053548
34.1  0.05629063081008784
34.2  0.05627815775517009
34.3  0.05626594629264564
34.4  0.056253990062431536
34.5  0.05624228288166527
34.6  0.056230818739074706
34.7  0.056219591789554824
34.8  0.0562085963489386
34.9  0.05619782688895225
35. 0.056187278032350574
35.1  0.056176944548223695
35.2  0.056166821347469864
35.3  0.05615690347842539
35.4  0.05614718612264624
35.5  0.056137664590837466
35.6  0.05612833431892339
35.7  0.056119190864250545
35.8  0.05611022990192238
35.9  0.05610144722125987
36. 0.0560928387223783
36.1  0.056084400412884206
36.2  0.05607612840468168
36.3  0.0560680189108832
36.4  0.05606006824282772
36.5  0.056052272807195824
36.6  0.0560446291032185
36.7  0.056037133719981314
36.8  0.0560297833338157
36.9  0.056022574705772876
37. 0.05601550467918481
37.1  0.05600857017730032
37.2  0.05600176820099956
37.3  0.05599509582658002
37.4  0.055988550203616094
37.5  0.05598212855288631
37.6  0.055975828164362584
37.7  0.0559696463952698
37.8  0.055963580668199675
37.9  0.0559576284692912
38. 0.05595178734646
38.1  0.055946054907689724
38.2  0.05594042881937261
38.3  0.05593490680470309
38.4  0.055929486642119386
38.5  0.055924166163795364
38.6  0.05591894325417629
38.7  0.0559138158485597
38.8  0.05590878193172172
38.9  0.05590383953658046
39. 0.055898986742904444
39.1  0.055894221676056455
39.2  0.05588954250577732
39.3  0.05588494744500394
39.4  0.05588043474872478
39.5  0.05587600271286686
39.6  0.055871649673217005
39.7  0.05586737400437597
39.8  0.055863174118740366
39.9  0.055859048465516814
40. 0.05585499552976303
40.1  0.055851013831459406
40.2  0.0558471019246044
40.3  0.0558432583963373
40.4  0.05583948186608667
40.5  0.055835770984744015
40.6  0.05583212443385738
40.7  0.055828540924852676
40.8  0.055825019198273707
40.9  0.05582155802304528
41. 0.055818156195756474
\end{filecontents}

\begin{tikzpicture}
  \begin{axis}[
      xlabel = {Separation (\si{\nano\meter})},
      ylabel = {$\rho_{00}(t = \SI{10}{\pico\second})$},
    ]
    \addplot[smooth, very thick] table [x index = {0}, y index = {1}]{screening.dat};
  \end{axis}
\end{tikzpicture}

  \caption{\label{fig:screening} Screening effect in a two-dot system.
    For the same incident pulse, quantum dots nearby each other experience very little excitation.
  }
\end{figure}

\Cref{fig:frame comparison} details the trajectory of a quantum dot in an $n=32$ particle simulation under the effects of an external ``$\pi$-pulse'' (a pulse constructed to rotate a wavefunction by $\pi$ radians on the Bloch sphere) in fixed and rotating reference frames.
The fixed-frame solution contains a very small, very high-frequency oscillating term that the rotating frame does not; on timescales significant to the evolution of $\rho_{00}$ (and thus the radiation dynamics) this oscillation all but completely vanishes which facilitates a rotating-frame solution in 160 timesteps (compared with a fixed-frame solution in 59959).

\Cref{fig:echo} shows the results of a two-pulse experiment.
Two pulses delayed by $\tau$ strike a collection of 64 nearly resonant quantum dots.
As each dot has a slightly different value of $\omega_0$, the initial pulse dephases the system by evolving the $\hat{\rho}$ at slightly different rates.
The second pulse, carefully constructed to undo the dephasing, results in an experimentally detectable flash at time $t = 2\tau$.

Finally, \cref{fig:decoherence} depicts the population distribution for $128$ homogeneous quantum dots after exposure to a variety of incident pulses.
The integrated pulse angle (analogus to the $\pi$-pulse mentioned earlier) measures the intensity of the incident light.
As each quantum dot absorbs and emits radiation, some dots feel a stronger field than others and therefore excite more than others which leads to the broadening of the population distribution in each column.
Without interactions, each dot would excite by exactly the same amount and the population curve in \cref{fig:decoherence} would follow a cosine.
Additionally, \cref{fig:screening} gives some explanation of the extreme outliers that appear in several of the columns in \cref{fig:decoherence}.
A system evolved with two dots very close to each other has very little change in the overall population though the effect diminishes rapidly with distance.
The outliers in \cref{fig:decoherence} have a small separaration relative to the rest of the distribution, thus they see an efffect similar to that of \cref{fig:screening} (though we note the effect becomes much more complex than simply screening an external field for many-body systems).


\section{Conclusions}
Here we have developed the tools required for a simulation of optically-active quantum dots.
In describing the coupling between particles, a semiclassical formulation allows us to treat fields as classical objects thereby reducing the complexity of the simulation while simultaneously offering a highly accurate integral equation methodology for propagating radiation throughout the system.
Additionally, a global rotating-wave approximation \emph{dramatically} reduces the frequency content of the systems under consideration which allows for timesteps several orders of magnitude larger than those available in a fixed-frame model.
We have shown the numerical validity of such an approximation as well as several features of results obtained with our model for moderate assemblages of dots.
In the near future, we hope to expand our research efforts in this area to include accelerated computational techniques---such as an Adaptive Integral\cite{Bleszynski1996} or Fast Multipole method\cite{Greengard1987}---for rotating-frame systems as well as phenomena arising from disorder (such as Anderson localization and pulse fragmentation effects).


\bibliography{random_lasing_reflist}{}
\bibliographystyle{plain}

\end{document}
