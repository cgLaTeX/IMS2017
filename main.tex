\documentclass[conference]{IEEEtran}

%\usepackage{array}
%\usepackage{cite}
\usepackage{amsmath}
\usepackage{cleveref}
\usepackage{filecontents}
\usepackage{graphicx}
\usepackage{paralist}
\usepackage{pgfplots}
\usepackage{physics}
\usepackage{siunitx}

\usepackage{microtype}

\hyphenation{op-tical net-works semi-conduc-tor}

\begin{document}
\title{Towards a Self-Consistent Integral-Equation Model of Optically Active Media}

\author{
  \IEEEauthorblockN{
    Connor Glosser\IEEEauthorrefmark{1}\IEEEauthorrefmark{2},
    B.\ Shanker\IEEEauthorrefmark{2}, and
    Carlo Piermarocchi\IEEEauthorrefmark{1}
  }
  \IEEEauthorblockA{\IEEEauthorrefmark{1}Department of Physics \& Astronomy\\
    Michigan State University, East Lansing, Michigan 48824}
  \IEEEauthorblockA{\IEEEauthorrefmark{1}Department of Electrical \& Computer Engineering\\
    Michigan State University, East Lansing, Michigan 48824}
}

% use for special paper notices
%\IEEEspecialpapernotice{(Invited Paper)}

\maketitle

\begin{abstract}
Conventional models of electromagnetic systems give rise to interesting phenomena through prescription of static boundary conditions.
  While these boundary conditions sufficiently describe passive scattering within systems, they fail to account for richer behavior in systems with dynamics.
  Such processes have proven essential in countless technological applications motivating the need for an efficient fullwave Maxwell solver that couples to an underlying description of the system behavior.

  Here we consider a disordered system of interacting quantum dots---nanoscale semiconductors with wide applicability in systems ranging from lasing to quantum computing to biological contrast imaging and next-generation displays.
  Much like atoms, individual quantum dots facilitate absorptive and emissive processes at specific frequencies over timescales independent of those in the incident radiation.
  These processes couple between dots due to the presence of electromagnetic fields, giving rise to emergent nonlinear behavior within the system.
  By treating quantum dots \emph{semiclassically} within our simulation, we maintain the discrete dynamics inherent to quantum objects without resorting to cumbersome second quantization to describe electromagnetic fields (i.e.\ fields behave classically).
  This has the advantage of partitoning the simulation into two distinct parts,
  \begin{inparaenum}[(1)]
  \item determination of source wavefunctions through evolution of the differential Liouville equations, and \label{enum:step 1}
  \item evaluation of radiation patterns through well-known computational electromagnetics techniques. \label{enum:step 2}
  \end{inparaenum}
  We employ a highly-tuned predictor-corrector integration scheme to advance the source wavefunctions in time; the subsequent polarizations that arise then serve as pointlike sources to electromagnetic integral equations (chosen to facilitate accurate point-to-point communication of fields without the computational overhead of a ``radiation grid'').
  The coupled solution of \cref{enum:step 1} and \cref{enum:step 2}, then, produces a complete description of both the quantum and electromagnetic dynamics at each timestep, giving rise to lasing effects and other optical phenomena.
\end{abstract}

\IEEEpeerreviewmaketitle

\section{Introduction}


\section{Problem Statement}
We wish to model the behavior of $N$ interacting quantum dots immersed in a homogeneous background medium amidst monochromatic radiation of frequency $\omega_L$.
Due to the bandlimitedness of the incident radiation, we approximate each quantum dot as a two-level system with transition frequency $\omega_0 \approx \omega_L$; as such, a density matrix formulation fully describes the time-evolution of each level as well as the coherences between levels that give rise to radiation physics.

The density matrix for a quantum dot, $\hat{\rho}$, evolves according to the Liouville-von Neumann equation
\begin{equation}
  i \hbar \pdv{\hat{\rho}}{t} = \commutator{\hat{\mathcal{H}}(t)}{\hat{\rho}} - \hat{\mathcal{D}}\qty[\hat{\rho}]
  \label{eq:liouville}
\end{equation}
where $\hat{\mathcal{H}}$ and $\hat{\mathcal{D}}$ denote the quantum dot's Hamiltonian and dissipator operators, respectively.
Here, $\hat{\mathcal{H}}(t)$ gives rise to the internal twol-level strucuture of the dot as well as an explicit coupling to an external electric field.
In contrast $\hat{\mathcal{D}}$ models the aggregate effect of processes that destroy coherences in the quantum dot system through assumed interactions with an external environment (such as spontaneous emission or phonon losses).
As matrices,
\begin{subequations}
  \begin{align}
    \hat{\mathcal{H}}(t) &\equiv \mqty(0 & \hbar \chi(t) \\ \hbar \chi^*(t) & \hbar \omega_0) \label{eq:hamiltonian} \\
    \hat{\mathcal{D}}[\hat{\rho}] &\equiv \mqty((\rho_{00} + 1)/T_2 & \rho_{01}/T_1 \\ \rho_{10}/T_1 & \rho_{11}/T_2). \label{eq:dissipator}
  \end{align}
\end{subequations}
where $\chi(t) \equiv \vb{d} \cdot \vb{E}/\hbar$, $\vb{d} \equiv \matrixel{0}{\hat{\vb{d}}}{1}$, and $\hat{\vb{d}}$ represents the dipole moment operator $\hat{\vb{d}} \equiv -e \hat{\vb{r}}$. 
Formally, $\chi(t)$ should contain $\hat{\vb{d}}$ and $\hat{\vb{E}}$ \emph{operators} to couple the quantum dot to a quantized electric field of photons.
By considering systems with a sufficiently high field intensity so as to ignore single-photon effects, however, we may instead treat $\vb{E}$ as a classical object arising from conventional electromagnetic sources.
Moreover, if $a$ characterizes the physical size of a quantum dot and $\vb{E}$ contains 


The evolution of \cref{eq:liouville} necessarily produces radiation as the quantum dot absorbs and re-radiates light.
Defining a classical polarization $\vb{P}$ from a set of dots each having a location $\vb{r}_i$, dipole moment $\hat{\vb{d}}_i$, and density matrix $\hat{\rho}_i$,
\begin{equation}
  \begin{aligned}
    \vb{P} & \equiv \sum_i \Tr[\hat{\vb{d}}_i \hat{\vb{\rho}}_i]\delta(\vb{r} - \vb{r}_i) \\
    & \equiv \sum_i 2\vb{d}_i \Re(\rho^{\qty(i)}_{01})\delta(\vb{r} - \vb{r}_i)
  \end{aligned}
  \label{eq:polarization}
\end{equation}
we may compute the electric fied at any other point in space through the dyadic electric field Green's function
\begin{equation}
  \begin{gathered}
  \vb{E}(\vb{r}, t) = \vb{E}_0(\vb{r}, t) - \frac{\mu_0}{4\pi} \int 
      \qty(I - \bar{\vb{r}}\bar{\vb{r}}) \frac{\ddot{\vb{P}}(\vb{r}', t_R)}{\abs{\vb{r} - \vb{r}'}} + \\ 
      \qty(I - 3\bar{\vb{r}}\bar{\vb{r}}) \frac{c \dot{\vb{P}}(\vb{r}', t_R)}{\abs{\vb{r} - \vb{r}'}^2} + 
      \qty(I - 3\bar{\vb{r}}\bar{\vb{r}}) \frac{c^2 \vb{P}(\vb{r}', t_R)}{\abs{\vb{r} - \vb{r}'}^3}  
  \dd[3]{\vb{r}'}
  \end{gathered}
\end{equation}
where $\bar{\vb{r}} = \vb{r} - \vb{r}'/\abs{\vb{r} - \vb{r}'}$ and $t_R = t - \abs{\bar{\vb{r}}}/c$.

\section{Solution of \Cref{eq:liouville}}
\subsection{The Rotating Wave Approximation}
The Hamiltonian detailed in \cref{eq:hamiltonian} fully describes the inner workings of a two-level system as well as its coupling to an external field, however it does so very stiffly. 
The systems under consideration contain quantum dots with a transition frequency well into the optical frequency range ($\omega_0 \sim \SI{1500}{\milli \eV}/\hbar$) while $\chi(t)$ contains a maximum frequency roughly a thousand times smaller. 


\section{Results}

\begin{figure}
  \centering
  \begin{filecontents}{screening.dat}
5.  0.9999969708869827
5.1 0.999996584341406
5.2 0.9999961578473717
5.3 0.9999956881502219
5.4 0.999995171796191
5.5 0.999994605124185
5.6 0.9999939842564228
5.7 0.9999933050890679
5.8 0.9999925632825548
5.9 0.9999917542513345
6.  0.9999908731534877
6.1 0.9999899148794416
6.2 0.99998887404059
6.3 0.9999877449571545
6.4 0.9999865216453998
6.5 0.9999851978045754
6.6 0.9999837668024254
6.7 0.9999822216610511
6.8 0.9999805550406343
6.9 0.9999787592233136
7.  0.9999768260956522
7.1 0.9999747471297896
7.2 0.9999725133637841
7.3 0.999970115380397
7.4 0.9999675432844903
7.5 0.9999647866788086
7.6 0.999961834637908
7.7 0.999958675680068
7.8 0.9999552977372679
7.9 0.999951688122596
8.  0.9999478334945633
8.1 0.9999437198185304
8.2 0.999939332325774
8.3 0.9999346554666029
8.4 0.9999296728610068
8.5 0.9999243672441174
8.6 0.9999187204046813
8.7 0.9999127131198783
8.8 0.9999063250797342
8.9 0.9998995348074716
9.  0.9998923195666556
9.1 0.9998846552569831
9.2 0.9998765162982495
9.3 0.9998678754990696
9.4 0.9998587039109255
9.5 0.9998489706403029
9.6 0.9998386426080649
9.7 0.9998276842575762
9.8 0.9998160571842911
9.9 0.9998037196272389
10. 0.9997905958321476
10.1  0.9997766948289732
10.2  0.9997619289495905
10.3  0.9997462310636236
10.4  0.9997295212427062
10.5  0.9997117013551837
10.6  0.9996926453545989
10.7  0.9996721842075371
10.8  0.9996500851540611
10.9  0.9996260207814038
11. 0.9995995167611674
11.1  0.9995698663057124
11.2  0.9995360064460355
11.3  0.9994963557109903
11.4  0.9994486017629833
11.5  0.999389403130976
11.6  0.9993139541137627
11.7  0.9992153811177747
11.8  0.9990839890177108
11.9  0.9989064160348181
12. 0.9986647350591142
12.1  0.9983354512296423
12.2  0.9978882505404906
12.3  0.9972843455975199
12.4  0.9964743931021984
12.5  0.9953961747859807
12.6  0.9939724036263209
12.7  0.992108995891713
12.8  0.9896938818995537
12.9  0.9865960096359075
13. 0.9826638640533641
13.1  0.9777228860040961
13.2  0.9715718642761402
13.3  0.9639796756397302
13.4  0.954685203315259
13.5  0.9434039173476882
13.6  0.9298433393740776
13.7  0.913726143821084
13.8  0.8948157275445259
13.9  0.8729384991550126
14. 0.8480025051116789
14.1  0.8200195308270347
14.2  0.7891372275622932
14.3  0.7556736578217932
14.4  0.7201306563908075
14.5  0.6831648073189316
14.6  0.6455180104740881
14.7  0.6079320386478734
14.8  0.5710745348966658
14.9  0.5354914299902651
15. 0.5015872296815752
15.1  0.4696273712903291
15.2  0.43975485771289563
15.3  0.4120139310011689
15.4  0.3863751774770793
15.5  0.3627584471700098
15.6  0.3410517996823668
15.7  0.3211260348212959
15.8  0.302845163119715
15.9  0.28607352498267324
16. 0.2706803368850458
16.1  0.25654236606822256
16.2  0.2435453030880707
16.3  0.23158426483635058
16.4  0.22056374186519068
16.5  0.21039720967755937
16.6  0.2010065531203306
16.7  0.19232140221142352
16.8  0.18427844218127615
16.9  0.1768207362147587
17. 0.16989708313129664
17.1  0.16346142160938754
17.2  0.15747228577800482
17.3  0.15189231280624887
17.4  0.1466878006501658
17.5  0.14182831276244062
17.6  0.13728632593100454
17.7  0.1330369172153108
17.8  0.1290574860153914
17.9  0.12532750752275723
18. 0.1218283140900071
18.1  0.11854290137192308
18.2  0.11545575640758532
18.3  0.11255270511528931
18.4  0.10982077695201137
18.5  0.10724808474365805
18.6  0.10482371792100248
18.7  0.10253764760014988
18.8  0.10038064212749753
18.9  0.09834419186966098
19. 0.09642044217091461
19.1  0.09460213352646007
19.2  0.09288254813116914
19.3  0.09125546206205387
19.4  0.08971510244006381
19.5  0.08825610899419256
19.6  0.08687349951943824
19.7  0.0855626387808388
19.8  0.08431921046953428
19.9  0.08313919186425561
20. 0.08201883089356954
20.1  0.08095462533117209
20.2  0.07994330388906246
20.3  0.07898180900204727
20.4  0.07806728112214567
20.5  0.07719704436349667
20.6  0.07636859335763602
20.7  0.0755795811958781
20.8  0.07482780835025193
20.9  0.07411121247728958
21. 0.07342785902015775
21.1  0.07277593253438552
21.2  0.07215372867094622
21.3  0.07155964675784915
21.4  0.07099218292786397
21.5  0.07044992374562786
21.6  0.06993154029229598
21.7  0.06943578267020656
21.8  0.06896147489378007
21.9  0.06850751013618156
22. 0.06807284630418192
22.1  0.06765650191620232
22.2  0.06725755226079799
22.3  0.0668751258148319
22.4  0.0665084009023878
22.5  0.06615660257704742
22.6  0.065818999711588
22.7  0.06549490228044297
22.8  0.06518365882140487
22.9  0.06488465406410665
23. 0.06459730671375852
23.1  0.06432106737947163
23.2  0.06405541663729653
23.3  0.06379986321880882
23.4  0.06355394231674188
23.5  0.06331721399976836
23.6  0.06308926172908674
23.7  0.06286969096998674
23.8  0.06265812789204106
23.9  0.06245421815200786
24. 0.06225762575393867
24.1  0.062068031981362126
24.2  0.06188513439676496
24.3  0.061708645903918724
24.4  0.06153829386889936
24.5  0.061373819295932996
24.6  0.061214976054462245
24.7  0.06106153015406873
24.8  0.0609132590641156
24.9  0.06076995107519434
25. 0.06063140469963696
25.1  0.060497428108560125
25.2  0.060367838603070545
25.3  0.0602424621174123
25.4  0.06012113275199732
25.5  0.06000369233439745
25.6  0.059889990006493976
25.7  0.05977988183611871
25.8  0.059673230451616266
25.9  0.059569904697870324
26. 0.05946977931243652
26.1  0.05937273462050424
26.2  0.05927865624750844
26.3  0.05918743484827582
26.4  0.05909896585167951
26.5  0.05901314921983042
26.6  0.05892988922090625
26.7  0.05884909421477458
26.8  0.05877067645062167
26.9  0.058694551875854595
27. 0.05862063995558381
27.1  0.05854886350204791
27.2  0.05847914851337843
27.3  0.058411424021139524
27.4  0.05834562194612036
27.5  0.058281676961885964
27.6  0.05821952636562672
27.7  0.05815910995587387
27.8  0.05810036991668277
27.9  0.05804325070789623
28. 0.05798769896114281
28.1  0.05793366338123551
28.2  0.05788109465265545
28.3  0.057829945350836076
28.4  0.05778016985796758
28.5  0.05773172428307194
28.6  0.0576845663861017
28.7  0.0576386555058383
28.8  0.057593952491380274
28.9  0.057550419637021843
29. 0.057508020620328226
29.1  0.0574667204432428
29.2  0.05742648537605105
29.3  0.05738728290405254
29.4  0.05734908167678876
29.5  0.057311851459696805
29.6  0.057275563088052595
29.7  0.057240188423084704
29.8  0.057205700310146546
29.9  0.057172072538833496
30. 0.05713927980494771
30.1  0.05710729767421219
30.2  0.057076102547647056
30.3  0.05704567162851548
30.4  0.05701598289076992
30.5  0.05698701504891018
30.6  0.05695874752919344
30.7  0.05693116044211827
30.8  0.056904234556128586
30.9  0.05687795127246692
31. 0.056852292601129895
31.1  0.05682724113786358
31.2  0.056802780042155165
31.3  0.0567788930161644
31.4  0.056755564284558246
31.5  0.05673277857519904
31.6  0.056710521100646616
31.7  0.05668877754043811
31.8  0.056667534024105515
31.9  0.05664677711489868
32. 0.05662649379418083
32.1  0.05660667144646098
32.2  0.05658729784504046
32.3  0.05656836113824237
32.4  0.05654984983619277
32.5  0.05653175279813849
32.6  0.056514059220267365
32.7  0.056496758624011434
32.8  0.05647984084481855
32.9  0.05646329602135891
33. 0.05644711458515589
33.1  0.05643128725062235
33.2  0.05641580500548021
33.3  0.056400659101549566
33.4  0.056385841045890506
33.5  0.0563713425922846
33.6  0.05635715573303668
33.7  0.056343272691087354
33.8  0.0563296859124221
33.9  0.056316388058763434
34. 0.05630337200053548
34.1  0.05629063081008784
34.2  0.05627815775517009
34.3  0.05626594629264564
34.4  0.056253990062431536
34.5  0.05624228288166527
34.6  0.056230818739074706
34.7  0.056219591789554824
34.8  0.0562085963489386
34.9  0.05619782688895225
35. 0.056187278032350574
35.1  0.056176944548223695
35.2  0.056166821347469864
35.3  0.05615690347842539
35.4  0.05614718612264624
35.5  0.056137664590837466
35.6  0.05612833431892339
35.7  0.056119190864250545
35.8  0.05611022990192238
35.9  0.05610144722125987
36. 0.0560928387223783
36.1  0.056084400412884206
36.2  0.05607612840468168
36.3  0.0560680189108832
36.4  0.05606006824282772
36.5  0.056052272807195824
36.6  0.0560446291032185
36.7  0.056037133719981314
36.8  0.0560297833338157
36.9  0.056022574705772876
37. 0.05601550467918481
37.1  0.05600857017730032
37.2  0.05600176820099956
37.3  0.05599509582658002
37.4  0.055988550203616094
37.5  0.05598212855288631
37.6  0.055975828164362584
37.7  0.0559696463952698
37.8  0.055963580668199675
37.9  0.0559576284692912
38. 0.05595178734646
38.1  0.055946054907689724
38.2  0.05594042881937261
38.3  0.05593490680470309
38.4  0.055929486642119386
38.5  0.055924166163795364
38.6  0.05591894325417629
38.7  0.0559138158485597
38.8  0.05590878193172172
38.9  0.05590383953658046
39. 0.055898986742904444
39.1  0.055894221676056455
39.2  0.05588954250577732
39.3  0.05588494744500394
39.4  0.05588043474872478
39.5  0.05587600271286686
39.6  0.055871649673217005
39.7  0.05586737400437597
39.8  0.055863174118740366
39.9  0.055859048465516814
40. 0.05585499552976303
40.1  0.055851013831459406
40.2  0.0558471019246044
40.3  0.0558432583963373
40.4  0.05583948186608667
40.5  0.055835770984744015
40.6  0.05583212443385738
40.7  0.055828540924852676
40.8  0.055825019198273707
40.9  0.05582155802304528
41. 0.055818156195756474
\end{filecontents}

\begin{tikzpicture}
  \begin{axis}[
      xlabel = {Separation (\si{\nano\meter})},
      ylabel = {End population ($\rho_{00}$)},
    ]
    \addplot[smooth, thick] table [x index = {0}, y index = {1}]{screening.dat};
  \end{axis}
\end{tikzpicture}

  \caption{\label{fig:screening} Screening effect in a two-dot system.
  Immersed in a ``$\pi$-pulse'' designed to excite a single quantum dot from $\ket{0}$ to $\ket{1}$ ($\rho_{00} = 1$ to $\rho_{00} = 0$, a rotation of $\pi$ radians around the Bloch sphere), the interaction between between two dots gives rise to a significant screening effect at short distances.
  Here, $\rho_{00}$ does not approach zero asymptotically at large separation values due to the action of the dissipator in \cref{eq:liouville}.
  }
\end{figure}
\end{document}
