\documentclass[conference]{IEEEtran}

%\usepackage{array}
%\usepackage{cite}
\usepackage{amsmath}
\usepackage{cleveref}
\usepackage{filecontents}
\usepackage{graphicx}
\usepackage{paralist}
\usepackage{pgfplots}
\usepackage{physics}
\usepackage{siunitx}

\usepackage{microtype}

%\usetikzlibrary{external}
%\tikzexternalize[prefix=build/tikz/]

\hyphenation{band-limitedness op-tical net-works semi-conduc-tor}

\begin{document}
\title{A Self-Consistent Integral-Equation Framework for Optically Active Media}

%\author{
  %\IEEEauthorblockN{
    %Connor Glosser\IEEEauthorrefmark{1}\IEEEauthorrefmark{2},
    %B.\ Shanker\IEEEauthorrefmark{2}, and
    %Carlo Piermarocchi\IEEEauthorrefmark{1}
  %}
  %\IEEEauthorblockA{\IEEEauthorrefmark{1}Department of Physics \& Astronomy\\
    %Michigan State University, East Lansing, Michigan 48824}
  %\IEEEauthorblockA{\IEEEauthorrefmark{1}Department of Electrical \& Computer Engineering\\
    %Michigan State University, East Lansing, Michigan 48824}
%}

% use for special paper notices
%\IEEEspecialpapernotice{(Invited Paper)}

\maketitle

\begin{abstract}
Here we consider a disordered system of interacting quantum dots---nanoscale semiconductors with wide applicability in systems ranging from lasing to quantum computing to biological contrast imaging and next-generation displays.
Quantum dots facilitate absorptive and emissive processes at specific frequencies over timescales independent of those in the incident radiation; by treating the system \emph{semiclassically} we maintain the discrete dynamics inherent to quantum objects without resorting to second quantization to describe electromagnetic fields (i.e.\ fields behave classically).
Our solution to the coupled system proceeds via
  \begin{inparaenum}[(1)]
  \item determination of source wavefunctions through evolution of the differential Liouville equations, and \label{enum:step 1}
  \item evaluation of radiation patterns through well-known computational electromagnetics techniques. \label{enum:step 2}
  \end{inparaenum}
We employ a highly-tuned predictor-corrector integration scheme to advance the source wavefunctions in time; the subsequent polarizations that arise then serve as point-like sources within time domain integral equations electromagnetic integral equations that is used to propagate the field.
The coupled solution of (1) and (2) produces a complete description of both the quantum and electromagnetic dynamics at each timestep giving rise to lasing effects, non-linear propagation, coupled Rabi oscillations, and other optical phenomena.
This paper gives a presentation of the theory, computational framework, and interesting physical observations for such a system.
\end{abstract}

\IEEEpeerreviewmaketitle

\section{Introduction}
Conventional models of electromagnetic systems give rise to interesting phenomena through prescription of static boundary conditions.
While these boundary conditions sufficiently describe passive scattering within systems, they fail to account for richer behavior in systems with dynamics.
Such behaviors have proven essential in countless technological applications motivating the need for an efficient fullwave Maxwell solver that couples to an underlying description of the system behavior.

\section{Problem Statement}

\subsection{The quantum Liouville-von Neumann equation}
We wish to model the behavior of $N$ interacting quantum dots immersed in a homogeneous background medium amidst monochromatic radiation of frequency $\omega_L$. Due to the bandlimitedness of the incident radiation, we approximate each quantum dot as a two-level system with transition frequency $\omega_0 \approx \omega_L$.
As such, a density matrix formulation fully describes the time-evolution of the population in each level as well as the coherences between levels that give rise to radiation physics.

The density matrix for a quantum dot, $\hat{\rho}$, evolves according to the Liouville-von Neumann equation
\begin{equation}
  i \hbar \pdv{\hat{\rho}}{t} = \commutator{\hat{\mathcal{H}}(t)}{\hat{\rho}} - \hat{\mathcal{D}}\qty[\hat{\rho}]
  \label{eq:liouville}
\end{equation}
where $\hat{\mathcal{H}}$ and $\hat{\mathcal{D}}$ denote the quantum dot's Hamiltonian and dissipator operators, respectively\cite{Breuer2002}.
Here, $\hat{\mathcal{H}}(t)$ describes the  two-level structure of the dot as well as an explicit coupling to an external electric field.
In contrast $\hat{\mathcal{D}}$ models the aggregate effect of processes that destroy coherences and relax populations in the quantum dot system (such as spontaneous emission or phonon losses) through assumed interactions with an external environment.
As matrices,
\begin{subequations}
  \begin{align}
    \hat{\mathcal{H}}(t) &\equiv \mqty(0 & \hbar \chi(t) \\ \hbar \chi^*(t) & \hbar \omega_0) \label{eq:hamiltonian} \\
    \hat{\mathcal{D}}[\hat{\rho}] &\equiv \mqty((\rho_{00} + 1)/T_2 & \rho_{01}/T_1 \\ \rho_{10}/T_1 & \rho_{11}/T_2) \label{eq:dissipator}
  \end{align}
\end{subequations}
where $\chi(t) \equiv \vb{d} \cdot \vb{E}/\hbar$, $\vb{d} \equiv \matrixel{0}{\hat{\vb{d}}}{1}$, $\hat{\vb{d}}$ represents the dipole moment operator $\hat{\vb{d}} \equiv -e \hat{\vb{r}}$, and $T_{1,2}$ denote characteristic relaxation times (determined empirically).
Physically, $\chi(t)$ should contain $\hat{\vb{E}}$ as an \emph{operator} to couple the quantum dot to a quantized electric field of photons; by considering systems with a sufficiently high field intensity so as to ignore single-photon effects, however, we instead treat $\vb{E}$ as a classical object arising from conventional electromagnetic source distributions in a manner consistent with the semiclassical nature of our approach.

\subsection{Determination of electric fields}

The evolution of \cref{eq:liouville} necessarily produces radiation as the quantum dot absorbs and re-radiates light.
Using the density matrix formulation detailed above, we may define a classical polarization as
\begin{equation}
  \begin{aligned}
    \vb{P}(\vb{r}, t) &\equiv \sum_i \Tr[\hat{\vb{d}}_i \hat{\rho}_i(t)] \delta(\vb{r} - \vb{r}_i) \\
                      &\equiv \sum_i 2 \vb{d}_i \Re\!\qty[\rho_{01}^{(i)}(t)] \delta(\vb{r} - \vb{r}_i)
  \end{aligned}
  \label{eq:polarization}
\end{equation}
for a set of dots each having a location $\vb{r}_i$, dipole moment $\vb{d}_i$, and off-diagonal matrix element $\rho_{01}^{(i)}(t)$.
From this we may compute the electric field at any other point in space via the the electric dyadic Green's function\cite{Rothwell2008}
\begin{equation}
  \begin{gathered}
    \vb{E}(\vb{r}, t) = \vb{E}_0^\text{rot}(\vb{r}, t) - \frac{\mu_0}{4\pi} \int
      \qty(I - 3\bar{\vb{r}}\bar{\vb{r}}) \frac{c^2 \vb{P}(\vb{r}', t_R)}{\abs{\vb{r} - \vb{r}'}^3} + \\
      \qty(I - 3\bar{\vb{r}}\bar{\vb{r}}) \frac{c \dot{\vb{P}}(\vb{r}', t_R)}{\abs{\vb{r} - \vb{r}'}^2} +
      \qty(I -  \bar{\vb{r}}\bar{\vb{r}}) \frac{\ddot{\vb{P}}(\vb{r}', t_R)}{\abs{\vb{r} - \vb{r}'}}
  \end{gathered}
  \label{eq:efield}
\end{equation}
where $\bar{\vb{r}} \equiv \vb{r} - \vb{r}'/\abs{\vb{r} - \vb{r}'}$ and $t_R \equiv t - \abs{\vb{r} - \vb{r}'}/c$.
These fields then couple back into the evolution of \cref{eq:liouville} through the definition of $\chi(t)$ in \cref{eq:hamiltonian}.

\section{Solution Methodology}
\subsection{Discretization and time-evolution}

To solve both \cref{eq:liouville,eq:efield} simultaneously we must first choose an appropriate discretization for $\hat{\rho}$ in both space and time.
As the physical dimensions of these nanostructures remain negligible when compared to the wavelengths present in radiated fields, $\vb{E}(\vb{r}, t)$ varies little throughout the volume occupied by a dot thus we may approximate each as a $\delta$-function in space.
To represent the temporal part of $\hat{\rho}$ we make use of an interpolatory set of Lagrange polynomials.
Such a set has two advantages: it easily accounts for nonintegral delay times in $t_R$ by interpolating $\vb{P}(\vb{r}, t)$ between timepoints and it facilitates straightforward evaluation of the polarization derivatives in \cref{eq:efield}.

This discretization, then, suggests a straightforward solution scheme in two iterated steps.
We first use \cref{eq:efield} to evaluate $\vb{E}(\vb{r}, t)$ from a given source distribution at the location of each dot and then step \cref{eq:rotating liouville} forward in time to update the source distribution.
The first step---evaluating \cref{eq:efield}---proceeds straightforwardly save for pre-calculating the spatial terms and interpolation weights at the start of the simulation.
Numerically integrating \cref{eq:liouville}, however, requires some care.
For this, we make use of a highly-tuned predictor/corrector ($PE(CE)^m$) algorithm detailed in \cite{Glaser2009}.
By approximating the solution to the differential equation as a weighted set of (complex) exponentials we recover the oscillating and decaying modes in $\hat{\rho}(t)$ with a high degree of accuracy without incurring a significant penalty in the size of our timestep $\Delta t$.
Moreover, repeated iterations of the corrector-evaluator step ($CE^m$) facilitate mutual interactions of quantum dots within one timestep.
As a result, the timestep becomes unconstrained by the time it takes information to traverse the space between quantum dots and we need only consider the frequencies in the system in choosing $\Delta t$.

\subsection{The rotating-wave approximation}

\Cref{eq:liouville,eq:efield} sufficiently detail the physics of semiclassical radiation processes in quantum dot systems.

In the systems we wish to investigate, $\omega_0 \gg \chi(t)$. By expanding the commutator in \cref{eq:liouville} for $\chi(t) \in \real$,
\begin{equation}
  \commutator{\hat{\mathcal{H}}}{\hat{\rho}} =
  \mqty(
    -2 i \chi \Im[\rho_{01}] & \chi\qty(1 - 2 \rho_{00}) - \omega_0 \rho_{01} \\
    \omega_0 \rho_{01}^* - \chi\qty(1 - 2\rho_{00}) & 2 i \chi \Im[\rho_{01}]
  ),
\end{equation}
it becomes apparent $\rho_{00}$ contains only low-frequency terms proportional to $\chi(t)$ while $\rho_{01}$ varies much more quickly due to the dependence on $\omega_0$.
By introducing a time-dependent unitary transformation $\hat{\rho}_\text{rot} = \hat{U}\hat{\rho}\hat{U}^\dagger$ where
\begin{equation}
  \hat{U} = \mqty(\dmat{1, e^{i \omega_L t}}),
  \label{eq:unitary transformation}
\end{equation}
we may write a rotating version of \cref{eq:liouville} as
\begin{equation}
  i \hbar \pdv{\hat{\rho}_\text{rot}}{t} = \commutator{\hat{U} \hat{\mathcal{H}} \hat{U}^\dagger - \hat{U} \, \partial_t \hat{U}^\dagger}{\hat{\rho}_\text{rot}} - \hat{\mathcal{D}}[\hat{\rho}_\text{rot}].
  \label{eq:rotating liouville}
\end{equation}
Defining $\hat{\mathcal{H}}_\text{rot} \equiv \hat{U} \hat{\mathcal{H}} \hat{U}^\dagger - \hat{U} \, \partial_t \hat{U}^\dagger$ as the rotating-frame Hamiltonian and assuming $\chi(t) \sim \cos(\omega_L t)$, the matrix elements in \cref{eq:rotating liouville} will contain frequencies proportional to $\omega_0 \pm \omega_L$.
As $\omega_0 \approx \omega_L$, the high-frequency terms will approximately integrate to zero over any appreciable timescale and we may safely ignore them, instead considering only the envelope function of $\chi(t)$. This forms the basis of the so-called Rotating Wave Approximation (RWA)\cite{Allen1987}.


The rotating wave approximation introduced in \cref{eq:rotating liouville} reduces the computational cost of our active media model dramatically, though constructing an appropriate solution scheme still requires some care.


\section{Results}
\Cref{fig:screening}

\begin{figure}
  \centering
  \begin{filecontents}{screening.dat}
5.  0.9999969708869827
5.1 0.999996584341406
5.2 0.9999961578473717
5.3 0.9999956881502219
5.4 0.999995171796191
5.5 0.999994605124185
5.6 0.9999939842564228
5.7 0.9999933050890679
5.8 0.9999925632825548
5.9 0.9999917542513345
6.  0.9999908731534877
6.1 0.9999899148794416
6.2 0.99998887404059
6.3 0.9999877449571545
6.4 0.9999865216453998
6.5 0.9999851978045754
6.6 0.9999837668024254
6.7 0.9999822216610511
6.8 0.9999805550406343
6.9 0.9999787592233136
7.  0.9999768260956522
7.1 0.9999747471297896
7.2 0.9999725133637841
7.3 0.999970115380397
7.4 0.9999675432844903
7.5 0.9999647866788086
7.6 0.999961834637908
7.7 0.999958675680068
7.8 0.9999552977372679
7.9 0.999951688122596
8.  0.9999478334945633
8.1 0.9999437198185304
8.2 0.999939332325774
8.3 0.9999346554666029
8.4 0.9999296728610068
8.5 0.9999243672441174
8.6 0.9999187204046813
8.7 0.9999127131198783
8.8 0.9999063250797342
8.9 0.9998995348074716
9.  0.9998923195666556
9.1 0.9998846552569831
9.2 0.9998765162982495
9.3 0.9998678754990696
9.4 0.9998587039109255
9.5 0.9998489706403029
9.6 0.9998386426080649
9.7 0.9998276842575762
9.8 0.9998160571842911
9.9 0.9998037196272389
10. 0.9997905958321476
10.1  0.9997766948289732
10.2  0.9997619289495905
10.3  0.9997462310636236
10.4  0.9997295212427062
10.5  0.9997117013551837
10.6  0.9996926453545989
10.7  0.9996721842075371
10.8  0.9996500851540611
10.9  0.9996260207814038
11. 0.9995995167611674
11.1  0.9995698663057124
11.2  0.9995360064460355
11.3  0.9994963557109903
11.4  0.9994486017629833
11.5  0.999389403130976
11.6  0.9993139541137627
11.7  0.9992153811177747
11.8  0.9990839890177108
11.9  0.9989064160348181
12. 0.9986647350591142
12.1  0.9983354512296423
12.2  0.9978882505404906
12.3  0.9972843455975199
12.4  0.9964743931021984
12.5  0.9953961747859807
12.6  0.9939724036263209
12.7  0.992108995891713
12.8  0.9896938818995537
12.9  0.9865960096359075
13. 0.9826638640533641
13.1  0.9777228860040961
13.2  0.9715718642761402
13.3  0.9639796756397302
13.4  0.954685203315259
13.5  0.9434039173476882
13.6  0.9298433393740776
13.7  0.913726143821084
13.8  0.8948157275445259
13.9  0.8729384991550126
14. 0.8480025051116789
14.1  0.8200195308270347
14.2  0.7891372275622932
14.3  0.7556736578217932
14.4  0.7201306563908075
14.5  0.6831648073189316
14.6  0.6455180104740881
14.7  0.6079320386478734
14.8  0.5710745348966658
14.9  0.5354914299902651
15. 0.5015872296815752
15.1  0.4696273712903291
15.2  0.43975485771289563
15.3  0.4120139310011689
15.4  0.3863751774770793
15.5  0.3627584471700098
15.6  0.3410517996823668
15.7  0.3211260348212959
15.8  0.302845163119715
15.9  0.28607352498267324
16. 0.2706803368850458
16.1  0.25654236606822256
16.2  0.2435453030880707
16.3  0.23158426483635058
16.4  0.22056374186519068
16.5  0.21039720967755937
16.6  0.2010065531203306
16.7  0.19232140221142352
16.8  0.18427844218127615
16.9  0.1768207362147587
17. 0.16989708313129664
17.1  0.16346142160938754
17.2  0.15747228577800482
17.3  0.15189231280624887
17.4  0.1466878006501658
17.5  0.14182831276244062
17.6  0.13728632593100454
17.7  0.1330369172153108
17.8  0.1290574860153914
17.9  0.12532750752275723
18. 0.1218283140900071
18.1  0.11854290137192308
18.2  0.11545575640758532
18.3  0.11255270511528931
18.4  0.10982077695201137
18.5  0.10724808474365805
18.6  0.10482371792100248
18.7  0.10253764760014988
18.8  0.10038064212749753
18.9  0.09834419186966098
19. 0.09642044217091461
19.1  0.09460213352646007
19.2  0.09288254813116914
19.3  0.09125546206205387
19.4  0.08971510244006381
19.5  0.08825610899419256
19.6  0.08687349951943824
19.7  0.0855626387808388
19.8  0.08431921046953428
19.9  0.08313919186425561
20. 0.08201883089356954
20.1  0.08095462533117209
20.2  0.07994330388906246
20.3  0.07898180900204727
20.4  0.07806728112214567
20.5  0.07719704436349667
20.6  0.07636859335763602
20.7  0.0755795811958781
20.8  0.07482780835025193
20.9  0.07411121247728958
21. 0.07342785902015775
21.1  0.07277593253438552
21.2  0.07215372867094622
21.3  0.07155964675784915
21.4  0.07099218292786397
21.5  0.07044992374562786
21.6  0.06993154029229598
21.7  0.06943578267020656
21.8  0.06896147489378007
21.9  0.06850751013618156
22. 0.06807284630418192
22.1  0.06765650191620232
22.2  0.06725755226079799
22.3  0.0668751258148319
22.4  0.0665084009023878
22.5  0.06615660257704742
22.6  0.065818999711588
22.7  0.06549490228044297
22.8  0.06518365882140487
22.9  0.06488465406410665
23. 0.06459730671375852
23.1  0.06432106737947163
23.2  0.06405541663729653
23.3  0.06379986321880882
23.4  0.06355394231674188
23.5  0.06331721399976836
23.6  0.06308926172908674
23.7  0.06286969096998674
23.8  0.06265812789204106
23.9  0.06245421815200786
24. 0.06225762575393867
24.1  0.062068031981362126
24.2  0.06188513439676496
24.3  0.061708645903918724
24.4  0.06153829386889936
24.5  0.061373819295932996
24.6  0.061214976054462245
24.7  0.06106153015406873
24.8  0.0609132590641156
24.9  0.06076995107519434
25. 0.06063140469963696
25.1  0.060497428108560125
25.2  0.060367838603070545
25.3  0.0602424621174123
25.4  0.06012113275199732
25.5  0.06000369233439745
25.6  0.059889990006493976
25.7  0.05977988183611871
25.8  0.059673230451616266
25.9  0.059569904697870324
26. 0.05946977931243652
26.1  0.05937273462050424
26.2  0.05927865624750844
26.3  0.05918743484827582
26.4  0.05909896585167951
26.5  0.05901314921983042
26.6  0.05892988922090625
26.7  0.05884909421477458
26.8  0.05877067645062167
26.9  0.058694551875854595
27. 0.05862063995558381
27.1  0.05854886350204791
27.2  0.05847914851337843
27.3  0.058411424021139524
27.4  0.05834562194612036
27.5  0.058281676961885964
27.6  0.05821952636562672
27.7  0.05815910995587387
27.8  0.05810036991668277
27.9  0.05804325070789623
28. 0.05798769896114281
28.1  0.05793366338123551
28.2  0.05788109465265545
28.3  0.057829945350836076
28.4  0.05778016985796758
28.5  0.05773172428307194
28.6  0.0576845663861017
28.7  0.0576386555058383
28.8  0.057593952491380274
28.9  0.057550419637021843
29. 0.057508020620328226
29.1  0.0574667204432428
29.2  0.05742648537605105
29.3  0.05738728290405254
29.4  0.05734908167678876
29.5  0.057311851459696805
29.6  0.057275563088052595
29.7  0.057240188423084704
29.8  0.057205700310146546
29.9  0.057172072538833496
30. 0.05713927980494771
30.1  0.05710729767421219
30.2  0.057076102547647056
30.3  0.05704567162851548
30.4  0.05701598289076992
30.5  0.05698701504891018
30.6  0.05695874752919344
30.7  0.05693116044211827
30.8  0.056904234556128586
30.9  0.05687795127246692
31. 0.056852292601129895
31.1  0.05682724113786358
31.2  0.056802780042155165
31.3  0.0567788930161644
31.4  0.056755564284558246
31.5  0.05673277857519904
31.6  0.056710521100646616
31.7  0.05668877754043811
31.8  0.056667534024105515
31.9  0.05664677711489868
32. 0.05662649379418083
32.1  0.05660667144646098
32.2  0.05658729784504046
32.3  0.05656836113824237
32.4  0.05654984983619277
32.5  0.05653175279813849
32.6  0.056514059220267365
32.7  0.056496758624011434
32.8  0.05647984084481855
32.9  0.05646329602135891
33. 0.05644711458515589
33.1  0.05643128725062235
33.2  0.05641580500548021
33.3  0.056400659101549566
33.4  0.056385841045890506
33.5  0.0563713425922846
33.6  0.05635715573303668
33.7  0.056343272691087354
33.8  0.0563296859124221
33.9  0.056316388058763434
34. 0.05630337200053548
34.1  0.05629063081008784
34.2  0.05627815775517009
34.3  0.05626594629264564
34.4  0.056253990062431536
34.5  0.05624228288166527
34.6  0.056230818739074706
34.7  0.056219591789554824
34.8  0.0562085963489386
34.9  0.05619782688895225
35. 0.056187278032350574
35.1  0.056176944548223695
35.2  0.056166821347469864
35.3  0.05615690347842539
35.4  0.05614718612264624
35.5  0.056137664590837466
35.6  0.05612833431892339
35.7  0.056119190864250545
35.8  0.05611022990192238
35.9  0.05610144722125987
36. 0.0560928387223783
36.1  0.056084400412884206
36.2  0.05607612840468168
36.3  0.0560680189108832
36.4  0.05606006824282772
36.5  0.056052272807195824
36.6  0.0560446291032185
36.7  0.056037133719981314
36.8  0.0560297833338157
36.9  0.056022574705772876
37. 0.05601550467918481
37.1  0.05600857017730032
37.2  0.05600176820099956
37.3  0.05599509582658002
37.4  0.055988550203616094
37.5  0.05598212855288631
37.6  0.055975828164362584
37.7  0.0559696463952698
37.8  0.055963580668199675
37.9  0.0559576284692912
38. 0.05595178734646
38.1  0.055946054907689724
38.2  0.05594042881937261
38.3  0.05593490680470309
38.4  0.055929486642119386
38.5  0.055924166163795364
38.6  0.05591894325417629
38.7  0.0559138158485597
38.8  0.05590878193172172
38.9  0.05590383953658046
39. 0.055898986742904444
39.1  0.055894221676056455
39.2  0.05588954250577732
39.3  0.05588494744500394
39.4  0.05588043474872478
39.5  0.05587600271286686
39.6  0.055871649673217005
39.7  0.05586737400437597
39.8  0.055863174118740366
39.9  0.055859048465516814
40. 0.05585499552976303
40.1  0.055851013831459406
40.2  0.0558471019246044
40.3  0.0558432583963373
40.4  0.05583948186608667
40.5  0.055835770984744015
40.6  0.05583212443385738
40.7  0.055828540924852676
40.8  0.055825019198273707
40.9  0.05582155802304528
41. 0.055818156195756474
\end{filecontents}

\begin{tikzpicture}
  \begin{axis}[
      xlabel = {Separation (\si{\nano\meter})},
      ylabel = {End population ($\rho_{00}$)},
    ]
    \addplot[smooth, thick] table [x index = {0}, y index = {1}]{screening.dat};
  \end{axis}
\end{tikzpicture}

  \caption{\label{fig:screening} Screening effect in a two-dot system.
  Immersed in a ``$\pi$-pulse'' designed to excite a single quantum dot from $\ket{0}$ to $\ket{1}$ ($\rho_{00} = 1$ to $\rho_{00} = 0$, a rotation of $\pi$ radians around the Bloch sphere), the interaction between between two dots gives rise to a significant screening effect at short distances.
  Here, $\rho_{00}$ does not approach zero asymptotically at large separation values due to the action of the dissipator in \cref{eq:liouville}.
  }
\end{figure}

\begin{figure}
  \centering
  \begin{filecontents}{echo.dat}
0.  0.00006118460941923795  -0.000022937795408427953
0.006133950247927229  0.0003282044434537532 -0.0002681399065804046
0.012267900495854458  0.0006319539847176276 -0.0005699639846424397
0.01840185074378169 0.0009675761329659023 -0.0009025630029862798
0.024535800991708916  0.0013461633366004856 -0.0012718823069151398
0.030669751239636146  0.0017745192612752141 -0.00168752584006078
0.03680370148756338 0.0022368629837918852 -0.002131936732534368
0.042937651735490606  0.0027618298306456806 -0.002639433077335105
0.04907160198341783 0.0033293981617384207 -0.0031898342044043176
0.055205552231345066  0.0039626734127889505 -0.0037836791700667626
0.06133950247927229 0.004665849175981548  -0.0044816249340415846
0.06747345272719953 0.005414477094039715  -0.005196620239359659
0.07360740297512676 0.0062743113624993815 -0.0059837741806931235
0.07974135322305398 0.0071669419941363935 -0.006941123283328351
0.08587530347098121 0.008174707546918224  -0.00785393330692725
0.09200925371890843 0.009295733790956643  -0.008929174644464645
0.09814320396683567 0.010431796827600485  -0.010191265100724597
0.10427715421476291 0.011786115238418576  -0.011346054808552731
0.11041110446269013 0.01320347144150243 -0.012836819257089266
0.11654505471061734 0.014653751842549082  -0.014439647165262876
0.12267900495854459 0.01647950093795158 -0.01599019522726918
0.12881295520647182 0.018227411930989732  -0.01793756058159242
0.13494690545439905 0.020088948551698408  -0.019933142267598947
0.14108085570232629 0.022507758356101602  -0.02201955562554118
0.14721480595025352 0.024633154060799214  -0.024501578293959204
0.15334875619818072 0.027177402133113992  -0.026972840875613305
0.15948270644610796 0.030162573719376296  -0.029712989600549404
0.1656166566940352  0.032806550268243966  -0.03276500724262505
0.17175060694196242 0.0361790109713085  -0.03583518897924085
0.17788455718988966 0.03977018394685434 -0.03936299077550867
0.18401850743781686 0.04322997857237249 -0.042917992660630726
0.19628640793367133 0.05170179935251594 -0.05104862575292261
0.2024203581815986  0.05616254893578861 -0.055337612864935666
0.20855430842952583 0.0613340640956221  -0.06027091433752427
0.214688258677453 0.06630477104183495 -0.06497308005366183
0.22082220892538026 0.07193595298773277 -0.07035357806786266
0.2269561591733075  0.0778373446109328  -0.07627164615790152
0.23309010942123468 0.083892887216346 -0.08157269202548549
0.23922405966916194 0.09080999700624469 -0.08832183285317315
0.24535800991708917 0.09731465484723986 -0.09523315620391047
0.2514919601650164  0.10476018619114141 -0.10121298144279103
0.25762591041294364 0.11222707857428359 -0.10988870851715099
0.2698938109087981  0.12905813035560976 -0.12454497287584322
0.27602776115672534 0.13688650318491324 -0.13525026436625462
0.28216171140465257 0.14648485131255035 -0.1435945974777046
0.2882956616525798  0.1567324352725454  -0.15273177416601424
0.29442961190050704 0.1650386579220788  -0.16468645141636198
0.30056356214843427 0.17695326975917133 -0.1741167833941215
0.30669751239636145 0.18822545137128366 -0.1854870926069769
0.3189654128922159  0.2119232148494281  -0.20957932410881328
0.32509936314014315 0.22370657903808647 -0.22286246818496955
0.3312333133880704  0.23430152759279366 -0.2362177437134504
0.34350121388392485 0.26328556944032344 -0.2646158455779388
0.3496351641318521  0.2769353853252722  -0.27800318337947383
0.3557691143797793  0.2946719467764216  -0.29337581126186
0.36803701487563373 0.32409656957029714 -0.3233422600841514
0.39257281586734266 0.3918140562537514  -0.3870190793261864
0.39922272083107024 0.41175791434877096 -0.40325339037696045
0.4058726257947978  0.42828874028524083 -0.4225786991706422
0.4125225307585254  0.44722058908921913 -0.43382848694863024
0.419172435722253 0.4582672600471715  -0.4615851792617994
0.4324722456497082  0.5004737784830585  -0.5035225317098451
0.4457720555771633  0.5446121356433427  -0.5405794723088617
0.45242196054089096 0.565212642888546 -0.5557316019983962
0.4590718655046185  0.5826574687296112  -0.5761164351615683
0.4723716754320737  0.6112820264208068  -0.6138784473150852
0.4790215803958012  0.6326345352299176  -0.6317615779120127
0.4856714853595288  0.6513753430360566  -0.6542859519318298
0.49232139032325645 0.6672117672786018  -0.6721898130793266
0.498971295286984 0.6928744400554564  -0.6908505746477571
0.5056212002507116  0.7105022231189145  -0.7014811569699453
0.5122711052144391  0.7265938287807325  -0.7195529144930918
0.5189210101781667  0.7409443501819963  -0.7227456933017956
0.5255709151418942  0.7487274295806179  -0.74932688753024
0.5322208201056218  0.7628722491523539  -0.7641226469168465
0.5388707250693494  0.7802238450646682  -0.782398112124769
0.545520630033077 0.7890731156893056  -0.7981859460131044
0.5521705349968046  0.813423963355451 -0.8136636166129103
0.5588204399605321  0.8256390968582774  -0.8184719347695228
0.5654703449242597  0.8389869059138242  -0.8315228532905165
0.5721202498879873  0.8470790455465875  -0.8303684455532524
0.5787701548517148  0.8517945232560005  -0.8492473152942416
0.5854200598154424  0.8562774397401409  -0.8602696309189427
0.5920699647791701  0.8714611126955396  -0.8723781681560615
0.5987198697428976  0.8724901910128348  -0.8856174069105514
0.6053697747066252  0.8934768487180137  -0.8953037816698246
0.6120196796703528  0.9006825506024697  -0.8962285108686747
0.6186695846340803  0.9113868446826968  -0.9033075509765903
0.6253194895978079  0.9132849345024845  -0.8991183758538261
0.6319693945615356  0.9154658666230301  -0.9089866740258288
0.6386192995252631  0.9117307295652717  -0.9178020591178502
0.6452692044889907  0.9239875366776553  -0.9233374807052742
0.6519191094527184  0.9185660990002156  -0.9349224887101475
0.6585690144164459  0.9359077630222744  -0.9393676705876591
0.6652189193801734  0.9397771348340267  -0.9382664345206271
0.6718688243439009  0.9489562237368399  -0.9399576863133088
0.6785187293076285  0.9462283625931865  -0.9348235576838306
0.6851686342713561  0.947040181588845 -0.9363188837840543
0.6918185392350839  0.9381625149568988  -0.9449246262064147
0.6984684441988114  0.9472804301188287  -0.9450268146097098
0.7051183491625389  0.9377259096131995  -0.9562791215638429
0.7117682541262664  0.9517810849012822  -0.956910592092213
0.718418159089994 0.9544653453032825  -0.9557319774671849
0.7250680640537215  0.9609124879886818  -0.9535742889072237
0.731717969017449 0.9578370865690741  -0.9488952915022052
0.7383678739811766  0.9582512467705939  -0.9433942292397197
0.7450177789449044  0.9465863285665396  -0.9530538329207947
0.7516676839086319  0.9531299036532793  -0.949488137414424
0.7583175888723594  0.9420582731099361  -0.9610173578611318
0.764967493836087 0.9526096756171683  -0.9595475786937873
0.7716173987998145  0.9558619767851899  -0.9596040773774389
0.7782673037635423  0.9600549076991689  -0.9548856804729006
0.7849172087272699  0.9583016935229868  -0.9513366581635574
0.7915671136909974  0.9587285058391746  -0.9401373208628478
0.7982170186547249  0.9460750167812678  -0.9514328371579904
0.8048669236184525  0.95023489389685  -0.9455930277991079
0.81151682858218  0.9409470844168882  -0.9562089224393077
0.8181667335459077  0.9461494021154363  -0.9551409824633628
0.8243759605723745  0.9499078719907136  -0.9583089448110336
0.8305851875988411  0.9455755185182226  -0.9544386840300854
0.8367944146253079  0.9517198785532798  -0.9545139522804147
0.8430036416517745  0.9502609471674877  -0.951294651700552
0.8492128686782412  0.9507202280050051  -0.9484470891353773
0.8554220957047078  0.9527862714080751  -0.9496883128295194
0.8616313227311745  0.9499725074434925  -0.9399100177799138
0.8678405497576412  0.9528123184132371  -0.9420579066216446
0.8740497767841079  0.9473993786713416  -0.9409982257912147
0.8802590038105745  0.9480544192944007  -0.9290066420082522
0.8864682308370412  0.9423621941626263  -0.9393598941407628
0.892677457863508 0.9404336136394346  -0.9329947454156028
0.8988866848899748  0.9413874686476577  -0.9277063152151721
0.9050959119164416  0.9302287733192035  -0.9377725272541825
0.9113051389429082  0.9345155230367496  -0.9295368108868883
0.9175143659693749  0.9327945558542592  -0.9324695848175886
0.9237235929958415  0.9195488523949769  -0.9368766469110322
0.9299328200223081  0.9321974543175057  -0.9331691272646376
0.9361420470487749  0.9253831672384322  -0.9378168396501015
0.9423512740752417  0.921512677549337 -0.9359694458174852
0.9485605011017085  0.9309699315042537  -0.9365461721694827
0.9547697281281752  0.9242110476638069  -0.9370261643734961
0.9609789551546418  0.9267897551377344  -0.9340988161901758
0.9671881821811085  0.9301693181021481  -0.935552412282463
0.9733974092075751  0.9275657711539115  -0.9301481514127511
0.9796066362340419  0.9319477614153634  -0.9307656372310968
0.9858158632605085  0.928647001370221 -0.9235470128691987
0.9920250902869753  0.9299543058966385  -0.9219949901718296
0.9982343173134421  0.9282825210036161  -0.9235157565566032
1.004443544339909 0.9259931137316605  -0.9105127447503316
1.0106527713663755  0.9252606060454134  -0.9162118482546928
1.0168619983928422  0.9205219685676753  -0.9155439673329597
1.0230712254193088  0.9217575385901063  -0.9001499738030764
1.0292804524457755  0.9123265808765535  -0.9144907638064022
1.0354896794722421  0.9119821854360607  -0.9089024470306322
1.041698906498709 0.9139425990476587  -0.9035204048120927
1.0479081335251759  0.8992562055883784  -0.9138451094628307
1.0541173605516425  0.9076607195254032  -0.9084603008802175
1.0603265875781092  0.9064613383871871  -0.9093336774189483
1.0665358146045758  0.892282228241529 -0.9129918643357187
1.0727450416310425  0.9065100200747686  -0.9108431534009923
1.078954268657509 0.9005654744932041  -0.9145618020171961
1.0851634956839757  0.8966938052300377  -0.9104143156715083
1.0913727227104426  0.9062084428484565  -0.9118025649323104
1.0975819497369095  0.9017636728633612  -0.9088952510137789
1.1037911767633761  0.9028261287247398  -0.9066270352058019
1.1100004037898428  0.9048027756804699  -0.9045371011843254
1.1162096308163094  0.9035877462901168  -0.9000830455020781
1.122418857842776 0.9060431984663077  -0.9013357870443632
1.1286280848692427  0.9015584920990833  -0.8909686367990762
1.1348373118957094  0.903382732680587 -0.8908504931204615
1.1410465389221762  0.8985608837817857  -0.8940834758463687
1.147255765948643 0.8970970729922334  -0.878416780919638
1.1534649929751097  0.8931840046000025  -0.8875354015761275
1.1596742200015764  0.8890992930827235  -0.8874285044557308
1.165883447028043 0.8910641684386646  -0.8738979981700438
1.1720926740545097  0.8799484361359547  -0.887099555371249
1.1783019010809763  0.8801880957086448  -0.8824895013494293
1.184511128107443 0.8841351937884427  -0.8788986230679849
1.1907203551339098  0.8684187257116693  -0.8873106197192736
1.1969295821603767  0.8786862730212576  -0.8834068794450765
1.2031388091868433  0.8780602375122804  -0.8845523904644239
1.20934803621331  0.8674112925503352  -0.885220923896001
1.2155572632397766  0.8787566364242821  -0.8840672631615287
1.221644689880003 0.8741844059256091  -0.8858006210055902
1.2277321165202293  0.8697966956642078  -0.881232623936706
1.2338195431604555  0.8773282952631672  -0.8819258862192672
1.2399069698006817  0.8728401789040096  -0.8798471598929667
1.245994396440908 0.8711341974043397  -0.876778899894911
1.2520818230811341  0.8757146747402953  -0.8786868331902957
1.2581692497213603  0.8714977252495768  -0.8729422774854996
1.2642566763615866  0.8720635904043139  -0.8719272834828561
1.2703441030018128  0.8721077001904208  -0.8707138774392035
1.276431529642039 0.8698525143457017  -0.8654926139772691
1.2886063829224914  0.8679538667114408  -0.8622951653867702
1.2946938095627176  0.8676378598566469  -0.8579139984697367
1.3007812362029438  0.8671553082986795  -0.8614468196159971
1.3068686628431703  0.8633048063880338  -0.8539000878903297
1.3129560894833963  0.8646241450262685  -0.8506308386176613
1.319043516123623 0.8602454848337271  -0.8562430381816953
1.325130942763849 0.8582435378679285  -0.8459576300176772
1.331218369404075 0.8586329197392251  -0.846081785523862
1.3373057960443016  0.852654737196712 -0.8512960494130416
1.3433932226845275  0.8528835335399154  -0.8388571034086252
1.3494806493247538  0.850574642470301 -0.8438442724527951
1.3555680759649802  0.8448029036224604  -0.8466862279085017
1.3616555026052064  0.8474709535894587  -0.8345359699183393
1.3677429292454328  0.8423758760606987  -0.8422475130555667
1.373830355885659 0.837117691720525 -0.8424608525975555
1.379917782525885 0.842345397170226 -0.8341636504504565
1.3860052091661117  0.8344439447857177  -0.8410261229822956
1.3920926358063377  0.830509036119156 -0.838810177497893
1.3981800624465637  0.8374698238836427  -0.8343284387707
1.4042674890867903  0.8271473701276065  -0.839888518258701
1.4103549157270163  0.8280027240198082  -0.8365563766327847
1.6051525682142558  0.7812555150568724  -0.7891249164417888
2.0277689749999572  0.6780629676628562  -0.6840640407995631
2.422181993800963 0.5794178935279127  -0.5700643754083785
2.428860651935529 0.5737638895783916  -0.5733126272057647
2.442217968204661 0.5663898041221831  -0.5706389655655224
2.455575284473793 0.5669205086456733  -0.572021525181843
2.4689326007429253  0.5673817266681503  -0.5656078682917968
2.52904052395402  0.5508774821114724  -0.5415324642505609
2.8496161144131906  0.4678742363014472  -0.4590389145134331
2.8561729721615166  0.463522978842951 -0.46251121849933213
2.869286687658167 0.4595855652389949  -0.45757380153456895
2.8824004031548185  0.4571051614499329  -0.45983220783117756
2.89551411865147  0.45546331961908765 -0.4567116664987003
2.9086278341481204  0.45487691987216056 -0.44897868461200013
2.921741549644772 0.44904600437247194 -0.43949844176990405
2.9348552651414233  0.4424975681998261  -0.44373691304655183
2.9479689806380738  0.4367560037663994  -0.4438807732344057
2.9610826961347256  0.4397057673689818  -0.4391887760319022
2.974196411631376 0.4371631924836927  -0.42913018233622485
2.987310127128028 0.4300636913086958  -0.4296820495209074
3.000423842624679 0.4195615211845428  -0.43072515782619575
3.013537558121331 0.42409072687950417 -0.42916257317666057
3.0266512736179814  0.4246029564418144  -0.42236600697558
3.0397649891146328  0.42059164464205323 -0.41888477491191367
3.052878704611284 0.41151824006293247 -0.4163157291748976
3.2692550103060283  0.3667879805656951  -0.36921603268994196
3.275371190117093 0.3675833061477652  -0.36880626574136616
3.2814873699281577  0.3653713133162675  -0.3657913782575847
3.2876035497392224  0.36522676373057655 -0.36489609305743315
3.293719729550287 0.36375895923514534 -0.3625319739719764
3.3059520891724166  0.36233970340874644 -0.3603805059251599
3.3181844487945455  0.359484589279748 -0.3548883261490146
3.3243006286056103  0.35739746813723683 -0.3556485551273174
3.3304168084166745  0.3556462806435859  -0.3498127366959682
3.3365329882277392  0.354956082657388 -0.3504928173292734
3.3426491680388044  0.3521185478303614  -0.3509869523678131
3.348765347849869 0.35141497538711153 -0.344029454192308
3.354881527660934 0.3488138624786094  -0.3473307819323029
3.3609977074719986  0.3467596330820546  -0.3464879177633248
3.3671138872830633  0.34696416962062876 -0.34030808329923234
3.373230067094128 0.34255925528845277 -0.344431408653206
3.3854624267162574  0.3423660090082017  -0.3385146940984223
3.3976947863383873  0.3375854181580509  -0.33898312704080885
3.4099271459605163  0.33099244049118265 -0.3388503944439796
3.4221595055826457  0.3334962642757079  -0.3356815148147999
3.4343918652047747  0.33174004451423095 -0.33385669636179544
3.446624224826904 0.3261784975790166  -0.33186356253162774
3.4588565844490335  0.32644092840066774 -0.3301947273939954
3.4649727642600987  0.3247904435607693  -0.32807387540967614
3.477205123882228 0.32394783712402325 -0.32544506227263026
3.4894374835043576  0.3229127309285816  -0.3228254169195234
3.501669843126487 0.3216649465957813  -0.32021637471452913
3.513902202748616 0.318747519773694 -0.3155727546981969
3.526134562370745 0.31565550021566463 -0.311217156832802
3.5444831018039396  0.31188386359201403 -0.30572063590277915
3.556715461426069 0.30817466619222844 -0.3074440262950825
3.5689478210481984  0.304590663786169 -0.3050429472612077
3.581180180670328 0.3038239882369119  -0.2990429242631515
3.5934125402924577  0.2991617629231734  -0.2998371860750635
3.605644899914587 0.29369344740992903 -0.2998027523141184
3.6178772595367166  0.2955670725859769  -0.29633040828432944
3.6301096191588464  0.2935286140463573  -0.2951642423462888
3.642341978780976 0.2877543148153931  -0.2938160040602956
3.654574338403105 0.2883949205522345  -0.2926456499380957
3.6606905182141696  0.2864061817094201  -0.29051426215167725
3.6673226527410345  0.28882499392000227 -0.2898276253667259
3.6739547872679 0.288024024117754 -0.286373637193768
3.680586921794765 0.28696497232770374 -0.2855454095084677
3.68721905632163  0.2852618511883515  -0.2787243790849417
3.6938511908484952  0.2808081050353604  -0.28243043390568334
3.70048332537536  0.28002876332113574 -0.28051896975119767
3.707115459902225 0.27920166225010534 -0.28075177106730653
3.7137475944290905  0.27560066152992063 -0.2809211608052252
3.7668046706440106  0.2654990501715301  -0.2715096348041698
3.7800689396977405  0.2678037055045858  -0.2694248869889397
3.793333208751471 0.267047486462723 -0.2633692992612485
3.806597477805201 0.2628583751340031  -0.26205628965685074
3.819861746858931 0.2560906573369722  -0.26229778209156923
3.8331260159126606  0.2580904085967617  -0.2611028379727979
3.846390284966391 0.2581993148529935  -0.25614659327231687
3.8596545540201213  0.2549744100819103  -0.2537798988940226
3.872918823073851 0.2485450162961166  -0.25322975154814237
3.879550957600716 0.2510693264301244  -0.25201315968736543
3.886183092127581 0.24875013264217874 -0.25282876120700437
3.8928152266544465  0.24941885594856533 -0.25108601183357543
3.8994473611813114  0.24953073341297355 -0.24907035378156958
3.9060794957081764  0.24950134873706373 -0.24697163868971533
3.9193437647619067  0.2463034133944274  -0.2415395600758105
3.9259758992887717  0.24158747958051802 -0.2445292975502362
3.9790329755036926  0.2349549282684248  -0.2362560961539158
3.992297244557423 0.2326636302601544  -0.2352055578516913
4.005561513611153 0.23282594991796216 -0.23488991403917375
4.018825782664883 0.23224882677066705 -0.2313873165573005
4.032090051718614 0.2285333311241856  -0.2284387925283663
4.085147127933533 0.2222261683401208  -0.2210881314979242
4.091338584523138 0.22096483190811392 -0.21902368444696357
4.097530041112742 0.22062605200395474 -0.21700181421277076
4.103721497702346 0.21795634605762224 -0.2186870925725673
4.10991295429195  0.21775488737633486 -0.2155085335837295
4.1161044108815545  0.21673069067564227 -0.21598427907077838
4.122295867471158 0.21391769464996724 -0.21638923841065527
4.128487324060763 0.2152122616825677  -0.21427684208330816
4.134678780650367 0.212875249149263 -0.21515858863774415
4.140870237239971 0.21147400559809654 -0.21436145246796606
4.147061693829576 0.21281162285159633 -0.213463325804531
4.15325315041918  0.20957333978161696 -0.21401863938098092
4.159444607008783 0.21045570058714735 -0.21257770845150942
4.165636063598388 0.21051469228340164 -0.212537624322539
4.171827520187992 0.20853893901404102 -0.21139441043961668
4.178018976777596 0.2095736609783041  -0.2107162071156328
4.1842104333672 0.20861054433824563 -0.20990810927932704
4.190401889956805 0.20784528683490908 -0.20856372146040394
4.196593346546409 0.2082730669554529  -0.20835653879943355
4.202784803136013 0.2069687653882569  -0.20615812899321304
4.208976259725617 0.20701538469088435 -0.20572048520149513
4.2151677163152215  0.20577308945809808 -0.20459678782346535
4.221359172904825 0.20523597515391687 -0.2024629256238603
4.227550629494429 0.20432314115609757 -0.20313726824466077
4.233742086084034 0.20310996921113997 -0.200752013119951
4.239933542673638 0.20282256078549124 -0.19965155636333895
4.246124999263242 0.20076273947946388 -0.20068638380197648
4.252316455852847 0.20043950892826823 -0.1972511989733322
4.258507912442451 0.19906774393460244 -0.19845614416907031
4.264699369032055 0.19727853385522176 -0.19833485558138092
4.27089082562166  0.19792316041064517 -0.19594558814176716
4.277082282211263 0.19525936324731313 -0.19745827505245908
4.283273738800868 0.19477034341077748 -0.19640558803275796
4.289465195390472 0.19547017522880397 -0.19546980324475524
4.295656651980075 0.1918478122322802  -0.19635146532744738
4.30184810856968  0.1935785565369208  -0.19508658538279877
4.308039565159285 0.1931249059356089  -0.1950349573597584
4.314231021748888 0.19088550499013565 -0.19432616743057052
4.320422478338493 0.19253922766286338 -0.19376011225605314
4.326613934928098 0.1913527712602415  -0.19335808698108217
4.332805391517701 0.19055436136418905 -0.1921106738868524
4.338996848107306 0.19137287098261552 -0.19209922995481224
4.34518830469691  0.19019472313455238 -0.19034436347698627
4.357571217876118 0.1894377390153388  -0.18852409989906593
4.363762674465723 0.18901611691607917 -0.18726467894141485
4.3699541310553265  0.18852907808996328 -0.1874862252022394
4.37614558764493  0.18740644493896397 -0.18476273806379767
4.382337044234535 0.18730662124290404 -0.1846159802675733
4.3885285008241395  0.18571566664932437 -0.18503741191644607
4.394719957413743 0.18533806254145166 -0.18130741755802265
4.400911414003347 0.1837575822971424  -0.18333022500451002
4.407102870592952 0.18286472804298554 -0.1826844235157156
4.413294327182556 0.18305563822182244 -0.1799430252828939
4.41948578377216  0.18007584361927995 -0.18224748822364673
4.425677240361764 0.18053591897575128 -0.18082691080427973
4.431868696951368 0.18067070480151384 -0.1798951769264286
4.438060153540973 0.17689995177057796 -0.18118875063036965
4.444251610130577 0.17934286407438002 -0.18006789227942263
4.45044306672018  0.17842852057800804 -0.18012484568615686
4.456634523309785 0.17617691731623472 -0.17984563239189108
4.46282597989939  0.1783574750531105  -0.17944264807431715
4.469017436488993 0.17688844314198157 -0.17948398650799202
4.475208893078597 0.17644449762694048 -0.17845689724445574
4.481400349668202 0.1774560093823604  -0.17865764538169934
4.487470005871565 0.17630349709987633 -0.17758368536089347
4.493539662074929 0.17632699530170765 -0.17714349766833296
4.499609318278293 0.17634270732620624 -0.1768926598292405
4.505678974481657 0.17563174163588044 -0.17597614287386876
4.51174863068502  0.17600551004147685 -0.17598591658051704
4.517818286888383 0.1754018652827595  -0.17504554957972063
4.523887943091748 0.17516627310182767 -0.17456968435099252
4.529957599295111 0.17554923014101376 -0.17493232895698083
4.536027255498475 0.17470773921209123 -0.17340306221075152
4.542096911701838 0.17491873021858134 -0.17345194838485112
4.548166567905202 0.17476018615035285 -0.1739147269202328
4.554236224108566 0.1743188213950777  -0.1720814310824918
4.56030588031193  0.1749099527403137  -0.1727181254637372
4.566375536515293 0.1742490059436478  -0.1732211324406214
4.572445192718657 0.17430233726538477 -0.17120726397844563
4.578514848922021 0.1745202821148817  -0.17275789643665088
4.584584505125385 0.17413920729610452 -0.17298687714264602
4.590654161328748 0.17473808216941175 -0.1709223889455345
4.596723817532112 0.17438573874696747 -0.1734452945331593
4.602793473735477 0.1745632543971352  -0.17337334901292853
4.60886312993884  0.1756254515313565  -0.1716886381803521
4.614932786142203 0.17481037522965026 -0.17484720621235658
4.621002442345568 0.17566706045183444 -0.17456618973319402
4.627072098548931 0.17702972292377067 -0.17396034958073886
4.633141754752295 0.1759689228603415  -0.17712185035213185
4.639211410955658 0.17761605336226385 -0.1767703376489631
4.645281067159022 0.17935808879174103 -0.17730145153804594
4.651350723362386 0.17805448142748972 -0.18044932215791135
4.65742037956575  0.18098871932595828 -0.18066467979028164
4.663490035769113 0.18284785203478868 -0.18186806612818
4.669559691972477 0.18128396459318338 -0.18501964818585756
4.675629348175841 0.1857515938030552  -0.18595180598040967
4.681699004379204 0.187748173376513 -0.18789554960829277
4.687768660582568 0.1868345737322986  -0.19092915381972025
4.693838316785932 0.1920705886619888  -0.19262023170317402
4.699907972989295 0.1943026992672301  -0.19551015402994837
4.705977629192659 0.19495863179287795 -0.1981616685680664
4.7120472853960225  0.20016076750016196 -0.20076296825835915
4.718116941599386 0.20272461180525214 -0.2045525742209196
4.72418659780275  0.20496145770027382 -0.20689892761037487
4.730256254006114 0.21019875822427395 -0.21044708596261502
4.7363259102094775  0.21338569168914195 -0.2141350870224814
4.748465222616205 0.2222828377256796  -0.2217133855011154
4.754534878819569 0.22589461026546045 -0.22499766947873479
4.7606045350229325  0.23052028366230926 -0.22905510335295795
4.77274384742966  0.24010096237247075 -0.23723705818220253
4.778813503633024 0.24580588651487337 -0.24244646034696954
4.7848831598363875  0.2508834446410331  -0.24848966588600738
4.790952816039751 0.25579868063487093 -0.25078863702037185
4.797022472243115 0.2623752292295987  -0.2572738536731174
4.8091617846498425  0.2726773887629891  -0.2664405549142444
4.82130109705657  0.28355298075470703 -0.2817799909388836
4.827370753259934 0.2903136019067998  -0.2839220874356467
4.8334404094632974  0.2960715953954747  -0.2925980364014355
4.839510065666661 0.30050510669808056 -0.3002654193521897
4.845579721870025 0.3082643478018373  -0.3030017680605784
4.8577190342767524  0.3171669619159641  -0.3191238083131493
4.869858346683479 0.33068521682935603 -0.3308092763898861
4.8764439576026435  0.3345082947942341  -0.3384374805073201
4.883029568521808 0.3446664929219909  -0.34571652931069213
4.8896151794409715  0.3494351608773626  -0.351005873430806
4.896200790360136 0.35809351116333066 -0.35641862458251417
4.9027864012793 0.36205475887693805 -0.3568792237572411
4.909372012198464 0.3666031247891982  -0.36220449522346476
4.915957623117628 0.3679653707522461  -0.3655055299093021
4.922543234036792 0.37146806251017683 -0.3662125030232219
4.929128844955956 0.3669725192709191  -0.37430038237926944
4.935714455875121 0.37488342455995666 -0.37613713840511664
4.942300066794284 0.37466012437421364 -0.37982125400448086
4.948885677713449 0.37657992547203706 -0.3790053903155881
4.955471288632612 0.37841676127313334 -0.3773291957830561
4.962056899551777 0.3782309217176407  -0.37423318376925807
4.968642510470941 0.37462543620177335 -0.37314308611634084
4.975228121390105 0.3715440104584407  -0.3628977746603125
4.981813732309269 0.3623881018007179  -0.36675171978440346
4.9883993432284335  0.3597322633493045  -0.36131880418322126
4.994984954147597 0.35410712178499076 -0.35803232685804454
5.0015705650667615  0.3456643683734175  -0.352693516233001
5.008156175985925 0.3434144042552986  -0.34696914358680025
5.014741786905089 0.33538254951630103 -0.336966917940676
5.021327397824254 0.3284385491086354  -0.32858164700475995
5.027913008743417 0.3182915527813109  -0.31464072781095676
5.034498619662582 0.3070549809690982  -0.30759435279760555
5.041084230581745 0.2955886658297949  -0.2952587906397715
5.04766984150091  0.2839882002812356  -0.28546785132737296
5.054255452420074 0.2697824405993605  -0.275857221664844
5.067426674258402 0.24671289469271002 -0.2533006302599659
5.080597896096729 0.22467822252623312 -0.22750387787121557
5.087183507015894 0.21337028028683075 -0.21482365568922393
5.093769117935057 0.19987846675021959 -0.19969475153528557
5.106940339773385 0.17388870924600078 -0.1736631499861849
5.11352595069255  0.16040473129185723 -0.1623242690358352
5.120111561611714 0.148097102518639 -0.14948878657938572
5.133282783450042 0.12396890580897837 -0.12641748693437932
5.139868394369206 0.11216559239242169 -0.11544786039157794
5.14645400528837  0.10198358583922758 -0.10439721091793625
5.1530396162075345  0.09042669316695161 -0.09434870740686402
5.159625227126698 0.08203274372692583 -0.08397760224947456
5.1662108380458625  0.07270642989917601 -0.07494770456024197
5.172796448965026 0.06462318903924535 -0.06639508978296346
5.17938205988419  0.058046420940750076  -0.05843508615732328
5.185967670803355 0.052389321778423044  -0.05241857253595831
5.192553281722518 0.04748908015076101 -0.04830095190108717
5.199138892641683 0.0436588595812999  -0.04515667255537642
5.205724503560847 0.04291583964242059 -0.042795439025686555
5.212310114480012 0.04369973760572152 -0.04292820357830898
5.218895725399176 0.044962993507088475  -0.04487416180564207
5.22548133631834  0.04701525982122592 -0.04741089748585024
5.232066947237503 0.05004692035661448 -0.050378210789911046
5.238652558156668 0.05418093457323227 -0.05351163061737026
5.245238169075833 0.05861230680494307 -0.05721964412709193
5.251823779994996 0.062420765005062336  -0.06169654300679271
5.2649950018333245  0.07048083194630282 -0.06951290170691686
5.271580612752489 0.0741531128502964  -0.0733895675475717
5.2781662236716524  0.07785396052600696 -0.07713167473398037
5.29133744550998  0.08458879394789912 -0.08457701093953744
5.2974823784918845  0.08773099803338949 -0.08738976604094514
5.303627311473788 0.08965899864291811 -0.09063183102887568
5.30977224445569  0.09352785027498556 -0.09297781294385224
5.315917177437593 0.09476164691776788 -0.09613371742716506
5.3220621104194965  0.09817188779509486 -0.09822358793818077
5.3282070434014 0.1004083251820644  -0.10052772408384854
5.340496909365205 0.10494199748013484 -0.10486549420852512
5.3466418423471085  0.10571104772671876 -0.10735729354508386
5.352786775329011 0.10810071572971633 -0.10883602614856
5.358931708310914 0.11024320685345426 -0.11074570704078342
5.365076641292818 0.11069429821701866 -0.11246564712475146
5.371221574274722 0.11350148378342495 -0.11392145980232024
5.3773665072566255  0.11450331600835442 -0.11577133315097035
5.383511440238528 0.11538179464732123 -0.1167101110006659
5.389656373220431 0.11786235223309235 -0.11831186914842869
5.395801306202334 0.11852214018343034 -0.11933009932879615
5.4019462391842366  0.11991785818499874 -0.12036364673779174
5.414236105148044 0.12217223678554004 -0.12222327857387462
5.420381038129948 0.12379462504633845 -0.12351137118257699
5.426525971111851 0.1243860560443405  -0.12380397984027257
5.4326709040937535  0.12548428222623445 -0.12468913614688061
5.438815837075657 0.12648016546084204 -0.12581090761454164
5.44496077005756  0.12709246976330235 -0.12560552040298492
5.451105703039462 0.1283691395880973  -0.12689282310785416
5.4572506360213655  0.1285758552102317  -0.12742961584700524
5.46339556900327  0.12951748406325403 -0.12721989125332422
5.469540501985174 0.12994838695638805 -0.12902993667369397
5.475685434967077 0.13042068793499711 -0.12865997879378105
5.481830367948979 0.1313813500388333  -0.12910621842037495
5.4879753009308825  0.13123009882816764 -0.1308873279659413
5.494120233912786 0.1321197876853288  -0.12973823978827703
5.500265166894688 0.13238728387489299 -0.1315155990196195
5.506410099876591 0.13239287589606025 -0.1325358160580421
5.512555032858495 0.13371485191560312 -0.1315838884761912
5.5186999658403995  0.13309282851941764 -0.13384583642231415
5.524844898822303 0.1335955369921897  -0.13407506139712524
5.530989831804205 0.1351455568557914  -0.134037294343108
5.537134764786108 0.13371231457234506 -0.13603453971511262
5.543279697768011 0.13556987285975733 -0.13608711888699496
5.549424630749914 0.13645230157613145 -0.13664175139986043
5.555569563731817 0.13478572704978697 -0.13797916486799702
5.561714496713721 0.13761001299709982 -0.13821076520829922
5.567859429695625 0.1377212919219477  -0.13924087246068187
5.574004362677528 0.1371032250422352  -0.13962026655038617
5.580149295659431 0.13960907181170007 -0.14031555544854932
5.586294228641334 0.13941411867563525 -0.14109234888391087
5.5985840946051395  0.14152108397354762 -0.14227551718682085
5.610873960568947 0.14258370983680047 -0.14281082305542359
5.623163826532754 0.1437645008830973  -0.14330446048197984
5.63545369249656  0.14496770072319184 -0.14365787975704786
5.647743558460366 0.14624736878898476 -0.1453299276460476
5.660033424424172 0.14741096425010705 -0.14580900404266012
5.672323290387977 0.148341714351118 -0.1450613777103675
5.6846131563517845  0.14860599937336877 -0.1471825149475658
5.6912740440494884  0.14892550415288902 -0.14780980461005555
5.6979349317471915  0.14740601143986126 -0.14967463545172205
5.7045958194448945  0.15008449598387266 -0.150515428629943
5.7112567071425975  0.14976471106283665 -0.15176258215708702
5.7179175948403005  0.15212517209595522 -0.15235920083637
5.724578482538005 0.1527443310105578  -0.15172441648841714
5.731239370235709 0.15389732892455027 -0.15235852712015047
5.737900257933412 0.15363210065338576 -0.15183692606542556
5.791187359515042 0.1587581042947767  -0.15671715177677775
5.897761562678297 0.16929586891000103 -0.16714705711751013
6.110909969004811 0.19134874125451465 -0.19072056912924026
6.117449056316274 0.19273038682402122 -0.19069367830757972
6.130527230939199 0.193674759157891 -0.18899308061992698
6.143605405562127 0.19336301391494706 -0.1926331121960422
6.156683580185052 0.19309085009441818 -0.19613988856447667
6.169761754807977 0.19670417646510716 -0.19776864348812545
6.182839929430905 0.19942082061591393 -0.19706145055200136
6.195918104053831 0.20003451131063207 -0.19846709240608637
6.208996278676756 0.1985532861642897  -0.20121646885389535
6.222074453299684 0.20074710254320127 -0.2040719406643525
6.235152627922609 0.20465479664882244 -0.20553031360920773
6.248230802545534 0.2070358791278721  -0.20585493424399395
6.261308977168462 0.2070709669288817  -0.20652003511159703
6.274387151791388 0.20782117073834167 -0.20748317663703572
6.287465326414313 0.20986992183805814 -0.21068939867805087
6.300543501037241 0.21219840993698222 -0.21269175679898641
6.313621675660166 0.21489559258987248 -0.21253058971885547
6.326699850283092 0.21556465702430117 -0.21120465240601274
6.339778024906018 0.21550595566026148 -0.2143370469008006
6.352856199528945 0.21560778015388868 -0.21817099013091593
6.365934374151871 0.21886831032925563 -0.2197606943084973
6.379012548774796 0.22172820770127805 -0.21882844534660065
6.3920907233977236  0.22216976301655925 -0.2199299052741786
6.405168898020649 0.2204510641447499  -0.22337181359132804
6.418247072643575 0.2225672613365592  -0.22656169020826836
6.431325247266502 0.2270134628780085  -0.22764227658078753
6.444403421889428 0.22938475005971987 -0.22789180051512797
6.457481596512353 0.2291213648020875  -0.22901107129082385
6.470559771135281 0.22971246265355835 -0.23048246150479196
6.483637945758206 0.23239015571404903 -0.23362316624141902
6.496716120381132 0.23529553065604727 -0.2354934257559815
6.509794295004059 0.23780055868617503 -0.23509054428575724
6.522872469626985 0.2379905203185351  -0.23408574814365504
6.529411556938447 0.2357695032403468  -0.23754970601420555
6.5416083756868515  0.23832602446580034 -0.23824635772222472
6.553805194435256 0.240320128920272 -0.23930974411229636
6.5660020131836605  0.23899674091859127 -0.24235162076487426
6.578198831932064 0.24093504405134825 -0.24480378046836257
6.590395650680469 0.24478536845761495 -0.24586046536734463
6.602592469428873 0.2452732333644372  -0.2474657610844016
6.614789288177277 0.24633616862366595 -0.24931582968250646
6.626986106925682 0.24956693830459953 -0.25124054102585053
6.639182925674088 0.25192998424147434 -0.2522553994268497
6.651379744422492 0.2529730985748421  -0.25264934032774294
6.6635765631708965  0.2547246168989947  -0.2531359475729192
6.681871791293503 0.25721346668650885 -0.25391061163202167
6.694068610041907 0.25845978155652816 -0.25646271751626043
6.706265428790311 0.25984650282256977 -0.2579727140834133
6.7184622475387155  0.2616596772516895  -0.25606702535009007
6.73065906628712  0.2615958405734213  -0.258843532052052
6.7428558850355245  0.2613745479622902  -0.26241208444893044
6.755052703783928 0.2639489587130843  -0.26224880789814603
6.767249522532333 0.2652141724233648  -0.2634641977977813
6.779446341280737 0.2631780957657011  -0.2671514542677359
6.791643160029142 0.26625482068667045 -0.26912759749019477
6.803839978777545 0.26964638639178495 -0.2701919267103012
6.81603679752595  0.2691701921177487  -0.2723178535803068
6.828233616274356 0.2705275166087 -0.2746846524710887
6.8404304350227605  0.27440313613712813 -0.27657537317741576
6.852627253771165 0.276130810346286 -0.277214288828215
6.8648240725195695  0.2772457540030484  -0.27797932514221324
6.877020891267973 0.2795231489349254  -0.27886499807879583
6.889217710016378 0.2820258423187349  -0.28095056897105336
6.901414528764782 0.28370157630040654 -0.2808236784111412
6.913611347513186 0.28440233334734266 -0.2799552785717979
6.919709756887388 0.28557213531495473 -0.2825837105618217
6.9329384850673925  0.2851597460695765  -0.2818532893242266
6.946167213247398 0.28600529891766846 -0.28632653523269935
6.959395941427404 0.2880882274019701  -0.28952028542761216
6.972624669607408 0.29187525113313806 -0.28894732440891535
6.985853397787413 0.2923083501251324  -0.2864394068876558
6.999082125967418 0.29177432757009825 -0.2910409821940546
7.012310854147424 0.29234177344090534 -0.2953168983047387
7.025539582327429 0.29685688896818946 -0.295720726440276
7.038768310507433 0.29876463703928663 -0.2924436276588678
7.051997038687439 0.297526969971065 -0.2958528454156923
7.065225766867445 0.2964780218998632  -0.3009840273923147
7.078454495047449 0.30154617706412773 -0.3025168906614469
7.091683223227454 0.30463778191341895 -0.29977076142297415
7.104911951407459 0.30379742950973176 -0.30177350252677887
7.118140679587465 0.30073104555504293 -0.3065589255002575
7.13136940776747  0.306134384481727 -0.3092009319906387
7.343029058647552 0.3294785581599586  -0.32690338641668093
7.3492027448002935  0.32453206379468386 -0.3312301951942728
7.355376430953035 0.329858239119757 -0.33063831750610445
7.361550117105777 0.33035250345899625 -0.3319475749621378
7.367723803258518 0.3264432775638332  -0.3338815790281568
7.37389748941126  0.33272316991506534 -0.3343242053053837
7.380071175564002 0.3316711629614458  -0.3366629767430269
7.386244861716744 0.33113861826325375 -0.33600175818083877
7.392418547869485 0.33578173730475946 -0.3375570809264807
7.398592234022227 0.33517612926227003 -0.3376349315272361
7.4047659201749685  0.33631083423703245 -0.3377918866773939
7.41093960632771  0.33847143284313136 -0.33859118446434355
7.417113292480452 0.33878612970485356 -0.3378328867286317
7.423286978633193 0.341067827100439 -0.3392590482151451
7.429460664785935 0.34051644713999324 -0.3373711804504811
7.435634350938677 0.3419794033588109  -0.33780277614538406
7.441808037091419 0.34216150322979616 -0.3400664405268877
7.44798172324416  0.3422171824576628  -0.3361187097922124
7.454155409396902 0.3432141669019734  -0.3388250954346003
7.4603290955496435  0.3422572867669588  -0.34087958104565474
7.466502781702385 0.34360111070336874 -0.3355917845558423
7.472676467855127 0.34202996854350953 -0.34155838375591135
7.478850154007868 0.34210472722659757 -0.34197244718972736
7.48502384016061  0.3444519037764743  -0.33918491275843793
7.491197526313352 0.3407741652003746  -0.3446863888870137
7.497371212466094 0.3429181529691624  -0.34403712350852017
7.503544898618835 0.345296999542567 -0.34407337871511917
7.509718584771577 0.34019281058019435 -0.34789505636852047
7.5158922709243186  0.34567246508250604 -0.3474522340215346
7.52206595707706  0.3464115128050234  -0.3490196457548661
7.528239643229802 0.34345898570485944 -0.3501395781173461
7.534413329382543 0.348859140396345 -0.35075403413404327
7.540587015535285 0.34835063243882797 -0.3524646085405732
7.546760701688027 0.3482563856061364  -0.3517941575412396
7.552934387840769 0.35213876981010733 -0.3534752621404156
7.55910807399351  0.3516693296351541  -0.35268115407995454
7.565281760146252 0.3531395503077957  -0.35309211044439354
7.571455446298994 0.3544098830316415  -0.3534268623208734
7.577629132451735 0.35488906754229177 -0.35228891715304944
7.583802818604477 0.35673363716782835 -0.354109475987861
7.589976504757218 0.355967800609031 -0.3521786475466029
7.5961501909099605  0.3575598874839734  -0.35190151486883375
7.602323877062702 0.3568543551655736  -0.35497617510974966
7.608497563215444 0.357155134036956 -0.3511288661752588
7.614671249368185 0.357709177618881 -0.3535713594847103
7.620844935520927 0.3563319750853028  -0.35601343163946325
7.627018621673669 0.35809511021289836 -0.3513935112681103
7.63319230782641  0.3564971275793894  -0.35667584099732447
7.639365993979152 0.35579935206994023 -0.35737907501833144
7.645539680131893 0.3589924777361641  -0.35557237415680526
7.6517133662846355  0.3554449252827489  -0.36012597001497487
7.657887052437377 0.3572611088559282  -0.35985738511177706
7.664060738590119 0.3600704821352607  -0.36029813710616404
7.67023442474286  0.35544177606285243 -0.36340688403903765
7.676408110895602 0.3604024059250461  -0.36294272865329147
7.682581797048344 0.3614646988869436  -0.3647838276871908
7.688755483201085 0.3595000010484144  -0.36499287094665195
7.694929169353827 0.3639270879439881  -0.3657474250621673
7.701102855506568 0.36381181208058333 -0.3666665183030573
7.7072765416593105  0.3641267673696533  -0.3660833327626106
7.713450227812052 0.36737635057393714 -0.36788286012209576
7.719623913964794 0.36678783279040067 -0.36610289721443356
7.725797600117535 0.3685384357816625  -0.366843020167253
7.731971286270277 0.36888339794266306 -0.36699012939285136
7.738144972423019 0.36949685196657217 -0.36513839504219187
7.7448346132915615  0.3690665607640284  -0.36730289282210093
7.7515242541601035  0.3686762804226488  -0.3627640325078691
7.7582138950286454  0.36628706921484777 -0.3681739167615495
7.764903535897187 0.36613555528500313 -0.3696319796930949
7.7715931767657285  0.3692879337627792  -0.37230590949776937
7.77828281763427  0.36930315859176865 -0.3722454444260249
7.784972458502812 0.37353942786233696 -0.3729472505646421
7.791662099371354 0.3740061763960847  -0.3689955431279779
7.845179226319691 0.3783106331112531  -0.3726496749375017
7.851868867188233 0.3768453565519276  -0.3757281903977486
7.858558508056775 0.3766480058187819  -0.37185984701499586
7.865248148925316 0.37460619591655026 -0.37710739309116725
7.871937789793859 0.37447081681329947 -0.37871620409460865
7.8786274306624025  0.37814726519853264 -0.3814049401457853
7.8853170715309435  0.3787437351049419  -0.3803837018774337
7.8920067123994855  0.3820573020859319  -0.3807352286114542
7.898696353268027 0.38240304434694755 -0.376097290239284
7.952213480216364 0.38627689413412913 -0.3793355338066398
7.958903121084906 0.3837217372170373  -0.38354834733440396
7.965592761953448 0.38398227324959927 -0.3807487193361717
7.97228240282199  0.38235128365520055 -0.38544069993501934
7.978972043690533 0.38220538912992597 -0.3869976604516653
7.9856616845590755  0.3864039736912209  -0.38961329064644445
7.9923513254276175  0.3873932471376187  -0.3876390347211551
7.999040966296159 0.38971740029695323 -0.3878301647621035
8.0057306071647 0.38992635377469587 -0.3823621967761651
8.059247734113036 0.3933463040326599  -0.3851757152559215
8.07262701585012  0.39062091688319944 -0.3889407350904558
8.086006297587204 0.3893116484423027  -0.3944078069289389
8.099385579324288 0.39517460736793836 -0.39396419218597994
8.112764861061372 0.3965323539302319  -0.3877751662048146
8.126144142798456 0.39367554381589104 -0.39275224589922775
8.13952342453554  0.3926206229016645  -0.39776707863725463
8.152902706272625 0.398717463727034 -0.39676619343711805
8.166281988009708 0.39948085605552874 -0.39016026061729275
8.17284982849201  0.39525112541988167 -0.39708369987499414
8.179417668974311 0.396361018856909 -0.39564157049412
8.185985509456613 0.3965963905649595  -0.3983102373853431
8.192553349938914 0.3933954896896353  -0.4006150677316191
8.199121190421215 0.39991251085758317 -0.40230780016815887
8.205689030903518 0.39982270401295833 -0.401253012058267
8.212256871385819 0.4035023304374195  -0.4016713943512358
8.218824711868121 0.4024152062576006  -0.39650498037011545
8.225392552350424 0.4024814985868848  -0.39866185151437517
8.231960392832725 0.4003574520959073  -0.398857491957273
8.238528233315026 0.4013350992330541  -0.39624879117461675
8.245096073797328 0.3939486971458653  -0.40268293559858465
8.251663914279629 0.401130788591136 -0.40285430904367364
8.258231754761931 0.4001443952951394  -0.40573016398750866
8.264799595244233 0.40194290252025305 -0.4046488761149652
8.271367435726534 0.40492614022269663 -0.4037495205355266
8.277935276208835 0.40634193873014796 -0.401887701553657
8.284503116691136 0.40539169147911597 -0.4031995544763587
8.291070957173439 0.4054882043464255  -0.39546093414276007
8.29763879765574  0.4006246701248188  -0.40357301749646457
8.30420663813804  0.4020950080595572  -0.4029953999742474
8.310774478620342 0.4031855856589894  -0.4051296089509974
8.317342319102643 0.39982860973866  -0.40718811031165636
8.323910159584946 0.40662627192602413 -0.40869138727787035
8.330478000067247 0.4066844461745988  -0.4068323590681702
8.337045840549548 0.409670010159023 -0.4071342232399147
8.343613681031849 0.40830093911832976 -0.40197484008953865
8.35018152151415  0.4078746408153697  -0.4040790924754517
8.356749361996453 0.40525817629427635 -0.4051513144545914
8.363317202478754 0.4069095670047314  -0.4029367787853589
8.369885042961055 0.39989182875160345 -0.40906424013181286
8.376452883443358 0.4068402115938589  -0.40915647373128766
8.383020723925661 0.4067343519004554  -0.4113205709475979
8.389588564407962 0.4085258753023945  -0.41000014004800944
8.396156404890263 0.410748516958461 -0.40892644133449313
8.402724245372564 0.41191815130521603 -0.4062681028465902
8.409292085854867 0.4101185038266874  -0.40845801889475036
8.415859926337168 0.4102100197951651  -0.4017452793906733
8.42242776681947  0.4059666904874498  -0.40897928214379753
8.42899560730177  0.4067042734627693  -0.40911601066099185
8.435563447784071 0.4087073662715652  -0.41124764653948664
8.442131288266374 0.40625278281424637 -0.4124796605234051
8.448699128748675 0.41223489005274566 -0.4137285505419821
8.455266969230976 0.4122068230628363  -0.4110030286977275
8.461834809713277 0.4144955143967214  -0.41142965072109766
8.468402650195578 0.412873156536536 -0.40701738066729415
8.474970490677881 0.4122692927309378  -0.4083718384659215
8.481538331160182 0.40883401025069915 -0.4103639477816864
8.488106171642483 0.4113708785257689  -0.4088997322846498
8.494674012124785 0.4052434993697061  -0.41431683412266385
8.501241852607087 0.41144436061724304 -0.4141401568899341
8.507809693089389 0.41208873895131143 -0.4155484359547702
8.51437753357169  0.4138832363984725  -0.4140251895154667
8.52094537405399  0.41526433631226  -0.4129653699188047
8.527513214536292 0.4161747131608878  -0.4093584568922339
8.534081055018595 0.4134958098700666  -0.41260158643333883
8.540648895500897 0.41369120341200516 -0.40722450075026034
8.547216735983199 0.41022092936628646 -0.41326997667823456
8.5537845764655 0.4101711897094307  -0.41397269953283156
8.560352416947802 0.41312482747802143 -0.41604815262026257
8.566920257430104 0.41144649449349413 -0.41644362704960924
8.573488097912405 0.4166949479854323  -0.4174379298929487
8.580055938394706 0.4164079280038931  -0.41382150152971053
8.586623778877007 0.41794308060564156 -0.4145439771651615
8.592750941422048 0.4167046648589799  -0.4126861203822157
8.59887810396709  0.4180307298298059  -0.41036619863279294
8.60500526651213  0.4163752388692775  -0.41459121814772426
8.611132429057172 0.4162729697411271  -0.41104908988320366
8.617259591602211 0.4171284157445765  -0.41094756094073864
8.623386754147253 0.41434599510431513 -0.4148488594587882
8.629513916692293 0.41569295030571135 -0.41001474604999566
8.635641079237333 0.41531655547583934 -0.4129822796739834
8.641768241782373 0.41228382779135725 -0.41534163141413666
8.647895404327414 0.41559243956742137 -0.41196938189909443
8.654022566872454 0.4135399943237366  -0.4154298296399889
8.660149729417496 0.4107007960544017  -0.41607691986048834
8.666276891962537 0.41577507877226805 -0.414786222145765
8.672404054507577 0.4121368106004421  -0.4179657449624652
8.678531217052619 0.4123493327526038  -0.41744933860737904
8.684658379597659 0.4161913482061617  -0.4176416988305139
8.6907855421427 0.4118941336242284  -0.4198820031199368
8.69691270468774  0.41463783746751703 -0.41870654888259734
8.703039867232782 0.41682999940181276 -0.4200985151862226
8.709167029777822 0.4143048099080194  -0.41980715613199815
8.715294192322862 0.4172373914536206  -0.4196791676195099
8.721421354867902 0.41777893857095955 -0.42070793339137297
8.727548517412943 0.41700711587060574 -0.41922330934771634
8.733675679957983 0.41978623548605054 -0.4202292926806255
8.739802842503025 0.4189460526934789  -0.41867122389815875
8.745930005048066 0.41956559258533455 -0.418311044271844
8.752057167593106 0.4208405764066167  -0.4195116139049261
8.758184330138148 0.4199129079832231  -0.41615318552306035
8.764311492683188 0.42155966099384684 -0.4172834788997133
8.77043865522823  0.42037542280029816 -0.41786022306346515
8.77656581777327  0.4205459425122026  -0.4136135448695156
8.78269298031831  0.42063813947832845 -0.4170398716913322
8.788820142863349 0.4194410966226079  -0.4162405668068138
8.79494730540839  0.4207435767631585  -0.41152902847423556
8.80107446795343  0.41815493702701007 -0.4173520443338707
8.807201630498472 0.418255182076246 -0.41493551424248504
8.813328793043512 0.4194171694130087  -0.41283612686641247
8.819455955588554 0.4153683732839471  -0.4179318355422679
8.825583118133595 0.41713011186872556 -0.4143194193372982
8.831710280678635 0.417812645379521 -0.4152345618850646
8.837837443223677 0.4127575299431238  -0.41871072950095584
8.843964605768717 0.41729050263957906 -0.4163307767945544
8.850091768313758 0.41635121505004363 -0.41801463612952466
8.856218930858798 0.41149942131382666 -0.4195768823697376
8.862346093403838 0.41777681080626017 -0.4185894336806079
8.868473255948878 0.41527943345509705 -0.4207562796000599
8.87460041849392  0.4134895310613318  -0.4203136249771985
8.88072758103896  0.4185161146279565  -0.4206866383717471
8.886854743584001 0.41560484102789563 -0.4220607899006116
8.892981906129041 0.41614490784552755 -0.42080686785486343
8.899109068674083 0.4194029238937797  -0.42222957608707207
8.905236231219124 0.4175952771954112  -0.42120979688337773
8.911363393764164 0.4190410238551858  -0.4209802526532682
8.917490556309206 0.4201649769320175  -0.42126188283594085
8.923617718854246 0.4196503313557649  -0.41980920252415826
8.929744881399285 0.4217208038187267  -0.42079002100247426
8.935872043944325 0.4206306135663928  -0.4183001535233946
8.941999206489367 0.42137938806977826 -0.4181346805001193
8.948126369034407 0.42183600512314323 -0.4196521775104457
8.954253531579448 0.4208052745900539  -0.41501945372485644
8.960380694124488 0.4224082651721691  -0.41649446483663033
8.96650785666953  0.4205640062080567  -0.4182180401589634
8.972635019214572 0.42064968402003944 -0.4119281536786869
8.978762181759611 0.41991977000251124 -0.41658114925978945
8.985405299020453 0.4172938479892504  -0.416039796219963
8.992048416281294 0.417875930042924 -0.41447564771566053
8.998691533542134 0.40982358730044477 -0.41998872780545726
9.005334650802975 0.41816053273426473 -0.4204479230368672
9.011977768063817 0.41738469221635177 -0.4210570255291531
9.018620885324658 0.42043485079881315 -0.42012519559820455
9.025264002585498 0.4203234461127336  -0.41548611808357366
9.03190711984634  0.4208712508091798  -0.4155679891168987
9.03855023710718  0.41746114314866284 -0.4163159248672279
9.045193354368022 0.4180182992038393  -0.41242974228289264
9.051836471628862 0.40981299500338586 -0.41926284593838575
9.058479588889703 0.4170229139210474  -0.4192403454953166
9.065122706150545 0.4163017670969508  -0.421096442678025
9.071765823411386 0.41850446217351733 -0.4195736216044739
9.078408940672226 0.4197680022218955  -0.4167569768478715
9.085052057933067 0.42111255744007603 -0.41456938696079343
9.091695175193909 0.41748972646948435 -0.4162888537280171
9.09833829245475  0.4177801169208132  -0.4108348205861367
9.10498140971559  0.4107599904376849  -0.41806169696073275
9.111624526976431 0.4155296343734065  -0.41785028807785957
9.118267644237273 0.4151682773427338  -0.4208354083001111
9.124910761498112 0.41639775660311545 -0.41885766831819227
9.131553878758954 0.41904867362994824 -0.41759649510432273
9.138196996019795 0.41988832322464037 -0.41419046197925247
9.144840113280637 0.4173848210390626  -0.41592520669540967
9.158126347802318 0.41154960764525  -0.4164605159611263
9.171412582324  0.41449567685284905 -0.419090289106731
9.184698816845682 0.4181439035852697  -0.41795652772013636
9.191341934106525 0.41835948240693976 -0.41375810817751085
9.197985051367366 0.4171198327080944  -0.41520662612535064
9.204628168628208 0.4161914540929564  -0.40840791687370304
9.21127128588905  0.41209429153562616 -0.414471804378265
9.217914403149889 0.4116249210198296  -0.4147382809863539
9.22455752041073  0.41370919821311786 -0.4167776261500498
9.231200637671572 0.41183835765130977 -0.416975678473816
9.237843754932411 0.4170183623413693  -0.41780465944678596
9.244486872193253 0.4165945861178617  -0.41325633725951916
9.257773106714936 0.4150083131911188  -0.408470425289817
9.271059341236617 0.4093238911216584  -0.41306043886617044
9.2843455757583 0.40948117104893267 -0.4157961341753818
9.29763181027998  0.4146478788134722  -0.4126622613147779
9.304274927540822 0.41597042875910606 -0.4127064061078888
9.310918044801664 0.41369258550773996 -0.4083763907648845
9.317561162062503 0.41217499940425656 -0.40945543968714004
9.324204279323345 0.4074527129547848  -0.41148274723659295
9.330847396584186 0.4112123293800086  -0.4115690082824494
9.337490513845028 0.40712296884653926 -0.4144255132015701
9.344133631105867 0.41244147305037665 -0.4142397868150196
9.350776748366709 0.4125591548584309  -0.41194627427680064
9.35741986562755  0.41497791488340363 -0.41096669710032296
9.364062982888392 0.4122545803961631  -0.4080466003259765
9.370706100149231 0.41161295175495183 -0.40652554057323453
9.377349217410073 0.40622966973583163 -0.40988202518234546
9.383992334670914 0.4094633483586744  -0.4087773361815467
9.390635451931756 0.4047918322665437  -0.4128229691028122
9.397278569192595 0.40948984371702946 -0.41204349940401697
9.403921686453437 0.4103537512086525  -0.41107229640901444
9.410124125777017 0.41074930707243734 -0.4100782050905538
9.416326565100597 0.4117763670363078  -0.4106058260170739
9.422529004424177 0.41082325315092  -0.4068527250770151
9.428731443747758 0.4122346410217012  -0.4079348338398232
9.434933883071336 0.41012041367074564 -0.40709301644361495
9.441136322394916 0.41064776370591427 -0.4028154357119608
9.447338761718498 0.4085320424776264  -0.4070053803804488
9.453541201042079 0.40786088430500567 -0.40389470730127447
9.459743640365659 0.40826356709086  -0.4024993401611699
9.465946079689239 0.4040894208934028  -0.406532046406228
9.47214851901282  0.4058383733474273  -0.4026029394411858
9.478350958336398 0.4048647039580883  -0.40470738675990753
9.484553397659978 0.400115497530235 -0.4064241018253442
9.490755836983558 0.4050961462388869  -0.4047610030554221
9.496958276307138 0.40193619632687044 -0.40726290478158295
9.503160715630719 0.4008674230022235  -0.40670350039554193
9.509363154954299 0.4048145285747335  -0.40699460682993444
9.515565594277879 0.4015163753145046  -0.40786247054099883
9.521768033601461 0.40331935960686544 -0.40672248824591006
9.52797047292504  0.4048424396156638  -0.40803140681815125
9.534172912248618 0.4036179735719724  -0.4058051191330607
9.5403753515722 0.4059062031166455  -0.4061392746792102
9.54657779089578  0.40497301000822267 -0.40388777629743017
9.55278023021936  0.4055373215497355  -0.40312856997956104
9.55898266954294  0.4056558328409567  -0.4038043641157827
9.56518510886652  0.4047060482917923  -0.3990639761111506
9.571387548190101 0.4053167163026204  -0.40087031198504625
9.577589987513681 0.40325691194097657 -0.40068840610401857
9.583792426837261 0.4038161198827407  -0.39481488027525224
9.589994866160842 0.4007621181976635  -0.4003465563584347
9.596197305484424 0.4003670586871273  -0.39802328311210183
9.602399744808002 0.4011274492373636  -0.39560879470706134
9.60860218413158  0.3958044761035379  -0.4003214151193406
9.614804623455163 0.39845250091475704 -0.3975364126803394
9.621007062778743 0.39817112418927214 -0.39827919018268687
9.627209502102323 0.3918815386028058  -0.40050906156317523
9.633411941425903 0.39814327410822115 -0.399259865982923
9.639614380749483 0.39577213072100687 -0.40106456196286494
9.645816820073064 0.3937012526502991  -0.4002604102387634
9.652019259396644 0.39829070753631113 -0.4007119451555468
9.658221698720224 0.39596322436809034 -0.40069811223978763
9.664424138043804 0.3965782134213444  -0.399556660596598
9.670626577367385 0.3984491821091557  -0.40006674136721926
9.676829016690965 0.3975468793906468  -0.39790757649245684
9.683031456014543 0.3993174237341716  -0.39829352900579423
9.689233895338123 0.39795847168171017 -0.3951050552942352
9.695436334661705 0.39863112734711426 -0.39463101673368745
9.701638773985286 0.3979180821941035  -0.39575443041090247
9.707841213308866 0.3969706971587569  -0.3898071302856157
9.714043652632446 0.39674182204669955 -0.39258095518509023
9.720246091956026 0.39480570866766684 -0.3930850139584303
9.726448531279605 0.39542852802920514 -0.385604087345289
9.732650970603185 0.3914485321122233  -0.392497913115042
9.738853409926765 0.3913778497188162  -0.3909314722681103
9.745055849250345 0.39278341420671237 -0.38761483776203026
9.751258288573926 0.3863074221653367  -0.39288770437446535
9.757460727897506 0.38988156604146246 -0.3909607362292339
9.763663167221086 0.39030141760369713 -0.3908200919903365
9.769865606544668 0.3836123843287262  -0.3931127424432759
9.776068045868247 0.39002280049500565 -0.3921223792825028
9.782270485191825 0.38839397680804144 -0.39360618561463767
9.788472924515407 0.3860470558601282  -0.39224171298904154
9.794675363838987 0.39055076486734225 -0.39274527955660143
9.800877803162567 0.38886188368352326 -0.3919168611934744
9.806958442099907 0.38846527307069373 -0.39090889474049983
9.813039081037248 0.3904394887855307  -0.39173851799997444
9.819119719974587 0.38890634095339455 -0.389554654882841
9.825200358911927 0.3894589169028467  -0.38943186016329123
9.831280997849266 0.38947032018751687 -0.38872109738002364
9.837361636786605 0.3888079202066348  -0.3869657833044126
9.843442275723945 0.3900596814258105  -0.38780608914433523
9.849522914661286 0.388289870658638 -0.3855203524339383
9.855603553598625 0.38845265907198845 -0.3843074899418098
9.861684192535964 0.3883966774027882  -0.3859628558789867
9.867764831473304 0.38690622988384016 -0.38232950444000063
9.873845470410643 0.3877410680344994  -0.3817416835990196
9.879926109347982 0.38605030645565724 -0.38413676926902496
9.886006748285324 0.385340487390344 -0.3793264781219139
9.892087387222663 0.3856849750891377  -0.38022666521164994
9.898168026160002 0.38339925336579844 -0.3824038387312972
9.904248665097342 0.38362650460362996 -0.37667402817644413
9.910329304034681 0.38264815418107273 -0.37964211575860884
9.916409942972022 0.38060253154401913 -0.3808072306357704
9.922490581909361 0.3818124564125518  -0.3750297679147492
9.9285712208467 0.3795217769124756  -0.37931779916726094
9.93465185978404  0.3778236156093606  -0.37937662472085826
9.94073249872138  0.3800054177906271  -0.3753665780405908
9.946813137658719 0.37647754811316525 -0.37916250670579127
9.95289377659606  0.37523024519713827 -0.37812842770516175
9.9589744155334 0.378279256771747 -0.3759736102892623
9.96505505447074  0.37367214836504065 -0.3790733634081007
9.971135693408078 0.3743756336715854  -0.37766461757664616
9.977216332345417 0.37667153518868307 -0.3766530962892387
9.983296971282757 0.37124739177522703 -0.37893611244384284
9.989377610220098 0.3739464808314028  -0.377244651121212
9.995458249157437 0.3752065480157142  -0.37721623069075766
10.001538888094778  0.3704738419236369  -0.37805524822103287
10.007619527032116  0.3737547627507305  -0.376725602594677
10.013700165969455  0.37389555281527415 -0.3774831436435472
10.019780804906796  0.37077073278731065 -0.3765638125571384
10.025861443844136  0.3736929352043037  -0.3760214681808927
10.031942082781475  0.372895233073544 -0.37670604799021423
10.038022721718816  0.3712334125492259  -0.37483045916770147
10.044103360656155  0.37364237625191443 -0.37506007045798007
10.050183999593493  0.3723112716523717  -0.37411903432873433
10.056264638530834  0.3716769511663968  -0.37290089059482645
10.062345277468173  0.37347369273411857 -0.37378284139252055
10.068425916405513  0.3716698156314031  -0.37117706827625
10.074506555342854  0.37192532248853116 -0.37083269230530636
10.080587194280193  0.37201032374617565 -0.3710277965273637
10.08666783321753 0.3708906058474378  -0.36804718367242784
10.092748472154872  0.3718113587780835  -0.3686950492687502
10.098829111092211  0.37025374317291465 -0.3681368408918172
10.104909750029552  0.36990698579002823 -0.36490016203770875
10.110990388966892  0.36994108450330115 -0.36682828741313045
10.117071027904231  0.36826353143785606 -0.3652625655591353
10.12315166684157 0.36866566913409377 -0.3619103227937179
10.12923230577891 0.3670257728600859  -0.3651985950040688
10.135312944716249  0.366102029436205 -0.3625329891224611
10.14139358365359 0.36676128173196454 -0.3598710189708285
10.14747422259093 0.3638345290405601  -0.3637030934176728
10.153554861528269  0.36384287437718177 -0.3600619276495606
10.159635500465608  0.364002274879624 -0.35935213804012733
10.165716139402948  0.36053940276241236 -0.36234100785648404
10.171796778340287  0.36159352073816664 -0.35801811626288893
10.177877417277628  0.3611933189870253  -0.35916394756005005
10.183958056214967  0.35731575774280216 -0.36109941759328695
10.190038695152307  0.35993002577174277 -0.3578034983623352
10.196635288805446  0.35708865470044737 -0.36107303582395645
10.203231882458587  0.35651688104610935 -0.361050277614016
10.209828476111726  0.3592383469821317  -0.36236045673828177
10.216425069764867  0.35877311997475825 -0.35982567976955626
10.223021663418006  0.360564688393038 -0.3593837799709231
10.229618257071147  0.35946304312916105 -0.3538449937180222
10.236214850724286  0.3573769565581891  -0.356077843513862
10.242811444377427  0.3556308864192453  -0.3528729945508471
10.249408038030568  0.3544996046304874  -0.3539493608480133
10.256004631683707  0.34967750711832374 -0.3559780180138356
10.262601225336848  0.35431403073586437 -0.3562301866307785
10.269197818989987  0.35232922174871156 -0.35631493708856804
10.275794412643126  0.35519438267070647 -0.3555648697763096
10.282391006296265  0.3544937657536734  -0.3517556846877346
10.288987599949406  0.35512005558998716 -0.35113422330254623
10.295584193602545  0.3521435206060253  -0.3498552968089408
10.302180787255686  0.3516225903079279  -0.34663820950719015
10.308777380908827  0.3454019271614415  -0.35085687102346436
10.315373974561966  0.34909963709627556 -0.3497690828889926
10.321970568215107  0.3462649186693627  -0.35233987232832464
10.328567161868246  0.3478424958392741  -0.35070362326215687
10.335163755521387  0.34941380013979423 -0.3496994099680351
10.341760349174528  0.3499318416016021  -0.3472018271558237
10.348356942827667  0.3486814780893022  -0.34675798050090073
10.354953536480805  0.3481034666341017  -0.34000466553148173
10.361550130133946  0.34302519910028056 -0.3454321683142781
10.368146723787087  0.3437063491289132  -0.34341822892853646
10.374743317440226  0.3425149747490129  -0.3453104755670904
10.381339911093367  0.34009039552943904 -0.34567556224692464
10.387936504746506  0.34419169791925386 -0.3464877071667726
10.394533098399647  0.34359575988515983 -0.34373554959073915
10.401129692052788  0.3448893459384466  -0.3432661048873128
10.407726285705927  0.343310884531049 -0.3371146540013192
10.414322879359066  0.34095873896538975 -0.33975926987278965
10.420919473012205  0.3389222042262004  -0.3381742423708777
10.427516066665344  0.338946110583628 -0.33740241559235423
10.434112660318485  0.33256850686483785 -0.34054833289564196
10.440709253971624  0.3384456316758585  -0.3404400665338874
10.447305847624765  0.33715549727182326 -0.34015620273302594
10.453902441277904  0.3389501240590572  -0.33891372463435304
10.460499034931045  0.3383447796208471  -0.3348850345620216
10.473692222237325  0.3353533559009722  -0.334264052198272
10.486885409543605  0.32837100696726884 -0.3348807733839644
10.500078596849885  0.3309613302920607  -0.3360232324635085
10.506675190503026  0.3311504025625405  -0.33402595141608293
10.513271784156167  0.33326252543780466 -0.3328958128444322
10.519868377809306  0.33333106292959025 -0.32994113532208447
10.526464971462447  0.33179629980615793 -0.33023618855766307
10.533061565115586  0.3308275749857922  -0.3233706343696267
10.539658158768727  0.3262871080655298  -0.3286650276978939
10.546254752421868  0.326244214784047 -0.3276287383907321
10.552851346075007  0.32653758383139764 -0.32860056855946107
10.559447939728148  0.3235125857710854  -0.32902647977072574
10.566044533381287  0.32778071796306574 -0.3293565719844487
10.572641127034428  0.3270304198265424  -0.326368704420773
10.57923772068757 0.3279132988972094  -0.3259018689590587
10.585834314340708  0.3259811677293056  -0.32072318053865756
10.59243090799385 0.32405198446178657 -0.32223670950211947
10.599027501646988  0.3210909212511998  -0.32211925323052665
10.60562409530013 0.3221096200163703  -0.32090208569132633
10.612220688953267  0.31652176466250254 -0.3238731970153589
10.618376604669146  0.3204384052210999  -0.32259124843329257
10.624532520385024  0.3195870094518939  -0.32326409765162517
10.630688436100904  0.316964572904686 -0.3220678581400251
10.636844351816784  0.32001403451935456 -0.32182058323811297
10.643000267532663  0.3185326727639018  -0.3214195042200219
10.649156183248543  0.31776269145761227 -0.31991923508588915
10.655312098964421  0.31951434013186114 -0.32042854355432654
10.6614680146803  0.31794612183252324 -0.31795037691182243
10.66762393039618 0.3183467709942087  -0.3175164957981901
10.67377984611206 0.31771030732688244 -0.3164743944031132
10.679935761827938  0.31714373796968437 -0.31417596210041915
10.686091677543818  0.31724639561258233 -0.31501014658945753
10.692247593259697  0.31552993147922975 -0.31233452079948926
10.698403508975575  0.31585537089614035 -0.3105644677597422
10.704559424691453  0.3138271327559396  -0.3126136330910319
10.710715340407333  0.31307715338208464 -0.3084862102256769
10.716871256123213  0.3125658711381627  -0.3089825248449213
10.723027171839092  0.31000757578963106 -0.31041319757447466
10.72918308755497 0.31051156580213396 -0.3056607952854435
10.73533900327085 0.30855144675274093 -0.3084392468606682
10.74149491898673 0.3062583860554826  -0.30847358020352705
10.747650834702606  0.308163469137206 -0.3057478786286695
10.753806750418486  0.3047094490025869  -0.30817074624931423
10.759962666134365  0.30401371728467813 -0.3071621379706717
10.766118581850245  0.306004244388029 -0.30618664163668824
10.772274497566125  0.3014522753389074  -0.30784211336874984
10.778430413282004  0.30341318564145786 -0.3062820193343738
10.784586328997884  0.30408297177293087 -0.306454685666655
10.790742244713764  0.3009681295406021  -0.306059675670431
10.796898160429643  0.30317367197258666 -0.3052163251229233
10.803054076145521  0.3025483960926167  -0.30551516456287336
10.809209991861401  0.30124000326643297 -0.30368112344068665
10.821521823293157  0.3015676482000731  -0.30196109813323163
10.833833654724916  0.30157890128248127 -0.30089187524117283
10.846145486156676  0.3011285399121534  -0.29827106487865856
10.858457317588435  0.29900278924501084 -0.29409285888281733
10.870769149020195  0.29651499884766097 -0.29357870873433833
10.88308098045195 0.2937565969845928  -0.294111128162059
10.895392811883708  0.2930558617760271  -0.29090670912390865
10.907704643315467  0.2913669989210539  -0.2894133869459216
10.920016474747227  0.2863561407804618  -0.2908306371926589
10.932328306178983  0.28633426672076545 -0.2907169855062541
10.94464013761074 0.2877234089953953  -0.2893003091441861
10.9569519690425  0.28556884496849205 -0.2882631642809839
10.969263800474259  0.28441441147933844 -0.2870958774063245
10.981575631906018  0.28486846171427843 -0.2856084289494289
10.993887463337778  0.285005338737003 -0.2844322933650049
11.006199294769534  0.28375480512085405 -0.28126506057513967
11.012871165201213  0.281845171682149 -0.2799847313145348
11.02621490606457 0.2762090070033339  -0.2794477658999403
11.03955864692793 0.2751969327554266  -0.2803215145556965
11.05290238779129 0.2774993315356159  -0.2776660155529882
11.112949221676407  0.27215661898960064 -0.26978548182381146
11.433199002397025  0.23731512991790055 -0.23543938385875204
11.852403485305125  0.19113397435850465 -0.19336777283506695
11.8585128774133  0.19175862948272238 -0.1932081527042111
11.870731661629659  0.19028800393790254 -0.19182269777300076
11.882950445846015  0.18887070847104914 -0.19055717445259893
11.895169230062368  0.1886853046138286  -0.1893300655915967
11.907388014278727  0.1884656423818683  -0.18831830495828122
11.919606798495083  0.18704181005994255 -0.18598247590166292
11.931825582711436  0.18562878040944344 -0.18355067014494503
11.944044366927795  0.18434864836387224 -0.18325078817317866
11.956263151144153  0.18309046780758953 -0.18191274141199146
11.96848193536051 0.18209798639095692 -0.1782652276891998
11.980700719576863  0.17979791370668902 -0.17842139442749982
11.992919503793221  0.17737675114855037 -0.1785979667289508
12.005138288009578  0.17719828089548903 -0.17627897902044207
12.017357072225936  0.1758353071548321  -0.1753502812788466
12.029575856442289  0.1722389210260632  -0.17555966320230543
12.041794640658646  0.17254475435463063 -0.1747620196818636
12.054013424875004  0.17264786442728305 -0.1734951884428762
12.06623220909136 0.17034295018444576 -0.17256696705804556
12.078450993307714  0.16956541141036635 -0.17180223556207397
12.090669777524072  0.16970518027437675 -0.17097754507504617
12.10288856174043 0.16888416061165434 -0.16929716768881417
12.115107345956787  0.16767084991909958 -0.16746537536122186
12.12732613017314 0.1667129006606392  -0.16562027623956038
12.145654306497677  0.16512597457166783 -0.16286779065783502
12.157873090714034  0.16382783365411502 -0.1629357169292716
12.194529443363102  0.1594513710021715  -0.15867926816555494
12.20674822757946 0.157111672688071 -0.15865046598119736
12.218967011795815  0.15725952398098275 -0.15642001887080156
12.23118579601217 0.1558857197121585  -0.15591447306373032
12.243404580228528  0.15252536734441757 -0.15591388976445475
12.250029927052507  0.15466768779591628 -0.1555533491234515
12.256655273876486  0.1535844266743343  -0.1553181694561173
12.263280620700465  0.15432346821306986 -0.1545158020266448
12.26990596752444 0.15375692175903952 -0.15246490234790439
12.276531314348418  0.15360293364770425 -0.15183929717882513
12.283156661172397  0.15190759035317072 -0.1511826555866588
12.289782007996376  0.15138831509948003 -0.1493476088472744
12.296407354820355  0.14814471931798814 -0.15101502287901414
12.349410129412185  0.14426777395153112 -0.14617456698619619
12.45541567859584 0.13698515268787811 -0.13681850830568112
12.667426776963154  0.12122880329560894 -0.12071573689412787
12.673611445849872  0.12115174705324538 -0.12063738375198094
12.679796114736588  0.12032059298841463 -0.1189274501956755
12.685980783623306  0.12030722585026402 -0.11876782898894318
12.692165452510025  0.11936832171041596 -0.11878693964701778
12.704534790283459  0.11814909943586377 -0.11748842184102257
12.716904128056893  0.11736086959663367 -0.11496327499129036
12.723088796943612  0.11567392518439616 -0.11645179049382663
12.735458134717046  0.11555036863375674 -0.11459942735091731
12.747827472490483  0.1143219514257457  -0.11454080406333877
12.760196810263917  0.11189362677903378 -0.11444018121477051
12.766381479150635  0.1133073067872921  -0.11389535249939534
12.778750816924072  0.11154303396894738 -0.11326175597470724
12.791120154697506  0.11144089285213202 -0.11257250480415154
12.803489492470941  0.1113273417377863  -0.11164185934762183
12.815858830244379  0.11099738041748752 -0.11066623272811332
12.828228168017814  0.11007540344944039 -0.10900398378442647
12.84059750579125 0.10896725846983848 -0.10710190218561176
12.852966843564683  0.10777605785367389 -0.10725755658931205
12.86533618133812 0.10636311443600022 -0.10655722743837756
12.871520850224838  0.1060382943803558  -0.10566047442169588
12.877705519111556  0.10598545295631553 -0.10466922945188555
12.883890187998272  0.10418103461195145 -0.10561976956692884
12.890074856884992  0.10467609656481544 -0.104615752876623
12.89625952577171 0.10430992558937936 -0.10442602714071539
12.902444194658425  0.10234377995270529 -0.10469158026649737
12.908628863545143  0.10367452478979461 -0.10403120685793919
12.914813532431863  0.10274827106693378 -0.10415922620781146
12.920998201318579  0.10188262854665356 -0.10365803320444716
12.927182870205296  0.10274952950909313 -0.10342616696742041
12.933367539092014  0.10176282990960264 -0.1031951074509836
12.93955220797873 0.10163603802804774 -0.10257401378191737
12.94573687686545 0.1018318995539753  -0.10253046779940327
12.951921545752167  0.10120580343879769 -0.1016629486193261
12.958106214638883  0.10135691088921046 -0.10141492111316401
12.964290883525601  0.10083585328473238 -0.10055171087156578
12.976660221299037  0.1003784879008382  -0.09996943697449343
12.989029559072472  0.09972688150845745 -0.09856920914077483
13.001398896845908  0.09867818716764729 -0.09661933020237613
13.013768234619342  0.0973510415661029  -0.09679583177893712
13.026137572392779  0.09583804054604132 -0.09663036236697777
13.038506910166213  0.095662878929824 -0.09538974680028753
13.050876247939646  0.09494076398151899 -0.09498671394619589
13.063245585713084  0.09319123645688658 -0.09485107306946283
13.069308454213559  0.09400367721458666 -0.09442267605917624
13.075371322714037  0.0928653201240209  -0.09454155504524829
13.081434191214516  0.09251993976497673 -0.09388432713553281
13.087497059714991  0.09304970979678703 -0.09365608320940247
13.09355992821547 0.09197813749076358 -0.09352134006155627
13.099622796715947  0.09188778171768897 -0.09292167392207405
13.105685665216424  0.09210631402639882 -0.0928569700663483
13.1117485337169  0.0912242796602791  -0.09238807473181129
13.117811402217379  0.0912671435406555  -0.09195860403661964
13.123874270717854  0.0911737023276095  -0.09193259447314497
13.129937139218333  0.09051164088688866 -0.09123377031797121
13.13600000771881 0.0906328925885371  -0.09098949258825423
13.142062876219287  0.09026731483739514 -0.09062412131964304
13.148125744719763  0.089818274762915 -0.09007548917551664
13.154188613220242  0.0899624027848155  -0.09000753303529742
13.160251481720717  0.08937358752678286 -0.0892887312058014
13.166314350221196  0.08912262904305437 -0.08892824592387036
13.172377218721675  0.08907440193507098 -0.08889641263259719
13.17844008722215 0.08848803112138943 -0.08795426936623203
13.184502955722628  0.08840347947091785 -0.0878049743746649
13.190565824223105  0.08803022032731252 -0.08768531956702759
13.202691561224059  0.08763986616466436 -0.08671649934130231
13.208754429724538  0.08696708211640124 -0.08646624829278267
13.220880166725491  0.0864477778614231  -0.08577238433908745
13.233005903726445  0.08581771231471796 -0.08420538278532116
13.2451316407274  0.08484190960761938 -0.0840689965039726
13.257257377728354  0.08398616545794646 -0.0839532425919587
13.263320246228831  0.08380358957108178 -0.0829252981204666
13.269383114729308  0.08366274375020977 -0.08258350473718745
13.275445983229785  0.08277254050915805 -0.08306870584524462
13.281508851730264  0.08279370083220905 -0.0818427169607557
13.287571720230739  0.08249401789853343 -0.08197137561546361
13.293634588731218  0.08160028205149591 -0.0821985683917323
13.299697457231696  0.08187461991494269 -0.08109327567516907
13.305760325732171  0.0813274174472849  -0.08138831461908777
13.31182319423265 0.08048945853387285 -0.08134328490152058
13.317886062733127  0.08099030907568958 -0.08048320084580232
13.323948931233604  0.08018064152557768 -0.08081403426249223
13.33001179973408 0.07951731204272444 -0.0805207725839455
13.33607466823456 0.08012099006152924 -0.07992303292780585
13.342137536735034  0.07907235912536648 -0.08023077443627899
13.348200405235513  0.07887098523815249 -0.07978989708661499
13.35426327373599 0.07926536533847707 -0.07939175596224904
13.360326142236467  0.07802226203293827 -0.07962329600009847
13.372451879237422  0.07842389853309248 -0.07886846093659135
13.384577616238376  0.07770590618404152 -0.0783918449413461
13.39670335323933 0.07668413333722265 -0.07803346348384513
13.408829090240285  0.07679421887854791 -0.0777624093044015
13.42095482724124 0.07658820017558919 -0.07701761232954447
13.451269169743625  0.07526015056150664 -0.07550990930713644
13.457847992959902  0.07532562174807103 -0.07505143519004628
13.46442681617618 0.07517600802830421 -0.07413482853025566
13.471005639392457  0.07450721997615004 -0.0742985164467615
13.477584462608736  0.07420912042387921 -0.0728496179637296
13.484163285825012  0.07316937655435538 -0.07377542537138274
13.490742109041289  0.07299294976961693 -0.07351662175030357
13.497320932257567  0.07303906010957126 -0.07360022767928533
13.503899755473846  0.07233378833479674 -0.07347461807050755
13.510478578690122  0.07308217718374452 -0.07341384848346864
13.523636225122678  0.07284425012400998 -0.07240980577360331
13.536793871555233  0.07178928460190198 -0.07138549740219421
13.549951517987788  0.07113305809950596 -0.0707640019858965
13.556530341204065  0.06963738215003415 -0.07138667163505419
13.563109164420343  0.0706986698396685  -0.07115509225056954
13.569687987636621  0.07036114866617268 -0.07105261702537917
13.576266810852898  0.07050349389974367 -0.07061767120298539
13.582845634069175  0.0703909494519751  -0.0698778900316415
13.589424457285453  0.07039516399393234 -0.06938067089579283
13.59600328050173 0.06964767142371524 -0.06940828871129545
13.602582103718008  0.06946402025467203 -0.06808634435795237
13.609160926934285  0.06807596109677785 -0.06915952922588674
13.622318573366842  0.06820552399016017 -0.06906959562397438
13.635476219799395  0.06842928512296108 -0.0687210626039047
13.64863386623195 0.06818839522578891 -0.06784630522241229
13.661791512664506  0.06708220676403631 -0.06711584111724365
13.668370335880784  0.06672478972104048 -0.06658839145612984
13.674949159097062  0.06660055887127461 -0.06659995293326555
13.681527982313337  0.06553307502754836 -0.06691943932094796
13.688106805529616  0.06642921878071607 -0.0668600959007205
13.694685628745894  0.06605101033993609 -0.06662692708861646
13.70126445196217 0.0664062415936948  -0.06638072292297402
13.707843275178448  0.06611639191068443 -0.06548628921625822
13.714422098394726  0.06606629277988883 -0.06535368638758837
13.721000921611004  0.06543534720765413 -0.06505225556796117
13.727579744827283  0.06525508044309443 -0.0643680258549692
13.734158568043558  0.0640335442211665  -0.06504880593307868
13.740737391259836  0.06461990040717193 -0.06473560280232603
13.747316214476115  0.06394118083463271 -0.06513335718023423
13.75389503769239 0.0641907564131276  -0.06475586650296355
13.760473860908668  0.06435967870304854 -0.0645586099749761
13.767052684124947  0.06436090633750084 -0.06403480795794678
13.773631507341225  0.06415530419092037 -0.06380445630249207
13.780210330557503  0.0639913981955003  -0.06267688964558053
13.786789153773778  0.06315856971122753 -0.06334372393232203
13.793367976990057  0.06307344891905092 -0.06267256392951384
13.799946800206335  0.06258947994215651 -0.0630842618403981
13.80652562342261 0.06219569223291971 -0.06308209936277123
13.813104446638889  0.0626558057146996  -0.06317344053644054
13.819683269855167  0.06235598209190561 -0.06282921153204937
13.826262093071445  0.06277763337586714 -0.06272537569124596
13.832840916287724  0.062478084149088474  -0.06181681667581049
13.839419739503999  0.06230615514872451 -0.06188740944139036
13.845998562720277  0.061863145945320525  -0.06121676869145704
13.852577385936556  0.0615486172310248  -0.061187329730760714
13.85915620915283 0.0606454887656459  -0.061435652239130834
13.865735032369109  0.06117085709429534 -0.06126761559442133
13.872313855585388  0.06041723786331005 -0.06155637820417474
13.878452000864405  0.06075134296906955 -0.06127326534396179
13.88459014614342 0.06072189316880219 -0.061359152834898456
13.890728291422436  0.0602763828079906  -0.06101622902370101
13.896866436701453  0.060622485142070484  -0.060921694947692284
13.90300458198047 0.060380311616670394  -0.06075782505368307
13.909142727259487  0.06019771590167938 -0.06045675492791395
13.915280872538505  0.06043440908959312 -0.06051314924756567
13.92141901781752 0.06009263802491923 -0.06000208320996694
13.927557163096536  0.06011501365410078 -0.059903699250310825
13.933695308375553  0.05997141983362481 -0.059750203232183365
13.93983345365457 0.05980955628276299 -0.05924771729881789
13.945971598933589  0.05983760791849895 -0.05938259007122479
13.952109744212606  0.0594801574902973  -0.058938909337567624
13.958247889491622  0.05949502683483617 -0.05855353248237824
13.964386034770637  0.05917519762548449 -0.05889538552398849
13.970524180049654  0.05898393919326815 -0.0581406643734625
13.976662325328672  0.05893147263566017 -0.05814312242403667
13.982800470607687  0.05847897359153759 -0.05841430784876933
13.988938615886703  0.05849719930985661 -0.057423429518878064
13.99507676116572 0.05814401380769593 -0.057961692163473
14.001214906444737  0.05780636464962738 -0.05794932661899451
14.007353051723754  0.05801565562650654 -0.05720041578678611
14.013491197002772  0.057335510504123846  -0.05781375573106247
14.019629342281787  0.057225483415649855  -0.05752866179870199
14.025767487560802  0.057530272957841265  -0.057191589615576015
14.03190563283982 0.056575587159959466  -0.0576595232933571
14.038043778118837  0.057009264874067836  -0.057339724689637814
14.044181923397856  0.05705333443553076 -0.057242763380375795
14.050320068676873  0.05608011090268437 -0.05743111740569201
14.056458213955889  0.056842632435851165  -0.05719057096412819
14.062596359234902  0.056604642324084214  -0.057282598148886296
14.068734504513921  0.056055889558249215  -0.05709010378575794
14.074872649792939  0.05668819697709257 -0.05703727351996367
14.081010795071954  0.05633407104567063 -0.057017243236333594
14.08714894035097 0.056132704693134146  -0.056739314947333794
14.093287085629987  0.05651732180956823 -0.05682908272472146
14.099425230909004  0.05622098880482845 -0.056497731680047375
14.105563376188021  0.056238482342724244  -0.05638409370141744
14.111701521467038  0.056248880570856294  -0.056200377262253846
14.117839666746054  0.05612527300722511 -0.05594717578004784
14.12397781202507 0.05627035546371634 -0.05601559264665258
14.130115957304087  0.055956046726662864  -0.055454156512228014
14.136254102583104  0.05599960105317258 -0.05541290836903637
14.142392247862121  0.05583951827829574 -0.0555437295986804
14.14853039314114 0.05565822839938513 -0.054737052860610876
14.154668538420154  0.05566993494914485 -0.05500970179468424
14.16080668369917 0.05535185458324997 -0.055057122529582156
14.166944828978188  0.05535101816986254 -0.054117960087184225
14.173082974257206  0.05499965775862777 -0.05480548104608896
14.179221119536221  0.05485824918142511 -0.05458350697834474
14.185359264815235  0.05495165482611006 -0.05390439216458342
14.191497410094254  0.05431785556862551 -0.05462915195923206
14.197635555373271  0.05440120759161664 -0.05415739859135221
14.203773700652288  0.054471116276675696  -0.05399347274618245
14.209911845931305  0.053692325036559664  -0.054460878431625274
14.21604999121032 0.05414735236760024 -0.054042597457965066
14.222188136489336  0.05398666548047641 -0.054128137366755036
14.228326281768354  0.05322158055539977 -0.054287956680194795
14.23446442704737 0.05395638588137345 -0.054041885990034205
14.240602572326388  0.0535344815373204  -0.05424213843003527
14.246740717605407  0.05325801017647654 -0.054112332095384665
14.25287886288442 0.053781320853959196  -0.0540603809710362
14.259017008163436  0.053256894161599594  -0.0541787084505972
14.265155153442453  0.05337017728894822 -0.05394186451469646
14.271809253437272  0.05368538553762559 -0.05398456105675655
14.27846335343209 0.053756809357559845  -0.05356857978241147
14.285117453426906  0.05372307230050677 -0.05341816529345983
14.291771553421723  0.05366612400171593 -0.052551566109285
14.298425653416539  0.05298233851617476 -0.05318524360142636
14.305079753411356  0.05305550377934988 -0.05274774954426795
14.311733853406173  0.052598947182328944  -0.053260608932666376
14.318387953400991  0.05264375088914475 -0.05324140935687915
14.331696153390626  0.052992637977432505  -0.05302138492097842
14.34500435338026 0.05306916159730393 -0.05211100706695544
14.358312553369892  0.05250921047406156 -0.05199659069050745
14.371620753359528  0.05201321753783695 -0.052598094198474755
14.378274853354345  0.05232031637631076 -0.05283773358288498
14.384928953349164  0.052284410602536656  -0.05252678296299114
14.39158305334398 0.052586680441361666  -0.05243792258970931
14.398237153338796  0.052523424958290474  -0.05173853880349345
14.404891253333611  0.052180082144810215  -0.052046656241557764
14.411545353328428  0.05204612210402063 -0.051328598421929154
14.418199453323247  0.05158612504565059 -0.05195404572342846
14.424853553318064  0.05147412858798176 -0.052012275565907186
14.431507653312881  0.05177486569825656 -0.05223952947411478
14.444815853302513  0.052094889984273654  -0.05206372982156938
14.458124053292147  0.05184851009791044 -0.051609033617816935
14.471432253281783  0.051217636533644854  -0.051427407536432275
14.478086353276598  0.051021456334373744  -0.0514847986478515
14.484740453271414  0.05131498795354316 -0.05168373450683887
14.49139455326623 0.05104494296218816 -0.05168435647853673
14.49804865326105 0.05165378249166927 -0.0517551591447124
14.504702753255867  0.05157542921060601 -0.051168513689317845
14.511356853250684  0.05155726349629098 -0.051250999627739255
14.5180109532455  0.05134382577807582 -0.05048001705637829
14.524665053240316  0.05093237507035941 -0.050981218603619205
14.531319153235133  0.05064978219788973 -0.05101686182531985
14.53797325322995 0.05093729603098354 -0.05117229349838723
14.544627353224769  0.050518233179849784  -0.051332846776340986
14.551281453219586  0.05121998177394686 -0.051411981625910416
14.557935553214401  0.05117024663335649 -0.050958485154491785
14.564589653209216  0.05130245388165792 -0.050965831564147246
14.571243753204033  0.05106793639471752 -0.05020329187712721
14.577897853198852  0.050722995378684334  -0.05061155633838288
14.58455195319367 0.050353894281161216  -0.05061014331736659
14.591206053188488  0.05063750021938515 -0.05070719040846163
14.597860153183305  0.050056449170125285  -0.05102623617424996
14.604514253178122  0.050846709069572366  -0.0510986692048963
14.611168353172937  0.05081017148124977 -0.050791376819584205
14.617822453167753  0.05108058215048523 -0.050746537297086586
14.624476553162571  0.05083571821895067 -0.050009366621627416
14.631130653157388  0.05058167577301438 -0.050313975513243736
14.637784753152205  0.05021076977402765 -0.050326792369211414
14.644438853147024  0.05041057539708968 -0.05029041023881246
14.651092953141841  0.04966163137414731 -0.050763776953231095
14.657747053136658  0.05053786302076503 -0.05082260829606739
14.664401153131474  0.05049530766741659 -0.05066265350714172
14.67105525312629 0.05087503074819006 -0.05058435260180465
14.677709353121108  0.05064433550772809 -0.049891710350669784
14.684363453115925  0.05050025606220239 -0.05008369066464479
14.691017553110743  0.05011983319749954 -0.05011501321195192
14.697230975168297  0.050347143865414674  -0.049646246071216295
14.703444397225855  0.049614483955715785  -0.050345491623180263
14.70965781928341 0.04998419623531467 -0.050138007903689115
14.715871241340967  0.05011677999983526 -0.0501197198673251
14.722084663398524  0.04924836609207145 -0.050482692968545705
14.72829808545608 0.05011267326996019 -0.05037133243587215
14.734511507513636  0.04993961931605426 -0.05058663492868685
14.740724929571192  0.04963559148655034 -0.05047019218906244
14.74693835162875 0.05027932347774033 -0.05057003992524109
14.753151773686305  0.05008298408437736 -0.050514519035050734
14.759365195743861  0.050146995109979234  -0.05041546944557879
14.765578617801419  0.05038039416663399 -0.05041656609753953
14.771792039858973  0.050340278415624884  -0.050199967727773014
14.77800546191653 0.05057689217015183 -0.0503152388427283
14.784218883974088  0.05037629448406584 -0.04982940043926032
14.790432306031642  0.050527038954593595  -0.0498727414579413
14.796645728089198  0.050366468371986536  -0.0500853469248367
14.802859150146753  0.05033054128990185 -0.04927413396402177
14.809072572204311  0.05020259694165439 -0.04983787010217487
14.815285994261865  0.050054641233211954  -0.04986240266170258
14.821499416319421  0.05020374810110147 -0.04911570596199502
14.827712838376979  0.0496270911273178  -0.04997883202606512
14.833926260434534  0.049761375141983165  -0.04971083303796976
14.84013968249209 0.049971757607466354  -0.049581394228433454
14.846353104549644  0.04912719088144385 -0.050148642402605484
14.852566526607202  0.04984015390276882 -0.049967612129122624
14.858779948664758  0.04976438714150872 -0.05012147718677221
14.864993370722313  0.049212971293903014  -0.050248776856332465
14.871206792779871  0.049996446963239405  -0.05026358579905157
14.877420214837427  0.049728175630079 -0.05044300899663736
14.883633636894983  0.049718805565788886  -0.0502796722310726
14.889847058952538  0.050168236590695935  -0.050466820361652565
14.896060481010096  0.05002893406149481 -0.05024256597821795
14.902273903067652  0.050250353624286655  -0.0502653681049768
14.908487325125208  0.05024975401143517 -0.05005411223283431
14.914700747182765  0.050327620617333806  -0.04996422413152837
14.92091416924032 0.050399371989337804  -0.05012873369460872
14.927127591297877  0.05028404923427443 -0.04951586473048067
14.933341013355435  0.0504142389358594  -0.049765156762313745
14.939554435412989  0.050179657249817176  -0.04990382525037004
14.945767857470544  0.05026864076664099 -0.04908036901056328
14.9519812795281  0.04992512220917845 -0.04987760400594207
14.958194701585658  0.049912691497194794  -0.04971893976277032
14.964408123643212  0.050088562924849186  -0.04934295048010269
14.970621545700768  0.049409311967425415  -0.05004922014388985
14.976834967758325  0.04982079509766669 -0.049790467764272726
14.983048389815881  0.049871894260083593  -0.04987531255451675
14.989261811873437  0.04909448374565208 -0.0502256471581221
14.995475233930991  0.04995926850818854 -0.05013220403629778
15.001688655988549  0.04971700665573317 -0.050402117038984354
15.007902078046106  0.04952906038886206 -0.05032254746429895
15.01411550010366 0.0501437822653835  -0.050447499416658985
15.020328922161218  0.049923798744222335  -0.050445982112016195
15.026542344218774  0.05008217145271499 -0.050377009581089496
15.03275576627633 0.05030694881282185 -0.05042775867228749
15.038969188333887  0.05028226364389749 -0.05023452689803823
15.045182610391443  0.05057734758455556 -0.05037306377742242
15.051396032448999  0.05040350394965246 -0.04994362232395669
15.057609454506554  0.05057756889252235 -0.04999464036253084
15.063822876564112  0.05046285921862792 -0.050173052306580375
15.070036298621666  0.05045283043738642 -0.04946338235876897
15.076249720679224  0.05037475637429952 -0.04998937871417345
15.082463142736781  0.050240575246197065  -0.04997625632164195
15.088676564794335  0.05039854709547143 -0.049324252053476465
15.09476818646565 0.049960185997665246  -0.05013245354526502
15.100859808136967  0.050089466627735606  -0.04989232566144547
15.106951429808284  0.05030602704184107 -0.049592149274487546
15.113043051479599  0.049706776020567646  -0.050260644329602346
15.119134673150912  0.05005363896114692 -0.0499459110301814
15.125226294822228  0.05021292518661484 -0.049912528532534056
15.131317916493545  0.049493674251951794  -0.050397275874770525
15.13740953816486 0.050141692915222004  -0.050160558950725895
15.143501159836175  0.0501327171711319  -0.05025615002603898
15.149592781507492  0.04940692268160699 -0.050528899903598226
15.155684403178808  0.050257504769679454  -0.05039699261460854
15.161776024850123  0.05007885837259203 -0.05059360745966927
15.16786764652144 0.049660385768615985  -0.05062893205780151
15.173959268192757  0.050393179554988864  -0.050632036502888104
15.18005088986407 0.050065574116909255  -0.050893917868709476
15.186142511535387  0.04997746966472772 -0.05071814667578398
15.192234133206703  0.05054064437437502 -0.05084238512935519
15.198325754878018  0.05027959215601574 -0.05088739670149833
15.204417376549335  0.0503288872150839  -0.05079359858265941
15.210508998220652  0.05069151239619889 -0.051004660733927336
15.216600619891967  0.05052233347812692 -0.050831641060859856
15.222692241563283  0.05068500521582485 -0.050852550538013866
15.2287838632346  0.05080997643788109 -0.050872760440796756
15.234875484905915  0.05077041583102256 -0.050745171092843454
15.240967106577228  0.05101570225776471 -0.05089260031134159
15.247058728248545  0.05091169368841891 -0.050644090501421525
15.253150349919862  0.05100039845885706 -0.05064688429329048
15.259241971591177  0.05112515379231096 -0.050863259899597536
15.265333593262492  0.05099921465693619 -0.05039579831573529
15.271425214933808  0.05118883751214625 -0.05055618874582945
15.277516836605125  0.05109058900300719 -0.050797534919010165
15.283608458276438  0.05107000850733389 -0.05015929349837467
15.289700079947753  0.05118021242204359 -0.0505638582833741
15.29579170161907 0.051017142274761215  -0.050734791179890125
15.301883323290387  0.05112202348495381 -0.04996649799871501
15.307974944961702  0.05096803808817515 -0.050683658053367026
15.314066566633016  0.050924015266783175  -0.05068810221042963
15.320158188304333  0.05113206123548333 -0.04994851484964532
15.32624980997565 0.05072442505556756 -0.05083274521373982
15.332341431646963  0.05083107056216946 -0.05067005550166579
15.338433053318278  0.05107942011945215 -0.050239570897766476
15.344524674989595  0.05048119205023574 -0.05100243426479994
15.350616296660911  0.05077979323599363 -0.050716457818752614
15.356707918332226  0.05102624682977058 -0.050593875434933946
15.36279954000354 0.05027074480961484 -0.05118337003171715
15.368891161674856  0.050898484491835046  -0.05095687941117297
15.374982783346173  0.05098534689466102 -0.05098094408739565
15.381074405017488  0.05012613620724282 -0.051365416708245626
15.387166026688803  0.051050888343785926  -0.05121879286188911
15.39325764836012 0.05096891315019742 -0.051369435942611764
15.399349270031436  0.05038671796560823 -0.05149265568394109
15.405440891702751  0.051227947674787944  -0.05147858837161365
15.411532513374064  0.05098842592831809 -0.05172704040264612
15.417624135045381  0.05073538584686464 -0.051601576559114855
15.423715756716698  0.051419804202163494  -0.05171280909389574
15.429807378388013  0.05118581896964937 -0.05180853182456706
15.435899000059328  0.05112877313994743 -0.05169216761031118
15.441990621730644  0.05161569597837445 -0.051898198702714185
15.448082243401961  0.05144696678966062 -0.05177142100013753
15.454173865073276  0.051534927260185166  -0.051763453411247363
15.460265486744593  0.0517698476736048  -0.05184267497371163
15.46635710841591 0.0517111745945482  -0.051693257255793205
15.472448730087223  0.051922175905289296  -0.051815346342164066
15.47854035175854 0.05188500511499064 -0.05162509239302745
15.49175550453277 0.051830441161540834  -0.051581300879261695
15.504970657307005  0.05107937536233644 -0.051747430214110934
15.518185810081237  0.05153857256197  -0.05209745229454515
15.531400962855468  0.05211224932136459 -0.052247197257201256
15.544616115629701  0.05227669698955367 -0.05196903847860161
15.557831268403932  0.05174139302728276 -0.05184350782725458
15.571046421178163  0.051896495827518416  -0.0519847084328019
15.584261573952398  0.0522585531035021  -0.052530624662101136
15.59747672672663 0.05267439222704885 -0.052346282200129
15.61069187950086 0.05240130015440304 -0.051950112659526926
15.623907032275094  0.05227488797827171 -0.05186843218497916
15.637122185049327  0.05235129472006055 -0.05264756674633921
15.650337337823558  0.05267577091460354 -0.05266925441393524
15.663552490597791  0.05298516846125376 -0.05210103026530691
15.676767643372022  0.052650983226487376  -0.051807750628806055
15.689982796146253  0.05243149468232284 -0.05273696168924048
15.703197948920485  0.05258268938991162 -0.052983411581345764
15.716413101694716  0.05316633881926751 -0.05255805881751215
15.72962825446895 0.0530065225350367  -0.05193160013158706
15.742843407243182  0.052524727115983195  -0.05283631720415503
15.756058560017413  0.052498622844005215  -0.05329139742990521
15.769273712791646  0.05329199850514191 -0.05306587931080806
15.782488865565877  0.05334028348854233 -0.052495445203973076
15.795704018340109  0.05275853217151138 -0.05302535786083633
15.808919171114342  0.052479528171926346  -0.0535823314006402
15.822134323888573  0.0534006376020282  -0.053582570756381766
15.835349476662806  0.05367275380669177 -0.053117500727858695
15.848564629437037  0.05324474859260202 -0.05333903436188823
15.861779782211268  0.052760474284935875  -0.0537564398137877
15.874994934985503  0.05352326719996086 -0.05406151221373631
15.888210087759735  0.05399416737800061 -0.053735826297540604
15.901425240533966  0.0537740505908066  -0.05365462051171525
15.907592138983821  0.053869882600949506  -0.05300846346011036
15.913759037433676  0.05381366222706683 -0.05342022473526249
15.91992593588353 0.05350759711503051 -0.05370997411929279
15.926092834333385  0.05385550231770936 -0.0531909942285959
15.932259732783242  0.053519826357722546  -0.05379028389216588
15.938426631233096  0.05330341392913846 -0.05381590187522743
15.94459352968295 0.053879809963373895  -0.053677667327180656
15.950760428132806  0.053290418490045836  -0.054181031052782616
15.956927326582663  0.05358209862120967 -0.05408903602438368
15.963094225032517  0.05394137557181431 -0.054187879247281225
15.96926112348237 0.053393349417919636  -0.054422046397483693
15.975428021932228  0.05395739434863473 -0.0543532289877948
15.981594920382083  0.05404794915540276 -0.05461886307394269
15.987761818831936  0.053861429803164426  -0.05445339837048666
15.993928717281792  0.05436207606687604 -0.054561211588957936
16.00009561573165 0.05429522962898832 -0.05450563492580909
16.006262514181504  0.05436353169042975 -0.05442231567105969
16.01242941263136 0.054648500918898155  -0.05458268875195459
16.01859631108121 0.05455105334004859 -0.05422811526672204
16.024763209531066  0.05479781363517278 -0.05436157335664795
16.030930107980925  0.054678485471188726  -0.0543234971900574
16.037097006430777  0.05475652347835399 -0.05393442615320216
16.043263904880632  0.054750520037351924  -0.054363978080569246
16.049430803330488  0.054639713152299214  -0.05405957343527824
16.055597701780343  0.0548244800151944  -0.05379130208137905
16.061764600230198  0.05446658572823103 -0.05442041746413504
16.067931498680053  0.05456767580080688 -0.05387051305944782
16.07409839712991 0.054580499746824856  -0.054114775695696594
16.080265295579764  0.05415577383292768 -0.05451675753981509
16.08643219402962 0.05455698552974695 -0.05406425297746795
16.092599092479475  0.05432262063993779 -0.05453060329824742
16.09876599092933 0.05393341513140517 -0.05465511986382612
16.104932889379185  0.05461870257567663 -0.0545231882204401
16.11109978782904 0.05412808003958159 -0.054962133194494105
16.117266686278892  0.05424807117125761 -0.05489303356175812
16.12343358472875 0.054720909863301886  -0.05499140695877261
16.129600483178606  0.054247403029443794  -0.055197224119825285
16.13576738162846 0.054668091969691725  -0.05511107056273068
16.141934280078313  0.05486008731951717 -0.05537035002099131
16.14810117852817 0.05468817668432551 -0.055181758757415
16.154268076978028  0.05511397366682613 -0.055272655183653326
16.16043497542788 0.055080077349375284  -0.05518725240389725
16.166601873877735  0.05514816082133092 -0.055098853193772654
16.17276877232759 0.05540308179234759 -0.05526850290064042
16.17893567077745 0.055289018107592694  -0.05485300192180398
16.1851025692273  0.05552982889068424 -0.05499312611797003
16.191269467677156  0.055386436033382766  -0.05503687564663786
16.19743636612701 0.05544609957027734 -0.05451470281256022
16.203603264576866  0.0554121179236705  -0.05500889545810466
16.20977016302672 0.055294610911204314  -0.054807644331544135
16.215937061476577  0.05547899892414624 -0.05436214698555846
16.222103959926432  0.055069393615152604  -0.05510207994749548
16.228270858376288  0.055174961790386653  -0.054650077220200585
16.234437756826143  0.05526005760936553 -0.054721753571836965
16.240604655275995  0.05471116586296104 -0.055236201578156445
16.246771553725853  0.055167455257482866  -0.05484851624484123
16.25293845217571 0.05503563350777441 -0.05518008312343925
16.259105350625564  0.05447008108061083 -0.055400768542372886
16.265272249075416  0.05526394113607094 -0.05527373189873663
16.27143914752527 0.05487259238123011 -0.05564817168503809
16.27760604597513 0.05481805675275339 -0.055598110749596044
16.283772944424985  0.055403328458591736  -0.05569432806887429
16.28993984287484 0.05500217373581041 -0.05587033437528699
16.296106741324696  0.055279217566669085  -0.0557653898751341
16.30278959449035 0.0556594571227649  -0.055801412773203135
16.309472447656006  0.055904880284032454  -0.05550505432406336
16.316155300821663  0.0558140556446825  -0.05544391066245013
16.32283815398732 0.055849714401550093  -0.054684987467238676
16.329521007152977  0.0550666954872421  -0.055562274951857234
16.33620386031863 0.05548027185436415 -0.055401065867602614
16.342886713484283  0.05512279690512286 -0.05601603660916846
16.34956956664994 0.0554665330876235  -0.05595681005534581
16.40303239197518 0.05563970729610239 -0.05613193470039006
16.409715245140838  0.05605608184457482 -0.05610812915346811
16.41639809830649 0.056287573105304384  -0.05579753324849156
16.423080951472148  0.05613745741100129 -0.055787777515090564
16.4297638046378  0.056153960334796985  -0.054962572288639606
16.436446657803458  0.05530660264586304 -0.05592552611750016
16.443129510969115  0.055792491450340626  -0.05581336349008319
16.44981236413477 0.05550973516258904 -0.05638834713496776
16.456495217300425  0.05579795273849911 -0.05628998062541428
16.50995804262567 0.055940555580357876  -0.056430256118104795
16.516640895791326  0.05638351121344803 -0.056346404850872334
16.52332374895698 0.05660001671333402 -0.05602402432009652
16.530006602122636  0.0563919159712029  -0.056068997063686996
16.53668945528829 0.05639504624030993 -0.05518182441253594
16.543372308453947  0.05548165013561163 -0.05622511972420809
16.550055161619603  0.05604281243321418 -0.056154658582481094
16.55673801478526 0.05582987486484723 -0.0566944238181184
16.563420867950914  0.05606687435100181 -0.056552158516929456
16.570103721116567  0.05651944813884132 -0.05643840366090784
16.58346942744788 0.05649178231356022 -0.05618434876339387
16.596835133779194  0.05554369303605121 -0.056349262274293176
16.6102008401105  0.0559629972744793  -0.05682086123982419
16.616883693276158  0.056176338560757925  -0.05665517159586292
16.623566546441815  0.056636073832564915  -0.05651169424005984
16.630249399607468  0.056836962641678354  -0.056180308251364365
16.63693225277312 0.056572773009473824  -0.056282073822299154
16.64361510593878 0.05656824060895793 -0.055338651597077335
16.650297959104435  0.05558827846850357 -0.0564555060600607
16.656980812270092  0.05622647009045832 -0.056419327582053556
16.66366366543575 0.056077312612038216  -0.056928783358540846
16.670346518601402  0.056268447196517996  -0.05673886277003345
16.677029371767055  0.05673290399229845 -0.05656591622577587
16.683712224932712  0.056925550538804326  -0.056231146545792386
16.69039507809837 0.056633959159746505  -0.056361674568740375
16.697077931264026  0.05662821526563217 -0.05541237959669486
16.703760784429683  0.05564244398766896 -0.05654334838098352
16.710443637595336  0.056292116031864105  -0.056519136388502714
16.71712649076099 0.056172293140946815  -0.05701794206528679
16.723809343926646  0.0563673226296188  -0.05680288013700145
16.730370396706064  0.056803218424052275  -0.05670737877291833
16.73693144948548 0.05693350921928903 -0.056360451580899217
16.74349250226489 0.05681884974430979 -0.0565132924101382
16.750053555044307  0.05676593294954506 -0.055438305854729635
16.75661460782372 0.056141184766526436  -0.056428934434778345
16.763175660603135  0.056185351986386196  -0.05631279786287144
16.76973671338255 0.05635752028637547 -0.05651736534693725
16.776297766161967  0.05573936196375877 -0.05685707944736515
16.78285881894138 0.05669407745422322 -0.057001904716644576
16.789419871720796  0.05664947256950123 -0.05679944158880051
16.79598092450021 0.05707710914934376 -0.05677414339320875
16.802541977279624  0.05687160151595618 -0.056097768829390715
16.809103030059042  0.05687880152243615 -0.056217343880377746
16.81566408283846 0.05645617215960148 -0.05637525753977473
16.822225135617874  0.05660317754959681 -0.05586917403103308
16.828786188397288  0.05561381042849461 -0.05676304934110567
16.835347241176702  0.05642167104130514 -0.056728010217637286
16.841908293956116  0.05633947261336636 -0.057116780336281044
16.848469346735534  0.0564685985035652  -0.05688046408649302
16.85503039951495 0.05690336486913047 -0.05681579576435447
16.861591452294366  0.05699791671130064 -0.05640020394599665
16.86815250507378 0.05688010345365051 -0.05658526184289594
16.874713557853195  0.0568010940150701  -0.05558946902898701
16.88127461063261 0.05625718954420418 -0.056462534205378334
16.887835663412027  0.056178729711404814  -0.05640186363257682
16.894396716191437  0.0564345484797203  -0.05657531995705371
16.90095776897085 0.05582169461573024 -0.0569152287886491
16.90751882175027 0.05673343727754394 -0.05702584621573948
16.914079874529683  0.05669149102852747 -0.05682012151623612
16.920640927309098  0.05709089642846183 -0.05677011260799173
16.927201980088512  0.056885326015771655  -0.05617752593005502
16.93376303286793 0.0569104543419198  -0.05617653676246225
16.940324085647344  0.05643455782737024 -0.05641579803810443
16.946885138426758  0.05659978190471683 -0.055907078609214
16.953446191206172  0.05568850430523969 -0.05676622880341687
16.960007243985586  0.05638051224172183 -0.056726956760026
16.966568296765004  0.05635941525039305 -0.057114173608400265
16.97312934954442 0.05645879653712642 -0.0568526703706999
16.979690402323833  0.05689366034289909 -0.05681264469933522
16.986251455103247  0.05695474689846764 -0.056339508980684024
16.992812507882665  0.056837049578789677  -0.056546896286754425
16.99937356066208 0.05673258209683944 -0.055624012204464716
17.00593461344149 0.056258711329906064  -0.056386663140283115
17.012495666220907  0.056068973814739034  -0.05637811869074391
17.01905671900032 0.05639756998607755 -0.056521830432140076
17.02561777177974 0.05578999366146451 -0.05686344123456812
17.032178824559153  0.0566591669719467  -0.05694193011452326
17.038739877338568  0.05662292461756134 -0.056736629598881684
17.045300930117982  0.056995270757858256  -0.056662270586212135
17.0518619828974  0.05679226257626442 -0.056142147863943996
17.058423035676814  0.05683063371312117 -0.0560319333955119
17.06498408845623 0.05631167160740601 -0.05634170556105554
17.071545141235646  0.056486593742688246  -0.05583077540758515
17.07810619401506 0.0556461205373599  -0.05665522634682896
17.084667246794474  0.05623082142811013 -0.056615426394849996
17.091228299573892  0.05626679442178608 -0.056999027321939795
17.097789352353306  0.056336556516531425  -0.0567184198875289
17.10435040513272 0.056772913452935364  -0.05669767355142342
17.110911457912135  0.056803206854674125  -0.05617790839010459
17.11747251069155 0.05668888332532016 -0.056398348869280135
17.124033563470967  0.05656004468735587 -0.05554133942021419
17.13059461625038 0.05614572427754869 -0.05620183477347556
17.137155669029795  0.05585683580756588 -0.05624080063949101
17.14371672180921 0.05624689331437159 -0.05635626558540518
17.149837096651364  0.055427198068423056  -0.05669679177236734
17.15595747149352 0.056118319411062476  -0.05651962153432805
17.162077846335674  0.05617547782463725 -0.0566727941561392
17.16819822117783 0.05567793396636452 -0.05666549356934667
17.17431859601998 0.056302191128437803  -0.056641699520815415
17.180438970862134  0.056173912042497165  -0.05681427597903813
17.186559345704293  0.05599680782233072 -0.056589576746478104
17.192679720546444  0.05648587707991653 -0.05671148824693841
17.1988000953886  0.05630431736931886 -0.05657572221285494
17.204920470230753  0.056327730363655244  -0.05647961219272398
17.211040845072908  0.05658084276862408 -0.05659999646901831
17.217161219915063  0.05643394294249996 -0.05626902293294996
17.223281594757214  0.056615375439200324  -0.05634728398247602
17.22940196959937 0.05652675237416141 -0.056238425188380195
17.23552234444152 0.05653375344386232 -0.05594102379799902
17.241642719283675  0.056633209937055944  -0.056211776285691434
17.247763094125833  0.056429912223625034  -0.05585181455976814
17.253883468967985  0.05657587527293071 -0.05563891970969195
17.26000384381014 0.05636263121102433 -0.05608785289293995
17.26612421865229 0.05630206667417213 -0.0554886597796424
17.272244593494445  0.05637392086628637 -0.055576554081003565
17.2783649683366  0.0560359250288291  -0.05597837813050673
17.284485343178755  0.05615594279268166 -0.055196286355079144
17.29060571802091 0.05600087269797232 -0.055703097138171696
17.29672609286306 0.05570025726413637 -0.05589116419855016
17.302846467705216  0.05601890365698311 -0.05527417834539969
17.308966842547374  0.055614587608619516  -0.05587387639508878
17.315087217389525  0.05540305570964398 -0.05583303467642412
17.32120759223168 0.05589577946703468 -0.05554373389465483
17.327327967073835  0.055262339697982756  -0.05605614549402895
17.33344834191599 0.05542227221180453 -0.055907564502587326
17.339568716758144  0.05578915637435006 -0.05584594148838133
17.345689091600295  0.05499032566856132 -0.05621637795643247
17.35180946644245 0.0555594669581118  -0.05601226085465299
17.35792984128461 0.05570582215414735 -0.05612671339651787
17.36405021612676 0.055168559172397574  -0.056167462499399745
17.370170590968915  0.05573325367187322 -0.056098926977155954
17.376290965811066  0.05565902051680801 -0.05631025973867421
17.38241134065322 0.055439375949481104  -0.05606056622571104
17.388531715495375  0.055910088526463765  -0.05614129766468141
17.39465209033753 0.05574862101945336 -0.05605002625011331
17.400772465179685  0.05572938435576772 -0.0559179423822319
17.406892840021836  0.05602514982483243 -0.05606795676033415
17.41301321486399 0.05584036642981514 -0.05571645386221585
17.419133589706142  0.05598525617658169 -0.05575358794529348
17.4252539645483  0.055944988421389445  -0.055709661690854284
17.431374339390455  0.05590805260897961 -0.055356975057073134
17.437494714232606  0.056049293846158074  -0.055591146374034416
17.44361508907476 0.0558185795358696  -0.055319750981336305
17.449735463916916  0.055926351215102046  -0.055019135967628315
17.45585583875907 0.05575806609508492 -0.05545690312323416
17.461976213601226  0.055659966952122476  -0.0549438680063437
17.468096588443377  0.05576766019294597 -0.054869697980324865
17.474216963285535  0.05540550899891981 -0.055336954770569206
17.48033733812769 0.05548383212794537 -0.05462673474720878
17.48645771296984 0.055395708134646594  -0.05497358136554719
17.492578087811996  0.055039134902755284  -0.055236798947018335
17.49869846265415 0.0553234122073814  -0.05459332606948261
17.504818837496305  0.055003272654241105  -0.055128025076000335
17.51093921233846 0.05470649209126557 -0.05516125169017343
17.51705958718061 0.05518896063022107 -0.05481549332142322
17.523179962022766  0.054635033391517554  -0.055298579081673904
17.529300336864917  0.054640491063389494  -0.055181354492452594
17.535420711707076  0.05507027167530244 -0.05507718969141376
17.54205704126503 0.054459788892215914  -0.05549564467093923
17.548693370822985  0.055211151540801105  -0.05547948732487206
17.555329700380938  0.055220517958451024  -0.05522995412670407
17.561966029938894  0.055552545784653865  -0.05509585894176922
17.568602359496847  0.05524679122482436 -0.05464652346598065
17.575238689054803  0.05517185850048625 -0.05447929457059085
17.58187501861276 0.054526937754121964  -0.054851849524056924
17.588511348170712  0.05485320242658288 -0.05462725056501294
17.64160198463435 0.0546233810130039  -0.05417392156604739
17.654874643750258  0.054384998054221215  -0.054830351219455356
17.668147302866167  0.05468937932538961 -0.054712092134345656
17.681419961982073  0.054870412993868675  -0.05390483921679994
17.694692621097982  0.05438103696697361 -0.05373015007286394
17.70796528021389 0.05395515162897058 -0.054467629524400124
17.7212379393298  0.05420461649381376 -0.0544934287355941
17.73451059844571 0.054547098389221414  -0.053776492736422155
17.74778325756162 0.05412838688983087 -0.05332208031355207
17.75441958711957 0.05369050040415485 -0.05399599256653815
17.761055916677527  0.05352800986212409 -0.05408792089795785
17.767692246235484  0.053899317232131434  -0.05431974411356721
17.774328575793437  0.053702466168277174  -0.054251842668202266
17.78096490535139 0.05428091214299713 -0.054301482793688204
17.787601234909346  0.0541878445549201  -0.053654523344845016
17.794237564467302  0.05412679371502759 -0.05380603415657475
17.800873894025255  0.05389138321568884 -0.05314142870270216
17.853964530488888  0.053640733935947676  -0.052978767689516065
17.860600860046844  0.053447951756777 -0.053060277302992814
17.867237189604797  0.05281696802056467 -0.053330193354745266
17.873873519162753  0.05327751145063234 -0.053298486553732864
17.880509848720706  0.052698818418737234  -0.0536839852666522
17.88714617827866 0.053380404719219954  -0.053640173443524355
17.893782507836615  0.05338840858082298 -0.05338146846426038
17.90041883739457 0.05368429990647149 -0.053229320063527416
17.907055166952524  0.053372679895672986  -0.05282319897750794
17.913691496510477  0.05329588152295455 -0.05259713904885481
17.920327826068434  0.052634547534087923  -0.052987479720961426
17.92696415562639 0.052955587515636296  -0.05275994533011428
17.933600485184343  0.05222311881603093 -0.05334985413106792
17.9402368147423  0.05287863594927695 -0.05323998998234013
17.946873144300255  0.052960982066405086  -0.053206603569691785
17.953509473858208  0.05318789915087872 -0.05295344536670284
17.960145803416165  0.053083399329968244  -0.05266422800243189
17.966341455036854  0.053149013306103415  -0.052249566200472095
17.97253710665755 0.05292585057372329 -0.05267753847474058
17.978732758278245  0.052820945337878164  -0.05215983591297257
17.984928409898934  0.05283472573945461 -0.05212464406796261
17.99112406151963 0.05239816341541673 -0.05252217969305002
17.997319713140325  0.05255442724768427 -0.05185283051288849
18.003515364761018  0.0523096063229629  -0.052308066842652365
18.00971101638171 0.051905487623558295  -0.05240610616999038
18.01590666800241 0.052368525891319885  -0.052101518144228746
18.022102319623098  0.05181565857856537 -0.05253250731904768
18.02829797124379 0.051857596110594664  -0.05241887576356981
18.03449362286449 0.05221574589947588 -0.05241270721770372
18.04068927448518 0.051594057274033114  -0.052624070384106526
18.046884926105875  0.052048286301892864  -0.052467862009476805
18.05308057772657 0.052102853667165216  -0.052650954814591096
18.05927622934726 0.051853574126528544  -0.0524268864428409
18.06547188096796 0.05226609271566673 -0.05246021901892047
18.07166753258865 0.05213159028307456 -0.05230280559002772
18.077863184209342  0.05215628592469513 -0.05216273667860228
18.08405883583004 0.05230314227746233 -0.05220734237007236
18.090254487450732  0.05217133928749474 -0.05177683923184051
18.096450139071422  0.05236783697241013 -0.051877666848779376
18.10264579069212 0.05209054281313274 -0.05168650916360558
18.108841442312812  0.05214675588840455 -0.05126666838222103
18.115037093933502  0.05192874677264182 -0.05168183360119817
18.1212327455542  0.05181772778428711 -0.05117963306106587
18.127428397174892  0.05183186303486712 -0.051122843032985595
18.133624048795586  0.0513959333676876  -0.05151804274734495
18.13981970041628 0.05154048859066319 -0.05085735438497397
18.146015352036972  0.05130628286911147 -0.05129055825062132
18.152211003657666  0.05089667028262941 -0.051392497573283645
18.15840665527836 0.05134663751950023 -0.05108474058005054
18.164602306899056  0.05080995662266297 -0.051499964090345
18.170797958519746  0.05082914463005586 -0.05138883689377573
18.17699361014044 0.05118569678350342 -0.05137462434392307
18.183189261761136  0.05057146270845464 -0.05158289230329948
18.18938491338183 0.05100433460369774 -0.051421912059210685
18.19558056500252 0.051063558237401205  -0.051594551631874534
18.201776216623216  0.050809845387655685  -0.051375911914791184
18.20797186824391 0.051207545792325664  -0.05140047841380554
18.2141675198646  0.051075466955182 -0.05124745195124297
18.220363171485296  0.05109173895833062 -0.05110248667043346
18.22655882310599 0.051236882977099225  -0.05114610427707763
18.232754474726683  0.051099795287420785  -0.05071770366019937
18.238950126347376  0.051285820880212704  -0.05080851503858346
18.24514577796807 0.05101494947069725 -0.050626330588074975
18.251341429588763  0.05106213878383412 -0.05020247680822797
18.257537081209456  0.05084972965110399 -0.05060465148084008
18.263732732830153  0.05073324916483433 -0.05011926164828651
18.269928384450843  0.05074794630034895 -0.05004115843750013
18.276124036071536  0.05031346922842306 -0.05043405787507217
18.282319687692233  0.05044659150900445 -0.04978432911647491
18.288515339312926  0.05022422565739822 -0.050194190537018755
18.29471099093362 0.04980917012271604 -0.050300902913051275
18.300906642554313  0.05024648515974441 -0.04999149640813304
18.307102294175007  0.04972817652304045 -0.05038980184642874
18.313297945795703  0.04972349656825251 -0.050282308482999284
18.319493597416397  0.05007941016777493 -0.05026065604578563
18.325689249037087  0.04947544416746011 -0.05046536921293639
18.331884900657784  0.04988460252573763 -0.05030021024570775
18.338080552278477  0.04995022017220614 -0.05046249862252601
18.344276203899167  0.04969354435156706 -0.0502495779418651
18.350471855519864  0.05007485524946134 -0.050265518562728054
18.356667507140557  0.0499465195007537  -0.05011689052542125
18.362741358375008  0.04989321250473642 -0.04999151975309022
18.374889060843916  0.049886500472416696  -0.04977252563287577
18.38703676331282 0.049940189315308525  -0.049789656101505146
18.399184465781726  0.04995731374661087 -0.049533078488395346
18.411332168250635  0.04972645597191976 -0.04906319270528738
18.42347987071954 0.049535592117798274  -0.04910086907060949
18.435627573188444  0.04941652242160616 -0.049160460243222036
18.447775275657353  0.04941274229911848 -0.04851498826923404
18.45992297812626 0.04907539947506635 -0.048457063471512775
18.472070680595166  0.04870495355621446 -0.048833328721289974
18.48421838306407 0.04874089832417856 -0.04847877018052508
18.49636608553298 0.04865170935684644 -0.04816859249289641
18.508513788001885  0.04802692741034923 -0.048546006293059496
18.52066149047079 0.04806540941956561 -0.048521157481539165
18.5328091929397  0.04830573406032802 -0.04825488836946474
18.544956895408603  0.04781561217628899 -0.04837100821073481
18.55710459787751 0.047522365651944835  -0.04851857614931538
18.569252300346417  0.04799897301861823 -0.04834033145927672
18.58140000281532 0.04783153678823264 -0.048210449661868396
18.59354770528423 0.04755294660777246 -0.04814308821246942
18.605695407753135  0.04776398270489636 -0.048128514534707276
18.617843110222044  0.04786142056837911 -0.047979725287690386
18.648212366394308  0.04766083576412038 -0.04743831549240811
18.660360068863213  0.04750808503616082 -0.047073345769060145
18.67250777133212 0.04741682025936511 -0.0471226998783444
18.684655473801026  0.04737959162471368 -0.046996470783511296
18.69680317626993 0.04725634131149131 -0.04637814729330602
18.70895087873884 0.046941730480377634  -0.046457924530577566
18.721098581207745  0.04667549434905645 -0.04664805109219087
18.73324628367665 0.04670600614646996 -0.04613222043399805
18.745393986145558  0.046444662185522324  -0.04590340794262122
18.75198379209581 0.04615260570113092 -0.04634641956571651
18.76516340399632 0.04626102735580715 -0.0465706142634356
18.778343015896827  0.04648088200664155 -0.04651696166775983
18.79152262779733 0.04640091985739586 -0.04601155707129338
18.80470223969784 0.04596857222487138 -0.045546014812614374
18.817881851598344  0.04575086372059909 -0.04576883784468742
18.831061463498855  0.0456894127232594  -0.045980202388130685
18.84424107539936 0.045901612703211477  -0.045607469721756284
18.857420687299868  0.045741321626810855  -0.04481566528858907
18.870600299200373  0.045251134402193825  -0.04497557774880124
18.883779911100884  0.0448677516489077  -0.04539371783269413
18.89695952300139 0.045121699246025576  -0.04525970790645967
18.910139134901897  0.04523947068802549 -0.04455243931746993
18.9233187468024  0.04479956075920379 -0.044367115438851565
18.936498358702913  0.04409366716093203 -0.044793037265034016
18.949677970603418  0.04432789447337905 -0.044888775789111816
18.962857582503926  0.044667241421909656  -0.044398199048304464
18.97603719440443 0.04444896785276977 -0.044087929766946074
18.989216806304942  0.0436877221814622  -0.04422158772833137
19.002396418205446  0.043612224683735185  -0.04444296843765432
19.061704671757727  0.04305610336825634 -0.04369659691503085
19.167141566961785  0.04200372268803415 -0.042520643871863974
19.173290694974774  0.04229163006991934 -0.04218768356385808
19.179439822987767  0.0422326120637449  -0.04223036246475877
19.18558895100076 0.041562145496180015  -0.042436228570200164
19.191738079013753  0.04216112917673142 -0.04223768816891477
19.19788720702675 0.041917937811395654  -0.04237138124490923
19.20403633503974 0.04155643176193645 -0.04232913946975428
19.21018546305273 0.042064873955293125  -0.0422835289517284
19.216334591065724  0.04172101572080172 -0.04239518246550616
19.222483719078713  0.04167632261730988 -0.0421956069012759
19.22863284709171 0.04198708845669213 -0.04227207208139547
19.234781975104703  0.041752012837615995  -0.04211753258806048
19.240931103117695  0.04182848943038927 -0.04202860035335105
19.24708023113069 0.041890069453276724  -0.04198093284967579
19.253229359143678  0.041793701765228446  -0.041776227843388775
19.25937848715667 0.04194883268342679 -0.04182316564909897
19.265527615169667  0.04175566752924567 -0.0414660043533662
19.27167674318266 0.041793741535674465  -0.0414114999930942
19.277825871195652  0.04172598373344326 -0.041509202198973846
19.28397499920864 0.041588997428472496  -0.040931013184866104
19.290124127221635  0.04163554552991739 -0.04110326108461326
19.296273255234627  0.041392421050570276  -0.041185248552814815
19.30242238324762 0.041386545861250734  -0.04044546927326096
19.308571511260617  0.04113319313668354 -0.04098019353303335
19.314720639273606  0.04102076741935111 -0.04088903922865476
19.3208697672866  0.041112973777138254  -0.04027336599989266
19.32701889529959 0.04058859516616481 -0.04090196678189496
19.333168023312584  0.04065562740402252 -0.04064428760877897
19.339317151325577  0.040800456465892566  -0.04038573469002353
19.34546627933857 0.04008262887064613 -0.04084812475142402
19.351615407351563  0.04049955355439355 -0.040618414139012175
19.357764535364556  0.04050842982080513 -0.04056863396762611
19.36391366337755 0.039773901770369796  -0.04078073553008435
19.370062791390538  0.04042107474210001 -0.04063213991451793
19.37621191940353 0.04025888263897645 -0.040725046710367314
19.382361047416524  0.0398782103563246  -0.040631250129936396
19.388510175429516  0.040375323653849805  -0.04061575006554172
19.394659303442513  0.040152578944721226  -0.04062778230903346
19.400808431455502  0.0400435233491464  -0.04044550386684467
19.413106687481488  0.04015224808027454 -0.0402884456419074
19.41925581549448 0.04019971012719674 -0.04022571314011353
19.425404943507473  0.04020143359651848 -0.040115256690535624
19.431554071520466  0.04013508891136927 -0.03989692299592986
19.43770319953346 0.04022585695136115 -0.039974087612999656
19.443852327546452  0.040014492968403516  -0.039623796271633135
19.450001455559445  0.040062230629975354  -0.03950394099811921
19.456150583572434  0.039900787597088226  -0.03969607971265896
19.46229971158543 0.039791481174909125  -0.03915730923371863
19.468448839598423  0.03977562514421075 -0.039267907653624055
19.474597967611416  0.039508596695387484  -0.03943761849283973
19.48074709562441 0.03954114535507699 -0.03877298820930661
19.486896223637398  0.03931023847177453 -0.039205189624916786
19.49304535165039 0.03910077705108339 -0.039213898282284755
19.499194479663384  0.03927643583044113 -0.038768967966051913
19.50534360767638 0.03885136669960508 -0.039188483393357204
19.511492735689373  0.03876723250708792 -0.039053194496376974
19.517641863702362  0.03902840747192769 -0.03888671820535047
19.523790991715355  0.038453901517122496  -0.03918576045171413
19.529940119728348  0.038695988117058634  -0.039015948625990386
19.53608924774134 0.038811676313579464  -0.03900902717311803
19.542238375754337  0.03830747650141693 -0.03909604046369346
19.548387503767326  0.038684079603598015  -0.03896844466671972
19.55453663178032 0.0386338068013503  -0.039073657519877354
19.560685759793312  0.03839590681040539 -0.03889008065780001
19.56755004479654 0.03884395211338844 -0.038849852201433104
19.57441432979977 0.03878541543895207 -0.038268982582301464
19.581278614803 0.038467661281035286  -0.0384166752013814
19.58814289980623 0.038375537259715944  -0.03795940129946244
19.595007184809457  0.03784959190540215 -0.03849310320566708
19.60187146981269 0.038161265011495 -0.038458117266602504
19.60873575481592 0.03825587340382744 -0.038391784435784074
19.61560003981915 0.03846122334252933 -0.0381457238055142
19.622464324822378  0.03817718362081223 -0.037781206385207496
19.629328609825606  0.037976028296490015  -0.037716005431611216
19.636192894828838  0.037427113096841706  -0.03793179288947887
19.643057179832066  0.03781171129726881 -0.038042046901486506
19.649921464835295  0.03767653191554683 -0.03799066107749052
19.656785749838527  0.0379899223990508  -0.0378959229169713
19.663650034841755  0.03790151047860279 -0.03724906472360539
19.670514319844983  0.03746035988844103 -0.03754469429990461
19.67737860484821 0.03744945715295618 -0.037253238171778294
19.684242889851443  0.03704934912036376 -0.037693040259161564
19.691107174854675  0.03732890660742508 -0.037606296154867436
19.697971459857904  0.03745135750052596 -0.037409125698323
19.704835744861132  0.03760099370432466 -0.037178070035681444
19.71170002986436 0.03722896060359235 -0.03693631054184712
19.718564314867592  0.03706711666248763 -0.03684802745666028
19.72542859987082 0.03649567251865448 -0.03715047405584101
19.73229288487405 0.03699654979594234 -0.03724976180398259
19.73915716987728 0.03694302501607921 -0.03706864188259617
19.746021454880513  0.03711798573309466 -0.0369545205464024
19.75288573988374 0.036994871217662206  -0.03621620948346301
19.75975002488697 0.036431260750923775  -0.0366937479845493
19.766614309890198  0.03653853440614886 -0.036542237181715474
19.773478594893426  0.03627799885534719 -0.03690214669831638
19.780342879896654  0.03651146689299305 -0.036730770298930855
19.787207164899886  0.036627826131752035  -0.03640014722186211
19.794071449903115  0.03665330904597949 -0.036228648203583416
19.800935734906343  0.03625803740292005 -0.03612149136727818
19.80780001990957 0.03618324392887891 -0.03599576210963802
19.8146643049128  0.03557605977211038 -0.036356626626233075
19.82152858991603 0.03620289774498203 -0.03642821251065319
19.828392874919263  0.03619073749940305 -0.03612227444248207
19.83525715992249 0.03622485162706764 -0.03603101728654756
19.84212144492572 0.03606645446988263 -0.03519683168277045
19.84898572992895 0.03542286500535797 -0.03586583033640793
19.85585001493218 0.03564666679130094 -0.035816662969080856
19.86271429993541 0.03556465048498802 -0.03605638854851601
19.869578584938637  0.03574268522201743 -0.03583050298985409
19.87644286994187 0.03578192225069692 -0.03544293645407135
19.883307154945097  0.03570353068356537 -0.03529238734095539
19.890171439948325  0.03526002890969666 -0.03533753496804923
19.897035724951554  0.03532787533554183 -0.035275583248968304
19.903900009954786  0.03488371126284797 -0.03554566241200328
19.910764294958017  0.03542996419129393 -0.03556807005452027
19.917628579961246  0.03539453861051451 -0.03513496651535839
19.924492864964474  0.03528749973756505 -0.035130271949411485
19.931357149967702  0.0351190919921819  -0.034513548149137055
19.938221434970934  0.03462415947721495 -0.035062029797490166
19.945085719974163  0.03477720612264798 -0.03506046066168303
19.95195000497739 0.03485364476329291 -0.03516803711027092
19.958814289980623  0.03501375967786582 -0.034905225923372134
19.965678574983855  0.03491070897928411 -0.034590572368549975
19.972542859987083  0.03478500742891713 -0.034373769956979765
19.97940714499031 0.03424253893934205 -0.03458398378262657
19.98627142999354 0.03450358748305156 -0.03459577172145213
19.99313571499677 0.03399207038657715 -0.035628743976049405
20. 0.05096923766651328 -0.04404391618484699
\end{filecontents}

\usepgfplotslibrary{fillbetween}
\begin{tikzpicture}[>=latex]
  \begin{axis}[
      xlabel={Time (\si{\pico\second})},
      ylabel={Average coherence ($\rho_{01}$)}
    ]
    \addplot[smooth, no marks, name path=A, black] table [x index=0, y index = 1]{echo.dat};
    \addplot[smooth, no marks, name path=B, black] table [x index=0, y index = 2]{echo.dat};

    \addplot[blue!25] fill between [of=A and B];

    \draw[<-] (axis cs:1.35,0.90) -- (axis cs:3,0.90) node[right]{Dephasing};

    \draw[<-] (axis cs:5.1,0.4) -- (axis cs:6.5,0.55) node[right]{Rephasing ($\pi$-pulse)};

    \draw[<-] (axis cs:9.1,-0.45) -- (axis cs:9.1,-.65) node[below]{Echo effect};

    \draw[|-|] (axis cs:0.5,0) -- (axis cs:5,0) node [midway, fill=blue!25]{$\tau$};
    \draw[-|] (axis cs:5,0) -- (axis cs:9.5,0) node [midway, fill=blue!25]{$\tau$};
  \end{axis}
\end{tikzpicture}

  \caption{\label{fig:echo} Photon echo effect.
    A $\pi$-pulse incident on a collection of 64 nearly-resonant quantum dots serves to rephase their polarizations as an ``echo'' long after both pulses have passed through the system.
  }
\end{figure}


\bibliography{random_lasing_reflist}{}
\bibliographystyle{plain}

\end{document}
