\documentclass[conference]{IEEEtran}

%\usepackage{array}
%\usepackage{cite}
\usepackage{amsmath}
\usepackage{cleveref}
\usepackage{graphicx}
\usepackage{paralist}
\usepackage{physics}
\usepackage{microtype}

\hyphenation{op-tical net-works semi-conduc-tor}

\begin{document}
\title{Towards a Self-Consistent Integral-Equation Model of Optically Active Media}

\author{
  \IEEEauthorblockN{
    Connor Glosser\IEEEauthorrefmark{1}\IEEEauthorrefmark{2},
    B.\ Shanker\IEEEauthorrefmark{2}, and
    Carlo Piermarocchi\IEEEauthorrefmark{1}
  }
  \IEEEauthorblockA{\IEEEauthorrefmark{1}Department of Physics \& Astronomy\\
    Michigan State University, East Lansing, Michigan 48824}
  \IEEEauthorblockA{\IEEEauthorrefmark{1}Department of Electrical \& Computer Engineering\\
    Michigan State University, East Lansing, Michigan 48824}
}

% use for special paper notices
%\IEEEspecialpapernotice{(Invited Paper)}

\maketitle

\begin{abstract}
Conventional models of electromagnetic systems give rise to interesting phenomena through prescription of static boundary conditions.
  While these boundary conditions sufficiently describe passive scattering within systems, they fail to account for richer behavior in systems with dynamics.
  Such processes have proven essential in countless technological applications motivating the need for an efficient fullwave Maxwell solver that couples to an underlying description of the system behavior.

  Here we consider a disordered system of interacting quantum dots---nanoscale semiconductors with wide applicability in systems ranging from lasing to quantum computing to biological contrast imaging and next-generation displays.
  Much like atoms, individual quantum dots facilitate absorptive and emissive processes at specific frequencies over timescales independent of those in the incident radiation.
  These processes couple between dots due to the presence of electromagnetic fields, giving rise to emergent nonlinear behavior within the system.
  By treating quantum dots \emph{semiclassically} within our simulation, we maintain the discrete dynamics inherent to quantum objects without resorting to cumbersome second quantization to describe electromagnetic fields (i.e.\ fields behave classically).
  This has the advantage of partitoning the simulation into two distinct parts,
  \begin{inparaenum}[(1)]
    \item determination of source wavefunctions through evolution of the differential Liouville equations, and
    \item evaluation of radiation patterns through well-known CEM techniques.
  \end{inparaenum}
  We employ a highly-tuned predictor-corrector integration scheme to advance the source wavefunctions in time; the subsequent polarizations that arise then serve as pointlike sources to electromagnetic integral equations (chosen to facilitate accurate point-to-point communication of fields without the computational overhead of a ``radiation grid'').
  The coupled solution of (1) and (2), then, gives a complete description of both the quantum and electromagnetic dynamics at each timestep, giving rise to lasing effects and other optical phenomena.
\end{abstract}

\IEEEpeerreviewmaketitle

\section{Introduction}


\section{Problem Statement}
We wish to model the behavior of $N$ interacting quantum dots immersed in a homogeneous background medium amidst monochromatic radiation of frequency $\omega_L$.
\begin{equation}
  \ket{\psi_i} = c_0 \ket{0_i} + c_1 \ket{1_i}
\end{equation}
where the kets $\ket{0_i}$ and $\ket{1_i}$ represent the highest valence and lowest conduction states of the $i^\text{th}$ dot.
The density operator for each dot, $\hat{\rho}_i \equiv \dyad{\psi_i}$, then evolves according to the Liouville-von Neumann equation
\begin{equation}
  i \hbar \pdv{\hat{\rho}_i}{t} = \commutator{\hat{\mathcal{H}}_i(t)}{\hat{\rho}_i} - \hat{\mathcal{D}}_i\qty[\hat{\rho}_i].
  \label{eq:liouville}
\end{equation}
Here, hats denote operators and the Hamiltonian $\hat{\mathcal{H}}_i$ accounts for both the dot's internal two-level structure as well as its interaction with an external field.
The dissipator, $\hat{\mathcal{D}}_i$, serves to return $\hat{\rho}_i$ to the ground state over long periods of time in a manner remeniscent of spontaneous emission.
As matrices,
\begin{subequations}
  \begin{align}
    \hat{\mathcal{H}}(t) &= \mqty(0 & \hbar \chi_i(t) \\ \hbar \chi_i^*(t) & \hbar \omega_0) \\
    \hat{\mathcal{D}}[\hat{\rho}] &= \mqty((\rho_{00} + 1)/T_2 & \rho_{01}/T_1 \\ \rho_{10}/T_1 & \rho_{11}/T_2).
  \end{align}
\end{subequations}
where $\hat{\chi}_i(t) \equiv \hat{\vb{d}}_i \cdot \hat{\vb{E}}/\hbar$ and $\hat{\vb{d}}_i \equiv -e \hat{\vb{r}}_i$.
If we elect to ignore single-photon effects by considering only fields of sufficient strength,
we may remove the \emph{operator} $\hat{\chi}_i(t)$ in favor of its classical counterpart: $\chi(t) \equiv \vb{d}_i \cdot \vb{E}(\vb{r}_i, t)$ with $\vb{d}_i \equiv \matrixelement{0_i}{\hat{\vb{d}}_i}{1_i}$.
\section{Solution of \Cref{eq:liouville}}
\subsection{The Rotating Wave Approximation}

\end{document}
